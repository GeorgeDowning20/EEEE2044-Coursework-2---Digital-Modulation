\begin{figure}[ht!]
    \begin{center}
        % This file was created by matlab2tikz.
%
%The latest updates can be retrieved from
%  http://www.mathworks.com/matlabcentral/fileexchange/22022-matlab2tikz-matlab2tikz
%where you can also make suggestions and rate matlab2tikz.
%
\definecolor{mycolor1}{rgb}{0.00000,0.44700,0.74100}%
\definecolor{mycolor2}{rgb}{0.85000,0.32500,0.09800}%
\definecolor{mycolor3}{rgb}{0.92900,0.69400,0.12500}%
%
\begin{tikzpicture}

\begin{axis}[%
width=4.568in,
height=3.603in,
at={(0.766in,0.486in)},
scale only axis,
xmode=log,
xmin=0.1,
xmax=100000,
xminorticks=true,
xlabel style={font=\color{white!15!black}},
xlabel={Simulation Stop Time [s]},
ymin=10,
ymax=100,
ylabel style={font=\color{white!15!black}},
ylabel={BER [\%]},
axis background/.style={fill=white},
axis x line*=bottom,
axis y line*=left,
legend style={legend cell align=left, align=left, draw=white!15!black}
]
\addplot [color=mycolor1, forget plot]
  table[row sep=crcr]{%
0.1	50\\
0.2	33.3333333333333\\
0.3	25\\
0.4	20\\
0.5	16.6666666666667\\
0.6	14.2857142857143\\
0.7	12.5\\
0.8	22.2222222222222\\
0.900000000000001	30\\
1	27.2727272727273\\
1.09999999999999	25\\
1.2	30.7692307692308\\
1.29999999999999	28.5714285714286\\
1.4	26.6666666666667\\
1.5	25\\
1.6	29.4117647058824\\
1.70000000000001	27.7777777777778\\
1.90000000000002	35\\
2.1	31.8181818181818\\
2.3	29.1666666666667\\
2.4	28\\
2.59999999999998	33.3333333333333\\
2.8	31.0344827586207\\
3	35.4838709677419\\
3.29999999999998	32.3529411764706\\
3.40000000000002	31.4285714285714\\
3.5	33.3333333333333\\
3.80000000000003	30.7692307692308\\
3.89999999999997	32.5\\
4.30000000000002	29.5454545454545\\
4.70000000000001	27.0833333333333\\
5.19999999999996	24.5283018867925\\
5.40000000000001	23.6363636363636\\
5.49999999999996	25\\
6	22.9508196721311\\
6.30000000000001	21.875\\
6.40000000000001	23.0769230769231\\
6.59999999999995	22.3880597014925\\
6.69999999999995	23.5294117647059\\
7.29999999999998	21.6216216216216\\
7.4	22.6666666666667\\
7.69999999999994	21.7948717948718\\
7.79999999999994	22.7848101265823\\
8.40000000000001	21.1764705882353\\
8.69999999999996	23.8636363636364\\
8.89999999999997	23.3333333333333\\
9.00000000000001	24.1758241758242\\
9.20000000000001	23.6559139784946\\
9.29999999999997	24.468085106383\\
9.50000000000007	23.9583333333333\\
9.60000000000001	24.7422680412371\\
9.69999999999993	24.4897959183673\\
9.80000000000001	25.2525252525253\\
9.89999999999993	25\\
10	25.7425742574257\\
10.3999999999999	24.7619047619048\\
10.5	25.4716981132076\\
10.6	25.2336448598131\\
10.7	25.9259259259259\\
10.8	25.6880733944954\\
11.2	28.3185840707965\\
11.3000000000001	28.0701754385965\\
11.5	29.3103448275862\\
11.6999999999999	28.8135593220339\\
11.9000000000001	30\\
12	29.7520661157025\\
12.1	30.327868852459\\
12.2000000000001	30.0813008130081\\
12.3000000000001	30.6451612903226\\
12.5	30.1587301587302\\
12.6	30.7086614173228\\
12.7000000000001	30.46875\\
12.8	31.0077519379845\\
12.9000000000001	30.7692307692308\\
13.3000000000001	32.8358208955224\\
13.6000000000001	32.1167883211679\\
13.7000000000001	32.6086956521739\\
14.1	31.6901408450704\\
14.2000000000001	32.1678321678322\\
14.8	30.8724832214765\\
14.9	31.3333333333333\\
15	31.1258278145695\\
15.1000000000001	31.5789473684211\\
15.4999999999999	30.7692307692308\\
15.5999999999999	31.2101910828025\\
16.2	30.0613496932515\\
16.3	30.4878048780488\\
16.4000000000001	30.3030303030303\\
16.4999999999999	30.7228915662651\\
17.5	28.9772727272727\\
17.5999999999999	29.3785310734463\\
18.2999999999999	28.2608695652174\\
18.4	28.6486486486487\\
18.5	28.494623655914\\
18.6	28.8770053475936\\
19.3999999999999	27.6923076923077\\
19.4999999999999	28.0612244897959\\
19.7000000000001	27.7777777777778\\
19.7999999999999	28.1407035175879\\
20	27.8606965174129\\
20.0999999999999	28.2178217821782\\
20.4000000000001	27.8048780487805\\
20.5000000000001	28.1553398058252\\
20.7999999999998	27.7511961722488\\
20.9	28.0952380952381\\
21	27.9620853080569\\
21.0999999999999	28.3018867924528\\
21.6999999999999	27.5229357798165\\
22.0999999999999	28.8288288288288\\
22.4	28.4444444444444\\
22.5	28.7610619469027\\
22.6000000000001	28.6343612334802\\
22.7000000000001	28.9473684210526\\
22.7999999999998	28.82096069869\\
22.9000000000001	29.1304347826087\\
23.0999999999998	28.8793103448276\\
23.2999999999999	29.4871794871795\\
23.6999999999998	28.9915966386555\\
24.2	30.4526748971193\\
24.5	30.0813008130081\\
24.6000000000001	30.3643724696356\\
24.7	30.241935483871\\
24.7999999999999	30.5220883534137\\
25.4000000000001	29.8039215686275\\
25.5000000000001	30.078125\\
25.8000000000002	29.7297297297297\\
25.9	30\\
26.5	29.3233082706767\\
26.6000000000002	29.5880149812734\\
27.0999999999999	29.0441176470588\\
27.2000000000001	29.3040293040293\\
27.4000000000002	29.0909090909091\\
27.4999999999998	29.3478260869565\\
28.4000000000002	28.4210526315789\\
28.6	28.9198606271777\\
29.1999999999999	28.3276450511945\\
29.2999999999999	28.5714285714286\\
29.4	28.4745762711864\\
29.4999999999998	28.7162162162162\\
29.6999999999998	28.5234899328859\\
29.7999999999999	28.7625418060201\\
29.8999999999998	28.6666666666667\\
30	28.9036544850498\\
30.5000000000003	28.4313725490196\\
30.6000000000002	28.6644951140065\\
31.3	28.0254777070064\\
31.3999999999998	28.2539682539683\\
31.5	28.1645569620253\\
31.5999999999998	28.391167192429\\
31.8	28.2131661442006\\
31.9000000000001	28.4375\\
32.1000000000002	28.2608695652174\\
32.2	28.4829721362229\\
33.2000000000001	27.6276276276276\\
33.3	27.8443113772455\\
33.4000000000001	27.7611940298507\\
33.6	28.1899109792285\\
33.7000000000001	28.1065088757396\\
33.8	28.3185840707965\\
33.9000000000002	28.2352941176471\\
34.1000000000002	28.6549707602339\\
34.5	28.3236994219653\\
34.6	28.5302593659942\\
34.9000000000003	28.2857142857143\\
35.0999999999998	28.6931818181818\\
35.3000000000001	28.5310734463277\\
35.3999999999998	28.7323943661972\\
36	28.2548476454294\\
36.1	28.4530386740332\\
37.1999999999999	27.6139410187668\\
37.2999999999998	27.807486631016\\
37.3999999999999	27.7333333333333\\
37.5	27.9255319148936\\
37.6000000000001	27.8514588859416\\
37.8000000000001	28.2321899736148\\
38.1000000000002	28.0104712041885\\
38.2000000000002	28.1984334203655\\
38.4	28.0519480519481\\
38.4999999999997	28.2383419689119\\
38.5999999999998	28.1653746770026\\
38.7000000000002	28.3505154639175\\
39.2	27.9898218829517\\
39.4999999999997	28.5353535353535\\
39.5999999999997	28.463476070529\\
39.7000000000001	28.643216080402\\
40	28.428927680798\\
40.1999999999997	28.7841191066998\\
41.0000000000002	28.2238442822384\\
41.1000000000003	28.3980582524272\\
41.2000000000001	28.3292978208232\\
41.2999999999997	28.5024154589372\\
41.9000000000002	28.0952380952381\\
42	28.2660332541568\\
42.0999999999999	28.1990521327014\\
42.1999999999999	28.3687943262411\\
42.3000000000001	28.3018867924528\\
42.3999999999999	28.4705882352941\\
42.5000000000002	28.4037558685446\\
42.6000000000003	28.5714285714286\\
42.8000000000002	28.4382284382284\\
42.9000000000001	28.6046511627907\\
43.2000000000001	28.4064665127021\\
43.3	28.5714285714286\\
43.7000000000004	28.310502283105\\
43.7999999999999	28.4738041002278\\
43.9999999999997	28.3446712018141\\
44.1999999999999	28.6681715575621\\
44.4999999999999	28.47533632287\\
44.6999999999999	28.7946428571429\\
44.9000000000004	28.6666666666667\\
45	28.8248337028825\\
45.4000000000002	28.5714285714286\\
45.7000000000001	29.0393013100437\\
46.1	28.7878787878788\\
46.1999999999997	28.9416846652268\\
46.3000000000001	28.8793103448276\\
46.3999999999998	29.0322580645161\\
46.4999999999999	28.9699570815451\\
46.5999999999999	29.1220556745182\\
47.5000000000004	28.5714285714286\\
47.7999999999999	29.0187891440501\\
47.8999999999998	28.9583333333333\\
48	29.1060291060291\\
48.4000000000001	28.8659793814433\\
48.4999999999996	29.0123456790123\\
48.6999999999997	28.8934426229508\\
48.7999999999996	29.038854805726\\
48.9000000000001	28.9795918367347\\
49	29.1242362525458\\
49.2000000000002	29.0060851926978\\
49.3	29.1497975708502\\
49.5999999999999	28.9738430583501\\
49.7000000000003	29.1164658634538\\
49.8000000000002	29.0581162324649\\
49.9	29.2\\
50.3000000000004	28.968253968254\\
50.4000000000001	29.1089108910891\\
50.7000000000001	28.9370078740157\\
50.8000000000003	29.0766208251474\\
51.2000000000001	28.8499025341131\\
51.2999999999996	28.988326848249\\
51.3999999999997	28.9320388349515\\
51.5000000000001	29.0697674418605\\
52.1999999999998	28.6806883365201\\
52.3999999999997	28.952380952381\\
52.5	28.8973384030418\\
52.5999999999997	29.0322580645161\\
52.7999999999997	28.9224952741021\\
52.9	29.0566037735849\\
53.2999999999998	28.8389513108614\\
53.3999999999999	28.9719626168224\\
53.5000000000002	28.9179104477612\\
53.5999999999996	29.0502793296089\\
54.4999999999998	28.5714285714286\\
54.5999999999996	28.7020109689214\\
54.8999999999996	28.5454545454545\\
55.1000000000002	28.804347826087\\
55.5	28.5971223021583\\
55.5999999999998	28.7253141831239\\
55.7000000000004	28.673835125448\\
56	29.0552584670232\\
56.0999999999999	29.0035587188612\\
56.1999999999998	29.1296625222025\\
56.4000000000001	29.0265486725664\\
56.5000000000003	29.1519434628975\\
56.9000000000003	28.9473684210526\\
57.0000000000005	29.0718038528897\\
57.2000000000001	28.9703315881326\\
57.4000000000002	29.2173913043478\\
57.6999999999999	29.0657439446367\\
57.7999999999996	29.1882556131261\\
58.9000000000003	28.6440677966102\\
58.9999999999996	28.7648054145516\\
59.1000000000004	28.7162162162162\\
59.2	28.8364249578415\\
59.3999999999996	28.7394957983193\\
59.5000000000003	28.8590604026846\\
60.2000000000003	28.5240464344942\\
60.2999999999996	28.6423841059603\\
60.5000000000001	28.5478547854785\\
60.7000000000004	28.7828947368421\\
60.7999999999995	28.735632183908\\
60.8999999999997	28.8524590163934\\
61.0000000000005	28.8052373158756\\
61.0999999999996	28.921568627451\\
61.2000000000003	28.8743882544861\\
61.3000000000005	28.9902280130293\\
62.1000000000001	28.6173633440514\\
62.1999999999996	28.7319422150883\\
62.2999999999998	28.6858974358974\\
62.3999999999996	28.8\\
62.9000000000003	28.5714285714286\\
63.0000000000001	28.6846275752773\\
63.1000000000001	28.6392405063291\\
63.1999999999996	28.7519747235387\\
64.0000000000001	28.393135725429\\
64.0999999999995	28.5046728971963\\
64.5999999999998	28.2843894899536\\
64.7000000000002	28.3950617283951\\
66.0999999999997	27.7945619335347\\
66.2999999999998	28.0120481927711\\
66.5000000000005	27.9279279279279\\
66.6000000000001	28.0359820089955\\
67.6000000000001	27.6218611521418\\
67.7000000000001	27.7286135693215\\
68.2000000000004	27.5256222547584\\
68.2999999999997	27.6315789473684\\
68.8000000000004	27.4310595065312\\
68.9000000000005	27.536231884058\\
69.5999999999997	27.2596843615495\\
69.6999999999994	27.3638968481375\\
70	27.246790299572\\
70.1000000000003	27.3504273504274\\
70.1999999999995	27.3115220483642\\
70.3000000000006	27.4147727272727\\
70.5000000000001	27.3371104815864\\
70.6000000000001	27.4398868458274\\
70.7000000000004	27.4011299435028\\
70.7999999999995	27.5035260930889\\
71.4000000000004	27.2727272727273\\
71.5000000000001	27.3743016759777\\
72.3999999999996	27.0344827586207\\
72.6000000000002	27.2352132049519\\
72.7999999999995	27.1604938271605\\
72.9999999999998	27.359781121751\\
73.0999999999995	27.3224043715847\\
73.3000000000003	27.5204359673025\\
74	27.2604588394062\\
74.1	27.3584905660377\\
74.1999999999999	27.321668909825\\
74.3000000000001	27.4193548387097\\
74.7000000000003	27.2727272727273\\
75	27.5632490013316\\
75.2999999999998	27.4535809018568\\
75.4000000000003	27.5496688741722\\
75.8999999999995	27.3684210526316\\
76.0999999999997	27.5590551181102\\
76.4000000000004	27.4509803921569\\
76.7000000000001	27.734375\\
76.8000000000001	27.6983094928479\\
76.9999999999994	27.8858625162127\\
77.0999999999997	27.8497409326425\\
77.1999999999996	27.9430789133247\\
78.5999999999996	27.4459974587039\\
78.8000000000006	27.6299112801014\\
78.9999999999994	27.5600505689001\\
79.1999999999995	27.7427490542245\\
79.4000000000002	27.6729559748428\\
79.4999999999999	27.7638190954774\\
79.9000000000005	27.625\\
80	27.7153558052434\\
80.2000000000006	27.6463262764633\\
80.3000000000006	27.7363184079602\\
80.6000000000004	27.633209417596\\
80.7000000000006	27.7227722772277\\
80.8999999999994	27.6543209876543\\
81.0000000000001	27.7435265104809\\
81.5	27.5735294117647\\
81.6000000000004	27.6621787025704\\
82.4000000000002	27.3939393939394\\
82.5999999999994	27.5695284159613\\
84.1000000000005	27.0783847980998\\
84.1999999999998	27.1648873072361\\
84.5000000000001	27.0685579196218\\
84.6000000000002	27.1546635182999\\
85.6000000000004	26.8378063010502\\
85.6999999999994	26.9230769230769\\
85.8000000000001	26.8917345750873\\
85.8999999999994	26.9767441860465\\
86.3000000000006	26.8518518518519\\
86.4999999999999	27.0207852193995\\
86.7999999999998	26.92750287687\\
86.9000000000001	27.0114942528736\\
87.1999999999997	26.9186712485682\\
87.4000000000007	27.0857142857143\\
87.6999999999998	26.9931662870159\\
87.8000000000001	27.0762229806598\\
89.0999999999995	26.6816143497758\\
89.2000000000006	26.7637178051512\\
89.3999999999999	26.7039106145251\\
89.4999999999998	26.7857142857143\\
89.6000000000001	26.7558528428094\\
89.8000000000008	26.9187986651835\\
90.0000000000001	26.8590455049944\\
90.1999999999997	27.0210409745293\\
90.4000000000005	26.9613259668508\\
90.4999999999994	27.0419426048565\\
91.1000000000006	26.8640350877193\\
91.1999999999993	26.9441401971522\\
91.2999999999997	26.9146608315098\\
91.4999999999993	27.0742358078603\\
91.9000000000005	26.9565217391304\\
92.0000000000001	27.0358306188925\\
92.7000000000002	26.8318965517241\\
92.9000000000005	26.989247311828\\
93.4999999999997	26.8162393162393\\
93.5999999999994	26.8943436499466\\
93.7000000000002	26.865671641791\\
93.7999999999994	26.9435569755059\\
94.0000000000001	26.8862911795962\\
94.1000000000004	26.963906581741\\
94.3000000000004	26.9067796610169\\
94.3999999999994	26.984126984127\\
94.5999999999998	26.9271383315734\\
94.7000000000001	27.0042194092827\\
95.0000000000007	26.919032597266\\
95.1	26.9957983193277\\
95.3000000000004	26.9392033542977\\
95.3999999999999	27.0157068062827\\
95.9000000000006	26.875\\
96.0000000000001	26.9510926118626\\
96.1999999999993	26.8951194184839\\
96.3000000000005	26.9709543568465\\
96.9999999999993	26.7765190525232\\
97.2999999999996	27.0020533880903\\
97.4999999999992	26.9467213114754\\
97.5999999999993	27.021494370522\\
97.8000000000001	26.9662921348315\\
98.0000000000001	27.1151885830785\\
98.5000000000007	26.9776876267749\\
98.7000000000002	27.1255060728745\\
98.9000000000006	27.0707070707071\\
98.9999999999993	27.1442986881937\\
99.8000000000001	26.9269269269269\\
99.9000000000001	27\\
100.1	26.9461077844311\\
100.3	27.0916334661355\\
100.4	27.0646766169154\\
100.600000000001	27.2095332671301\\
101.199999999999	27.0483711747285\\
101.299999999999	27.120315581854\\
101.4	27.0935960591133\\
101.499999999999	27.1653543307087\\
101.8	27.0853778213935\\
101.9	27.156862745098\\
102.9	26.8932038834951\\
103.1	27.0348837209302\\
104.600000000001	26.647564469914\\
104.700000000001	26.7175572519084\\
104.799999999999	26.6920877025739\\
104.9	26.7619047619048\\
105.299999999999	26.6603415559772\\
105.4	26.7298578199052\\
106	26.5786993402451\\
106.199999999999	26.7168391345249\\
106.299999999999	26.6917293233083\\
106.400000000001	26.7605633802817\\
106.500000000001	26.7354596622889\\
106.6	26.8041237113402\\
106.8	26.7539756782039\\
106.900000000001	26.8224299065421\\
107.4	26.6976744186046\\
107.500000000001	26.7657992565056\\
107.600000000001	26.7409470752089\\
107.799999999999	26.8767377201112\\
108.1	26.8022181146026\\
108.2	26.8698060941828\\
108.3	26.8450184501845\\
108.399999999999	26.9124423963134\\
109.7	26.5938069216758\\
109.799999999999	26.6606005459509\\
109.899999999999	26.6363636363636\\
109.999999999999	26.7029972752044\\
110.5	26.5822784810127\\
110.599999999999	26.6485998193315\\
110.899999999999	26.5765765765766\\
111	26.6426642664266\\
111.2	26.5947888589398\\
111.3	26.6606822262119\\
111.500000000001	26.6129032258064\\
111.6	26.6786034019696\\
111.8	26.6309204647006\\
111.899999999999	26.6964285714286\\
112.2	26.6251113089938\\
112.3	26.6903914590747\\
113.2	26.4783759929391\\
113.600000000001	26.7370272647318\\
113.7	26.713532513181\\
113.800000000001	26.7778753292362\\
114.900000000001	26.5217391304348\\
115	26.5855777584709\\
115.100000000001	26.5625\\
115.2	26.6261925411969\\
116.2	26.3972484952709\\
116.399999999999	26.5236051502146\\
117.300000000001	26.320272572402\\
117.400000000001	26.3829787234043\\
117.5	26.3605442176871\\
117.6	26.4231096006797\\
117.800000000001	26.3782866836302\\
117.899999999999	26.4406779661017\\
118.600000000001	26.284751474305\\
118.700000000001	26.3468013468013\\
119.000000000001	26.2804366078925\\
119.2	26.4040234702431\\
119.599999999999	26.3157894736842\\
119.699999999999	26.3772954924875\\
120.599999999999	26.1806130903065\\
120.699999999999	26.2417218543046\\
121.400000000001	26.0905349794239\\
121.5	26.1513157894737\\
121.7	26.1083743842365\\
121.799999999999	26.16899097621\\
121.9	26.1475409836066\\
122.000000000001	26.2080262080262\\
122.3	26.1437908496732\\
122.5	26.2642740619902\\
123.300000000001	26.0940032414911\\
123.5	26.2135922330097\\
123.700000000001	26.1712439418417\\
123.8	26.2308313155771\\
124.1	26.1674718196457\\
124.2	26.2268704746581\\
124.5	26.1637239165329\\
124.799999999999	26.3410728582866\\
125	26.2989608313349\\
125.1	26.3578274760383\\
125.3	26.3157894736842\\
125.4	26.3745019920319\\
126.399999999999	26.1660079051383\\
126.499999999999	26.2243285939968\\
126.9	26.1417322834646\\
127.000000000001	26.1998426435877\\
127.1	26.1792452830189\\
127.2	26.2372348782404\\
127.700000000001	26.1345852895149\\
127.800000000001	26.192337763878\\
127.9	26.171875\\
128	26.2295081967213\\
128.199999999999	26.1886204208885\\
128.300000000001	26.2461059190031\\
128.6	26.1849261849262\\
128.7	26.2422360248447\\
128.8	26.22187742436\\
128.900000000001	26.2790697674419\\
129.100000000001	26.2383900928793\\
129.299999999999	26.3523956723338\\
129.4	26.3320463320463\\
129.8	26.5588914549654\\
130.000000000001	26.5180630284397\\
130.100000000001	26.5745007680492\\
130.4	26.5134099616858\\
130.499999999999	26.5696784073507\\
131.099999999999	26.4481707317073\\
131.200000000001	26.5041888804265\\
131.899999999999	26.3636363636364\\
131.999999999999	26.4193792581378\\
132.1	26.3993948562784\\
132.199999999999	26.4550264550265\\
132.400000000001	26.4150943396226\\
132.5	26.4705882352941\\
132.6	26.4506405425772\\
132.7	26.5060240963855\\
133.099999999999	26.4264264264264\\
133.2	26.4816204051013\\
134.3	26.264880952381\\
134.4	26.3197026022305\\
134.9	26.2222222222222\\
135	26.2768319763138\\
135.3	26.218611521418\\
135.499999999999	26.3274336283186\\
135.799999999999	26.2693156732892\\
135.900000000001	26.3235294117647\\
136.099999999999	26.2848751835536\\
136.200000000001	26.3389581804842\\
136.799999999999	26.2235208181154\\
137.000000000001	26.3311451495259\\
137.300000000001	26.27365356623\\
137.4	26.3272727272727\\
137.700000000001	26.2699564586357\\
137.800000000001	26.3234227701233\\
138	26.2853005068791\\
138.100000000001	26.3386396526773\\
138.4	26.2815884476534\\
138.5	26.3347763347763\\
139.199999999999	26.2024407753051\\
139.3	26.2553802008608\\
139.7	26.1802575107296\\
140.100000000001	26.3908701854494\\
140.299999999999	26.3532763532764\\
140.399999999999	26.4056939501779\\
140.600000000001	26.3681592039801\\
140.699999999999	26.4204545454545\\
141.100000000001	26.3456090651558\\
141.2	26.3977353149328\\
141.299999999999	26.3790664780764\\
141.400000000001	26.4310954063604\\
141.5	26.4124293785311\\
141.599999999999	26.4643613267466\\
141.700000000001	26.4456981664316\\
141.799999999999	26.4975334742777\\
142.500000000001	26.367461430575\\
142.6	26.4190609670638\\
142.800000000001	26.3820853743877\\
142.900000000001	26.4335664335664\\
143	26.4150943396226\\
143.1	26.4664804469274\\
143.9	26.3194444444444\\
144	26.3705759888966\\
144.2	26.3340263340263\\
144.599999999999	26.5376641326883\\
144.9	26.4827586206897\\
144.999999999999	26.5334252239835\\
145.2	26.4969029593944\\
145.3	26.5474552957359\\
145.7	26.4746227709191\\
145.8	26.525017135024\\
146.399999999999	26.4163822525597\\
146.5	26.4665757162347\\
146.7	26.4305177111717\\
146.799999999999	26.4805990469707\\
147.4	26.3728813559322\\
147.600000000001	26.4725795531483\\
148.1	26.3832658569501\\
148.2	26.4329062710721\\
148.6	26.3618022864828\\
148.9	26.510067114094\\
149.499999999999	26.403743315508\\
149.6	26.4529058116232\\
149.7	26.4352469959947\\
149.800000000001	26.4843228819213\\
149.899999999999	26.4666666666667\\
150	26.5156562291805\\
150.1	26.4980026631158\\
150.199999999999	26.5469061876248\\
150.300000000001	26.5292553191489\\
150.6	26.6755142667551\\
150.800000000001	26.6401590457256\\
150.900000000001	26.6887417218543\\
151.000000000001	26.6710787557909\\
151.100000000001	26.7195767195767\\
151.399999999999	26.6666666666667\\
151.500000000001	26.7150395778364\\
151.600000000001	26.6974291364535\\
151.700000000001	26.7457180500659\\
151.799999999999	26.7281105990783\\
151.9	26.7763157894737\\
152.199999999999	26.7235718975706\\
152.3	26.7716535433071\\
152.500000000001	26.7365661861075\\
152.599999999999	26.784544859201\\
152.700000000001	26.7670157068063\\
152.800000000001	26.814911706998\\
153.1	26.7624020887728\\
153.4	26.9055374592834\\
153.699999999999	26.853055916775\\
153.799999999999	26.9005847953216\\
153.899999999999	26.8831168831169\\
153.999999999999	26.930564568462\\
154.099999999999	26.9130998702983\\
154.199999999999	26.9604666234608\\
154.399999999999	26.9255663430421\\
154.5	26.9728331177232\\
154.599999999999	26.9553975436328\\
154.7	27.0025839793282\\
155.7	26.8292682926829\\
155.799999999999	26.8762026940346\\
156.2	26.80742162508\\
156.300000000001	26.8542199488491\\
156.5	26.8199233716475\\
156.700000000001	26.9132653061224\\
156.999999999999	26.861871419478\\
157.1	26.9083969465649\\
157.699999999999	26.8060836501901\\
157.799999999999	26.8524382520583\\
158.1	26.8015170670038\\
158.200000000001	26.8477574226153\\
158.3	26.8308080808081\\
158.399999999999	26.8769716088328\\
158.7	26.8261964735516\\
158.8	26.8722466960352\\
159.500000000001	26.7543859649123\\
159.600000000001	26.8002504696306\\
160.2	26.6999376169682\\
160.3	26.7456359102244\\
160.500000000001	26.7123287671233\\
160.600000000001	26.7579340385812\\
161.3	26.6418835192069\\
161.400000000001	26.687306501548\\
161.499999999999	26.6707920792079\\
161.600000000001	26.7161410018553\\
161.799999999999	26.6831377393453\\
162.2	26.8638324091189\\
162.500000000001	26.8142681426814\\
162.599999999999	26.859250153657\\
162.899999999999	26.8098159509202\\
163	26.8546903740037\\
163.1	26.8382352941176\\
163.200000000001	26.8830373545622\\
163.699999999999	26.8009768009768\\
163.799999999999	26.8456375838926\\
164.299999999999	26.7639902676399\\
164.9	27.030303030303\\
165.400000000001	26.9486404833837\\
165.500000000001	26.9927536231884\\
165.6	26.9764634882317\\
165.700000000001	27.0205066344994\\
165.899999999999	26.9879518072289\\
166.000000000001	27.0319084888621\\
166.5	26.9507803121248\\
166.600000000001	26.994601079784\\
167.100000000001	26.9138755980861\\
167.2	26.9575612671847\\
167.499999999999	26.909307875895\\
167.600000000001	26.9528920691711\\
167.7	26.9368295589988\\
167.8	26.9803454437165\\
167.9	26.9642857142857\\
168.4	27.1810089020772\\
168.699999999999	27.132701421801\\
168.8	27.1758436944938\\
169.4	27.0796460176991\\
169.500000000001	27.122641509434\\
170.000000000001	27.0429159318048\\
170.1	27.0857814336075\\
170.700000000001	26.9906323185012\\
170.799999999999	27.0333528379169\\
172.399999999999	26.7826086956522\\
172.5	26.8250289687138\\
172.700000000001	26.7939814814815\\
172.9	26.878612716763\\
173.2	26.8320830929025\\
173.3	26.8742791234141\\
173.399999999999	26.8587896253602\\
173.5	26.9009216589862\\
173.999999999999	26.8236645605974\\
174.2	26.9076305220884\\
174.500000000001	26.8613974799542\\
174.599999999999	26.9032627361191\\
175.499999999999	26.7653758542141\\
175.6	26.8070574843483\\
175.999999999999	26.7461669505963\\
176.100000000001	26.7877412031782\\
176.4	26.7422096317281\\
176.5	26.7836919592299\\
177.099999999999	26.6930022573363\\
177.300000000001	26.7756482525366\\
177.6	26.7304445694992\\
177.800000000001	26.8128161888702\\
178.1	26.7676767676768\\
178.199999999999	26.8087492989344\\
178.299999999999	26.7937219730942\\
178.500000000001	26.8756998880179\\
178.699999999999	26.8456375838926\\
178.8	26.8865287870319\\
179.2	26.8265476854434\\
179.299999999999	26.8673355629877\\
179.399999999999	26.8523676880223\\
179.499999999999	26.8930957683742\\
180.200000000001	26.7886855241265\\
180.300000000001	26.8292682926829\\
182.100000000001	26.5642151481888\\
182.200000000001	26.6044980800878\\
182.399999999999	26.5753424657534\\
182.5	26.6155531215772\\
183.099999999999	26.528384279476\\
183.200000000001	26.5684669939989\\
184.2	26.4243081931633\\
184.3	26.4642082429501\\
184.699999999999	26.4069264069264\\
184.899999999999	26.4864864864865\\
185.100000000001	26.4578833693305\\
185.200000000001	26.4975715056665\\
185.3	26.4832793959008\\
185.5	26.5625\\
185.7	26.5339074273412\\
185.800000000001	26.5734265734266\\
186.200000000002	26.5163714439077\\
186.4	26.5951742627346\\
187.299999999999	26.4674493062967\\
187.4	26.5066666666667\\
187.5	26.4925373134328\\
187.7	26.5708200212993\\
187.8	26.5566790846195\\
187.899999999999	26.5957446808511\\
188.799999999999	26.4690312334569\\
188.900000000002	26.5079365079365\\
189.4	26.4379947229551\\
189.500000000001	26.4767932489451\\
189.699999999999	26.4488935721812\\
189.800000000001	26.4876250658241\\
190.500000000001	26.3903462749213\\
190.600000000001	26.4289459884636\\
190.8	26.4012572027239\\
191.099999999999	26.5167364016736\\
191.400000000001	26.4751958224543\\
191.500000000001	26.5135699373695\\
191.599999999999	26.4997391757955\\
191.700000000001	26.5380604796663\\
192	26.4966163456533\\
192.099999999999	26.5348595213319\\
192.499999999999	26.4797507788162\\
192.700000000001	26.5560165975104\\
192.799999999999	26.5422498703992\\
192.9	26.580310880829\\
193.299999999999	26.5253360910031\\
193.399999999999	26.5633074935401\\
193.9	26.4948453608247\\
193.999999999999	26.5327150953117\\
194.4	26.4781491002571\\
194.499999999999	26.5159301130524\\
194.599999999999	26.502311248074\\
194.899999999999	26.6153846153846\\
194.999999999999	26.6017426960533\\
195.3	26.7144319344933\\
195.500000000001	26.6871165644172\\
195.6	26.7245784363822\\
195.699999999999	26.7109295199183\\
195.800000000001	26.7483409903012\\
195.9	26.734693877551\\
196	26.7720550739419\\
196.199999999999	26.7447784004075\\
196.3	26.7820773930754\\
196.499999999999	26.7548321464903\\
196.699999999999	26.8292682926829\\
197.000000000001	26.7884322678843\\
197.2	26.862645717182\\
197.299999999999	26.8490374873354\\
197.4	26.8860759493671\\
197.499999999999	26.8724696356275\\
197.6	26.9094587759231\\
197.800000000001	26.8822637695806\\
197.900000000001	26.9191919191919\\
198.199999999999	26.8784669692385\\
198.299999999999	26.9153225806452\\
198.5	26.8882175226586\\
198.600000000001	26.9250125817816\\
198.699999999999	26.9114688128773\\
198.899999999999	26.9849246231156\\
200	26.8365817091454\\
200.099999999999	26.8731268731269\\
200.3	26.8463073852295\\
200.400000000001	26.8827930174564\\
200.6	26.8560039860488\\
200.700000000001	26.8924302788845\\
202.399999999999	26.6666666666667\\
202.7	26.7751479289941\\
202.9	26.7487684729064\\
202.999999999999	26.7848350566224\\
203.200000000001	26.7584849975406\\
203.299999999999	26.7944936086529\\
203.400000000001	26.7813267813268\\
203.499999999998	26.8172888015717\\
203.600000000001	26.8041237113402\\
203.699999999999	26.8400392541708\\
203.9	26.8137254901961\\
204.000000000001	26.8495835374816\\
205.9	26.6019417475728\\
206	26.6375545851528\\
206.2	26.6117304895783\\
206.3	26.6472868217054\\
206.400000000001	26.634382566586\\
206.499999999999	26.6698935140368\\
206.6	26.6569908079342\\
206.700000000001	26.6924564796905\\
206.799999999999	26.6795553407443\\
207	26.7503621438918\\
207.999999999998	26.6218164344065\\
208.2	26.6922707633221\\
208.900000000002	26.6028708133971\\
209.099999999998	26.6730401529637\\
209.599999999999	26.6094420600858\\
209.7	26.6444232602479\\
210.999999999999	26.4803410705827\\
211.099999999999	26.5151515151515\\
211.8	26.4275601698915\\
212	26.4969354078265\\
212.200000000001	26.4719736222327\\
212.300000000001	26.5065913370998\\
212.500000000001	26.4816556914393\\
212.599999999999	26.5162200282087\\
213.4	26.4168618266979\\
213.500000000002	26.4513108614232\\
214.000000000001	26.3895375992527\\
214.099999999999	26.4239028944911\\
214.200000000001	26.4115725618292\\
214.300000000001	26.4458955223881\\
214.9	26.3720930232558\\
215.000000000001	26.4063226406323\\
215.1	26.3940520446097\\
215.499999999999	26.530612244898\\
215.599999999999	26.5183124710246\\
215.899999999998	26.6203703703704\\
216.3	26.5711645101664\\
216.5	26.6389658356417\\
216.6	26.6266728195662\\
216.799999999999	26.6943291839557\\
218.299999999998	26.510989010989\\
218.500000000002	26.5782250686185\\
218.599999999999	26.5660722450846\\
218.700000000001	26.599634369287\\
219.999999999998	26.4425261244889\\
220.100000000001	26.4759309718438\\
220.599999999998	26.4159492523788\\
220.7	26.4492753623188\\
221.199999999999	26.3895164934478\\
221.3	26.4227642276423\\
221.699999999999	26.3751127141569\\
221.799999999999	26.408292023434\\
222	26.3845114813147\\
222.2	26.4507422402159\\
222.3	26.4388489208633\\
222.6	26.5379434216435\\
223.300000000001	26.4547896150403\\
223.5	26.520572450805\\
223.799999999999	26.4850379633765\\
223.900000000001	26.5178571428571\\
224.599999999999	26.4352469959947\\
224.700000000001	26.4679715302491\\
224.900000000002	26.4444444444444\\
225	26.4771212794314\\
225.4	26.430155210643\\
225.6	26.4953478068232\\
225.800000000002	26.4718902169101\\
225.899999999999	26.5044247787611\\
226.599999999999	26.4225849139832\\
226.699999999999	26.4550264550265\\
227.000000000001	26.4200792602378\\
227.099999999999	26.4524647887324\\
227.600000000001	26.3943785682916\\
227.799999999999	26.4589732338745\\
227.900000000001	26.4473684210526\\
228.100000000001	26.5118317265557\\
228.899999999999	26.4192139737991\\
229.100000000001	26.4834205933682\\
229.600000000001	26.4257727470614\\
229.700000000001	26.4577893820714\\
229.899999999998	26.4347826086956\\
230	26.4667535853977\\
230.1	26.4552562988706\\
230.4	26.5509761388286\\
230.5	26.539462272333\\
230.699999999999	26.6031195840555\\
230.999999999998	26.5685850281263\\
231.1	26.6003460207612\\
231.999999999999	26.4971994829815\\
232.100000000001	26.5288544358312\\
232.499999999999	26.4832330180568\\
232.600000000001	26.5148259561667\\
232.799999999998	26.4920566766853\\
232.899999999999	26.5236051502146\\
233.299999999999	26.4781491002571\\
233.500000000001	26.541095890411\\
233.900000000001	26.4957264957265\\
233.999999999998	26.527125160188\\
234.099999999999	26.5157984628523\\
234.199999999999	26.5471617584294\\
234.499999999998	26.5132139812447\\
234.600000000001	26.5445249254367\\
234.800000000002	26.5219242230737\\
235	26.5844321565291\\
235.2	26.5618359541011\\
235.300000000001	26.5930331350892\\
235.700000000001	26.5479219677693\\
235.799999999999	26.5790589232726\\
236.200000000001	26.5340668641557\\
236.3	26.5651438240271\\
239	26.2651610204935\\
239.299999999999	26.3575605680869\\
239.599999999999	26.3245723821443\\
239.800000000001	26.3859941642351\\
240.900000000002	26.2655601659751\\
240.999999999998	26.2961426793862\\
241.100000000001	26.2852404643449\\
241.299999999999	26.3463131731566\\
241.600000000002	26.3136119155979\\
241.700000000001	26.3440860215054\\
242.600000000001	26.2463947259992\\
242.699999999998	26.2767710049423\\
243.4	26.2012320328542\\
243.499999999999	26.2315270935961\\
243.599999999999	26.2207632334838\\
243.699999999998	26.2510254306809\\
244.300000000002	26.1865793780687\\
244.5	26.2469337694195\\
245	26.1933904528764\\
245.099999999999	26.2234910277325\\
245.200000000002	26.2128006522625\\
245.3	26.2428687856561\\
245.5	26.2214983713355\\
245.6	26.2515262515263\\
246.100000000001	26.1982128350934\\
246.200000000001	26.2281770198944\\
246.800000000001	26.1644390441474\\
246.9	26.1943319838057\\
247	26.1837312828814\\
247.1	26.2135922330097\\
247.299999999999	26.1924009700889\\
247.499999999998	26.2520193861066\\
247.999999999999	26.1991132607819\\
248.100000000001	26.228847703465\\
248.700000000001	26.1655948553055\\
248.799999999998	26.195259140217\\
249.3	26.1427425821973\\
249.4	26.1723446893788\\
249.599999999998	26.1513816579896\\
249.699999999999	26.1809447558046\\
249.900000000001	26.16\\
250	26.1895241903239\\
250.199999999999	26.1685976827807\\
250.299999999998	26.1980830670927\\
250.999999999999	26.1250497809638\\
251.1	26.1544585987261\\
252	26.0610868702896\\
252.100000000001	26.0904044409199\\
252.2	26.0800634165676\\
252.400000000001	26.1386138613861\\
252.999999999998	26.0766495456341\\
253.300000000001	26.1641673243883\\
253.6	26.1332282223098\\
253.7	26.1623325453113\\
254.100000000001	26.1211644374508\\
254.2	26.1502162799843\\
254.299999999998	26.1399371069182\\
254.4	26.1689587426326\\
255.3	26.0767423649178\\
255.400000000001	26.105675146771\\
255.600000000002	26.085256159562\\
255.700000000001	26.1141516810008\\
255.899999999999	26.09375\\
256	26.1226083561109\\
256.899999999999	26.0311284046693\\
256.999999999999	26.0598988720342\\
257.2	26.0396424407307\\
257.3	26.0683760683761\\
258.000000000002	25.9976753196435\\
258.200000000002	26.0549748354626\\
259.1	25.9645061728395\\
259.200000000001	25.9930582337061\\
259.8	25.9330511735283\\
259.900000000001	25.9615384615385\\
259.999999999998	25.9515570934256\\
260.099999999998	25.9800153727902\\
260.399999999999	25.9500959692898\\
260.600000000001	26.0069044879171\\
261.2	25.947187141217\\
261.400000000001	26.0038240917782\\
261.899999999998	25.9541984732824\\
262.199999999998	26.038886770873\\
262.599999999999	25.999238675295\\
262.700000000002	26.027397260274\\
263.099999999999	25.9878419452888\\
263.2	26.0159513862514\\
263.4	25.9962049335863\\
263.5	26.0242792109256\\
263.699999999999	26.0045489006823\\
263.799999999998	26.0325881015536\\
263.999999999998	26.0128739113972\\
264.100000000001	26.0408781226344\\
264.399999999999	26.0113421550095\\
264.5	26.0393046107332\\
264.700000000002	26.0196374622357\\
264.9	26.0754716981132\\
265.4	26.0263653483992\\
265.600000000001	26.0820474219044\\
265.800000000001	26.0624294847687\\
265.900000000001	26.0902255639098\\
266.1	26.0706235912848\\
266.199999999998	26.0983852797597\\
266.499999999999	26.0690172543136\\
266.600000000001	26.0967379077615\\
266.999999999999	26.0576563084987\\
267.100000000002	26.0853293413174\\
267.400000000002	26.0560747663551\\
267.500000000001	26.0837070254111\\
268.100000000002	26.0253542132737\\
268.2	26.0529258292956\\
268.4	26.0335195530726\\
268.5	26.0610573343261\\
268.8	26.031982149498\\
269.099999999998	26.1144130757801\\
270	26.027397260274\\
270.099999999999	26.0547742413027\\
270.199999999999	26.0451350351461\\
270.4	26.0998151571165\\
270.599999999999	26.0805319541928\\
270.700000000001	26.1078286558346\\
270.999999999999	26.0789376613796\\
271.099999999998	26.1061946902655\\
271.499999999998	26.0677466863034\\
271.599999999998	26.094957673905\\
272.000000000001	26.0565968393973\\
272.099999999998	26.0837619397502\\
272.400000000001	26.0550458715596\\
272.499999999999	26.0821716801174\\
272.9	26.0439560439561\\
272.999999999998	26.0710362504577\\
274.000000000002	25.9759211966436\\
274.200000000001	26.0298942763398\\
274.499999999998	26.0014566642389\\
274.600000000002	26.0283946123043\\
275.000000000002	25.9905488913123\\
275.099999999999	26.0174418604651\\
275.200000000002	26.0079912822376\\
275.300000000001	26.0348583877996\\
276.200000000001	25.9500542888165\\
276.400000000001	26.003616636528\\
277.900000000001	25.863309352518\\
278.100000000001	25.9166067577283\\
278.6	25.870111230714\\
278.700000000002	25.896700143472\\
278.799999999998	25.8874148440301\\
278.899999999998	25.9139784946237\\
279.299999999998	25.8768790264853\\
279.399999999999	25.9033989266547\\
279.499999999999	25.8941344778255\\
279.6	25.9206292456203\\
279.799999999999	25.902107895677\\
279.899999999998	25.9285714285714\\
280.400000000001	25.8823529411765\\
280.499999999999	25.9087669280114\\
280.599999999998	25.8995368721055\\
280.700000000002	25.9259259259259\\
281.200000000002	25.879843583363\\
281.300000000001	25.9061833688699\\
281.399999999998	25.8969804618117\\
281.499999999999	25.9232954545455\\
281.599999999998	25.9140930067448\\
282.100000000002	26.045357902197\\
282.300000000001	26.0269121813031\\
282.400000000001	26.0530973451327\\
282.500000000001	26.0438782731776\\
282.599999999999	26.0700389105058\\
282.900000000001	26.0424028268551\\
283	26.0685270222536\\
283.100000000002	26.0593220338983\\
283.199999999998	26.0854218143311\\
283.5	26.0578279266573\\
283.599999999998	26.083891434614\\
284.1	26.0380014074595\\
284.199999999999	26.0640168835737\\
284.500000000002	26.0365425158117\\
284.700000000002	26.0884831460674\\
285.000000000002	26.0610312171168\\
285.100000000001	26.0869565217391\\
285.4	26.0595446584939\\
285.500000000001	26.0854341736695\\
286.3	26.0125698324022\\
286.4	26.0383944153578\\
286.500000000002	26.0293091416609\\
286.600000000002	26.0551098709452\\
287.000000000001	26.0188087774295\\
287.100000000001	26.0445682451253\\
287.300000000002	26.026443980515\\
287.399999999999	26.0521739130435\\
287.5	26.0431154381085\\
287.6	26.0688216892597\\
287.700000000002	26.0597637248089\\
287.800000000001	26.0854463355332\\
287.900000000002	26.0763888888889\\
288	26.1020479000347\\
288.200000000001	26.0839403399237\\
288.299999999998	26.1095700416089\\
288.699999999999	26.0734072022161\\
288.8	26.0989961924541\\
289.599999999998	26.0269244045564\\
289.700000000001	26.0524499654934\\
289.8	26.0434632631942\\
289.9	26.0689655172414\\
290.099999999998	26.0509993108201\\
290.200000000001	26.0764726145367\\
290.299999999999	26.0674931129477\\
290.400000000001	26.0929432013769\\
290.699999999999	26.0660247592847\\
290.799999999999	26.0914403575112\\
291.899999999999	25.9931506849315\\
291.999999999999	26.0184868195823\\
292.199999999998	26.0006842285323\\
292.399999999998	26.0512820512821\\
292.499999999998	26.0423786739576\\
292.6	26.0676460539802\\
293.099999999999	26.0231923601637\\
293.200000000001	26.0484145925673\\
293.300000000001	26.0395364689843\\
293.4	26.0647359454855\\
293.500000000002	26.0558583106267\\
293.599999999999	26.0810350697991\\
293.700000000002	26.072157930565\\
293.799999999999	26.0973120108881\\
293.900000000001	26.0884353741497\\
294	26.1135668140088\\
294.199999999999	26.0958205912334\\
294.4	26.1460101867572\\
295.200000000002	26.0751777853031\\
295.299999999999	26.1002031144211\\
295.599999999999	26.0737233682787\\
295.699999999999	26.0987153482083\\
295.8	26.0898952348766\\
295.9	26.1148648648649\\
297.500000000001	25.9744623655914\\
297.599999999999	25.9993281827343\\
297.800000000001	25.9818731117825\\
297.999999999999	26.0315330426032\\
298.699999999999	25.9705488621151\\
298.800000000001	25.9953161592506\\
298.899999999998	25.9866220735786\\
298.999999999998	26.0113674356403\\
299.600000000001	25.9592926259593\\
299.7	25.9839893262175\\
300.2	25.9407259407259\\
300.300000000001	25.965379494008\\
300.399999999999	25.9567387687188\\
300.500000000001	25.981370592149\\
301.600000000001	25.8866423599602\\
301.699999999999	25.9111994698476\\
302.000000000002	25.885468387951\\
302.100000000002	25.9099933818663\\
302.299999999998	25.8928571428571\\
302.400000000001	25.9173553719008\\
302.900000000002	25.8745874587459\\
303.000000000002	25.8990432200594\\
303.599999999998	25.8478761936121\\
303.700000000003	25.87228439763\\
304.1	25.8382642998028\\
304.200000000001	25.8626355570161\\
304.800000000002	25.8117415546081\\
304.899999999998	25.8360655737705\\
305.100000000002	25.8191349934469\\
305.199999999999	25.8434326891582\\
305.800000000001	25.7927427263812\\
306.100000000001	25.8654474199869\\
306.500000000003	25.8317025440313\\
306.6	25.8558852298663\\
306.700000000001	25.8474576271186\\
306.900000000002	25.8957654723127\\
307.100000000001	25.87890625\\
307.200000000001	25.9030263586072\\
307.399999999998	25.8861788617886\\
307.599999999999	25.934351641209\\
308.799999999998	25.8336031078019\\
308.900000000002	25.8576051779935\\
309.000000000001	25.8492397282433\\
309.100000000001	25.8732212160414\\
309.5	25.8397932816537\\
309.700000000003	25.8876694641704\\
309.9	25.8709677419355\\
309.999999999999	25.8948726217349\\
310.3	25.8698453608247\\
310.399999999998	25.8937198067633\\
310.999999999998	25.8437801350048\\
311.099999999999	25.8676092544987\\
311.499999999999	25.8344030808729\\
311.599999999999	25.8581969842798\\
311.700000000002	25.8499037844772\\
311.800000000001	25.8736774607246\\
312.700000000002	25.7992327365729\\
312.800000000002	25.8229466283158\\
313	25.8064516129032\\
313.199999999999	25.853814235557\\
313.699999999999	25.8126195028681\\
313.900000000001	25.859872611465\\
314.500000000001	25.8105530832804\\
314.599999999998	25.8341277407054\\
314.7	25.8259212198221\\
314.8	25.8494760241346\\
315	25.8330688670263\\
315.100000000002	25.8565989847716\\
315.599999999998	25.8156477668673\\
315.999999999998	25.9095223030686\\
316.200000000001	25.8931394245969\\
316.3	25.9165613147914\\
317.600000000001	25.8105130626377\\
317.700000000001	25.8338577721838\\
317.9	25.8176100628931\\
318.200000000002	25.8875274897895\\
318.300000000002	25.8793969849246\\
318.4	25.9026687598116\\
318.699999999998	25.8782936010038\\
318.800000000001	25.9015365318282\\
319.000000000001	25.8853024130367\\
319.100000000001	25.9085213032581\\
319.400000000002	25.8841940532081\\
319.500000000002	25.9073842302879\\
320.100000000001	25.8588382261087\\
320.199999999998	25.8819856384639\\
320.800000000003	25.8335930196323\\
320.899999999999	25.8566978193146\\
321.100000000003	25.840597758406\\
321.200000000002	25.8636788048553\\
321.899999999998	25.8074534161491\\
322.1	25.8535071384233\\
323.200000000002	25.765542839468\\
323.300000000002	25.7884972170686\\
323.599999999998	25.7645968489342\\
323.699999999999	25.787523162446\\
323.8	25.7795615930843\\
324.000000000001	25.8253625424252\\
324.400000000001	25.7935285053929\\
324.500000000001	25.8163894023413\\
325.199999999999	25.760836151245\\
325.399999999998	25.8064516129032\\
325.799999999998	25.7747775391224\\
325.899999999999	25.7975460122699\\
326.100000000002	25.7817290006131\\
326.300000000002	25.8272058823529\\
326.400000000002	25.8192955589587\\
326.5	25.8420085731782\\
326.600000000002	25.8340985613713\\
326.700000000001	25.8567931456548\\
327.399999999998	25.8015267175573\\
327.499999999998	25.8241758241758\\
327.899999999999	25.7926829268293\\
328.000000000001	25.8153002133496\\
328.200000000001	25.7995735607676\\
328.299999999998	25.8221680876979\\
328.800000000002	25.7829127394345\\
328.900000000001	25.8054711246201\\
329.699999999998	25.7428744693754\\
329.799999999999	25.7653834495302\\
330.099999999999	25.7419745608722\\
330.299999999999	25.7869249394673\\
330.399999999998	25.7791225416036\\
330.499999999998	25.8015728977616\\
330.800000000001	25.7781807192505\\
331.000000000001	25.8230141951072\\
331.099999999999	25.8152173913043\\
331.200000000001	25.8376094174464\\
331.7	25.7986738999397\\
331.799999999998	25.8210304308527\\
332.100000000002	25.7977122215533\\
332.2	25.8200421306049\\
332.599999999998	25.7889990982868\\
332.699999999999	25.8112980769231\\
332.799999999998	25.8035446079904\\
332.900000000002	25.8258258258258\\
333	25.8180726508556\\
333.099999999998	25.8403361344538\\
333.499999999998	25.8093525179856\\
333.599999999999	25.8315852562182\\
334.100000000001	25.7929383602633\\
334.200000000003	25.8151361052946\\
334.299999999998	25.8074162679426\\
334.500000000002	25.851763299462\\
334.6	25.844039438303\\
334.7	25.8661887694146\\
335.200000000001	25.8276170593498\\
335.399999999999	25.8718330849478\\
336.499999999999	25.787284610814\\
336.599999999999	25.8093258093258\\
336.799999999999	25.7940041555358\\
337.199999999999	25.8820041506078\\
337.799999999999	25.8360461675052\\
337.899999999998	25.8579881656805\\
338.700000000003	25.7969303423849\\
338.800000000001	25.818825612275\\
339.000000000002	25.8035977587732\\
339.099999999999	25.8254716981132\\
339.499999999997	25.7950530035336\\
339.699999999999	25.83872866392\\
340.000000000002	25.8159364892679\\
340.100000000002	25.8377425044092\\
340.200000000001	25.8301498677637\\
340.4	25.8737151248164\\
340.500000000001	25.8661186142102\\
340.600000000001	25.8878778984444\\
341.599999999998	25.8121158911326\\
341.800000000002	25.8555133079848\\
342.3	25.8177570093458\\
342.600000000002	25.8826962357747\\
343.200000000001	25.8374599475677\\
343.399999999998	25.8806404657933\\
344.000000000002	25.8355129322871\\
344.099999999999	25.857059848925\\
344.500000000002	25.8270458502612\\
344.599999999998	25.8485639686684\\
346	25.7440046229413\\
346.100000000003	25.7654534950895\\
346.9	25.7060518731989\\
347	25.7274560645347\\
347.100000000002	25.7200460829493\\
347.300000000003	25.7628094415659\\
347.600000000001	25.7405809605982\\
347.7	25.7619321449109\\
347.900000000002	25.7471264367816\\
347.999999999998	25.7684573398449\\
348.100000000001	25.7610568638713\\
348.200000000002	25.7823715188056\\
348.500000000003	25.7601835915089\\
348.799999999999	25.8240183433649\\
349.199999999998	25.794446034927\\
349.3	25.8156840297653\\
349.800000000003	25.778793941126\\
349.899999999999	25.8\\
350	25.7926306769494\\
350.200000000002	25.834998572652\\
350.999999999998	25.7761321560809\\
351.100000000001	25.7972665148064\\
351.299999999999	25.7825839499146\\
351.399999999999	25.8036984352774\\
351.599999999999	25.7890247369918\\
351.799999999999	25.8312020460358\\
352.200000000003	25.8018734033494\\
352.300000000001	25.8229284903519\\
352.800000000001	25.7863417398697\\
353.000000000001	25.8283772302464\\
353.400000000002	25.7991513437058\\
353.699999999998	25.8620689655172\\
353.999999999998	25.840158147416\\
354.099999999998	25.8610954263128\\
354.300000000002	25.8465011286682\\
354.499999999997	25.8883248730965\\
354.9	25.8591549295775\\
355.100000000003	25.9009009009009\\
355.800000000003	25.8499578533296\\
355.999999999999	25.8916034821679\\
356.199999999999	25.8770698849284\\
356.399999999998	25.9186535764376\\
356.7	25.8968609865471\\
356.899999999997	25.9383753501401\\
357.5	25.8948545861298\\
357.6	25.9155717081353\\
357.999999999999	25.8866238480871\\
358.1	25.9073143495254\\
358.599999999998	25.8712015611932\\
358.799999999998	25.9125104485929\\
359.099999999997	25.890868596882\\
359.299999999998	25.9321090706733\\
359.5	25.9176863181313\\
359.700000000002	25.9588660366871\\
359.9	25.9444444444444\\
360	25.9650097195224\\
360.700000000003	25.9146341463415\\
360.799999999999	25.9351620947631\\
361.099999999998	25.9136212624585\\
361.200000000002	25.9341267644617\\
363.499999999999	25.7700770077008\\
363.600000000002	25.7904866648337\\
363.900000000001	25.7692307692308\\
364.099999999999	25.8099945085118\\
364.200000000002	25.8029096898161\\
364.299999999999	25.8232711306257\\
364.600000000003	25.8020290649849\\
364.700000000002	25.8223684210526\\
365.1	25.7940854326397\\
365.199999999999	25.8143991240077\\
365.399999999998	25.8002735978112\\
365.7	25.8611262985238\\
365.9	25.8469945355191\\
365.999999999997	25.8672493854138\\
366.400000000001	25.8390177353342\\
366.500000000002	25.8592471358429\\
366.900000000001	25.8310626702997\\
366.999999999998	25.851266684827\\
367.1	25.8442265795207\\
367.200000000002	25.8644160087122\\
367.399999999999	25.8503401360544\\
367.5	25.8705114254625\\
367.699999999999	25.8564437194127\\
367.899999999997	25.8967391304348\\
368.1	25.8826724606192\\
368.199999999998	25.9027966331795\\
368.3	25.8957654723127\\
368.7	25.9761388286334\\
369.399999999998	25.9269282814614\\
369.499999999999	25.9469696969697\\
370.400000000001	25.8839406207827\\
370.5	25.9039395574744\\
370.699999999998	25.8899676375405\\
370.999999999999	25.9498787388844\\
371.100000000003	25.9428879310345\\
371.199999999998	25.962833288446\\
371.299999999998	25.9558427571352\\
371.400000000001	25.9757738896366\\
371.799999999998	25.9478354396343\\
371.899999999999	25.9677419354839\\
372.1	25.9537882858678\\
372.199999999998	25.9736771420897\\
372.300000000001	25.9667024704619\\
372.400000000003	25.9865771812081\\
373.099999999999	25.9378349410504\\
373.199999999998	25.9576747923922\\
373.500000000002	25.9368308351178\\
373.600000000002	25.956649719026\\
373.700000000002	25.9497057249866\\
373.799999999999	25.969510564322\\
373.999999999999	25.9556268377439\\
374.200000000002	25.9951910232434\\
374.800000000001	25.9535876233662\\
375	25.9930685150627\\
375.299999999999	25.9722962173681\\
375.499999999998	26.0117145899894\\
375.7	25.9978712080894\\
375.9	26.0372340425532\\
376.099999999998	26.0233918128655\\
376.200000000001	26.0430507573744\\
376.300000000002	26.0361317747078\\
376.499999999999	26.0754115772703\\
376.600000000003	26.0684895142023\\
376.699999999998	26.088110403397\\
376.799999999998	26.0811886442027\\
376.899999999997	26.1007957559682\\
377.299999999997	26.0731319554849\\
377.400000000002	26.0927152317881\\
377.500000000002	26.0858050847458\\
377.700000000003	26.1249338274219\\
378.200000000002	26.0904044409199\\
378.300000000001	26.1099365750529\\
378.399999999999	26.1030383091149\\
378.499999999998	26.1225567881669\\
380.099999999998	26.0126249342451\\
380.2	26.0320799368919\\
381.200000000002	25.963808025177\\
381.3	25.9832197168327\\
381.6	25.9627980089075\\
381.700000000003	25.9821896280775\\
382.100000000001	25.9549973835688\\
382.199999999997	25.974365681402\\
382.4	25.9607843137255\\
382.500000000002	25.9801359121798\\
382.599999999999	25.9733472694016\\
382.800000000002	26.0120135805693\\
383.100000000002	25.9916492693111\\
383.199999999999	26.010957474563\\
383.299999999999	26.0041731872718\\
383.400000000003	26.0234680573664\\
383.9	25.9895833333333\\
384.000000000001	26.0088518614944\\
384.100000000002	26.0020822488287\\
384.299999999997	26.0405827263267\\
384.499999999998	26.0270410816433\\
384.599999999997	26.0462698206395\\
384.799999999997	26.0327357755261\\
384.900000000003	26.0519480519481\\
385.400000000002	26.0181582360571\\
385.499999999998	26.0373443983403\\
385.8	26.0171028763929\\
385.900000000003	26.0362694300518\\
386.100000000001	26.0227861211807\\
386.400000000001	26.0802069857697\\
386.499999999998	26.0734609415416\\
386.599999999997	26.092578226015\\
386.799999999998	26.0790902041871\\
386.899999999998	26.0981912144703\\
387.400000000003	26.0645161290323\\
387.599999999999	26.1026566933196\\
387.999999999997	26.0757536717341\\
388.099999999998	26.0947964966512\\
388.299999999998	26.081359423275\\
388.400000000002	26.1003861003861\\
388.700000000001	26.0802469135802\\
388.800000000001	26.0992543070198\\
389.7	26.03899435608\\
389.799999999997	26.0579635804052\\
389.900000000003	26.0512820512821\\
390.100000000002	26.0891850333162\\
391.000000000002	26.0291485553567\\
391.200000000001	26.0669562995144\\
391.299999999999	26.060296371998\\
391.600000000002	26.1169262190452\\
392.699999999999	26.0437881873727\\
392.8	26.0626113514889\\
392.899999999998	26.0559796437659\\
392.999999999998	26.074790129738\\
393.100000000001	26.0681586978637\\
393.2	26.0869565217391\\
393.700000000001	26.0538344337227\\
393.800000000003	26.0726072607261\\
394.999999999997	25.9934193874968\\
395.2	26.0308626359727\\
395.300000000001	26.0242792109256\\
395.500000000002	26.061678463094\\
395.900000000002	26.0353535353535\\
396.100000000001	26.0726905603231\\
396.399999999997	26.0529634300126\\
396.5	26.0716086737267\\
396.900000000001	26.0453400503778\\
397.000000000001	26.0639637370939\\
397.499999999999	26.0311871227364\\
397.699999999999	26.0683760683761\\
397.799999999999	26.06182457904\\
397.9	26.0804020100503\\
398.200000000001	26.0607582224454\\
398.300000000002	26.0793172690763\\
398.600000000003	26.0596940055179\\
398.799999999999	26.0967661067937\\
399.000000000003	26.083688298672\\
399.099999999997	26.1022044088176\\
399.399999999997	26.0826032540676\\
399.500000000002	26.1011011011011\\
399.799999999999	26.081520380095\\
399.900000000001	26.1\\
400.400000000001	26.0674157303371\\
400.499999999999	26.0858711932102\\
400.800000000002	26.0663507109005\\
400.900000000002	26.0847880299252\\
401.599999999999	26.0393328354493\\
401.699999999998	26.0577401692384\\
402.200000000001	26.0253542132737\\
402.400000000003	26.0621118012422\\
402.899999999999	26.029776674938\\
403.000000000002	26.0481270156289\\
403.299999999998	26.0287555775905\\
403.500000000003	26.0654112983152\\
403.600000000001	26.0589546693089\\
403.700000000002	26.0772659732541\\
404.000000000002	26.0579064587973\\
404.099999999997	26.0761999010391\\
404.200000000003	26.0697501855058\\
404.300000000001	26.0880316518299\\
405.199999999998	26.0301011596348\\
405.299999999998	26.0483473112975\\
405.600000000001	26.0290855311807\\
405.700000000001	26.0473139477575\\
405.800000000001	26.0408967726041\\
405.899999999999	26.0591133004926\\
406.300000000001	26.0334645669291\\
406.499999999998	26.0698475159862\\
406.800000000002	26.0506266896043\\
406.900000000003	26.0687960687961\\
407.5	26.0304219823356\\
407.600000000002	26.0485651214128\\
407.700000000003	26.0421775380088\\
407.800000000001	26.06030889924\\
408.200000000001	26.034778349253\\
408.299999999998	26.052889324192\\
408.400000000002	26.046511627907\\
408.499999999999	26.064610866373\\
408.799999999999	26.0454878943507\\
409.100000000002	26.099706744868\\
409.699999999999	26.0614934114202\\
409.799999999998	26.0795315930715\\
409.899999999998	26.0731707317073\\
410.000000000002	26.0911972689588\\
410.1	26.0848366650414\\
410.199999999999	26.1028515720205\\
410.499999999999	26.0837798343887\\
410.600000000001	26.1017774531288\\
411.400000000002	26.0510328068044\\
411.5	26.0689990281827\\
411.900000000003	26.0436893203883\\
412.300000000002	26.1154219204656\\
412.399999999999	26.1090909090909\\
412.499999999997	26.126999515269\\
412.800000000003	26.1080164688787\\
412.900000000002	26.1259079903148\\
413.2	26.1069441083958\\
413.300000000001	26.1248185776488\\
413.400000000003	26.1185006045949\\
413.500000000001	26.1363636363636\\
413.599999999998	26.1300459270002\\
413.700000000003	26.1478975350411\\
414.300000000002	26.1100386100386\\
414.4	26.1278648974668\\
414.600000000002	26.1152640462985\\
414.799999999999	26.1508797300554\\
415.000000000001	26.1382799325464\\
415.100000000003	26.1560693641619\\
415.2	26.149771249699\\
415.300000000003	26.1675493500241\\
415.5	26.1549566891242\\
415.600000000002	26.172720712052\\
415.799999999998	26.1601346477519\\
415.900000000001	26.1778846153846\\
416.100000000003	26.1653051417588\\
416.200000000001	26.183041076147\\
416.299999999998	26.1767531219981\\
416.4	26.1944777911164\\
416.500000000002	26.1881901104177\\
416.600000000001	26.2059035277178\\
416.899999999998	26.1870503597122\\
416.999999999998	26.2047470630544\\
417.099999999999	26.1984659635666\\
417.199999999999	26.2161514497963\\
417.299999999999	26.2098706276953\\
417.399999999999	26.2275449101796\\
418.299999999999	26.1711281070746\\
418.499999999999	26.2064022933588\\
418.9	26.18138424821\\
419.099999999999	26.2166030534351\\
419.699999999999	26.1791329204383\\
419.799999999998	26.1967135032151\\
420.300000000003	26.1655566127498\\
420.600000000002	26.2182077489898\\
420.999999999999	26.193303253384\\
421.099999999998	26.2108262108262\\
421.699999999997	26.1735419630156\\
421.800000000004	26.1910405309315\\
421.900000000003	26.1848341232227\\
422.099999999997	26.2198010421601\\
422.199999999998	26.2135922330097\\
422.399999999997	26.2485207100592\\
422.800000000003	26.2236935445732\\
423.000000000001	26.2585677144883\\
423.100000000003	26.2523629489603\\
423.200000000001	26.2697850224427\\
423.999999999999	26.220231077576\\
424.099999999999	26.2376237623762\\
425.899999999999	26.1267605633803\\
426.000000000003	26.1440976296644\\
426.2	26.1318320431621\\
426.299999999998	26.1491557223265\\
426.599999999997	26.130771033513\\
426.7	26.1480787253983\\
426.800000000001	26.1419536191145\\
426.899999999999	26.1592505854801\\
427.000000000004	26.1531257316788\\
427.100000000003	26.1704119850187\\
427.300000000003	26.1581656527843\\
427.500000000004	26.1927034611787\\
427.699999999997	26.1804581580178\\
428.000000000002	26.2321887409484\\
428.799999999997	26.1832595010492\\
428.900000000001	26.2004662004662\\
429.100000000004	26.18825722274\\
429.3	26.2226362366092\\
429.499999999997	26.2104283054004\\
429.6	26.2276006516174\\
429.699999999997	26.2214983713355\\
429.800000000001	26.2386601535241\\
430.299999999999	26.2081784386617\\
430.500000000002	26.2424523920111\\
430.999999999998	26.2120157736024\\
431.099999999999	26.2291280148423\\
431.199999999997	26.2230466032924\\
431.299999999998	26.2401483541956\\
431.900000000001	26.2037037037037\\
432.000000000001	26.2207822263365\\
432.199999999999	26.2086513994911\\
432.400000000001	26.242774566474\\
432.5	26.2367082755432\\
432.600000000001	26.2537554887913\\
432.799999999999	26.2416262416262\\
432.899999999997	26.2586605080831\\
433.799999999998	26.2041945148652\\
433.900000000003	26.221198156682\\
433.999999999999	26.2151577977425\\
434.099999999999	26.2321510824505\\
434.800000000002	26.1899287192458\\
434.9	26.2068965517241\\
435.200000000002	26.1888352860096\\
435.300000000001	26.2057877813505\\
436.199999999999	26.1517304606922\\
436.300000000003	26.1686526122823\\
436.9	26.1327231121281\\
437.199999999999	26.1833981248571\\
438.099999999997	26.1296211775445\\
438.2	26.1464750171116\\
439	26.0988385333637\\
439.099999999998	26.115664845173\\
439.900000000001	26.0681818181818\\
439.999999999997	26.0849806862077\\
440.100000000001	26.0790549750114\\
440.399999999998	26.1293984108967\\
440.600000000001	26.1175402768323\\
440.699999999999	26.1343012704174\\
441.1	26.1106074342702\\
441.4	26.1608154020385\\
443.000000000001	26.0663507109005\\
443.099999999999	26.0830324909747\\
443.299999999999	26.0712674785747\\
443.399999999999	26.0879368658399\\
443.5	26.0820559062218\\
443.599999999998	26.0987153482083\\
444.300000000001	26.0576057605761\\
444.399999999999	26.07424071991\\
444.500000000002	26.0683760683761\\
444.6	26.0850011243535\\
444.899999999998	26.0674157303371\\
444.999999999999	26.0840260615592\\
445.700000000002	26.043068640646\\
445.800000000001	26.0596546310832\\
445.899999999997	26.0538116591928\\
446.000000000003	26.0703878054248\\
446.100000000002	26.0645450470641\\
446.199999999997	26.0811113600717\\
446.3	26.0752688172043\\
446.399999999999	26.0918253079507\\
446.800000000003	26.0684716938912\\
446.9	26.0850111856823\\
447.899999999997	26.0267857142857\\
448.099999999997	26.0597947344935\\
448.499999999997	26.0365581810076\\
448.699999999997	26.0695187165775\\
449.400000000002	26.0289210233593\\
449.500000000004	26.0453736654804\\
450.900000000002	25.9645232815965\\
451.100000000003	25.9973404255319\\
451.299999999997	25.9858218874612\\
451.400000000004	26.0022148394241\\
451.700000000002	25.9849490925188\\
451.900000000002	26.0176991150442\\
452.1	26.0061919504644\\
452.300000000003	26.0389036251105\\
452.499999999997	26.027397260274\\
452.599999999998	26.0437375745527\\
452.700000000004	26.0379858657244\\
453.100000000003	26.1032656663725\\
453.199999999997	26.0975071696448\\
453.300000000002	26.1138067931187\\
453.499999999997	26.1022927689594\\
453.600000000001	26.1185805598413\\
454.000000000002	26.0955736621889\\
454.100000000003	26.1118450022017\\
454.300000000001	26.1003521126761\\
454.400000000003	26.1166116611661\\
454.600000000003	26.1051242577524\\
454.7	26.1213720316623\\
454.800000000002	26.1156298087492\\
454.900000000004	26.1318681318681\\
455.200000000003	26.1146496815287\\
455.300000000003	26.1308739569609\\
456.100000000002	26.085050416484\\
456.400000000001	26.1336254107338\\
456.499999999998	26.1279018834866\\
456.599999999998	26.1440770746661\\
457.000000000001	26.1211988623933\\
457.1	26.1373578302712\\
457.299999999998	26.1259291648448\\
457.600000000001	26.1743500109242\\
457.900000000001	26.1572052401747\\
458.000000000001	26.1733246016154\\
458.300000000001	26.1561954624782\\
458.400000000003	26.1723009814613\\
458.499999999997	26.1665939816834\\
458.600000000002	26.1826902114672\\
458.899999999997	26.1655773420479\\
459.000000000003	26.1816597691135\\
459.200000000002	26.1702590899194\\
459.299999999999	26.1863299956465\\
459.700000000001	26.163549369291\\
460	26.2116931101934\\
460.3	26.1946133796699\\
460.500000000001	26.2266608771168\\
460.7	26.2152777777778\\
460.800000000001	26.2312866131482\\
461	26.2199089134678\\
461.099999999997	26.2359063313096\\
461.299999999997	26.2245340268747\\
461.399999999998	26.240520043337\\
461.499999999999	26.234835355286\\
461.800000000003	26.28274518294\\
461.999999999997	26.2713698333694\\
462.199999999999	26.3032662773091\\
463.100000000003	26.2521588946459\\
463.199999999998	26.2680768400604\\
463.300000000004	26.2624082865775\\
463.400000000002	26.2783171521036\\
463.900000000002	26.25\\
463.999999999998	26.2658909717733\\
464.200000000003	26.2545767822529\\
464.299999999998	26.2704565030146\\
464.899999999998	26.236559139785\\
464.999999999999	26.2524188346592\\
465.100000000001	26.2467755803955\\
465.200000000002	26.2626262626263\\
465.299999999997	26.2569832402235\\
465.400000000003	26.2728249194415\\
465.500000000004	26.2671821305842\\
465.700000000002	26.2988407041649\\
465.799999999997	26.2931959647993\\
465.999999999999	26.3248229993564\\
466.100000000001	26.3191763191763\\
466.200000000001	26.3349774823075\\
466.700000000001	26.3067694944302\\
466.799999999999	26.3225530092097\\
466.899999999998	26.3169164882227\\
467.199999999999	26.3642199871603\\
467.700000000001	26.3360410431808\\
467.800000000002	26.3517845693524\\
467.999999999997	26.3405255287332\\
468.100000000002	26.3562580093977\\
468.300000000003	26.3450042698548\\
468.399999999999	26.3607257203842\\
469.400000000004	26.3045793397231\\
469.600000000004	26.3359591228444\\
470.000000000001	26.313550308445\\
470.2	26.3448862428237\\
470.700000000004	26.3169073916738\\
470.799999999999	26.3325546825228\\
471.500000000002	26.2934690415606\\
471.599999999998	26.3090947636209\\
471.699999999998	26.303518440017\\
471.999999999997	26.3503495022241\\
472.500000000001	26.322471434617\\
472.6	26.3380579648826\\
472.699999999998	26.3324873096447\\
472.800000000004	26.3480651300486\\
473.399999999998	26.3146779303062\\
473.6	26.3457884737175\\
473.699999999997	26.3402279442803\\
473.800000000001	26.3557712597594\\
473.900000000001	26.3502109704641\\
473.999999999997	26.3657456232862\\
474.400000000004	26.3435194942044\\
474.600000000003	26.3745523488519\\
474.700000000003	26.3689974726201\\
474.800000000003	26.3845020004211\\
475.5	26.3456686291001\\
475.6	26.3611519865461\\
476.600000000003	26.3058527375708\\
476.700000000002	26.3213087248322\\
477.200000000001	26.2937355960612\\
477.300000000003	26.3091746962715\\
477.899999999997	26.2761506276151\\
478.099999999998	26.3069845253032\\
479.199999999997	26.246609639057\\
479.300000000004	26.2619941593659\\
480.499999999997	26.1964211402414\\
480.599999999999	26.2117744955274\\
480.800000000002	26.2008733624454\\
480.900000000002	26.2162162162162\\
481.100000000003	26.2053200332502\\
481.200000000004	26.2206523997507\\
481.500000000002	26.2043189368771\\
481.700000000003	26.2349522623495\\
481.900000000001	26.2240663900415\\
481.999999999997	26.2393694254304\\
482.599999999998	26.2067536772322\\
482.700000000002	26.2220381110191\\
482.899999999997	26.2111801242236\\
483.099999999999	26.2417218543046\\
483.499999999997	26.2200165425972\\
483.600000000003	26.2352697953277\\
484.900000000001	26.1649484536083\\
484.999999999996	26.1801690373119\\
485.7	26.1424454508028\\
485.799999999999	26.1576456060918\\
486.000000000001	26.1468833573339\\
486.100000000002	26.1620732208968\\
486.399999999996	26.1459403905447\\
486.499999999998	26.1611179613646\\
486.899999999999	26.1396303901437\\
487.099999999997	26.1699507389163\\
488.099999999997	26.1163457599345\\
488.300000000001	26.1466011466011\\
488.600000000004	26.1305504399427\\
488.699999999998	26.1456628477905\\
489.000000000001	26.1296258433858\\
489.100000000003	26.1447260834015\\
489.800000000003	26.1073688507859\\
489.900000000004	26.1224489795918\\
490.300000000002	26.1011419249592\\
490.400000000004	26.1162079510703\\
490.599999999999	26.1055634807418\\
490.700000000003	26.120619396903\\
490.899999999999	26.1099796334012\\
491	26.1250254530646\\
491.299999999999	26.1090761090761\\
491.399999999997	26.1241098677518\\
492.000000000002	26.092257671205\\
492.100000000004	26.1072734660707\\
492.400000000002	26.0913705583756\\
492.500000000003	26.1063743402355\\
492.599999999999	26.1010757052973\\
492.899999999998	26.1460446247465\\
493	26.1407422429527\\
493.099999999999	26.1557177615572\\
493.200000000003	26.1504155686195\\
493.300000000004	26.1653830563437\\
494.200000000001	26.1177422617843\\
494.300000000004	26.1326860841424\\
494.800000000004	26.1062840977975\\
494.900000000003	26.1212121212121\\
494.999999999997	26.1159361745102\\
495.200000000001	26.1457702402584\\
495.299999999999	26.1404925312878\\
495.499999999996	26.1702986279257\\
495.900000000002	26.1491935483871\\
496.100000000003	26.1789600967352\\
496.400000000001	26.1631419939577\\
496.500000000002	26.1780104712042\\
496.899999999997	26.1569416498994\\
497.100000000002	26.1866452131939\\
497.599999999997	26.1603375527426\\
497.699999999997	26.1751707513057\\
497.899999999999	26.1646586345382\\
498.000000000002	26.1794820317205\\
498.100000000001	26.1742272179848\\
498.2	26.1890427453341\\
498.399999999999	26.1785356068205\\
498.499999999999	26.1933413557962\\
498.6	26.1880890314819\\
498.699999999997	26.2028869286287\\
499.700000000001	26.1504601840736\\
499.800000000003	26.1652330466093\\
500.100000000003	26.1495401839264\\
500.199999999998	26.1643014191485\\
500.900000000002	26.127744510978\\
501.000000000002	26.1424865296348\\
501.399999999998	26.1216350947158\\
501.499999999999	26.1363636363636\\
501.600000000001	26.1311540761411\\
501.699999999997	26.1458748505381\\
501.9	26.1354581673307\\
501.999999999998	26.1501692889863\\
503.500000000003	26.0722795869738\\
503.600000000004	26.0869565217391\\
503.699999999999	26.0817784835252\\
504.000000000001	26.1257686966872\\
504.700000000001	26.0895404120444\\
504.899999999999	26.1188118811881\\
505.000000000003	26.1136408631954\\
505.199999999999	26.1428854146052\\
505.700000000003	26.1170423092131\\
505.9	26.1462450592885\\
506.399999999999	26.1204343534057\\
506.499999999997	26.1350177654955\\
507.000000000001	26.1092486689016\\
507.100000000004	26.1238170347003\\
508.000000000003	26.0775437905924\\
508.100000000001	26.0920897284534\\
508.500000000003	26.0715690129768\\
508.599999999997	26.0861018281895\\
508.8	26.0758498722735\\
508.899999999998	26.090373280943\\
509.299999999997	26.0698861405575\\
509.400000000001	26.0843964671246\\
510.099999999998	26.0486083888671\\
510.199999999996	26.0631001371742\\
510.800000000002	26.0324916813466\\
510.900000000002	26.0469667318982\\
511.099999999999	26.0367762128326\\
511.200000000004	26.0512419323294\\
511.599999999999	26.030877467266\\
511.699999999997	26.0453302071122\\
511.899999999996	26.03515625\\
512.000000000001	26.049599687561\\
512.700000000003	26.0140405616225\\
512.799999999996	26.0284655878339\\
513.999999999997	25.9677105621474\\
514.100000000004	25.9821081291326\\
514.6	25.9568680784923\\
514.699999999997	25.971250971251\\
515.200000000001	25.9460508441684\\
515.300000000002	25.9604190919674\\
516.000000000003	25.9252082929665\\
516.099999999997	25.9395583107323\\
516.5	25.9194734804491\\
516.700000000003	25.9481424148607\\
516.899999999997	25.9381044487427\\
516.999999999997	25.9524269967124\\
517.5	25.9273570324575\\
517.600000000002	25.9416650569828\\
517.700000000001	25.9366550791812\\
517.800000000004	25.9509557829697\\
518.799999999998	25.9009443052611\\
519	25.9294933538817\\
519.500000000003	25.9045419553503\\
519.700000000004	25.9330511735283\\
520.199999999997	25.9081299250432\\
520.300000000004	25.9223674096849\\
521.400000000001	25.8676893576222\\
521.600000000001	25.8961088748323\\
521.799999999998	25.8861850929297\\
521.900000000003	25.9003831417625\\
521.999999999998	25.8954223328864\\
522.199999999997	25.9237985831897\\
522.4	25.9138755980861\\
522.5	25.928052047455\\
522.600000000001	25.9230916395638\\
522.699999999997	25.9372609028309\\
522.800000000002	25.9323006310958\\
522.900000000002	25.9464627151052\\
523.000000000004	25.9415025807685\\
523.100000000001	25.9556574923547\\
523.599999999999	25.9308764559863\\
523.700000000003	25.9450171821306\\
523.899999999999	25.9351145038168\\
524.100000000002	25.9633727584891\\
524.900000000003	25.9238095238095\\
525	25.9379165873167\\
525.100000000003	25.9329779131759\\
525.199999999999	25.9470778602703\\
525.300000000003	25.942139322421\\
525.400000000004	25.9562321598478\\
525.999999999997	25.9266299182665\\
526.199999999999	25.9547786433593\\
526.300000000004	25.9498480243161\\
526.400000000001	25.9639126305793\\
526.600000000001	25.9540535409151\\
526.800000000001	25.9821598026191\\
526.900000000003	25.977229601518\\
527.099999999999	26.00531107739\\
527.599999999997	25.9806708357021\\
528.099999999997	26.0507383566831\\
528.699999999999	26.0211800302572\\
528.799999999997	26.0351673284175\\
529.400000000004	26.0056657223796\\
529.500000000001	26.0196374622357\\
529.699999999999	26.0098150245376\\
529.800000000001	26.0237780713342\\
530.399999999999	25.9943449575872\\
530.500000000002	26.0082924990577\\
530.600000000002	26.0033917467496\\
530.800000000001	26.0312676586928\\
531.4	26.0018814675447\\
531.499999999997	26.0158013544018\\
532.500000000004	25.9669545625235\\
532.699999999998	25.9947447447448\\
532.999999999998	25.9801163008816\\
533.099999999996	25.9939984996249\\
533.500000000001	25.9745127436282\\
533.700000000004	26.0022480329712\\
533.999999999999	25.9876427635274\\
534.099999999998	26.0014975664545\\
535.400000000001	25.9383753501401\\
535.599999999997	25.966025760687\\
535.700000000002	25.9611795446062\\
535.799999999996	25.9749953349506\\
536.299999999998	25.9507829977629\\
536.399999999999	25.9645852749301\\
537.400000000001	25.9162790697674\\
537.500000000003	25.9300595238095\\
538.100000000003	25.9011519881085\\
538.199999999997	25.9149173323426\\
538.599999999998	25.8956747726007\\
538.799999999996	25.9231768417146\\
538.999999999996	25.9135596364311\\
539.099999999997	25.9272997032641\\
539.599999999999	25.9032795997777\\
539.699999999998	25.9170062986291\\
539.900000000005	25.9074074074074\\
540.000000000001	25.9211257174597\\
541.400000000002	25.854108956602\\
541.5	25.8677991137371\\
541.999999999997	25.8439402324294\\
542.100000000003	25.8576171154556\\
542.300000000005	25.8480825958702\\
542.400000000003	25.8617511520737\\
542.499999999998	25.8569848875783\\
542.6	25.8706467661692\\
543.500000000002	25.8278145695364\\
543.600000000004	25.8414566856722\\
543.9	25.8272058823529\\
544.100000000002	25.8544652701213\\
544.199999999997	25.8497152305714\\
544.399999999997	25.8769513314968\\
544.699999999998	25.8627019089574\\
544.900000000002	25.8899082568807\\
545.5	25.8614369501466\\
545.800000000001	25.9021798864261\\
546.399999999998	25.8737419945105\\
546.499999999998	25.8873033296744\\
546.600000000004	25.8825681360893\\
546.7	25.8961228968544\\
547.099999999998	25.8771929824561\\
547.199999999996	25.8907363420428\\
548.199999999999	25.8435163231807\\
548.299999999997	25.8570386579139\\
549.499999999997	25.8005822416303\\
549.799999999997	25.8410620112748\\
549.999999999996	25.8316669696419\\
550.199999999997	25.8586225695075\\
550.499999999997	25.8445332364693\\
550.600000000002	25.8579989104776\\
551.499999999996	25.8158085569253\\
551.600000000004	25.8292550299076\\
551.699999999996	25.8245741210584\\
551.799999999997	25.8380141329951\\
551.899999999999	25.8333333333333\\
552.100000000001	25.8601955813111\\
552.600000000001	25.8368011579519\\
552.699999999997	25.8502170767004\\
552.999999999996	25.8361959862593\\
553.099999999998	25.8496023138106\\
553.2	25.844930417495\\
553.500000000003	25.8851156069364\\
553.6	25.880440671844\\
553.700000000003	25.8938244853738\\
553.9	25.884476534296\\
554	25.8978523732178\\
554.299999999997	25.8838383838384\\
554.399999999997	25.8972046889089\\
554.800000000003	25.8785366732745\\
554.899999999997	25.8918918918919\\
555.2	25.8779038357645\\
555.300000000002	25.891249549874\\
555.699999999998	25.8726160489385\\
555.800000000001	25.8859507105595\\
555.900000000001	25.8812949640288\\
556.099999999998	25.9079467817332\\
556.300000000002	25.8986340762042\\
556.499999999999	25.9252605102408\\
556.600000000001	25.9206035566733\\
556.700000000003	25.933908045977\\
557.2	25.9106405885519\\
557.500000000003	25.9505021520803\\
558.400000000005	25.9086839749329\\
558.500000000004	25.921947726459\\
558.700000000004	25.9126700071582\\
558.900000000001	25.9391771019678\\
560.300000000002	25.8743754461099\\
560.400000000003	25.8876003568243\\
561.100000000004	25.8553100498931\\
561.400000000005	25.8949243098842\\
562.100000000004	25.8626823194593\\
562.200000000001	25.8758669749244\\
562.299999999998	25.871266002845\\
562.400000000005	25.8844444444444\\
562.5	25.879843583363\\
562.699999999996	25.9061833688699\\
562.799999999996	25.9015810978859\\
562.999999999998	25.9278991298171\\
563.4	25.9094942324756\\
563.599999999997	25.9357814440305\\
563.699999999998	25.9311812699539\\
563.799999999999	25.9443163681504\\
564.300000000001	25.921332388377\\
564.500000000005	25.9475735033652\\
564.700000000005	25.9383852691218\\
564.800000000001	25.9514958399717\\
565.000000000003	25.9423110953813\\
565.099999999996	25.9554140127389\\
565.299999999996	25.9462327555713\\
565.400000000002	25.9593280282935\\
566.200000000003	25.9226558361293\\
566.299999999996	25.9357344632768\\
566.399999999997	25.9311562224184\\
566.499999999997	25.9442287327921\\
567.500000000002	25.8985200845666\\
567.599999999999	25.9115730139158\\
567.800000000005	25.9024476140166\\
567.900000000002	25.9154929577465\\
568.1	25.9063709961281\\
568.200000000001	25.9194087629773\\
568.9	25.8875219683656\\
569.000000000003	25.9005447197329\\
570.3	25.8415147265077\\
570.399999999999	25.8545135845749\\
571.500000000003	25.8047585724283\\
571.600000000003	25.8177365751268\\
572.4	25.7816593886463\\
572.6	25.8075781386415\\
572.700000000003	25.8030726256983\\
572.799999999999	25.8160237388724\\
573.399999999996	25.7890148212729\\
573.499999999998	25.8019525801953\\
573.899999999999	25.7839721254355\\
574.000000000003	25.7968994948615\\
574.600000000003	25.7699669392727\\
574.799999999999	25.7957905722734\\
575.100000000005	25.7823365785814\\
575.2	25.7952372675126\\
576.199999999999	25.7504771820233\\
576.299999999996	25.763358778626\\
577.099999999997	25.7276507276507\\
577.199999999996	25.7405161960852\\
577.299999999998	25.7360581918947\\
577.399999999999	25.7489177489177\\
577.999999999996	25.7221933921467\\
578.199999999996	25.7478817222895\\
578.299999999998	25.7434301521438\\
578.399999999997	25.7562662057044\\
579.199999999997	25.7206973934058\\
579.300000000003	25.733517431826\\
579.400000000003	25.7290767903365\\
579.5	25.7418909592823\\
580.599999999998	25.6931289822628\\
580.800000000001	25.7187123429162\\
581.300000000003	25.6965944272446\\
581.399999999998	25.7093723129837\\
581.500000000002	25.7049518569464\\
581.599999999998	25.7177239126698\\
582.300000000002	25.6868131868132\\
582.399999999996	25.6995708154506\\
582.499999999999	25.6951596292482\\
582.600000000003	25.7079114467136\\
583.000000000004	25.6902761104442\\
583.100000000003	25.7030178326475\\
583.999999999999	25.663413799007\\
584.200000000003	25.6888584631183\\
584.399999999997	25.6800684345595\\
584.599999999996	25.7054899948692\\
584.900000000003	25.6923076923077\\
584.999999999996	25.705007690993\\
585.299999999997	25.6918346429792\\
585.399999999998	25.7045260461144\\
585.499999999999	25.7001366120219\\
585.599999999996	25.7128222639577\\
585.700000000002	25.7084329122567\\
585.8	25.721112817887\\
585.899999999999	25.7167235494881\\
585.999999999998	25.7293977137007\\
586.500000000003	25.7074667575861\\
586.600000000002	25.7201295380944\\
587.099999999996	25.6982288828338\\
587.199999999997	25.7108802996765\\
588.700000000004	25.6453804347826\\
588.800000000001	25.6580064527084\\
589.000000000003	25.6492955355627\\
589.099999999997	25.6619144602851\\
589.299999999996	25.6532066508314\\
589.399999999999	25.665818490246\\
589.600000000001	25.6571137866712\\
589.699999999997	25.6697185486606\\
591.500000000001	25.5916159567275\\
591.6	25.6041913131655\\
591.8	25.5955397871262\\
591.899999999999	25.6081081081081\\
592.699999999999	25.5735492577598\\
592.800000000001	25.5861022094788\\
593.000000000004	25.5774742876412\\
593.199999999997	25.6025619416821\\
593.599999999999	25.585312447364\\
593.700000000004	25.5978443920512\\
594.200000000001	25.5763082618206\\
594.300000000001	25.5888290713324\\
594.499999999999	25.5802219979818\\
594.599999999996	25.5927358331932\\
594.800000000002	25.5841317868549\\
594.899999999997	25.5966386554622\\
595.199999999999	25.5837392911137\\
595.299999999999	25.5962378233121\\
595.800000000003	25.5747608659171\\
595.899999999999	25.5872483221477\\
596.099999999998	25.5786648775579\\
596.200000000003	25.5911453966124\\
596.600000000002	25.5739902798726\\
596.699999999999	25.5864611260054\\
597.399999999998	25.5564853556485\\
597.499999999999	25.5689424364123\\
597.600000000003	25.5646645474318\\
597.700000000001	25.5771160923386\\
598.199999999999	25.555741266923\\
598.300000000004	25.5681818181818\\
598.399999999999	25.5639097744361\\
598.699999999996	25.6012024048096\\
598.800000000001	25.5969277007848\\
598.900000000002	25.6093489148581\\
599.099999999996	25.6008010680908\\
599.200000000003	25.6132154179877\\
599.500000000003	25.6004002668446\\
599.599999999996	25.6128064032016\\
599.699999999999	25.6085361787262\\
599.899999999996	25.6333333333333\\
600.499999999997	25.6077256077256\\
600.600000000001	25.6201098718162\\
600.799999999997	25.6115826260609\\
601.200000000003	25.6610676866789\\
601.500000000005	25.6482712765958\\
601.600000000001	25.6606282200432\\
602.199999999998	25.6350655819359\\
602.300000000004	25.6474103585657\\
603.1	25.6133952254642\\
603.200000000002	25.6257251781866\\
603.399999999998	25.6172328086164\\
603.499999999997	25.6295559973492\\
603.900000000001	25.612582781457\\
604.000000000004	25.6248965403079\\
604.100000000004	25.6206554121152\\
604.400000000003	25.6575682382134\\
604.699999999999	25.6448412698413\\
604.899999999997	25.6694214876033\\
605.100000000002	25.6609385327165\\
605.200000000001	25.6732198909632\\
605.699999999999	25.6520303730604\\
605.800000000004	25.6643010397755\\
606.000000000004	25.6558323708959\\
606.099999999997	25.6680963378423\\
606.500000000002	25.6511704582921\\
606.600000000002	25.6634250865337\\
607.699999999997	25.6169792694965\\
607.900000000004	25.6414473684211\\
608.400000000001	25.620377978636\\
608.600000000002	25.644816822737\\
608.900000000002	25.632183908046\\
608.999999999997	25.6443933672632\\
609.499999999999	25.6233595800525\\
609.600000000004	25.6355584713794\\
610.399999999998	25.6019656019656\\
610.499999999996	25.6141500163773\\
610.700000000002	25.6057629338572\\
610.800000000002	25.6179407431658\\
610.999999999996	25.6095565373916\\
611.200000000004	25.6338949779159\\
611.900000000005	25.6045751633987\\
612.000000000003	25.6167292925993\\
612.5	25.5958210904342\\
612.600000000003	25.6079647462053\\
612.800000000004	25.5996084189917\\
612.900000000003	25.6117455138662\\
613.100000000002	25.6033920417482\\
613.199999999999	25.6155225827491\\
613.400000000002	25.6071719641402\\
613.5	25.619295958279\\
614.200000000002	25.5901025557545\\
614.400000000001	25.6143205858422\\
614.5	25.6101529450049\\
614.600000000001	25.6222547584187\\
614.699999999998	25.6180871828237\\
614.799999999996	25.6301837697187\\
615.100000000005	25.6176853055917\\
615.199999999998	25.6297740939379\\
615.399999999999	25.621445978879\\
615.500000000003	25.6335282651072\\
615.699999999997	25.6252029879831\\
615.799999999998	25.6372787790226\\
615.900000000002	25.6331168831169\\
615.999999999996	25.6451874695666\\
616.399999999996	25.6285482562855\\
616.500000000004	25.6406097956536\\
617.5	25.5990932642487\\
617.599999999997	25.6111380929254\\
618.100000000001	25.5904238110644\\
618.200000000003	25.6024583535501\\
618.399999999997	25.5941794664511\\
618.500000000005	25.6062075654704\\
618.799999999998	25.5937954435288\\
618.9	25.6058158319871\\
619.000000000001	25.6016798578582\\
619.099999999996	25.6136950904393\\
620.799999999996	25.5435657915928\\
620.900000000001	25.5555555555556\\
621.000000000001	25.5514409917888\\
621.100000000002	25.5634256278171\\
621.199999999999	25.5593111218413\\
621.300000000003	25.5712906340521\\
621.400000000005	25.5671761866452\\
621.499999999998	25.5791505791506\\
621.600000000001	25.575036191089\\
621.700000000001	25.5870054679961\\
622.199999999998	25.5664470512614\\
622.300000000003	25.5784061696658\\
622.8	25.5578744581795\\
622.900000000005	25.569823434992\\
623.100000000004	25.5616174582798\\
623.199999999998	25.5735600834269\\
623.400000000004	25.5653568564555\\
623.5	25.577293136626\\
624.400000000004	25.5404323458767\\
624.6	25.564270850008\\
625.300000000001	25.5356571794052\\
625.400000000005	25.5475619504397\\
625.499999999998	25.5434782608696\\
625.600000000003	25.5553779766661\\
625.699999999998	25.5512943432407\\
625.799999999999	25.5631890078287\\
626.3	25.5427841634738\\
626.499999999999	25.5665496329397\\
627.200000000001	25.5380200860832\\
627.299999999996	25.5498884284348\\
627.899999999996	25.5254777070064\\
627.999999999997	25.5373348192963\\
628.099999999998	25.5332696593442\\
628.299999999996	25.5569700827498\\
628.699999999997	25.5407124681934\\
628.799999999997	25.5525520750517\\
629.399999999999	25.5281969817315\\
629.500000000005	25.5400254129606\\
630.299999999999	25.507614213198\\
630.500000000001	25.5312400888043\\
631.299999999998	25.4988913525499\\
631.399999999998	25.5106888361045\\
631.799999999996	25.49454027536\\
631.900000000002	25.5063291139241\\
632.400000000001	25.4861660079051\\
632.499999999995	25.4979449889346\\
632.899999999998	25.4818325434439\\
632.999999999998	25.4936029063339\\
633.299999999998	25.4815282601831\\
633.4	25.4932912391476\\
634.200000000005	25.4611382626517\\
634.300000000002	25.4728877679697\\
634.500000000001	25.4648597541759\\
634.599999999997	25.4766031195841\\
634.800000000001	25.4685777287762\\
634.899999999996	25.4803149606299\\
635.099999999996	25.4722921914358\\
635.3	25.4957507082153\\
635.600000000003	25.4837187352525\\
635.699999999996	25.4954388172381\\
635.800000000001	25.4914294700425\\
635.900000000001	25.5031446540881\\
635.999999999999	25.4991353560761\\
636.199999999995	25.5225522552255\\
636.400000000004	25.5145326001571\\
636.5	25.526233113415\\
637.1	25.5021971123666\\
637.3	25.5255726388453\\
637.500000000003	25.5175658720201\\
637.600000000001	25.5292457268308\\
637.899999999996	25.5172413793103\\
638.200000000002	25.5522481591728\\
638.299999999995	25.5482456140351\\
638.399999999995	25.5599060297573\\
638.500000000003	25.5559035389915\\
638.599999999999	25.5675591044309\\
639.3	25.5395683453237\\
639.399999999998	25.5512118842846\\
639.499999999999	25.5472170106316\\
639.599999999998	25.5588557136158\\
639.800000000002	25.5508673230192\\
639.899999999996	25.5625\\
640.200000000001	25.5505231922536\\
640.299999999996	25.5621486570893\\
640.399999999996	25.5581576893052\\
640.700000000002	25.5930087390762\\
641.200000000002	25.5730547325745\\
641.3	25.5846585594013\\
641.600000000005	25.5726975222066\\
641.699999999999	25.5842941726395\\
641.799999999998	25.5803084592616\\
641.899999999997	25.5919003115265\\
642.000000000003	25.5879146550382\\
642.099999999995	25.599501712862\\
642.199999999995	25.5955161139654\\
642.299999999996	25.607098381071\\
642.400000000005	25.6031128404669\\
642.499999999997	25.6146903205727\\
642.800000000001	25.6027375952714\\
642.900000000003	25.6143079315708\\
643.900000000005	25.5745341614907\\
644.000000000001	25.5860891165968\\
644.500000000005	25.566242631089\\
644.600000000004	25.5777881185047\\
644.700000000001	25.5738213399504\\
644.900000000001	25.5968992248062\\
645.200000000002	25.5849992251666\\
645.4	25.6080557707204\\
645.500000000004	25.6040892193309\\
645.600000000005	25.6156109648444\\
645.900000000005	25.6037151702786\\
645.999999999997	25.615229840582\\
646.199999999996	25.6073031100108\\
646.299999999997	25.6188118811881\\
646.799999999996	25.5990106662545\\
647.000000000002	25.6220058723536\\
647.1	25.6180469715698\\
647.199999999995	25.6295380812606\\
647.6	25.6137100509495\\
647.699999999996	25.6251929607904\\
649.000000000001	25.5738715144046\\
649.099999999997	25.5853357979051\\
649.199999999999	25.5813953488372\\
649.299999999997	25.5928549430243\\
649.599999999997	25.5810374018778\\
649.699999999997	25.5924899969221\\
650.699999999997	25.5531653349723\\
650.799999999995	25.5646028575818\\
651.199999999995	25.5489021956088\\
651.599999999997	25.5945987417523\\
652.000000000001	25.5788989418801\\
652.100000000005	25.5903097209445\\
652.3	25.5824647455549\\
652.399999999999	25.5938697318008\\
652.600000000003	25.5860272713345\\
652.800000000004	25.6088221779752\\
652.900000000006	25.6049004594181\\
653	25.6162915326902\\
653.299999999998	25.6045301499847\\
653.400000000002	25.6159143075746\\
653.700000000001	25.6041602936678\\
653.800000000004	25.615537543967\\
654.499999999998	25.5881454323251\\
654.600000000001	25.599511226516\\
654.799999999995	25.5916933883036\\
654.899999999996	25.6030534351145\\
655.900000000001	25.5640243902439\\
656.000000000003	25.5753696082914\\
656.900000000001	25.5403348554033\\
657.100000000001	25.5629945222155\\
657.3	25.5552175235777\\
657.400000000005	25.5665399239544\\
657.899999999996	25.5471124620061\\
658.200000000001	25.5810420780799\\
658.999999999996	25.5499924138977\\
659.200000000002	25.5725769755802\\
659.800000000002	25.5493256554023\\
659.999999999995	25.571883048023\\
660.100000000003	25.5680096940321\\
660.199999999998	25.5792821444798\\
660.300000000002	25.5754088431254\\
660.399999999998	25.5866767600303\\
660.699999999995	25.5750605326877\\
660.799999999996	25.5863216825541\\
661.399999999995	25.5631141345427\\
661.500000000001	25.5743651753325\\
662.199999999999	25.5473350445417\\
662.299999999999	25.5585748792271\\
662.499999999999	25.5508602475098\\
662.599999999996	25.5620944620492\\
662.900000000001	25.5505279034691\\
663.099999999999	25.5729794933655\\
663.400000000001	25.561416729465\\
663.500000000001	25.5726341169379\\
663.599999999996	25.5687810757873\\
663.799999999996	25.5912034945022\\
664.000000000003	25.5834964613763\\
664.099999999999	25.5947003914484\\
664.900000000003	25.5639097744361\\
665.099999999998	25.5862898376428\\
665.199999999997	25.582444010221\\
665.499999999996	25.6159855769231\\
665.599999999996	25.6121375995193\\
665.7	25.6233103033944\\
665.800000000004	25.619462381739\\
665.900000000004	25.6306306306306\\
666.900000000001	25.592203898051\\
667.199999999998	25.6256556271542\\
667.799999999999	25.6026351250187\\
667.900000000002	25.6137724550898\\
668.200000000003	25.6022744276523\\
668.300000000003	25.6134051466188\\
668.400000000005	25.6095736724009\\
668.500000000004	25.6206999700867\\
668.599999999995	25.6168685509197\\
668.700000000001	25.627990430622\\
670.200000000005	25.570640011935\\
670.300000000005	25.5817422434368\\
670.599999999998	25.5702996868943\\
670.799999999999	25.5924877030854\\
671.499999999998	25.565812983919\\
671.700000000004	25.5879726108961\\
672.200000000001	25.5689424364123\\
672.300000000003	25.58001189768\\
672.400000000006	25.5762081784387\\
672.500000000005	25.5872732679155\\
672.800000000004	25.5758656561153\\
672.899999999999	25.5869242199108\\
672.999999999999	25.5831228643589\\
673.500000000006	25.6383610451306\\
673.899999999998	25.6231454005935\\
674.100000000004	25.6452091367547\\
674.300000000001	25.6376037959668\\
674.399999999998	25.648628613788\\
674.499999999999	25.6448265638897\\
674.599999999999	25.6558470431303\\
674.700000000006	25.6520450503853\\
674.799999999996	25.663061194251\\
675.399999999996	25.640266469282\\
675.499999999996	25.6512729425696\\
675.599999999998	25.6474766908391\\
675.699999999995	25.6584788398935\\
676.299999999997	25.6357185097575\\
676.400000000004	25.6467110125647\\
676.499999999998	25.6429204847768\\
676.600000000003	25.6539086744495\\
676.699999999999	25.6501182033097\\
676.800000000002	25.6611020830256\\
677.8	25.623248266706\\
677.9	25.6342182890856\\
678.000000000003	25.6304379884973\\
678.199999999997	25.6523662096418\\
678.399999999999	25.6448047162859\\
678.499999999995	25.6557618626584\\
678.699999999997	25.6482027106659\\
678.999999999995	25.6810484464733\\
679.300000000002	25.6697085663821\\
679.399999999999	25.6806475349522\\
679.9	25.6617647058824\\
680.000000000003	25.6726951918835\\
680.800000000001	25.6425319430166\\
680.899999999996	25.6534508076358\\
681.000000000003	25.6496843341653\\
681.099999999997	25.6605989430417\\
681.399999999997	25.6493030080704\\
681.499999999997	25.6602112676056\\
681.600000000005	25.6564471175004\\
681.699999999997	25.6673511293634\\
681.900000000005	25.6598240469208\\
682.000000000004	25.6707227679226\\
682.800000000004	25.6406501683995\\
682.900000000004	25.6515373352855\\
682.999999999997	25.6477821695213\\
683.099999999996	25.6586651053864\\
684.600000000001	25.6024536293267\\
684.800000000003	25.6241787122208\\
686.000000000001	25.5793616090949\\
686.100000000005	25.5902069367531\\
688.399999999999	25.5047204066812\\
688.500000000004	25.5155387743247\\
688.600000000006	25.5118338899376\\
688.699999999999	25.5226480836237\\
689.000000000005	25.5115367871136\\
689.100000000004	25.5223447475334\\
689.699999999995	25.5001449695564\\
689.800000000003	25.5109436150167\\
690.099999999996	25.4998551144596\\
690.300000000002	25.5214368482039\\
690.400000000005	25.5177407675597\\
690.500000000003	25.5285259194903\\
691.000000000003	25.5100564317754\\
691.100000000001	25.5208333333333\\
691.200000000002	25.5171416172429\\
691.400000000001	25.5386840202458\\
691.899999999998	25.5202312138728\\
692	25.5309926311227\\
692.4	25.5162454873646\\
692.499999999999	25.526999711233\\
692.600000000001	25.5233145661903\\
692.700000000005	25.534064665127\\
692.800000000001	25.5303795641507\\
692.900000000003	25.5411255411255\\
693.500000000004	25.5190311418685\\
693.599999999995	25.5297679112008\\
694.200000000004	25.5077056027654\\
694.299999999995	25.5184331797235\\
694.399999999998	25.5147588192945\\
694.500000000002	25.5254822919666\\
694.699999999998	25.5181347150259\\
694.799999999997	25.5288530723845\\
695.300000000003	25.5104975553638\\
695.600000000001	25.5426189449475\\
695.699999999996	25.5389479735556\\
695.800000000005	25.5496479379221\\
695.900000000004	25.5459770114943\\
696.099999999998	25.5673656995116\\
696.300000000004	25.5600229753015\\
696.400000000004	25.5707106963388\\
697.100000000005	25.5450372920252\\
697.300000000002	25.5663894465156\\
697.499999999998	25.5590596330275\\
697.599999999998	25.5697291099326\\
698.000000000005	25.5550780690446\\
698.100000000005	25.5657404755085\\
698.600000000001	25.5474452554745\\
698.699999999996	25.5580995993131\\
699.300000000001	25.5361738633114\\
699.4	25.5468191565404\\
700.500000000004	25.506708535541\\
700.600000000001	25.5173398030541\\
700.900000000004	25.5064194008559\\
701.000000000003	25.5170446441307\\
701.499999999995	25.4988597491448\\
701.599999999998	25.5094769844663\\
702.700000000001	25.4695503699488\\
702.799999999998	25.4801536491677\\
703.100000000004	25.4692832764505\\
703.199999999999	25.4798805630599\\
703.300000000004	25.4762581745806\\
703.499999999995	25.4974417282547\\
704.699999999998	25.4540295119183\\
704.799999999999	25.4646049084977\\
705.100000000001	25.4537719795803\\
705.199999999995	25.4643414150007\\
705.399999999997	25.4571226080794\\
705.500000000006	25.4676870748299\\
705.900000000002	25.4532577903683\\
706.000000000001	25.4638153236086\\
706.100000000001	25.4602095723591\\
706.400000000001	25.4918612880396\\
706.600000000004	25.4846469506155\\
706.800000000004	25.5057292403452\\
707.399999999997	25.4840989399293\\
707.5	25.4946297343132\\
708.300000000001	25.4658385093168\\
708.399999999995	25.4763585038814\\
709.400000000002	25.4404510218464\\
709.499999999999	25.4509582863585\\
710.099999999997	25.4294564911293\\
710.200000000005	25.4399549486133\\
710.799999999995	25.4184836123224\\
711.099999999996	25.4499437570304\\
711.600000000005	25.4320640719404\\
711.699999999996	25.4425400393369\\
711.900000000004	25.435393258427\\
712.100000000001	25.4563324908734\\
712.200000000005	25.4527586691001\\
712.300000000006	25.4632229084784\\
712.399999999999	25.459649122807\\
712.600000000002	25.4805668584257\\
712.700000000002	25.4769921436588\\
712.799999999996	25.4874456445504\\
713.300000000003	25.4695822820297\\
713.4	25.4800280308339\\
713.799999999995	25.4657515058131\\
713.899999999996	25.4761904761905\\
714.500000000003	25.4547998880493\\
714.600000000002	25.4652301665034\\
714.900000000005	25.4545454545455\\
715.000000000001	25.4649699342749\\
715.1	25.4614093959732\\
715.299999999995	25.482247693598\\
716.600000000004	25.4360262313381\\
716.699999999995	25.4464285714286\\
717.300000000002	25.4251463618623\\
717.400000000001	25.4355400696864\\
717.599999999995	25.4284519994427\\
717.700000000003	25.4388409027584\\
717.899999999998	25.4317548746518\\
718.099999999995	25.4525201893623\\
718.399999999995	25.4418928322895\\
718.499999999998	25.4522682994712\\
718.900000000001	25.4381084840056\\
719	25.4484772632457\\
719.400000000004	25.4343293954135\\
719.500000000001	25.4446914952752\\
720.000000000001	25.427024024441\\
720.099999999997	25.4373785059706\\
721.200000000003	25.3985858865937\\
721.400000000005	25.4192654192654\\
721.899999999995	25.4016620498615\\
722.099999999999	25.4223206867904\\
723.799999999996	25.3626191462909\\
723.899999999998	25.3729281767956\\
724.099999999999	25.3659210162938\\
724.199999999997	25.3762253209996\\
724.699999999996	25.3587196467991\\
724.800000000006	25.3690164160574\\
725.100000000002	25.3585217870932\\
725.200000000001	25.3688129050048\\
725.299999999995	25.3653156878963\\
725.400000000005	25.3756030323915\\
725.699999999998	25.3651143565721\\
725.899999999998	25.3856749311295\\
726.100000000003	25.3786835582484\\
726.300000000006	25.3992290748899\\
726.6	25.388743635613\\
726.799999999995	25.4092722520292\\
726.900000000002	25.4057771664374\\
727.100000000003	25.4262926292629\\
727.499999999995	25.4123144584937\\
727.599999999995	25.4225642435069\\
728.099999999995	25.4051084866795\\
728.199999999998	25.415350816971\\
728.900000000005	25.3909465020576\\
729.000000000002	25.4011795364148\\
729.1	25.3976961053209\\
729.200000000006	25.4079254079254\\
730.600000000003	25.3592445600109\\
730.700000000001	25.3694581280788\\
731.099999999996	25.355579868709\\
731.200000000002	25.3657869547381\\
731.600000000005	25.3519201858685\\
731.699999999997	25.3621207980323\\
731.800000000001	25.3586555540374\\
731.900000000003	25.3688524590164\\
732.499999999998	25.3480753480753\\
732.599999999995	25.3582639552341\\
732.7	25.3548034934498\\
732.800000000003	25.3649884022377\\
733.000000000003	25.3580684763334\\
733.100000000001	25.3682487725041\\
733.299999999996	25.3613307881102\\
733.400000000002	25.371506475801\\
733.900000000002	25.3542234332425\\
733.999999999996	25.3643917722381\\
734.799999999998	25.3367805143557\\
734.899999999997	25.3469387755102\\
735.399999999998	25.3297076818491\\
735.499999999998	25.3398586188146\\
735.699999999998	25.3329709160098\\
735.799999999995	25.3431172713684\\
736.000000000001	25.3362314902867\\
736.100000000004	25.3463732681337\\
736.200000000001	25.3429308705691\\
736.399999999996	25.3632043448744\\
736.5	25.3597610643497\\
736.699999999995	25.3800217155266\\
736.800000000001	25.3765775546207\\
736.899999999998	25.3867028493894\\
737.499999999995	25.3660520607375\\
737.6	25.3761691744612\\
737.699999999997	25.3727297370561\\
737.800000000001	25.3828432036861\\
737.899999999997	25.3794037940379\\
738.099999999995	25.3996206989976\\
738.600000000004	25.3824285907676\\
738.700000000001	25.3925284244721\\
739.4	25.368492224476\\
739.5	25.3785830178475\\
740	25.3614376435617\\
740.100000000005	25.3715212104837\\
740.899999999997	25.3441295546559\\
741	25.3542032114425\\
741.099999999995	25.3507825148408\\
741.200000000001	25.36085255632\\
741.399999999996	25.354012137559\\
741.499999999996	25.3640776699029\\
741.799999999998	25.3538212697129\\
742.099999999999	25.3839935327405\\
745.099999999998	25.2818035426731\\
745.200000000002	25.2918287937743\\
745.999999999996	25.2647098244203\\
746.100000000002	25.2747252747253\\
746.300000000004	25.2679528403001\\
746.399999999995	25.2779638312123\\
746.800000000003	25.2644262953541\\
747.000000000003	25.2844331414804\\
747.3	25.2742841851753\\
747.400000000004	25.2842809364548\\
747.499999999994	25.2808988764045\\
747.599999999998	25.2908920690116\\
748.200000000001	25.2706133903515\\
748.400000000005	25.2905811623246\\
749.299999999998	25.2602081665332\\
749.400000000001	25.2701801200801\\
751.599999999998	25.1962218970334\\
751.699999999995	25.2061718542165\\
752.000000000001	25.1961175375615\\
752.199999999997	25.2160042536222\\
752.500000000004	25.205952697316\\
752.600000000004	25.2158894645941\\
752.700000000002	25.2125398512221\\
752.800000000003	25.2224731039979\\
752.999999999998	25.2157747975037\\
753.100000000005	25.2257036643654\\
753.399999999996	25.2156602521566\\
753.499999999999	25.2255838641189\\
754.100000000002	25.2055157783081\\
754.200000000003	25.2154315259181\\
754.599999999995	25.2020670465085\\
754.7	25.2119766825649\\
754.900000000002	25.205298013245\\
755.099999999999	25.2251059322034\\
755.400000000007	25.2150893448048\\
755.600000000006	25.2348815667593\\
755.799999999996	25.2282047889933\\
756.000000000001	25.2479830710224\\
756.200000000006	25.2413063599101\\
756.300000000004	25.251189846642\\
756.600000000001	25.2411788026959\\
756.700000000001	25.2510570824524\\
756.799999999999	25.2477209671027\\
756.900000000005	25.2575957727873\\
756.999999999996	25.254259675076\\
757.200000000002	25.2739997359039\\
757.3	25.2706627937682\\
757.500000000004	25.2903907074974\\
757.9	25.2770448548813\\
758.000000000003	25.2869014641868\\
758.299999999996	25.2768987341772\\
758.399999999995	25.2867501647989\\
758.500000000003	25.2834168204587\\
758.600000000004	25.2932647950442\\
758.699999999998	25.2899314707433\\
758.799999999997	25.2997759915667\\
760.099999999998	25.2565114443568\\
760.399999999999	25.2859960552268\\
760.700000000005	25.2760252365931\\
760.8	25.2858457090288\\
760.999999999997	25.2792011562213\\
761.099999999999	25.2890173410405\\
761.9	25.2624671916011\\
762.100000000002	25.2820781946996\\
762.300000000002	25.2754459601259\\
762.400000000003	25.2852459016393\\
762.899999999995	25.2686762778506\\
762.999999999997	25.278469401127\\
763.500000000003	25.261917234154\\
763.700000000001	25.2814873003404\\
763.800000000001	25.2781777719597\\
763.899999999999	25.2879581151832\\
764.000000000004	25.2846486062034\\
764.100000000002	25.2944255430516\\
764.200000000003	25.2911160539055\\
764.500000000003	25.3204289824745\\
764.600000000001	25.3171178239833\\
764.700000000005	25.3268828451883\\
765.000000000004	25.3169520324141\\
765.099999999998	25.3267119707266\\
765.400000000002	25.3167864141084\\
765.499999999999	25.3265412748171\\
765.800000000002	25.3166209687949\\
765.900000000001	25.3263707571802\\
766.200000000004	25.3164556962025\\
766.399999999998	25.3359425962166\\
766.800000000006	25.3227278654323\\
766.899999999998	25.3324641460235\\
767.299999999998	25.3192598384154\\
767.399999999999	25.328990228013\\
768.3	25.2993232691307\\
768.599999999995	25.3284766488877\\
768.900000000005	25.3185955786736\\
768.999999999997	25.328305811988\\
769.099999999996	25.32501300052\\
769.199999999995	25.3347198752112\\
769.900000000002	25.3116883116883\\
769.999999999994	25.3213868328788\\
770.199999999999	25.3148124107491\\
770.400000000004	25.3341985723556\\
770.800000000004	25.321053314308\\
770.899999999999	25.3307392996109\\
771.900000000001	25.2979274611399\\
771.999999999996	25.3076026421448\\
772.400000000004	25.294498381877\\
772.500000000001	25.3041677452757\\
773.199999999995	25.2812621233674\\
773.599999999996	25.3198914307871\\
773.799999999997	25.3133479777749\\
773.900000000002	25.3229974160207\\
774.199999999995	25.3131861035774\\
774.300000000007	25.3228305785124\\
776.499999999999	25.2510945145506\\
776.700000000006	25.2703398558187\\
776.800000000004	25.2670871412022\\
777	25.2863209368164\\
777.699999999998	25.2635638981743\\
777.799999999997	25.2731713587865\\
779.300000000006	25.2245316910444\\
779.400000000001	25.2341244387428\\
779.500000000002	25.2308876346844\\
779.599999999995	25.2404771065795\\
780.399999999996	25.2146060217809\\
780.600000000005	25.2337645702575\\
781.100000000002	25.2176139272914\\
781.299999999997	25.2367545431277\\
781.800000000002	25.220616447116\\
781.900000000005	25.230179028133\\
782.200000000005	25.2205036431037\\
782.400000000001	25.2396166134185\\
783.099999999994	25.2170582226762\\
783.200000000005	25.2266053874633\\
784.500000000006	25.184807545246\\
784.599999999999	25.1943417866701\\
785.100000000001	25.1782985226694\\
785.199999999999	25.1878263084172\\
785.799999999997	25.1685965135513\\
785.900000000005	25.1781170483461\\
785.999999999996	25.174914133062\\
786.099999999998	25.1844314423811\\
786.300000000001	25.178026449644\\
786.4	25.1875397329943\\
786.499999999995	25.1843376557335\\
786.599999999994	25.193847718317\\
787.200000000004	25.1746475295313\\
787.300000000001	25.1841503683007\\
787.400000000002	25.1809523809524\\
787.600000000006	25.1999492192459\\
787.899999999995	25.1903553299492\\
788.000000000006	25.199847735059\\
788.599999999995	25.1806770635223\\
789.099999999999	25.22807906741\\
789.500000000007	25.2152988855117\\
789.800000000002	25.2437017343968\\
790.100000000007	25.2341179448241\\
790.200000000001	25.2435783879539\\
790.500000000001	25.2339994940551\\
790.600000000002	25.2434551663083\\
791.800000000003	25.2052026771057\\
791.900000000002	25.2146464646465\\
792.200000000002	25.2050990786318\\
792.400000000001	25.2239747634069\\
792.799999999995	25.2112498423509\\
792.899999999999	25.2206809583859\\
793	25.2175009456563\\
793.099999999996	25.2269288956127\\
793.300000000003	25.2205697000252\\
793.400000000004	25.2299936988028\\
794.100000000007	25.207756232687\\
794.200000000007	25.2171723530152\\
794.500000000003	25.2076516486282\\
794.600000000001	25.2170630426576\\
794.700000000001	25.2138902868646\\
794.799999999997	25.2232985281167\\
795.099999999995	25.2137826961771\\
795.200000000006	25.2231862190368\\
795.599999999998	25.2105064722886\\
795.699999999995	25.2199044986177\\
796.200000000002	25.2040688182846\\
796.3	25.2134605725766\\
796.400000000002	25.2102950408035\\
796.600000000005	25.2290699133927\\
797.000000000001	25.2164094843809\\
797.200000000005	25.2351686943434\\
797.299999999998	25.2320040130424\\
797.400000000002	25.2413793103448\\
797.500000000003	25.2382146439318\\
797.599999999999	25.2475868120847\\
797.899999999998	25.2380952380952\\
798.000000000007	25.2474627239694\\
798.099999999999	25.2442996742671\\
798.199999999995	25.2536640360767\\
798.400000000001	25.2473387601753\\
798.599999999996	25.2660573431827\\
798.799999999994	25.2597321316811\\
798.900000000003	25.2690863579474\\
799.200000000001	25.2596021518829\\
799.400000000004	25.2782989368355\\
799.499999999997	25.2751375687844\\
799.599999999998	25.2844816806302\\
799.700000000004	25.2813203300825\\
799.800000000002	25.2906613326666\\
800.499999999994	25.2685485885586\\
800.699999999999	25.2872127872128\\
800.800000000001	25.2840554376327\\
800.899999999997	25.2933832709114\\
800.999999999998	25.2902259393334\\
801.099999999994	25.299550673989\\
801.600000000004	25.2837719845329\\
801.700000000006	25.2930905462709\\
802.500000000004	25.2678793919761\\
802.600000000004	25.2771894854865\\
804.100000000002	25.2300422780403\\
804.200000000002	25.2393385552655\\
804.699999999995	25.2236580516899\\
804.800000000007	25.232948192322\\
804.900000000002	25.2298136645963\\
805	25.2391007328282\\
805.299999999997	25.2296995281848\\
805.399999999997	25.2389819987585\\
805.499999999999	25.2358490566038\\
805.600000000005	25.2451284597245\\
805.900000000005	25.2357320099256\\
806.100000000002	25.2542793351526\\
806.900000000003	25.2292441140025\\
807.000000000006	25.2385082393755\\
807.799999999999	25.2135165243223\\
807.899999999997	25.2227722772277\\
808.299999999999	25.2102919346858\\
808.400000000006	25.2195423623995\\
810.000000000001	25.1697321318356\\
810.100000000003	25.1789681560109\\
810.799999999999	25.1572327044025\\
810.900000000003	25.1664611590629\\
811.100000000006	25.1602564102564\\
811.400000000003	25.1879235982748\\
811.499999999998	25.1848201084278\\
811.600000000002	25.1940372058642\\
811.900000000002	25.1847290640394\\
812.300000000002	25.2215657311669\\
812.900000000004	25.2029520295203\\
812.999999999996	25.212151026934\\
813.1	25.2090506640433\\
813.199999999997	25.2182466494529\\
813.299999999995	25.2151462994836\\
813.400000000004	25.2243392747388\\
813.600000000005	25.2181393634017\\
813.699999999995	25.2273285819612\\
814.100000000004	25.2149349054286\\
814.199999999995	25.2241188751075\\
814.300000000001	25.221021611002\\
814.400000000003	25.2302025782689\\
815	25.2116304747884\\
815.099999999994	25.2208047105005\\
815.699999999999	25.2022554547683\\
815.800000000001	25.211422968501\\
816.000000000004	25.2052444553364\\
816.100000000005	25.2144082332762\\
816.500000000006	25.2020573108009\\
816.599999999996	25.2112158687401\\
817.200000000004	25.1927076960724\\
817.400000000001	25.2110091743119\\
817.499999999997	25.2079256360078\\
817.599999999999	25.2170722758958\\
817.699999999997	25.2139887503057\\
817.900000000005	25.2322738386308\\
818	25.2291895856252\\
818.100000000006	25.2383280371547\\
818.400000000005	25.2290775809407\\
818.599999999997	25.2473433492122\\
819.799999999997	25.2103915111599\\
819.900000000006	25.219512195122\\
820.000000000003	25.2164370198756\\
820.199999999999	25.2346702425942\\
820.600000000003	25.2223711465822\\
820.699999999997	25.2314814814815\\
820.799999999994	25.2284078450481\\
820.899999999995	25.237515225335\\
821.299999999995	25.2252252252252\\
821.4	25.234327449787\\
821.699999999998	25.2251155999027\\
821.800000000006	25.2342134079572\\
822.299999999999	25.2188715953307\\
822.399999999996	25.2279635258359\\
822.500000000001	25.224896669098\\
822.600000000003	25.2339856569831\\
823.200000000001	25.2155957731082\\
823.300000000001	25.2246781637114\\
823.399999999999	25.2216150576806\\
823.499999999994	25.230694511899\\
823.600000000001	25.227631419206\\
823.699999999995	25.2367079388201\\
823.800000000006	25.2336448598131\\
823.899999999998	25.2427184466019\\
824.200000000007	25.2335314812568\\
824.299999999996	25.2426006792819\\
824.499999999997	25.2364782925055\\
824.600000000004	25.2455438341215\\
825.400000000002	25.221078134464\\
825.499999999997	25.2301356589147\\
825.999999999994	25.2148650284469\\
826.100000000003	25.2239167271847\\
826.600000000002	25.2086609410911\\
826.700000000004	25.2177068214804\\
826.800000000006	25.2146571532229\\
826.900000000002	25.223700120919\\
827.000000000003	25.2206504654818\\
827.100000000005	25.2296905222437\\
827.500000000003	25.2174963750604\\
827.599999999994	25.2265313519391\\
828.000000000001	25.214346093467\\
828.100000000001	25.2233759961362\\
828.300000000005	25.2172863351038\\
828.400000000004	25.2263126131563\\
828.600000000004	25.2202244479305\\
828.900000000002	25.2472858866104\\
829.100000000004	25.2411963338157\\
829.200000000004	25.2502110213433\\
829.399999999997	25.244122965642\\
829.499999999994	25.2531340405015\\
830.1	25.2348831606842\\
830.200000000006	25.2438877514152\\
830.3	25.2408477842004\\
830.4	25.2498494882601\\
830.800000000005	25.2376940666747\\
830.900000000006	25.2466907340554\\
831	25.2436529900132\\
831.099999999999	25.2526467757459\\
831.300000000005	25.2465720471494\\
831.4	25.2555622369212\\
831.499999999996	25.2525252525253\\
831.700000000004	25.2704977157971\\
832.200000000005	25.2553165925748\\
832.400000000002	25.2732732732733\\
832.700000000003	25.2641690682036\\
832.799999999999	25.2731420338576\\
833.700000000002	25.2458623171024\\
833.799999999997	25.2548267178319\\
833.900000000004	25.2517985611511\\
834.099999999994	25.2697194917286\\
834.300000000003	25.2636625119847\\
834.399999999999	25.2726183343319\\
835.199999999997	25.2484137435652\\
835.400000000003	25.2663076002394\\
836.899999999997	25.221027479092\\
836.999999999998	25.2299605781866\\
837.100000000004	25.2269469660774\\
837.199999999994	25.2358772244118\\
837.300000000005	25.2328636255075\\
837.400000000006	25.2417910447761\\
838.199999999998	25.2177024931409\\
838.299999999994	25.2266221374046\\
838.699999999997	25.2145922746781\\
838.799999999999	25.2235069734176\\
838.9	25.2205005959476\\
838.999999999998	25.2294124657371\\
839.399999999999	25.2173913043478\\
839.599999999995	25.2352030487079\\
839.800000000004	25.2291939516609\\
840.200000000005	25.2647863858146\\
840.699999999997	25.2497621313035\\
840.799999999998	25.2586514448805\\
841.000000000005	25.252645345381\\
841.099999999996	25.2615311459819\\
841.400000000003	25.2525252525253\\
841.499999999998	25.2614068441065\\
841.999999999998	25.2464077900487\\
842.100000000007	25.2552837805747\\
843.900000000004	25.2014218009479\\
844.099999999996	25.2191423833215\\
844.499999999997	25.2071986739285\\
844.600000000005	25.216053036581\\
845.100000000001	25.20113582584\\
845.200000000002	25.2099846208447\\
845.299999999997	25.2070026023184\\
845.399999999995	25.2158486102898\\
845.600000000006	25.2098853021166\\
845.699999999999	25.2187278316387\\
845.799999999994	25.2157465421445\\
846.000000000002	25.2334239451601\\
846.799999999994	25.209587908844\\
846.900000000004	25.2184179456907\\
847.000000000001	25.2154409160666\\
847.099999999997	25.224268177526\\
847.300000000002	25.2183148454095\\
847.399999999998	25.2271386430678\\
848.900000000007	25.1825677267373\\
849.000000000001	25.1913791072901\\
849.100000000003	25.1884126236458\\
849.200000000003	25.197221241022\\
849.599999999995	25.1853595386607\\
849.700000000007	25.1941633325488\\
850.599999999995	25.1675091101446\\
850.699999999996	25.1763046544429\\
851	25.1674303842087\\
851.099999999996	25.1762218045113\\
851.399999999999	25.1673517322372\\
851.500000000002	25.1761390324096\\
851.6	25.1731830456734\\
851.699999999994	25.1819675980277\\
851.799999999998	25.1790116210823\\
851.900000000001	25.18779342723\\
852.200000000001	25.1789276076499\\
852.500000000005	25.2052545155993\\
852.599999999996	25.2022985809781\\
852.799999999997	25.2198381990855\\
853.199999999995	25.2080159381226\\
853.400000000001	25.2255418863503\\
853.500000000005	25.2225866916589\\
853.899999999999	25.2576112412178\\
854.300000000003	25.2457865168539\\
854.399999999998	25.2545348156817\\
854.600000000007	25.2486252486253\\
854.699999999994	25.2573701450632\\
854.900000000005	25.2514619883041\\
855.000000000007	25.2602034849725\\
855.399999999995	25.2483927527762\\
855.500000000003	25.2571294997663\\
855.599999999998	25.2541778660746\\
855.700000000004	25.2629118953026\\
856.199999999999	25.2481606913465\\
856.299999999995	25.2568893040635\\
856.399999999995	25.2539404553415\\
856.500000000004	25.2626663553584\\
856.800000000005	25.2538219162096\\
856.900000000004	25.2625437572929\\
857.200000000004	25.2537034876939\\
857.400000000004	25.2711370262391\\
857.5	25.2681902985075\\
857.800000000003	25.2943233477095\\
857.899999999996	25.2913752913753\\
858.299999999994	25.3261882572227\\
860.099999999994	25.2731922808649\\
860.3	25.2905625290563\\
860.800000000007	25.2758740852596\\
860.899999999997	25.2845528455285\\
861.200000000007	25.2757459654011\\
861.300000000004	25.2844207104713\\
861.599999999995	25.2756179644888\\
861.799999999994	25.2929574196543\\
861.999999999996	25.2870896647721\\
862.099999999998	25.2957550452331\\
862.599999999996	25.281094239017\\
862.700000000003	25.2897542883635\\
863.000000000006	25.2809639670954\\
863.100000000006	25.2896200185357\\
863.300000000005	25.2837618716701\\
863.400000000002	25.2924145917776\\
863.500000000003	25.2894858730894\\
863.599999999994	25.2981359268264\\
863.800000000003	25.2922791989814\\
863.899999999999	25.3009259259259\\
864.299999999999	25.2892179546506\\
864.399999999998	25.2978600347021\\
864.600000000003	25.2920087891754\\
864.700000000004	25.3006475485661\\
864.899999999995	25.2947976878613\\
865.100000000001	25.3120665742025\\
865.300000000007	25.3062167783684\\
865.400000000004	25.314846909301\\
865.499999999999	25.3119223659889\\
865.599999999998	25.3205498440568\\
866.3	25.3000923361034\\
866.5	25.3173321024694\\
866.8	25.3085707694082\\
866.900000000004	25.3171856978085\\
867.100000000003	25.3113468634686\\
867.199999999996	25.3199584918713\\
867.299999999995	25.3170394281762\\
867.499999999999	25.334255417243\\
867.599999999995	25.3313357151089\\
867.800000000007	25.3485424588086\\
867.999999999998	25.3427024536344\\
868.100000000005	25.3513015434232\\
868.199999999999	25.3483818956582\\
868.299999999995	25.3569783509903\\
869.000000000001	25.3365550569555\\
869.100000000004	25.3451449608836\\
869.699999999999	25.3276615313865\\
869.800000000001	25.336245545465\\
870.400000000005	25.3187823090178\\
870.500000000004	25.3273604410751\\
870.600000000002	25.3244515906742\\
870.699999999995	25.3330271015159\\
871.000000000001	25.3243026059006\\
871.200000000002	25.3414438195799\\
871.400000000005	25.3356282271945\\
871.500000000004	25.3441945846719\\
872.299999999994	25.3209536909674\\
872.399999999998	25.3295128939828\\
873.1	25.3092075125973\\
873.299999999995	25.3263109686284\\
873.399999999996	25.3234115626789\\
873.500000000002	25.3319597069597\\
873.8	25.3232635312965\\
873.900000000005	25.3318077803204\\
874.399999999998	25.3173241852487\\
874.500000000006	25.3258632517722\\
876.699999999995	25.2623175182482\\
876.899999999994	25.2793614595211\\
876.999999999998	25.2764793068065\\
877.099999999999	25.2849977200182\\
877.299999999998	25.2792341007522\\
877.400000000007	25.2877492877493\\
877.499999999994	25.2848678213309\\
877.600000000002	25.2933804261137\\
877.700000000003	25.2904989747095\\
877.800000000001	25.299008998747\\
878.000000000001	25.2932467828266\\
878.099999999998	25.3017535868823\\
878.199999999996	25.298872822498\\
878.300000000001	25.3073770491803\\
879.100000000006	25.2843494085532\\
879.199999999996	25.2928465825088\\
879.399999999996	25.287094940307\\
879.499999999997	25.2955889040473\\
879.699999999995	25.2898385996818\\
879.800000000002	25.2983293556086\\
879.999999999994	25.2925803885922\\
880.099999999995	25.3010679391047\\
880.200000000002	25.298193797569\\
880.299999999996	25.3066787823716\\
880.900000000006	25.2894438138479\\
881.200000000002	25.3148757517304\\
881.700000000005	25.3005216602404\\
881.799999999993	25.3089919492006\\
882.300000000004	25.2946509519492\\
882.399999999994	25.3031161473088\\
883.600000000002	25.2687563652823\\
883.800000000001	25.2856657992986\\
883.899999999998	25.2828054298643\\
883.999999999997	25.2912566451759\\
884.7	25.2712477396022\\
884.799999999994	25.2796926206351\\
884.899999999994	25.2768361581921\\
884.999999999994	25.2852784996046\\
885.699999999994	25.265296906751\\
885.800000000006	25.2737329269669\\
885.899999999998	25.2708803611738\\
886.100000000005	25.2877454299255\\
886.199999999998	25.2848922486743\\
886.4	25.3017484489566\\
886.500000000007	25.2988946537334\\
886.599999999998	25.3073192737115\\
886.799999999997	25.3016123576502\\
886.899999999995	25.3100338218715\\
887.099999999997	25.3043282236249\\
887.199999999996	25.3127465344303\\
887.299999999997	25.3098940725716\\
887.500000000005	25.3267237494367\\
887.6	25.3238706770305\\
887.799999999995	25.3406915193152\\
888.400000000007	25.3235790658413\\
888.600000000002	25.3403848317768\\
889.500000000007	25.3147482014388\\
889.999999999997	25.3567014942141\\
890.600000000006	25.339620523184\\
890.700000000006	25.3480017961383\\
890.999999999995	25.339468073168\\
891.100000000001	25.3478456014363\\
891.699999999994	25.3307916573223\\
891.799999999994	25.3391635833614\\
892.299999999998	25.324966382788\\
892.399999999994	25.3333333333333\\
892.900000000006	25.3191489361702\\
892.999999999994	25.3275109170306\\
894.199999999999	25.2935256625294\\
894.300000000004	25.3018783542039\\
894.499999999998	25.296221775095\\
894.600000000007	25.3045713647033\\
895.199999999996	25.2876130905842\\
895.299999999998	25.2959571141389\\
895.5	25.2903081732916\\
895.600000000005	25.2986491012616\\
895.900000000005	25.2901785714286\\
896.099999999997	25.306851149297\\
896.199999999995	25.3040276693072\\
896.299999999996	25.3123605533244\\
897.499999999994	25.2785204991087\\
897.599999999998	25.2868441572909\\
897.700000000006	25.2840276230786\\
897.800000000002	25.2923488138991\\
898.000000000008	25.2867164012916\\
898.100000000006	25.2950345134714\\
898.200000000001	25.2922186351998\\
898.299999999997	25.3005342831701\\
898.499999999995	25.2949031827287\\
898.599999999995	25.3032157560921\\
898.700000000005	25.3004005340454\\
898.800000000005	25.3087106463455\\
899.100000000002	25.3002669039146\\
899.199999999997	25.308573334816\\
899.600000000007	25.297321329332\\
899.799999999994	25.3139237693077\\
899.900000000002	25.3111111111111\\
900.099999999993	25.3277049544546\\
900.2	25.3248917027657\\
900.299999999997	25.3331852509996\\
900.500000000004	25.3275594048412\\
900.600000000001	25.3358498945265\\
900.700000000005	25.3330373001776\\
900.799999999997	25.3413253413253\\
901.099999999996	25.3328894806924\\
901.199999999996	25.34117385998\\
901.699999999996	25.3271235307163\\
901.800000000005	25.3354030380308\\
901.999999999997	25.3297860547611\\
902.100000000006	25.338062513855\\
903.100000000006	25.3100088573959\\
903.200000000008	25.3182774272113\\
903.400000000005	25.3126729385722\\
903.500000000004	25.3209384683488\\
903.599999999998	25.31813654974\\
903.800000000003	25.3346609138179\\
903.899999999996	25.3318584070796\\
904.000000000005	25.3401172436677\\
904.199999999999	25.3345128828928\\
904.300000000002	25.3427686864219\\
904.399999999997	25.3399668325041\\
904.7	25.3647214854111\\
904.900000000002	25.3591160220994\\
904.999999999994	25.3673627223511\\
905.399999999993	25.3561568194368\\
905.499999999999	25.3643992932862\\
905.700000000005	25.3587988518437\\
905.800000000003	25.3670383044486\\
906.600000000005	25.3446564464542\\
906.699999999993	25.3528892809881\\
907.699999999995	25.3249614452523\\
907.800000000002	25.3331864742813\\
908.499999999994	25.313669381466\\
908.600000000002	25.3218884120172\\
908.800000000007	25.3163164264496\\
908.900000000006	25.3245324532453\\
909.300000000001	25.3133934462283\\
909.500000000008	25.3298153034301\\
910.100000000004	25.3131179960448\\
910.200000000004	25.3213226408876\\
910.300000000005	25.3185413005272\\
910.500000000006	25.3349439929717\\
910.700000000006	25.3293807641634\\
910.799999999994	25.3375782193435\\
911.199999999998	25.3264567101942\\
911.299999999997	25.3346499890279\\
912.3	25.3068829460763\\
912.400000000003	25.3150684931507\\
912.999999999997	25.2984339064725\\
913.100000000003	25.3066141042488\\
913.199999999997	25.3038432059564\\
913.300000000001	25.3120210203635\\
914.699999999995	25.2732837778749\\
915.000000000008	25.2977816632062\\
915.499999999994	25.2839667977283\\
915.700000000006	25.3002839047827\\
916.400000000004	25.2809601745772\\
916.600000000001	25.2972619177485\\
916.699999999999	25.2945026178011\\
916.899999999998	25.3107960741549\\
917.100000000005	25.3052769297863\\
917.300000000001	25.3215609330717\\
917.799999999994	25.3077677306896\\
917.899999999999	25.3159041394336\\
918.499999999997	25.299368604398\\
918.599999999999	25.3074997278763\\
918.7	25.3047453199826\\
918.800000000005	25.3128740885842\\
919.099999999998	25.3046127067015\\
919.200000000007	25.3127379527902\\
919.400000000002	25.3072321914084\\
919.500000000008	25.3153545019574\\
920.400000000002	25.2906029331885\\
920.499999999995	25.2987182272431\\
921.000000000001	25.2849853436109\\
921.200000000002	25.3012048192771\\
921.700000000001	25.2874810154046\\
921.799999999997	25.295585204469\\
921.900000000001	25.29284164859\\
922	25.300943498536\\
922.900000000006	25.2762730227519\\
923.199999999999	25.3005523665114\\
923.699999999997	25.2868586274085\\
923.799999999995	25.2949453404048\\
923.899999999999	25.2922077922078\\
923.999999999994	25.3002921761714\\
924.799999999994	25.2784084765921\\
924.900000000006	25.2864864864865\\
925.099999999999	25.2810203199308\\
925.199999999996	25.2890954285097\\
925.700000000001	25.2754374594945\\
925.900000000008	25.2915766738661\\
926.200000000006	25.283385512253\\
926.300000000003	25.2914507772021\\
926.399999999995	25.2887209929844\\
926.500000000001	25.2967839412907\\
927.800000000005	25.2613428171139\\
927.999999999996	25.2774485508027\\
928.199999999998	25.2720025853711\\
928.300000000005	25.2800517018527\\
928.599999999997	25.271885431248\\
928.699999999997	25.2799310938846\\
929.400000000003	25.2608929532007\\
929.499999999994	25.2689328743546\\
929.999999999997	25.2553488872164\\
930.100000000001	25.2633842184476\\
931.199999999994	25.2335445076774\\
931.300000000005	25.2415718273567\\
931.400000000004	25.2388620504563\\
931.500000000002	25.2468870759983\\
932.200000000002	25.2279309235225\\
932.299999999999	25.2359502359502\\
932.400000000001	25.2332439678284\\
932.499999999994	25.2412609907785\\
932.999999999994	25.2277355053049\\
933.100000000002	25.2357479639949\\
933.199999999996	25.233044037287\\
933.299999999997	25.2410542104135\\
933.500000000004	25.235646958012\\
933.599999999998	25.2436542786762\\
934.599999999999	25.2166470525302\\
934.700000000008	25.2246469833119\\
934.900000000001	25.2192513368984\\
934.999999999997	25.2272484226286\\
935.300000000001	25.2191575796451\\
935.400000000003	25.2271512560128\\
935.500000000006	25.2244548952544\\
935.600000000004	25.23244629689\\
935.700000000008	25.2297499465698\\
935.799999999999	25.2377390746875\\
935.900000000005	25.2350427350427\\
935.999999999994	25.2430295908557\\
936.099999999993	25.2403332621235\\
936.299999999993	25.2563007261854\\
937.2	25.2320495038942\\
937.299999999994	25.240025602731\\
937.800000000006	25.2265699968014\\
937.900000000008	25.2345415778252\\
939	25.2049834948355\\
939.099999999993	25.2129471890971\\
939.499999999997	25.2022137079608\\
939.599999999997	25.2101734596148\\
940.000000000001	25.1994468673545\\
940.200000000005	25.215356801021\\
940.300000000006	25.2126754572522\\
940.500000000001	25.228577503721\\
942.300000000008	25.1803904923599\\
942.400000000005	25.1883289124668\\
942.900000000008	25.1749734888653\\
943.000000000004	25.1829074329339\\
943.199999999996	25.1775681119474\\
943.3	25.185499258003\\
943.600000000002	25.1774928473032\\
943.699999999996	25.1854206399661\\
943.900000000006	25.1800847457627\\
944.099999999999	25.1959330650286\\
944.200000000002	25.1932648522715\\
944.299999999998	25.2011859381618\\
944.6	25.1931830210649\\
944.699999999995	25.201100762066\\
944.900000000002	25.1957671957672\\
945.000000000001	25.203682150037\\
945.2	25.1983497302444\\
945.300000000006	25.206261899725\\
945.399999999996	25.2035959809625\\
945.500000000005	25.2115059221658\\
945.999999999998	25.1981820103583\\
946.1	25.2060875079264\\
946.799999999996	25.1874537965994\\
946.899999999996	25.19535374868\\
947.299999999999	25.1847160650201\\
947.399999999993	25.1926121372032\\
947.500000000003	25.1899535669059\\
947.600000000002	25.1978474200696\\
947.800000000002	25.1925308576854\\
947.999999999994	25.2083113595612\\
948.200000000005	25.2029948328588\\
948.300000000007	25.2108814846057\\
949.000000000006	25.192287430197\\
949.100000000008	25.2001685630004\\
949.200000000005	25.197513957653\\
949.299999999999	25.2053928797135\\
949.899999999994	25.1894736842105\\
950.000000000007	25.197347647616\\
950.100000000001	25.1946958535045\\
950.199999999996	25.2025676102283\\
951.199999999999	25.176074844949\\
951.300000000007	25.1839394576414\\
951.600000000001	25.176000840601\\
951.699999999998	25.1838621559151\\
952.1	25.1732829237555\\
952.200000000002	25.1811403969337\\
952.899999999997	25.1626442812172\\
953.000000000003	25.1704962753121\\
953.200000000007	25.1652155669779\\
953.299999999997	25.1730648206419\\
953.599999999999	25.1651462724127\\
953.800000000004	25.1808365656777\\
953.900000000008	25.1781970649895\\
953.999999999999	25.1860391992454\\
954.099999999996	25.1833997065605\\
954.200000000003	25.191239652101\\
955.4	25.1596023024594\\
955.499999999993	25.1674340728338\\
955.799999999995	25.1595355162674\\
955.899999999994	25.1673640167364\\
956.599999999999	25.1489495139542\\
956.799999999994	25.1645940014631\\
957.000000000004	25.159335492634\\
957.199999999997	25.174971273373\\
957.8	25.1592024219647\\
957.899999999999	25.1670146137787\\
957.999999999997	25.164387850955\\
958.099999999994	25.1721978710081\\
958.700000000001	25.1564455569462\\
958.800000000006	25.1642507039316\\
958.899999999995	25.1616266944734\\
959.000000000006	25.1694296736524\\
959.300000000008	25.1615593079008\\
959.499999999997	25.177157148812\\
959.899999999999	25.1666666666667\\
960.100000000001	25.1822536971464\\
961.100000000007	25.1560549313358\\
961.199999999998	25.1638406324769\\
961.400000000001	25.1586063442538\\
961.499999999993	25.1663893510815\\
961.800000000003	25.1585403888138\\
961.899999999995	25.1663201663202\\
962.400000000008	25.1532467532468\\
962.599999999994	25.1687960943181\\
962.699999999997	25.1661819692563\\
962.799999999999	25.1739536815869\\
963.000000000005	25.1687259889939\\
963.099999999995	25.1764950166113\\
963.200000000006	25.1738814491851\\
963.299999999993	25.1816483288354\\
963.500000000005	25.1764217517642\\
963.700000000004	25.1919485370409\\
964.100000000006	25.1814976146028\\
964.200000000002	25.1892564554599\\
964.400000000005	25.1840331778123\\
964.600000000007	25.1995438996579\\
964.900000000001	25.1917098445596\\
964.999999999994	25.199461195731\\
965.199999999997	25.1942401326013\\
965.300000000007	25.2019888129273\\
965.400000000004	25.1993785603314\\
965.500000000004	25.2071251035626\\
966.000000000001	25.1940792878584\\
966.100000000006	25.2018215690333\\
966.6	25.1887865935657\\
966.700000000005	25.1965246172942\\
966.800000000003	25.1939187092771\\
966.900000000006	25.2016546018614\\
967.099999999995	25.1964433416046\\
967.200000000006	25.2041765739688\\
967.800000000003	25.1885525364191\\
967.900000000004	25.1962809917355\\
968.200000000003	25.1884746462873\\
968.299999999998	25.1961999173895\\
968.900000000008	25.1805985552116\\
968.999999999997	25.1883190589206\\
969.200000000003	25.1831218405035\\
969.299999999995	25.1908396946565\\
969.399999999995	25.1882413615266\\
969.499999999999	25.1959570957096\\
969.600000000006	25.1933587707538\\
969.699999999997	25.201072386059\\
970.400000000004	25.1828954147347\\
970.700000000009	25.2060156571899\\
970.799999999994	25.2034195076733\\
970.900000000006	25.211122554068\\
973.799999999998	25.1360509292535\\
973.9	25.1437371663244\\
974.199999999994	25.135995073386\\
974.300000000006	25.1436781609195\\
974.900000000007	25.1282051282051\\
975.000000000008	25.1358834991283\\
975.699999999993	25.1178520188563\\
975.800000000009	25.125525156266\\
975.900000000001	25.1229508196721\\
976.099999999993	25.1382913337431\\
976.900000000004	25.1177072671443\\
976.999999999995	25.1253709958039\\
977.099999999997	25.1227998362669\\
977.200000000008	25.1304614754937\\
977.899999999998	25.1124744376278\\
978.000000000001	25.1201308659646\\
978.499999999993	25.1072961373391\\
978.799999999995	25.130248237818\\
979.200000000006	25.1199836617992\\
979.299999999998	25.1276291607106\\
979.600000000006	25.1199346738798\\
979.699999999998	25.1275770565422\\
980.000000000001	25.1198857259463\\
980.299999999998	25.1427988576091\\
980.399999999998	25.1402345741968\\
980.600000000004	25.1555011726318\\
981.400000000005	25.1349974528783\\
981.499999999999	25.1426242868786\\
982.400000000002	25.1195928753181\\
982.499999999995	25.1272135151639\\
983.699999999998	25.096564342346\\
983.900000000001	25.1117886178862\\
984.800000000005	25.088841506752\\
984.899999999994	25.0964467005076\\
985.399999999996	25.0837138508371\\
985.499999999999	25.0913149350649\\
986.499999999997	25.0658828299209\\
986.600000000008	25.0734772473903\\
986.700000000002	25.0709363599514\\
986.800000000007	25.0785287263147\\
987.000000000002	25.0734474723939\\
987.200000000003	25.0886255444141\\
987.600000000001	25.0784651209882\\
987.700000000008	25.0860498076534\\
987.800000000001	25.0835104767689\\
988	25.0986742232568\\
988.199999999994	25.0935950622281\\
988.299999999993	25.1011736139215\\
988.700000000002	25.0910194174757\\
988.800000000003	25.0985943978158\\
989.300000000004	25.085910652921\\
989.399999999998	25.0934815563416\\
989.600000000008	25.0884106294837\\
989.700000000004	25.0959789856537\\
989.800000000006	25.0934437822002\\
989.899999999997	25.1010101010101\\
989.999999999993	25.0984749015251\\
990.099999999994	25.1060391840032\\
990.400000000002	25.0984351337708\\
990.500000000001	25.1059963658389\\
990.799999999999	25.0983953981229\\
990.899999999996	25.1059535822402\\
991.099999999995	25.1008878127522\\
991.299999999996	25.115997579181\\
991.499999999995	25.1109318273497\\
991.599999999993	25.1184834123223\\
991.699999999998	25.1159507965316\\
991.800000000004	25.1235003528582\\
992.900000000006	25.0956696878147\\
993.100000000001	25.1107531212243\\
993.499999999996	25.1006441223833\\
993.600000000002	25.1081815437255\\
994.499999999997	25.0854614920571\\
994.600000000004	25.0929928621695\\
994.699999999995	25.0904704463209\\
994.900000000004	25.105527638191\\
995.500000000003	25.0903977501004\\
995.600000000003	25.0979210605604\\
996.200000000002	25.0828063836194\\
996.300000000006	25.0903251706142\\
996.4	25.0878073256397\\
996.500000000006	25.0953241019466\\
996.999999999998	25.0827399458429\\
997.099999999994	25.0902527075812\\
997.2	25.0877368896019\\
997.299999999997	25.0952476438741\\
997.399999999995	25.0927318295739\\
997.500000000008	25.1002405773857\\
998.199999999998	25.082640488831\\
998.300000000007	25.0901442307692\\
998.500000000009	25.0851191668336\\
998.600000000006	25.0926204065285\\
999.000000000001	25.0825743168852\\
999.099999999994	25.0900720576461\\
999.299999999995	25.0850510306184\\
999.499999999997	25.1000400160064\\
999.700000000004	25.0950190038008\\
999.800000000004	25.1025102510251\\
999.899999999999	25.1\\
1000	25.1074892510749\\
1000.3	25.0999600159936\\
1000.4	25.1074462768616\\
1000.9	25.0949050949051\\
1001.10000000001	25.1098681582101\\
1001.5	25.0998402555911\\
1001.6	25.1073175601477\\
1002.3	25.0897845171588\\
1002.4	25.0972568578554\\
1002.50000000001	25.0947536405346\\
1002.60000000001	25.1022239952129\\
1003.10000000001	25.0897129186603\\
1003.2	25.0971793082827\\
1004	25.0771835474554\\
1004.2	25.0921039530021\\
1004.30000000001	25.089605734767\\
1004.4	25.0970632155301\\
1004.5	25.0945650009954\\
1004.6	25.1020205036329\\
1005.4	25.0820487319741\\
1005.5	25.0894988066826\\
1005.7	25.0845098429111\\
1006.2	25.121733081586\\
1006.5	25.1142459765547\\
1006.6	25.1216847124267\\
1006.9	25.1142005958292\\
1007.00000000001	25.12163638169\\
1007.1	25.1191421763304\\
1007.20000000001	25.1265759952348\\
1007.69999999999	25.1141099424489\\
1007.79999999999	25.1215398353011\\
1008	25.1165558972324\\
1008.1	25.1239833366396\\
1008.8	25.1065516899594\\
1008.9	25.1139742319128\\
1009.7	25.0940780352545\\
1009.8	25.1014951975443\\
1010.00000000001	25.0965250965251\\
1010.2	25.1113530634465\\
1010.8	25.0964487090711\\
1010.9	25.1038575667656\\
1011.8	25.0815297954343\\
1011.9	25.0889328063241\\
1012.9	25.0641658440276\\
1012.99999999999	25.0715625308459\\
1013.4	25.0616674888999\\
1013.5	25.0690607734807\\
1013.80000000001	25.061643160075\\
1013.89999999999	25.069033530572\\
1014.4	25.0566781665845\\
1014.69999999999	25.0788332676389\\
1014.8	25.0763622031727\\
1014.9	25.0837438423645\\
1014.99999999999	25.0812727810068\\
1015.2	25.0960307298336\\
1015.30000000001	25.0935591884971\\
1015.8	25.1304262230534\\
1016.09999999999	25.1230072820311\\
1016.30000000001	25.137741046832\\
1016.4	25.1352680767339\\
1016.6	25.1499950821285\\
1017.00000000001	25.1401042178744\\
1017.1	25.147463625639\\
1017.50000000001	25.1375786163522\\
1017.7	25.1522892513264\\
1017.8	25.1498182532665\\
1017.9	25.1571709233792\\
1018.6	25.139884166094\\
1018.8	25.1545784669742\\
1018.9	25.1521099116781\\
1019	25.1594544205672\\
1019.5	25.1471165162809\\
1019.60000000001	25.1544571932921\\
1019.79999999999	25.1495244631827\\
1019.9	25.156862745098\\
1020.09999999999	25.1519309939228\\
1020.2	25.1592668822895\\
1020.30000000001	25.15680125441\\
1020.39999999999	25.1641352278295\\
1020.5	25.1616696061141\\
1020.6	25.1690016655237\\
1021.6	25.1443672310854\\
1021.70000000001	25.1516930906244\\
1021.9	25.146771037182\\
1022	25.1540945113003\\
1022.5	25.1417954234305\\
1022.6	25.1491150875135\\
1022.8	25.1441978688044\\
1022.90000000001	25.1515151515152\\
1023.09999999999	25.1465989053948\\
1023.2	25.1539138082674\\
1023.6	25.1440851812054\\
1023.7	25.1513967571791\\
1024.1	25.1415739113454\\
1024.20000000001	25.1488821634287\\
1024.4	25.1439726695949\\
1024.5	25.1512785477259\\
1024.69999999999	25.1463700234192\\
1024.79999999999	25.1536735291248\\
1025.1	25.1463129145533\\
1025.20000000001	25.1536135765142\\
1025.80000000001	25.1389024271372\\
1025.9	25.1461988304094\\
1027.20000000001	25.1143774944028\\
1027.50000000001	25.1362397820164\\
1027.80000000001	25.1289035898434\\
1027.90000000001	25.136186770428\\
1028.3	25.1264099572151\\
1028.50000000001	25.1409683064359\\
1028.6	25.1385243511228\\
1028.70000000001	25.145800933126\\
1029.1	25.1360279828993\\
1029.2	25.1433012727096\\
1030.69999999999	25.1067132324408\\
1030.79999999999	25.1139780774081\\
1031.1	25.1066718386346\\
1031.20000000001	25.113933869873\\
1031.3	25.1114989334885\\
1031.40000000001	25.1187590887058\\
1031.8	25.1090221920729\\
1031.89999999999	25.1162790697674\\
1032.00000000001	25.113845557601\\
1032.3	25.1356063541263\\
1032.89999999999	25.1210067763795\\
1033	25.1282547672055\\
1034.10000000001	25.1015277509186\\
1034.2	25.1087692158948\\
1034.3	25.1063418406806\\
1034.6	25.1280564414806\\
1035	25.1183460535214\\
1035.4	25.1472718493481\\
1036.8	25.113318545665\\
1036.89999999999	25.1205400192864\\
1037.5	25.1060138781804\\
1037.6	25.11323118435\\
1038.7	25.0866384289565\\
1038.8	25.0938492636442\\
1038.9	25.0914340712223\\
1039.2	25.1130568651977\\
1040.60000000001	25.0792735658691\\
1040.7	25.086471944658\\
1040.80000000001	25.084061869536\\
1041	25.098453558736\\
1041.39999999999	25.0888142102736\\
1041.69999999999	25.1103858706086\\
1042.4	25.0935251798561\\
1042.5	25.1007097640514\\
1042.7	25.0958956655159\\
1042.8	25.1030779557004\\
1043.2	25.0934534649669\\
1043.30000000001	25.1006325474411\\
1043.4	25.0982271202683\\
1043.5	25.1054043694902\\
1043.9	25.095785440613\\
1044.10000000001	25.1101321585903\\
1044.3	25.1053236307928\\
1044.50000000001	25.1196630289106\\
1044.60000000001	25.1172585431224\\
1044.7	25.1244257274119\\
1044.90000000001	25.1196172248804\\
1045	25.1267821261123\\
1045.2	25.12197455276\\
1045.30000000001	25.1291371723742\\
1045.49999999999	25.1243305279265\\
1045.9	25.1529636711281\\
1046.00000000001	25.1505592199599\\
1046.10000000001	25.157713630281\\
1046.2	25.1553091847463\\
1046.30000000001	25.1624617737003\\
1046.4	25.1600573339704\\
1046.49999999999	25.1672081024269\\
};
\addplot [color=mycolor1, forget plot]
  table[row sep=crcr]{%
1046.49999999999	25.1672081024269\\
1046.70000000001	25.1623996943065\\
1046.8	25.169548189894\\
1046.89999999999	25.1671442215855\\
1047.00000000001	25.1742908986725\\
1047.1	25.1718869365928\\
1047.2	25.179031796047\\
1047.40000000001	25.1742243436754\\
1047.5	25.1813669339443\\
1047.9	25.1717557251908\\
1047.99999999999	25.1788951435932\\
1048.4	25.169289461135\\
1048.5	25.1764257104711\\
1049	25.1644266514155\\
1049.1	25.1715592832634\\
1049.29999999999	25.1667619592148\\
1049.40000000001	25.1738923296808\\
1049.49999999999	25.171493902439\\
1049.60000000001	25.178622463561\\
1049.90000000001	25.1714285714286\\
1050.3	25.1999238385377\\
1051.2	25.1783506135261\\
1051.3	25.1854669963858\\
1051.60000000001	25.1782827802605\\
1051.7	25.1853964632059\\
1051.9	25.180608365019\\
1051.99999999999	25.1877197984982\\
1052.2	25.1829326237765\\
1052.30000000001	25.190041809198\\
1052.5	25.1852555576667\\
1052.60000000001	25.1923624964377\\
1052.8	25.1875771678222\\
1052.89999999999	25.1946818613485\\
1053.6	25.1779443864478\\
1053.7	25.1850446004935\\
1053.9	25.180265654649\\
1054	25.1873636277393\\
1054.2	25.1825856018211\\
1054.29999999999	25.1896813353566\\
1054.39999999999	25.1872925557136\\
1054.50000000001	25.1943864972501\\
1054.9	25.1848341232227\\
1055.10000000001	25.1990144048522\\
1055.30000000001	25.1942391510328\\
1055.6	25.21549682675\\
1055.9	25.2083333333333\\
1055.99999999999	25.2154152068933\\
1056.19999999999	25.2106409164063\\
1056.30000000001	25.2177205603938\\
1057.00000000001	25.2010216630404\\
1057.1	25.2080968596292\\
1058.2	25.1818954927714\\
1058.4	25.1960321209258\\
1058.80000000001	25.1865143073\\
1058.9	25.1935788479698\\
1059	25.1912000755358\\
1059.1	25.1982628398792\\
1059.90000000001	25.1792452830189\\
1060	25.1863031789454\\
1060.4	25.1768033946252\\
1060.50000000001	25.1838581934754\\
1060.9	25.1743638077286\\
1061	25.1814155122043\\
1062.4	25.1482352941176\\
1062.70000000001	25.1693639442981\\
1063.1	25.1598946576373\\
1063.39999999999	25.1810061118947\\
1063.49999999999	25.1786385859346\\
1063.60000000001	25.1856726520636\\
1064.00000000001	25.1762052438681\\
1064.1	25.1832362337906\\
1064.6	25.1714097867944\\
1064.7	25.1784372652141\\
1064.79999999999	25.1760728706921\\
1065.00000000001	25.1901229931462\\
1065.1	25.1877581674803\\
1065.2	25.1947808129166\\
1065.6	25.1853242000563\\
1065.7	25.1923437793207\\
1066	25.1852546665416\\
1066.19999999999	25.1992872549939\\
1066.4	25.1945616502579\\
1066.49999999999	25.2015750984437\\
1066.7	25.1968503937008\\
1066.80000000001	25.2038616552629\\
1067.1	25.1967766116942\\
1067.2	25.2037852525063\\
1067.30000000001	25.2014240209856\\
1067.40000000001	25.2084309133489\\
1067.50000000001	25.2060696890221\\
1067.6	25.2130748337548\\
1068.40000000001	25.1941974730931\\
1068.5	25.2011978289351\\
1069.29999999999	25.1823452403217\\
1069.39999999999	25.1893408134642\\
1069.60000000001	25.1846312050108\\
1069.7	25.1916246027295\\
1069.90000000001	25.1869158878505\\
1070	25.1939071114849\\
1070.2	25.1891992899187\\
1070.30000000001	25.1961883408072\\
1070.40000000001	25.1938346567025\\
1070.49999999999	25.2008219689894\\
1071.5	25.177304964539\\
1071.59999999999	25.1842866473827\\
1071.9	25.1772388059701\\
1071.99999999999	25.1842178901222\\
1072.3	25.1771726967549\\
1072.5	25.191124370688\\
1072.8	25.1840805294063\\
1072.90000000001	25.1910531220876\\
1074.10000000001	25.1629119344629\\
1074.2	25.1698780601322\\
1074.5	25.1628512935046\\
1074.6	25.1698148320461\\
1074.90000000001	25.1627906976744\\
1075.09999999999	25.1767113095238\\
1075.39999999999	25.1696885169689\\
1075.5	25.1766455931573\\
1075.7	25.1719650492657\\
1075.89999999999	25.185873605948\\
1076.00000000001	25.1835331288914\\
1076.1	25.1904850399554\\
1076.99999999999	25.1694364497261\\
1077.1	25.1763832157445\\
1078.6	25.1413738759618\\
1078.69999999999	25.148312940304\\
1079.3	25.1343338891977\\
1079.4	25.1412691060676\\
1079.80000000001	25.1319566626539\\
1080	25.1458198314971\\
1080.19999999999	25.141164491345\\
1080.30000000001	25.1480932987782\\
1080.5	25.1434388302795\\
1080.60000000001	25.1503655038401\\
1081	25.1410600314494\\
1081.1	25.1479837217906\\
1081.2	25.1456580042541\\
1081.30000000001	25.1525799889033\\
1081.70000000001	25.1432797189869\\
1081.8	25.1501987244662\\
1081.9	25.1478743068392\\
1082	25.1547916089086\\
1082.6	25.1408515747668\\
1082.69999999999	25.1477650535648\\
1083.5	25.1291989664083\\
1083.60000000001	25.1361077789056\\
1083.99999999999	25.1268333179596\\
1084.3	25.147547030616\\
1084.49999999999	25.1429098285082\\
1084.60000000001	25.1498110076519\\
1085.3	25.1335913027455\\
1085.5	25.1473839351511\\
1085.69999999999	25.1427518880088\\
1085.80000000001	25.1496454553826\\
1085.99999999999	25.1450142712457\\
1086.10000000001	25.1519057263856\\
1086.2	25.149590352573\\
1086.29999999999	25.1564801178203\\
1086.39999999999	25.1541647491947\\
1086.7	25.1748251748252\\
1089.10000000001	25.119353654058\\
1089.2	25.1262278527495\\
1089.6	25.1170046801872\\
1089.7	25.1238759405395\\
1089.9	25.1192660550459\\
1090	25.1261352169526\\
1090.19999999999	25.1215261854535\\
1090.3	25.1283932501834\\
1090.70000000001	25.1191785845251\\
1090.80000000001	25.1260427170226\\
1091.3	25.114531794026\\
1091.39999999999	25.1213925790197\\
1091.8	25.1121897609671\\
1091.90000000001	25.1190476190476\\
1091.99999999999	25.1167475505906\\
1092.09999999999	25.1236037355796\\
1092.3	25.1190040278286\\
1092.50000000001	25.1327109646714\\
1092.8	25.1258120596578\\
1092.9	25.1326623970723\\
1093.7	25.1142804900347\\
1093.80000000001	25.1211262455435\\
1094	25.1165341376474\\
1094.10000000001	25.1233778102723\\
1095.3	25.0958553952894\\
1095.4	25.1026928343222\\
1095.6	25.0981107967509\\
1095.80000000001	25.1117802719226\\
1096.2	25.1026178965612\\
1096.30000000001	25.1094491061656\\
1096.70000000001	25.100291757841\\
1096.8	25.1071200656395\\
1097	25.1025430680886\\
1097.1	25.1093693036821\\
1097.40000000001	25.1025056947608\\
1097.5	25.1093294460641\\
1098.40000000001	25.0887573964497\\
1098.49999999999	25.0955761878755\\
1099.29999999999	25.0773148990358\\
1099.4	25.0841291496135\\
1099.7	25.0772867794144\\
1099.79999999999	25.084098554414\\
1099.90000000001	25.0818181818182\\
1100.1	25.0954371932376\\
1100.20000000001	25.0931564118877\\
1100.30000000001	25.099963649582\\
1100.39999999999	25.0976828714221\\
1100.5	25.1044884608395\\
1101.60000000001	25.0794227103567\\
1101.7	25.0862225449265\\
1102.30000000001	25.0725689404935\\
1102.40000000001	25.0793650793651\\
1104.3	25.0362187613184\\
1104.59999999999	25.056576446094\\
1105.39999999999	25.0384441429218\\
1105.5	25.0452243125904\\
1105.8	25.0384302378154\\
1105.9	25.0452079566004\\
1106.8	25.024844159364\\
1106.90000000001	25.0316169828365\\
1107.00000000001	25.02935597507\\
1107.09999999999	25.0361271676301\\
1107.7	25.0225672504062\\
1107.8	25.029334777507\\
1108.5	25.0135305791088\\
1108.90000000001	25.0405770964833\\
1109.3	25.0315485848206\\
1109.39999999999	25.0383055430374\\
1110.1	25.0225184651414\\
1110.3	25.036023054755\\
1110.8	25.0247547033936\\
1110.9	25.031503150315\\
1111	25.0292502925029\\
1111.1	25.0359971202304\\
1111.2	25.0337442634752\\
1111.30000000001	25.0404894727371\\
1111.7	25.0314804821011\\
1111.8	25.0382228617681\\
1111.9	25.0359712230216\\
1112	25.0427119863322\\
1112.09999999999	25.0404603488581\\
1112.2	25.0471994965387\\
1112.3	25.0449478604818\\
1112.4	25.0516853932584\\
1112.8	25.0426812831342\\
1113	25.0561494924086\\
1113.2	25.0516482529417\\
1113.5	25.0718390804598\\
1113.6	25.0695878602855\\
1113.70000000001	25.076315316933\\
1114.7	25.05382131324\\
1114.80000000001	25.0605435465064\\
1115.19999999999	25.0515556352551\\
1115.3	25.0582750582751\\
1116.10000000001	25.040315355671\\
1116.3	25.0537441777141\\
1116.40000000001	25.051500223914\\
1116.5	25.0582124305929\\
1117.3	25.0402720601396\\
1117.40000000001	25.0469798657718\\
1117.8	25.038017711781\\
1117.90000000001	25.0447227191413\\
1118.7	25.0268144440472\\
1118.80000000001	25.0335150594334\\
1119.50000000001	25.0178635226867\\
1119.59999999999	25.0245601500402\\
1119.90000000001	25.0178571428571\\
1120	25.0245513793411\\
1120.20000000001	25.0200839060966\\
1120.3	25.0267761513745\\
1120.69999999999	25.0178443968594\\
1120.80000000001	25.0245338567223\\
1122	24.997772034578\\
1122.1	25.0044555337729\\
1122.39999999999	24.9977728285078\\
1122.50000000001	25.0044539461963\\
1122.9	24.9955476402493\\
1123	25.0022259816579\\
1125.69999999999	24.9422632794457\\
1125.79999999999	24.9489297450928\\
1126.49999999999	24.9334280134919\\
1126.6	24.940090529866\\
1126.9	24.9334516415262\\
1127	24.9401117913229\\
1127.4	24.9312638580931\\
1127.5	24.9379212486697\\
1128.10000000001	24.9246587484489\\
1128.3	24.9379652605459\\
1128.80000000001	24.9269200106298\\
1129.00000000001	24.9402178726419\\
1129.40000000001	24.9313855688358\\
1129.5	24.9380311614731\\
1129.80000000001	24.9314098592796\\
1129.9	24.9380530973451\\
1130.3	24.929228591649\\
1130.5	24.9425084026181\\
1130.59999999999	24.940302467498\\
1130.69999999999	24.9469402193138\\
1131.30000000001	24.9337104472335\\
1131.4	24.9403446752099\\
1131.49999999999	24.9381406857547\\
1131.60000000001	24.9447733498277\\
1131.70000000001	24.9425693585439\\
1131.79999999999	24.9492004594045\\
1132.40000000001	24.9359823399558\\
1132.50000000001	24.9426099240685\\
1132.90000000001	24.9338040600177\\
1133.09999999999	24.9470525944229\\
1133.40000000001	24.9404499338333\\
1133.49999999999	24.9470712773465\\
1134.30000000001	24.9294781382228\\
1134.39999999999	24.9360951961216\\
1134.9	24.9251101321586\\
1135	24.9317240771738\\
1135.10000000001	24.9295278365046\\
1135.2	24.9361402272527\\
1135.9	24.9207746478873\\
1136.00000000001	24.9273831528915\\
1136.50000000001	24.9164173851839\\
1136.6	24.9230227852556\\
1136.8	24.9186384026739\\
1136.9	24.9252418645559\\
1137	24.9230498636883\\
1137.09999999999	24.9296517762927\\
1138.00000000001	24.9099376153238\\
1138.2	24.9231309848019\\
1138.49999999999	24.9165642016512\\
1138.59999999999	24.9231579871784\\
1139	24.9144061100869\\
1139.10000000001	24.9209971910112\\
1139.59999999999	24.9100640519435\\
1139.8	24.9232388806036\\
1140.00000000001	24.9188667660732\\
1140.29999999999	24.9386180287618\\
1140.9	24.9255039439089\\
1141	24.932083077732\\
1141.6	24.9189804677236\\
1141.80000000001	24.9321306594273\\
1142.00000000001	24.9277646440767\\
1142.1	24.9343372439153\\
1142.20000000001	24.9321544252823\\
1142.3	24.9387254901961\\
1142.6	24.932178174499\\
1142.70000000001	24.9387469373469\\
1143.4	24.923480542195\\
1143.7	24.9431718831964\\
1144.5	24.92573824917\\
1144.6	24.932296671617\\
1145.2	24.9192351348992\\
1145.5	24.9388966480447\\
1145.80000000001	24.9323675713413\\
1145.89999999999	24.9389179755672\\
1146.10000000001	24.9345663932996\\
1146.2	24.9411148913897\\
1146.70000000001	24.9302406696896\\
1146.9	24.9433304272014\\
1147.30000000001	24.9346348265644\\
1147.4	24.9411764705882\\
1147.6	24.9368301821033\\
1147.7	24.9433699250741\\
1147.89999999999	24.9390243902439\\
1148.00000000001	24.9455622332549\\
1148.40000000001	24.9368741837179\\
1148.50000000001	24.9434093679262\\
1148.6	24.9412379211282\\
1148.79999999999	24.9543041169815\\
1149.00000000001	24.9499608389174\\
1149.10000000001	24.9564914723286\\
1149.49999999999	24.9478079331942\\
1149.6	24.9543359137166\\
1149.80000000001	24.9499956517958\\
1149.9	24.9565217391304\\
1150	24.954351795496\\
1150.09999999999	24.9608763693271\\
1150.6	24.9500304162684\\
1150.69999999999	24.9565519638512\\
1151.00000000001	24.950047780384\\
1151.3	24.9696022233802\\
1151.5	24.9652657172629\\
1151.60000000001	24.9717808457063\\
1151.7	24.9696127799965\\
1151.90000000001	24.9826388888889\\
1152.1	24.9783023780594\\
1152.30000000001	24.99132245748\\
1152.4	24.9891540130152\\
1152.5	24.9956619816068\\
1153.09999999999	24.9826569545612\\
1153.19999999999	24.9891615364606\\
1153.40000000001	24.9848287819679\\
1153.79999999999	25.0108328278014\\
1154.39999999999	24.9978345604158\\
1154.50000000001	25.0043305040707\\
1154.89999999999	24.995670995671\\
1154.99999999999	25.0021643147779\\
1155.60000000001	24.9891840443022\\
1155.7	24.9956739920401\\
1156.20000000001	24.9848655193289\\
1156.30000000001	24.9913524731927\\
1156.7	24.9827109266943\\
1156.90000000001	24.9956784788245\\
1157.4	24.9848812095032\\
1157.6	24.99784054591\\
1158.1	24.9870488689346\\
1158.20000000001	24.993524993525\\
1158.7	24.98274076631\\
1158.9	24.9956859361519\\
1159.10000000001	24.9913733609386\\
1159.2	24.9978435262659\\
1159.6	24.9892213503492\\
1159.90000000001	25.0086206896552\\
1160.09999999999	25.0043096017928\\
1160.20000000001	25.0107730759286\\
1160.30000000001	25.0086177180283\\
1160.39999999999	25.0150797070228\\
1160.9	25.0043066322136\\
1161.1	25.0172235618326\\
1161.29999999999	25.0129154468745\\
1161.6	25.0322802789016\\
1161.90000000001	25.025817555938\\
1162	25.0322691678857\\
1163.00000000001	25.0107471412604\\
1163.10000000001	25.0171939477304\\
1163.3	25.0128932439402\\
1163.4	25.0193382036957\\
1163.59999999999	25.0150382400962\\
1163.7	25.0214813541846\\
1163.79999999999	25.019331557694\\
1163.90000000001	25.0257731958763\\
1164.2	25.0193249162587\\
1164.29999999999	25.0257643421505\\
1164.89999999999	25.0128755364807\\
1165.1	25.0257466529351\\
1165.3	25.0214518620216\\
1165.39999999999	25.027885027885\\
1166.80000000001	24.9978575713429\\
1167.1	25.017135023989\\
1167.20000000001	25.0149918615609\\
1167.3	25.021415110502\\
1167.6	25.0149867260427\\
1167.69999999999	25.021407775304\\
1167.8	25.0192653480606\\
1167.90000000001	25.0256849315069\\
1168.2	25.0192587520329\\
1168.40000000001	25.0320924261874\\
1169.19999999999	25.0149662191055\\
1169.3	25.021378484693\\
1169.39999999999	25.0192389910218\\
1169.9	25.0512820512821\\
1170.4	25.0405809483127\\
1170.6	25.0533868625609\\
1170.90000000001	25.0469684030743\\
1171	25.0533686277858\\
1171.1	25.0512295081967\\
1171.19999999999	25.0576282762742\\
1171.5	25.0512120177535\\
1171.70000000001	25.0640040962622\\
1171.8	25.0618653468726\\
1171.89999999999	25.0682593856655\\
1172.39999999999	25.0575692963753\\
1172.49999999999	25.0639604298141\\
1172.9	25.0554134697357\\
1173.09999999999	25.0681895669963\\
1173.40000000001	25.0617809970175\\
1173.50000000001	25.068166325835\\
1174.10000000001	25.0553568386987\\
1174.19999999999	25.0617389082858\\
1174.3	25.0596049046322\\
1174.39999999999	25.0659855257556\\
1174.80000000001	25.0574516980169\\
1174.9	25.063829787234\\
1175.2	25.0574321449843\\
1175.3	25.0638080653395\\
1175.7	25.055281510461\\
1175.9	25.0680272108844\\
1176	25.0658957571635\\
1176.09999999999	25.0722666213229\\
1176.6	25.0616129854678\\
1176.8	25.07434786303\\
1176.9	25.072217502124\\
1177	25.0785829581174\\
1177.7	25.0636780438105\\
1177.89999999999	25.0764006791171\\
1178.00000000001	25.0742721330957\\
1178.19999999999	25.0869897309683\\
1178.4	25.0827322868053\\
1178.5	25.0890887493637\\
1178.90000000001	25.0805767599661\\
1178.99999999999	25.0869307098635\\
1179.6	25.0741713995084\\
1179.7	25.0805221223936\\
1180.1	25.0720216912388\\
1180.2	25.0783699059561\\
1180.5	25.0719972895138\\
1180.6	25.0783433556365\\
1181.1	25.0677277345073\\
1181.2	25.0740709387962\\
1181.7	25.0634625148079\\
1181.8	25.069802859802\\
1181.90000000001	25.0676818950931\\
1182.00000000001	25.0740208104221\\
1182.19999999999	25.0697792438467\\
1182.3	25.0761163734777\\
1182.6	25.0697556438657\\
1182.7	25.0760906323977\\
1184	25.0485600878304\\
1184.3	25.0675447483958\\
1184.4	25.0654284508231\\
1184.60000000001	25.0780788385245\\
1184.99999999999	25.0696143785335\\
1185.1	25.0759365507931\\
1185.4	25.0695908899199\\
1185.69999999999	25.0885478158205\\
1187.09999999999	25.0589622641509\\
1187.2	25.065274151436\\
1190.00000000001	25.0063019914293\\
1190.10000000001	25.0126029238783\\
1190.6	25.0020996052742\\
1190.7	25.0083977158213\\
1191.4	24.9937054133445\\
1191.50000000001	25\\
1195.3	24.9205286933244\\
1195.4	24.926808866583\\
1195.80000000001	24.9184714441007\\
1195.99999999999	24.9310258339604\\
1196.2	24.9268578115857\\
1196.4	24.9394066025909\\
1196.69999999999	24.9331550802139\\
1196.9	24.9456975772765\\
1197.2	24.9394470892842\\
1197.29999999999	24.9457157173877\\
1197.39999999999	24.9436325678497\\
1197.5	24.9498997995992\\
1198.40000000001	24.9311639549437\\
1198.50000000001	24.9374269981645\\
1198.60000000001	24.935346625511\\
1198.8	24.9478688798065\\
1198.89999999999	24.9457881567973\\
1199.00000000001	24.95204736886\\
1200.09999999999	24.929178470255\\
1200.20000000001	24.9354328084646\\
1200.5	24.9292020656339\\
1200.6	24.9354543183143\\
1202.40000000001	24.8981288981289\\
1202.59999999999	24.9106177766692\\
1203.30000000001	24.896127638358\\
1203.39999999999	24.9023680930619\\
1203.5	24.9002991026919\\
1203.6	24.9065381739636\\
1204.7	24.8837981407703\\
1205.09999999999	24.908728841686\\
1206.5	24.8798276147853\\
1206.6	24.8860528714676\\
1206.8	24.8819289087745\\
1207.10000000001	24.9005964214712\\
1207.19999999999	24.8985339186615\\
1207.29999999999	24.9047540168958\\
1207.8	24.8944449043795\\
1207.9	24.9006622516556\\
1208.3	24.8924197285667\\
1208.40000000001	24.8986346710799\\
1208.7	24.8924553275976\\
1208.80000000001	24.8986682107701\\
1209.00000000001	24.8945496650401\\
1209.30000000001	24.9131800893005\\
1209.4	24.9111202976437\\
1209.5	24.917328042328\\
1209.6	24.915268248326\\
1209.7	24.9214746239048\\
1209.79999999999	24.9194148276717\\
1209.9	24.9256198347107\\
1211.4	24.8947585637639\\
1211.5	24.900957411687\\
1211.8	24.8947932997772\\
1211.9	24.9009900990099\\
1212.19999999999	24.8948280128681\\
1212.29999999999	24.9010227647641\\
1212.50000000001	24.8969157182913\\
1212.60000000001	24.9031087655644\\
1212.9	24.8969497114592\\
1213	24.9031407138735\\
1213.40000000001	24.8949320148331\\
1213.49999999999	24.901120632828\\
1213.7	24.8970176305816\\
1213.79999999999	24.9032045473268\\
1214.5	24.8888522970525\\
1214.60000000001	24.8950358113114\\
1215.50000000001	24.8766041461007\\
1215.59999999999	24.8827835814757\\
1216.2	24.8705089204966\\
1216.29999999999	24.8766853008879\\
1216.4	24.8746403616934\\
1216.5	24.8808153871445\\
1217.6	24.8583394924858\\
1217.7	24.8645097717195\\
1217.8	24.8624681829378\\
1217.99999999999	24.874805024218\\
1218.9	24.856439704676\\
1219	24.8626035600033\\
1219.20000000001	24.8585253834167\\
1219.3	24.8646875512547\\
1219.4	24.8626486264863\\
1219.49999999999	24.8688094457199\\
1220.00000000001	24.8586181460536\\
1220.10000000001	24.8647762661859\\
1220.6	24.8545916277546\\
1220.70000000001	24.860747051114\\
1220.89999999999	24.8566748566749\\
1220.99999999999	24.8628285971665\\
1221.9	24.8445171849427\\
1222.19999999999	24.862963265974\\
1222.60000000001	24.8548294757504\\
1222.70000000001	24.8609748119071\\
1223.59999999999	24.8426902018469\\
1223.7	24.8488315084164\\
1223.80000000001	24.8468012092491\\
1223.9	24.8529411764706\\
1224.19999999999	24.8468512619456\\
1224.30000000001	24.8529892192094\\
1224.50000000001	24.848930262943\\
1224.70000000001	24.86120182887\\
1224.8	24.8591721773206\\
1224.89999999999	24.865306122449\\
1225.5	24.8531331592689\\
1225.60000000001	24.8592640939871\\
1226.00000000001	24.8511540657369\\
1226.1	24.8572826618822\\
1226.5	24.849176585684\\
1226.60000000001	24.8553028450314\\
1227.1	24.8451760104303\\
1227.2	24.8512996007496\\
1227.40000000001	24.847250509165\\
1227.5	24.8533724340176\\
1227.60000000001	24.8513480491977\\
1227.8	24.8635882400847\\
1228	24.8595391254784\\
1228.1	24.8656570591109\\
1228.2	24.8636326630302\\
1228.30000000001	24.8697492673396\\
1228.60000000001	24.8636770570522\\
1228.7	24.8697916666667\\
1228.8	24.8677679225324\\
1228.9	24.8738812042311\\
1229.29999999999	24.8657881893607\\
1229.4	24.8718991459943\\
1229.59999999999	24.8678539481174\\
1229.7	24.8739632460563\\
1229.80000000001	24.8719408081958\\
1230.1	24.8902617460576\\
1230.20000000001	24.8882386409819\\
1230.3	24.8943433029909\\
1230.8	24.8842310504509\\
1230.9	24.8903330625508\\
1231.19999999999	24.8842686591407\\
1231.3	24.8903686860484\\
1231.5	24.8863267294576\\
1231.6	24.8924251035155\\
1231.99999999999	24.8843438032627\\
1232.1	24.8904398636585\\
1232.4	24.8843813387424\\
1232.5	24.890475417816\\
1233.29999999999	24.8743311172369\\
1233.8	24.904773482454\\
1233.9	24.902755267423\\
1234	24.9088404505308\\
1236.09999999999	24.8665264520304\\
1236.2	24.8726037369571\\
1236.7	24.8625485122898\\
1236.79999999999	24.8686231708303\\
1237.49999999999	24.8545572074984\\
1237.7	24.866698982065\\
1237.8	24.8646902011471\\
1237.9	24.8707592891761\\
1238	24.8687505048058\\
1238.1	24.8748182846067\\
1238.19999999999	24.8728094968909\\
1238.3	24.8788759689922\\
1238.79999999999	24.8688352570829\\
1238.89999999999	24.8748991121873\\
1239.69999999999	24.8588482013228\\
1240	24.8770260462866\\
1240.3	24.871009351822\\
1240.5	24.8831210704498\\
1241.2	24.8690888584548\\
1241.29999999999	24.8751409698727\\
1241.7	24.8671283620551\\
1241.8	24.8731781947017\\
1242.2	24.8651694437736\\
1242.29999999999	24.8712169993561\\
1242.60000000001	24.8652128430031\\
1242.7	24.8712584486643\\
1242.90000000001	24.8672566371681\\
1243	24.8733006194192\\
1243.10000000001	24.8712998712999\\
1243.2	24.8773425561007\\
1244.1	24.8593473718052\\
1244.4	24.8774608276416\\
1244.90000000001	24.8674698795181\\
1245.00000000001	24.873504136214\\
1245.19999999999	24.8695093551755\\
1245.40000000001	24.8815736651947\\
1245.6	24.8775788713173\\
1245.70000000001	24.8836089259913\\
1246.09999999999	24.8756218905473\\
1246.20000000001	24.8816496830619\\
1246.4	24.8776574408343\\
1246.5	24.8836836194449\\
1246.80000000001	24.8776966877857\\
1247	24.8897442065592\\
1247.2	24.8857532269703\\
1247.30000000001	24.8917748917749\\
1247.39999999999	24.8897795591182\\
1247.5	24.8957999358769\\
1247.8	24.8898148890135\\
1247.99999999999	24.9018508132361\\
1248.30000000001	24.895866709388\\
1248.6	24.9139104668856\\
1249.50000000001	24.895966709347\\
1249.6	24.9019764743538\\
1249.7	24.8999839974396\\
1249.8	24.9059924793983\\
1250	24.9020078393728\\
1250.1	24.9080147176452\\
1250.40000000001	24.9020391843263\\
1250.6	24.9140481330455\\
1250.89999999999	24.9080735411671\\
1251.10000000001	24.9200767263427\\
1251.4	24.9141030763084\\
1251.6	24.9261005033155\\
1252.3	24.9121686362185\\
1252.50000000001	24.9241577518761\\
1252.6	24.9221681168676\\
1252.70000000001	24.9281609195402\\
1253	24.9221929614556\\
1253.1	24.9281838493457\\
1253.2	24.9261948456076\\
1253.5	24.9441608168475\\
1253.9	24.9362041467305\\
1254	24.9421896180528\\
1254.10000000001	24.9402009248924\\
1254.19999999999	24.9461851231763\\
1254.59999999999	24.9382322467522\\
1254.70000000001	24.9442142174052\\
1255.09999999999	24.93626513703\\
1255.19999999999	24.9422448817016\\
1255.40000000001	24.9382716049383\\
1255.6	24.9502269650394\\
1255.69999999999	24.9482401656315\\
1255.79999999999	24.954216100008\\
1256.10000000001	24.9482566470307\\
1256.2	24.954230677386\\
1256.40000000001	24.950258654994\\
1256.49999999999	24.9562310997931\\
1256.7	24.9522597071929\\
1256.8	24.9582305672687\\
1256.90000000001	24.9562450278441\\
1257.00000000001	24.962214620953\\
1257.89999999999	24.9443561208267\\
1258.00000000001	24.9503219139973\\
1258.8	24.9344665978235\\
1258.9	24.9404289118348\\
1260	24.9186572494247\\
1260.30000000001	24.9365280863218\\
1260.6	24.9305941143809\\
1260.80000000001	24.9425013878975\\
1261.3	24.9326145552561\\
1261.4	24.9385652001585\\
1261.50000000001	24.9365884590996\\
1261.59999999999	24.9425378457637\\
1261.70000000001	24.9405611031859\\
1261.80000000001	24.9465092321103\\
1263.5	24.9129471351694\\
1263.59999999999	24.9188889768141\\
1263.8	24.9149458026743\\
1264.20000000001	24.9387012576129\\
1264.69999999999	24.9288425047438\\
1264.8	24.9347774527631\\
1264.99999999999	24.9308355070745\\
1265.09999999999	24.936768890294\\
1266.4	24.9111725227004\\
1266.5	24.9171009000474\\
1266.8	24.9112005683164\\
1266.99999999999	24.9230526398864\\
1267.19999999999	24.9191193876746\\
1267.3	24.9250433959287\\
1268.09999999999	24.9093202964832\\
1268.3	24.921160517187\\
1268.5	24.917231593883\\
1268.6	24.9231496807756\\
1268.69999999999	24.921185372005\\
1268.8	24.9271022145165\\
1269.1	24.9212102111566\\
1269.19999999999	24.927125187111\\
1269.40000000001	24.9231981094919\\
1269.5	24.9291115311909\\
1270.90000000001	24.9016522423289\\
1271	24.9075603807726\\
1271.49999999999	24.8977665932683\\
1271.8	24.9154807767906\\
1272.3	24.9056900345803\\
1272.39999999999	24.9115913555992\\
1272.60000000001	24.9076765930699\\
1272.80000000001	24.9194752140781\\
1273	24.9155604430131\\
1273.09999999999	24.9214577442664\\
1273.8	24.9077635607191\\
1274.1	24.9254434154764\\
1274.50000000001	24.9176212144987\\
1274.6	24.9235114144505\\
1274.79999999999	24.9196015373755\\
1274.89999999999	24.9254901960784\\
1275.40000000001	24.9157193257546\\
1275.49999999999	24.9216055189715\\
1275.59999999999	24.919651955789\\
1275.7	24.9255369180122\\
1275.89999999999	24.9216300940439\\
1276.00000000001	24.9275135177494\\
1276.10000000001	24.925560257013\\
1276.19999999999	24.9314424508344\\
1276.3	24.9294891883422\\
1276.4	24.9353701527615\\
1276.7	24.9295112781955\\
1277.00000000001	24.9471458773784\\
1277.30000000001	24.9412869891968\\
1277.5	24.9530369442705\\
1277.60000000001	24.9510839790248\\
1277.8	24.962829642382\\
1277.90000000001	24.9608763693271\\
1278.00000000001	24.9667475158438\\
1278.40000000001	24.958936253422\\
1278.70000000001	24.9765405067251\\
1278.8	24.9745875361639\\
1278.9	24.9804534792807\\
1279.2	24.9745954819042\\
1279.3	24.980459590433\\
1279.40000000001	24.9785072293865\\
1279.5	24.9843701156611\\
1279.79999999999	24.9785139464021\\
1279.9	24.984375\\
1280.7	24.9687695190506\\
1280.79999999999	24.9746272152393\\
1280.89999999999	24.9726775956284\\
1281.2	24.9902442831499\\
1281.99999999999	24.9746509632634\\
1282.09999999999	24.9805022617376\\
1282.2	24.9785541604929\\
1282.3	24.9844042420462\\
1283.00000000001	24.9707739069441\\
1283.3	24.9883122954652\\
1284.49999999999	24.964969640355\\
1284.59999999999	24.970810305908\\
1284.7	24.9688667496887\\
1284.9	24.9805447470817\\
1285.6	24.9669440771564\\
1285.7	24.9727795924716\\
1285.80000000001	24.9708375456878\\
1285.9	24.9766718506998\\
1286.10000000001	24.9727880578448\\
1286.20000000001	24.9786208505014\\
1286.9	24.965034965035\\
1287.10000000001	24.9766935985084\\
1287.69999999999	24.9650566858208\\
1287.9	24.9767080745342\\
1288.3	24.9689537410742\\
1288.4	24.9747768723322\\
1288.8	24.9670261463263\\
1288.90000000001	24.9728471683476\\
1289.20000000001	24.9670363763282\\
1289.4	24.9786739046142\\
1289.60000000001	24.9748003411646\\
1289.70000000001	24.9806171499457\\
1291.00000000001	24.9554643327395\\
1291.1	24.9612763320942\\
1291.50000000001	24.9535459894704\\
1291.59999999999	24.95935588759\\
1292	24.9516291308722\\
1292.39999999999	24.9748549323017\\
1292.6	24.9709909491761\\
1292.69999999999	24.9767945544554\\
1294.2	24.9478482577455\\
1294.39999999999	24.9594438006953\\
1294.70000000001	24.9536607970343\\
1294.8	24.959456328674\\
1295.6	24.9440456895886\\
1295.7	24.9498379379534\\
1295.8	24.9479126475808\\
1296.10000000001	24.9652831353186\\
1296.20000000001	24.9633572475507\\
1296.29999999999	24.9691453255168\\
1296.4	24.9672194369456\\
1296.5	24.9730063242326\\
1299.19999999999	24.9211113676595\\
1299.3	24.9268893335386\\
1301.10000000001	24.8924070089149\\
1301.29999999999	24.9039495927463\\
1301.39999999999	24.9020361121783\\
1301.5	24.9078057775046\\
1301.8	24.9020662109225\\
1302	24.9136011059058\\
1302.1	24.911687912763\\
1302.19999999999	24.9174537356984\\
1302.50000000001	24.9117150314755\\
1302.6	24.9174790819068\\
1302.89999999999	24.911742133538\\
1303	24.9175044125547\\
1303.19999999999	24.9136806567943\\
1303.30000000001	24.9194414607948\\
1303.5	24.9156182878184\\
1303.6	24.92137761755\\
1303.8	24.917555027226\\
1303.90000000001	24.9233128834356\\
1304.40000000001	24.9137600613262\\
1304.6	24.9252701770522\\
1306	24.8985529438787\\
1306.1	24.9043025570357\\
1306.29999999999	24.9004898958971\\
1306.40000000001	24.9062380405664\\
1306.5	24.904331853666\\
1306.90000000001	24.9273144605968\\
1307.3	24.9196879302432\\
1307.4	24.925430210325\\
1307.60000000001	24.9216181081288\\
1307.7	24.9273589233828\\
1307.8	24.9254530162857\\
1307.90000000001	24.9311926605505\\
1308	24.9292867517774\\
1308.10000000001	24.9350252255007\\
1308.40000000001	24.9293083683607\\
1308.5	24.9350450863518\\
1308.6	24.9331397570108\\
1308.70000000001	24.9388753056235\\
1309.49999999999	24.9236408063531\\
1309.59999999999	24.9293731388868\\
1309.79999999999	24.9255668371631\\
1309.99999999999	24.9370277078086\\
1310.39999999999	24.9294162533384\\
1310.5	24.9351442087593\\
1310.6	24.933241779202\\
1310.7	24.9389685688129\\
1311.6	24.9218571319662\\
1311.8	24.9333028432045\\
1312.29999999999	24.9238037183785\\
1312.4	24.9295238095238\\
1312.89999999999	24.9200304645849\\
1313.10000000001	24.9314651233628\\
1313.50000000001	24.9238733252132\\
1313.6	24.9295881860394\\
1313.9	24.923896499239\\
1314	24.9296096187505\\
1314.2	24.9258160237389\\
1314.29999999999	24.9315276932441\\
1314.6	24.9258385943561\\
1314.69999999999	24.9315485244904\\
1315.20000000001	24.9220710104159\\
1315.3	24.9277786224723\\
1315.5	24.9239890544238\\
1315.6	24.9296952192749\\
1317	24.9031964163693\\
1317.1	24.9088976617067\\
1318.29999999999	24.8862257281553\\
1318.4	24.8919226393629\\
1318.60000000001	24.8881474179116\\
1318.70000000001	24.8938428874735\\
1318.9	24.8900682335102\\
1318.99999999999	24.8957622621484\\
1319.9	24.8787878787879\\
1319.99999999999	24.8844784486024\\
1320.60000000001	24.8731733171803\\
1320.7	24.8788612961841\\
1320.90000000001	24.8750946252839\\
1321	24.8807811672091\\
1322.4	24.8544423440454\\
1322.5	24.8601239981854\\
1322.90000000001	24.8526077097506\\
1323	24.8582873554531\\
1323.1	24.8564087061669\\
1323.3	24.8677648481185\\
1323.59999999999	24.8621288811664\\
1323.70000000001	24.8678048043511\\
1324.09999999999	24.8602930070986\\
1324.2	24.8659669259231\\
1324.4	24.8622121555304\\
1324.7	24.8792270531401\\
1324.8	24.8773492339044\\
1324.90000000001	24.8830188679245\\
1325.19999999999	24.8773862521693\\
1325.3	24.8830541723253\\
1325.4	24.8811769143719\\
1325.60000000001	24.8925096175605\\
1326.50000000001	24.8756218905473\\
1326.6	24.8812843898395\\
1327.3	24.8681633268043\\
1327.4	24.8738229755179\\
1327.49999999999	24.8719493823441\\
1327.7	24.883265552041\\
1327.8	24.8813916710596\\
1327.89999999999	24.8870481927711\\
1328.40000000001	24.8776815957847\\
1328.5	24.88333584224\\
1328.70000000001	24.8795906080674\\
1328.8	24.8852434344194\\
1329.09999999999	24.879626843214\\
1329.2	24.8852779658467\\
1329.6	24.8777919831541\\
1329.69999999999	24.8834411189653\\
1329.9	24.8796992481203\\
1330.00000000001	24.8853469663935\\
1330.59999999999	24.8741263996393\\
1330.7	24.8797715659754\\
1331.4	24.866691701089\\
1331.5	24.8723340342445\\
1331.60000000001	24.8704663212435\\
1331.69999999999	24.8761075236522\\
1332	24.870505217326\\
1332.20000000001	24.8817833821211\\
1332.80000000001	24.8705829394553\\
1332.9	24.8762190547637\\
1333	24.8743530117771\\
1333.09999999999	24.8799879987999\\
1333.70000000001	24.8687959214275\\
1333.89999999999	24.880059970015\\
1334.09999999999	24.8763303852496\\
1334.20000000001	24.8819605785805\\
1334.80000000001	24.8707768372163\\
1335	24.8820313085162\\
1335.09999999999	24.8801677651288\\
1335.19999999999	24.8857934546544\\
1335.6	24.8783409448229\\
1335.7	24.8839646653691\\
1336.2	24.8746538950834\\
1336.30000000001	24.8802753666567\\
1336.40000000001	24.8784137673027\\
1336.49999999999	24.8840341164148\\
1336.89999999999	24.8765893792072\\
1337.00000000001	24.8822077630693\\
1337.6	24.8710473200269\\
1337.69999999999	24.8766631783525\\
1337.90000000001	24.8729446935725\\
1338.00000000001	24.878559151035\\
1338.69999999999	24.8655512399163\\
1338.9	24.8767737117252\\
1338.99999999999	24.8749159883504\\
1339.1	24.8805256869773\\
1339.99999999999	24.8638161331244\\
1340.1	24.8694224742576\\
1340.30000000001	24.8657117278424\\
1340.40000000001	24.8713166728833\\
1341	24.8601893967639\\
1341.1	24.8657918282135\\
1341.3	24.8620843894439\\
1341.39999999999	24.8676854267611\\
1342.3	24.8510131108462\\
1342.4	24.8566108007449\\
1343.2	24.8418074890196\\
1343.3	24.847402114039\\
1343.69999999999	24.8400059532669\\
1343.79999999999	24.8455986308505\\
1344	24.8419016442229\\
1344.1	24.8474929325993\\
1344.2	24.8456445733839\\
1344.29999999999	24.851234751562\\
1344.4	24.8493863889922\\
1344.5	24.8549754573851\\
1344.60000000001	24.8531270915446\\
1344.69999999999	24.8587150505651\\
1344.9	24.8550185873606\\
1345.1	24.8661909009813\\
1345.2	24.8643425258307\\
1345.29999999999	24.8699271592092\\
1345.49999999999	24.8662306777646\\
1345.60000000001	24.8718139258379\\
1346.09999999999	24.8625761402466\\
1346.2	24.8681571715071\\
1346.90000000001	24.8552338530067\\
1347.00000000001	24.8608121149135\\
1347.2	24.8571216507088\\
1347.39999999999	24.8682745825603\\
1347.8	24.860894725128\\
1348	24.8720421333729\\
1348.10000000001	24.8701973001038\\
1348.20000000001	24.8757694875028\\
1348.50000000001	24.870235800089\\
1348.7	24.8813760379597\\
1349	24.8758431546957\\
1349.09999999999	24.881411206641\\
1349.90000000001	24.8666666666667\\
1350.10000000001	24.8777958820915\\
1350.3	24.8741113744076\\
1350.5	24.8852361913224\\
1350.99999999999	24.876026941011\\
1351.09999999999	24.8815867377146\\
1351.30000000001	24.8779043954418\\
1351.39999999999	24.8834628190899\\
1351.90000000001	24.8742603550296\\
1352	24.8798165816138\\
1352.19999999999	24.8761369518598\\
1352.3	24.8816918071577\\
1352.4	24.8798521256932\\
1352.5	24.8854058849623\\
1352.70000000001	24.8817267888823\\
1352.80000000001	24.8872791780619\\
1353.09999999999	24.8817617499261\\
1353.20000000001	24.8873124953817\\
1353.79999999999	24.8762833296403\\
1354	24.8873790709697\\
1354.1	24.8855412789839\\
1354.19999999999	24.8910876467548\\
1354.3	24.8892498523331\\
1354.50000000001	24.9003395836409\\
1354.79999999999	24.8948261864344\\
1354.90000000001	24.90036900369\\
1355.4	24.8911840649207\\
1355.49999999999	24.8967246975509\\
1355.6	24.8948882496127\\
1355.8	24.9059665167048\\
1356.39999999999	24.8949502395872\\
1356.5	24.9004865103936\\
1356.6	24.8986511387927\\
1356.8	24.9097206868598\\
1356.99999999999	24.9060496647263\\
1357.20000000001	24.9171148603846\\
1357.89999999999	24.9042709867452\\
1358.09999999999	24.9153291120601\\
1358.2	24.9134948096886\\
1358.4	24.9245491350755\\
1358.49999999999	24.922714559105\\
1358.6	24.9282402296313\\
1359.00000000001	24.9209035391068\\
1359.1	24.9264273101825\\
1359.2	24.9245935407931\\
1359.30000000001	24.9301162277475\\
1359.7	24.9227827621709\\
1359.8	24.9283035517317\\
1360.10000000001	24.9228054697839\\
1360.19999999999	24.9283246342719\\
1360.3	24.9264922081741\\
1360.40000000001	24.9320102903344\\
1360.49999999999	24.9301778627076\\
1360.6	24.9356948629382\\
1360.7	24.9338624338624\\
1360.8	24.9393783525608\\
1360.99999999999	24.9357137609287\\
1361.1	24.9412283279459\\
1361.4	24.9357326478149\\
1361.5	24.9412455934195\\
1361.7	24.9375826112498\\
1361.9	24.9486049926579\\
1362.00000000001	24.9467733646575\\
1362.09999999999	24.952283071502\\
1362.19999999999	24.9504514424136\\
1362.30000000001	24.9559600704639\\
1362.59999999999	24.9504659866442\\
1362.79999999999	24.9614791987673\\
1362.99999999999	24.9578167412516\\
1363.10000000001	24.9633215962441\\
1363.90000000001	24.9486803519062\\
1364.00000000001	24.9541822447035\\
1364.40000000001	24.9468669842433\\
1364.8	24.9688621877061\\
1364.99999999999	24.9652040143579\\
1365.09999999999	24.9707002636976\\
1365.30000000001	24.9670426248718\\
1365.39999999999	24.9725375320395\\
1365.80000000001	24.9652243941723\\
1365.90000000001	24.9707174231332\\
1366.6	24.9579278554182\\
1366.70000000001	24.9634182031021\\
1367.1	24.9561146869514\\
1367.20000000001	24.9616031595114\\
1367.5	24.9561275226674\\
1367.69999999999	24.9671004532826\\
1368.50000000001	24.9525062107263\\
1368.6	24.9579893329437\\
1370.00000000001	24.9324866798044\\
1370.09999999999	24.9379652605459\\
1370.5	24.9306872902379\\
1370.60000000001	24.9361640037937\\
1371	24.9288892130406\\
1371.1	24.9343640606768\\
1371.50000000001	24.9270924467775\\
1371.60000000001	24.9325654297587\\
1371.89999999999	24.9271137026239\\
1372	24.9325850885504\\
1372.4	24.9253187613843\\
1372.49999999999	24.9307882850066\\
1372.60000000001	24.9289720987834\\
1372.7	24.9344405594406\\
1372.89999999999	24.9308084486526\\
1373.00000000001	24.9362755808026\\
1373.20000000001	24.9326439962135\\
1373.3	24.9381098004951\\
1373.4	24.9362941390608\\
1373.5	24.9417588817705\\
1373.7	24.9381278206435\\
1373.8	24.9435912366257\\
1374.19999999999	24.9363312231682\\
1374.30000000001	24.941792782305\\
1374.60000000001	24.9363497490362\\
1374.8	24.9472688922831\\
1375.00000000001	24.9436404625118\\
1375.10000000001	24.9490983129727\\
1375.6	24.940030529912\\
1375.80000000001	24.9509412021223\\
1376.10000000001	24.9455021072519\\
1376.30000000001	24.9564080209242\\
1376.8	24.9473454862372\\
1376.89999999999	24.9527959331881\\
1377.00000000001	24.9509839517827\\
1377.1	24.9564333430148\\
1377.60000000001	24.9473760615519\\
1377.80000000001	24.9582698309021\\
1378.20000000001	24.9510266270043\\
1378.30000000001	24.9564712710389\\
1378.39999999999	24.9546608632572\\
1378.50000000001	24.9601044537937\\
1379.00000000001	24.951055035893\\
1379.1	24.9564965197216\\
1379.2	24.9546871601537\\
1379.3	24.9601275917065\\
1379.8	24.9510834118414\\
1380	24.9619592783132\\
1380.19999999999	24.9583423893357\\
1380.29999999999	24.9637786148942\\
1380.6	24.9583544578837\\
1380.80000000001	24.9692229705265\\
1381.5	24.9565720903301\\
1381.60000000001	24.9620033292321\\
1382.30000000001	24.9493634259259\\
1382.4	24.9547920433996\\
1382.8	24.9475739388242\\
1382.9	24.9530007230658\\
1383	24.9511965873762\\
1383.1	24.9566223250434\\
1383.39999999999	24.9512106975063\\
1383.69999999999	24.9674808498338\\
1384.09999999999	24.960265857535\\
1384.60000000001	24.9873618834405\\
1385.09999999999	24.9783424776206\\
1385.2	24.9837580307515\\
1385.40000000001	24.9801515698304\\
1385.50000000001	24.9855658198614\\
1385.69999999999	24.9819598787704\\
1385.80000000001	24.9873728263223\\
1386.19999999999	24.9801630238765\\
1386.3	24.9855741488748\\
1386.89999999999	24.9747656813266\\
1387.00000000001	24.9801744647106\\
1387.39999999999	24.972972972973\\
1387.5	24.9783799365811\\
1387.6	24.9765799524393\\
1387.7	24.9819858769275\\
1387.9	24.978386167147\\
1388	24.9837907931705\\
1388.40000000001	24.9765934461649\\
1388.59999999999	24.9873982861669\\
1388.7	24.985599078341\\
1388.8	24.9910000719994\\
1388.9	24.9892008639309\\
1389.10000000001	25\\
1389.89999999999	24.9856115107914\\
1389.99999999999	24.9910078411625\\
1391.2	24.9694530295407\\
1391.29999999999	24.9748454793733\\
1391.5	24.971256108077\\
1391.70000000001	24.9820376490875\\
1391.80000000001	24.9802428335369\\
1391.90000000001	24.985632183908\\
1391.99999999999	24.9838373680052\\
1392.10000000001	24.9892256859647\\
1392.4	24.983842010772\\
1392.50000000001	24.989228780698\\
1392.80000000001	24.9838466508723\\
1392.9	24.989231873654\\
1393.09999999999	24.985644559288\\
1393.20000000001	24.9910284935046\\
1393.6	24.9838559230824\\
1393.7	24.9892380542402\\
1393.80000000001	24.9874452973671\\
1393.89999999999	24.9928263988522\\
1394.20000000001	24.9874488990892\\
1394.3	24.9928284566839\\
1394.7	24.9856610266705\\
1395	25.0017919862375\\
1395.10000000001	25\\
1395.2	25.0053751881316\\
1395.3	25.0035832019493\\
1395.40000000001	25.0089573629523\\
1396.20000000001	24.9946286614624\\
1396.29999999999	25\\
1396.40000000001	24.9982098102399\\
1396.50000000001	25.0035801231562\\
1396.79999999999	24.9982103228578\\
1396.89999999999	25.0035790980673\\
1397.30000000001	24.9964219264348\\
1397.39999999999	25.0017889087657\\
1397.7	24.9964229503505\\
1397.80000000001	25.001788396881\\
1397.89999999999	25\\
1398	25.0053644231457\\
1398.39999999999	24.9982123703969\\
1398.50000000001	25.003575003575\\
1398.8	24.9982128815498\\
1398.9	25.0035739814153\\
1399.20000000001	24.9982133924105\\
1399.3	25.0035729598399\\
1399.49999999999	25\\
1399.59999999999	25.0053582910624\\
1399.79999999999	25.0017858418459\\
1399.90000000001	25.0071428571429\\
1400.80000000001	24.9910771646799\\
1400.90000000001	24.9964311206281\\
1401.5	24.9857305936073\\
1401.6	24.991082257259\\
1401.80000000001	24.987516941294\\
1401.9	24.9928673323823\\
1402.2	24.9875205020324\\
1402.30000000001	24.9928693667998\\
1402.4	24.9910873440285\\
1402.60000000001	25.0017822770371\\
1402.99999999999	24.9946546931794\\
1403.1	25\\
1403.6	24.9910949633113\\
1403.7	24.9964382390654\\
1403.99999999999	24.991097500178\\
1404.2	25.0017802463861\\
1404.40000000001	24.99822000712\\
1404.49999999999	25.0035597323081\\
1404.59999999999	25.0017797394461\\
1404.7	25.0071184510251\\
1405.29999999999	24.9964422940088\\
1405.4	25.0017787264319\\
1405.6	24.9982215266415\\
1405.7	25.0035566936975\\
1406.30000000001	24.9928896473265\\
1406.4	24.9982225382154\\
1406.60000000001	24.9946683727874\\
1406.70000000001	25\\
1406.80000000001	24.998223043571\\
1407.1	25.0142126208073\\
1407.60000000001	25.0053278397386\\
1407.8	25.0159812486682\\
1408.60000000001	25.0017746858806\\
1408.70000000001	25.0070982396366\\
1408.99999999999	25.0017741821021\\
1409.1	25.0070962248084\\
1409.30000000001	25.0035476089116\\
1409.4	25.0088683930472\\
1409.5	25.0070942111237\\
1409.6	25.0124139887919\\
1409.70000000001	25.0106398070648\\
1409.80000000001	25.0159585786226\\
1410	25.0124104673427\\
1410.10000000001	25.0177279818465\\
1410.2	25.0159540523293\\
1410.30000000001	25.0212705615428\\
1410.50000000001	25.017722954771\\
1410.6	25.0230382079819\\
1411.3	25.0106277455009\\
1411.40000000001	25.0159404888417\\
1411.69999999999	25.0106247343816\\
1412.09999999999	25.0318651749044\\
1412.6	25.0230055921285\\
1412.70000000001	25.0283125707814\\
1412.8	25.0265411564867\\
1412.90000000001	25.031847133758\\
1413.20000000001	25.0265336446614\\
1413.3	25.0318381208434\\
1413.8	25.0229860669071\\
1413.9	25.02828854314\\
1414.4	25.0194414987628\\
1414.60000000001	25.0300417049551\\
1414.90000000001	25.0247349823322\\
1415	25.0300332132005\\
1415.09999999999	25.0282645562465\\
1415.20000000001	25.03356178902\\
1415.50000000001	25.0282565696524\\
1415.6	25.0335523062796\\
1415.79999999999	25.030016244085\\
1416.09999999999	25.0458974721085\\
1416.3	25.0423609149958\\
1416.39999999999	25.0476526650194\\
1416.5	25.0458845122123\\
1416.6	25.0511752664643\\
1417.70000000001	25.0317393144308\\
1417.8	25.037026588617\\
1419.2	25.0123300218418\\
1419.29999999999	25.0176130759476\\
1419.50000000001	25.0140884756269\\
1419.6	25.0193702894978\\
1419.70000000001	25.0176081138188\\
1419.8	25.0228889358406\\
1420.40000000001	25.0123196057726\\
1420.5	25.0175981979445\\
1420.60000000001	25.0158372633209\\
1420.8	25.0263917235555\\
1421.59999999999	25.0123092072871\\
1421.7	25.0175833450556\\
1421.99999999999	25.012305745025\\
1422.1	25.0175783996625\\
1422.19999999999	25.015819447374\\
1422.30000000001	25.0210911136108\\
1423	25.0087836413464\\
1423.09999999999	25.0140528386734\\
1423.20000000001	25.012295369915\\
1423.29999999999	25.0175635801602\\
1424.9	24.9894736842105\\
1425.09999999999	25\\
1426.19999999999	24.9807193437566\\
1426.3	24.9859786876052\\
1426.39999999999	24.9842271293375\\
1426.60000000001	24.9947431134787\\
1426.8	24.9912397505081\\
1427	25.0017518043585\\
1427.1	25\\
1427.20000000001	25.0052546766622\\
1427.7	24.9964981089788\\
1427.79999999999	25.0017508228868\\
1428.00000000001	24.9982494223094\\
1428.10000000001	25.0035009102367\\
1428.19999999999	25.0017503325632\\
1428.3	25.0070008401008\\
1428.4	25.0052502625131\\
1428.5	25.0104997900042\\
1428.79999999999	25.0052487927777\\
1429.09999999999	25.0209907640638\\
1429.2	25.0192401875044\\
1429.29999999999	25.024485798237\\
1429.9	25.013986013986\\
1430.10000000001	25.024472101804\\
1430.8	25.0122300649941\\
1430.89999999999	25.0174703004892\\
1431.30000000001	25.0104792510829\\
1431.5	25.0209555741827\\
1432.10000000001	25.0104733975702\\
1432.20000000001	25.0157089995113\\
1433.00000000001	25.00174447003\\
1433.1	25.0069773932459\\
1433.49999999999	25\\
1433.6	25.0052312199205\\
1433.69999999999	25.0034872367136\\
1433.8	25.0087174837855\\
1435.19999999999	24.9843238347384\\
1435.40000000001	24.9947753396029\\
1435.7	24.9895528625157\\
1435.9	25\\
1436.29999999999	24.9930381509329\\
1436.39999999999	24.9982596588931\\
1436.59999999999	24.9947797034872\\
1436.7	25\\
1436.9	24.9965205288796\\
1437	25.0017396145014\\
1438.10000000001	24.9826171603393\\
1438.20000000001	24.9878328582354\\
1438.80000000001	24.9774133018278\\
1438.89999999999	24.9826268241835\\
1439.6	24.970479961103\\
1439.7	24.9756910682039\\
1439.8	24.9739565247587\\
1439.90000000001	24.9791666666667\\
1440.5	24.9687630154102\\
1440.7	24.9791782343143\\
1440.80000000001	24.9774446526477\\
1440.89999999999	24.9826509368494\\
1441	24.9809173547984\\
1441.1	24.9861226755482\\
1441.19999999999	24.9843890931798\\
1441.3	24.9895934508117\\
1441.49999999999	24.9861265260821\\
1441.8	25.0017338234274\\
1442.1	24.9965330744696\\
1442.5	25.0173298211563\\
1443.59999999999	24.9982683382974\\
1443.79999999999	25.0086571092181\\
1444.30000000001	25\\
1444.39999999999	25.0051921079958\\
1444.50000000001	25.0034611657206\\
1444.59999999999	25.0086523153596\\
1445.40000000001	24.9948114839156\\
1445.7	25.0103748789597\\
1445.90000000001	25.0069156293223\\
1445.99999999999	25.0121015144181\\
1446.2	25.0086427435525\\
1446.3	25.0138274336283\\
1446.5	25.0103691414351\\
1446.60000000001	25.015552637036\\
1446.70000000001	25.0138236107271\\
1446.8	25.0190061510816\\
1447.19999999999	25.0120914806882\\
1447.30000000001	25.0172723504214\\
1447.8	25.0086331929001\\
1447.99999999999	25.0189904012154\\
1448.30000000001	25.0138083402375\\
1448.50000000001	25.0241612591468\\
1448.6	25.0224339062608\\
1448.8	25.0327834909242\\
1449.10000000001	25.0276014352746\\
1449.19999999999	25.0327744428345\\
1449.39999999999	25.0293204553294\\
1449.5	25.0344922737307\\
1449.60000000001	25.0327653997379\\
1449.70000000001	25.0379362670713\\
1450.3	25.0275785990072\\
1450.4	25.0327473285074\\
1450.49999999999	25.0310216462154\\
1450.59999999999	25.0361894257944\\
1451.10000000001	25.0275633958104\\
1451.3	25.0378944467411\\
1451.89999999999	25.0275482093664\\
1452	25.032711245782\\
1452.3	25.0275406224181\\
1452.4	25.0327022375215\\
1452.7	25.0275330396476\\
1452.80000000001	25.0326932342212\\
1453.7	25.0171963131105\\
1453.90000000001	25.0275103163686\\
1454	25.0257891479266\\
1454.09999999999	25.0309448494017\\
1454.50000000001	25.0240615976901\\
1454.9	25.0446735395189\\
1455.19999999999	25.0395107537965\\
1455.4	25.0498110614909\\
1455.9	25.0412087912088\\
1456.1	25.0515039142975\\
1456.9	25.0377487989019\\
1457.19999999999	25.0531805393536\\
1457.69999999999	25.0445877349431\\
1457.80000000001	25.04972906235\\
1458	25.0462931211851\\
1458.1	25.0514332738993\\
1459.10000000001	25.0342653508772\\
1459.3	25.0445388515828\\
1459.6	25.0393916558197\\
1459.69999999999	25.044526647486\\
1460	25.0393808643244\\
1460.10000000001	25.0445144500753\\
1460.2	25.0427994247757\\
1460.3	25.0479320734045\\
1460.4	25.0462170489558\\
1460.50000000001	25.0513487607832\\
1460.7	25.0479189485214\\
1460.8	25.0530494900404\\
1461.29999999999	25.0444778979061\\
1461.4	25.0496065685939\\
1461.99999999999	25.0393269954176\\
1462.09999999999	25.0444535631241\\
1462.8	25.0324697518627\\
1462.90000000001	25.0375939849624\\
1463	25.0358827147837\\
1463.1	25.0410060142154\\
1463.3	25.0375837091704\\
1463.39999999999	25.0427058421592\\
1463.69999999999	25.0375734389944\\
1463.99999999999	25.0529335427908\\
1464.1	25.051222510586\\
1464.2	25.0563409137472\\
1465.09999999999	25.04095004095\\
1465.19999999999	25.0460656520849\\
1466.2	25.028984518857\\
1466.39999999999	25.0392090010228\\
1466.89999999999	25.0306748466258\\
1467	25.0357848817395\\
1467.19999999999	25.0323723846521\\
1467.40000000001	25.0425894378194\\
1468.19999999999	25.0289450384799\\
1468.29999999999	25.0340506673931\\
1470.2	25.0017003332653\\
1470.29999999999	25.0068008705114\\
1471.3	24.9898056272937\\
1471.4	24.9949031600408\\
1471.69999999999	24.9898083978801\\
1471.8	24.9949045451457\\
1472	24.9915087290266\\
1472.10000000001	24.9966037223203\\
1472.29999999999	24.9932083672915\\
1472.4	24.9983022071307\\
1472.7	24.9932102118414\\
1472.79999999999	24.9983026682056\\
1473.1	24.9932120553896\\
1473.29999999999	25.0033935116058\\
1473.39999999999	25.0016966406515\\
1473.5	25.006786102063\\
1473.6	25.0050892311868\\
1473.7	25.0101777717465\\
1473.9	25.0067842605156\\
1474	25.0118716504986\\
1474.29999999999	25.0067824199674\\
1474.5	25.0169537501695\\
1474.8	25.0118652112008\\
1474.90000000001	25.0169491525424\\
1475.2	25.0254185589372\\
1475.3	25.0237223803714\\
1475.49999999999	25.0338845215506\\
1476.1	25.0237095244547\\
1476.5	25.0440200460517\\
1476.8	25.0389328999932\\
1476.90000000001	25.0440081245768\\
1477.1	25.0406173842405\\
1477.20000000001	25.0456914641576\\
1477.4	25.0490693739425\\
1477.59999999999	25.0456790958923\\
1478.09999999999	25.0710323366256\\
1478.29999999999	25.0676406926407\\
1478.39999999999	25.0727088265134\\
1478.69999999999	25.0676223965377\\
1479	25.0828206341694\\
1479.3	25.0777342165743\\
1479.4	25.0827982426495\\
1479.60000000001	25.0794079881057\\
1479.70000000001	25.0844708744425\\
1479.90000000001	25.0878378378378\\
1480.10000000001	25.0844480475611\\
1480.60000000001	25.1097453906936\\
1481.6	25.0927988121752\\
1481.7	25.0978539613983\\
1482	25.0927737669523\\
1482.09999999999	25.0978275536365\\
1482.29999999999	25.0944414463033\\
1482.4	25.0994940978078\\
1482.79999999999	25.0927237170409\\
1482.90000000001	25.0977747808496\\
1483.29999999999	25.0910071457463\\
1483.40000000001	25.0960566228514\\
1483.70000000001	25.1044615177248\\
1483.90000000001	25.1078167115903\\
1484.1	25.1044333647756\\
1484.2	25.109479215792\\
1484.79999999999	25.0993332884369\\
1484.99999999999	25.1094202410612\\
1485.19999999999	25.1060391840032\\
1485.3	25.1110811902518\\
1485.6	25.1060106347176\\
1485.7	25.1110512855028\\
1485.9	25.114401076716\\
1486.2	25.122788131602\\
1486.40000000001	25.1194080053818\\
1486.7	25.1345170836696\\
1486.90000000001	25.1311365164761\\
1487	25.1361710712124\\
1487.3	25.1311012505042\\
1487.39999999999	25.1361344537815\\
1487.6	25.1394770451032\\
1488.19999999999	25.1293422025129\\
1488.30000000001	25.134372480516\\
1488.60000000001	25.1293074494525\\
1488.69999999999	25.1343363782912\\
1488.89999999999	25.137676292814\\
1489.2	25.1326126368092\\
1489.29999999999	25.1376393178461\\
1489.50000000001	25.140977443609\\
1489.8	25.1359151620914\\
1489.9	25.1409395973154\\
1490.2	25.1492987988996\\
1490.4	25.1526333445153\\
1490.60000000001	25.1492587375059\\
1490.69999999999	25.1542795814328\\
1491.2	25.1458459062563\\
1491.30000000001	25.1508649590988\\
1491.99999999999	25.1390657462637\\
1492.2	25.1490987066944\\
1492.60000000001	25.1423594828164\\
1492.7	25.1473740621651\\
1493.4	25.1355875460328\\
1493.49999999999	25.1405998928763\\
1494.00000000001	25.1321866006291\\
1494.1	25.1371971623611\\
1494.3	25.1405246252677\\
1494.6	25.1354786913762\\
1494.7	25.1404870216751\\
1495.10000000001	25.1337613697164\\
1495.59999999999	25.1587885271111\\
1495.8	25.1554248278628\\
1495.89999999999	25.1604278074866\\
1496.2	25.1553832787543\\
1496.3	25.1603849238172\\
1496.60000000001	25.1553417518541\\
1496.69999999999	25.1603420630679\\
1497.1	25.1536200908362\\
1497.2	25.1586188472584\\
1497.39999999999	25.1619365609349\\
1498.00000000001	25.185234630532\\
1499.1	25.1667556029883\\
1499.2	25.1717468151804\\
1499.80000000001	25.161677445163\\
1500.10000000001	25.1766431142514\\
1500.30000000001	25.1799520127966\\
1500.49999999999	25.1832600293216\\
1500.7	25.1799040511727\\
1500.9	25.1898734177215\\
1501.79999999999	25.1747786137559\\
1501.9	25.1797603195739\\
1502.3	25.1730564430245\\
1502.5	25.1830161054173\\
1502.8	25.1912968261361\\
1503.8	25.1745461799322\\
1504	25.1844957117213\\
1504.2	25.1877949877019\\
1504.39999999999	25.1844466600199\\
1504.59999999999	25.1943909084867\\
1505.09999999999	25.1860217911241\\
1505.20000000001	25.1909918288713\\
1505.4	25.194287612089\\
1505.7	25.1892681631027\\
1506	25.2041697098466\\
1506.79999999999	25.190789037096\\
1507.10000000001	25.2056794055202\\
1507.3	25.2023351466101\\
1507.40000000001	25.2072968490879\\
1507.79999999999	25.2006101200345\\
1508.00000000001	25.2105298057158\\
1508.2	25.2138168799311\\
1509.60000000001	25.1904351857985\\
1509.70000000001	25.1953901178964\\
1509.9	25.1920529801325\\
1510.00000000001	25.1970068207403\\
1510.3	25.1920021186441\\
1510.39999999999	25.1969546507779\\
1510.59999999999	25.1936188521877\\
1510.70000000001	25.198570293884\\
1511.59999999999	25.1835681682874\\
1511.69999999999	25.1885169996031\\
1511.89999999999	25.1917989417989\\
1512.09999999999	25.188467133977\\
1512.20000000001	25.1934140051577\\
1512.40000000001	25.1900826446281\\
1512.60000000001	25.1999735572156\\
1513.2	25.1899821581973\\
1513.3	25.1949253336857\\
1514.50000000001	25.174963686782\\
1514.7	25.184842883549\\
1514.89999999999	25.1881188118812\\
1515.5	25.1781472684086\\
1515.6	25.183083723692\\
1517.00000000001	25.1598444400501\\
1517.10000000001	25.1647772211969\\
1517.3	25.1680506128905\\
1517.50000000001	25.164733790195\\
1517.59999999999	25.1696646241023\\
1517.80000000001	25.1663482442849\\
1517.90000000001	25.171277997365\\
1518.29999999999	25.1646469968388\\
1518.40000000001	25.1695752387224\\
1519.8	25.146391209948\\
1519.89999999999	25.1513157894737\\
1520.59999999999	25.1397382784244\\
1520.70000000001	25.1446607048922\\
1521.5	25.1314405888538\\
1521.6	25.1363606492738\\
1521.99999999999	25.1297549438276\\
1522.30000000001	25.1445086705202\\
1522.49999999999	25.1412058321293\\
1522.60000000001	25.1461220200959\\
1523.30000000001	25.1345674149928\\
1523.50000000001	25.1443948542925\\
1523.8	25.1394448454623\\
1523.89999999999	25.1443569553806\\
1524.09999999999	25.1476184227792\\
1524.6	25.1393716796747\\
1524.7	25.1442812172088\\
1525.7	25.1278018088871\\
1525.79999999999	25.1327085654368\\
1525.99999999999	25.1359674988533\\
1527.3	25.1145737855179\\
1527.4	25.1194762684124\\
1527.6	25.1227335209793\\
1527.89999999999	25.1178010471204\\
1528.00000000001	25.1227013938878\\
1528.50000000001	25.1144838414235\\
1528.59999999999	25.1193824818473\\
1528.89999999999	25.1144538914323\\
1529.1	25.1242479727962\\
1529.40000000001	25.1323962079111\\
1529.6	25.129110283062\\
1529.69999999999	25.1340044450255\\
1529.9	25.1372549019608\\
1530.4	25.1290427964717\\
1530.50000000001	25.1339344048086\\
1531.8	25.11260526144\\
1531.9	25.1174934725849\\
1532.1	25.1142148544577\\
1532.2	25.1191020035241\\
1532.50000000001	25.1141850450215\\
1532.6	25.1190709205976\\
1532.9	25.1141552511416\\
1533	25.1190398538908\\
1533.19999999999	25.122285267071\\
1533.60000000001	25.1157331942362\\
1533.7	25.1206154648585\\
1534.20000000001	25.1124291207717\\
1534.29999999999	25.1173096976017\\
1534.6	25.1254316804587\\
1534.8	25.1286728777119\\
1535.09999999999	25.1237623762376\\
1535.2	25.1286393538722\\
1535.50000000001	25.1367543631154\\
1535.8	25.1318445211277\\
1535.9	25.13671875\\
1536.1	25.1334461658638\\
1536.2	25.1383193386708\\
1536.5	25.1334114278277\\
1536.6	25.1382833344179\\
1536.99999999999	25.1317415913083\\
1537.1	25.1366120218579\\
1537.29999999999	25.1333420059841\\
1537.50000000001	25.1430801248699\\
1537.89999999999	25.1560468140442\\
1538.30000000001	25.1690067602704\\
1538.70000000001	25.1624642578633\\
1538.80000000001	25.1673273117162\\
1539.19999999999	25.1607873708829\\
1539.40000000001	25.1705099058136\\
1539.69999999999	25.1656059228471\\
1539.99999999999	25.1801831049932\\
1540.19999999999	25.176913588262\\
1540.29999999999	25.1817709685796\\
1540.80000000001	25.1735998442469\\
1540.89999999999	25.1784555483452\\
1541.50000000001	25.1686559418786\\
1541.60000000001	25.1735097619511\\
1541.8	25.1767300084312\\
1542.5	25.1653053286659\\
1542.80000000001	25.1798561151079\\
1543	25.1765925733912\\
1543.1	25.1814411612234\\
1543.3	25.1846572502268\\
1544.50000000001	25.1650912857698\\
1544.6	25.1699359098854\\
1545	25.1634198433758\\
1545.10000000001	25.1682630080249\\
1545.50000000001	25.1617494824017\\
1545.6	25.1665911884583\\
1545.80000000001	25.1633352739505\\
1545.90000000001	25.1681759379043\\
1546.3	25.161665804449\\
1546.6	25.1761815478115\\
1547.30000000001	25.164792555254\\
1547.4	25.1696284329564\\
1548.00000000001	25.1598733931917\\
1548.09999999999	25.1647074021444\\
1548.39999999999	25.1598320955764\\
1548.50000000001	25.164664858582\\
1548.69999999999	25.1678719008264\\
1548.9	25.1646223369916\\
1549	25.1694532309083\\
1549.29999999999	25.1645798373564\\
1549.4	25.1694094869313\\
1549.9	25.1612903225806\\
1550.10000000001	25.1709456844278\\
1551.2	25.1530974021788\\
1551.29999999999	25.1579218770143\\
1551.5	25.1546790409899\\
1551.59999999999	25.1595024811497\\
1552.1	25.1513980157196\\
1552.30000000001	25.1610409688225\\
1553.50000000001	25.1416065911432\\
1553.60000000001	25.146424663706\\
1554.5	25.1318667181269\\
1554.6	25.1366823181321\\
1555.10000000001	25.1286008230453\\
1555.2	25.1334147752845\\
1555.4	25.1301832208293\\
1555.49999999999	25.1349961429673\\
1556.49999999999	25.1188487729667\\
1556.79999999999	25.1332776671591\\
1557	25.1364716460086\\
1557.20000000001	25.1396648044693\\
1558.39999999999	25.1203079884504\\
1558.50000000001	25.1251122802515\\
1559.60000000001	25.1073924472655\\
1559.7	25.1121938710091\\
1559.90000000001	25.1153846153846\\
1560.09999999999	25.1121651070376\\
1560.3	25.1217636503461\\
1560.49999999999	25.1249519415609\\
1560.79999999999	25.1201230059581\\
1560.90000000001	25.1249199231262\\
1561.5	25.1152663934426\\
1561.59999999999	25.1200614714734\\
1561.8	25.1168448684295\\
1561.90000000001	25.1216389244558\\
1562.09999999999	25.1248239662015\\
1562.29999999999	25.1216077828981\\
1562.4	25.1264\\
1563.29999999999	25.1119355251375\\
1563.40000000001	25.1167252958107\\
1564.10000000001	25.1054852320675\\
1564.3	25.1150600869343\\
1564.60000000001	25.1102447753563\\
1564.7	25.1150306748466\\
1565.30000000001	25.1054043694902\\
1565.39999999999	25.1101884381987\\
1565.59999999999	25.1133678226991\\
1565.79999999999	25.1101602912063\\
1565.99999999999	25.1197241555456\\
1566.59999999999	25.1101040403396\\
1566.70000000001	25.1148838396732\\
1566.90000000001	25.111678366305\\
1567.00000000001	25.11645715015\\
1567.2	25.1132520895808\\
1567.30000000001	25.1180298583642\\
1568.2	25.1036153797105\\
1568.30000000001	25.1083907166539\\
1568.5	25.1115644523779\\
1568.70000000001	25.1083630800612\\
1568.8	25.1131365925171\\
1569.4	25.1035361580121\\
1569.50000000001	25.1083078491335\\
1569.9	25.1019108280255\\
1570.1	25.1114507706025\\
1570.8	25.1002609968808\\
1570.89999999999	25.1050286441757\\
1571.10000000001	25.10183299389\\
1571.29999999999	25.1113656611938\\
1572.2	25.0969916682567\\
1572.4	25.1065182829889\\
1572.6	25.1033254911935\\
1572.7	25.1080874872838\\
1572.99999999999	25.1032992181044\\
1573.10000000001	25.1080600050852\\
1573.39999999999	25.1032729583731\\
1573.59999999999	25.1127915104531\\
1574.6	25.0968438432717\\
1574.70000000001	25.1016002032004\\
1575.20000000001	25.0936329588015\\
1575.30000000001	25.0983877110575\\
1577.09999999999	25.0697438498605\\
1577.19999999999	25.074494389146\\
1578.30000000001	25.0570197668525\\
1578.4	25.0617675007919\\
1578.6	25.058592512827\\
1578.7	25.0633392449962\\
1579.70000000001	25.0474743638435\\
1579.90000000001	25.0569620253165\\
1580.30000000001	25.0506200961782\\
1580.70000000001	25.0695850202429\\
1581.09999999999	25.0632431065014\\
1581.3	25.0727203743518\\
1581.59999999999	25.0679648479484\\
1581.69999999999	25.0727019850803\\
1582.40000000001	25.0616113744076\\
1582.6	25.0710810640045\\
1583.40000000001	25.0584149036943\\
1583.5	25.0631472594089\\
1583.70000000001	25.0599823210001\\
1583.8	25.0647136814193\\
1584.19999999999	25.0583854068043\\
1584.29999999999	25.0631153749053\\
1584.50000000001	25.0599520383693\\
1584.70000000001	25.0694093891974\\
1585.2	25.0615025547215\\
1585.29999999999	25.0662293427526\\
1586.19999999999	25.0520078169325\\
1586.3	25.0567322239032\\
1586.49999999999	25.0535736795664\\
1586.60000000001	25.0582970945989\\
1587.1	25.0504032258064\\
1587.2	25.0551250551251\\
1587.4	25.0582677165354\\
1588.1	25.0472232716283\\
1588.20000000001	25.0519423282755\\
1588.40000000001	25.0487881649355\\
1588.6	25.058223704916\\
1589.2	25.0487636066193\\
1589.30000000001	25.0534793003649\\
1589.59999999999	25.0613323268541\\
1590.19999999999	25.0518770043388\\
1590.30000000001	25.0565895372233\\
1591.00000000001	25.0455659606562\\
1591.2	25.0549864890341\\
1591.39999999999	25.0581212692429\\
1592	25.0486778468689\\
1592.4	25.0675039246468\\
1593.70000000001	25.0470573472205\\
1593.89999999999	25.0564617314931\\
1594.10000000001	25.0533182787605\\
1594.2	25.0580191933764\\
1594.69999999999	25.050163029847\\
1594.8	25.0548623738165\\
1595.2	25.0673854447439\\
1595.59999999999	25.0611017108479\\
1595.70000000001	25.0657977190124\\
1596.7	25.0501002004008\\
1596.8	25.0547936627215\\
1597.00000000001	25.0579174754242\\
1597.19999999999	25.0547799411507\\
1597.30000000001	25.0594716414173\\
1597.59999999999	25.0547662264505\\
1597.70000000001	25.0594567530354\\
1598.4	25.0484829527682\\
1598.49999999999	25.0531715250845\\
1598.69999999999	25.0500375281461\\
1598.80000000001	25.0547251235224\\
1599.1	25.0500250125063\\
1599.5	25.0687671917979\\
1600.3	25.0562359410147\\
1600.5	25.0656003998501\\
1600.70000000001	25.0624687656172\\
1600.79999999999	25.0671497282778\\
1601.30000000001	25.0593230922942\\
1601.4	25.0640024976584\\
1601.79999999999	25.0577439290842\\
1601.9	25.0624219725343\\
1602.1	25.0655348895269\\
1602.4	25.0608424336974\\
1602.50000000001	25.0655185323849\\
1602.70000000001	25.0623908160719\\
1602.80000000001	25.0670659429784\\
1603.00000000001	25.0701765329674\\
1603.20000000001	25.0670492110023\\
1603.3	25.0717225894973\\
1603.60000000001	25.0670324873729\\
1603.79999999999	25.0763763326891\\
1604.19999999999	25.0888237860749\\
1606.1	25.0591458099863\\
1606.2	25.0638112432298\\
1606.4	25.066915655151\\
1606.90000000001	25.0591163658992\\
1607	25.0637794785639\\
1607.20000000001	25.0668823492814\\
1608.1	25.0528541226216\\
1608.4	25.0668324525956\\
1608.60000000001	25.0699322434264\\
1608.89999999999	25.077688004972\\
1609.1	25.0807854834701\\
1609.90000000001	25.0683229813665\\
1610.1	25.0776301080611\\
1610.4	25.085377212046\\
1611.20000000001	25.07292248495\\
1611.3	25.0775722973812\\
1612.00000000001	25.0666832082377\\
1612.40000000001	25.0852713178295\\
1612.8	25.0790501581003\\
1612.89999999999	25.0836949783013\\
1613.5	25.0743678730788\\
1613.6	25.0790109685815\\
1613.80000000001	25.0759030918892\\
1613.9	25.0805452292441\\
1614.30000000001	25.0743310208127\\
1614.40000000001	25.0789718179003\\
1614.59999999999	25.0758654858488\\
1614.80000000001	25.0851445909963\\
1615.2	25.0789327059989\\
1615.3	25.0835706326606\\
1615.49999999999	25.080465461748\\
1615.6	25.0851024323823\\
1615.79999999999	25.0819976483693\\
1615.90000000001	25.0866336633663\\
1616.2	25.0819773556889\\
1616.30000000001	25.0866122246969\\
1616.90000000001	25.0773036487322\\
1616.99999999999	25.0819368004452\\
1617.2	25.0788350955296\\
1617.5	25.0927299703264\\
1618.30000000001	25.0803262481463\\
1618.39999999999	25.0849552054371\\
1618.8	25.0787571808018\\
1619	25.0880118584399\\
1619.29999999999	25.0833642089663\\
1619.40000000001	25.0879901204075\\
1620	25.0786988457503\\
1620.10000000001	25.0833230465375\\
1620.5	25.077131926447\\
1620.6	25.0817547973098\\
1621.30000000001	25.0709263599359\\
1621.5	25.0801677355698\\
1621.90000000001	25.0739827373613\\
1622	25.0786018124653\\
1622.20000000001	25.0755100782839\\
1622.29999999999	25.0801282051282\\
1623.2	25.0662231257315\\
1623.29999999999	25.0708389799187\\
1623.6	25.0662068116031\\
1623.69999999999	25.070821529745\\
1624.09999999999	25.0646472109346\\
1624.20000000001	25.0692606045681\\
1625.20000000001	25.0538362148526\\
1625.30000000001	25.0584471514704\\
1625.50000000001	25.0615157480315\\
1625.89999999999	25.0553505535055\\
1625.99999999999	25.0599594120903\\
1627.1	25.0430186823992\\
1627.20000000001	25.0476249001413\\
1627.5	25.0430081101008\\
1627.60000000001	25.047613196535\\
1627.8	25.0445359051539\\
1627.90000000001	25.04914004914\\
1628.20000000001	25.0445249646871\\
1628.3	25.0491279783837\\
1628.49999999999	25.0521920668058\\
1628.90000000001	25.0460405156538\\
1628.99999999999	25.050641458474\\
1629.5	25.0429553264605\\
1629.59999999999	25.0475547646806\\
1630	25.0414085025459\\
1630.10000000001	25.046006624954\\
1630.29999999999	25.042934249264\\
1630.40000000001	25.047531432076\\
1630.9	25.0398528510117\\
1631	25.0444485316658\\
1631.4	25.0383083052406\\
1631.6	25.0474964760679\\
1631.80000000001	25.0505545682946\\
1632.1	25.0582036515133\\
1632.59999999999	25.0750290929136\\
1632.9	25.0826699326393\\
1633.30000000001	25.0765274886739\\
1633.40000000001	25.0811141720233\\
1633.7	25.0887501530175\\
1633.9	25.0917992656059\\
1634.40000000001	25.0841235851943\\
1634.70000000001	25.0978712992415\\
1635.80000000001	25.080995170854\\
1635.89999999999	25.0855745721271\\
1636.40000000001	25.0779101741522\\
1636.5	25.0824880850544\\
1636.7	25.0794232649071\\
1636.80000000001	25.0840002443643\\
1636.99999999999	25.0809358011117\\
1637.09999999999	25.0855118494991\\
1637.30000000001	25.0885550262611\\
1637.49999999999	25.0915974596971\\
1637.7	25.0885333984614\\
1637.80000000001	25.093107027291\\
1638.1	25.1007203027713\\
1638.29999999999	25.103759765625\\
1638.69999999999	25.0976324139614\\
1638.8	25.1022026969309\\
1642.10000000001	25.0517598343685\\
1642.20000000001	25.0563234488218\\
1642.59999999999	25.0502221951665\\
1642.7	25.054784514244\\
1643.4	25.0441131731062\\
1643.50000000001	25.0486736432222\\
1643.9	25.0425790754258\\
1644.00000000001	25.0471382519312\\
1644.5	25.0395232883376\\
1644.6	25.0440809874141\\
1645	25.0379916114522\\
1645.10000000001	25.042548018478\\
1645.3	25.0455816214902\\
1645.5	25.0425376762275\\
1645.60000000001	25.0470924226773\\
1645.90000000001	25.0425273390036\\
1646	25.0470809792844\\
1646.2	25.0440381461459\\
1646.30000000001	25.0485908649174\\
1646.8	25.0409860950877\\
1646.89999999999	25.0455373406193\\
1647.2	25.053117222121\\
1647.49999999999	25.0606943432872\\
1647.7	25.0576526277461\\
1647.89999999999	25.0667475728155\\
1648.2	25.0743189953285\\
1648.59999999999	25.0682355795475\\
1648.70000000001	25.0727802037846\\
1649.39999999999	25.0621400424371\\
1649.50000000001	25.0666828322017\\
1649.69999999999	25.0697054188386\\
1649.90000000001	25.0727272727273\\
1650.4	25.065131778249\\
1650.69999999999	25.0787496971165\\
1651	25.0741929622676\\
1651.09999999999	25.078730620155\\
1651.39999999999	25.0741749924311\\
1651.49999999999	25.078711552434\\
1652.40000000001	25.0650529500756\\
1652.49999999999	25.0695873169551\\
1653.60000000001	25.0529116526577\\
1653.8	25.0619747264043\\
1654.00000000001	25.0589444410858\\
1654.1	25.06347479144\\
1654.30000000001	25.0604448742747\\
1654.39999999999	25.0649743124811\\
1654.7	25.060430263476\\
1654.80000000001	25.0649586077709\\
1655.19999999999	25.0589017096599\\
1655.30000000001	25.0634287785429\\
1655.5	25.0604010630587\\
1655.8	25.073977897216\\
1656.10000000001	25.069436058447\\
1656.2	25.0739600313953\\
1656.7	25.0663930468373\\
1656.80000000001	25.0709155652121\\
1657.20000000001	25.0648645387075\\
1657.3	25.0693857849644\\
1657.49999999999	25.0723938223938\\
1658.7	25.0542560887388\\
1658.79999999999	25.0587738863102\\
1658.99999999999	25.0617804833946\\
1659.49999999999	25.0542299349241\\
1659.6	25.0587455564259\\
1659.9	25.0542168674699\\
1660.00000000001	25.0587314017228\\
1660.40000000001	25.0526949713942\\
1660.6	25.0617209610405\\
1660.9	25.057194461168\\
1661	25.0617060983686\\
1661.80000000001	25.0496419760515\\
1661.9	25.0541516245487\\
1663.30000000001	25.0330648070218\\
1663.40000000001	25.0375713856327\\
1663.6	25.0345615195047\\
1663.7	25.0390671955764\\
1664.19999999999	25.0315447936069\\
1664.29999999999	25.0360490266763\\
1664.69999999999	25.0480538202787\\
1665.6	25.0345200216125\\
1665.69999999999	25.0390202905511\\
1665.90000000001	25.0360144057623\\
1666.00000000001	25.0405137746834\\
1666.19999999999	25.0375082518154\\
1666.30000000001	25.0420067210754\\
1666.5	25.045001800072\\
1666.80000000001	25.0404943307937\\
1666.90000000001	25.0449910017996\\
1667.29999999999	25.0389828475471\\
1667.40000000001	25.0434782608696\\
1668.4	25.0284686844471\\
1668.60000000001	25.037454305747\\
1669.10000000001	25.0299544692068\\
1669.2	25.0344455759899\\
1669.40000000001	25.0314465408805\\
1669.5	25.0359367513177\\
1669.69999999999	25.0329380764163\\
1669.8	25.0374273908617\\
1670.00000000001	25.0404167415125\\
1670.30000000001	25.0478927203065\\
1670.60000000001	25.0553660142455\\
1670.80000000001	25.0583517864624\\
1671.00000000001	25.055352761654\\
1671.10000000001	25.0598372426999\\
1671.4	25.0673048160335\\
1671.6	25.0702877310522\\
1672.1	25.0627915321134\\
1672.20000000001	25.0672726185493\\
1672.7	25.0597800095648\\
1672.79999999999	25.0642596688385\\
1673	25.0672404518558\\
1673.2	25.0642443076555\\
1673.29999999999	25.0687223616589\\
1673.5	25.0657265774379\\
1673.60000000001	25.0702037402163\\
1673.79999999999	25.067208315909\\
1674	25.0761603249507\\
1675.20000000001	25.0581985316063\\
1675.29999999999	25.0626716008117\\
1676.59999999999	25.0432396970239\\
1676.7	25.0477099236641\\
1677.3	25.0387504471205\\
1677.39999999999	25.043219076006\\
1678	25.0342649424945\\
1678.10000000001	25.0387319747348\\
1678.3	25.0417063870353\\
1678.8	25.0342486151647\\
1678.9	25.0387135199524\\
1679.10000000001	25.0357313006193\\
1679.19999999999	25.0401953194784\\
1679.49999999999	25.0357227911407\\
1679.60000000001	25.0401857474549\\
1680.30000000001	25.0297548202809\\
1680.40000000001	25.0342160071407\\
1682.40000000001	25.0044576523031\\
1682.5	25.0089147747534\\
1682.7	25.0118849536487\\
1682.9	25.0148544266191\\
1683.19999999999	25.0103962454702\\
1683.30000000001	25.0148508969942\\
1683.7	25.0089084214277\\
1683.79999999999	25.013361838589\\
1684	25.0103913069295\\
1684.10000000001	25.014843842774\\
1684.30000000001	25.0118736642128\\
1684.39999999999	25.0163253190858\\
1685.09999999999	25.0059340137669\\
1685.39999999999	25.0192821121329\\
1685.60000000001	25.0163136975737\\
1685.69999999999	25.020761656187\\
1685.90000000001	25.017793594306\\
1686.20000000001	25.0311332503113\\
1686.6	25.0251971304915\\
1686.79999999999	25.0340861936096\\
1687.20000000001	25.0281514846204\\
1687.29999999999	25.03259452412\\
1687.60000000001	25.0281448124667\\
1687.70000000001	25.0325867993838\\
1688.19999999999	25.0251732511994\\
1688.3	25.029613835584\\
1688.49999999999	25.0266492952742\\
1688.89999999999	25.044404973357\\
1689.70000000001	25.0325482305598\\
1689.9	25.0414201183432\\
1690.5	25.0325328285816\\
1690.7	25.0414005204637\\
1691.1	25.0354777672658\\
1691.4	25.0487732781555\\
1692.1	25.0384115352795\\
1692.2	25.0428411038232\\
1692.5	25.0384024577573\\
1692.60000000001	25.042830980091\\
1693.10000000001	25.0354358610914\\
1693.19999999999	25.0398629894289\\
1693.59999999999	25.033949341678\\
1693.89999999999	25.047225501771\\
1694.70000000001	25.0354024073637\\
1694.8	25.0398253584282\\
1695.1	25.035394053799\\
1695.19999999999	25.0398159617767\\
1695.89999999999	25.0294811320755\\
1696	25.0339013029892\\
1696.30000000001	25.0294741806178\\
1696.50000000001	25.0383119179536\\
1696.7	25.0412541254125\\
1696.90000000001	25.0441956393636\\
1697.1	25.0412444025454\\
1697.20000000001	25.0456607553173\\
1698.40000000001	25.0279658522226\\
1698.60000000001	25.0367928415847\\
1698.89999999999	25.0441436138905\\
1699.3	25.0382487936919\\
1699.40000000001	25.0426596057664\\
1699.69999999999	25.0500058830451\\
1700.89999999999	25.0323339212228\\
1701	25.0367409323379\\
1701.50000000001	25.0293841090738\\
1701.70000000001	25.0381948525091\\
1703.00000000001	25.0190828489226\\
1703.1	25.0234852043213\\
1703.3	25.0205471410121\\
1703.39999999999	25.0249486351629\\
1703.6	25.0278804953924\\
1703.89999999999	25.0234741784038\\
1704.00000000001	25.0278739510592\\
1704.20000000001	25.0249369242504\\
1704.3	25.0293358366581\\
1704.49999999999	25.0322656341664\\
1704.69999999999	25.0351947442515\\
1705.1	25.0293220736571\\
1705.19999999999	25.0337184073184\\
1705.59999999999	25.0278478044205\\
1705.7	25.0322429358659\\
1705.89999999999	25.0293083235639\\
1706	25.0337025965653\\
1706.2	25.0366289632538\\
1706.39999999999	25.0395546440082\\
1707.49999999999	25.0234246896229\\
1707.6	25.0278151900217\\
1707.89999999999	25.0351288056206\\
1708.1	25.0321976349374\\
1708.20000000001	25.0365860797284\\
1708.60000000001	25.0307251126588\\
1708.8	25.0394990929838\\
1709.00000000001	25.0424199871277\\
1709.59999999999	25.0336316312803\\
1709.80000000001	25.0424001403591\\
1709.99999999999	25.0453189871937\\
1710.3	25.0409260991581\\
1710.40000000001	25.0453083893598\\
1710.60000000001	25.0423803121529\\
1710.69999999999	25.0467617488894\\
1711.00000000001	25.0423704050026\\
1711.1	25.0467508181393\\
1711.40000000001	25.0423605024832\\
1711.5	25.0467398924983\\
1712.59999999999	25.0306533543528\\
1712.69999999999	25.035030359645\\
1713.1	25.0291851505954\\
1713.30000000001	25.0379362670713\\
1713.9	25.0291715285881\\
1713.99999999999	25.0335453007409\\
1714.80000000001	25.0218671642661\\
1714.89999999999	25.0262390670554\\
1715.19999999999	25.0218620649449\\
1715.40000000001	25.0306033226465\\
1716.40000000001	25.01602097291\\
1716.50000000001	25.0203891413259\\
1716.70000000001	25.0174743709226\\
1716.79999999999	25.0218416914206\\
1717.1	25.0291171674819\\
1717.29999999999	25.0262023989752\\
1717.39999999999	25.0305676855895\\
1717.70000000001	25.0261962975899\\
1717.90000000001	25.034924330617\\
1718.3	25.0465549348231\\
1718.5	25.0436401722332\\
1718.60000000001	25.0480013964043\\
1718.79999999999	25.0509046483216\\
1719.5	25.0407071411956\\
1719.60000000001	25.0450659998837\\
1719.80000000001	25.0421536135822\\
1720.10000000001	25.0552261364958\\
1720.89999999999	25.0435793143521\\
1721	25.0479344605194\\
1721.60000000001	25.0392054364872\\
1721.7	25.0435590660936\\
1722.20000000001	25.0362886837369\\
1722.30000000001	25.0406409660938\\
1722.50000000001	25.0377336584233\\
1722.60000000001	25.0420850989725\\
1723.40000000001	25.0304612706701\\
1723.59999999999	25.0391599466264\\
1723.99999999999	25.0333507337161\\
1724.29999999999	25.0463929482719\\
1724.50000000001	25.04928679114\\
1725.09999999999	25.0405750057964\\
1725.19999999999	25.044919724106\\
1725.5	25.0405656003709\\
1725.6	25.0449093121632\\
1725.8	25.0420070687757\\
1725.9	25.0463499420626\\
1726.50000000001	25.0376462411676\\
1726.60000000001	25.0419876064169\\
1727.29999999999	25.0318397591756\\
1727.40000000001	25.0361794500724\\
1727.69999999999	25.0318323880079\\
1727.8	25.036171074715\\
1728.5	25.0260326275599\\
1728.6	25.0303696419275\\
1730.2	25.0072241807779\\
1730.50000000001	25.0202241996995\\
1730.69999999999	25.0231107002542\\
1730.9	25.0259965337955\\
1731.5	25.017325017325\\
1731.59999999999	25.0216550210776\\
1731.9	25.0173210161663\\
1732.2	25.0303065288922\\
1732.4	25.033189033189\\
1732.6	25.0302995325215\\
1733	25.0476025618833\\
1733.9	25.0346020761246\\
1733.99999999999	25.0389250908252\\
1735.39999999999	25.0187265917603\\
1735.50000000001	25.0230467849735\\
1735.69999999999	25.0201636133195\\
1735.80000000001	25.02448297713\\
1736.70000000001	25.0115154306771\\
1736.79999999999	25.0158328055731\\
1737.90000000001	25\\
1738	25.0043150566711\\
1738.20000000001	25.0071909336708\\
1738.4	25.0043140638481\\
1738.5	25.0086276314276\\
1739.20000000001	24.9985626401426\\
1739.29999999999	25.0028745544441\\
1740.2	24.9899442624835\\
1740.30000000001	24.9942541944381\\
1741.90000000001	24.9712973593571\\
1742	24.9756041559038\\
1742.19999999999	24.9727371864776\\
1742.3	24.9770431588613\\
1742.80000000001	24.9698777898904\\
1742.9	24.974182444062\\
1743.30000000001	24.9684524492371\\
1743.40000000001	24.9727559506739\\
1743.8	24.9670279259132\\
1743.90000000001	24.9713302752294\\
1744.10000000001	24.9684669189313\\
1744.2	24.9727684457949\\
1744.50000000001	24.968474148802\\
1744.59999999999	24.9727746890583\\
1745.20000000001	24.9641895376153\\
1745.29999999999	24.968488598602\\
1745.6	24.9641977430257\\
1745.69999999999	24.9684958185359\\
1745.90000000001	24.9656357388316\\
1745.99999999999	24.9699329935284\\
1746.5	24.962784839116\\
1746.79999999999	24.9756711889633\\
1747	24.9785358594242\\
1747.80000000001	24.9671033812003\\
1747.89999999999	24.9713958810069\\
1748.09999999999	24.9685390687564\\
1748.20000000001	24.9728307498713\\
1748.5	24.9799839871898\\
1748.70000000001	24.9828453796889\\
1749.2	24.9757045675413\\
1749.4	24.9842812232066\\
1749.80000000001	24.9785702040117\\
1749.89999999999	24.9828571428571\\
1750.1	24.9800022854531\\
1750.29999999999	24.9885740402194\\
1750.6	24.9842919974867\\
1751.00000000001	25.0014276740335\\
1751.20000000002	25.0042825329755\\
1751.5	25\\
1751.6	25.0042815550608\\
1752.09999999999	24.9971464444698\\
1752.19999999999	25.001426696342\\
1752.50000000001	25.0085587127696\\
1752.70000000001	25.0114103149247\\
1753	25.007130226456\\
1753.09999999999	25.0114077116131\\
1753.90000000001	25\\
1754.09999999999	25.0085509063961\\
1754.3	25.0056999544004\\
1754.49999999999	25.0142482617121\\
1754.70000000001	25.0113973102348\\
1754.80000000001	25.0156704085703\\
1757.4	24.9786628733997\\
1757.50000000001	24.9829312699135\\
1758.09999999999	24.974405642134\\
1758.30000000001	24.9829390354868\\
1758.50000000001	24.9857841464802\\
1759.6	24.970165369097\\
1759.69999999999	24.9744289123764\\
1759.89999999999	24.9715909090909\\
1759.99999999999	24.9758536446793\\
1760.19999999999	24.9730159631881\\
1760.4	24.9815393354161\\
1760.80000000001	24.9929013572605\\
1761.00000000001	24.9900630287888\\
1761.39999999999	25.0070962248084\\
1762.70000000001	24.9886544134332\\
1762.8	24.9929094106302\\
1763.59999999999	24.9815728298463\\
1763.69999999999	24.9858260573761\\
1763.9	24.9886621315193\\
1764.1	24.9914975626346\\
1764.49999999999	24.9858324832823\\
1764.60000000001	24.9900833002777\\
1764.9	24.985835694051\\
1765	24.990085547561\\
1765.2	24.9929190505863\\
1765.6	24.9872571784561\\
1765.70000000001	24.9915052667346\\
1766.2	24.9844307309064\\
1766.4	24.9929238607416\\
1766.7	25\\
1767.8	24.9844448215397\\
1767.99999999999	24.9929302641253\\
1768.49999999999	24.9858645256135\\
1768.6	24.9901057273704\\
1768.8	24.9929334614732\\
1770.1	24.9745791435996\\
1770.2	24.9788171496357\\
1770.49999999999	24.9745848864792\\
1770.59999999999	24.978821934828\\
1770.89999999999	24.9745906267645\\
1771.09999999999	24.9830623306233\\
1772	24.9703741323853\\
1772.10000000001	24.9746078320731\\
1772.49999999999	24.985896423333\\
1772.89999999999	24.9971799210378\\
1774	24.9816808522631\\
1774.29999999999	24.9943642921551\\
1774.59999999999	24.9901391784527\\
1774.8	24.998591469942\\
1775.4	24.9901436215151\\
1775.50000000001	24.9943681009236\\
1775.80000000001	24.9901458415451\\
1775.9	24.9943693693694\\
1776.3	24.9887412744877\\
1776.4	24.9929636926541\\
1776.80000000001	24.9873374978896\\
1776.90000000001	24.9915588069781\\
1777.3	24.9859345110836\\
1777.4	24.9901547116737\\
1777.70000000001	24.9859376757791\\
1778.00000000001	24.9985940048366\\
1778.19999999999	25.0014058370354\\
1779.70000000002	24.9803348690864\\
1779.89999999999	24.9887640449438\\
1781	24.9733310875302\\
1781.20000000001	24.9817548981081\\
1781.50000000001	24.9775482712169\\
1781.7	24.9859692445841\\
1781.9	24.983164983165\\
1781.99999999999	24.9873744458785\\
1782.2	24.9901812265051\\
1783.2	24.976167778837\\
1783.3	24.9803745654368\\
1784.40000000001	24.964976183805\\
1784.49999999999	24.9691807687997\\
1784.9	24.9635854341737\\
1785.1	24.9719919336769\\
1785.69999999999	24.9636017471161\\
1785.9	24.9720044792833\\
1786.19999999999	24.9678105581369\\
1786.40000000001	24.9762104673943\\
1786.59999999999	24.9790115856047\\
1786.80000000001	24.981812076781\\
1787.1	24.9776186213071\\
1787.20000000001	24.9818161472612\\
1787.5	24.9776236294473\\
1787.6	24.9818202159199\\
1787.80000000001	24.9790256725768\\
1787.89999999999	24.9832214765101\\
1788.29999999999	24.9776336390069\\
1788.5	24.9860225874986\\
1788.69999999999	24.9888193202147\\
1789	24.9846291431446\\
1789.1	24.9888218198077\\
1789.5	24.9832364774251\\
1789.60000000001	24.9874280605688\\
1790.2	24.9790537898676\\
1790.29999999999	24.9832439678284\\
1790.5	24.9804534792807\\
1790.70000000001	24.9888318070136\\
1790.89999999999	24.9916247906198\\
1791.80000000001	24.9790724928846\\
1791.90000000001	24.9832589285714\\
1792.09999999999	24.9860506639884\\
1792.39999999999	24.9818688981869\\
1792.5	24.9860537766373\\
1792.7	24.9832663989291\\
1792.80000000001	24.9874504991913\\
1793.2	24.9818769865611\\
1793.29999999999	24.9860599977696\\
1794.10000000001	24.9749191840375\\
1794.19999999999	24.9791004848688\\
1794.6	24.973533181033\\
1794.7	24.9777133942501\\
1795.40000000001	24.9679754942913\\
1795.49999999999	24.9721541546001\\
1796.20000000001	24.9624227578912\\
1796.3	24.9665998663995\\
1796.99999999999	24.9568749652218\\
1797.09999999999	24.9610505230358\\
1797.5	24.9554962171785\\
1797.60000000001	24.9596706903265\\
1798.1	24.9527305082861\\
1798.2	24.9569037424234\\
1798.69999999999	24.9499666444296\\
1798.79999999999	24.9541386402802\\
1798.99999999999	24.9569229058974\\
1799.29999999999	24.9527620317884\\
1799.5	24.9611024672149\\
1799.69999999999	24.9583287031892\\
1799.90000000001	24.9666666666667\\
1800.19999999999	24.9625062489585\\
1800.30000000001	24.9666740724284\\
1800.7	24.9611283873834\\
1800.80000000001	24.9652951302127\\
1801.00000000001	24.9680750652379\\
1801.30000000001	24.9639169534806\\
1801.40000000001	24.9680821537608\\
1801.59999999999	24.9653105400455\\
1801.69999999999	24.969474969475\\
1801.89999999999	24.9667036625971\\
1802.00000000001	24.9708673214583\\
1802.3	24.9778073679538\\
1802.5	24.9750360590259\\
1802.8	24.9875201064951\\
1803.50000000001	24.9778221335108\\
1803.60000000001	24.9819814825082\\
1804.20000000001	24.9736740009976\\
1804.3	24.9778319663046\\
1804.90000000001	24.9695290858726\\
1805	24.973685668384\\
1805.69999999999	24.9640048731864\\
1805.9	24.9723145071982\\
1807.2	24.954351795496\\
1807.30000000001	24.9585039282948\\
1808.70000000001	24.9391862007961\\
1808.99999999999	24.9516334088773\\
1809.3	24.9474964076489\\
1809.4	24.9516441005803\\
1809.60000000001	24.9488865557827\\
1809.70000000001	24.9530334843629\\
1809.9	24.9558011049724\\
1810.1	24.9530438625566\\
1810.39999999999	24.9654791494062\\
1810.69999999999	24.9613430527943\\
1810.79999999999	24.965486774532\\
1811	24.9682513389653\\
1812.00000000001	24.954472711219\\
1812.1	24.9586138395321\\
1812.40000000001	24.9544827586207\\
1812.49999999999	24.9586229725257\\
1813.1	24.9503639973527\\
1813.29999999999	24.9586412264255\\
1813.99999999999	24.9490105286368\\
1814.1	24.9531473927902\\
1814.49999999999	24.9476468643227\\
1814.6	24.9517826638012\\
1815.70000000001	24.9366670338143\\
1815.79999999999	24.9408007048846\\
1816.00000000001	24.9435603766312\\
1816.20000000001	24.9408137422232\\
1816.5	24.9532092920841\\
1817.2	24.9435976448578\\
1817.40000000001	24.9518569463549\\
1818.19999999999	24.9408788428752\\
1818.40000000001	24.9491339015672\\
1818.6	24.9463902787706\\
1818.8	24.9546429160482\\
1818.99999999999	24.95739651476\\
1819.20000000001	24.9601495080526\\
1819.39999999999	24.9574058807365\\
1819.50000000001	24.9615300065949\\
1820.60000000001	24.9464491679025\\
1820.69999999999	24.9505711775044\\
1820.90000000001	24.9533223503569\\
1821.1	24.9505820338239\\
1821.20000000001	24.9547026848954\\
1821.40000000001	24.9574526489157\\
1822.30000000001	24.9451273046532\\
1822.4	24.9492455418381\\
1823.09999999999	24.9396665204037\\
1823.20000000001	24.9437832501508\\
1824.7	24.9232792634809\\
1824.80000000001	24.9273932818237\\
1825.2	24.9219306415384\\
1825.30000000001	24.9260436068807\\
1825.6	24.92194774607\\
1825.80000000001	24.9301714223123\\
1826.99999999999	24.9137978216846\\
1827.10000000002	24.9179071803853\\
1827.4	24.9138166894665\\
1827.49999999999	24.9179251477347\\
1827.69999999999	24.9151985994091\\
1828.00000000001	24.927520376347\\
1829.00000000001	24.9138920780712\\
1829.1	24.9179969385524\\
1829.39999999999	24.9248428532386\\
1829.60000000001	24.9221183800623\\
1829.69999999999	24.9262214449667\\
1830.60000000001	24.9139673348992\\
1830.8	24.922169424873\\
1830.99999999999	24.9194473267435\\
1831.10000000001	24.9235474006116\\
1832.6	24.9031483603427\\
1832.7	24.9072457442165\\
1832.90000000001	24.9099836333879\\
1833.2	24.916816669394\\
1833.5	24.912739965096\\
1833.60000000001	24.9168348148552\\
1833.90000000001	24.9127589967285\\
1834.20000000001	24.9250395246143\\
1834.40000000001	24.927773235214\\
1834.6	24.930506349812\\
1834.9	24.9264305177112\\
1834.99999999999	24.9305214974661\\
1836.1	24.9155865374142\\
1836.2	24.9196754342972\\
1836.40000000001	24.9224067519739\\
1836.60000000001	24.925137474819\\
1837.5	24.9129299085764\\
1837.70000000001	24.9211013167918\\
1838.40000000001	24.9116127277672\\
1838.49999999999	24.9156967257696\\
1839.30000000001	24.9048602805263\\
1839.4	24.9089426474586\\
1839.6	24.9116703810404\\
1839.90000000001	24.9076086956522\\
1840	24.9116895820879\\
1840.20000000001	24.908982231158\\
1840.30000000001	24.913062377744\\
1840.70000000001	24.9076488483268\\
1840.8	24.9117279591504\\
1840.99999999999	24.9144533159524\\
1841.90000000001	24.9022801302932\\
1842	24.9063568753054\\
1842.3	24.9023013460703\\
1842.4	24.9063772048847\\
1842.60000000001	24.903673956694\\
1842.7	24.9077490774908\\
1843.1	24.90234375\\
1843.2	24.9064178375739\\
1843.4	24.909140222403\\
1843.59999999999	24.9118620165971\\
1843.8	24.9091599327512\\
1843.89999999999	24.9132321041215\\
1844.09999999999	24.9105303112461\\
1844.20000000001	24.9146017459199\\
1845.29999999999	24.8997507315487\\
1845.39999999999	24.9038201029531\\
1845.9	24.8970747562297\\
1846	24.9011429500027\\
1846.4	24.8957487137828\\
1846.59999999999	24.9038826013971\\
1846.79999999999	24.906600249066\\
1846.99999999999	24.9093173082129\\
1847.20000000001	24.9120337790289\\
1847.49999999999	24.9188135960165\\
1847.70000000001	24.9161164628207\\
1847.8	24.9201796634017\\
1848.5	24.9107432651737\\
1848.70000000001	24.9188662916486\\
1848.9	24.9161709031909\\
1848.99999999999	24.9202314639554\\
1849.3	24.9161890342814\\
1849.4	24.9202487158692\\
1849.59999999999	24.9229604800778\\
1850.2	24.9148786683241\\
1850.29999999999	24.9189364461738\\
1850.89999999999	24.9108589951378\\
1851.00000000001	24.914915455675\\
1851.90000000001	24.902807775378\\
1852.00000000001	24.9068624804276\\
1852.29999999999	24.9136255668322\\
1853	24.904214559387\\
1853.1	24.9082667817829\\
1853.3	24.9109744253804\\
1853.49999999999	24.9082865774709\\
1853.60000000001	24.9123374871878\\
1853.90000000001	24.908306364617\\
1854	24.9123563993312\\
1855	24.8989272815482\\
1855.10000000001	24.9029754204398\\
1855.29999999999	24.9056807157486\\
1855.8	24.8989708497225\\
1855.90000000002	24.9030172413793\\
1856.19999999999	24.8989926197274\\
1856.30000000001	24.9030381383323\\
1856.7	24.8976734166308\\
1856.90000000001	24.9057619816909\\
1857.19999999999	24.901739083616\\
1857.29999999999	24.9057822763002\\
1857.5	24.9084840654608\\
1857.69999999999	24.9058025621703\\
1857.90000000001	24.9138858988159\\
1858.29999999999	24.9246663796814\\
1858.49999999999	24.9273646830948\\
1858.69999999999	24.9300624058532\\
1858.90000000001	24.9327595481442\\
1860.4	24.9126578876646\\
1860.5	24.9166935397184\\
1861.40000000001	24.9046467902229\\
1861.50000000001	24.9086807047701\\
1862.39999999999	24.896644295302\\
1862.5	24.9006764737464\\
1862.9	24.8953301127214\\
1863	24.8993612795878\\
1863.30000000001	24.895352581303\\
1863.39999999999	24.8993828816743\\
1863.7	24.8953750402404\\
1863.80000000001	24.899404474489\\
1864.10000000001	24.8953974895398\\
1864.29999999999	24.9034541943789\\
1865.4	24.8887697668185\\
1865.49999999999	24.8927958833619\\
1865.99999999999	24.886126145437\\
1866.1	24.8901511092059\\
1866.39999999999	24.8861505491562\\
1866.50000000001	24.8901746490946\\
1866.70000000001	24.8928647953718\\
1867.50000000001	24.8822017562647\\
1867.79999999999	24.8942662883452\\
1868.10000000001	24.8902687078471\\
1868.19999999999	24.8942889257614\\
1868.50000000001	24.890292197367\\
1868.6	24.8943115534864\\
1868.8	24.8969982342554\\
1869.2	24.8916706788637\\
1869.30000000001	24.8956884561892\\
1869.59999999999	24.8916938546291\\
1869.69999999999	24.8957107712055\\
1870.5	24.8850636159521\\
1870.69999999999	24.8930938635878\\
1871.6	24.88112411177\\
1871.7	24.8851373009937\\
1872.19999999999	24.878491694707\\
1872.40000000001	24.8865153538051\\
1872.79999999999	24.8812002776443\\
1872.90000000001	24.8852108916177\\
1873.6	24.8759139670171\\
1873.70000000001	24.8799231508165\\
1874.3	24.8719590268886\\
1874.59999999999	24.8839814370299\\
1875.29999999999	24.8746933987416\\
1875.40000000001	24.8786990135964\\
1875.60000000001	24.8760462760569\\
1875.70000000001	24.880051178164\\
1875.9	24.8773987206823\\
1875.99999999999	24.8814029102926\\
1876.89999999999	24.8694725625999\\
1877	24.8734750412871\\
1877.80000000001	24.8628787475371\\
1877.89999999999	24.8668796592119\\
1878.1	24.869555957832\\
1879.19999999999	24.8549992018305\\
1879.30000000001	24.8589975524103\\
1879.5	24.8563524154075\\
1879.60000000001	24.86035005586\\
1879.89999999999	24.8563829787234\\
1880.09999999999	24.8643761301989\\
1880.29999999999	24.8617315464795\\
1880.40000000001	24.8657272002127\\
1880.8	24.86043915147\\
1880.9	24.8644338118022\\
1881.10000000001	24.8617903465873\\
1881.19999999999	24.8657842980917\\
1881.4	24.8684560191337\\
1882.70000000001	24.8512853197366\\
1882.79999999999	24.855276435286\\
1883.00000000001	24.8579470022835\\
1883.30000000001	24.8539874694701\\
1883.49999999999	24.8619664472287\\
1883.69999999999	24.8593268924514\\
1883.80000000002	24.8633154626042\\
1884.19999999999	24.8580374674946\\
1884.29999999999	24.8620250477606\\
1885.00000000001	24.8527929552809\\
1885.09999999999	24.8567791215786\\
1885.3	24.8541423570595\\
1885.4	24.858127817555\\
1886.00000000001	24.8502200307513\\
1886.09999999999	24.8542042201251\\
1886.29999999999	24.8568702290076\\
1886.49999999999	24.8542351319835\\
1886.6	24.8582180526846\\
1886.79999999999	24.8555832317558\\
1886.89999999999	24.8595654478007\\
1887.3	24.8542969163929\\
1887.39999999999	24.8582781456954\\
1887.59999999999	24.8556444350268\\
1887.90000000002	24.8675847457627\\
1888.2	24.8636339564688\\
1888.3	24.8676127938996\\
1888.50000000002	24.8702742772424\\
1889.09999999999	24.8623756087233\\
1889.2	24.8663526173715\\
1889.49999999999	24.8624047417443\\
1889.60000000001	24.8663809070223\\
1889.8	24.8637494047304\\
1889.9	24.8677248677249\\
1890.10000000001	24.8703840863401\\
1890.30000000001	24.8677528565383\\
1890.4	24.8717270563343\\
1890.60000000001	24.8743851483578\\
1890.79999999999	24.8770426780898\\
1891.5	24.867836751956\\
1891.60000000001	24.8718084262832\\
1892	24.8665503937424\\
1892.10000000001	24.8705210865659\\
1892.49999999999	24.8652647152066\\
1892.6	24.869234427009\\
1892.8	24.8666067938084\\
1893	24.8745443980772\\
1893.2	24.8771985422279\\
1893.4	24.8745709004489\\
1893.59999999999	24.8825051486508\\
1894.19999999999	24.8746238716148\\
1894.4	24.8825547637899\\
1894.7	24.8786151572725\\
1894.79999999999	24.8825795556494\\
1895.49999999999	24.8733910107618\\
1895.70000000001	24.8813165945775\\
1895.90000000001	24.8839662447257\\
1896.40000000001	24.8774057474295\\
1896.50000000001	24.8813666561215\\
1897.60000000001	24.8669441956052\\
1897.69999999999	24.870903151017\\
1898.30000000001	24.8630425621576\\
1898.40000000001	24.8670002633658\\
1898.6	24.8696476536578\\
1898.80000000001	24.8670282795303\\
1898.89999999999	24.8709847288046\\
1899.30000000001	24.8657470780246\\
1899.40000000001	24.8697025533035\\
1899.6	24.8723482655156\\
1899.89999999999	24.8684210526316\\
1900.00000000002	24.8723751381506\\
1900.79999999999	24.8619075174917\\
1900.9	24.8658600736455\\
1901.29999999999	24.860629010203\\
1901.40000000001	24.8645805942677\\
1901.70000000001	24.8606583236933\\
1901.89999999999	24.8685594111462\\
1902.4	24.862023653088\\
1902.49999999999	24.8659728792179\\
1902.90000000001	24.860746190226\\
1903.1	24.8686422866751\\
1903.3	24.8712829673216\\
1903.6	24.867363555182\\
1903.90000000001	24.8792016806723\\
1904.10000000002	24.8818401428421\\
1904.50000000001	24.8766145122335\\
1904.6	24.8805586181551\\
1904.80000000001	24.8779463488897\\
1905.00000000001	24.8858327646843\\
1905.29999999999	24.8924110423008\\
1905.60000000001	24.8884924174844\\
1905.70000000001	24.8924336236751\\
1905.9	24.8950682056663\\
1906.09999999999	24.8924561955723\\
1906.20000000001	24.8963961601007\\
1906.6	24.9069072219017\\
1908.19999999999	24.8860242100299\\
1908.30000000001	24.8899601760637\\
1908.8	24.8834407250249\\
1908.90000000001	24.8873755893138\\
1910.00000000001	24.8730432961625\\
1910.10000000001	24.8769762328552\\
1910.3	24.8743718592965\\
1910.4	24.878304108872\\
1910.6	24.8757000052337\\
1910.9	24.887493458922\\
1911.5	24.8796819418288\\
1911.59999999999	24.8836114453105\\
1912	24.8784059411119\\
1912.1	24.8823344838406\\
1912.8	24.8732291285483\\
1912.89999999999	24.8771562990068\\
1913.2	24.8732556316312\\
1913.3	24.877181979722\\
1913.59999999999	24.8732821236348\\
1913.69999999999	24.8772076497022\\
1914.2	24.8707099200752\\
1914.29999999999	24.8746343501881\\
1914.69999999999	24.8694380614163\\
1914.79999999999	24.8733615332393\\
1915.50000000001	24.8642722906661\\
1915.6	24.8681943936942\\
1915.90000000001	24.8643006263048\\
1915.99999999999	24.8682219090862\\
1916.40000000001	24.8786851030524\\
1916.90000000001	24.8721961398018\\
1917.00000000001	24.8761149653122\\
1917.29999999999	24.8722228017106\\
1917.5	24.8800584063413\\
1917.9	24.8748696558916\\
1918.00000000001	24.8787862989417\\
1918.30000000001	24.8748957464554\\
1918.40000000001	24.8788115715403\\
1918.7	24.8749218261413\\
1918.99999999999	24.8866656245115\\
1919.2	24.88928255093\\
1920.3	24.8750260362425\\
1920.50000000001	24.8828491096532\\
1920.8	24.8789629861003\\
1920.89999999999	24.8828735033837\\
1921.19999999999	24.878988185083\\
1921.49999999999	24.8907160699417\\
1921.69999999999	24.8881257154751\\
1921.8	24.892033924762\\
1922.10000000001	24.8985537405057\\
1922.49999999999	24.9089774264017\\
1923.20000000001	24.8999116102532\\
1923.29999999999	24.9038161588853\\
1923.99999999999	24.8947559898134\\
1924.19999999999	24.9025619705867\\
1924.59999999999	24.8973866057048\\
1924.70000000001	24.9012884455528\\
1925.19999999999	24.91559756921\\
1925.39999999999	24.9130096078941\\
1925.5	24.9169090153718\\
1925.69999999999	24.919513968221\\
1925.9	24.9169262720665\\
1926.00000000001	24.9208244639427\\
1926.19999999999	24.923428334112\\
1926.5	24.9299283712239\\
1927.6	24.9157026508274\\
1927.70000000001	24.9195974686171\\
1928.4	24.9105522426757\\
1928.50000000001	24.9144457119154\\
1928.89999999999	24.9248315189217\\
1929.09999999999	24.9274310595065\\
1929.30000000001	24.9248471027262\\
1929.40000000001	24.9287380150298\\
1929.7	24.9248626800705\\
1929.8	24.9287527851184\\
1930.5	24.9197140785248\\
1930.60000000001	24.9236028383488\\
1930.80000000001	24.9262002175152\\
1931.19999999999	24.9210376430384\\
1931.29999999999	24.9249249249249\\
1931.79999999999	24.9184740410994\\
1931.9	24.9223602484472\\
1932.20000000001	24.9184909175594\\
1932.49999999999	24.930145917417\\
1933.50000000002	24.9172527927182\\
1933.60000000001	24.9211356466877\\
1933.80000000001	24.923729251771\\
1933.99999999999	24.9211519569826\\
1934.19999999999	24.9289148529184\\
1934.40000000001	24.9315068493151\\
1934.59999999999	24.928929549801\\
1934.80000000001	24.9366892345858\\
1935.29999999999	24.930246977369\\
1935.39999999999	24.9341255489538\\
1936.7	24.9173895084676\\
1936.80000000002	24.9212659404203\\
1937.39999999999	24.9135483870968\\
1937.49999999999	24.9174236168456\\
1937.90000000001	24.9122807017544\\
1937.99999999999	24.9161549971622\\
1938.20000000001	24.9187432286024\\
1938.50000000001	24.9148870318787\\
1938.59999999999	24.9187599938103\\
1939	24.9136197204889\\
1939.10000000001	24.9174917491749\\
1939.70000000001	24.9097845138674\\
1939.80000000001	24.9136553430589\\
1940.9	24.8995363214838\\
1941.00000000001	24.9034052856628\\
1941.49999999999	24.917593737124\\
1941.69999999999	24.9201771552168\\
1942.00000000001	24.9163276865249\\
1942.10000000001	24.9201935948924\\
1942.90000000002	24.9099330931549\\
1943.2	24.9215252405702\\
1943.39999999999	24.9189606380242\\
1943.5	24.9228236262606\\
1943.7	24.9254038481325\\
1944.49999999999	24.9151496451712\\
1944.60000000001	24.9190106443153\\
1944.80000000001	24.9215897989614\\
1945.39999999999	24.9139038807505\\
1945.49999999999	24.9177631578947\\
1945.8	24.9139215787039\\
1945.89999999999	24.917780061665\\
1946.09999999999	24.9152194019114\\
1946.19999999999	24.9190772234496\\
1946.80000000002	24.9113976064513\\
1947.10000000001	24.9229663105998\\
1947.6	24.916568259999\\
1947.8	24.9242774269726\\
1947.99999999999	24.9217185976079\\
1948.1	24.9255723231701\\
1949.19999999999	24.9115066947109\\
1949.3	24.9153585718683\\
1949.80000000001	24.9089696907534\\
1949.9	24.9128205128205\\
1950.10000000001	24.9153932929956\\
1951.90000000002	24.8924180327869\\
1951.99999999999	24.896265560166\\
1952.29999999999	24.8924400737554\\
1952.5	24.9001331557923\\
1952.8	24.9065492344718\\
1953.80000000001	24.8938021393111\\
1953.89999999999	24.8976458546571\\
1954.30000000001	24.8925501432665\\
1954.40000000002	24.8963929393707\\
1954.7	24.902803355842\\
1955.19999999999	24.8964353296169\\
1955.4	24.9041165942214\\
1955.60000000001	24.9015697704147\\
1955.8	24.9092489391073\\
1956.20000000001	24.9041558043245\\
1956.30000000001	24.9079942751993\\
1956.5	24.9105591331902\\
1956.69999999999	24.9080130825838\\
1956.80000000001	24.9118503755941\\
1957.09999999999	24.9080318822808\\
1957.2	24.9118683901293\\
1957.80000000001	24.9042341284029\\
1958.00000000001	24.9119043971197\\
1958.69999999999	24.9030018378599\\
1958.79999999999	24.9068354688856\\
1959	24.9093971721709\\
1959.20000000001	24.9119583524728\\
1959.60000000001	24.9068735010461\\
1959.70000000001	24.9107051739973\\
1959.89999999999	24.9132653061224\\
1960.29999999999	24.9081820036727\\
1960.39999999999	24.9120122417751\\
1960.70000000002	24.9082007343941\\
1961.20000000001	24.9273441084995\\
1961.40000000001	24.929900586286\\
1961.59999999999	24.9324565427945\\
1961.80000000001	24.9350119781844\\
1961.99999999999	24.9324703124204\\
1962.1	24.9362959942921\\
1962.4	24.9324840764331\\
1962.59999999999	24.9401334895807\\
1962.80000000001	24.9426868408987\\
1963.40000000001	24.9350649350649\\
1963.6	24.9427101899475\\
1963.9	24.938900203666\\
1964	24.9427218573392\\
1964.19999999999	24.94018225322\\
1964.29999999999	24.9440032579923\\
1964.49999999999	24.9465540059045\\
1964.69999999999	24.9440146579805\\
1964.90000000002	24.9516539440204\\
1965.19999999999	24.9478451127054\\
1965.5	24.9592999593\\
1966	24.9529525456487\\
1966.29999999999	24.9644019528072\\
1967.5	24.9491766619232\\
1967.8	24.9606179175771\\
1968.5	24.9517423549731\\
1968.6	24.9555544267791\\
1969.40000000002	24.9454176186849\\
1969.7	24.9568484110062\\
1970.00000000001	24.953048068626\\
1970.09999999999	24.9568571718607\\
1971.09999999999	24.9441964285714\\
1971.19999999999	24.9480038553239\\
1971.49999999999	24.9442077500507\\
1971.59999999999	24.948014403814\\
1972.20000000002	24.9404248846524\\
1972.3	24.9442303792334\\
1972.49999999999	24.9467707594038\\
1973.3	24.9366575453532\\
1973.4	24.9404611097036\\
1973.8	24.9354070621612\\
1973.9	24.9392097264438\\
1974.09999999999	24.9417485563773\\
1974.30000000001	24.9442868719611\\
1975.1	24.9341838801134\\
1975.30000000001	24.9417839424927\\
1975.5	24.9443207126949\\
1975.69999999999	24.9417957283126\\
1975.79999999999	24.9455944126727\\
1976.5	24.9367600930891\\
1976.70000000001	24.9443545123432\\
1976.89999999999	24.9468892261002\\
1977.10000000002	24.9494234270686\\
1977.29999999999	24.9519571154041\\
1977.50000000001	24.9544902912621\\
1977.7	24.9519668318334\\
1977.8	24.9557611608271\\
1978.39999999999	24.9481930755623\\
1978.49999999999	24.9519862529061\\
1979.10000000001	24.9444219886823\\
1979.20000000002	24.9482140150558\\
1979.40000000001	24.950745137661\\
1979.99999999999	24.943184687642\\
1980.09999999999	24.9469750530249\\
1980.4	24.9431961625852\\
1980.5	24.9469857618903\\
1980.8	24.953304053713\\
1981	24.9558326182424\\
1981.2	24.9583606722859\\
1982.19999999999	24.9457700650759\\
1982.5	24.957127004943\\
1982.90000000001	24.952092788704\\
1983.1	24.959661153691\\
1983.4	24.955886059995\\
1983.50000000001	24.9596692881629\\
1983.80000000001	24.9659761076667\\
1984.19999999999	24.960943405735\\
1984.30000000001	24.9647248538601\\
1984.6	24.9609512772711\\
1984.70000000001	24.9647319629182\\
1985.19999999999	24.9584445675717\\
1985.39999999999	24.9660035255603\\
1985.60000000001	24.9634889459636\\
1985.7	24.9672675999597\\
1986.00000000001	24.9735662856855\\
1986.2	24.9760861904043\\
1986.80000000001	24.9685439629574\\
1986.9	24.9723200805234\\
1987.29999999999	24.967293951897\\
1987.5	24.9748440330046\\
1987.79999999999	24.9811358720258\\
1988.49999999999	24.972342351403\\
1988.59999999999	24.9761150500327\\
1988.8	24.9786314042938\\
1989.39999999999	24.971098265896\\
1989.5	24.9748693204664\\
1989.69999999999	24.9773846617751\\
1990.30000000001	24.9698553054662\\
1990.39999999999	24.9736247174077\\
1990.90000000002	24.9673530889\\
1991.00000000001	24.9711214906333\\
1991.50000000001	24.964852379996\\
1991.6	24.968619772054\\
1992.00000000001	24.9636062446664\\
1992.10000000002	24.9673727537396\\
1992.4	24.9636135508156\\
1992.60000000001	24.971144678075\\
1992.8	24.9736564805058\\
1993.00000000001	24.976167778837\\
1993.7	24.9673989367038\\
1993.80000000001	24.9711620442349\\
1994.29999999999	24.9649017248295\\
1994.4	24.9686638255202\\
1994.70000000001	24.9649087627832\\
1994.79999999999	24.9686701087774\\
1995.40000000001	24.9862189927337\\
1995.89999999999	24.9799599198397\\
1996.10000000001	24.9874762047891\\
1997.30000000001	24.9724642034645\\
1997.50000000001	24.9799759711654\\
1997.70000000001	24.977475222745\\
1997.8	24.9812302918064\\
1998.90000000002	24.9674837418709\\
1999.1	24.9749899959984\\
1999.3	24.9724917475243\\
1999.40000000001	24.9762440610153\\
1999.60000000001	24.9737460619093\\
1999.70000000001	24.977497749775\\
2000	24.9837508124594\\
2000.20000000001	24.9812528120782\\
2000.29999999999	24.9850029994001\\
2001.39999999999	24.9712715463402\\
2001.50000000001	24.9750199840128\\
2001.7	24.972524727745\\
2001.80000000001	24.976272541086\\
2002.20000000002	24.9712830245218\\
2002.39999999999	24.9787765293383\\
2002.90000000002	24.9725411882177\\
2003	24.9762867555289\\
2003.4	24.9713002246069\\
2003.50000000001	24.9750449191455\\
2004.00000000001	24.9688139314406\\
2004.1	24.9725576289791\\
2004.39999999999	24.968820154652\\
2004.50000000001	24.9725631048588\\
2004.8	24.9688263753803\\
2005.00000000002	24.9763104084584\\
2005.20000000002	24.9788061636663\\
2006.4	24.9638674308497\\
2006.50000000001	24.9676068972391\\
2007.6	24.9539273795886\\
2007.69999999999	24.9576651060863\\
2008.4	24.9489668907145\\
2008.49999999999	24.9527033754854\\
2009.30000000002	24.9427689857669\\
2009.39999999999	24.9465041054989\\
2009.59999999999	24.9440214957456\\
2009.70000000001	24.9477559956215\\
2010.79999999999	24.9341091053757\\
2011.10000000001	24.9453062848051\\
2011.3	24.9477975539425\\
2011.60000000001	24.9440771486802\\
2011.70000000001	24.9478079331942\\
2011.9	24.9453280318091\\
2012.00000000002	24.9490581979027\\
2012.20000000001	24.9465785419669\\
2012.30000000001	24.950308089843\\
2012.80000000001	24.9441104873566\\
2013	24.9515672346133\\
2013.30000000001	24.9577828548724\\
2014.1	24.9478701221329\\
2014.19999999999	24.951596087971\\
2014.90000000002	24.9429280397022\\
2015.00000000001	24.9466527715746\\
2015.89999999999	24.9355158730159\\
2016	24.9392391250434\\
2016.39999999999	24.9342920902554\\
2016.49999999999	24.9380144798175\\
2016.90000000001	24.9330689142291\\
2017.00000000001	24.9367904417233\\
2017.40000000001	24.9318463444857\\
2017.50000000002	24.9355670103093\\
2017.8	24.931859854304\\
2017.89999999999	24.9355797819623\\
2018.09999999999	24.9380636210485\\
2018.40000000001	24.9442655437206\\
2019.09999999999	24.935618066561\\
2019.20000000001	24.939335413262\\
2019.70000000001	24.9331616991781\\
2019.8	24.9368780632705\\
2020.70000000001	24.9257719714964\\
2020.8	24.9294868622891\\
2021.2	24.9245535051699\\
2021.30000000001	24.9282675373504\\
2021.6	24.9344610970965\\
2022.39999999999	24.9245982694685\\
2022.49999999998	24.9283100959162\\
2022.70000000001	24.9258453628634\\
2022.80000000001	24.9295565771912\\
2022.99999999999	24.9270920864021\\
2023.1	24.9308026888098\\
2023.3	24.9283384402491\\
2023.4	24.9320484309365\\
2023.9	24.9258893280632\\
2023.99999999999	24.929598340003\\
2024.70000000001	24.9209798498617\\
2024.8	24.9246876388957\\
2025.8	24.9123846191816\\
2025.90000000001	24.9160908193485\\
2026.19999999999	24.9222721216009\\
2026.40000000001	24.9198124845793\\
2026.59999999999	24.9272215917501\\
2027.19999999999	24.9198441276575\\
2027.29999999999	24.9235474006116\\
2027.79999999999	24.9174022387692\\
2027.90000000001	24.9211045364892\\
2028.10000000001	24.923577556454\\
2029.3	24.9088400512467\\
2029.39999999999	24.9125400344913\\
2029.70000000001	24.908858015568\\
2029.80000000001	24.912557268831\\
2029.99999999999	24.9101029505936\\
2030.1	24.9138015959019\\
2030.30000000001	24.9162726556344\\
2030.50000000002	24.9138185757904\\
2030.69999999999	24.9212133149498\\
2031	24.9175323716213\\
2031.2	24.9249249249249\\
2031.40000000001	24.9273935515629\\
2031.6	24.929861692179\\
2032.09999999999	24.9237279795296\\
2032.19999999999	24.9274221325592\\
2032.69999999999	24.9212908303817\\
2032.8	24.9249840129864\\
2032.99999999999	24.9274506910629\\
2033.19999999999	24.9299168838833\\
2033.4	24.9274649618884\\
2033.6	24.9348478143286\\
2033.79999999999	24.9373125522395\\
2034.2	24.9324091825198\\
2034.29999999999	24.9360990955564\\
2034.59999999999	24.932422470143\\
2034.7	24.9361116571653\\
2035.2	24.9299857514863\\
2035.29999999999	24.9336739707183\\
2035.89999999999	24.926326129666\\
2036.00000000001	24.9300132606454\\
2036.19999999999	24.9275647006826\\
2036.29999999998	24.9312512276567\\
2036.8	24.9251313270165\\
2036.90000000002	24.9288168875798\\
2037.19999999999	24.9251460266038\\
2037.3	24.9288308628644\\
2037.6	24.9251607204201\\
2037.80000000001	24.9325285833456\\
2038.00000000001	24.9349884696531\\
2038.2	24.9374478732277\\
2038.8	24.930109372701\\
2038.90000000001	24.9337910740559\\
2039.10000000001	24.9313456257356\\
2039.20000000001	24.9350267248566\\
2039.8	24.9276925339477\\
2040.1	24.9387314969121\\
2040.29999999999	24.9362870025485\\
2040.39999999999	24.9399656946827\\
2041.00000000001	24.9326343638234\\
2041.09999999999	24.936311973349\\
2042.4	24.9204406364749\\
2042.49999999999	24.9241163223343\\
2042.8	24.9204562142053\\
2042.90000000001	24.9241311796378\\
2043.40000000002	24.9180327868852\\
2043.5	24.9217067919358\\
2043.8	24.9180488282206\\
2044	24.9253950393816\\
2044.29999999999	24.9217374290745\\
2044.4	24.9254096356077\\
2044.80000000002	24.9205340114431\\
2045	24.9278763874627\\
2045.90000000001	24.9169110459433\\
2046.09999999999	24.9242498289512\\
2046.3	24.9218139171228\\
2046.49999999999	24.9291507866706\\
2047.10000000001	24.9218444704963\\
2047.2	24.9255116494896\\
2047.49999999999	24.9218597382301\\
2047.69999999999	24.9291923039359\\
2047.89999999999	24.9267578125\\
2048.09999999999	24.9340884679231\\
2048.40000000001	24.9304369050525\\
2048.60000000001	24.9377654122126\\
2049.49999999999	24.9268149882904\\
2049.59999999999	24.9304776308728\\
2049.8	24.928045270501\\
2049.9	24.9317073170732\\
2050.10000000001	24.9292751926641\\
2050.2	24.932936643418\\
2050.60000000001	24.9280733408105\\
2050.69999999999	24.93173395748\\
2050.90000000002	24.9293027791321\\
2050.99999999999	24.9329628004485\\
2051.29999999999	24.9390660037048\\
2051.69999999999	24.9342041134614\\
2051.8	24.9378624689312\\
2051.99999999999	24.9354319964914\\
2052.09999999999	24.9390897573336\\
2052.29999999999	24.9366595205613\\
2052.40000000001	24.9403166869671\\
2052.7	24.9366718628215\\
2052.80000000002	24.9403283160407\\
2053.00000000001	24.9378987871998\\
2053.10000000001	24.9415546464056\\
2053.39999999999	24.9379108838568\\
2053.5	24.9415660303857\\
2054.29999999999	24.9318535825545\\
2054.40000000001	24.9355074227306\\
2054.69999999999	24.9416001557329\\
2054.9	24.9391727493917\\
2055.00000000001	24.9428251666586\\
2055.19999999999	24.9452634651876\\
2056.10000000001	24.9343449080829\\
2056.2	24.9379954286826\\
2056.70000000001	24.9319330999611\\
2057.00000000001	24.9428807544602\\
2057.30000000001	24.9489647127442\\
2057.5	24.951399688958\\
2058	24.945337933045\\
2058.1	24.9489845496065\\
2058.4	24.9453485547729\\
2058.5	24.9489944622559\\
2058.79999999999	24.9453591723736\\
2059	24.9526492156768\\
2059.50000000002	24.9465915711789\\
2059.60000000001	24.9502354711851\\
2060	24.9599534003204\\
2060.3	24.9660260143661\\
2061.00000000001	24.9575469409539\\
2061.10000000002	24.9611876576751\\
2061.29999999999	24.963616959348\\
2061.69999999999	24.9587738868949\\
2061.79999999999	24.9624133081139\\
2062.39999999998	24.9551515151515\\
2062.49999999998	24.9587898768545\\
2062.70000000001	24.956369982548\\
2062.80000000001	24.9600077560716\\
2063.3	24.9539594843462\\
2063.40000000001	24.9575963169372\\
2063.90000000001	24.9515503875969\\
2064.1	24.9588218195911\\
2064.80000000002	24.9503607922902\\
2064.89999999999	24.953995157385\\
2065.1	24.9515785396088\\
2065.2	24.9552123178231\\
2066.5	24.9395141778767\\
2066.60000000001	24.9431460782891\\
2067.1	24.937113003096\\
2067.19999999999	24.9407439655589\\
2067.70000000002	24.9347132217816\\
2067.89999999999	24.9419729206963\\
2068.10000000002	24.9443960932212\\
2068.39999999999	24.9407783417936\\
2068.6	24.9480349978247\\
2069.80000000001	24.933571670129\\
2070	24.9408241147771\\
2071.3	24.9251713816742\\
2071.39999999998	24.9287955587738\\
2071.80000000001	24.9239828177036\\
2072	24.9312291877805\\
2072.30000000001	24.9372707971434\\
2072.5	24.9348644214996\\
2072.8	24.94572820686\\
2073.20000000001	24.9409154488014\\
2073.3	24.9445355454809\\
2074.8	24.9265024820473\\
2075.00000000001	24.933738133102\\
2075.19999999999	24.9361538090878\\
2075.40000000001	24.9337509033968\\
2075.60000000002	24.9409837645132\\
2075.89999999999	24.9373795761079\\
2076.29999999999	24.9518397225968\\
2076.50000000001	24.9542521429259\\
2076.70000000001	24.9566640986133\\
2077.30000000001	24.9494560508328\\
2077.4	24.9530685920578\\
2077.6	24.9554796168841\\
2077.8	24.9530776264498\\
2077.9	24.9566891241578\\
2078.19999999999	24.9530866573642\\
2078.30000000001	24.9566974595843\\
2078.59999999999	24.953095684803\\
2078.70000000001	24.956705791803\\
2078.90000000001	24.959114959115\\
2079.39999999999	24.9531137292618\\
2079.5	24.9567224466244\\
2080.4	24.9459264599856\\
2080.50000000001	24.9495337883303\\
2080.69999999999	24.9471357170319\\
2080.79999999999	24.9507424672017\\
2081.49999999999	24.9423520368947\\
2081.60000000001	24.9459576307825\\
2082	24.9411651697805\\
2082.10000000001	24.9447699548554\\
2082.60000000001	24.9387813895424\\
2082.69999999999	24.9423852506242\\
2083.09999999999	24.9375960061444\\
2083.2	24.9411990591849\\
2083.60000000001	24.9364111916303\\
2083.7	24.9400134369901\\
2084.00000000001	24.9364233961902\\
2084.1	24.940024949621\\
2084.29999999999	24.9376319324506\\
2084.39999999999	24.9412329095706\\
2084.70000000002	24.9472371450499\\
2084.99999999999	24.9436477866769\\
2085.09999999999	24.9472472664493\\
2085.4	24.9436585950611\\
2085.5	24.9472573839662\\
2086.09999999999	24.9400824465535\\
2086.2	24.9436801993961\\
2086.99999999999	24.9341191126443\\
2087.09999999999	24.9377155998467\\
2087.50000000001	24.9329373443188\\
2087.60000000001	24.9365330267759\\
2087.8	24.9341443555726\\
2087.89999999998	24.9377394636015\\
2088.10000000001	24.9401398333493\\
2088.29999999999	24.9377513886229\\
2088.40000000001	24.9413454632511\\
2088.69999999999	24.937763309077\\
2088.79999999999	24.9413566949112\\
2089.00000000002	24.9389689339907\\
2089.09999999999	24.9425617461229\\
2089.30000000001	24.9401742126926\\
2089.4	24.9437664513041\\
2090	24.9366059040237\\
2090.1	24.9401971103244\\
2090.60000000002	24.9342325536902\\
2090.70000000001	24.9378228429309\\
2091.20000000001	24.9318605651987\\
2091.39999999999	24.9390389672484\\
2091.60000000001	24.9414351962519\\
2092.29999999998	24.9330911871535\\
2092.4	24.936678614098\\
2092.99999999998	24.9295303616645\\
2093.10000000001	24.9331167590292\\
2093.3	24.9355116079106\\
2094.30000000002	24.9236058059588\\
2094.4	24.9271902602053\\
2094.7	24.923620393355\\
2094.80000000001	24.9272041624899\\
2095.19999999999	24.9367632319954\\
2096.8	24.9177357050885\\
2096.89999999999	24.9213161659514\\
2097.20000000001	24.9177513946503\\
2097.3	24.9213311719272\\
2097.49999999999	24.9189549961861\\
2097.80000000002	24.9296915963583\\
2099.09999999999	24.9142530487805\\
2099.20000000001	24.9178297527747\\
2099.50000000001	24.9237950085731\\
2099.69999999999	24.9261834460425\\
2100.40000000001	24.9178766960248\\
2100.50000000001	24.921451013996\\
2100.8	24.9178923318578\\
2100.9	24.9214659685864\\
2101.50000000001	24.914350970689\\
2101.60000000001	24.9179235856687\\
2102.29999999999	24.9096270928463\\
2102.4	24.9131985731272\\
2102.70000000001	24.9191554118318\\
2104.1	24.9025758007794\\
2104.30000000001	24.9097129823228\\
2104.49999999998	24.9120973106529\\
2105	24.9061802289677\\
2105.10000000001	24.9097472924188\\
2105.30000000001	24.9073810202337\\
2105.39999999999	24.9109475184042\\
2105.99999999999	24.9038507193391\\
2106.09999999998	24.9074161997911\\
2106.4	24.903868976976\\
2106.5	24.90743377955\\
2107.4	24.8967971530249\\
2107.49999999999	24.9003605997343\\
2107.69999999999	24.9027421956542\\
2107.89999999999	24.9003795066414\\
2108	24.9039419382382\\
2108.20000000002	24.9015794716122\\
2108.30000000002	24.9051413394043\\
2108.49999999999	24.9075215782984\\
2108.7	24.9099013657056\\
2109.00000000002	24.9063581622493\\
2109.09999999999	24.9099184524938\\
2110.70000000001	24.8910365738109\\
2110.79999999999	24.8945947226302\\
2110.99999999999	24.8922362749278\\
2111.1	24.8957938613111\\
2111.30000000001	24.8981718291181\\
2111.50000000001	24.9005493464671\\
2111.8	24.8970121691368\\
2111.89999999999	24.9005681818182\\
2112.10000000002	24.9029447968942\\
2113.1	24.8911603255726\\
2113.30000000001	24.8982681934324\\
2113.70000000001	24.893556627874\\
2113.80000000001	24.8971096078339\\
2114.00000000002	24.8994844141715\\
2114.49999999998	24.8935968977584\\
2114.70000000001	24.9006998297711\\
2114.89999999999	24.903073286052\\
2115.1	24.9054462934947\\
2115.40000000001	24.9019144410305\\
2115.59999999999	24.9090135652503\\
2116.50000000002	24.8984219975432\\
2116.70000000001	24.9055177626606\\
2116.99999999998	24.9019885692693\\
2117.10000000002	24.9055356130739\\
2118.2	24.8926025586555\\
2118.40000000001	24.8996931791362\\
2118.70000000002	24.8961676420615\\
2118.8	24.8997121147765\\
2118.99999999998	24.8973620876787\\
2119.10000000001	24.900906002265\\
2120.89999999998	24.8797736916549\\
2121.00000000001	24.8833152609495\\
2121.30000000001	24.8797963608937\\
2121.6	24.8904180609888\\
2121.80000000001	24.8880720109336\\
2122.09999999999	24.8986900386392\\
2122.3	24.8963437617791\\
2122.40000000001	24.8998822143698\\
2123.4	24.8881563456558\\
2123.50000000001	24.8916933509135\\
2123.69999999999	24.8940578208871\\
2123.90000000002	24.891713747646\\
2124.20000000001	24.9023207644871\\
2124.50000000001	24.8988044808435\\
2124.59999999999	24.9023391537629\\
2124.80000000001	24.9047013977128\\
2125.50000000001	24.8964998118178\\
2125.80000000001	24.9070981701867\\
2126.89999999999	24.8942172073343\\
2127.09999999999	24.9012786761941\\
2127.40000000001	24.8977673325499\\
2127.60000000002	24.9048268082906\\
2129.49999999999	24.8826070623591\\
2129.59999999999	24.8861341973048\\
2130.20000000001	24.8791250058677\\
2130.40000000001	24.8861769537667\\
2130.79999999999	24.8815054671735\\
2130.90000000001	24.8850305021117\\
2131.70000000002	24.8756919035557\\
2131.80000000001	24.8792157230639\\
2131.99999999999	24.8815721589044\\
2132.20000000001	24.883928152699\\
2133.1	24.8734295893493\\
2133.19999999999	24.8769512023625\\
2133.4	24.8746191703773\\
2133.70000000001	24.8851813665761\\
2133.89999999999	24.8828491096532\\
2134	24.8863689611546\\
2134.49999999999	24.8805396795653\\
2134.69999999999	24.8875772906127\\
2135.00000000002	24.8840803709428\\
2135.09999999999	24.8875983514425\\
2135.29999999999	24.8899503605882\\
2135.60000000001	24.8864540899939\\
2135.70000000001	24.8899709710647\\
2135.90000000001	24.8876404494382\\
2135.99999999999	24.8911567810496\\
2136.19999999999	24.8888264756823\\
2136.39999999999	24.8958577112099\\
2136.60000000002	24.8935274020686\\
2136.70000000001	24.8970423062523\\
2136.90000000002	24.8947122133833\\
2137.2	24.9052542927993\\
2137.90000000001	24.8971000935454\\
2138.00000000002	24.9006126935129\\
2138.2	24.9029602955619\\
2138.39999999999	24.90063128361\\
2138.49999999999	24.9041428972225\\
2139.20000000001	24.8959940167344\\
2139.29999999999	24.8995045339815\\
2139.50000000001	24.9018508132361\\
2140.2	24.8937064897444\\
2140.29999999999	24.8972154737432\\
2140.60000000001	24.9030690895502\\
2140.80000000002	24.9054136110981\\
2142.00000000001	24.8914616497829\\
2142.1	24.8949677901223\\
2143.00000000001	24.8845130885166\\
2143.10000000002	24.8880179171333\\
2143.30000000001	24.8856956237753\\
2143.50000000001	24.8927038626609\\
2143.80000000001	24.8892205793181\\
2143.9	24.892723880597\\
2144.10000000001	24.8950657587912\\
2144.29999999999	24.8974072001492\\
2145	24.88928255093\\
2145.1	24.892783889614\\
2145.3	24.890463316864\\
2145.50000000002	24.8974645786726\\
2145.99999999999	24.8916639485578\\
2146.10000000001	24.89516354487\\
2146.40000000001	24.8916841369672\\
2146.5	24.8951830802199\\
2146.70000000001	24.8928637972797\\
2146.90000000001	24.8998602701444\\
2147.40000000001	24.8940628637951\\
2147.49999999998	24.8975600670516\\
2148	24.8917648154183\\
2148.1	24.8952611488688\\
2148.49999999998	24.8906264544354\\
2148.6	24.8941220272723\\
2148.8	24.8964586532645\\
2150.19999999999	24.8802492675441\\
2150.30000000001	24.8837425595238\\
2150.60000000001	24.8802715394988\\
2150.69999999999	24.88376418077\\
2151.1	24.8791372257345\\
2151.29999999999	24.8861206656131\\
2152.10000000001	24.8768701793514\\
2152.2	24.8803605445338\\
2152.50000000001	24.8768930595559\\
2152.60000000001	24.8803827751196\\
2152.79999999999	24.8827163361048\\
2153.19999999999	24.8780940881438\\
2153.29999999999	24.8815826135414\\
2154.29999999999	24.8700334199777\\
2154.39999999999	24.873520538408\\
2154.59999999999	24.8758527869309\\
2155.29999999999	24.8677739630695\\
2155.49999999999	24.8747448506216\\
2156.09999999999	24.8678230219831\\
2156.49999999999	24.8817583232867\\
2156.80000000001	24.8782975566786\\
2157	24.8852626211117\\
2157.2	24.8875909701942\\
2157.80000000001	24.8806710227536\\
2157.89999999999	24.8841519925857\\
2159.30000000002	24.8680188941373\\
2159.4	24.8714980319518\\
2159.60000000002	24.8691947955735\\
2159.7	24.8726733956848\\
2160	24.8692190176381\\
2160.09999999999	24.8726969725025\\
2160.29999999999	24.8703943714127\\
2160.49999999998	24.8773488845691\\
2160.70000000001	24.8750462791559\\
2160.8	24.8785228377065\\
2160.99999999999	24.8808477164407\\
2161.39999999999	24.8762433495258\\
2161.50000000001	24.879718726869\\
2161.7	24.877416967342\\
2161.79999999999	24.8808918081317\\
2162.5	24.8728382502543\\
2162.7	24.8797854632883\\
2162.89999999999	24.8774849745724\\
2163.1	24.8844304733728\\
2163.40000000001	24.8809798936908\\
2163.60000000001	24.8879234644359\\
2163.99999999999	24.8833233214731\\
2164.10000000001	24.8867941964698\\
2164.30000000001	24.889114766217\\
2164.6	24.8856654501779\\
2164.69999999999	24.8891352549889\\
2165	24.8856865733684\\
2165.39999999998	24.8995613022397\\
2165.9	24.8938134810711\\
2166	24.8972808272933\\
2166.7	24.8892375853794\\
2166.89999999999	24.8961698200277\\
2167.8	24.8858342174455\\
2168.10000000002	24.8962272853058\\
2168.40000000001	24.8927830297441\\
2168.50000000002	24.8962464262658\\
2168.90000000001	24.9054863992623\\
2169.09999999999	24.9078001106399\\
2169.60000000001	24.9020601926534\\
2169.7	24.9055212461978\\
2169.90000000001	24.9078341013825\\
2170.10000000001	24.9101465302737\\
2170.5	24.9055560674468\\
2170.6	24.9090155249459\\
2170.79999999999	24.9067207149109\\
2170.89999999998	24.9101796407186\\
2171.19999999999	24.9159489706627\\
2171.39999999999	24.9136541561133\\
2171.49999999999	24.9171118069626\\
2171.69999999999	24.9148172023207\\
2171.80000000001	24.9182743220222\\
2171.99999999999	24.9205837668616\\
2172.4	24.9159953970081\\
2172.5	24.9194513486146\\
2172.7	24.9217599410898\\
2172.9	24.9194661757938\\
2173.00000000001	24.9229211725185\\
2173.4	24.9183344835519\\
2173.59999999998	24.925242673782\\
2174.6	24.9137812112015\\
2174.70000000001	24.9172337686224\\
2175.2	24.9115064588792\\
2175.30000000001	24.9149581686127\\
2175.7	24.9103777920765\\
2175.8	24.9138287605129\\
2176.3	24.9081051277339\\
2176.40000000001	24.9115552492534\\
2176.6	24.909266320577\\
2176.79999999999	24.9161651890303\\
2177.2	24.9115877462913\\
2177.29999999999	24.915036281804\\
2178.20000000001	24.9047422301795\\
2178.30000000001	24.9081894968784\\
2178.50000000001	24.9059028734049\\
2178.59999999999	24.909349612154\\
2179.99999999999	24.8933535158938\\
2180.2	24.9002430858139\\
2180.99999999999	24.8911099903718\\
2181.09999999999	24.8945534568128\\
2181.30000000001	24.8922710186119\\
2181.40000000001	24.8957139582856\\
2182.30000000001	24.8854472140762\\
2182.40000000001	24.8888888888889\\
2182.60000000001	24.8911898107848\\
2182.9	24.8969308291342\\
2183.30000000002	24.8923696986352\\
2183.40000000001	24.8958094801924\\
2183.6	24.8935293309521\\
2183.70000000002	24.8969685868669\\
2183.9	24.8946886446886\\
2183.99999999998	24.8981273751202\\
2184.29999999999	24.9038637612159\\
2184.59999999999	24.9004439968874\\
2184.69999999998	24.9038813621384\\
2185.3	24.8970440194015\\
2185.40000000001	24.9004804392588\\
2185.59999999999	24.8982019490323\\
2185.70000000001	24.9016378442675\\
2186.1	24.897081694264\\
2186.3	24.903951701427\\
2187.10000000002	24.8948427212875\\
2187.2	24.8982764138436\\
2187.5	24.9040043883708\\
2188.20000000001	24.8960380203811\\
2188.39999999999	24.9029015307288\\
2188.70000000002	24.9086257309942\\
2188.90000000002	24.9109182275011\\
2189.20000000001	24.9075046818618\\
2189.3	24.9109345026035\\
2189.5	24.9086591158202\\
2189.60000000001	24.912088413938\\
2189.8	24.9143796520389\\
2190.2	24.9098297036936\\
2190.4	24.916685688199\\
2190.6	24.9144109188844\\
2190.8	24.9212652334657\\
2191.10000000001	24.9178532311062\\
2191.19999999999	24.9212796057135\\
2191.70000000001	24.9155944885482\\
2191.90000000001	24.9224452554745\\
2192.60000000001	24.9144889861814\\
2192.7	24.9179131703758\\
2192.89999999999	24.9156406748746\\
2192.99999999999	24.9190643381515\\
2193.40000000002	24.9145201732391\\
2193.60000000001	24.9213657291334\\
2193.90000000002	24.9179580674567\\
2194	24.921380064719\\
2194.2	24.9236658615504\\
2194.80000000001	24.9168527039956\\
2194.90000000001	24.9202733485194\\
2195.09999999999	24.9180029154519\\
2195.2	24.9214230401312\\
2195.4	24.9237075836939\\
2195.59999999999	24.9259917110716\\
2195.80000000001	24.9237214809418\\
2196.20000000001	24.9373947092838\\
2196.59999999998	24.9465106751036\\
2196.99999999998	24.9419689590824\\
2197.10000000001	24.9453850354997\\
2197.59999999999	24.9397096965009\\
2197.7	24.9431249431249\\
2197.90000000001	24.9454049135578\\
2198.29999999999	24.940866084425\\
2198.49999999999	24.9476939870827\\
2199.3	24.9386196235337\\
2199.4	24.9420322800636\\
2199.59999999999	24.9443105878074\\
2200.09999999998	24.9386419416417\\
2200.20000000001	24.9420533563605\\
2200.59999999998	24.9375198800382\\
2200.70000000001	24.9409305707016\\
2201.00000000001	24.9466176002908\\
2201.40000000001	24.9420849420849\\
2201.50000000001	24.9454941860465\\
2201.89999999999	24.9409627611263\\
2202.29999999999	24.9545949872866\\
2203.00000000001	24.9466660614589\\
2203.09999999999	24.9500726216412\\
2203.29999999998	24.952346373786\\
2203.6	24.9489494940328\\
2203.7	24.9523550231418\\
2205	24.9376445512675\\
2205.10000000001	24.9410484309813\\
2205.39999999999	24.9376558603491\\
2205.5	24.9410591222343\\
2205.79999999999	24.9376671653293\\
2205.9	24.9410698096102\\
2206.10000000001	24.9388088115311\\
2206.19999999999	24.9422109413951\\
2207.19999999999	24.9309110678204\\
2207.39999999999	24.9377123442809\\
2207.79999999999	24.9331944381539\\
2207.9	24.9365942028986\\
2208.1	24.9388642333122\\
2209.10000000002	24.9275755929748\\
2209.30000000002	24.934371322531\\
2209.89999999999	24.9276018099548\\
2210.09999999999	24.9343950773686\\
2210.79999999999	24.9265005201502\\
2210.9	24.9298959746721\\
2211.79999999999	24.9197522491975\\
2212.09999999999	24.9299340023506\\
2213.50000000001	24.9141669678352\\
2213.6	24.9175588381443\\
2213.89999999999	24.9141824751581\\
2214.00000000001	24.9175737319904\\
2214.80000000001	24.9085737505079\\
2215	24.9153537086362\\
2215.6	24.9086067608431\\
2215.70000000001	24.911995667479\\
2215.9	24.9097472924188\\
2216	24.9131356888227\\
2216.2	24.9108875152281\\
2216.29999999999	24.9142754015521\\
2217.5	24.9007936507937\\
2217.59999999999	24.9041800063128\\
2218.30000000001	24.896321673278\\
2218.6	24.9064767656736\\
2218.79999999999	24.9087385641534\\
2219.40000000002	24.9020049560712\\
2219.49999999999	24.9053883582628\\
2220.5	24.8941727461047\\
2220.6	24.8975548250552\\
2221.20000000001	24.8908296943231\\
2221.29999999999	24.8942108580175\\
2221.8	24.888608848283\\
2222	24.8953692453085\\
2222.2	24.8931287404941\\
2222.30000000001	24.8965082793377\\
2222.50000000001	24.8987672095744\\
2223.6	24.8864505104106\\
2223.70000000001	24.8898282219624\\
2223.90000000001	24.8920863309352\\
2224.19999999999	24.8887290383491\\
2224.39999999999	24.8954821308159\\
};
\addplot [color=mycolor1, forget plot]
  table[row sep=crcr]{%
2224.39999999999	24.8954821308159\\
2224.70000000002	24.8921251348436\\
2224.80000000001	24.8955009213897\\
2224.99999999999	24.8932632241248\\
2225.10000000002	24.8966385044041\\
2225.3	24.8944010065606\\
2225.4	24.8977757807234\\
2225.69999999999	24.894419983826\\
2225.80000000001	24.8977941506806\\
2226	24.900049413773\\
2227.09999999999	24.8877514367816\\
2227.19999999999	24.8911237821578\\
2227.6	24.8866543969116\\
2227.69999999999	24.890026034653\\
2228.00000000002	24.8956510030968\\
2228.2	24.8979042319257\\
2229.20000000001	24.8867357466469\\
2229.3	24.890104960976\\
2229.60000000001	24.8867560658385\\
2229.70000000002	24.8901246748587\\
2230.69999999999	24.8789671866595\\
2230.79999999999	24.8823344838406\\
2231.29999999999	24.8767589853903\\
2231.5	24.883491665173\\
2232	24.877917656019\\
2232.1	24.8812830391542\\
2232.29999999999	24.8835334169504\\
2232.49999999998	24.8857833915614\\
2233.10000000001	24.8790972595379\\
2233.19999999999	24.8824609322527\\
2233.4	24.8847100962615\\
2234.00000000002	24.8780269459738\\
2234.20000000002	24.88475137627\\
2234.39999999998	24.8869993287089\\
2234.7	24.8926078396277\\
2235.29999999999	24.8859264561152\\
2235.39999999999	24.8892865130843\\
2235.60000000001	24.8915328532451\\
2235.90000000002	24.8881932021467\\
2236	24.8915522561603\\
2236.20000000001	24.8893261190359\\
2236.30000000001	24.892684671794\\
2236.59999999999	24.8893459113873\\
2236.69999999999	24.8927038626609\\
2236.9	24.8949485918641\\
2237.79999999999	24.88493677108\\
2237.89999999999	24.8882931188561\\
2238.30000000001	24.8838456040029\\
2238.4	24.8872012508376\\
2238.59999999999	24.8849778889534\\
2238.70000000001	24.8883330355548\\
2239.19999999999	24.8827758674586\\
2239.30000000001	24.88613021345\\
2239.90000000001	24.8794642857143\\
2240	24.8828177313513\\
2240.19999999999	24.8850600366022\\
2240.40000000001	24.8873019415309\\
2240.90000000001	24.8817492190986\\
2241.00000000001	24.8851010664406\\
2241.30000000002	24.8817703221201\\
2241.39999999999	24.885121570377\\
2241.60000000001	24.8829013694964\\
2241.70000000001	24.8862521188331\\
2241.9	24.8884924174844\\
2242.8	24.8785055062642\\
2242.89999999999	24.8818546589389\\
2243.20000000001	24.8785271697945\\
2243.3	24.881875724347\\
2243.6	24.8874626732629\\
2244.70000000002	24.8752672843906\\
2244.79999999998	24.8786137467148\\
2245.09999999999	24.8752895065028\\
2245.20000000001	24.8786353716653\\
2245.5	24.8753117206983\\
2245.70000000001	24.8820019592128\\
2246.09999999998	24.8775710088149\\
2246.19999999999	24.8809152829097\\
2248.2	24.8587821909887\\
2248.40000000002	24.865465866133\\
2248.59999999999	24.8677013385512\\
2248.80000000001	24.8699364133576\\
2249.00000000002	24.8677248677249\\
2249.10000000001	24.8710652676507\\
2249.4	24.8677483885308\\
2249.49999999998	24.8710881934566\\
2249.70000000001	24.8688772335319\\
2249.8	24.8722165429575\\
2250	24.870005777521\\
2250.09999999999	24.8733445915919\\
2250.6	24.8678189007864\\
2250.79999999998	24.8744946465858\\
2251.09999999998	24.8800639658849\\
2251.70000000001	24.8734345856648\\
2251.79999999999	24.8767707269417\\
2251.99999999999	24.8745615203588\\
2252.09999999999	24.8778971672143\\
2252.4	24.8834628190899\\
2253.2	24.8746283229042\\
2253.29999999999	24.8779621904677\\
2254	24.8702364580098\\
2254.09999999998	24.8735693372372\\
2254.39999999999	24.8791306276336\\
2254.60000000001	24.8769237592584\\
2254.79999999999	24.8835868552929\\
2255.2	24.8791735024165\\
2255.3	24.8825042121132\\
2255.50000000002	24.8847313353431\\
2255.79999999999	24.8902876900572\\
2256.80000000001	24.8792591607958\\
2256.89999999998	24.8825875055383\\
2257.09999999998	24.8848130427078\\
2257.40000000001	24.8815060908084\\
2257.50000000001	24.8848334514529\\
2257.7	24.8870581982461\\
2260.10000000002	24.8606318024954\\
2260.19999999999	24.8639561120205\\
2260.79999999999	24.8573576894157\\
2260.9	24.8606811145511\\
2262.40000000001	24.8441988950276\\
2262.49999999999	24.8475205515778\\
2262.8	24.8530646515533\\
2263.29999999999	24.8475744455244\\
2263.40000000002	24.8508946322068\\
2263.59999999999	24.8486990325573\\
2263.90000000002	24.8586572438163\\
2264.50000000001	24.8520710059172\\
2264.59999999999	24.8553892347772\\
2264.80000000001	24.8576096074882\\
2265.50000000002	24.8499293785311\\
2265.69999999999	24.8565628034248\\
2265.99999999999	24.8532721415648\\
2266.19999999998	24.8599038079689\\
2266.90000000001	24.8522276135862\\
2266.99999999999	24.8555423227912\\
2267.20000000001	24.857760331672\\
2267.39999999999	24.8599779492834\\
2267.70000000001	24.8655084222595\\
2268.30000000002	24.8589314053959\\
2268.40000000001	24.8622437734186\\
2268.70000000001	24.8589562764457\\
2268.79999999999	24.8622680594121\\
2269.39999999998	24.8556950870236\\
2269.59999999999	24.8623166057188\\
2270.60000000001	24.8513674197384\\
2270.70000000002	24.8546767658975\\
2270.9	24.8568912373404\\
2271.7	24.8481380403204\\
2271.8	24.8514459263172\\
2272.3	24.8459778208062\\
2272.4	24.8492849284928\\
2272.70000000001	24.8548046462513\\
2273.00000000001	24.8515243500066\\
2273.1	24.8548301953194\\
2273.30000000001	24.8570423154746\\
2273.5	24.8548557353976\\
2273.59999999999	24.8581607072173\\
2274.59999999999	24.8472326021014\\
2274.80000000001	24.8538397292189\\
2275.19999999999	24.8494703995078\\
2275.30000000001	24.8527731387888\\
2275.6	24.8494968581096\\
2275.80000000001	24.8561008831671\\
2276.2	24.8517330756051\\
2276.50000000001	24.8616357726434\\
2276.99999999999	24.8561767159984\\
2277.19999999999	24.8627760944979\\
2277.40000000001	24.860592755214\\
2277.50000000002	24.8638918159466\\
2277.99999999999	24.8584346604627\\
2278.20000000002	24.865030944125\\
2279.89999999999	24.8464912280702\\
2280.00000000002	24.8497872900311\\
2280.20000000001	24.8476077709073\\
2280.29999999999	24.8509033502894\\
2280.69999999999	24.8465450719046\\
2280.80000000001	24.8498399754483\\
2281.2	24.8454828387323\\
2281.3	24.8487770667134\\
2281.80000000001	24.8433323107936\\
2282	24.8499189343149\\
2282.2	24.8521228585199\\
2283.50000000002	24.8379751269925\\
2283.59999999998	24.841266365985\\
2284.29999999999	24.833654351252\\
2284.59999999999	24.8435243139143\\
2286.10000000002	24.8272242148543\\
2286.20000000002	24.8305121812536\\
2286.6	24.8392880570254\\
2287.20000000001	24.8327722642417\\
2287.30000000002	24.8360584069249\\
2287.69999999999	24.8317160590961\\
2287.8	24.8350015297871\\
2288.6	24.8263206186918\\
2288.69999999999	24.8296050332052\\
2288.99999999999	24.8263509676292\\
2289.09999999999	24.8296348069195\\
2289.5	24.8252969951083\\
2289.6	24.8285801633402\\
2289.89999999999	24.825327510917\\
2290.3	24.838456164862\\
2290.50000000001	24.8406531039902\\
2290.89999999998	24.8363160192056\\
2291.00000000001	24.839596700275\\
2291.2	24.8417928686772\\
2291.4	24.8396246999782\\
2291.5	24.8429045208588\\
2291.7	24.840736538965\\
2291.79999999998	24.8440158820193\\
2292.10000000001	24.8407643312102\\
2292.49999999999	24.8538776934485\\
2292.69999999998	24.856071179344\\
2293.19999999999	24.8506518990102\\
2293.30000000001	24.8539286648644\\
2293.70000000001	24.8495945592467\\
2293.8	24.8528706569598\\
2294.10000000002	24.8496207828437\\
2294.19999999999	24.8528963082422\\
2294.4	24.8550882545217\\
2295.29999999999	24.8453428596323\\
2295.4	24.8486168590721\\
2295.6	24.8464520625517\\
2295.79999999998	24.8529988239906\\
2296.10000000002	24.8584618064628\\
2296.40000000002	24.8639233616373\\
2296.7	24.8606757227447\\
2296.90000000002	24.867218110579\\
2297.2	24.8639707482697\\
2297.30000000001	24.8672412292156\\
2297.59999999999	24.8639944292118\\
2297.7	24.8672643398033\\
2298.00000000002	24.8640181019103\\
2298.09999999998	24.8672874423462\\
2298.40000000001	24.8640417663694\\
2298.60000000001	24.870579022926\\
2298.8	24.8684153290704\\
2298.9	24.8716833405829\\
2299.5	24.8651939467734\\
2299.69999999999	24.8717279763458\\
2301.30000000001	24.8544364299991\\
2301.39999999998	24.8577014990224\\
2301.9	24.8523023457863\\
2302.09999999999	24.8588306836939\\
2302.59999999999	24.8534329265645\\
2302.70000000002	24.8566961959354\\
2303.3	24.850221411826\\
2303.4	24.8534838289559\\
2303.7	24.8502474173105\\
2303.80000000001	24.8535092668953\\
2304	24.8513519378499\\
2304.09999999999	24.8546133148164\\
2304.3	24.8567956951918\\
2304.49999999999	24.8589776967804\\
2305.4	24.8492734764693\\
2305.60000000001	24.8557921672377\\
2306.3	24.8482483524107\\
2306.39999999999	24.8515066117494\\
2306.80000000002	24.8471975378213\\
2306.9	24.850455136541\\
2307.20000000001	24.8472240280848\\
2307.29999999999	24.8504810609344\\
2307.89999999999	24.844020797227\\
2308.10000000001	24.8505328827658\\
2308.90000000001	24.8419229103508\\
2309.00000000001	24.8451777748907\\
2309.19999999999	24.8473563417486\\
2309.40000000002	24.8495345312838\\
2309.6	24.8517123435944\\
2309.79999999999	24.8495605870384\\
2309.99999999998	24.8560668369335\\
2311.89999999999	24.8356401384083\\
2312.09999999999	24.8421416832454\\
2312.40000000001	24.8475675675676\\
2312.60000000002	24.8454187745925\\
2312.7	24.8486682808717\\
2312.90000000002	24.8508430609598\\
2313.3	24.846546209043\\
2313.49999999999	24.8530428769018\\
2313.7	24.8552165269254\\
2313.89999999999	24.85738980121\\
2314.09999999999	24.8552415521563\\
2314.20000000001	24.8584885278486\\
2314.40000000001	24.8563404623029\\
2314.5	24.8595869696708\\
2314.70000000001	24.8574390876102\\
2314.8	24.8606851267873\\
2315.00000000001	24.8585374281889\\
2315.09999999999	24.8617829993089\\
2315.60000000001	24.8564149069396\\
2315.7	24.8596597288194\\
2316.10000000002	24.8553665486573\\
2316.29999999998	24.8618546019686\\
2316.9	24.8554164868364\\
2317.00000000001	24.8586595313107\\
2318.4	24.8436489109338\\
2318.60000000001	24.8501315392246\\
2318.99999999999	24.8458453710491\\
2319.29999999999	24.8555660946797\\
2319.9	24.8491379310345\\
2320.10000000002	24.8556158951815\\
2320.30000000001	24.853473539045\\
2320.40000000002	24.8567119155354\\
2321.1	24.8492159227986\\
2321.2	24.852453366648\\
2321.8	24.8460312674964\\
2322.10000000001	24.8557402463181\\
2322.50000000001	24.8514595711702\\
2322.59999999999	24.8546949670642\\
2323.00000000001	24.8504153932246\\
2323.1	24.853650137741\\
2323.80000000002	24.8461637764104\\
2323.90000000001	24.8493975903614\\
2324.40000000002	24.8440524844053\\
2324.49999999999	24.847285554504\\
2324.70000000001	24.8494494150034\\
2324.90000000002	24.8516129032258\\
2325.10000000001	24.8494753139515\\
2325.30000000001	24.8559387632235\\
2325.5	24.8538011695906\\
2325.59999999999	24.8570322913531\\
2326.39999999999	24.8484848484849\\
2326.49999999999	24.8517149488524\\
2327.10000000001	24.8453076658646\\
2327.19999999998	24.8485369312078\\
2327.39999999999	24.8506981740064\\
2328.50000000001	24.8389590311775\\
2328.7	24.8454139470972\\
2328.99999999999	24.8508007384827\\
2329.20000000001	24.8529601167733\\
2329.4	24.8551191242756\\
2329.79999999999	24.8508519678956\\
2329.89999999999	24.8540772532189\\
2330.40000000001	24.8487449045269\\
2330.5	24.8519694499271\\
2330.70000000002	24.854127338253\\
2331	24.8595083865986\\
2331.2	24.8616651653584\\
2332.10000000001	24.8520710059172\\
2332.2	24.8552930583544\\
2332.6	24.851030994127\\
2332.7	24.8542524005487\\
2332.89999999998	24.852121731676\\
2333.1	24.8585633464769\\
2333.3	24.8564326733522\\
2333.39999999999	24.859652881937\\
2333.60000000001	24.8575223893388\\
2333.7	24.8607421372868\\
2333.9	24.8586118251928\\
2333.99999999999	24.8618311126344\\
2334.59999999999	24.8554418126526\\
2334.7	24.858660270687\\
2335.5	24.8501455728721\\
2335.60000000001	24.8533630175108\\
2336.10000000001	24.8651656536255\\
2336.40000000001	24.8619730365932\\
2336.59999999999	24.8684041597124\\
2336.80000000001	24.8705550087723\\
2337.4	24.8641711229947\\
2337.49999999999	24.8673853524983\\
2338.00000000001	24.8620674906976\\
2338.10000000001	24.8652809853734\\
2338.39999999998	24.8620910840282\\
2338.70000000001	24.871729091842\\
2339.00000000001	24.8685391817366\\
2339.19999999999	24.8749625956483\\
2339.40000000002	24.8771104936952\\
2339.79999999999	24.8728578144365\\
2339.89999999999	24.8760683760684\\
2340.30000000001	24.8718167834558\\
2340.39999999999	24.8750267036958\\
2340.60000000001	24.8729012688512\\
2340.70000000002	24.8761107313739\\
2341.19999999999	24.8707982744629\\
2341.50000000002	24.8804236419542\\
2341.80000000002	24.877236431957\\
2341.89999999998	24.8804440649018\\
2342.09999999999	24.8825890188711\\
2342.29999999999	24.8847336065574\\
2342.80000000001	24.8794229373853\\
2342.9	24.8826291079812\\
2343.10000000002	24.8805052919085\\
2343.5	24.8933265062297\\
2343.70000000001	24.8912023210172\\
2343.89999999999	24.8976109215017\\
2344.20000000001	24.8944247749861\\
2344.29999999999	24.8976283910595\\
2344.59999999999	24.8944427858575\\
2344.69999999999	24.8976458546571\\
2345.40000000001	24.8902153059049\\
2345.5	24.893417462483\\
2346.50000000002	24.8828091707151\\
2346.60000000001	24.8860101419014\\
2346.80000000001	24.8881503259619\\
2347.00000000002	24.8902901452857\\
2347.2	24.8924295999659\\
2347.39999999999	24.8945686900959\\
2347.60000000002	24.8924479277591\\
2347.69999999998	24.8956469886702\\
2347.9	24.8935264054515\\
2348.09999999999	24.8999233455413\\
2348.30000000001	24.9020609776869\\
2349.09999999999	24.8935807934616\\
2349.19999999999	24.8967777635892\\
2349.8	24.8904208689732\\
2349.89999999999	24.8936170212766\\
2350.70000000001	24.885145482389\\
2350.8	24.8883406355013\\
2351.29999999999	24.8830483966998\\
2351.4	24.8862428237295\\
2352.50000000001	24.8746068179886\\
2352.7	24.8809928595716\\
2353.00000000001	24.8778207470996\\
2353.1	24.8810130885603\\
2353.59999999999	24.8757275778561\\
2353.8	24.882110539955\\
2354.60000000001	24.8736569414363\\
2354.80000000002	24.8800373688904\\
2356.2	24.8652548487035\\
2356.30000000001	24.8684433882193\\
2356.6	24.8737641617516\\
2357.50000000001	24.8642687478792\\
2357.70000000002	24.8706421240139\\
2358.29999999999	24.8643147896879\\
2358.40000000001	24.8675005299979\\
2359.60000000001	24.854854430648\\
2359.70000000001	24.8580388168489\\
2360.09999999999	24.8538259469536\\
2360.19999999999	24.8570097021565\\
2360.39999999999	24.8591400127092\\
2360.79999999999	24.8549282053454\\
2361.19999999999	24.8676576462118\\
2362.1	24.8581830496994\\
2362.4	24.8677248677249\\
2363.29999999999	24.8582550562749\\
2363.4	24.8614343135181\\
2363.69999999999	24.8582790422202\\
2363.80000000001	24.8614577604806\\
2364.50000000001	24.8540979446841\\
2364.59999999999	24.8572757643676\\
2364.79999999999	24.8594020888833\\
2365.70000000001	24.8499450503001\\
2365.9	24.8562975486052\\
2366.2	24.8615982757892\\
2366.6	24.857396374699\\
2366.7	24.8605712354234\\
2366.89999999999	24.8584706379383\\
2367.1	24.8648191956742\\
2367.3	24.8627185942384\\
2367.40000000001	24.8658922914467\\
2367.9	24.8606418918919\\
2368	24.8638148726827\\
2368.3	24.8606654281371\\
2368.40000000001	24.8638378720709\\
2368.70000000001	24.8606889564336\\
2368.90000000002	24.8670325031659\\
2369.40000000001	24.8617851867483\\
2369.5	24.864956110736\\
2369.9	24.8607594936709\\
2370.10000000001	24.8670998227998\\
2371.10000000001	24.8566126855601\\
2371.3	24.862950156026\\
2371.5	24.8650699949401\\
2371.89999999999	24.8608768971332\\
2372.00000000002	24.8640445175161\\
2372.39999999999	24.8598524762908\\
2372.6	24.866186201374\\
2373.10000000002	24.8609472442272\\
2373.19999999999	24.8641132600177\\
2374.00000000002	24.8557348047681\\
2374.09999999999	24.8588998399461\\
2374.60000000001	24.853665726197\\
2374.70000000001	24.8568300488462\\
2374.9	24.8589473684211\\
2375.10000000001	24.8568541596497\\
2375.20000000002	24.860017681977\\
2375.49999999999	24.8568782623337\\
2375.70000000001	24.8632039733984\\
2376.49999999998	24.8548346377178\\
2376.60000000001	24.8579963815374\\
2376.8	24.8601119104716\\
2376.99999999999	24.8580202768079\\
2377.10000000001	24.8611812216052\\
2377.29999999999	24.8590897619248\\
2377.5	24.8654104979812\\
2377.7	24.8633190344015\\
2378	24.8727976115386\\
2378.30000000001	24.8696602758157\\
2378.39999999998	24.8728190035737\\
2378.89999999998	24.8675914249685\\
2379	24.8707494430667\\
2379.40000000001	24.8665686068502\\
2379.5	24.8697260043705\\
2379.8	24.8749947476785\\
2380.3	24.8697697865905\\
2380.40000000002	24.8729258559126\\
2380.70000000002	24.8697916666667\\
2380.8	24.8729472048385\\
2381.00000000001	24.8708580068036\\
2381.1	24.8740131026373\\
2381.69999999999	24.8677470820388\\
2381.80000000001	24.8709013812503\\
2382.00000000001	24.8730112085975\\
2382.70000000001	24.8657042135303\\
2382.8	24.8688572747493\\
2383.00000000002	24.8667701733037\\
2383.20000000001	24.8730751479042\\
2383.89999999998	24.8657718120805\\
2384	24.8689232834193\\
2384.20000000001	24.8710313299501\\
2384.39999999999	24.8731390228559\\
2384.60000000002	24.8752463622259\\
2384.80000000001	24.8773533481488\\
2385.20000000002	24.8731815704524\\
2385.29999999999	24.8763310136665\\
2385.50000000001	24.874245472837\\
2385.60000000002	24.877394475416\\
2386.69999999999	24.865929277694\\
2386.8	24.8690770455402\\
2387.00000000001	24.8669934229819\\
2387.09999999999	24.8701407506702\\
2387.4	24.8670157068063\\
2387.50000000001	24.8701625062825\\
2387.70000000001	24.8722673590753\\
2388.29999999999	24.8660190922794\\
2388.4	24.8691647477496\\
2389.10000000001	24.8618784530387\\
2389.30000000002	24.8681677408554\\
2390.9	24.8515265579256\\
2391.00000000001	24.8546693990214\\
2391.39999999999	24.8505122308175\\
2391.5	24.8536544572671\\
2391.70000000001	24.8515762187474\\
2391.9	24.8578595317726\\
2392.20000000001	24.8547422982068\\
2392.30000000001	24.8578832971075\\
2392.50000000001	24.8558053999833\\
2392.60000000001	24.8589459606303\\
2393.6	24.8485608054476\\
2393.8	24.8548393834329\\
2393.99999999998	24.8527630424794\\
2394.10000000001	24.8559017625929\\
2394.9	24.8475991649269\\
2394.99999999999	24.8507369212141\\
2395.3	24.8476246138432\\
2395.40000000001	24.8507618451263\\
2395.7	24.8476500542616\\
2395.79999999999	24.8507867607162\\
2395.99999999999	24.8487124911314\\
2396.1	24.8518487605375\\
2396.30000000002	24.8539475880487\\
2396.80000000001	24.848762985523\\
2396.89999999999	24.851898206091\\
2397.20000000002	24.8487882200809\\
2397.30000000001	24.8519229164929\\
2398.00000000001	24.844668696051\\
2398.09999999999	24.8478025185556\\
2398.39999999999	24.8530331457161\\
2398.60000000002	24.8509609371743\\
2398.8	24.8572262286881\\
2399.10000000001	24.8541180393465\\
2399.30000000001	24.8603817621072\\
2401.49999999998	24.8376082611592\\
2401.6	24.8407378107174\\
2403.29999999998	24.8231671798286\\
2403.4	24.8262949864781\\
2403.6	24.8283895660856\\
2404.09999999999	24.8232260211297\\
2404.20000000002	24.8263527845943\\
2404.50000000001	24.8232554270981\\
2404.60000000001	24.8263816692311\\
2404.80000000001	24.8284751964739\\
2405.10000000002	24.825378346915\\
2405.19999999998	24.8285037209496\\
2406.40000000001	24.8161230002078\\
2406.49999999999	24.819247070556\\
2406.69999999999	24.8171846435101\\
2406.79999999999	24.8203082803606\\
2407.09999999998	24.8172150216019\\
2407.2	24.8203381381631\\
2407.59999999999	24.8162146446817\\
2407.7	24.8193371542487\\
2408.10000000002	24.8152146831659\\
2408.19999999999	24.8183365859735\\
2408.69999999999	24.813184988376\\
2409.09999999999	24.8256682716254\\
2409.69999999998	24.8194870943647\\
2409.89999999999	24.8257261410788\\
2410.59999999999	24.8185174430663\\
2410.70000000001	24.8216359714618\\
2410.89999999998	24.8237245956035\\
2411.2	24.8206361713599\\
2411.39999999998	24.8268712419656\\
2411.90000000002	24.8217247097844\\
2411.99999999999	24.8248414244849\\
2412.19999999999	24.8227832359159\\
2412.4	24.8290155440415\\
2412.99999999999	24.822841987485\\
2413.09999999998	24.8259572352064\\
2413.29999999999	24.8238998922682\\
2413.50000000001	24.8301292674843\\
2413.69999999998	24.8280719197945\\
2413.99999999999	24.8374135288513\\
2414.30000000001	24.8343273691186\\
2414.39999999998	24.8374404638642\\
2414.59999999999	24.8395245786226\\
2415.10000000001	24.8343822457768\\
2415.20000000001	24.8374943071254\\
2416.30000000001	24.8261877172654\\
2416.5	24.8324091699081\\
2416.7	24.8303541873552\\
2416.90000000001	24.8365742656185\\
2417.2	24.8334919124643\\
2417.29999999999	24.8366013071895\\
2418.00000000001	24.8294115214425\\
2418.10000000002	24.8325200562402\\
2418.39999999999	24.8294397353732\\
2418.50000000001	24.8325477548995\\
2418.70000000002	24.8346287415247\\
2419.30000000001	24.8284698685625\\
2419.50000000001	24.8346834187469\\
2419.69999999999	24.8326307959336\\
2419.79999999999	24.8357370139262\\
2420.29999999999	24.8306065113204\\
2420.40000000001	24.8337120429663\\
2420.6	24.8316602635601\\
2420.8	24.8378702135569\\
2421.00000000002	24.8358184296394\\
2421.09999999999	24.8389228481745\\
2421.30000000001	24.836871231519\\
2421.40000000002	24.8399752219699\\
2421.60000000001	24.842053103192\\
2422	24.8379505387887\\
2422.1	24.8410535876476\\
2422.49999999999	24.8369520350037\\
2422.70000000002	24.8431566782235\\
2423.09999999999	24.8514361175305\\
2424.00000000001	24.8422094798069\\
2424.2	24.8484098502661\\
2424.99999999998	24.8402127747309\\
2425.10000000001	24.8433118918027\\
2425.40000000001	24.8402391259534\\
2425.49999999998	24.8433377308707\\
2426.00000000001	24.8382177156754\\
2426.2	24.8444133042081\\
2426.80000000002	24.8382710453665\\
2426.99999999998	24.8444645873676\\
2427.9	24.835255354201\\
2428.00000000002	24.8383509740126\\
2428.3	24.8352824905287\\
2428.39999999999	24.8383775993412\\
2428.59999999999	24.836332194178\\
2428.7	24.8394268774704\\
2429	24.8363591453625\\
2429.10000000002	24.8394533179648\\
2429.70000000001	24.8538974401185\\
2430	24.8590592979713\\
2431.10000000002	24.8478117801909\\
2431.29999999999	24.8539935839434\\
2431.49999999998	24.8560618522783\\
2432.09999999999	24.8499301044322\\
2432.2	24.8530197755211\\
2432.49999999999	24.8499547808929\\
2432.59999999999	24.8530439429441\\
2432.8	24.8551111841835\\
2433.80000000002	24.8448991330786\\
2433.9	24.847986852917\\
2434.30000000002	24.8439040420638\\
2434.5	24.8500780415674\\
2434.80000000001	24.847016304571\\
2434.9	24.8501026694045\\
2435.10000000001	24.8521681997372\\
2435.30000000002	24.8542333908188\\
2435.60000000001	24.8593833394917\\
2436.19999999999	24.8532610926405\\
2436.30000000001	24.8563454276802\\
2436.49999999999	24.8543051793483\\
2436.60000000002	24.8573890918045\\
2436.99999999998	24.8533092610069\\
2437.1	24.8563925816511\\
2437.40000000001	24.8533333333333\\
2437.60000000002	24.8594987077983\\
2437.9	24.856439704676\\
2438	24.8595217587466\\
2438.29999999998	24.8564632545932\\
2438.40000000002	24.8595448021325\\
2438.60000000001	24.8575060483044\\
2438.70000000001	24.86058717402\\
2439.09999999999	24.8688094457199\\
2440.00000000002	24.859636900127\\
2440.20000000001	24.8657951891161\\
2440.40000000001	24.8637574267568\\
2440.49999999998	24.8668360239285\\
2440.7	24.8688954441167\\
2441.30000000001	24.8627836487261\\
2441.59999999999	24.8720153991072\\
2441.79999999998	24.8699782955895\\
2441.89999999999	24.8730548730549\\
2442.70000000001	24.8649091206812\\
2442.80000000001	24.867984772197\\
2443.20000000001	24.8639135595301\\
2443.40000000002	24.8700634335993\\
2443.60000000001	24.8721201456807\\
2443.80000000001	24.8741765211343\\
2443.99999999998	24.8721410744241\\
2444.09999999998	24.8752147942067\\
2444.50000000001	24.8711445635278\\
2444.80000000002	24.8803632050391\\
2445	24.8824178970185\\
2445.29999999998	24.8875439600883\\
2446.40000000001	24.8763539750664\\
2446.5	24.8794245074798\\
2446.69999999999	24.8814778486186\\
2446.9	24.8835308541071\\
2447.1	24.8814972213142\\
2447.19999999998	24.8845666653046\\
2448.30000000002	24.8733867015194\\
2448.49999999999	24.8795229927305\\
2448.80000000001	24.8764751521091\\
2448.89999999999	24.8795426704777\\
2449.1	24.8775110240078\\
2449.20000000001	24.8805781243621\\
2449.40000000001	24.8785466421719\\
2449.6	24.8846797567049\\
2449.79999999999	24.8867300706151\\
2450.30000000002	24.8816519751877\\
2450.40000000001	24.8847174046113\\
2450.90000000002	24.8796409628723\\
2450.99999999999	24.8827057239607\\
2451.60000000001	24.8766162254762\\
2451.70000000001	24.8796802349294\\
2451.89999999999	24.8817292006525\\
2452.40000000001	24.8766564729868\\
2452.49999999999	24.8797194813667\\
2452.70000000002	24.8817677756034\\
2453.3	24.8756827260129\\
2453.39999999999	24.8787446504993\\
2453.60000000001	24.8767167950442\\
2453.80000000002	24.8828395615143\\
2454.60000000001	24.8747301095857\\
2454.80000000001	24.8808505438103\\
2455	24.8788236731701\\
2455.10000000001	24.8818833496253\\
2455.29999999998	24.8839292986886\\
2456	24.8768372623265\\
2456.10000000001	24.8798957739598\\
2456.60000000001	24.8748320918305\\
2456.70000000001	24.8778899381309\\
2456.99999999998	24.8829921452118\\
2457.20000000002	24.8809669149066\\
2457.4	24.8870803662258\\
2457.70000000001	24.8840426397591\\
2457.8	24.8870987428292\\
2458.79999999999	24.8769775102688\\
2458.90000000001	24.8800325335502\\
2459.39999999999	24.874974588331\\
2459.49999999999	24.8780289477964\\
2459.70000000002	24.8760061793642\\
2459.79999999999	24.8790601243953\\
2460.1	24.8841557596943\\
2460.40000000002	24.881121723227\\
2460.6	24.887227211769\\
2461.30000000002	24.8801495084099\\
2461.50000000001	24.8862528436789\\
2461.7	24.8882931188561\\
2461.90000000002	24.8862713241267\\
2462.00000000001	24.8893221233906\\
2462.19999999998	24.8913617349632\\
2462.69999999999	24.8863082670132\\
2462.79999999999	24.8893580738154\\
2464.49999999999	24.8721902134221\\
2464.59999999999	24.875238365724\\
2465	24.8712019796357\\
2465.09999999999	24.8742495537887\\
2465.40000000002	24.8712228756844\\
2465.59999999998	24.8773167863081\\
2466.30000000001	24.8702562439183\\
2466.40000000002	24.873302250152\\
2467.1	24.8662451361868\\
2467.4	24.8753799392097\\
2467.59999999999	24.8733638610852\\
2467.69999999999	24.876408136802\\
2468.00000000001	24.8733843847494\\
2468.10000000002	24.876428166275\\
2469.3	24.8643395156718\\
2469.4	24.867382061146\\
2469.79999999999	24.863354791692\\
2470.10000000001	24.8724799611368\\
2470.60000000002	24.8674464726596\\
2470.7	24.8704872915655\\
2471.40000000002	24.8634432530852\\
2471.50000000002	24.8664832497168\\
2472.09999999999	24.8604481838039\\
2472.19999999999	24.8634874408446\\
2472.4	24.8655207280081\\
2472.80000000001	24.8614986453152\\
2472.89999999999	24.8645369995956\\
2473.70000000002	24.856496078907\\
2474.00000000002	24.8656076957277\\
2474.3	24.8625929518267\\
2474.4	24.8656294200849\\
2474.9	24.8606060606061\\
2474.99999999998	24.8636418730556\\
2475.19999999999	24.8656728477356\\
2475.39999999999	24.8636639062816\\
2475.5	24.866698982065\\
2477	24.8516410318518\\
2477.20000000001	24.8577079885359\\
2477.49999999998	24.8546980949306\\
2477.59999999998	24.8577309601647\\
2478.30000000001	24.8507101355713\\
2478.49999999998	24.8567739853143\\
2478.69999999999	24.8588026464418\\
2479.10000000001	24.8547918683446\\
2479.2	24.8578227725568\\
2479.80000000001	24.851808540667\\
2479.99999999999	24.8578686343293\\
2480.50000000001	24.8528581794727\\
2480.60000000001	24.8558874511227\\
2480.80000000001	24.8579144665242\\
2481.10000000001	24.8549089150411\\
2481.20000000001	24.8579373715391\\
2481.49999999998	24.8549323017408\\
2481.59999999999	24.8579602691703\\
2482.50000000001	24.8489486828325\\
2482.59999999998	24.8519756716478\\
2482.99999999999	24.8479722926986\\
2483.09999999999	24.8509987113402\\
2483.29999999999	24.8489973423532\\
2483.4	24.8520233541373\\
2483.8	24.8480212568944\\
2483.90000000002	24.8510466988728\\
2484.1	24.8530714113195\\
2484.3	24.8550957977781\\
2484.7	24.8510946555055\\
2484.8	24.8541188780233\\
2485.00000000002	24.852118627017\\
2485.1	24.8551424432641\\
2485.50000000001	24.8632121017058\\
2485.70000000002	24.8652345321426\\
2485.99999999999	24.8622340211576\\
2486.09999999998	24.865256214303\\
2486.4	24.8702996179369\\
2487.40000000001	24.8603015075377\\
2487.5	24.8633220775044\\
2488.30000000001	24.8553287252853\\
2488.49999999998	24.8613678373383\\
2490.1	24.8453939442615\\
2490.2	24.8484118379312\\
2490.39999999999	24.8464163822526\\
2490.5	24.8494338713563\\
2490.9	24.845443596949\\
2491	24.8484605194492\\
2492.60000000001	24.8325109319212\\
2492.70000000002	24.8355263157895\\
2493	24.83253780434\\
2493.1	24.8355527033531\\
2493.4	24.8325646681372\\
2493.60000000002	24.8385932550026\\
2493.90000000002	24.8356054530874\\
2494	24.8386191411732\\
2494.40000000001	24.8346361996392\\
2494.49999999999	24.8376493225367\\
2494.70000000001	24.8356581689915\\
2494.79999999999	24.8386708886128\\
2495.20000000002	24.8346892157256\\
2495.29999999999	24.8377013705218\\
2495.8	24.8487519532033\\
2495.99999999998	24.8467609470774\\
2496.1	24.8497716529124\\
2496.70000000002	24.843800064082\\
2496.79999999998	24.8468100444551\\
2497.5	24.8398462524023\\
2497.60000000002	24.8428554269928\\
2497.79999999999	24.8408663277153\\
2497.9	24.8438751000801\\
2498.19999999999	24.8408918064284\\
2498.29999999998	24.8439000960615\\
2499.39999999999	24.8329665933187\\
2499.50000000001	24.8359737558009\\
2499.69999999999	24.8379870389631\\
2500.10000000001	24.8340132789377\\
2500.20000000001	24.8370195576531\\
2500.39999999998	24.8390321935613\\
2500.60000000002	24.8370456272244\\
2500.80000000002	24.8430564996601\\
2501.99999999999	24.8311418408537\\
2502.1	24.8341459515626\\
2502.79999999999	24.8272004474809\\
2502.99999999998	24.8332068235388\\
2503.30000000001	24.8302308859951\\
2503.40000000002	24.8332334731376\\
2504	24.827283255461\\
2504.10000000001	24.8302851209967\\
2504.39999999999	24.8273108404871\\
2504.50000000001	24.8303122255051\\
2504.80000000001	24.8273384167033\\
2505.09999999999	24.8363404119432\\
2505.30000000001	24.8383491658019\\
2505.49999999999	24.8403575989783\\
2506.50000000002	24.8304476182877\\
2506.59999999999	24.8334463637452\\
2506.99999999999	24.8294842646883\\
2507.1	24.8324824505424\\
2507.30000000001	24.8305017149238\\
2507.49999999999	24.8364970489711\\
2508.20000000001	24.8534864250688\\
2508.39999999999	24.8515048833965\\
2508.7	24.8604910714286\\
2509.3	24.8545469036423\\
2509.40000000002	24.857541342897\\
2509.59999999999	24.8595449655337\\
2509.99999999999	24.8555834428907\\
2510.09999999999	24.8585770058163\\
2510.29999999999	24.860579987253\\
2510.5	24.8585995379591\\
2510.59999999999	24.8615923845939\\
2511.30000000001	24.8546627379151\\
2511.49999999999	24.860646599777\\
2511.69999999999	24.8626483000239\\
2512.20000000002	24.8577001154321\\
2512.39999999999	24.8636815920398\\
2512.9	24.858734580183\\
2513.10000000001	24.8647143084514\\
2514.39999999998	24.851859216544\\
2514.50000000001	24.8548476894934\\
2515.19999999999	24.8479306643343\\
2515.29999999999	24.8509183430071\\
2515.5	24.8489425981873\\
2515.7	24.8549169250338\\
2515.90000000002	24.8529411764706\\
2516.00000000001	24.8559278248082\\
2516.90000000002	24.8470401271355\\
2517.1	24.8530112823772\\
2517.30000000002	24.8510367839835\\
2517.40000000001	24.8540218470705\\
2518.00000000002	24.8480997577539\\
2518.2	24.854068220625\\
2518.4	24.8560651181259\\
2518.79999999999	24.8521179880106\\
2518.90000000002	24.8551012306471\\
2519.40000000001	24.8501686842627\\
2519.60000000002	24.8561336667064\\
2520	24.852188405222\\
2520.30000000002	24.8611331534677\\
2521.1	24.8532444867523\\
2521.2	24.8562249633126\\
2521.60000000002	24.8522821905857\\
2521.80000000002	24.8582418018161\\
2522.80000000001	24.8483887589679\\
2522.90000000001	24.8513674197384\\
2523.1	24.8533608116677\\
2523.5	24.8494214614043\\
2523.60000000001	24.852399255062\\
2524.40000000002	24.8445236680531\\
2524.49999999999	24.8475005941535\\
2524.79999999999	24.8445482989425\\
2525.00000000001	24.8505009702586\\
2525.19999999999	24.8524927731359\\
2525.39999999998	24.8505246485844\\
2525.5	24.8535001583782\\
2526	24.8485808162781\\
2526.09999999998	24.8515556963027\\
2526.29999999998	24.8535465484484\\
2526.9	24.8476454293629\\
2527	24.8506192869297\\
2527.19999999999	24.8486527123808\\
2527.30000000001	24.851626177099\\
2527.60000000001	24.8486766625786\\
2527.8	24.8546224138613\\
2527.99999999998	24.852656144931\\
2528.09999999999	24.8556285104027\\
2528.30000000001	24.8576174655909\\
2528.60000000002	24.8546684066912\\
2528.70000000002	24.8576399873458\\
2528.99999999999	24.8546913921949\\
2529.1	24.8576625019769\\
2529.30000000001	24.8556970032419\\
2529.39999999998	24.8586677208935\\
2529.60000000001	24.8567023757758\\
2529.70000000002	24.8596727013993\\
2530.49999999999	24.851813799099\\
2530.7	24.8577524893314\\
2531	24.8548062107384\\
2531.09999999999	24.8577749683944\\
2532.69999999998	24.8420720151611\\
2532.90000000002	24.8480063166206\\
2533.19999999999	24.8450637508388\\
2533.29999999999	24.8480303149917\\
2534.29999999998	24.838226010101\\
2534.39999999999	24.84119155652\\
2534.80000000001	24.837271687246\\
2534.89999999999	24.8402366863905\\
2535.1	24.8422215209845\\
2535.3	24.8442060424391\\
2535.50000000002	24.8422464111059\\
2535.60000000001	24.8452103955515\\
2535.79999999999	24.8471942899957\\
2536	24.8452348093529\\
2536.19999999999	24.8511611402437\\
2537.39999999998	24.8394088669951\\
2537.6	24.8453323875951\\
2538.09999999999	24.8404381057442\\
2538.30000000001	24.8463599117554\\
2538.80000000002	24.8414667769507\\
2539	24.8473868693632\\
2541.10000000001	24.8268534550606\\
2541.2	24.8298115137922\\
2541.60000000001	24.8259039225715\\
2541.69999999999	24.8288614367771\\
2542.10000000002	24.8249547635906\\
2542.2	24.8279117334697\\
2542.7	24.8230297310052\\
2542.79999999998	24.8259860788863\\
2542.99999999998	24.8279658684283\\
2543.50000000002	24.8230853907847\\
2543.6	24.8260408066989\\
2543.89999999999	24.8231132075472\\
2544	24.8260681576982\\
2544.49999999999	24.8211899709188\\
2544.59999999999	24.8241442999175\\
2544.79999999998	24.8261228339031\\
2545.20000000002	24.8340077790437\\
2545.60000000001	24.830105668382\\
2545.7	24.8330583706497\\
2546.49999999999	24.825257205686\\
2546.70000000002	24.8311606722161\\
2546.90000000001	24.829210836278\\
2547	24.832162066664\\
2547.90000000002	24.8233908948195\\
2547.99999999998	24.8263411954005\\
2549.89999999999	24.8078431372549\\
2550.00000000001	24.8107917336575\\
2550.40000000001	24.8186630072535\\
2550.60000000001	24.8206374720665\\
2550.90000000001	24.8177185417483\\
2550.99999999998	24.8206655952334\\
2551.90000000002	24.8119122257053\\
2552.00000000001	24.8148583519455\\
2552.20000000002	24.8168318771304\\
2552.39999999998	24.8148873653281\\
2552.50000000001	24.8178327979315\\
2552.7	24.8198057035412\\
2552.90000000002	24.8217783000392\\
2553.70000000001	24.8140026626987\\
2553.79999999998	24.8169466306433\\
2554.50000000001	24.8101464025679\\
2554.60000000002	24.8130895995616\\
2554.89999999998	24.8101761252446\\
2555.09999999998	24.8160613650595\\
2555.3	24.8141191202943\\
2555.39999999999	24.8170612404618\\
2555.59999999999	24.8190319677583\\
2556.39999999999	24.8112654019167\\
2556.70000000001	24.8200876095119\\
2556.90000000001	24.8220570981619\\
2557.10000000001	24.8240262787424\\
2557.30000000001	24.8259951513256\\
2557.70000000001	24.8221127531472\\
2557.79999999999	24.8250518003049\\
2558.1	24.8221405675866\\
2558.19999999999	24.8250791541258\\
2558.49999999998	24.8221683733292\\
2558.59999999999	24.8251064993942\\
2558.80000000001	24.8231662042284\\
2558.9	24.8261039468542\\
2559.09999999999	24.8241638011879\\
2559.20000000001	24.8271011604736\\
2559.4	24.8290681773784\\
2560.50000000002	24.818401937046\\
2560.60000000002	24.8213379154138\\
2561.09999999999	24.8164922692488\\
2561.19999999998	24.8194276344044\\
2561.79999999998	24.813614895195\\
2562.10000000002	24.8224182343299\\
2562.30000000002	24.8204807992507\\
2562.40000000001	24.8234146341463\\
2563.00000000002	24.8176036830401\\
2563.10000000002	24.8205368289638\\
2563.7	24.8147281379203\\
2563.80000000002	24.817660595187\\
2563.99999999998	24.8157248157248\\
2564.09999999998	24.8186568910381\\
2564.50000000002	24.8147859315293\\
2564.70000000001	24.8206487835309\\
2564.90000000002	24.8187134502924\\
2565.10000000002	24.824575081865\\
2565.39999999998	24.8216721886572\\
2565.50000000001	24.8246024321796\\
2565.7	24.8265648140931\\
2566.1	24.822695035461\\
2566.19999999998	24.825624439855\\
2566.70000000001	24.820788530466\\
2566.8	24.8237173243991\\
2567.5	24.8169496806356\\
2567.59999999999	24.8198777115707\\
2568.6	24.8102152839958\\
2568.69999999998	24.8131423232638\\
2568.89999999999	24.8112105877773\\
2569.10000000002	24.8170636774093\\
2569.30000000001	24.8151319374173\\
2569.39999999999	24.8180579879354\\
2570.60000000001	24.8064729451122\\
2570.70000000001	24.8093978528085\\
2570.90000000001	24.8113574484636\\
2572.10000000002	24.799782287536\\
2572.20000000001	24.8027057497182\\
2572.89999999999	24.795958025651\\
2573.10000000001	24.8018032022385\\
2574.20000000001	24.7912053762188\\
2574.3	24.7941267868241\\
2574.59999999998	24.7912378141143\\
2574.8	24.7970794982329\\
2575.10000000001	24.7941907424666\\
2575.39999999998	24.8029508833236\\
2575.60000000002	24.8049074038126\\
2575.8	24.8068636204822\\
2576.09999999999	24.8039748466734\\
2576.20000000002	24.806893607111\\
2576.6	24.8030426514534\\
2576.7	24.8059608817138\\
2578.00000000002	24.7934525425701\\
2578.1	24.7963695601582\\
2578.6	24.7915616395858\\
2578.7	24.794478051807\\
2579.50000000001	24.786788649403\\
2579.6	24.7897042291739\\
2579.90000000002	24.7868217054264\\
2580.00000000002	24.7897368319057\\
2580.2	24.7878153703058\\
2580.49999999999	24.7965589397814\\
2580.7	24.7946373217607\\
2580.9	24.8004649360713\\
2581.10000000002	24.8024174802418\\
2581.29999999999	24.8004958549624\\
2581.39999999999	24.8034088708115\\
2582.2	24.7957247415095\\
2582.3	24.7986369268897\\
2582.5	24.8005885541702\\
2582.8	24.7977080026327\\
2583.00000000001	24.8035306414773\\
2583.30000000001	24.8006503057986\\
2583.69999999998	24.8122919730629\\
2583.99999999999	24.8094114004876\\
2584.09999999999	24.8123210277842\\
2584.49999999999	24.8084810028631\\
2584.59999999998	24.8113901033002\\
2585.3	24.8046723911194\\
2585.39999999998	24.8075807387353\\
2585.60000000002	24.8095293344162\\
2586.10000000002	24.8047328126208\\
2586.20000000001	24.807640258284\\
2586.40000000002	24.8095882466654\\
2587.00000000002	24.803834409184\\
2587.19999999998	24.8096471224829\\
2587.5	24.8067707528212\\
2587.70000000002	24.8125821160832\\
2587.9	24.8145285935085\\
2588.4	24.8097353679737\\
2588.59999999999	24.8155444817862\\
2589.00000000002	24.8117106330385\\
2589.10000000002	24.8146145527576\\
2591.89999999998	24.7878086419753\\
2592.00000000001	24.7907102349446\\
2592.69999999999	24.7840172786177\\
2592.80000000001	24.7869181225655\\
2593.5	24.7802282541641\\
2593.59999999998	24.7831283494622\\
2594.2	24.777396600239\\
2594.3	24.7802960222017\\
2594.6	24.777430916869\\
2594.69999999999	24.7803298905503\\
2595.10000000002	24.7765104808878\\
2595.20000000001	24.7794089315301\\
2595.40000000001	24.7774995183972\\
2595.49999999998	24.7803975959316\\
2595.90000000002	24.7881355932203\\
2596.1	24.7900778060242\\
2596.29999999999	24.7920197196118\\
2596.59999999998	24.7968575499673\\
2596.8	24.7987985675228\\
2597	24.7968888375496\\
2597.19999999999	24.8026797058484\\
2597.39999999998	24.8046198267565\\
2598	24.7988914976329\\
2598.09999999998	24.8017858517435\\
2598.50000000002	24.797968136689\\
2598.6	24.8008619694463\\
2598.79999999999	24.8028011851168\\
2599.00000000001	24.800892616675\\
2599.09999999998	24.8037857802401\\
2599.60000000001	24.7990152709928\\
2599.70000000002	24.8019078390645\\
2600.59999999999	24.7933248740724\\
2600.70000000001	24.7962165487542\\
2600.89999999998	24.7943098808151\\
2600.99999999998	24.7972011841144\\
2601.40000000001	24.7933884297521\\
2601.50000000002	24.7962792127921\\
2601.70000000002	24.7982166192636\\
2602.3	24.7924992314786\\
2602.4	24.7953890489914\\
2603.19999999998	24.7877693696462\\
2603.29999999999	24.7906583698241\\
2603.6	24.7954833506164\\
2603.8	24.7935788624755\\
2603.99999999999	24.7993548634845\\
2604.30000000002	24.7964982337583\\
2604.39999999999	24.7993856786331\\
2605.09999999999	24.7927222478121\\
2605.19999999999	24.7956089509845\\
2605.59999999999	24.791802586637\\
2605.7	24.7946887712027\\
2607.49999999999	24.7775732474306\\
2607.60000000001	24.7804578747555\\
2607.80000000002	24.7785574600253\\
2607.90000000002	24.7814417177914\\
2608.10000000001	24.7795414462081\\
2608.20000000001	24.7824253345091\\
2608.40000000002	24.7805252060571\\
2608.5	24.7834087249866\\
2609.2	24.7767600505883\\
2609.3	24.7796428297693\\
2609.7	24.7873400260556\\
2610.09999999999	24.7835414910735\\
2610.19999999999	24.7864230165115\\
2610.50000000001	24.7835746571669\\
2610.69999999999	24.7893366018079\\
2610.99999999999	24.7864884531424\\
2611.09999999999	24.789368872549\\
2611.30000000002	24.7912996859922\\
2611.50000000001	24.7932302037065\\
2612.5	24.7837403352982\\
2612.59999999999	24.7866192061852\\
2612.89999999999	24.7914274779946\\
2613.10000000002	24.7895300780652\\
2613.19999999999	24.7924080664294\\
2613.80000000001	24.7867171659207\\
2614.00000000001	24.7924715963429\\
2614.20000000001	24.7905749148912\\
2614.29999999998	24.7934516523868\\
2614.50000000001	24.7915551135929\\
2614.80000000001	24.800183563425\\
2615.20000000002	24.7963904714564\\
2615.3	24.7992658866713\\
2615.90000000001	24.7935779816514\\
2615.99999999999	24.7964527349872\\
2616.29999999999	24.8012536309433\\
2616.5	24.8031796988458\\
2617.50000000002	24.7937041564792\\
2617.6	24.7965771478779\\
2617.99999999999	24.7927886635346\\
2618.2	24.7985333995341\\
2618.6	24.7947454844007\\
2618.7	24.7976172292653\\
2618.99999999998	24.8024130426482\\
2619.3	24.7995724211652\\
2619.49999999999	24.8053137883646\\
2620.50000000001	24.7958482790201\\
2620.70000000002	24.8015873015873\\
2620.89999999999	24.7996947729874\\
2621	24.8025638090878\\
2621.50000000001	24.7978333841929\\
2621.6	24.8007018346874\\
2621.79999999998	24.8026240512605\\
2622.30000000002	24.7978950579622\\
2622.49999999999	24.8036299855106\\
2622.69999999999	24.8055513192009\\
2622.9	24.8036599313763\\
2622.99999999999	24.806526628798\\
2623.39999999999	24.8141795311607\\
2623.60000000001	24.8122879902428\\
2623.8	24.8180189793818\\
2624.3	24.8132906569121\\
2624.40000000001	24.8161554581825\\
2624.59999999998	24.8180744466034\\
2624.90000000002	24.8152380952381\\
2625	24.8181021675365\\
2625.40000000001	24.8143210816987\\
2625.50000000002	24.8171846435101\\
2625.79999999999	24.8143493659317\\
2625.9	24.8172124904798\\
2626.09999999998	24.8153225192293\\
2626.20000000002	24.818185279671\\
2626.39999999998	24.8201027984009\\
2626.60000000001	24.8182129668405\\
2626.69999999998	24.8210750723314\\
2626.90000000002	24.8229920060906\\
2627.2	24.8201575762189\\
2627.40000000001	24.825880114177\\
2628.2	24.8183236312445\\
2628.40000000001	24.824044131634\\
2628.70000000001	24.8288192331102\\
2629.00000000002	24.8259860788863\\
2629.10000000001	24.8288452761296\\
2629.3	24.8269567201643\\
2629.39999999998	24.8298155542879\\
2629.70000000001	24.8345881816108\\
2629.9	24.8365019011407\\
2630.20000000002	24.8336691632133\\
2630.30000000001	24.8365267639903\\
2630.69999999998	24.8327504941463\\
2630.79999999999	24.8356075867574\\
2630.99999999999	24.8337197369921\\
2631.20000000001	24.8394329798959\\
2631.4	24.8375451263538\\
2631.50000000002	24.8404012767898\\
2632.69999999999	24.8290793072015\\
2632.80000000001	24.8319343689468\\
2633.6	24.8243915404184\\
2633.7	24.827245804541\\
2634	24.8244182073574\\
2634.39999999999	24.8358322262289\\
2634.80000000001	24.8320619378345\\
2634.90000000001	24.8349146110057\\
2635.10000000002	24.8368245294475\\
2635.29999999999	24.8387341580026\\
2635.7	24.8349647165946\\
2635.80000000002	24.8378163056262\\
2636.5	24.8312220283699\\
2636.59999999998	24.834072894148\\
2636.80000000001	24.8359816451136\\
2636.99999999999	24.8378901065565\\
2637.20000000002	24.8397982785425\\
2638.19999999999	24.8303832013039\\
2638.29999999998	24.833232261977\\
2639.09999999999	24.8257047590179\\
2639.20000000001	24.8285530254234\\
2639.59999999999	24.8247906959124\\
2639.70000000001	24.8276384574589\\
2640.20000000002	24.8229367874863\\
2640.29999999999	24.8257839721254\\
2641.00000000001	24.8192041194957\\
2641.20000000001	24.8248968311059\\
2641.80000000001	24.8192588667247\\
2641.90000000001	24.8221044663134\\
2642.60000000002	24.8155295720286\\
2642.70000000002	24.8183744513395\\
2642.90000000001	24.8202799848657\\
2643.60000000001	24.8137080606725\\
2643.70000000001	24.816551932824\\
2644.1	24.8127978216474\\
2644.29999999998	24.8184843442747\\
2644.49999999998	24.8203887166301\\
2644.7	24.8222928009679\\
2645.10000000001	24.8185392408892\\
2645.2	24.8213813178089\\
2645.59999999998	24.8289677590052\\
2645.90000000002	24.8261526832955\\
2646	24.8289936132421\\
2646.50000000001	24.8243028791657\\
2646.6	24.8271432349719\\
2646.79999999999	24.8290452982735\\
2647.00000000002	24.8309470741566\\
2647.19999999998	24.8328485626865\\
2647.40000000002	24.8347497639282\\
2647.6	24.8366506779469\\
2647.89999999999	24.833836858006\\
2648.00000000001	24.8366753521393\\
2648.19999999998	24.8347996828154\\
2648.3	24.8376378190606\\
2648.7	24.8338870431894\\
2648.8	24.8367246781683\\
2649	24.8386244384885\\
2649.39999999999	24.8348745046235\\
2649.49999999999	24.837711352657\\
2649.69999999998	24.8358366669183\\
2649.80000000002	24.8386731574776\\
2649.99999999999	24.8405720538848\\
2650.89999999999	24.8321388155413\\
2651.00000000001	24.8349741616687\\
2651.79999999999	24.8274821825861\\
2651.90000000001	24.8303167420815\\
2652.09999999999	24.8284443103838\\
2652.19999999999	24.8312785129887\\
2652.50000000001	24.8284701802006\\
2652.60000000001	24.8313039544615\\
2653.39999999998	24.823817599397\\
2653.6	24.8294833628519\\
2653.80000000002	24.8276121933758\\
2653.89999999999	24.8304446119066\\
2654.1	24.8285735814935\\
2654.39999999998	24.837069127896\\
2654.59999999999	24.8389648547859\\
2655.79999999999	24.8277420083588\\
2655.89999999999	24.8305722891566\\
2656.20000000002	24.8277679478974\\
2656.3	24.8305978015359\\
2656.49999999998	24.8287284498984\\
2656.70000000002	24.8343872327612\\
2657.49999999999	24.8269114990969\\
2657.6	24.8297400007525\\
2657.8	24.827871627977\\
2657.90000000002	24.8306997742664\\
2658.10000000002	24.8288315401399\\
2658.19999999998	24.8316593311515\\
2658.40000000001	24.8297912356592\\
2658.50000000002	24.8326186714812\\
2658.9	24.8288830387364\\
2659.2	24.8373632158839\\
2659.50000000002	24.8345615882088\\
2659.60000000001	24.8373876753017\\
2659.89999999999	24.8345864661654\\
2660.09999999999	24.8402375761221\\
2660.3	24.8421290031574\\
2660.59999999999	24.8393279963919\\
2660.69999999999	24.8421527360192\\
2660.90000000002	24.8440435926343\\
2661.10000000002	24.8421764617466\\
2661.19999999999	24.8450005636343\\
2661.4	24.8431335712944\\
2661.49999999999	24.8459573189059\\
2663.1	24.8310303394413\\
2663.20000000002	24.8338527390831\\
2663.70000000001	24.8291913807343\\
2664	24.837656244135\\
2665.40000000001	24.8246107672107\\
2665.49999999998	24.827430972389\\
2665.7	24.8293195288469\\
2666.4	24.8228014250891\\
2666.49999999999	24.825620640516\\
2667.20000000002	24.8191054624527\\
2667.29999999999	24.821923970908\\
2667.5	24.8238116659169\\
2667.7	24.8256990778919\\
2668.10000000001	24.8332208979837\\
2668.30000000002	24.8313596162494\\
2668.40000000001	24.8341765036537\\
2669.19999999999	24.8267336005694\\
2669.30000000002	24.8295497115457\\
2669.79999999999	24.8248998089816\\
2669.89999999999	24.8277153558052\\
2670.30000000002	24.823996405033\\
2670.49999999999	24.8296263012057\\
2670.8	24.8268373956344\\
2671.00000000002	24.8324660252331\\
2671.29999999999	24.8296773227521\\
2671.4	24.8324911098634\\
2672.10000000002	24.8259860788863\\
2672.19999999998	24.8287991617708\\
2672.50000000002	24.8260121230263\\
2672.60000000001	24.8288247839264\\
2673.1	24.824180757145\\
2673.20000000001	24.8269928552725\\
2673.79999999999	24.8214218931149\\
2673.90000000001	24.8242333582648\\
2674.09999999999	24.8223767855807\\
2674.20000000002	24.8251878996373\\
2674.80000000001	24.8196194250252\\
2675.00000000001	24.8252401779373\\
2675.2	24.8271221919037\\
2675.90000000002	24.8206278026906\\
2676.00000000002	24.8234370912896\\
2677.19999999998	24.8123109102454\\
2677.30000000001	24.8151191454396\\
2677.6	24.8123389476043\\
2677.70000000002	24.8151467622675\\
2678.09999999999	24.8114405197521\\
2678.2	24.8142478437815\\
2678.79999999998	24.8086901340102\\
2678.90000000001	24.8114968271743\\
2679.09999999998	24.8133771275008\\
2681.49999999999	24.791169451074\\
2681.70000000001	24.7967782832426\\
2682.10000000002	24.7930803072105\\
2682.2	24.795884129292\\
2682.39999999999	24.7977632805219\\
2682.6	24.7996421515637\\
2682.90000000001	24.8043235184495\\
2683.5	24.7987777612163\\
2683.59999999999	24.8015799083355\\
2683.80000000002	24.8034576549052\\
2684.59999999999	24.7960665996201\\
2684.70000000001	24.7988676996424\\
2684.89999999998	24.8007448789572\\
2685.09999999998	24.7988976612543\\
2685.20000000002	24.8016981342867\\
2685.7	24.7970809442252\\
2685.79999999998	24.7998808593023\\
2686.29999999999	24.7952650387135\\
2686.5	24.8008635450011\\
2687.00000000001	24.7962487439991\\
2687.1	24.7990473355165\\
2687.50000000001	24.795356451853\\
2687.70000000002	24.8009524518193\\
2687.9	24.8028273809524\\
2688.19999999999	24.800059517167\\
2688.40000000001	24.8056537102473\\
2688.59999999999	24.8075278015398\\
2688.89999999999	24.8047601338788\\
2689	24.8075564315198\\
2689.29999999999	24.8047891723061\\
2689.49999999998	24.8103807257585\\
2689.89999999999	24.8066914498141\\
2690.09999999998	24.8122816147498\\
2690.30000000001	24.8141540291407\\
2690.59999999999	24.811387371316\\
2690.79999999998	24.8169757330261\\
2690.99999999998	24.8151313589239\\
2691.10000000001	24.8179250891796\\
2691.29999999999	24.8197963884967\\
2691.6	24.8170301296578\\
2691.80000000001	24.8226159961366\\
2692.09999999999	24.8272788054379\\
2692.29999999999	24.8254345565295\\
2692.40000000001	24.8282265552461\\
2692.99999999999	24.822695035461\\
2693.10000000002	24.8254864102183\\
2693.70000000001	24.8199569381543\\
2693.80000000002	24.8227476892238\\
2694.89999999999	24.8126159554731\\
2695.09999999999	24.8181953101811\\
2695.49999999999	24.8145125389524\\
2695.59999999999	24.8173016285195\\
2695.79999999999	24.8191698505137\\
2696.80000000001	24.8099669991472\\
2696.90000000002	24.8127549128661\\
2697.4	24.808155699722\\
2697.49999999999	24.8109430604982\\
2697.80000000001	24.8081841432225\\
2698	24.8137578295838\\
2698.19999999999	24.8119186154245\\
2698.30000000002	24.8147050103765\\
2698.49999999999	24.8128659304825\\
2698.80000000002	24.8212234614102\\
2699.1	24.8184647302905\\
2699.19999999999	24.8212499536917\\
2699.50000000002	24.8184916283894\\
2699.60000000002	24.8212764381228\\
2699.80000000002	24.823141597837\\
2700	24.8250064812414\\
2700.20000000002	24.8231677961708\\
2700.29999999998	24.8259517108577\\
2701.30000000002	24.8167616791293\\
2701.39999999999	24.8195446973903\\
2701.59999999998	24.8214087426435\\
2701.8	24.8195714127096\\
2701.89999999999	24.8223538119911\\
2702.19999999999	24.8195981201199\\
2702.5	24.8279434618516\\
2702.79999999998	24.8251877612934\\
2702.90000000001	24.8279689234184\\
2703.70000000001	24.8206228271322\\
2703.80000000002	24.8234032323681\\
2704.69999999998	24.8151434486838\\
2704.90000000001	24.8207024029575\\
2705.09999999999	24.8225639509094\\
2705.29999999998	24.8207289125453\\
2705.40000000002	24.823507669562\\
2705.60000000001	24.8216727649037\\
2705.70000000001	24.8244511789489\\
2705.90000000001	24.8226164079823\\
2705.99999999999	24.8253944791397\\
2706.19999999998	24.8272549236966\\
2706.5	24.824503066578\\
2706.8	24.832834607854\\
2707.5	24.8264145368592\\
2707.59999999998	24.8291908261624\\
2708.6	24.820024365932\\
2708.89999999999	24.828349944629\\
2709.09999999999	24.8265170530046\\
2709.19999999999	24.8292916989628\\
2709.50000000002	24.8265426631237\\
2709.59999999998	24.8293168985497\\
2709.80000000002	24.8311745820879\\
2709.99999999999	24.8330319914394\\
2710.20000000001	24.8348891266649\\
2710.40000000001	24.8330566316178\\
2710.50000000001	24.8358297056002\\
2710.70000000001	24.8339973439575\\
2710.79999999999	24.8367700763584\\
2710.99999999998	24.8349378481059\\
2711.10000000002	24.8377102390086\\
2711.30000000002	24.8358781441322\\
2711.40000000001	24.8386501936198\\
2712.10000000001	24.8322395103606\\
2712.29999999998	24.8377820380475\\
2712.49999999999	24.8396372483964\\
2712.7	24.8414921851961\\
2713.7	24.8323384184538\\
2713.80000000002	24.8351081469472\\
2714.1	24.8323631272567\\
2714.20000000002	24.8351324466713\\
2714.70000000001	24.8305584205098\\
2714.79999999999	24.8333271943718\\
2715.59999999999	24.8260117096881\\
2715.7	24.8287797334119\\
2716.19999999998	24.824209402496\\
2716.30000000002	24.8269768811663\\
2716.59999999998	24.8315971583171\\
2717.29999999999	24.8252005593582\\
2717.50000000001	24.8307329997056\\
2717.90000000002	24.8270787343635\\
2718.09999999998	24.832609815319\\
2718.80000000001	24.8262164846077\\
2718.90000000002	24.8289812431041\\
2719.09999999999	24.8271550456016\\
2719.4	24.8354476925906\\
2719.6	24.8336213552965\\
2719.69999999999	24.8363850283109\\
2720.20000000001	24.8318200198508\\
2720.30000000001	24.8345831495368\\
2720.50000000001	24.8327574799677\\
2720.6	24.8355202705186\\
2721.29999999999	24.8291320643786\\
2721.4	24.8318941760059\\
2721.79999999999	24.8282449759359\\
2722.10000000001	24.83652927779\\
2722.40000000002	24.8337924701561\\
2722.50000000001	24.8365532946448\\
2722.90000000002	24.8329048843188\\
2722.99999999999	24.8356652344754\\
2723.80000000001	24.8283710855758\\
2723.89999999998	24.8311306901615\\
2724.30000000001	24.8274849508149\\
2724.39999999999	24.8302440814828\\
2724.99999999999	24.824777072401\\
2725.10000000002	24.8275355937179\\
2725.39999999999	24.8321408915795\\
2725.70000000001	24.8294078802553\\
2725.90000000001	24.8349229640499\\
2726.10000000002	24.8331010197344\\
2726.20000000001	24.8358581227304\\
2726.40000000002	24.8377040161379\\
2726.79999999999	24.8340606549562\\
2726.90000000001	24.8368170150348\\
2727.19999999999	24.8414182524841\\
2727.40000000002	24.8432630614115\\
2728.00000000002	24.8377992009091\\
2728.2	24.843309020269\\
2729	24.8360265288923\\
2729.09999999999	24.8387805950462\\
2729.5	24.8351406799531\\
2729.59999999999	24.8378942740961\\
2730.40000000001	24.8306171030947\\
2730.49999999998	24.8333699553212\\
2730.70000000001	24.8352131243592\\
2731.30000000002	24.8297576334481\\
2731.39999999999	24.8325096101043\\
2731.60000000002	24.8306915107808\\
2731.69999999999	24.833443151036\\
2731.90000000002	24.8316251830161\\
2731.99999999999	24.8343764869514\\
2732.39999999998	24.8307410795974\\
2732.49999999999	24.8334919124643\\
2732.70000000001	24.8353337236534\\
2733.00000000002	24.8326076616297\\
2733.1	24.8353578223328\\
2734	24.8271826195092\\
2734.09999999998	24.8299319727891\\
2734.7	24.8244844229925\\
2734.80000000001	24.8272331712311\\
2735.00000000001	24.8290738912654\\
2735.30000000002	24.8263508079257\\
2735.49999999999	24.8318467612224\\
2735.99999999998	24.8273089433866\\
2736.09999999998	24.8300562824355\\
2736.39999999999	24.827334186004\\
2736.50000000002	24.8300811225608\\
2737.29999999999	24.8228245780668\\
2737.4	24.8255707762557\\
2738	24.8201307475987\\
2738.10000000001	24.8228763421226\\
2738.5	24.8192507120427\\
2738.70000000001	24.8247407623777\\
2738.9	24.8265790434465\\
2741.40000000001	24.8039394492066\\
2741.49999999999	24.8066822293551\\
2741.7	24.8085199503976\\
2742.29999999998	24.8030921820303\\
2742.39999999998	24.8058340929809\\
2742.89999999998	24.8013124316442\\
2743.20000000002	24.8095359603397\\
2743.39999999998	24.8113723346091\\
2743.60000000002	24.8132084411561\\
2743.79999999999	24.8150442800394\\
2744.2	24.8114273220858\\
2744.30000000001	24.814167031045\\
2744.49999999998	24.8123588136705\\
2744.59999999999	24.8150981892374\\
2744.9	24.8123861566485\\
2745.20000000001	24.8206024842458\\
2745.60000000001	24.8169865608042\\
2745.70000000002	24.8197246704057\\
2746.2	24.8152059134108\\
2746.29999999999	24.8179434896592\\
2746.49999999998	24.8197771790577\\
2746.70000000001	24.8216106014271\\
2746.89999999998	24.8198034219148\\
2746.99999999999	24.8225401332314\\
2747.40000000001	24.8189262966333\\
2747.69999999999	24.8271344348206\\
2748.30000000002	24.8217144520448\\
2748.39999999999	24.8244496998363\\
2749.9	24.8109090909091\\
2750.1	24.8163769907643\\
2750.90000000001	24.8091603053435\\
2751.10000000001	24.8146263448677\\
2751.30000000001	24.8164570763975\\
2751.50000000001	24.8146532926297\\
2751.60000000001	24.8173856161646\\
2752.1	24.8128769711504\\
2752.30000000001	24.8183403575062\\
2752.49999999999	24.8165370922037\\
2752.79999999999	24.8247302844273\\
2753.09999999998	24.8220252796746\\
2753.30000000001	24.8274860172877\\
2753.49999999998	24.8256827425915\\
2753.6	24.8284126811199\\
2755.00000000001	24.815796159849\\
2755.3	24.8239819989838\\
2755.49999999999	24.825809261141\\
2755.8	24.8231067890707\\
2755.90000000001	24.8258345428157\\
2756.20000000001	24.8303885643798\\
2756.40000000002	24.8322147651007\\
2756.79999999999	24.8286118466393\\
2756.89999999998	24.8313384113167\\
2757.1	24.8331640795009\\
2757.29999999999	24.8349894828462\\
2757.60000000002	24.8395401965406\\
2757.80000000002	24.837738859277\\
2757.89999999999	24.8404641044235\\
2758.09999999999	24.8422884489885\\
2758.39999999999	24.8395867319195\\
2758.49999999998	24.8423113173349\\
2759.4	24.8342090958507\\
2759.5	24.8369328888245\\
2759.7	24.8351329806508\\
2759.80000000002	24.8378564440741\\
2760	24.8360566646136\\
2760.09999999998	24.8387797985653\\
2760.59999999998	24.8342811605752\\
2760.70000000002	24.8370037670241\\
2760.90000000001	24.8388265121333\\
2761.8	24.8307324667801\\
2762.00000000001	24.8361753738098\\
2762.29999999999	24.8334781349551\\
2762.49999999999	24.8389198581047\\
2762.7	24.8407412769654\\
2762.90000000001	24.842562432139\\
2763.1	24.8407643312102\\
2763.20000000001	24.8434842398581\\
2763.60000000001	24.8398885551977\\
2763.80000000001	24.8453272549658\\
2764.00000000001	24.847147353569\\
2764.5	24.8426535484338\\
2764.60000000001	24.8453720114298\\
2764.80000000001	24.8471915801656\\
2765.20000000001	24.8435974396991\\
2765.3	24.8463151804441\\
2765.49999999999	24.8481342204223\\
2765.89999999999	24.8445408532176\\
2766.09999999999	24.8499746945268\\
2766.4	24.8545093077896\\
2766.59999999998	24.8563270322044\\
2768.50000000001	24.8392689445929\\
2768.6	24.8419836024127\\
2768.80000000002	24.8401892448265\\
2769	24.8456177097252\\
2769.3	24.8501480465083\\
2769.5	24.8519641825534\\
2769.8	24.8492725369147\\
2769.89999999998	24.8519855595668\\
2770.10000000001	24.8538011695906\\
2771.09999999999	24.8448325635104\\
2771.2	24.8475444737127\\
2771.39999999999	24.8457513981598\\
2771.49999999998	24.8484629816712\\
2771.7	24.846670033913\\
2771.8	24.8493812908114\\
2772.20000000002	24.8566172492155\\
2772.39999999998	24.858431018936\\
2773.09999999999	24.8521563536708\\
2773.2	24.854866044063\\
2773.4	24.8530737335497\\
2773.49999999998	24.8557830977791\\
2773.70000000001	24.8539909149903\\
2773.79999999998	24.8566999531346\\
2774.3	24.8522202998847\\
2774.49999999999	24.8576371368846\\
2774.79999999998	24.8549497279181\\
2774.90000000002	24.8576576576577\\
2775.3	24.8540750882756\\
2775.39999999999	24.8567825617006\\
2776.19999999999	24.8496199978389\\
2776.30000000001	24.85232675407\\
2776.60000000001	24.8568444556488\\
2777.1	24.8523692928129\\
2777.19999999998	24.8550750729125\\
2777.8	24.849706612909\\
2777.89999999999	24.8524118070554\\
2779.09999999999	24.8416810592976\\
2779.19999999999	24.8443852768683\\
2779.70000000002	24.8399165407583\\
2779.79999999998	24.8426202381381\\
2779.99999999999	24.8408330635589\\
2780.10000000002	24.8435364362276\\
2780.3	24.8417493885772\\
2780.40000000001	24.8444524366121\\
2780.59999999999	24.8426655158773\\
2780.80000000001	24.8480707684563\\
2781.1	24.84539047893\\
2781.20000000001	24.8480926185597\\
2781.6	24.8445195384118\\
2781.90000000002	24.8526240115025\\
2782.10000000002	24.8544317446625\\
2783.20000000002	24.8446089174721\\
2783.30000000001	24.8473090464899\\
2783.60000000002	24.8446312461831\\
2783.7	24.8473309864214\\
2784.29999999999	24.8419767274817\\
2784.50000000001	24.8473748473749\\
2784.80000000001	24.844698193831\\
2784.90000000002	24.8473967684022\\
2785.50000000001	24.842044801838\\
2785.60000000002	24.8447427935528\\
2785.89999999999	24.8492462311558\\
2786.09999999999	24.8510516115139\\
2786.60000000001	24.8465927441059\\
2786.70000000001	24.8492895076791\\
2786.90000000001	24.8475062791532\\
2787.00000000002	24.8502027196728\\
2787.69999999998	24.8439629815625\\
2787.79999999998	24.8466587754224\\
2789.20000000001	24.834187789051\\
2789.30000000001	24.8368824836882\\
2789.50000000001	24.8351018067106\\
2789.60000000002	24.8377961788006\\
2789.80000000001	24.8360156278003\\
2789.90000000001	24.8387096774194\\
2790.39999999999	24.8342590933525\\
2790.5	24.8369526266753\\
2790.80000000001	24.8414489949479\\
2791.70000000001	24.8334407908876\\
2791.79999999999	24.836133099323\\
2792.00000000002	24.8379356040256\\
2792.20000000002	24.8361565734341\\
2792.29999999998	24.8388483025355\\
2792.50000000002	24.8406502900523\\
2792.80000000002	24.8451430412833\\
2793.20000000001	24.8415852217807\\
2793.29999999998	24.8442757929405\\
2793.79999999999	24.8398296288342\\
2793.90000000001	24.8425196850394\\
2794.09999999999	24.8407415360389\\
2794.20000000001	24.8434312708013\\
2794.50000000001	24.8407643312102\\
2794.70000000002	24.8461428366967\\
2795.4	24.8399213020926\\
2795.5	24.8426098154242\\
2795.90000000001	24.8390557939914\\
2796	24.8417438575158\\
2796.19999999998	24.8399670993813\\
2796.4	24.8453423922761\\
2796.9	24.84090096532\\
2797.10000000002	24.8462748462748\\
2797.30000000001	24.8444984628584\\
2797.39999999999	24.8471849865952\\
2797.8	24.8436327245434\\
2797.99999999999	24.8490046817483\\
2798.59999999999	24.8436774216601\\
2798.70000000002	24.846362726883\\
2798.9	24.8445873526259\\
2798.99999999998	24.8472723375371\\
2799.59999999999	24.841947351502\\
2799.80000000002	24.8473159755706\\
2800.10000000001	24.8446539532891\\
2800.20000000001	24.8473377852373\\
2800.50000000001	24.844676140827\\
2800.59999999998	24.8473595886743\\
2801.09999999999	24.8429244609453\\
2801.30000000002	24.848290140644\\
2801.90000000001	24.8429693076374\\
2802.00000000002	24.845651475679\\
2802.5	24.8412188681938\\
2802.6	24.8439005244942\\
2802.99999999999	24.8403553208947\\
2803.1	24.8430365296804\\
2803.4	24.8403780988051\\
2803.49999999999	24.8430589242403\\
2803.7	24.8448534132249\\
2804.10000000002	24.8413094643749\\
2804.2	24.8439895874193\\
2805.20000000002	24.8351334973087\\
2805.40000000001	24.8404918909285\\
2806.20000000001	24.8334105405694\\
2806.39999999999	24.8387671476929\\
2806.60000000001	24.8369971853066\\
2806.80000000001	24.8423527735224\\
2807.10000000002	24.8396979196352\\
2807.19999999999	24.8423752359919\\
2807.4	24.8441674087266\\
2807.59999999999	24.8459593261388\\
2807.8	24.844189607892\\
2808.29999999998	24.8575701467027\\
2810.4	24.8389966198185\\
2810.50000000002	24.841670817619\\
2811.00000000001	24.8372523211554\\
2811.10000000001	24.8399260102447\\
2811.49999999999	24.8363920899132\\
2811.6	24.8390653341395\\
2811.80000000002	24.8372986237064\\
2812.00000000002	24.8426442871875\\
2812.20000000001	24.8408775735163\\
2812.3	24.8435499928886\\
2813.29999999998	24.8347195564086\\
2813.40000000002	24.8373911498134\\
2813.80000000002	24.8338604783397\\
2813.99999999998	24.8392025869727\\
2814.30000000002	24.8365548607163\\
2814.39999999999	24.8392254396873\\
2814.70000000001	24.8436833878073\\
2814.90000000002	24.841918294849\\
2814.99999999999	24.844588114099\\
2815.4	24.8410584265672\\
2815.5	24.8437278022446\\
2815.99999999998	24.8393167856255\\
2816.1	24.841985654428\\
2816.59999999999	24.8375758866759\\
2816.70000000002	24.840244248793\\
2817.70000000001	24.831428774221\\
2817.80000000002	24.8340963128571\\
2818.40000000001	24.8288096505233\\
2818.60000000002	24.8341433994395\\
2819.09999999999	24.8297389330306\\
2819.3	24.8350712917642\\
2819.80000000002	24.8306677541757\\
2819.9	24.8333333333333\\
2820.10000000002	24.8315722289199\\
2820.20000000002	24.8342374924653\\
2820.60000000002	24.830715779771\\
2820.70000000002	24.8333806012479\\
2820.90000000002	24.8316199929103\\
2821.1	24.8369488161066\\
2821.29999999999	24.838732544127\\
2821.59999999999	24.8360917177588\\
2821.79999999999	24.8414189021581\\
2822.20000000002	24.8378981681607\\
2822.40000000001	24.8432240921169\\
2823.00000000001	24.8379441039991\\
2823.09999999999	24.8406064040805\\
2823.29999999999	24.8423886094779\\
2823.70000000001	24.8388696083292\\
2823.80000000001	24.8415312156946\\
2824.00000000001	24.8397719627492\\
2824.10000000001	24.8424332554352\\
2824.29999999998	24.840674125478\\
2824.40000000001	24.8433351035582\\
2824.89999999998	24.8389380530973\\
2825.00000000001	24.8415985274858\\
2825.29999999998	24.8389608551002\\
2825.40000000001	24.8416209520439\\
2826.10000000002	24.8354681197368\\
2826.19999999999	24.838127587305\\
2826.49999999998	24.8354914030991\\
2826.7	24.8408093957832\\
2827.00000000001	24.8381733932298\\
2827.1	24.8408319185059\\
2827.3	24.8390747683384\\
2827.50000000001	24.8443910029707\\
2827.89999999998	24.8408769448373\\
2828.00000000002	24.843534528482\\
2828.60000000002	24.8382649273518\\
2828.9	24.8462354188759\\
2830	24.836578212784\\
2830.1	24.8392339763974\\
2830.29999999998	24.8410118711136\\
2831.00000000002	24.8348698385786\\
2831.2	24.8401794228799\\
2831.49999999998	24.8446108207374\\
2831.79999999998	24.8419788834352\\
2831.89999999999	24.8446327683616\\
2832.40000000002	24.8402471315093\\
2832.60000000001	24.8455537120062\\
2833	24.8420458155378\\
2833.10000000002	24.8446985740505\\
2834.09999999999	24.8359325382824\\
2834.3	24.8412362404742\\
2834.49999999999	24.8394835250123\\
2834.60000000001	24.8421349701908\\
2834.80000000001	24.8439098380895\\
2835	24.8421572431308\\
2835.10000000001	24.8448081264108\\
2835.30000000002	24.8430556535233\\
2835.40000000002	24.8457062246517\\
2835.6	24.8474803399513\\
2835.8	24.8492542050143\\
2836.19999999999	24.8457497443853\\
2836.3	24.8483993794951\\
2836.50000000002	24.8466473947684\\
2836.59999999998	24.8492967180174\\
2836.90000000002	24.8466690165668\\
2836.99999999998	24.8493179655282\\
2837.50000000001	24.8449393853961\\
2837.69999999999	24.8502360983861\\
2838	24.8476093160918\\
2838.1	24.8502572052709\\
2838.3	24.8485062006764\\
2838.4	24.8511537784041\\
2838.59999999999	24.8529256349738\\
2838.9	24.8502994011976\\
2839.00000000002	24.8529463562397\\
2839.20000000001	24.8547177121121\\
2839.40000000002	24.8529670716675\\
2839.6	24.8582596753178\\
2840.1	24.8538835293289\\
2840.2	24.856529239869\\
2841	24.8495301115765\\
2841.10000000001	24.8521751372659\\
2841.5	24.8486768018018\\
2841.6	24.8513213921244\\
2841.80000000002	24.8495724691228\\
2841.99999999999	24.8548608423349\\
2842.40000000002	24.8513632365875\\
2842.70000000002	24.8592936541438\\
2843.29999999998	24.8540479707393\\
2843.40000000001	24.8566906980833\\
2844.29999999998	24.8488257629025\\
2844.39999999999	24.8514677447706\\
2844.60000000001	24.8532358420923\\
2845.10000000002	24.848868269366\\
2845.19999999999	24.8515095069061\\
2845.40000000001	24.8497627833421\\
2845.49999999999	24.8524037109924\\
2846.19999999999	24.8462916769139\\
2846.30000000002	24.8489319842608\\
2847.00000000002	24.8428225211619\\
2847.09999999999	24.8454622084855\\
2847.30000000002	24.8472290510641\\
2847.49999999999	24.8489956454558\\
2847.79999999999	24.8463780329365\\
2847.9	24.8490168539326\\
2848.4	24.8446550816219\\
2848.49999999999	24.8472934072878\\
2849.10000000002	24.8420609293837\\
2849.20000000002	24.8446986979258\\
2849.4	24.8464642919811\\
2850.59999999999	24.8360051917073\\
2850.69999999998	24.8386417847622\\
2851.00000000001	24.8360281996422\\
2851.10000000001	24.8386644219978\\
2851.79999999998	24.832567761843\\
2851.99999999999	24.8378387854563\\
2852.19999999998	24.8360971847281\\
2852.39999999999	24.8413672217353\\
2853.1	24.8352726762933\\
2853.40000000001	24.8431750481864\\
2853.6	24.8414339278831\\
2853.70000000001	24.8440675590441\\
2855.1	24.8318856822639\\
2855.19999999998	24.8345182642805\\
2855.50000000002	24.838913013027\\
2855.80000000002	24.8433068384747\\
2856.39999999998	24.8380885699282\\
2856.50000000001	24.84071973675\\
2856.70000000002	24.8424810977317\\
2857	24.839872598089\\
2857.1	24.842503149937\\
2857.50000000002	24.8390257558791\\
2857.59999999998	24.841655877104\\
2857.79999999998	24.8399174218832\\
2857.89999999999	24.8425472358293\\
2858.69999999999	24.8355953546943\\
2858.80000000001	24.8382244919375\\
2859.49999999999	24.832144355854\\
2859.60000000002	24.8347728782739\\
2860	24.8312996049089\\
2860.09999999998	24.8339276973638\\
2860.4	24.8383149798986\\
2862.2	24.822695035461\\
2862.50000000001	24.8305736044156\\
2862.7	24.8323319826743\\
2862.89999999998	24.8340901152637\\
2863.10000000001	24.8323554065381\\
2863.20000000001	24.8349806167709\\
2863.50000000001	24.8323788238581\\
2863.60000000002	24.8350036665852\\
2863.80000000002	24.8332693180628\\
2864.1	24.8411423783255\\
2864.6	24.8368066464202\\
2864.89999999998	24.8446771378709\\
2865.09999999999	24.8464330587743\\
2865.49999999999	24.8429648241206\\
2865.6	24.8455874655407\\
2865.79999999999	24.8473428940298\\
2866.00000000002	24.849098077527\\
2866.2	24.8473641977462\\
2866.3	24.8499860452135\\
2866.49999999998	24.8482522849369\\
2866.79999999998	24.8561163626216\\
2866.99999999998	24.8578703219281\\
2867.59999999999	24.8526693866165\\
2867.69999999998	24.855289769161\\
2868.1	24.8518234432745\\
2868.2	24.8544433985287\\
2868.79999999998	24.8492453553627\\
2869	24.8544839845248\\
2869.20000000001	24.852751542188\\
2869.29999999999	24.8553704607235\\
2869.90000000001	24.8501742160279\\
2870.00000000001	24.8527925856242\\
2870.50000000002	24.8484637358044\\
2870.6	24.8510816177239\\
2870.80000000001	24.8493503779303\\
2870.89999999999	24.8519679554162\\
2871.3	24.8485059552831\\
2871.39999999999	24.8511231063904\\
2871.69999999998	24.8485270562017\\
2871.80000000001	24.8511438420558\\
2872.7	24.84335839599\\
2872.79999999998	24.8459744509033\\
2873.00000000002	24.8442448922766\\
2873.19999999998	24.849476212021\\
2873.4	24.851226726988\\
2873.59999999998	24.8529769982949\\
2874.29999999998	24.8469245755636\\
2874.39999999999	24.8495390502696\\
2875.19999999999	24.8426251173791\\
2875.30000000001	24.8452389232802\\
2875.50000000002	24.8435109194603\\
2875.70000000002	24.8487377425412\\
2876	24.8461458224679\\
2876.20000000001	24.8513715537322\\
2876.40000000002	24.8531201112463\\
2877.19999999999	24.8462099885309\\
2877.29999999999	24.8488218530618\\
2878.10000000001	24.8419150858175\\
2878.19999999999	24.8445262828753\\
2878.39999999999	24.8428000694806\\
2878.49999999999	24.8454109636629\\
2878.69999999998	24.8471585382798\\
2879.10000000001	24.8437065851625\\
2879.20000000001	24.8463168131143\\
2879.80000000002	24.8411403173721\\
2879.90000000001	24.84375\\
2881.49999999998	24.8299555802332\\
2881.60000000002	24.832564111462\\
2881.79999999998	24.8343106978035\\
2882.00000000001	24.8325873495021\\
2882.10000000002	24.8351953368954\\
2882.29999999999	24.8369414376908\\
2882.90000000002	24.8317724592438\\
2882.99999999998	24.8343796607818\\
2883.40000000001	24.8309346280562\\
2883.69999999998	24.8387544212497\\
2885.30000000002	24.8249809385181\\
2885.40000000001	24.8275862068966\\
2885.69999999998	24.8319356850787\\
2886.10000000001	24.8284942138452\\
2886.2	24.8310986383952\\
2886.70000000002	24.826797838437\\
2886.8	24.8294017804565\\
2887.19999999998	24.825961971392\\
2887.40000000002	24.8311688311688\\
2887.70000000001	24.8285892374818\\
2887.89999999998	24.8337950138504\\
2888.2	24.831215593948\\
2888.30000000002	24.8338180307437\\
2888.60000000002	24.8312389656247\\
2888.79999999999	24.8364429367579\\
2889.00000000002	24.8347236163511\\
2889.10000000001	24.8373252111311\\
2889.29999999999	24.8356060081678\\
2889.40000000001	24.8382073023014\\
2889.79999999999	24.8451503512232\\
2889.99999999998	24.8468911110342\\
2890.4	24.843452689846\\
2890.50000000002	24.8460527226181\\
2890.70000000001	24.8443337484433\\
2890.80000000002	24.8469334809229\\
2890.99999999998	24.8452146241915\\
2891.10000000001	24.8478140564472\\
2891.29999999998	24.8495538493463\\
2892.50000000002	24.8392449699233\\
2892.60000000001	24.8418432606216\\
2892.89999999998	24.8392671966816\\
2893.10000000001	24.8444628784737\\
2893.29999999999	24.8427455588581\\
2893.5	24.8479402820017\\
2894.39999999998	24.8402141993436\\
2894.50000000001	24.8428107510537\\
2894.90000000001	24.839378238342\\
2894.99999999998	24.8419743704881\\
2895.70000000003	24.8359693348988\\
2895.80000000002	24.8385648675714\\
2896.1	24.8359919895035\\
2896.20000000001	24.838587162932\\
2896.4	24.8403245296047\\
2896.59999999998	24.8420616563676\\
2896.79999999999	24.8437985432704\\
2897.00000000001	24.8455351903628\\
2897.6	24.8403906546571\\
2897.69999999998	24.8429843329422\\
2897.89999999999	24.8412698412698\\
2898.39999999999	24.8542349491116\\
2898.59999999999	24.8525200952151\\
2898.79999999998	24.8577046465901\\
2900.50000000002	24.8431359029166\\
2900.70000000002	24.8483177054606\\
2900.89999999999	24.8500517063082\\
2901.09999999998	24.8517854680822\\
2901.39999999999	24.8492159227986\\
2901.5	24.851805900193\\
2901.69999999999	24.8500930456958\\
2901.80000000001	24.8526827251111\\
2902.39999999998	24.8475452196382\\
2902.79999999999	24.8579007199697\\
2903.70000000001	24.8501962945106\\
2903.79999999999	24.8527841867833\\
2904.1	24.8502169272089\\
2904.2	24.8528044623489\\
2904.9	24.8468158347676\\
2905.00000000001	24.8494027744312\\
2905.20000000003	24.8511341341686\\
2905.60000000001	24.8477131156004\\
2905.70000000002	24.8502994011976\\
2906.30000000001	24.8451692815855\\
2906.50000000002	24.8503406041423\\
2906.99999999999	24.8460665267793\\
2907.10000000002	24.8486516235553\\
2907.50000000001	24.8452331820058\\
2907.7	24.8504023660499\\
2908.30000000002	24.8452757529913\\
2908.40000000001	24.8478597215059\\
2908.59999999999	24.8461512015677\\
2908.70000000003	24.8487348734874\\
2908.90000000002	24.8504640770024\\
2909.09999999999	24.8487556716623\\
2909.20000000002	24.851338810023\\
2909.49999999998	24.8487764641188\\
2909.60000000002	24.8513592466577\\
2909.8	24.8496511907626\\
2909.9	24.8522336769759\\
2910.09999999998	24.8539619270153\\
2910.50000000001	24.8505462791177\\
2910.59999999998	24.8531281135122\\
2910.79999999999	24.8548558864956\\
2911.20000000001	24.8514409370384\\
2911.30000000001	24.8540221199423\\
2911.50000000001	24.8557494161286\\
2912.90000000001	24.8438036388603\\
2913.00000000002	24.8463835776321\\
2913.19999999998	24.8481103902791\\
2913.40000000001	24.8464046679252\\
2913.49999999998	24.8489840746842\\
2913.70000000002	24.84727846798\\
2913.79999999999	24.8498575791894\\
2913.99999999999	24.8481520881233\\
2914.2	24.8533095426003\\
2914.39999999999	24.8516040487219\\
2914.59999999998	24.8567605585481\\
2915.00000000002	24.8533497993208\\
2915.09999999999	24.8559275521405\\
2915.39999999998	24.8602298062082\\
2915.70000000001	24.8645311749777\\
2915.90000000002	24.8662551440329\\
2916.09999999999	24.8645497565325\\
2916.20000000001	24.8671261530021\\
2916.49999999998	24.8645683329905\\
2916.60000000001	24.8671443754929\\
2916.80000000002	24.8654393362817\\
2916.90000000001	24.8680150839904\\
2917.1	24.8663101604278\\
2917.20000000002	24.8688856134097\\
2917.60000000002	24.8654762312781\\
2917.7	24.8680512715059\\
2918.00000000002	24.8654946711902\\
2918.09999999999	24.8680693578233\\
2918.60000000001	24.8638092301367\\
2918.70000000001	24.8663834452515\\
2918.99999999999	24.8638278921585\\
2919.2	24.8689754393176\\
2919.39999999998	24.8672717931153\\
2919.50000000001	24.8698451842718\\
2919.69999999999	24.8715665456538\\
2921.00000000002	24.860497757694\\
2921.09999999999	24.8630699712447\\
2921.79999999999	24.8571135220233\\
2921.99999999998	24.8622565962835\\
2923.99999999998	24.8452515303854\\
2924.30000000002	24.852961291205\\
2924.70000000002	24.8598194748359\\
2924.89999999999	24.8581196581197\\
2924.99999999998	24.8606885234693\\
2925.20000000002	24.8624072744676\\
2925.4	24.8641257904632\\
2925.6	24.8658440715043\\
2925.80000000002	24.867562117639\\
2926	24.8692799289156\\
2926.40000000002	24.8658807449171\\
2926.49999999999	24.8684480284289\\
2926.70000000001	24.8701653683204\\
2926.99999999999	24.874449113457\\
2927.39999999998	24.8710503842869\\
2927.60000000001	24.8761826689893\\
2928.20000000001	24.8710856128129\\
2928.29999999998	24.8736511405546\\
2928.69999999999	24.8702540289538\\
2929	24.877948858011\\
2929.29999999999	24.8754011060285\\
2929.4	24.8779655231268\\
2930.19999999998	24.8711735999727\\
2930.30000000001	24.8737373737374\\
2930.70000000001	24.870342568582\\
2930.8	24.8729059333311\\
2930.99999999999	24.8712087612159\\
2931.1	24.8737718340611\\
2932.10000000001	24.8652888616056\\
2932.3	24.870413313327\\
2932.59999999998	24.8746888532751\\
2932.89999999998	24.8789635185817\\
2933.19999999999	24.8764190502165\\
2933.60000000001	24.8866618945359\\
2935.00000000002	24.8747913188648\\
2935.09999999998	24.8773507767784\\
2935.49999999998	24.8739610301131\\
2935.60000000001	24.8765200803897\\
2936.09999999999	24.8722839043662\\
2936.3	24.8774008990601\\
2937.00000000002	24.871471859998\\
2937.10000000001	24.8740296881384\\
2937.30000000002	24.8757404507387\\
2937.5	24.8774509803922\\
2938.30000000001	24.8706779199564\\
2938.39999999998	24.8732346435256\\
2938.89999999999	24.8690030622661\\
2939.19999999999	24.8766713162998\\
2939.70000000001	24.8724403020614\\
2939.89999999998	24.8775510204082\\
2940.70000000002	24.8707834602829\\
2940.80000000002	24.8733380937808\\
2941.00000000001	24.8750467512155\\
2942.2	24.8716990109778\\
2942.3	24.8742523110386\\
2943.39999999999	24.87175131646\\
2943.50000000002	24.8743035738551\\
2944.20000000002	24.8683897700642\\
2944.40000000002	24.8734929529632\\
2944.90000000001	24.8692699490662\\
2945.00000000001	24.8718209907983\\
2945.2	24.8735273147048\\
2945.80000000001	24.8684612512305\\
2945.9	24.8710115410726\\
2946.10000000002	24.8727173986831\\
2946.59999999998	24.8752842162419\\
2947.3	24.8693763995386\\
2947.40000000001	24.871925360475\\
2947.59999999999	24.8736302880212\\
2948.39999999998	24.8668814651518\\
2948.59999999998	24.8719774816021\\
2949.50000000001	24.8643883916463\\
2949.6	24.8669356205716\\
2949.90000000002	24.864406779661\\
2950.10000000001	24.8695003728561\\
2950.90000000002	24.8627583869875\\
2951.09999999999	24.8678503659528\\
2951.8	24.8721162641011\\
2952.69999999998	24.8679219723652\\
2952.79999999998	24.8704663212435\\
2953.30000000002	24.8662558407259\\
2953.4	24.8687997291349\\
2954.69999999998	24.8612427237038\\
2954.8	24.8637855765\\
2955.1	24.8612615051435\\
2955.19999999998	24.863804013129\\
2955.80000000001	24.8587570621469\\
2955.89999999999	24.861299052774\\
2956.99999999999	24.8520509959082\\
2957.2	24.8571331958205\\
2958	24.8504107366215\\
2958.20000000001	24.8554913294798\\
2958.70000000002	24.8512910639448\\
2958.79999999999	24.8538308155058\\
2959.20000000002	24.8504713952624\\
2959.30000000001	24.8530107454214\\
2959.69999999998	24.856409216839\\
2959.89999999998	24.8581081081081\\
2960.40000000002	24.8539098125317\\
2960.50000000001	24.8564480172938\\
2960.9	24.8598446470787\\
2961.20000000002	24.8640799648803\\
2961.80000000002	24.865795604173\\
2962.49999999998	24.8599203402417\\
2962.6	24.8624565430182\\
2963.90000000001	24.8549257759784\\
2964	24.8574609493607\\
2964.50000000001	24.8532685691156\\
2964.8	24.8608722047961\\
2965.70000000001	24.8566997100276\\
2965.80000000002	24.8592332850062\\
2966.40000000001	24.8542052924322\\
2966.49999999999	24.8567383536709\\
2967.10000000002	24.8584524130493\\
2967.7	24.8534267807804\\
2967.90000000001	24.8584905660377\\
2968.69999999999	24.8517919698195\\
2968.80000000003	24.8543231499882\\
2969.19999999999	24.8577105715152\\
2970.6	24.8459958932238\\
2970.70000000001	24.8485256496567\\
2971	24.8460166268385\\
2971.10000000002	24.8485460420032\\
2971.59999999999	24.8443651781808\\
2971.79999999999	24.8494229280931\\
2972.29999999999	24.8452429013592\\
2972.4	24.8477712363331\\
2973.50000000002	24.8385794995964\\
2973.69999999999	24.8436344071558\\
2974.19999999998	24.8394580237367\\
2974.30000000002	24.8419849381388\\
2974.69999999999	24.8453677558155\\
2976.8	24.8278410426954\\
2976.9	24.8303661404098\\
2978.00000000001	24.8211947214667\\
2978.20000000002	24.826243158849\\
2979.00000000002	24.8229331006009\\
2979.10000000002	24.8254564983888\\
2979.59999999999	24.8212907339665\\
2979.69999999999	24.8238136787704\\
2980.6	24.8196732311202\\
2980.7	24.8221953837896\\
2981.39999999999	24.8163676002012\\
2981.50000000002	24.8188891870137\\
2982.30000000002	24.8155847639485\\
2982.5	24.820626299202\\
2982.79999999998	24.8181300077106\\
2983.00000000001	24.8231705273038\\
2983.7	24.8274013003553\\
2985.10000000002	24.8191075974809\\
2985.40000000001	24.8266622006364\\
2985.90000000002	24.8292029470864\\
2986.19999999998	24.8267086361049\\
2986.30000000002	24.8292258237343\\
2986.80000000001	24.8250694700191\\
2986.9	24.8275862068966\\
2987.20000000001	24.8250928932481\\
2987.29999999999	24.8276092923613\\
2987.60000000001	24.8251163102052\\
2987.7	24.8276323716447\\
2989.10000000001	24.8160042820822\\
2989.2	24.8185193858094\\
2989.70000000002	24.8244029700983\\
2990.39999999999	24.8219361310818\\
2990.50000000001	24.8244499431552\\
2990.90000000002	24.8211300568372\\
2991	24.8236434756444\\
2991.50000000002	24.8261799705843\\
2991.79999999999	24.8236906313714\\
2991.99999999999	24.8287156177935\\
2992.8	24.8220789201109\\
2992.9	24.8245907116605\\
2993.39999999999	24.8204442959746\\
2993.49999999998	24.8229556386959\\
2994.09999999999	24.8179814307662\\
2994.20000000001	24.8204922686438\\
2994.59999999999	24.8238554780112\\
2995.09999999998	24.8197115384615\\
2995.40000000001	24.8272408612919\\
2996.10000000002	24.821440491289\\
2996.30000000001	24.8264584167668\\
2996.59999999999	24.8239730370074\\
2996.7	24.8264815803524\\
2997.29999999998	24.8348568759592\\
2997.6	24.832371484805\\
2997.70000000002	24.8348789112015\\
2998.40000000001	24.8324162081041\\
2998.49999999999	24.8349229640499\\
2999.10000000001	24.8299546545746\\
2999.2	24.8324609075451\\
2999.7	24.8349889992666\\
3000.1	24.8316778881408\\
3000.20000000001	24.8341832483418\\
3000.90000000002	24.8283905364878\\
3000.99999999999	24.8308953383759\\
3001.7	24.8350989406356\\
3002	24.8326171679824\\
3002.20000000002	24.8376244878926\\
3002.49999999999	24.8418037700659\\
3002.89999999998	24.8384948384948\\
3003.00000000001	24.8409976357764\\
3003.39999999998	24.8376893624105\\
3003.59999999998	24.8426940107201\\
3005.5	24.8303167420815\\
3005.6	24.8328176464717\\
3006.10000000001	24.8353402967201\\
3006.79999999998	24.8328843659583\\
3007.10000000002	24.8403830806065\\
3007.70000000002	24.8354278874925\\
3007.80000000003	24.837926792779\\
3008.39999999999	24.8396210736247\\
3009.30000000002	24.832192463614\\
3009.4	24.8346901478651\\
3009.80000000001	24.8313897471677\\
3009.89999999999	24.8338870431894\\
3010.40000000002	24.8364059126391\\
3010.9	24.8322816340086\\
3010.99999999999	24.8347779881107\\
3012.60000000001	24.8215886082252\\
3012.69999999998	24.8240839086564\\
3014.50000000002	24.8092615935779\\
3014.6	24.8117557302551\\
3015.5	24.8076667993103\\
3015.59999999998	24.8101601618198\\
3016.00000000001	24.8068697987467\\
3016.10000000001	24.8093627743518\\
3017.19999999998	24.8003181652471\\
3017.29999999999	24.8028103665407\\
3018.30000000001	24.7945931619401\\
3018.39999999998	24.7970846446911\\
3019.09999999999	24.7913354531002\\
3019.29999999999	24.7963171491025\\
3019.70000000002	24.7996556063315\\
3020.00000000002	24.7971921459554\\
3020.10000000001	24.7996821402556\\
3020.4	24.7972190034762\\
3020.6	24.802198165988\\
3021.00000000002	24.7989143027374\\
3021.09999999999	24.8014034158613\\
3022.1	24.7965058566607\\
3022.2	24.7989941435331\\
3022.59999999998	24.8023290435703\\
3022.99999999998	24.8056630610962\\
3023.49999999999	24.8015610530493\\
3023.60000000001	24.804048020637\\
3024.19999999999	24.7991270707271\\
3024.29999999999	24.8016135431821\\
3024.59999999999	24.7991536350712\\
3024.69999999998	24.8016397778366\\
3025.50000000001	24.7950819672131\\
3025.60000000002	24.7975675050402\\
3027.2	24.7910679483368\\
3027.30000000001	24.7935522230297\\
3027.70000000001	24.7968822247176\\
3029.1	24.7887230952067\\
3029.2	24.7912058891493\\
3029.9	24.7854785478548\\
3030.10000000001	24.7904428750578\\
3030.49999999999	24.7937702105194\\
3031	24.7962785787338\\
3032.00000000001	24.7881006563108\\
3032.09999999999	24.7905810962338\\
3032.50000000001	24.7873112180967\\
3032.60000000001	24.7897912751014\\
3033.8	24.7799861564323\\
3034.00000000001	24.7849444645859\\
3035.1	24.7825513969425\\
3035.19999999999	24.785029486377\\
3035.99999999998	24.7817924310793\\
3036.10000000003	24.7842698109479\\
3037.09999999999	24.7859870933755\\
3037.89999999999	24.7794601711652\\
3038.10000000001	24.7844118227898\\
3038.39999999999	24.7819647852559\\
3038.49999999998	24.7844402027249\\
3039.39999999999	24.7803915117618\\
3039.79999999999	24.7902891542485\\
3040.59999999999	24.7870556121946\\
3040.69999999999	24.789529071297\\
3041.19999999999	24.7854535889258\\
3041.30000000001	24.7879266127441\\
3041.6	24.7854818029391\\
3041.69999999998	24.7879545006246\\
3042.00000000001	24.792084415371\\
3043.30000000001	24.7814943812841\\
3043.40000000002	24.7839658288155\\
3043.99999999999	24.7790808449131\\
3044.09999999998	24.7815518034295\\
3044.50000000001	24.7848650068975\\
3045.20000000002	24.7923028929826\\
3045.90000000002	24.7898883782009\\
3046.00000000001	24.7923574406618\\
3046.30000000002	24.7899159663866\\
3046.40000000002	24.7923847037584\\
3046.8	24.7956939840494\\
3047.10000000001	24.7932528222631\\
3047.20000000002	24.7957208020215\\
3047.6	24.7990287757981\\
3048.00000000002	24.8023358813687\\
3048.69999999999	24.7966413014957\\
3048.90000000002	24.8015742866514\\
3049.79999999999	24.8040919374406\\
3050.50000000001	24.7984003146922\\
3050.60000000001	24.8008653751598\\
3051.1	24.7968012585212\\
3051.49999999999	24.80665880194\\
3051.89999999999	24.8099606815203\\
3052.20000000002	24.8075221963765\\
3052.30000000003	24.8099855851134\\
3052.99999999999	24.8141233500377\\
3053.50000000001	24.8100602567461\\
3053.60000000002	24.8125225136719\\
3054.50000000001	24.8052118116938\\
3054.59999999999	24.8076734212852\\
3054.99999999998	24.8044253870577\\
3055.09999999999	24.8068866195339\\
3056.89999999999	24.7922800130847\\
3057.00000000001	24.7947401131792\\
3057.50000000001	24.7906855049712\\
3057.59999999998	24.7931451744776\\
3059.09999999999	24.7842573221757\\
3059.2	24.7867159154055\\
3059.60000000002	24.7834755041344\\
3059.8	24.7883917775091\\
3060.39999999999	24.7835321025976\\
3060.49999999998	24.7859896752271\\
3061.1	24.781131582386\\
3061.19999999998	24.7835886714794\\
3061.50000000001	24.7811601776849\\
3061.60000000001	24.7836169448346\\
3062.10000000003	24.7795702436157\\
3062.3	24.7844827586207\\
3062.59999999999	24.7820550494662\\
3062.79999999999	24.7869666002808\\
3064.60000000003	24.7756713544556\\
3064.80000000002	24.7805801168064\\
3065.10000000002	24.784679629388\\
3065.40000000001	24.7822541184146\\
3065.50000000001	24.7847077244259\\
3065.8	24.7888058971265\\
3066.1	24.7929032678886\\
3066.50000000001	24.7896693406378\\
3066.6	24.7921218247628\\
3066.9	24.78969677209\\
3067.00000000001	24.7921489354765\\
3067.70000000002	24.7897516135341\\
3067.80000000001	24.7922031356954\\
3069.19999999999	24.7939269540286\\
3069.6	24.7906961592338\\
3069.79999999999	24.7955959477507\\
3070.1	24.7931730831868\\
3070.3	24.7980719124544\\
3070.8	24.8038034452441\\
3072.10000000002	24.8063277130395\\
3072.5	24.8030983531862\\
3072.60000000002	24.8055456113516\\
3073	24.8023168787218\\
3073.10000000002	24.8047637641546\\
3073.79999999999	24.8023683268812\\
3073.9	24.8048145738452\\
3074.50000000002	24.7999739803552\\
3074.59999999998	24.8024197482681\\
3075.60000000002	24.7976070488019\\
3075.79999999998	24.802496830196\\
3076.30000000001	24.7984657391757\\
3076.39999999999	24.8009101251422\\
3077.60000000003	24.7912402118465\\
3077.80000000002	24.7961272296046\\
3079.39999999999	24.7832440331223\\
3079.50000000001	24.7856864527861\\
3079.9	24.7824675324675\\
3079.99999999998	24.7849095808578\\
3080.49999999999	24.787379081997\\
3081.00000000001	24.7833565934244\\
3081.1	24.7857977411398\\
3081.79999999998	24.7834128297479\\
3081.9	24.7858533419857\\
3082.70000000002	24.7826651096406\\
3082.79999999999	24.7851049336664\\
3084.69999999999	24.7698392116183\\
3085.10000000002	24.7795928951122\\
3085.90000000001	24.7731691510045\\
3086.00000000001	24.7756067528596\\
3087.4	24.7708502024292\\
3087.5	24.7732866951678\\
3089.09999999999	24.7766412016056\\
3090.10000000001	24.7686233900718\\
3090.20000000003	24.771057826101\\
3091.60000000001	24.7630753307242\\
3091.70000000002	24.7655087651206\\
3091.99999999998	24.7695740758708\\
3094.09999999999	24.7527632344386\\
3094.20000000002	24.755195036034\\
3094.89999999998	24.7592891760905\\
3095.69999999998	24.7528910136314\\
3095.8	24.7553215543138\\
3096.79999999998	24.7473279731344\\
3096.89999999999	24.7497578301582\\
3097.70000000002	24.7465943572858\\
3097.9	24.7514525500323\\
3098.40000000002	24.747458447636\\
3098.50000000001	24.7498870457626\\
3099.70000000001	24.7435318407639\\
3099.80000000001	24.7459595470822\\
3100.40000000001	24.7411707789066\\
3100.50000000002	24.7435980132878\\
3101.00000000001	24.7396085260069\\
3101.09999999999	24.7420353411583\\
3101.69999999999	24.7372493390934\\
3101.79999999998	24.7396756826461\\
3103	24.7333311849441\\
3103.09999999999	24.7357566383088\\
3103.79999999999	24.7333999162344\\
3103.9	24.735824742268\\
3104.40000000003	24.7318408761475\\
3104.5	24.7342652837725\\
3105.29999999999	24.7278933470728\\
3105.7	24.7375877390688\\
3106.10000000001	24.7344021634151\\
3106.2	24.736825161768\\
3106.6	24.7400778961599\\
3106.9	24.7441261667203\\
3107.40000000003	24.7465808527755\\
3107.80000000002	24.7433958621577\\
3107.89999999999	24.7458172458172\\
3108.40000000003	24.7418368988258\\
3108.5	24.744257865277\\
3108.9	24.7410743004181\\
3109.10000000002	24.7459153479995\\
3109.50000000002	24.7491638795987\\
3110.69999999998	24.7428314259997\\
3110.90000000002	24.7476695596271\\
3111.19999999999	24.7517115032302\\
3111.6	24.7485297425844\\
3111.70000000002	24.7509480043705\\
3113.30000000002	24.7478640714332\\
3113.50000000002	24.7526978417266\\
3113.90000000003	24.7495183044316\\
3114	24.7519347484024\\
3114.90000000003	24.7447833065811\\
3114.99999999999	24.7471991268338\\
3115.30000000001	24.7448160749823\\
3115.39999999999	24.7472315840154\\
3115.70000000003	24.7512677322036\\
3116.09999999999	24.7480906231949\\
3116.49999999999	24.757748828852\\
3116.90000000002	24.7609881296118\\
3117.6	24.7586361741027\\
3117.8	24.7634625869977\\
3118.30000000001	24.7594920472037\\
3118.39999999998	24.7619047619048\\
3118.89999999998	24.7579352356525\\
3119.09999999999	24.7627596819697\\
3119.99999999998	24.7556168071536\\
3120.1	24.7580283315172\\
3120.60000000001	24.7604704072804\\
3120.99999999998	24.7637051039698\\
3122.79999999999	24.7526337698934\\
3122.9	24.7550432276657\\
3123.19999999998	24.7526654500048\\
3123.30000000002	24.7550745981943\\
3124.1	24.7487356763331\\
3124.19999999998	24.7511442563134\\
3124.59999999998	24.7479758056773\\
3124.89999999999	24.7552\\
3125.8	24.7480725551041\\
3125.9	24.7504798464491\\
3126.29999999999	24.7473132036847\\
3126.59999999999	24.7545335337576\\
3127.49999999999	24.7506074945645\\
3127.60000000002	24.7530133964255\\
3129.40000000001	24.7451669595782\\
3129.60000000001	24.7499760360418\\
3130.60000000002	24.7452646373016\\
3130.69999999999	24.747668327584\\
3131.19999999998	24.7437166671989\\
3131.5	24.7509260441947\\
3132.30000000002	24.7477972161921\\
3132.39999999998	24.7501995211492\\
3133.50000000003	24.7478937962727\\
3133.60000000001	24.7502951782238\\
3133.90000000001	24.7543075941289\\
3135.29999999998	24.7432544491931\\
3135.40000000001	24.7456546005422\\
3136.09999999999	24.7401313691729\\
3136.29999999998	24.7449304935595\\
3136.90000000002	24.7401976410583\\
3136.99999999999	24.7425966657104\\
3137.49999999999	24.7386537480877\\
3137.69999999999	24.7434508254191\\
3138.00000000003	24.7410853701284\\
3138.09999999999	24.7434835255879\\
3138.60000000001	24.739541848536\\
3139.00000000001	24.7491319167914\\
3139.49999999998	24.7451904701236\\
3139.60000000002	24.7475873491098\\
3140.89999999998	24.7373447946514\\
3141.09999999999	24.7421367630205\\
3141.7	24.7501432299955\\
3142.50000000001	24.7438426780373\\
3142.59999999998	24.7462373118656\\
3143.19999999999	24.7415136957974\\
3143.39999999999	24.7463018927947\\
3143.89999999998	24.7423664122137\\
3143.99999999998	24.7447600267167\\
3144.60000000001	24.7495786561516\\
3146.10000000001	24.7377789078889\\
3146.20000000003	24.7401709945015\\
3147.39999999999	24.7339158061954\\
3147.50000000003	24.7363070275766\\
3147.89999999998	24.7395171537484\\
3148.20000000003	24.7371597370009\\
3148.29999999999	24.7395502477449\\
3148.60000000003	24.7371931273224\\
3148.7	24.7395833333333\\
3149.60000000001	24.742038924342\\
3150	24.7388971778674\\
3150.10000000002	24.7412862675386\\
3150.9	24.7350047603935\\
3151.00000000002	24.7373932912316\\
3151.30000000001	24.7350383956337\\
3151.40000000002	24.7374266222434\\
3151.80000000002	24.7342872553063\\
3151.89999999998	24.7366751269036\\
3152.6	24.7311827956989\\
3152.79999999999	24.7359573725776\\
3153.19999999999	24.7328195858307\\
3153.30000000001	24.7352064438384\\
3153.9	24.7305009511731\\
3154.10000000002	24.7352736034494\\
3154.50000000002	24.7416471184936\\
3155.00000000001	24.737726221039\\
3155.10000000001	24.7401115618661\\
3155.8	24.7377927057258\\
3155.90000000002	24.7401774397972\\
3156.39999999999	24.7425946459686\\
3156.70000000001	24.7402432843386\\
3156.90000000002	24.7450110864745\\
3157.19999999999	24.7489943939442\\
3157.80000000002	24.7442920928465\\
3157.90000000002	24.7466751108296\\
3158.20000000002	24.7506569990185\\
3158.59999999998	24.747522715041\\
3158.69999999998	24.7499050272255\\
3159.09999999998	24.7467713345151\\
3159.20000000001	24.7491532934511\\
3159.99999999998	24.7460523401158\\
3160.10000000003	24.7484336434403\\
3160.39999999998	24.7460844803038\\
3160.50000000002	24.7484654812377\\
3161.00000000002	24.7508778589731\\
3161.7	24.7453981909039\\
3161.90000000002	24.7501581277672\\
3162.19999999998	24.7478101381906\\
3162.29999999998	24.7501897293195\\
3163.30000000001	24.7455269646583\\
3163.59999999998	24.7526630211461\\
3164.00000000002	24.7558547454252\\
3164.99999999999	24.7480332374964\\
3165.1	24.7504107165424\\
3165.50000000001	24.7536012130402\\
3165.79999999998	24.7512555671373\\
3166	24.7560089700262\\
3166.80000000001	24.7497552811898\\
3166.9	24.7521313545943\\
3167.60000000001	24.7466616156833\\
3167.80000000001	24.7514126077212\\
3168.19999999998	24.7482877252785\\
3168.30000000003	24.7506627951016\\
3168.59999999998	24.7483195001105\\
3168.7	24.750694269124\\
3169.90000000001	24.7413249211357\\
3170	24.7436989369421\\
3170.59999999998	24.7390166209354\\
3170.70000000001	24.7413901854422\\
3171.1	24.7477295660949\\
3171.6	24.7438282309172\\
3171.70000000002	24.7462008953906\\
3172.30000000002	24.7541293657799\\
3172.79999999999	24.750228497589\\
3172.90000000001	24.7526000630318\\
3173.29999999998	24.7557824415453\\
3174.10000000003	24.7495431919854\\
3174.2	24.7519138077686\\
3174.6	24.7487951617476\\
3174.8	24.7535355444266\\
3175.3	24.7496378409019\\
3175.40000000001	24.7520075578649\\
3175.80000000001	24.748890078403\\
3175.9	24.7512594458438\\
3176.20000000001	24.7489217013506\\
3176.30000000002	24.7512907694245\\
3176.89999999999	24.7560591753226\\
3177.20000000001	24.7537217134045\\
3177.30000000001	24.7560898848115\\
3178.4	24.7475224162341\\
3178.49999999998	24.7498898886302\\
3178.89999999998	24.7467757156339\\
3179.00000000001	24.7491428391683\\
3179.39999999999	24.7460292498821\\
3179.5	24.7483960246572\\
3180.50000000002	24.740614978306\\
3180.69999999998	24.745347082495\\
3180.99999999998	24.7493005564113\\
3181.29999999999	24.7469667442007\\
3181.4	24.7493320760648\\
3181.8	24.7462208114649\\
3181.90000000001	24.7485857950974\\
3182.39999999999	24.7572663000786\\
3182.70000000001	24.7549327636044\\
3182.79999999998	24.7572968048007\\
3183.2	24.7604686960073\\
3184.50000000001	24.7566413364316\\
3184.60000000001	24.7590039878168\\
3184.99999999998	24.7558946343914\\
3185.1	24.7582569383398\\
3185.5	24.7551481667504\\
3185.70000000001	24.7598719316969\\
3186.40000000002	24.7669857210105\\
3187.79999999998	24.7717933435804\\
3188.10000000003	24.7694623925726\\
3188.29999999999	24.7741814076026\\
3188.70000000002	24.7773457099849\\
3188.99999999998	24.7750148944843\\
3189.30000000001	24.7820906753621\\
3190.49999999998	24.77277001191\\
3190.7	24.7774852701517\\
3191.10000000001	24.7743795437453\\
3191.19999999998	24.7767367530474\\
3191.59999999999	24.7798978600746\\
3191.90000000002	24.7775689223058\\
3191.99999999998	24.7799254409323\\
3192.70000000002	24.7744926083688\\
3192.79999999998	24.7768486329043\\
3193.30000000001	24.7792321663431\\
3193.9	24.7745773324984\\
3194.00000000002	24.7769324692402\\
3195.40000000003	24.7660772961978\\
3195.49999999999	24.768431593441\\
3196.00000000002	24.7739432433278\\
3197.1	24.7654197422745\\
3197.19999999998	24.767772808307\\
3197.80000000001	24.7631258013071\\
3197.90000000002	24.765478424015\\
3198.30000000001	24.7686343171586\\
3198.80000000002	24.7647628872425\\
3198.90000000001	24.767114723351\\
3199.60000000002	24.7616964090384\\
3199.9	24.76875\\
3200.20000000001	24.7664281473612\\
3200.29999999999	24.7687789026372\\
3200.90000000001	24.7641362074352\\
3201.00000000001	24.7664865202587\\
3201.49999999998	24.7626186906547\\
3201.59999999999	24.7649686104257\\
3202.10000000001	24.761101742552\\
3202.29999999999	24.7658006495129\\
3202.60000000003	24.7697255440722\\
3203.00000000001	24.7666323249352\\
3203.10000000002	24.768981018981\\
3204.09999999999	24.7612508582486\\
3204.29999999999	24.7659468231182\\
3204.7	24.7628557164254\\
3204.8	24.7652032824737\\
3205.30000000001	24.7675797092407\\
3205.60000000002	24.7715007642637\\
3206.00000000001	24.7746483266274\\
3207.50000000002	24.7630627260257\\
3207.59999999998	24.7654082364311\\
3208.00000000003	24.7685545961784\\
3208.80000000002	24.7654959643492\\
3208.9	24.7678404487379\\
3209.30000000002	24.7709852308843\\
3209.59999999999	24.768669969156\\
3209.8	24.773357425465\\
3210.09999999999	24.7710423026603\\
3210.20000000001	24.7733856648911\\
3211.20000000002	24.765671223492\\
3211.29999999999	24.768013950302\\
3211.80000000001	24.7703851302967\\
3212.40000000001	24.7657587548638\\
3212.59999999998	24.7704423070937\\
3213.80000000002	24.7611935654501\\
3213.89999999999	24.7635345364032\\
3214.29999999999	24.766674962668\\
3214.6	24.7643637042337\\
3214.79999999998	24.7690441382314\\
3215.70000000001	24.7652217177685\\
3215.80000000002	24.767561180385\\
3216.50000000002	24.7621712367096\\
3216.59999999998	24.7645102123294\\
3217.00000000001	24.7676478816325\\
3218.10000000002	24.7591821515133\\
3218.2	24.761520057173\\
3218.70000000002	24.7576736672052\\
3218.79999999998	24.7600111839448\\
3219.60000000001	24.7631766934808\\
3220.10000000002	24.7593317185268\\
3220.2	24.761668167562\\
3220.69999999998	24.7640337804272\\
3221.50000000003	24.7609883287807\\
3221.59999999999	24.7633237110842\\
3222.1	24.768791508907\\
3223.29999999998	24.7657752683502\\
3223.39999999998	24.7681091980766\\
3225.29999999999	24.7628201153345\\
3225.39999999999	24.7651526895055\\
3227.09999999999	24.7552057511155\\
3227.20000000001	24.7575372602485\\
3227.69999999998	24.7629964681827\\
3228.00000000003	24.7668907406834\\
3229.19999999998	24.7576874245192\\
3229.3	24.7600173406825\\
3230.5	24.7570110815328\\
3230.60000000001	24.759340081097\\
3231.49999999999	24.7524446094814\\
3231.70000000001	24.7571013057739\\
3233.60000000002	24.7487398336271\\
3233.69999999999	24.75106685633\\
3235.09999999999	24.7403560830861\\
3235.20000000001	24.7426822860322\\
3235.69999999998	24.7388590147722\\
3235.8	24.7411848326586\\
3236.09999999998	24.7388912922564\\
3236.30000000001	24.7435422073909\\
3236.99999999999	24.7412807760032\\
3237.2	24.7459302505174\\
3238.39999999998	24.7398486953837\\
3238.49999999998	24.7421725436917\\
3238.89999999998	24.7391170114233\\
3238.99999999999	24.7414405236022\\
3239.79999999999	24.735331337387\\
3239.90000000003	24.7376543209877\\
3240.20000000003	24.7353640095053\\
3240.30000000001	24.737686705345\\
3240.59999999998	24.7353966735582\\
3240.69999999998	24.7377190817082\\
3241.70000000002	24.74551175273\\
3242.29999999999	24.740932642487\\
3242.39999999998	24.7432536622976\\
3242.69999999999	24.7409645984951\\
3242.79999999998	24.7432853310309\\
3243.29999999999	24.7394709255719\\
3243.39999999998	24.7417912748574\\
3243.80000000002	24.738740405068\\
3243.89999999998	24.7410604192355\\
3244.69999999998	24.7380424063116\\
3244.89999999999	24.7426810477658\\
3245.30000000003	24.7396314784002\\
3245.40000000002	24.7419503928516\\
3245.70000000002	24.7396635652227\\
3245.80000000002	24.7419821929203\\
3246.70000000002	24.7382037698657\\
3246.80000000002	24.7405217284179\\
3247.39999999998	24.7359507313318\\
3247.60000000001	24.7405856452259\\
3247.99999999999	24.7375388688772\\
3248.40000000001	24.7468062182546\\
3248.90000000002	24.7429978454909\\
3249	24.7453140869779\\
3249.30000000003	24.743029482366\\
3249.40000000002	24.7453454377597\\
3249.89999999999	24.7415384615385\\
3250.00000000003	24.7438540352605\\
3250.5	24.7400479911401\\
3250.60000000003	24.7423631833144\\
3251.79999999998	24.7455333804853\\
3252.29999999999	24.7417291846021\\
3252.4	24.7440430438125\\
3252.90000000001	24.7402397786658\\
3253	24.7425532568934\\
3253.60000000001	24.7379905953223\\
3253.70000000002	24.740303644969\\
3254.40000000002	24.7349823321555\\
3254.5	24.737294905672\\
3255.40000000002	24.7304561511289\\
3255.49999999998	24.7327681533358\\
3255.99999999998	24.7289702404717\\
3256.09999999998	24.7312818622935\\
3257.09999999998	24.7236890580867\\
3257.20000000001	24.7260000614005\\
3258.40000000001	24.7168942765076\\
3258.50000000003	24.7192045663782\\
3258.80000000003	24.7169290251312\\
3258.89999999999	24.7192390303774\\
3259.8	24.7124144912421\\
3260	24.7170332198399\\
3260.39999999998	24.7232019628891\\
3260.89999999998	24.7286108555658\\
3261.20000000003	24.7324686474719\\
3261.8	24.7279193108311\\
3261.99999999999	24.7325342570737\\
3262.40000000001	24.7295019157088\\
3262.49999999998	24.7318089866977\\
3263.60000000001	24.7265373655667\\
3263.80000000002	24.7311498514048\\
3264.30000000002	24.7334885430707\\
3264.60000000001	24.7312157319202\\
3264.7	24.7335211957854\\
3265.1	24.7366164400343\\
3266.00000000002	24.7420470898013\\
3266.69999999998	24.7398065385086\\
3266.80000000003	24.7421102574306\\
3267.30000000002	24.7383240497031\\
3267.39999999999	24.7406273909717\\
3267.89999999999	24.7368421052632\\
3267.99999999999	24.7391450690003\\
3269.40000000001	24.7285517663251\\
3269.5	24.7308539270859\\
3269.80000000002	24.7285849720175\\
3269.90000000002	24.7308868501529\\
3270.4	24.7271059471029\\
3270.49999999998	24.7294074481746\\
3270.79999999999	24.7271393194534\\
3270.9	24.7294405380618\\
3272.3	24.7219166361081\\
3272.40000000002	24.7242169595111\\
3272.80000000001	24.7273060588469\\
3273.60000000002	24.7212634022666\\
3273.90000000001	24.728161270617\\
3274.60000000001	24.7320365224295\\
3276.40000000002	24.7215016023195\\
3276.49999999999	24.7237990600012\\
3276.80000000001	24.7276389270347\\
3277.1	24.7253753203955\\
3277.20000000002	24.7276721691636\\
3277.6	24.7246544833267\\
3277.70000000002	24.726951003722\\
3279.39999999999	24.7141332520201\\
3279.60000000003	24.7187242735616\\
3280.00000000001	24.7218072619737\\
3280.8	24.7249230394099\\
3281.49999999999	24.7196489517309\\
3281.59999999999	24.7219428954505\\
3282.00000000001	24.7189299533835\\
3282.1	24.721223569557\\
3282.99999999998	24.7174926136883\\
3283.10000000002	24.7197855750487\\
3283.50000000003	24.7167742721403\\
3283.70000000001	24.7213594006943\\
3284.10000000001	24.7244382193533\\
3284.90000000002	24.7214611872146\\
3285	24.723752701592\\
3285.6	24.71923790973\\
3285.70000000002	24.7215290035912\\
3286.79999999999	24.7193404119383\\
3287.00000000002	24.7239207812357\\
3287.70000000002	24.721698400146\\
3287.79999999999	24.7239879558381\\
3288.40000000002	24.7194769651817\\
3288.49999999999	24.7217661010764\\
3288.90000000002	24.7248403770143\\
3289.39999999998	24.7210822313422\\
3289.49999999998	24.7233706225681\\
3290.40000000001	24.7196474699894\\
3290.50000000001	24.7219352093843\\
3290.90000000001	24.7250075964752\\
3292.09999999999	24.719032865561\\
3292.4	24.7258921791951\\
3293.50000000001	24.7206703910615\\
3293.6	24.7229559462003\\
3294.30000000002	24.7207382224381\\
3294.6	24.7275928005585\\
3296.30000000001	24.7178740444121\\
3296.39999999999	24.7201577430608\\
3296.69999999998	24.7179082746906\\
3296.80000000003	24.7201916952289\\
3297.3	24.7225086431734\\
3297.60000000001	24.7202595748552\\
3297.7	24.7225423009279\\
3298.10000000002	24.7256079073434\\
3298.7	24.7211107069237\\
3298.89999999998	24.7256744468021\\
3299.99999999998	24.7204630162722\\
3300.10000000002	24.7227440761166\\
3300.40000000002	24.7204968944099\\
3300.50000000002	24.722777676786\\
3301.39999999998	24.7190670907163\\
3301.60000000001	24.72362722234\\
3301.99999999999	24.7206323248842\\
3302.19999999998	24.7251915331738\\
3302.90000000002	24.7199515591886\\
3303.10000000003	24.724509566481\\
3304.09999999998	24.7291326190909\\
3304.50000000001	24.7321914906494\\
3305.20000000001	24.7269536804526\\
3305.30000000002	24.7292309554063\\
3305.69999999998	24.7322887047008\\
3306.1	24.7292964732926\\
3306.30000000002	24.7338495039923\\
3307.89999999999	24.7218863361548\\
3308.00000000001	24.7241619056256\\
3308.99999999998	24.7197123084827\\
3309.2	24.7242619285045\\
3309.70000000001	24.7265695812436\\
3310.00000000001	24.7243285701338\\
3310.09999999999	24.7266026221981\\
3310.39999999998	24.7304032623471\\
3311.2	24.7334883580467\\
3312.00000000001	24.739591195918\\
3312.49999999999	24.7418945843144\\
3313.09999999999	24.7374139804419\\
3313.20000000002	24.7396855099146\\
3314.5	24.7329994569481\\
3314.6	24.7352701601955\\
3315.69999999999	24.7270643585259\\
3315.80000000002	24.7293344190114\\
3316.80000000002	24.7218788627936\\
3316.89999999999	24.7241483268013\\
3317.59999999999	24.7189317900955\\
3317.69999999999	24.721200795708\\
3318.39999999999	24.7159861383155\\
3318.50000000001	24.7182546857108\\
3318.89999999998	24.7213015968665\\
3319.2	24.7190672732203\\
3319.29999999998	24.7213351810568\\
3319.8	24.7176119762643\\
3320.00000000001	24.7221469232854\\
3320.4	24.7191687998795\\
3320.49999999999	24.721435885081\\
3321.2	24.7192364435613\\
3321.29999999998	24.7215029806708\\
3321.70000000002	24.7185261003071\\
3321.8	24.7207923176495\\
3322.60000000003	24.7148403406868\\
3322.70000000001	24.7171060551342\\
3323.60000000002	24.7104130938412\\
3323.70000000002	24.7126782598231\\
3324	24.710447940796\\
3324.09999999999	24.7127128331629\\
3324.50000000002	24.7097395175359\\
3324.59999999998	24.7120040905946\\
3325.79999999998	24.7091012958898\\
3325.90000000002	24.7113650030066\\
3326.30000000003	24.714405964406\\
3326.80000000002	24.7106916348553\\
3326.89999999999	24.7129546137662\\
3327.99999999998	24.7107959496409\\
3328.09999999999	24.7130581094886\\
3330	24.7079667277259\\
3330.09999999998	24.7102276139571\\
3331.09999999999	24.7028097982709\\
3331.2	24.7050700927566\\
3331.50000000003	24.7088486012727\\
3332.90000000001	24.6984698469847\\
3332.99999999999	24.7007290510336\\
3333.3	24.6985060298794\\
3333.40000000002	24.7007649617519\\
3333.70000000001	24.6985422040914\\
3333.80000000002	24.7008008638531\\
3334.59999999998	24.6948751012085\\
3334.69999999999	24.697133261365\\
3335.50000000001	24.6912099772155\\
3335.60000000001	24.693467637977\\
3336.1	24.7017564894191\\
3336.39999999998	24.6995354413307\\
3336.59999999999	24.7040489106003\\
3336.90000000001	24.7018279892119\\
3337.00000000003	24.7040843846453\\
3337.39999999999	24.7071161048689\\
3337.69999999999	24.7048954401103\\
3337.80000000002	24.7071512028521\\
3338.30000000002	24.7034507548526\\
3338.39999999999	24.705706155459\\
3338.80000000002	24.7027464134895\\
3338.89999999998	24.7050014974543\\
3339.60000000001	24.6998233374255\\
3339.70000000001	24.7020779687406\\
3340.00000000001	24.7058471303254\\
3341.00000000001	24.6984526054294\\
3341.20000000002	24.7029599257774\\
3341.9	24.6977857570317\\
3342.00000000003	24.7000388976991\\
3342.39999999998	24.7030665669409\\
3343.4	24.6986690593689\\
3343.50000000001	24.7009211628185\\
3344	24.6972279537095\\
3344.1	24.6994796961904\\
3344.40000000002	24.7032441321573\\
3344.80000000002	24.7002899937218\\
3344.89999999998	24.7025411061285\\
3346	24.694420369983\\
3346.09999999998	24.696670850517\\
3346.4	24.7004332885104\\
3346.90000000001	24.6967433522558\\
3347	24.6989931582564\\
3347.99999999999	24.6946029091126\\
3348.1	24.696852039902\\
3348.40000000002	24.694639390772\\
3348.49999999998	24.6968882518067\\
3349.20000000002	24.6947123279491\\
3349.3	24.6969606496686\\
3349.90000000002	24.7014925373134\\
3351.00000000002	24.6933842618842\\
3351.09999999999	24.6956314156123\\
3351.4	24.693420856333\\
3351.50000000003	24.6956677407805\\
3352.00000000001	24.6979505384684\\
3352.49999999999	24.6942671359542\\
3352.70000000003	24.6987592460033\\
3353.19999999998	24.7040229028122\\
3353.59999999998	24.7010764230551\\
3353.70000000002	24.7033216053432\\
3355	24.6967303508092\\
3355.10000000003	24.6989747257988\\
3355.70000000002	24.6945586745336\\
3355.80000000002	24.696802646086\\
3356.40000000003	24.7013257857888\\
3356.69999999999	24.699118207817\\
3356.8	24.7013613750782\\
3357.20000000002	24.7043755398684\\
3357.70000000001	24.7006968848651\\
3357.80000000001	24.7029393370857\\
3358.10000000001	24.7007325352868\\
3358.20000000002	24.7029747193521\\
3358.6	24.705987435615\\
3359.10000000001	24.7023100738271\\
3359.20000000002	24.7045515434763\\
3359.49999999998	24.7023455173235\\
3359.60000000001	24.7045867190523\\
3360.19999999999	24.7001755795614\\
3360.40000000003	24.7046570450826\\
3361.30000000002	24.7010174332124\\
3361.4	24.7032574743418\\
3361.89999999998	24.6995835812017\\
3362.00000000003	24.7018232652211\\
3362.69999999999	24.6996550493636\\
3362.79999999999	24.7018941984597\\
3363.30000000002	24.7041684010228\\
3364.00000000002	24.707945661544\\
3364.29999999998	24.7057424800856\\
3364.39999999998	24.7079803834151\\
3365.50000000001	24.702876158783\\
3365.60000000001	24.7051133493775\\
3365.90000000003	24.7029114676174\\
3365.99999999999	24.7051483913134\\
3367.50000000003	24.6941441976482\\
3367.59999999998	24.696380318912\\
3367.99999999999	24.6993854101719\\
3368.40000000003	24.6964524268963\\
3368.49999999998	24.6986878822063\\
3368.79999999999	24.6964884680459\\
3368.89999999999	24.6987236568715\\
3369.39999999999	24.6950586140377\\
3369.49999999999	24.6972934472934\\
3370.19999999998	24.7010651870753\\
3370.79999999999	24.6966685454923\\
3371.00000000002	24.7011361276735\\
3371.30000000002	24.6989381265943\\
3371.60000000002	24.7056381054068\\
3371.99999999999	24.7027075116396\\
3372.10000000001	24.7049403949944\\
3372.40000000002	24.7086730911787\\
3372.80000000001	24.7057428325773\\
3372.89999999999	24.7079750963534\\
3373.3	24.7050453548349\\
3373.59999999998	24.7117408186857\\
3374.50000000002	24.7051502400285\\
3374.70000000001	24.7096124214768\\
3375.30000000002	24.7052201220596\\
3375.40000000002	24.7074507480373\\
3376.3	24.7008648264424\\
3376.39999999999	24.703094920776\\
3376.89999999998	24.6994373704471\\
3377.09999999998	24.7038967191757\\
3377.59999999999	24.7002398081535\\
3377.69999999999	24.7024690627035\\
3378.09999999999	24.7054644485229\\
3378.40000000002	24.7032706822554\\
3378.49999999997	24.7054993192447\\
3379.10000000002	24.7129498106061\\
3379.40000000003	24.7166740642107\\
3379.9	24.7248520710059\\
3381	24.7168081393629\\
3381.10000000002	24.7190346622501\\
3381.59999999998	24.7153798385427\\
3381.69999999999	24.7176060086345\\
3382.19999999999	24.7198651804985\\
3382.69999999998	24.7162114224902\\
3383.00000000001	24.7228872927197\\
3383.60000000002	24.7185034134232\\
3383.69999999999	24.7207281754241\\
3385.20000000002	24.7097746137713\\
3385.3	24.7119985821469\\
3385.69999999998	24.7149861184949\\
3386.39999999998	24.7098774545991\\
3386.49999999999	24.7121006319022\\
3386.90000000002	24.7091821671095\\
3387.00000000003	24.7114050367571\\
3387.49999999998	24.7136615893258\\
3388.00000000001	24.7159174758714\\
3388.39999999999	24.7129998524421\\
3388.49999999999	24.71522162545\\
3388.90000000003	24.7123045146061\\
3389	24.7145259803488\\
3389.40000000003	24.7116093819147\\
3389.5	24.7138305404768\\
3390.99999999999	24.7087965556899\\
3391.09999999998	24.7110167492333\\
3392.09999999998	24.7066800306586\\
3392.30000000002	24.7111189718194\\
3393.20000000002	24.716352812896\\
3393.7	24.7127114149331\\
3393.80000000002	24.7149297268629\\
3394.19999999999	24.7120172053148\\
3394.30000000001	24.7142352109357\\
3394.99999999998	24.7091396424258\\
3395.10000000002	24.7113572101791\\
3395.49999999999	24.717281187419\\
3395.89999999997	24.7143698468787\\
3396.09999999998	24.7188033684706\\
3396.60000000002	24.7151647186976\\
3396.69999999999	24.7173810645313\\
3397.39999999999	24.7122884473878\\
3397.50000000002	24.714504356016\\
3398.19999999999	24.7094135302946\\
3398.30000000001	24.7116290018832\\
3398.59999999999	24.7094477300144\\
3398.80000000003	24.7138780193592\\
3401.00000000002	24.6978918585164\\
3401.09999999999	24.7001058449959\\
3401.59999999998	24.6964752917659\\
3401.79999999999	24.7009024368735\\
3403.00000000003	24.6980694073051\\
3403.09999999997	24.7002820874471\\
3404	24.6966892864487\\
3404.09999999998	24.6989013571471\\
3404.79999999998	24.6938236071544\\
3404.90000000002	24.6960352422907\\
3405.30000000002	24.699007458742\\
3405.79999999999	24.6953815437917\\
3405.90000000002	24.697592483852\\
3407.89999999999	24.6830985915493\\
3408.00000000003	24.6853085296793\\
3408.99999999998	24.6810008506644\\
3409.20000000001	24.6854192942833\\
3409.6	24.6825233891545\\
3409.69999999998	24.6847322423603\\
3410.20000000001	24.6811130985544\\
3410.59999999998	24.689946345325\\
3411.19999999999	24.6856037287837\\
3411.29999999998	24.6878114557073\\
3411.60000000001	24.6915027698801\\
3412.49999999997	24.693781867198\\
3413.10000000003	24.6894409937888\\
3413.2	24.6916473793689\\
3413.49999999998	24.6894773845793\\
3413.6	24.6916835105604\\
3414.00000000002	24.688790603673\\
3414.10000000001	24.6909964266885\\
3414.50000000002	24.6881040238974\\
3414.70000000002	24.6925149349889\\
3414.99999999998	24.6903458171064\\
3415.1	24.6925509486999\\
3415.39999999999	24.6903820816864\\
3415.49999999998	24.6925869539759\\
3416.69999999998	24.6897682041676\\
3416.80000000002	24.6919722555533\\
3417.30000000003	24.6883595716041\\
3417.40000000001	24.6905632772495\\
3419.00000000001	24.6819338422392\\
3419.10000000002	24.6841366401497\\
3419.69999999999	24.6798058365986\\
3419.80000000002	24.6820082458551\\
3421.50000000001	24.6872808043021\\
3421.89999999999	24.6843950905903\\
3422.00000000001	24.6865959498554\\
3423.30000000001	24.6801425483438\\
3423.39999999999	24.6823426318096\\
3423.70000000003	24.6801799170512\\
3423.79999999998	24.682379742399\\
3425.39999999997	24.6708509706612\\
3425.59999999998	24.6752488542488\\
3426.00000000002	24.6782055398266\\
3426.50000000002	24.6746045642911\\
3426.60000000002	24.676802754837\\
3427.60000000002	24.6725209324037\\
3427.80000000001	24.6769158960296\\
3428.7	24.6704386374242\\
3428.80000000002	24.6726355390942\\
3429.1	24.6704770792022\\
3429.20000000002	24.6726737235004\\
3429.50000000003	24.6763470958712\\
3429.79999999998	24.6800198256509\\
3430.60000000002	24.6830092983939\\
3432.10000000002	24.6897033972379\\
3432.50000000002	24.6868263124162\\
3432.59999999998	24.6890203047164\\
3433.19999999998	24.6847056767541\\
3433.40000000003	24.6890927624873\\
3434.4	24.6848158392779\\
3434.69999999997	24.6913939676255\\
3435.6	24.6878365398609\\
3435.69999999998	24.6900285231969\\
3436.09999999999	24.6929747977417\\
3437	24.6865089755899\\
3437.20000000001	24.6908911063916\\
3437.50000000003	24.6887363276705\\
3437.59999999998	24.6909270733339\\
3438.00000000002	24.6880544486781\\
3438.10000000001	24.6902448955849\\
3440.10000000003	24.6817045520609\\
3440.3	24.6860830136031\\
3441.29999999999	24.6789097460336\\
3441.40000000002	24.681098358274\\
3441.7	24.6789470625835\\
3441.90000000003	24.6833236490413\\
3442.59999999998	24.6783048188922\\
3442.69999999999	24.6804926222842\\
3443.4	24.6783795556846\\
3443.5	24.6805668486468\\
3443.80000000002	24.6784169110601\\
3443.89999999999	24.6806039488966\\
3444.50000000003	24.6763049410672\\
3444.60000000003	24.6784915957848\\
3446.19999999999	24.6670342106027\\
3446.29999999998	24.6692200557103\\
3448.09999999998	24.6563424395337\\
3448.19999999999	24.6585273903083\\
3448.89999999999	24.6535227602204\\
3449.00000000001	24.6557072859587\\
3449.50000000001	24.652133580705\\
3449.59999999998	24.6543177667623\\
3451.1	24.6436022253129\\
3451.2	24.6457856459885\\
3451.89999999998	24.6407879490151\\
3452.00000000002	24.6429709452217\\
3452.70000000002	24.6408711770158\\
3452.79999999999	24.6430536650352\\
3453.3	24.6452771182024\\
3453.69999999999	24.6482135618739\\
3454.19999999999	24.6446458037808\\
3454.30000000001	24.6468272348309\\
3455.20000000001	24.6490898040691\\
3456.3	24.641245226247\\
3456.50000000001	24.645605508303\\
3457.2	24.6435079397218\\
3457.3	24.6456875108463\\
3457.89999999998	24.6414112203586\\
3458	24.6435904109193\\
3458.40000000001	24.6407402052913\\
3458.59999999999	24.6450978691416\\
3459.89999999999	24.6473988439306\\
3460.3	24.653219281008\\
3460.60000000001	24.6510821510099\\
3460.70000000002	24.6532593619972\\
3461.1	24.6504102623368\\
3461.20000000003	24.6525871782278\\
3462.19999999999	24.6454668861739\\
3462.5	24.6519956102351\\
3462.89999999999	24.6549234767543\\
3463.50000000002	24.6506525002887\\
3463.60000000003	24.6528279007997\\
3464.59999999997	24.6457124714983\\
3464.90000000001	24.6522366522367\\
3465.49999999998	24.6479686057248\\
3465.7	24.6523169253852\\
3466	24.6501832030236\\
3466.40000000001	24.6588778306649\\
3467.00000000001	24.6546104813821\\
3467.09999999999	24.6567835717582\\
3467.50000000002	24.6539393240281\\
3467.60000000002	24.6561121204256\\
3468.19999999999	24.6633797537699\\
3469.79999999999	24.6577711173233\\
3470	24.6621134837613\\
3470.69999999997	24.6600207444969\\
3470.79999999998	24.6621913624708\\
3471.10000000001	24.6600599216409\\
3471.20000000003	24.6622302883646\\
3472.10000000002	24.6558377973619\\
3472.2	24.6580076606284\\
3472.60000000003	24.655167448959\\
3472.70000000002	24.65733701912\\
3473.29999999999	24.6530776760523\\
3473.40000000001	24.6552468691522\\
3473.99999999998	24.6509887452865\\
3474.09999999998	24.653157561453\\
3474.69999999998	24.6489006561529\\
3474.80000000001	24.6510690955135\\
3476.19999999999	24.6411414434888\\
3476.30000000001	24.6433091704062\\
3477.29999999997	24.6477253120147\\
3477.60000000001	24.6513500301924\\
3478	24.6485149938185\\
3478.10000000001	24.6506813869243\\
3478.40000000002	24.6485554118154\\
3478.59999999999	24.6528875729439\\
3481.60000000002	24.6460062613091\\
3481.70000000001	24.64817048653\\
3482.09999999999	24.6453391534088\\
3482.19999999998	24.6475030870402\\
3483.80000000003	24.6390539338098\\
3483.89999999998	24.6412169919633\\
3484.29999999998	24.638388244748\\
3484.39999999998	24.6405510116229\\
3485.1	24.6384712498565\\
3485.2	24.6406335179181\\
3486.00000000002	24.6435845213849\\
3486.30000000001	24.6414639743001\\
3486.40000000001	24.6436254123046\\
3487.4	24.6480286738351\\
3489.79999999999	24.636809077624\\
3490.10000000003	24.6432869176552\\
3490.59999999998	24.6397570687828\\
3490.79999999998	24.6440745939443\\
3491.09999999999	24.6476856095325\\
3491.6	24.6441561417075\\
3491.70000000002	24.6463142218913\\
3492.10000000001	24.64349120898\\
3492.19999999999	24.6456489992269\\
3492.60000000002	24.6428264666304\\
3492.70000000003	24.6449839670179\\
3493	24.6428673670951\\
3493.09999999998	24.6450246192603\\
3494.60000000002	24.6344464474776\\
3494.69999999998	24.6366029529587\\
3495.2	24.6388006751924\\
3496.5	24.6353600640622\\
3496.8	24.641825616975\\
3497.2	24.6390072341521\\
3497.40000000002	24.6433166547534\\
3497.79999999999	24.6404985848652\\
3497.90000000001	24.6426529445397\\
3498.7	24.6370184063107\\
3498.79999999999	24.6391723112978\\
3499.69999999998	24.6328361620664\\
3499.79999999998	24.6349895711306\\
3500.19999999999	24.6378881810131\\
3501.1	24.6344110590655\\
3501.2	24.6365635621055\\
3502.00000000002	24.6309357242797\\
3502.2	24.635239699626\\
3503.80000000002	24.6268443734125\\
3503.9	24.62899543379\\
3504.39999999998	24.6340419460693\\
3504.70000000001	24.6319333485506\\
3504.79999999998	24.6340837113755\\
3506.1	24.6278021789972\\
3506.30000000001	24.6321013004791\\
3507.30000000001	24.6250784056566\\
3507.40000000002	24.6272273699216\\
3508.79999999998	24.6316509447405\\
3509.10000000003	24.6295451954862\\
3509.30000000001	24.6338405425429\\
3509.69999999999	24.6310331072996\\
3509.89999999998	24.6353276353276\\
3510.40000000001	24.6375160233585\\
3510.90000000002	24.642551979493\\
3511.30000000002	24.6454405650168\\
3511.79999999999	24.6419317178735\\
3511.89999999998	24.6440774487472\\
3512.90000000001	24.6370623398804\\
3512.99999999999	24.6392075375025\\
3515.99999999999	24.6210289809732\\
3516.09999999998	24.6231727433024\\
3516.89999999998	24.6175717941427\\
3517.00000000002	24.6197151061954\\
3518.40000000002	24.6127611197954\\
3518.5	24.6149036548627\\
3518.79999999998	24.6128051379693\\
3518.89999999998	24.6149474282467\\
3519.29999999998	24.6121497982611\\
3519.5	24.6164336856461\\
3520.30000000002	24.6136802636064\\
3520.50000000003	24.6179628472419\\
3520.80000000001	24.6215456275384\\
3522.00000000003	24.62451378439\\
3522.29999999997	24.6224165341812\\
3522.40000000002	24.624556422995\\
3523.1	24.619663941871\\
3523.19999999999	24.6218034229274\\
3524	24.6275644845492\\
3524.30000000001	24.625468164794\\
3524.40000000002	24.6276067527309\\
3525.5	24.6312684365782\\
3526.30000000002	24.6256805807622\\
3526.40000000001	24.6278179498086\\
3526.89999999998	24.6243266231925\\
3526.99999999998	24.6264636670352\\
3528.4	24.619526711067\\
3528.50000000003	24.621662982486\\
3529.09999999998	24.6288110619971\\
3529.40000000002	24.6323841903952\\
3529.69999999999	24.6302906680265\\
3529.8	24.6324258477577\\
3530.10000000003	24.6303325590618\\
3530.19999999999	24.6324674956803\\
3530.7	24.6289792681545\\
3530.80000000001	24.6311138803138\\
3531.50000000001	24.637558047344\\
3531.80000000001	24.6354653302755\\
3531.90000000001	24.6375990939977\\
3533.09999999999	24.6320615872297\\
3533.3	24.6363276164601\\
3534.10000000001	24.6392394318375\\
3534.40000000001	24.6371481114726\\
3534.5	24.6392802580207\\
3534.80000000003	24.6428470395202\\
3535.59999999999	24.6372712617021\\
3535.7	24.6394026811471\\
3536.50000000003	24.6423118249166\\
3536.90000000002	24.6451795306757\\
3537.8	24.6473896944515\\
3538.29999999998	24.6439068505539\\
3538.50000000001	24.6481659413327\\
3539.19999999998	24.6461164637075\\
3539.30000000002	24.6482454653331\\
3540.09999999999	24.6426755550534\\
3540.2	24.6448041126458\\
3540.59999999999	24.64201993956\\
3540.69999999998	24.6441482150926\\
3541.19999999999	24.6406686809929\\
3541.3	24.6427966340995\\
3541.60000000002	24.6407092639128\\
3541.70000000003	24.6428369755492\\
3541.99999999998	24.6407498376669\\
3542.09999999998	24.6428773078878\\
3542.60000000002	24.639399328196\\
3542.69999999998	24.6415264762335\\
3543.09999999998	24.644389252653\\
3543.90000000002	24.6388261851016\\
3544.00000000001	24.6409525690584\\
3544.59999999999	24.6367816740486\\
3544.7	24.6389076957797\\
3544.99999999997	24.6368226566246\\
3545.09999999999	24.6389484373237\\
3545.6	24.6354739543673\\
3545.70000000002	24.6375994133905\\
3546.89999999999	24.6292641669016\\
3547.00000000001	24.631389022018\\
3547.59999999998	24.6272232714153\\
3547.69999999998	24.629347764812\\
3548.2	24.6315136826086\\
3548.50000000003	24.6350673505044\\
3548.90000000001	24.6322907861369\\
3549.00000000001	24.6344143585698\\
3549.9	24.6281690140845\\
3550.00000000002	24.6302921044478\\
3550.8	24.6247430228956\\
3551.00000000003	24.6289882008392\\
3551.29999999999	24.6325392802838\\
3551.7	24.6297651894814\\
3551.90000000001	24.634009009009\\
3552.19999999999	24.6319286096332\\
3552.40000000003	24.6361717100633\\
3552.8	24.6333980691829\\
3552.9	24.6355192794821\\
3553.60000000003	24.6306666291471\\
3553.70000000001	24.6327874387979\\
3554.99999999998	24.6350313633934\\
3555.59999999998	24.6393115279692\\
3556.00000000002	24.6365400298079\\
3556.10000000003	24.6386592430122\\
3557.09999999998	24.6345440233892\\
3557.20000000002	24.6366626373935\\
3557.49999999999	24.6345851135597\\
3557.59999999998	24.6367034882087\\
3558.00000000003	24.633933841095\\
3558.09999999999	24.6360519363723\\
3558.39999999999	24.6395953351131\\
3559.50000000002	24.6319811214743\\
3559.79999999999	24.6383325374308\\
3560.70000000002	24.6321051449113\\
3560.80000000003	24.6342216855289\\
3561.09999999999	24.6377625519488\\
3561.39999999999	24.635687210445\\
3561.6	24.6399191397366\\
3561.90000000001	24.6378439079169\\
3561.99999999999	24.6399595744084\\
3562.40000000002	24.6428070175439\\
3563.10000000002	24.6379658733723\\
3563.2	24.6400808239553\\
3563.69999999999	24.6366238284977\\
3563.8	24.6387384606751\\
3564.20000000001	24.6359734029122\\
3564.4	24.6402019918642\\
3564.70000000002	24.6381283662478\\
3564.80000000003	24.6402423630396\\
3566.10000000003	24.6368683753014\\
3566.2	24.6389815775454\\
3566.89999999999	24.6341463414634\\
3567	24.6362591460851\\
3568.7	24.6273257117238\\
3568.90000000002	24.6315494536285\\
3569.59999999998	24.6267193321568\\
3569.7	24.6288307468206\\
3570.99999999999	24.6226652852062\\
3571.19999999999	24.6268865679164\\
3572.60000000002	24.6172362638901\\
3572.70000000001	24.6193461710703\\
3573.19999999998	24.6159012677357\\
3573.30000000002	24.6180108580064\\
3573.70000000001	24.6152554703677\\
3573.80000000003	24.6173647835698\\
3574.50000000002	24.6125440608739\\
3574.60000000002	24.6146529778723\\
3574.9	24.6181818181818\\
3575.29999999998	24.6154276444594\\
3575.4	24.6175360089498\\
3575.79999999999	24.6203752901367\\
3576.20000000002	24.623213936191\\
3576.69999999999	24.6253634533661\\
3577.10000000002	24.6226098624623\\
3577.20000000002	24.624716965309\\
3577.59999999999	24.6219638315119\\
3577.80000000002	24.6261773666117\\
3578.59999999999	24.6234666219577\\
3578.89999999999	24.6297848561051\\
3579.69999999998	24.6326610425163\\
3580.39999999999	24.6306381790253\\
3580.59999999998	24.6348479347614\\
3580.90000000002	24.6327841385088\\
3580.99999999999	24.6348887213426\\
3581.49999999998	24.6314496314496\\
3581.60000000001	24.6335538989865\\
3582.10000000001	24.6301155714365\\
3582.19999999998	24.6322195237696\\
3584.09999999999	24.6191618771274\\
3584.19999999998	24.6212649610803\\
3584.49999999999	24.6247837973554\\
3584.79999999998	24.6227230885101\\
3584.90000000002	24.6248256624826\\
3586.39999999999	24.6145266973372\\
3586.50000000001	24.6166285618692\\
3587.80000000002	24.6077092449622\\
3587.89999999999	24.6098104793757\\
3588.60000000002	24.6133697439184\\
3589.4	24.6078841064215\\
3589.49999999999	24.609984399376\\
3590.39999999999	24.6038156245648\\
3590.49999999997	24.6059154458865\\
3590.90000000003	24.6031746031746\\
3590.99999999997	24.6052741499819\\
3591.89999999998	24.6074610244989\\
3592.49999999999	24.6117018315426\\
3593.29999999998	24.6090053987867\\
3593.40000000001	24.6111033811048\\
3594.10000000002	24.6063101663792\\
3594.19999999998	24.6084077567259\\
3595.29999999999	24.6008789008177\\
3595.50000000002	24.6050728668372\\
3595.89999999999	24.6078976640712\\
3596.40000000001	24.6044765744474\\
3596.59999999998	24.6086690577474\\
3597.10000000003	24.6108084065384\\
3597.50000000001	24.6136313097621\\
3598.40000000003	24.607475336946\\
3598.5	24.6095703884844\\
3599.00000000001	24.6117084826762\\
3599.70000000003	24.6152564031335\\
3600.09999999998	24.6125215265819\\
3600.29999999998	24.6167092545273\\
3600.60000000002	24.6146582608937\\
3600.7	24.616751832926\\
3601.09999999998	24.6195712540264\\
3601.5	24.6223900488672\\
3601.89999999998	24.6196557468073\\
3602.00000000002	24.6217484245301\\
3602.80000000002	24.6162813289295\\
3602.89999999998	24.6183735775742\\
3603.60000000001	24.6135915864251\\
3603.70000000003	24.6156834452522\\
3604.70000000003	24.6088548601864\\
3604.79999999999	24.6109462121002\\
3605.90000000002	24.6145313366611\\
3606.19999999999	24.6124837090647\\
3606.30000000003	24.6145740905058\\
3606.79999999998	24.6167068673931\\
3607.20000000002	24.6195215257949\\
3607.60000000002	24.616791861851\\
3607.70000000001	24.6188813127113\\
3608.2	24.6154698888673\\
3608.30000000001	24.6175590289325\\
3609.90000000002	24.6149584487535\\
3610.10000000002	24.6191346739793\\
3610.5	24.6164072453332\\
3610.70000000001	24.6205826963554\\
3612.29999999999	24.612446019267\\
3612.40000000003	24.6145328719723\\
3612.69999999998	24.6180248007086\\
3613.10000000001	24.6152994575446\\
3613.30000000001	24.6194719654619\\
3614.00000000002	24.6174704629092\\
3614.1	24.6195561950086\\
3614.89999999998	24.6141078838174\\
3615.09999999998	24.6182783801726\\
3616.39999999997	24.6149592147104\\
3616.60000000002	24.6191279343048\\
3617.09999999997	24.6240185779056\\
3617.40000000002	24.6219765031099\\
3617.49999999999	24.6240601503759\\
3618.19999999998	24.6303512699334\\
3618.90000000001	24.6255871787787\\
3618.99999999998	24.627669862673\\
3619.30000000002	24.6256285572194\\
3619.40000000002	24.627711009808\\
3620.09999999998	24.6257112866692\\
3620.2	24.6277932768003\\
3620.5	24.6312765839916\\
3622.09999999998	24.62039644415\\
3622.20000000002	24.6224774314662\\
3623.09999999999	24.6163612276441\\
3623.20000000001	24.618441751994\\
3624.00000000003	24.6130073673464\\
3624.1	24.6150874675791\\
3624.99999999998	24.6172519378776\\
3626.2	24.6118633317707\\
3626.29999999998	24.6139422016325\\
3626.70000000001	24.6112275283997\\
3626.79999999997	24.6133061292012\\
3627.09999999998	24.616784296427\\
3627.59999999998	24.6133914050225\\
3627.7	24.6154694305089\\
3628.39999999998	24.610720683478\\
3628.49999999999	24.6127983244226\\
3628.89999999999	24.6100854229815\\
3628.99999999999	24.6121627951834\\
3629.6	24.608094332865\\
3629.69999999999	24.6101713593035\\
3630.09999999998	24.6074596440967\\
3630.2	24.6095364019503\\
3631.10000000002	24.6034368803701\\
3631.20000000001	24.6055131770991\\
3631.80000000001	24.601448277761\\
3632.00000000002	24.6056000660775\\
3632.80000000003	24.6029342949159\\
3632.9	24.6050096339114\\
3633.80000000002	24.6016676298192\\
3633.99999999998	24.6058171211579\\
3634.70000000002	24.6010784637394\\
3634.90000000002	24.60522696011\\
3635.60000000002	24.6087410952499\\
3636.30000000001	24.6040039599604\\
3636.39999999998	24.6060772721023\\
3637.2	24.6006653286779\\
3637.3	24.6027382196074\\
3637.80000000003	24.6048544489953\\
3638.89999999998	24.5974168727672\\
3639.00000000001	24.5994888846143\\
3639.90000000001	24.6016483516484\\
3640.40000000002	24.5982694684796\\
3640.49999999998	24.6003406031973\\
3640.99999999999	24.6024553019692\\
3641.39999999998	24.5997528491006\\
3641.49999999997	24.601823374341\\
3642.3	24.5991653854601\\
3642.40000000002	24.6012354152368\\
3642.80000000003	24.6067693321255\\
3643.19999999998	24.6095572695084\\
3644.10000000001	24.6062235881675\\
3644.19999999999	24.608292401833\\
3645.5	24.6022602589423\\
3645.59999999999	24.6043283868667\\
3646.79999999998	24.596232416573\\
3646.89999999999	24.5982999725802\\
3648.40000000001	24.588186926134\\
3648.5	24.5902537959765\\
3649	24.5868844372585\\
3649.1	24.5889510029596\\
3649.99999999999	24.582888140051\\
3650.20000000001	24.5870202449114\\
3650.70000000002	24.5891311493371\\
3651	24.5871107337515\\
3651.10000000001	24.5891761612621\\
3653.50000000001	24.573023866871\\
3653.59999999997	24.5750882666886\\
3653.90000000001	24.5785440613027\\
3655.49999999998	24.5677864098917\\
3655.6	24.5698498235632\\
3656.00000000001	24.572632039605\\
3656.50000000003	24.5692720013127\\
3656.59999999999	24.5713348100747\\
3657.80000000003	24.5660078186938\\
3657.90000000001	24.5680699835976\\
3658.29999999998	24.5653837743276\\
3658.39999999999	24.5674456744567\\
3659.79999999999	24.5580480340993\\
3659.90000000002	24.5601092896175\\
3660.30000000002	24.5574254179871\\
3660.39999999998	24.5594864089605\\
3660.70000000002	24.5574737762238\\
3661.10000000001	24.5657161586365\\
3661.39999999999	24.5637034002458\\
3661.50000000001	24.5657636006118\\
3661.79999999999	24.5637510581938\\
3661.89999999999	24.5658110322228\\
3662.30000000001	24.563128003495\\
3662.39999999999	24.5651877133106\\
3663.5	24.5605415438367\\
3663.7	24.5646596429936\\
3664.10000000001	24.5674362753125\\
3664.80000000002	24.573658217141\\
3665.10000000003	24.5716468405544\\
3665.19999999999	24.5737047444957\\
3665.60000000002	24.5710232697711\\
3665.70000000001	24.5730809100333\\
3666.20000000001	24.5697297002428\\
3666.30000000003	24.5717870390574\\
3666.59999999998	24.5697766383942\\
3666.8	24.5738907524067\\
3667.99999999998	24.5685777377934\\
3668.2	24.5726903470272\\
3668.6	24.5781884591272\\
3669.2	24.574169460115\\
3669.30000000003	24.5762249959121\\
3669.70000000001	24.5789961305793\\
3670.7	24.5723003160074\\
3670.79999999997	24.5743550627911\\
3671.19999999998	24.571677607387\\
3671.30000000002	24.5737320913003\\
3671.70000000001	24.5710550683588\\
3671.80000000001	24.5731092894687\\
3672.09999999998	24.5711017918414\\
3672.19999999999	24.5731557879258\\
3673.29999999999	24.5712419012359\\
3673.39999999998	24.5732952225398\\
3673.99999999999	24.5692822732098\\
3674.10000000002	24.5713352566545\\
3674.8	24.5666548749626\\
3674.89999999998	24.5687074829932\\
3675.30000000002	24.5660336289928\\
3675.49999999999	24.5701382087278\\
};
\addplot [color=mycolor1, forget plot]
  table[row sep=crcr]{%
3675.49999999999	24.5701382087278\\
3676.2	24.5654598373365\\
3676.29999999999	24.5675116962246\\
3677.10000000002	24.5621668660938\\
3677.19999999998	24.5642183123487\\
3677.99999999997	24.5724694815258\\
3678.89999999998	24.5746126664855\\
3679.30000000003	24.5719410773496\\
3679.39999999998	24.5739910313901\\
3680	24.5781364636831\\
3680.40000000002	24.5836163564733\\
3680.70000000002	24.5816126928928\\
3680.79999999998	24.5836616044989\\
3681.10000000002	24.5816581549495\\
3681.20000000001	24.5837068426914\\
3681.69999999997	24.5885164865012\\
3682.5	24.5831749307554\\
3682.60000000001	24.5852227984902\\
3683.39999999999	24.5880276910547\\
3684.10000000003	24.5914988328538\\
3684.49999999999	24.5888291809152\\
3684.59999999999	24.5908757836459\\
3684.90000000002	24.5943012211669\\
3685.49999999998	24.6011504232689\\
3686	24.6059520902851\\
3686.70000000001	24.6039926223283\\
3686.80000000001	24.6060375925574\\
3687.3	24.6081249661008\\
3687.89999999998	24.6122559652928\\
3688.49999999998	24.6082524535054\\
3688.59999999998	24.6102963103532\\
3689.09999999999	24.6123820882576\\
3689.49999999998	24.6097137901127\\
3689.60000000002	24.611757053419\\
3690.20000000002	24.6185946941983\\
3690.49999999998	24.6165935078307\\
3690.59999999997	24.6186360311052\\
3691.29999999997	24.6139676003684\\
3691.39999999999	24.6160097521333\\
3692.70000000001	24.6073440207972\\
3692.8	24.6093855777302\\
3693.39999999997	24.6053878435089\\
3693.5	24.6074290664934\\
3694.30000000001	24.604807275877\\
3694.7	24.6129695788676\\
3695.89999999998	24.6049783549784\\
3696.3	24.6131371063738\\
3697	24.6111817370371\\
3697.09999999998	24.6132208157525\\
3697.6	24.6153014035752\\
3698.39999999998	24.6126808165473\\
3698.49999999999	24.6147190828962\\
3699.29999999998	24.6120992593394\\
3699.40000000002	24.6141370455467\\
3699.80000000002	24.6114759858375\\
3700	24.6155509310559\\
3700.89999999999	24.6176708997568\\
3702.40000000002	24.6076975016881\\
3702.50000000001	24.6097337006428\\
3703.7	24.6071602138344\\
3703.80000000002	24.6091957126272\\
3705	24.6066233030148\\
3705.20000000002	24.6106927914069\\
3706.10000000001	24.6047164211322\\
3706.29999999999	24.6087848046622\\
3706.6	24.6067931043786\\
3706.7	24.6088270206108\\
3707.89999999999	24.6035598705502\\
3708.00000000001	24.605593160918\\
3708.99999999999	24.5989593162762\\
3709.09999999997	24.6009921276825\\
3709.40000000002	24.599002560992\\
3709.49999999999	24.601035152038\\
3710.79999999999	24.5924169338974\\
3710.89999999998	24.5944489355969\\
3713.20000000002	24.5792152532788\\
3713.29999999999	24.5812462971939\\
3713.69999999999	24.5785987398352\\
3713.80000000003	24.5806295269124\\
3714.20000000002	24.5779823923754\\
3714.30000000001	24.5800129226793\\
3715.20000000003	24.5740586224531\\
3715.30000000001	24.576088711848\\
3715.80000000002	24.5727818294357\\
3715.89999999997	24.5748116254037\\
3716.39999999997	24.5715054486748\\
3716.50000000003	24.5735349512996\\
3716.90000000001	24.5762711864407\\
3717.29999999999	24.5736267283585\\
3717.39999999997	24.5756556825824\\
3717.90000000003	24.5777299623453\\
3719.8	24.5651764832388\\
3719.99999999999	24.5692320098922\\
3721.60000000002	24.5721041459548\\
3721.89999999997	24.570123589468\\
3722.60000000002	24.5843070889408\\
3723.20000000003	24.5910885504794\\
3724.19999999997	24.5871707434954\\
3724.3	24.5891955751262\\
3724.99999999999	24.5979973691981\\
3725.49999999999	24.5946961563238\\
3725.60000000002	24.5967200794482\\
3725.90000000002	24.5947396672034\\
3726.00000000001	24.5967633718902\\
3727.00000000001	24.5901639344262\\
3727.10000000001	24.5921871646276\\
3727.70000000003	24.5882289822415\\
3727.80000000002	24.590251884439\\
3728.09999999999	24.5882731613111\\
3728.29999999997	24.5923184207703\\
3729.30000000002	24.596449831072\\
3729.70000000003	24.5991742184568\\
3729.99999999998	24.5971957856358\\
3730.10000000002	24.5992172001501\\
3730.50000000001	24.6019407065888\\
3731.80000000001	24.5933706690962\\
3731.9	24.5953912111468\\
3732.70000000001	24.5927989712816\\
3732.79999999998	24.5948190414959\\
3733.60000000001	24.5895492406996\\
3733.7	24.591568911029\\
3734.1	24.5889347115848\\
3734.2	24.590954127949\\
3735.90000000003	24.5797644539615\\
3736.09999999999	24.5838017236765\\
3736.5	24.586522507092\\
3736.90000000003	24.5892427080546\\
3737.50000000003	24.5852953767123\\
3737.6	24.5873130534821\\
3738.10000000001	24.5893745652988\\
3738.49999999998	24.5947680950088\\
3738.90000000001	24.6001604707141\\
3739.60000000001	24.6035778270984\\
3740.1	24.6002887546121\\
3740.19999999998	24.602304627971\\
3741.1	24.5990591254143\\
3741.2	24.6010744928234\\
3741.70000000003	24.5977871612593\\
3741.8	24.599802239504\\
3742.30000000002	24.5965156049594\\
3742.40000000002	24.5985303941216\\
3743.10000000001	24.5966018379996\\
3743.2	24.5986161942671\\
3743.9	24.596688034188\\
3743.99999999998	24.5987019577469\\
3744.29999999997	24.6020724281594\\
3745.00000000002	24.6001441884062\\
3745.19999999997	24.6041705604357\\
3746.40000000003	24.5989590284265\\
3746.49999999999	24.6009715475364\\
3747	24.6030263403699\\
3747.29999999998	24.601056732668\\
3747.40000000002	24.603068712475\\
3747.80000000001	24.6004429146989\\
3747.90000000001	24.602454642476\\
3749.50000000002	24.6052912310646\\
3749.90000000001	24.6026666666667\\
3750	24.6046772086078\\
3751.20000000003	24.6021379255192\\
3751.40000000003	24.6061575369852\\
3752.29999999997	24.6002558362648\\
3752.40000000003	24.6022651565623\\
3753.20000000002	24.5970212879333\\
3753.30000000003	24.5990302126072\\
3753.60000000002	24.5970642299598\\
3753.69999999999	24.5990729394214\\
3754.19999999999	24.5957968196468\\
3754.40000000003	24.5998135570649\\
3754.99999999998	24.5958829325451\\
3755.10000000003	24.5978909245846\\
3755.79999999998	24.5959690087596\\
3755.90000000003	24.59797657082\\
3757	24.5934364270315\\
3757.09999999997	24.5954434153093\\
3757.50000000002	24.6008090270385\\
3759.60000000001	24.587068117137\\
3759.79999999999	24.591079549988\\
3760.09999999999	24.5944364661454\\
3761.30000000003	24.5919072685702\\
3761.4	24.5939120031902\\
3762.50000000002	24.5867219475894\\
3762.69999999999	24.590730307218\\
3764.00000000002	24.5822374538402\\
3764.19999999999	24.5862444544803\\
3764.89999999997	24.5843293492696\\
3764.99999999999	24.5863323683302\\
3765.69999999997	24.5897286101227\\
3766.20000000003	24.5917744205188\\
3766.99999999998	24.5865519895941\\
3767.09999999998	24.5885538330856\\
3767.49999999999	24.5912517252362\\
3768.09999999999	24.5873361286556\\
3768.20000000003	24.5893373669825\\
3769.10000000001	24.5914252361244\\
3769.60000000002	24.588163514338\\
3769.70000000003	24.5901639344262\\
3771.70000000002	24.5771249801156\\
3771.9	24.5811240721103\\
3772.99999999997	24.5792584347089\\
3773.1	24.5812572882434\\
3773.90000000002	24.5866454689984\\
3774.6	24.5820859935889\\
3774.70000000002	24.5840839249762\\
3775.40000000002	24.5795258906105\\
3775.49999999999	24.5815234664689\\
3776.20000000003	24.5769668723354\\
3776.29999999999	24.5789640927868\\
3776.70000000001	24.5816564287227\\
3776.99999999997	24.5797040057187\\
3777.19999999999	24.583697349959\\
3777.79999999997	24.5797930066968\\
3777.89999999998	24.5817893065114\\
3778.29999999998	24.5844802032606\\
3779.10000000002	24.5819220999153\\
3779.50000000001	24.5899036935125\\
3780.20000000001	24.5853503690183\\
3780.30000000001	24.5873452544704\\
3780.90000000003	24.5834435334568\\
3781.09999999999	24.5874325610917\\
3781.59999999998	24.5894703440252\\
3781.90000000001	24.5875198307774\\
3782.00000000002	24.5895137621956\\
3782.5	24.5941944694126\\
3782.79999999998	24.5975309947395\\
3783.50000000001	24.5956232159848\\
3783.59999999999	24.5976160900706\\
3784.10000000002	24.5996511812272\\
3784.50000000003	24.5970512075252\\
3784.60000000001	24.5990435173197\\
3785.49999999999	24.5958368554522\\
3785.7	24.5998203814253\\
3786.80000000001	24.595315429507\\
3786.99999999999	24.5992976155898\\
3787.3	24.5973491049269\\
3787.39999999998	24.5993399339934\\
3788.09999999998	24.5947943614382\\
3788.20000000003	24.5967848375261\\
3789.69999999998	24.5896881101905\\
3789.80000000002	24.5916778806829\\
3790.60000000001	24.5864879837497\\
3790.70000000003	24.5884773662551\\
3791.20000000001	24.5852346160947\\
3791.3	24.5872237168328\\
3791.6	24.5852783711792\\
3791.7	24.5872672609315\\
3791.99999999998	24.5905962395506\\
3792.39999999997	24.5959129861569\\
3792.80000000002	24.5985921063039\\
3793.10000000002	24.596646630813\\
3793.20000000003	24.5986344343975\\
3793.99999999998	24.5934477214623\\
3794.3	24.5994096563357\\
3795.29999999998	24.592928281604\\
3795.39999999998	24.5949150309577\\
3796.00000000003	24.591027633624\\
3796.10000000003	24.5930140666983\\
3797.60000000003	24.5885667640941\\
3797.79999999998	24.5925379815161\\
3798.2	24.5899481346918\\
3798.29999999999	24.5919334456613\\
3798.60000000001	24.5899913128175\\
3798.69999999999	24.5919764136043\\
3799.10000000001	24.5972836386608\\
3799.79999999999	24.595384089055\\
3799.90000000001	24.5973684210526\\
3800.60000000002	24.5928381613913\\
3800.70000000002	24.5948221427068\\
3801.09999999998	24.5922340313585\\
3801.20000000001	24.5942177676058\\
3802	24.5890428973462\\
3802.1	24.5910262479617\\
3803.00000000001	24.5852068049749\\
3803.1	24.58718973496\\
3804	24.5840014720959\\
3804.10000000001	24.5859839125177\\
3804.89999999999	24.5886990801577\\
3805.6	24.5841763670284\\
3805.70000000003	24.5861579694151\\
3806.89999999998	24.5836616758603\\
3807.00000000001	24.5856426151139\\
3807.59999999999	24.5817685216798\\
3807.70000000003	24.5837491464888\\
3808.80000000001	24.5924020058285\\
3809.20000000001	24.5898196519045\\
3809.29999999998	24.5917992334751\\
3810.39999999997	24.5847001705813\\
3810.69999999999	24.5906371365593\\
3811.10000000002	24.5933039462636\\
3811.59999999999	24.5900779179893\\
3811.70000000001	24.5920562463928\\
3812.10000000001	24.589475893185\\
3812.19999999999	24.5914539779136\\
3812.50000000003	24.589518963437\\
3812.59999999997	24.5914968395101\\
3813.29999999999	24.5869827450569\\
3813.40000000001	24.5889602727154\\
3814.19999999998	24.5838030569174\\
3814.4	24.5877572421025\\
3814.80000000001	24.5851791659021\\
3814.90000000003	24.5871559633028\\
3815.30000000001	24.5845782879908\\
3815.40000000001	24.5865548420915\\
3815.8	24.5892187950418\\
3816.10000000001	24.5872857816676\\
3816.29999999998	24.5912378157426\\
3816.69999999998	24.5886606581429\\
3816.89999999999	24.5926119989521\\
3817.4	24.5893909626719\\
3817.50000000001	24.5913663034367\\
3818.29999999998	24.5888330190656\\
3818.40000000002	24.5908079088647\\
3819.20000000002	24.585657057576\\
3819.29999999999	24.5876315651673\\
3820.60000000001	24.5845002224723\\
3820.69999999999	24.5864740368509\\
3823.10000000002	24.5736555764804\\
3823.19999999998	24.5756283838569\\
3823.50000000003	24.5737001778429\\
3823.7	24.5776452743344\\
3824	24.5757171622081\\
3824.30000000003	24.5816337203221\\
3825.29999999999	24.5752078214043\\
3825.49999999999	24.5791509828524\\
3825.99999999999	24.5759389456627\\
3826.1	24.5779101981078\\
3826.49999999997	24.5753410338159\\
3826.59999999999	24.5773120443202\\
3827.00000000001	24.5747432782002\\
3827.1	24.5767140468227\\
3827.39999999998	24.5747877204442\\
3827.49999999999	24.5767582819521\\
3828.20000000003	24.5748765770708\\
3828.30000000003	24.5768467244802\\
3828.59999999998	24.5801446966333\\
3830.39999999997	24.5685941783057\\
3830.50000000001	24.5705633582206\\
3831.4	24.5647918569751\\
3831.50000000001	24.5667606221944\\
3831.89999999997	24.5694154488518\\
3832.29999999998	24.5668510593884\\
3832.39999999998	24.5688193085453\\
3832.70000000003	24.566896263828\\
3832.80000000002	24.5688643063999\\
3833.70000000003	24.5630966664928\\
3833.80000000002	24.5650642948434\\
3834.30000000001	24.5618610473607\\
3834.39999999997	24.5638284000522\\
3834.80000000001	24.566481524942\\
3835.20000000003	24.5639193804917\\
3835.30000000002	24.5658862178652\\
3836.80000000002	24.5588886861789\\
3836.89999999999	24.5608548345061\\
3837.79999999999	24.555095234373\\
3837.89999999999	24.5570609692548\\
3838.20000000002	24.5603522392726\\
3838.49999999999	24.5636429948419\\
3839	24.5656534083509\\
3840.00000000001	24.559256269368\\
3840.20000000002	24.5631851678254\\
3841.79999999998	24.5555584476431\\
3841.89999999998	24.5575221238938\\
3843.19999999999	24.5544193791794\\
3843.39999999998	24.5583452582282\\
3843.79999999999	24.5609927417467\\
3844.60000000001	24.5558821234427\\
3844.70000000002	24.5578443612151\\
3845.70000000003	24.5540589734256\\
3845.79999999999	24.556020697366\\
3846.20000000001	24.5534669682552\\
3846.29999999998	24.555428452579\\
3846.9	24.5515986482974\\
3847.00000000003	24.5535598242832\\
3847.39999999998	24.5562053281352\\
3847.9	24.5530145530146\\
3847.99999999997	24.5549751825576\\
3849.40000000001	24.5486426808676\\
3849.59999999997	24.5525625373406\\
3849.90000000002	24.5506493506494\\
3849.99999999997	24.5526090231423\\
3850.70000000003	24.5481458398255\\
3850.79999999999	24.55010517022\\
3851.39999999998	24.5462806698689\\
3851.49999999998	24.5482396925953\\
3852.2	24.5463748929211\\
3852.30000000001	24.5483335063856\\
3852.59999999997	24.5516131544112\\
3852.9	24.5548922917207\\
3853.19999999997	24.5581709184336\\
3853.60000000003	24.5556218698913\\
3853.69999999999	24.5575795318906\\
3854.40000000003	24.5557141004021\\
3854.59999999999	24.5596285054609\\
3854.99999999998	24.5570802313818\\
3855.20000000001	24.5609939563718\\
3855.69999999999	24.5578090149904\\
3855.90000000001	24.5617219917012\\
3856.89999999997	24.5553539019964\\
3857.00000000001	24.5573098960359\\
3857.90000000002	24.559357179886\\
3858.7	24.5568570540064\\
3858.80000000002	24.5588120967115\\
3859.19999999998	24.5614489674293\\
3859.90000000002	24.5569948186529\\
3860.10000000001	24.5609035801254\\
3861.60000000002	24.5513633891809\\
3862.00000000002	24.5591776494653\\
3862.39999999998	24.5618122977346\\
3862.79999999998	24.5644464003728\\
3863.3	24.5664440648134\\
3863.89999999999	24.5626293995859\\
3864.00000000001	24.5645816619653\\
3864.40000000002	24.5620390736188\\
3864.50000000001	24.5639910986907\\
3864.99999999998	24.5685751985718\\
3865.40000000002	24.5660328547407\\
3865.49999999998	24.5679842715232\\
3866.30000000003	24.5629008897165\\
3866.5	24.5668028759117\\
3867.20000000001	24.5701135158896\\
3868.1	24.5643968771005\\
3868.20000000001	24.566346974123\\
3868.60000000003	24.5638069635795\\
3868.7	24.5657568238213\\
3869.19999999998	24.5625823792417\\
3869.30000000001	24.5645319687807\\
3870.00000000002	24.5626727991525\\
3870.10000000001	24.56462198336\\
3870.49999999998	24.5620833979228\\
3870.59999999997	24.5640323455706\\
3871.50000000003	24.5660708750904\\
3872.09999999999	24.5622643458499\\
3872.19999999998	24.5642124835369\\
3872.69999999999	24.5662053294774\\
3873.00000000003	24.5694663189693\\
3874.09999999998	24.5624903205823\\
3874.2	24.5644374467646\\
3875.2	24.5606791732253\\
3875.29999999999	24.5626257934665\\
3875.69999999997	24.5600908199598\\
3875.90000000002	24.5639834881321\\
3876.50000000002	24.5601816024351\\
3876.70000000002	24.5640734626496\\
3877.10000000003	24.5615392551326\\
3877.19999999998	24.5634848992856\\
3878	24.560996364199\\
3878.20000000003	24.5648866771524\\
3878.89999999997	24.5604537251869\\
3878.99999999999	24.5623984944962\\
3879.70000000002	24.5579669055106\\
3879.79999999999	24.5599113379211\\
3880.50000000003	24.5580580322631\\
3880.70000000001	24.5619459905174\\
3880.99999999998	24.5600474092397\\
3881.1	24.5619911367618\\
3882.09999999998	24.5582401730977\\
3882.19999999998	24.5601833964403\\
3882.89999999998	24.5634818439351\\
3883.39999999998	24.5603192996009\\
3883.50000000003	24.5622618189309\\
3883.90000000001	24.5597322348095\\
3884.00000000002	24.5616745191936\\
3885.19999999998	24.5592360950248\\
3885.30000000001	24.5611777423174\\
3886.10000000003	24.5586948690237\\
3886.19999999998	24.5606360805908\\
3886.59999999997	24.5581084210256\\
3886.7	24.5600493979623\\
3887.39999999999	24.5556270096463\\
3887.60000000001	24.5595081925046\\
3888.39999999997	24.5621705027646\\
3888.69999999999	24.5654186381403\\
3889.5	24.5629370629371\\
3889.60000000002	24.5648764686223\\
3889.89999999999	24.5681233933162\\
3890.19999999997	24.5713698172377\\
3890.70000000002	24.5733525239025\\
3891.79999999998	24.5664071533184\\
3892.1	24.5722213658085\\
3892.80000000001	24.5678029232706\\
3892.90000000003	24.5697405599795\\
3893.50000000001	24.5659543866858\\
3893.70000000001	24.5698289588577\\
3894.10000000002	24.5673052231524\\
3894.30000000002	24.5711791290058\\
3895.39999999999	24.5642407906559\\
3895.5	24.5661772255878\\
3895.79999999999	24.5642855309428\\
3895.90000000002	24.5662217659138\\
3897.40000000001	24.5567671584349\\
3897.50000000001	24.5587027914614\\
3898.30000000001	24.5613585060538\\
3900.1	24.5500230757397\\
3900.20000000002	24.5519575417276\\
3900.69999999998	24.5539376538146\\
3901.00000000002	24.5520494219579\\
3901.10000000001	24.5539833897262\\
3901.60000000001	24.5559627854525\\
3901.9	24.5540748334188\\
3901.99999999998	24.5560083032213\\
3903.09999999999	24.551649928264\\
3903.30000000003	24.5555157042578\\
3903.6	24.5536286087558\\
3903.70000000002	24.5555612480148\\
3905.10000000003	24.5467581685957\\
3905.20000000002	24.5486902414667\\
3905.50000000001	24.5468045882835\\
3905.60000000001	24.5487364620939\\
3905.90000000003	24.5468509984639\\
3905.99999999999	24.5487826732547\\
3906.30000000001	24.5468973991399\\
3906.40000000001	24.548828874952\\
3906.79999999999	24.546315493102\\
3906.9	24.5482467366266\\
3907.60000000002	24.5438493231313\\
3907.99999999998	24.551572375323\\
3908.60000000002	24.5478036175711\\
3908.69999999998	24.5497339336881\\
3909.50000000003	24.5447104563127\\
3909.59999999998	24.5466404071924\\
3909.90000000002	24.5447570332481\\
3910.00000000002	24.5466867855042\\
3910.29999999999	24.5448036006547\\
3910.39999999998	24.5467331543281\\
3911.20000000001	24.5417124741135\\
3911.30000000001	24.5436416628317\\
3911.59999999999	24.5417593373725\\
3911.70000000002	24.5436883276241\\
3912.00000000001	24.5469185348023\\
3913.29999999998	24.5387642459243\\
3913.39999999997	24.5406924747668\\
3914.10000000001	24.53885851515\\
3914.19999999998	24.5407863474951\\
3915.19999999998	24.5345184277067\\
3915.29999999998	24.5364458292895\\
3915.60000000002	24.5396736215747\\
3916.10000000003	24.5365405239773\\
3916.19999999999	24.5384674309935\\
3917.1	24.5328295721434\\
3917.19999999999	24.5347560819953\\
3917.70000000001	24.5367297973352\\
3918.10000000003	24.5342248991884\\
3918.30000000001	24.538076766027\\
3918.59999999998	24.5361982290045\\
3918.70000000003	24.5381239154843\\
3919.20000000003	24.5349934937361\\
3919.29999999998	24.5369189161606\\
3920.30000000003	24.5306601367207\\
3920.40000000001	24.5325851294478\\
3920.70000000002	24.5307080187717\\
3920.79999999998	24.5326328138948\\
3921.50000000002	24.5282537739698\\
3921.59999999999	24.530178239029\\
3921.9	24.5283018867925\\
3921.99999999999	24.5302261543561\\
3922.40000000002	24.5328234544296\\
3923.4	24.5291194086912\\
3923.5	24.5310429197676\\
3923.90000000002	24.5285423037717\\
3923.99999999999	24.5304655844652\\
3924.30000000003	24.5285903577617\\
3924.69999999999	24.5362821035467\\
3924.99999999998	24.5395021782885\\
3925.70000000002	24.5351265984003\\
3925.79999999999	24.5370488295677\\
3926.3	24.533924205379\\
3926.49999999998	24.5377680436001\\
3926.80000000003	24.5358934528508\\
3926.9	24.5378151260504\\
3927.19999999999	24.5410332798615\\
3928.29999999998	24.5341614906832\\
3928.39999999997	24.5360824742268\\
3928.79999999997	24.5335844638448\\
3928.90000000001	24.5355052176126\\
3929.50000000002	24.5317589576547\\
3929.60000000001	24.5336794157315\\
3929.90000000001	24.5318066157761\\
3929.99999999998	24.5337268771787\\
3930.59999999998	24.5299819370596\\
3930.79999999999	24.5338217710957\\
3931.5	24.529453657544\\
3931.59999999998	24.5313731973447\\
3932	24.5339640395717\\
3932.9	24.5308924485126\\
3932.99999999997	24.5328112684651\\
3933.40000000002	24.537943307487\\
3934.39999999998	24.5342483161774\\
3934.49999999999	24.5361663193209\\
3934.79999999999	24.5342956618974\\
3934.89999999998	24.5362134688691\\
3936.00000000002	24.5344376413201\\
3936.09999999998	24.5363548600173\\
3936.80000000003	24.5319921765856\\
3937.00000000001	24.5358258616748\\
3938.00000000003	24.529595490211\\
3938.29999999997	24.5353443022547\\
3938.60000000002	24.5334755122248\\
3938.80000000003	24.5373073700779\\
3939.20000000002	24.5398928743685\\
3939.69999999999	24.5443931164018\\
3940.09999999999	24.5469773107964\\
3940.60000000002	24.548938005938\\
3940.99999999998	24.5464464235873\\
3941.10000000001	24.548360905308\\
3941.59999999999	24.5452469746556\\
3941.7	24.547161195393\\
3942.19999999997	24.5440478908252\\
3942.3	24.5459618506494\\
3942.80000000001	24.5428491719293\\
3942.99999999998	24.546676472826\\
3944.40000000003	24.5430346051464\\
3944.69999999998	24.5487730683431\\
3945.19999999998	24.5507312498416\\
3945.60000000003	24.5482423904504\\
3945.69999999998	24.550154594759\\
3946.40000000003	24.5458000760167\\
3946.49999999999	24.5477119545938\\
3947.29999999998	24.5427369914374\\
3947.50000000003	24.5465599351505\\
3948.00000000001	24.5510498720904\\
3948.49999999998	24.5479410423948\\
3948.69999999998	24.551762560778\\
3949.00000000001	24.5549618900509\\
3949.40000000002	24.5575389289784\\
3949.90000000001	24.5594936708861\\
3950.4	24.5563852676876\\
3950.50000000003	24.5582949425404\\
3951.00000000001	24.5551871630685\\
3951.20000000001	24.5590058967935\\
3952.09999999999	24.5534132888012\\
3952.20000000002	24.555322217443\\
3953.00000000001	24.5503528876072\\
3953.19999999997	24.5541699339792\\
3953.79999999997	24.5504438655505\\
3953.99999999998	24.5542601350497\\
3955.20000000001	24.5468106085506\\
3955.30000000003	24.5487182080194\\
3956.00000000002	24.5469022522181\\
3956.30000000001	24.5526235972096\\
3956.99999999998	24.5482803062849\\
3957.09999999997	24.5501870009097\\
3958.20000000002	24.5433645757017\\
3958.40000000003	24.5471769609701\\
3959.99999999997	24.5372591601222\\
3960.09999999999	24.5391646886521\\
3960.60000000002	24.5360668568687\\
3960.90000000001	24.5417823781873\\
3961.70000000001	24.5368266949367\\
3961.80000000002	24.5387314167445\\
3962.19999999999	24.5362541957954\\
3962.39999999999	24.5400630914827\\
3963.2	24.5376327807635\\
3963.30000000002	24.5395367613665\\
3964.39999999998	24.5352503468281\\
3964.5	24.5371538112294\\
3965	24.5340596706262\\
3965.09999999998	24.5359628770302\\
3966.2	24.531679398936\\
3966.29999999999	24.5335820895522\\
3966.59999999999	24.531726624146\\
3966.79999999999	24.5355315233558\\
3967.20000000002	24.5330577470824\\
3967.29999999997	24.5349599233755\\
3967.60000000002	24.5381455251153\\
3967.89999999998	24.5362903225806\\
3967.99999999999	24.5381920818528\\
3970.60000000001	24.5271614576775\\
3970.70000000003	24.5290621537222\\
3971.20000000001	24.5259738624632\\
3971.60000000003	24.5335750434323\\
3971.89999999999	24.5317220543807\\
3972.00000000002	24.5336220135445\\
3972.79999999999	24.5286818193259\\
3972.9	24.5305814246162\\
3974.09999999999	24.5332393940919\\
3974.70000000002	24.5295360772869\\
3974.8	24.5314347530756\\
3975.10000000002	24.5295834171866\\
3975.19999999998	24.5314819007371\\
3975.59999999998	24.5290137585834\\
3975.7	24.5309120177071\\
3976.40000000003	24.529108512511\\
3976.50000000001	24.5310063873661\\
3976.99999999999	24.5279223554852\\
3977.10000000001	24.5298199738509\\
3978.2	24.5280647512757\\
3978.29999999998	24.5299617936859\\
3978.79999999998	24.5268792882455\\
3978.90000000002	24.5287760743906\\
3979.39999999998	24.5332328181932\\
3980.09999999997	24.5389678910608\\
3980.79999999998	24.5371649626969\\
3980.90000000001	24.5390605375534\\
3981.40000000001	24.5435137510988\\
3981.89999999999	24.5454545454545\\
3982.20000000002	24.5436054541345\\
3982.40000000001	24.5473948524796\\
3983.4	24.553784360487\\
3983.89999999999	24.5557228915663\\
3984.39999999999	24.5526414857573\\
3984.59999999999	24.5564283383944\\
3985.5	24.5584102769972\\
3986.10000000003	24.5647483819176\\
3987.89999999999	24.5536609829488\\
3988.10000000001	24.5574444611604\\
3988.59999999997	24.559380249204\\
3989.40000000002	24.5569620253165\\
3989.49999999999	24.5588530178464\\
3989.89999999998	24.5614035087719\\
3990.20000000001	24.5595569255444\\
3990.30000000001	24.5614474739375\\
3990.59999999999	24.5596010724935\\
3990.69999999999	24.5614914302897\\
3991.59999999998	24.5634691985871\\
3992.50000000002	24.5579321745229\\
3992.60000000003	24.5598216745561\\
3993.80000000002	24.554946293097\\
3993.99999999997	24.5587241180742\\
3994.29999999998	24.5568796314841\\
3994.40000000003	24.5587683064213\\
3994.80000000003	24.5563092943503\\
3995.09999999998	24.5619743692431\\
3996.60000000003	24.5552580879225\\
3996.70000000001	24.5571457165733\\
3997.00000000002	24.5553025943809\\
3997.30000000001	24.5609646270076\\
3997.60000000001	24.5641243715136\\
3997.89999999998	24.5622811405703\\
3997.99999999999	24.5641679797904\\
3998.70000000001	24.5623687106132\\
3998.80000000002	24.5642551701718\\
3999.70000000001	24.5587279363968\\
3999.8	24.5606140153504\\
4000.19999999999	24.5581581381396\\
4000.40000000002	24.5619297587802\\
4000.80000000003	24.5644729935764\\
4001.20000000002	24.5620173443631\\
4001.40000000001	24.5657878295639\\
4001.80000000001	24.5633324171019\\
4001.99999999998	24.5671022713076\\
4002.40000000001	24.5646470955653\\
4002.60000000002	24.5684163189847\\
4003.60000000002	24.5622798911007\\
4003.69999999999	24.564164044158\\
4004.09999999998	24.5617102042855\\
4004.20000000001	24.5635941363035\\
4005.7	24.5543961256179\\
4005.80000000003	24.5562794877556\\
4006.49999999997	24.5519892177906\\
4006.60000000001	24.5538722639579\\
4007.1	24.5508085446197\\
4007.20000000002	24.5526913383076\\
4009.30000000003	24.5473138125405\\
4009.39999999997	24.5491956603068\\
4011.59999999997	24.5432111075105\\
4011.69999999999	24.5450919786629\\
4012.50000000002	24.5401983751184\\
4012.7	24.5439593301435\\
4013.09999999997	24.548988338483\\
4013.39999999998	24.5521365391803\\
4013.79999999997	24.5546725130173\\
4014.40000000001	24.5510026155187\\
4014.49999999999	24.5528819807702\\
4016.90000000002	24.5456808563605\\
4016.99999999998	24.5475591844863\\
4017.40000000001	24.5500933416304\\
4018.00000000002	24.5464274159429\\
4018.10000000001	24.5483052112886\\
4018.39999999998	24.5464725643897\\
4018.49999999997	24.5483501717016\\
4020.09999999999	24.5435550470126\\
4020.30000000001	24.5473087255\\
4021.50000000003	24.5399840859359\\
4021.59999999998	24.5418604072905\\
4022.50000000001	24.5388554666136\\
4022.59999999999	24.5407313495911\\
4023	24.5382913673535\\
4023.10000000003	24.5401670312189\\
4023.49999999997	24.5377274082911\\
4023.9	24.5452286282306\\
4024.30000000003	24.5477586721002\\
4024.60000000002	24.5459288891097\\
4024.69999999999	24.5478036175711\\
4025.60000000002	24.5522517822987\\
4026.69999999998	24.5505115724645\\
4026.99999999998	24.5561322043158\\
4027.30000000003	24.5592690073\\
4027.80000000003	24.5611857295365\\
4028.09999999999	24.5593565364182\\
4028.30000000003	24.5631019759706\\
4028.60000000001	24.561272867178\\
4028.69999999997	24.5631453534551\\
4029.20000000001	24.5650609286973\\
4029.6	24.5626225277316\\
4029.70000000002	24.5644945158569\\
4030.20000000003	24.561447038682\\
4030.30000000001	24.5633187772926\\
4031.09999999998	24.5584441357412\\
4031.39999999998	24.5640580429121\\
4031.70000000002	24.5622302693586\\
4031.79999999999	24.5641012921947\\
4033.29999999998	24.5549660336193\\
4033.39999999999	24.5568364943597\\
4033.69999999998	24.5599682681343\\
4034.00000000001	24.5581418408071\\
4034.09999999997	24.5600118982698\\
4034.39999999997	24.5581856487793\\
4034.50000000001	24.5600555197541\\
4035.00000000003	24.5570122177889\\
4035.09999999998	24.5588818398097\\
4035.40000000002	24.557056126874\\
4035.60000000001	24.5607949054687\\
4036.3	24.556535526707\\
4036.40000000002	24.5584045584046\\
4036.90000000003	24.5553628932376\\
4037.00000000002	24.5572316762032\\
4037.29999999998	24.5554069450637\\
4037.4	24.5572755417957\\
4037.89999999997	24.5591877166914\\
4038.29999999997	24.5617076069731\\
4038.80000000001	24.5586669637773\\
4038.89999999998	24.5605347858381\\
4039.3	24.5581026885181\\
4039.40000000001	24.5599702933531\\
4040.30000000001	24.5619245619246\\
4040.8	24.5638347892796\\
4041.49999999999	24.562054631829\\
4041.6	24.5639211223\\
4042.09999999999	24.5608826876453\\
4042.19999999998	24.5627489300646\\
4042.49999999997	24.5609261366447\\
4042.60000000001	24.5627921933361\\
4043.39999999999	24.557932484234\\
4043.49999999998	24.5597981996241\\
4043.89999999998	24.5573689416419\\
4044.09999999999	24.561099846694\\
4044.50000000003	24.5586708203531\\
4044.60000000002	24.5605360100873\\
4044.90000000002	24.5587144622991\\
4044.99999999997	24.560579466515\\
4045.50000000004	24.557543998418\\
4045.60000000002	24.5594087549744\\
4046.2	24.5557669970096\\
4046.29999999997	24.5576314748913\\
4046.6	24.5558109076532\\
4046.70000000002	24.5576752001582\\
4047.99999999997	24.5497887898026\\
4048.09999999999	24.5516525863347\\
4048.59999999999	24.5486205448663\\
4048.70000000003	24.5504840940526\\
4049.30000000001	24.5542549513508\\
4049.80000000001	24.5586310773106\\
4050.20000000001	24.5611436189912\\
4051.19999999999	24.5550810850838\\
4051.30000000004	24.5569432788666\\
4052.2	24.5539570120672\\
4052.30000000001	24.5558187740598\\
4053.19999999998	24.5602348703526\\
4053.49999999998	24.5584172093941\\
4053.90000000002	24.565860878145\\
4054.39999999998	24.5677642126033\\
4055.00000000001	24.5715272126458\\
4055.90000000001	24.5660749506903\\
4056.00000000001	24.5679347156135\\
4056.3	24.5661177398679\\
4056.40000000003	24.5679773203501\\
4056.79999999997	24.5704848529666\\
4057.09999999997	24.5686680469289\\
4057.20000000001	24.5705271978902\\
4057.70000000002	24.572428409483\\
4058.79999999998	24.568232772426\\
4058.90000000001	24.570091155457\\
4059.40000000002	24.5719916245843\\
4060.10000000001	24.5751440815723\\
4060.50000000002	24.5776486233562\\
4061.50000000001	24.5715974000394\\
4061.60000000003	24.5734544648792\\
4062.10000000001	24.5704298163557\\
4062.2	24.5722866356498\\
4062.59999999997	24.5772515814606\\
4063.40000000001	24.572412944506\\
4064.19999999998	24.5872597987353\\
4064.60000000001	24.5848402096096\\
4064.70000000003	24.5866955323755\\
4065.1	24.5842762963692\\
4065.20000000003	24.5861314048164\\
4065.8	24.5825032588111\\
4065.9	24.5843580914904\\
4066.49999999997	24.580730831653\\
4066.70000000003	24.584439854431\\
4067.20000000001	24.5814176480712\\
4067.3	24.5832718690072\\
4068.40000000001	24.5766252918766\\
4068.49999999998	24.5784790837143\\
4068.79999999997	24.5815822458158\\
4069.10000000002	24.5846849503588\\
4069.60000000003	24.5890360468831\\
4070.70000000003	24.5848481870885\\
4070.89999999998	24.5885531810366\\
4071.19999999999	24.5867413356913\\
4071.29999999997	24.5885936041656\\
4072.69999999997	24.5825967393439\\
4072.89999999998	24.5863000245519\\
4073.29999999997	24.5887956007267\\
4073.89999999998	24.5851742758959\\
4073.99999999997	24.5870253552932\\
4074.29999999997	24.5852150009817\\
4074.39999999998	24.5870658976562\\
4075.10000000001	24.5852964271692\\
4075.30000000002	24.5889973990283\\
4076.50000000003	24.5915714075455\\
4077.00000000003	24.5934610384832\\
4078.40000000003	24.5899227657227\\
4078.6	24.593620516341\\
4080.69999999998	24.5809645167614\\
4080.79999999998	24.5828126148644\\
4081.09999999997	24.5810055865922\\
4081.20000000003	24.5828535025605\\
4082.99999999998	24.5793637187431\\
4083.09999999997	24.581210815047\\
4085.10000000001	24.5789679819837\\
4085.20000000003	24.5808141384966\\
4085.70000000001	24.5778060600127\\
4085.80000000003	24.5796519738613\\
4086.10000000002	24.5827419118007\\
4087.70000000001	24.5780126229268\\
4087.79999999998	24.5798576286113\\
4088.09999999999	24.5780539112568\\
4088.2	24.5798987354157\\
4088.49999999999	24.5829868414616\\
4088.90000000003	24.5854732208364\\
4089.30000000002	24.5830684207952\\
4089.40000000002	24.5849125810001\\
4089.9	24.5819070904645\\
4090	24.5837510085328\\
4090.7	24.5795443434047\\
4090.90000000003	24.5832314837448\\
4092.39999999997	24.5791081246182\\
4092.49999999997	24.5809509847041\\
4092.79999999999	24.584035769259\\
4093.99999999999	24.5866002295987\\
4094.50000000001	24.5884823914424\\
4095.90000000002	24.5849609375\\
4096.00000000001	24.5868020800273\\
4096.70000000002	24.5826010544816\\
4096.90000000004	24.5862826458384\\
4097.29999999999	24.5838824620491\\
4097.40000000003	24.5857230018304\\
4098.39999999999	24.5797242893742\\
4098.49999999999	24.5815644366369\\
4098.79999999998	24.5797653028861\\
4098.90000000002	24.5816052695779\\
4099.19999999997	24.5846851901544\\
4099.6	24.5822865087689\\
4099.69999999998	24.5841260549295\\
4100.29999999997	24.5805287289045\\
4100.39999999999	24.582368003902\\
4101.09999999999	24.5903637959622\\
4101.49999999998	24.5928418178272\\
4101.99999999999	24.5898442261281\\
4102.1	24.5916825118229\\
4102.59999999997	24.5886854997928\\
4102.70000000002	24.5905235448962\\
4103.09999999998	24.5954377071554\\
4104.59999999998	24.5888859112724\\
4104.70000000003	24.5907230559345\\
4105.1	24.5931988697262\\
4105.69999999998	24.5993472648449\\
4106.20000000002	24.6036577941212\\
4107	24.6013001874802\\
4107.1	24.6031359563693\\
4107.79999999997	24.5989434991115\\
4108.00000000001	24.6026143472652\\
4109.3	24.5972648075145\\
4109.49999999998	24.6009343975083\\
4110.39999999998	24.5979807809269\\
4110.59999999998	24.6016493541246\\
4111.00000000003	24.6041205516772\\
4111.5	24.6059928008561\\
4111.80000000003	24.6041975728982\\
4111.90000000004	24.6060311284047\\
4112.4	24.6030395136778\\
4112.5	24.60487282984\\
4112.80000000003	24.6030781200613\\
4112.90000000001	24.60491125699\\
4113.40000000002	24.6019205056521\\
4113.59999999997	24.605586211926\\
4114.09999999999	24.6025958874143\\
4114.19999999998	24.6044284568456\\
4114.50000000002	24.6026345209741\\
4114.60000000002	24.604466911318\\
4114.9	24.6026731470231\\
4115	24.6045053583145\\
4115.40000000003	24.6069736362532\\
4115.69999999998	24.6051800379027\\
4115.79999999997	24.6070118321631\\
4116.60000000003	24.6046590715865\\
4116.70000000002	24.6064904780412\\
4116.99999999999	24.6046974812368\\
4117.09999999997	24.6065287088312\\
4118.69999999998	24.5969699912596\\
4118.79999999999	24.5988006506592\\
4119.40000000001	24.6025003034349\\
4120.7	24.5971655989128\\
4120.79999999998	24.5989953650902\\
4123.00000000002	24.5955712934443\\
4123.1	24.5974000776096\\
4123.89999999997	24.5950533462658\\
4123.99999999999	24.5968817438956\\
4124.70000000001	24.5927075252133\\
4124.79999999997	24.5945356251061\\
4125.19999999999	24.5921508738758\\
4125.30000000002	24.5939787656955\\
4126.10000000003	24.5892104115167\\
4126.49999999998	24.5965201376436\\
4126.90000000002	24.5941361763993\\
4127.00000000001	24.5959632671852\\
4127.29999999999	24.5941755100063\\
4127.49999999999	24.5978292470201\\
4128.09999999998	24.594254154353\\
4128.30000000002	24.5979071795369\\
4128.6	24.6009639838206\\
4128.90000000003	24.5991765560668\\
4129.19999999999	24.604654541932\\
4130.09999999998	24.5992930124449\\
4130.19999999999	24.6011185628163\\
4130.69999999999	24.6029824731287\\
4131.1	24.6054415182029\\
4131.59999999997	24.6121451218627\\
4132.39999999999	24.6073805202662\\
4132.6	24.611029109299\\
4133.09999999998	24.6080518726411\\
4133.20000000003	24.6098758860959\\
4134.19999999999	24.603923276008\\
4134.30000000001	24.6057469040248\\
4135.10000000001	24.6009866511898\\
4135.3	24.6046331672873\\
4135.89999999999	24.6010638297872\\
4135.99999999998	24.602886777399\\
4137.2	24.5981678872695\\
4137.30000000001	24.5999903320926\\
4138.49999999998	24.5952737640748\\
4138.60000000002	24.5970957063812\\
4139	24.5995506269479\\
4139.29999999999	24.6025994105426\\
4140.00000000001	24.6008550518103\\
4140.09999999997	24.6026761992174\\
4140.7	24.5991112828439\\
4140.80000000001	24.6009321645053\\
4141.90000000001	24.5943988411395\\
4142.00000000002	24.5962193090461\\
4143.29999999999	24.598156103683\\
4143.70000000002	24.6006081374584\\
4144	24.5988272483772\\
4144.1	24.600646686936\\
4144.39999999998	24.5988659669441\\
4144.50000000001	24.6006852289726\\
4144.80000000002	24.5989046780381\\
4144.89999999999	24.6007237635706\\
4145.2	24.5989433816612\\
4145.30000000001	24.6007622907319\\
4145.59999999999	24.5989820778156\\
4145.70000000002	24.6008008104588\\
4146.60000000001	24.5978730074517\\
4146.70000000001	24.5996913282531\\
4147.20000000001	24.5967255804982\\
4147.29999999998	24.5985436659112\\
4147.59999999997	24.6015864213902\\
4148.9	24.5962882622319\\
4149.09999999997	24.5999228766991\\
4150.00000000001	24.6066359846751\\
4150.90000000003	24.6013008913515\\
4151.00000000003	24.6031172460312\\
4151.30000000001	24.6013393072217\\
4151.5	24.6049715772232\\
4152	24.602008622143\\
4152.19999999999	24.6056402475736\\
4152.5	24.6038626402736\\
4152.59999999999	24.6056782334385\\
4153.09999999998	24.6075315419436\\
4153.70000000001	24.6039770812268\\
4153.80000000003	24.6057921471388\\
4154.50000000003	24.6040533384682\\
4154.60000000002	24.6058680530484\\
4155.7	24.6113864959815\\
4156.10000000002	24.6138299408113\\
4156.90000000002	24.6163098388261\\
4157.30000000001	24.6139414056862\\
4157.40000000002	24.6157546602526\\
4157.7	24.6139785463466\\
4157.90000000002	24.6176046176046\\
4158.30000000003	24.6200461716045\\
4159.1	24.6153106366609\\
4159.40000000001	24.62074768602\\
4159.70000000002	24.6189720659647\\
4159.99999999997	24.6244080671138\\
4160.49999999997	24.6262558284863\\
4161.29999999998	24.6287307156245\\
4161.79999999997	24.6257718830342\\
4161.89999999997	24.62758289284\\
4162.3	24.6252162214107\\
4162.40000000003	24.627027027027\\
4162.7	24.6252522340732\\
4162.90000000003	24.6288734085996\\
4163.29999999998	24.6265071816304\\
4163.39999999999	24.6283175213162\\
4163.89999999998	24.6253602305475\\
4164	24.6271703369275\\
4164.90000000001	24.6218487394958\\
4165.00000000002	24.6236584955943\\
4165.70000000003	24.619520860339\\
4165.79999999997	24.6213303247798\\
4166.30000000002	24.6183755760369\\
4166.39999999999	24.6201848073923\\
4166.70000000001	24.6232120572142\\
4167.40000000001	24.6190761847631\\
4167.60000000003	24.6226935719941\\
4168.30000000001	24.6209576816045\\
4168.39999999999	24.6227659829675\\
4169.1	24.6282260385686\\
4169.50000000002	24.6306600153492\\
4170.60000000002	24.6337545256192\\
4171.40000000003	24.6290303248232\\
4171.50000000001	24.6308370888868\\
4171.80000000002	24.6290658932381\\
4171.89999999998	24.6308724832215\\
4172.4	24.6327142001198\\
4172.9	24.634555475677\\
4173.40000000002	24.6363963100515\\
4174.3	24.6382713683404\\
4174.70000000001	24.6359107023091\\
4174.89999999998	24.6395209580838\\
4175.40000000003	24.6413603161298\\
4176.50000000003	24.6348704688024\\
4176.59999999999	24.6366748868724\\
4177.69999999999	24.6445497630332\\
4178.89999999999	24.6422589136157\\
4179.2	24.6476682698059\\
4180.3	24.6411826619462\\
4180.39999999998	24.6429852888411\\
4180.90000000003	24.6400382683569\\
4181.00000000002	24.6418406639401\\
4182.40000000002	24.6383741781231\\
4182.49999999999	24.6401759671018\\
4182.90000000002	24.6378197465934\\
4182.99999999999	24.6396213334608\\
4183.39999999998	24.6420461336202\\
4183.80000000001	24.6396902411626\\
4183.90000000001	24.6414913957935\\
4184.49999999998	24.6379582277876\\
4184.59999999997	24.6397591225177\\
4184.99999999999	24.6374041241547\\
4185.10000000003	24.6392048169741\\
4185.8	24.6422513676868\\
4186.10000000004	24.6404854044241\\
4186.20000000002	24.6422855504861\\
4186.89999999997	24.6453307857655\\
4187.4	24.6423880597015\\
4187.49999999997	24.6441876014901\\
4188.00000000003	24.6412454334901\\
4188.2	24.6448439701072\\
4189	24.6425246472989\\
4189.09999999998	24.64432349852\\
4189.49999999997	24.6491311819744\\
4190.10000000004	24.6551477256456\\
4190.50000000002	24.6575669355224\\
4191.10000000004	24.6540370299676\\
4191.19999999999	24.6558347052227\\
4191.60000000001	24.6582532146862\\
4192.40000000003	24.6559332140727\\
4192.49999999999	24.6577302866956\\
4193.19999999999	24.6559988553168\\
4193.30000000003	24.657795583536\\
4194.09999999998	24.6530923656478\\
4194.29999999997	24.6566851039481\\
4195.19999999997	24.6513956093724\\
4195.30000000002	24.6531915907899\\
4195.99999999999	24.6490789066038\\
4196.29999999997	24.6544657325326\\
4196.69999999999	24.6568814334731\\
4197.60000000003	24.6539771779784\\
4197.80000000001	24.6575668786774\\
4198.29999999999	24.6546303353659\\
4198.40000000002	24.6564249136596\\
4198.89999999998	24.6534889259347\\
4199.00000000002	24.6552832749875\\
4199.29999999997	24.6582845168357\\
4200.4	24.6565885013689\\
4200.60000000001	24.6601756850049\\
4201.30000000003	24.6560670252773\\
4201.50000000004	24.6596534653465\\
4202.59999999998	24.6531991338901\\
4202.70000000003	24.6549919101551\\
4203.40000000001	24.653265136196\\
4203.49999999998	24.6550575697022\\
4204.1	24.6515389372532\\
4204.20000000001	24.6533311133839\\
4204.8	24.6498133130396\\
4204.89999999999	24.6516052318668\\
4205.29999999997	24.6492604746279\\
4205.40000000003	24.6510521935561\\
4205.7	24.6492938323268\\
4205.90000000003	24.6528768426058\\
4206.20000000002	24.6511185602549\\
4206.40000000003	24.6547010578866\\
4206.70000000003	24.6529428544262\\
4206.80000000001	24.6547338895624\\
4207.29999999997	24.6518039644436\\
4207.40000000003	24.6535947712418\\
4208.40000000002	24.6501128668172\\
4208.49999999997	24.6519032457349\\
4209.49999999997	24.6460471303687\\
4209.59999999999	24.6478371380383\\
4209.99999999999	24.6454953564048\\
4210.20000000002	24.6490748877752\\
4210.60000000002	24.651483126321\\
4211.89999999998	24.653371320038\\
4212.30000000002	24.6557781787105\\
4212.59999999999	24.6540223609562\\
4212.70000000001	24.6558108621345\\
4213	24.6540552087537\\
4213.10000000003	24.6558435393525\\
4213.69999999998	24.6523328112393\\
4213.79999999998	24.6541208856404\\
4214.29999999999	24.6511958997722\\
4214.40000000001	24.6529837465892\\
4214.70000000003	24.655974186201\\
4215.20000000003	24.6577942257965\\
4216.30000000002	24.6513613509155\\
4216.39999999997	24.6531483457844\\
4217.10000000001	24.6514274874324\\
4217.19999999998	24.6532141417495\\
4217.59999999998	24.6508760698959\\
4217.70000000003	24.6526625254872\\
4218.10000000002	24.6503247830828\\
4218.19999999998	24.6521110399924\\
4218.90000000002	24.6503910879355\\
4219.00000000003	24.6521770045744\\
4219.9	24.654028436019\\
4220.30000000002	24.6516917827694\\
4220.70000000003	24.6588324488249\\
4221.10000000001	24.6564957831896\\
4221.3	24.6600653811532\\
4221.80000000002	24.6618820910017\\
4222.30000000003	24.6684350132626\\
4222.60000000002	24.6714187605087\\
4223.69999999997	24.6744637530186\\
4224.29999999997	24.6709591894707\\
4224.40000000003	24.6727423363712\\
4225.40000000002	24.6692699088865\\
4225.5	24.6710526315789\\
4226.90000000002	24.6652472202508\\
4226.99999999998	24.6670294055026\\
4227.30000000001	24.665278894829\\
4227.40000000003	24.6670609107037\\
4228.19999999999	24.6623938698768\\
4228.49999999999	24.6677387314951\\
4229.59999999999	24.6636877319904\\
4229.69999999997	24.6654688164925\\
4230.10000000001	24.6678644035743\\
4230.80000000002	24.6637831194309\\
4230.90000000002	24.6655636965256\\
4232.30000000003	24.6574047821567\\
4232.5	24.660964891556\\
4233.00000000004	24.6675013583426\\
4233.40000000003	24.6651706625723\\
4233.60000000001	24.6687294801238\\
4235.00000000001	24.6605747207858\\
4235.09999999997	24.662353607858\\
4235.40000000001	24.6606067760595\\
4235.60000000002	24.6641641287154\\
4236.00000000001	24.6665565024433\\
4236.50000000001	24.6683661426616\\
4236.90000000002	24.6660372905358\\
4236.99999999999	24.6678152509971\\
4237.80000000001	24.6655182991576\\
4237.99999999997	24.6690734055355\\
4238.79999999998	24.6644176555238\\
4238.99999999998	24.6679719751834\\
4240.39999999999	24.6598278504893\\
4240.50000000002	24.6616044899307\\
4240.80000000003	24.6645759154896\\
4241.10000000003	24.6628312741677\\
4241.30000000001	24.6663837412175\\
4241.70000000001	24.664057711349\\
4241.79999999997	24.6658337065937\\
4242.69999999998	24.6606014895824\\
4242.79999999998	24.6623771477056\\
4243.10000000001	24.6606334841629\\
4243.19999999999	24.6624089741475\\
4243.79999999997	24.6589222177714\\
4243.89999999999	24.6606974552309\\
4244.19999999997	24.6636665645689\\
4244.9	24.6595995288575\\
4245.00000000001	24.6613742903583\\
4245.50000000001	24.6631807047296\\
4245.8	24.6661485197485\\
4246.50000000003	24.6691470823718\\
4247.00000000001	24.666242848061\\
4247.20000000002	24.669790219669\\
4247.80000000004	24.6663057039949\\
4247.90000000003	24.6680790960452\\
4248.90000000001	24.662273476112\\
4249.00000000003	24.6640465039655\\
4249.39999999997	24.6617249088128\\
4249.5	24.6634977409639\\
4250.2	24.6711996800226\\
4250.60000000003	24.6688780671419\\
4250.70000000001	24.6706502305448\\
4251.00000000002	24.6689092234951\\
4251.10000000003	24.6706812194204\\
4251.50000000001	24.6730642581616\\
4252.80000000001	24.6749276963954\\
4253.10000000003	24.6778895890153\\
4253.4	24.6808510638298\\
4255	24.6715705858852\\
4255.30000000003	24.6768811392584\\
4256.20000000001	24.6740126400865\\
4256.40000000001	24.6775519793257\\
4257.10000000001	24.6734943155125\\
4257.20000000003	24.6752636647641\\
4257.8	24.6717865614505\\
4258.10000000001	24.6770936076276\\
4258.39999999999	24.6753551720089\\
4258.49999999997	24.6771239374442\\
4258.90000000001	24.6818501995774\\
4259.69999999999	24.677214892718\\
4259.79999999997	24.6789830747201\\
4260.40000000001	24.6755075695341\\
4260.50000000002	24.6772755011031\\
4261.50000000001	24.6714848883049\\
4261.70000000003	24.6750199446243\\
4262	24.673283123343\\
4262.20000000001	24.676817680595\\
4262.99999999998	24.6721869062419\\
4263.09999999997	24.673953837493\\
4263.90000000003	24.671669793621\\
4263.99999999999	24.6734363640628\\
4265.5	24.667104276069\\
4265.60000000001	24.6688702909253\\
4266.39999999999	24.67596390484\\
4267.50000000002	24.6696035242291\\
4267.60000000002	24.6713686529044\\
4268.00000000001	24.6737424146576\\
4269.20000000003	24.6668072049282\\
4269.39999999999	24.6703361049303\\
4269.89999999997	24.672131147541\\
4270.70000000001	24.6745340451438\\
4271.00000000003	24.6728009177964\\
4271.10000000001	24.674564525192\\
4271.40000000003	24.6728315580007\\
4271.50000000002	24.6745949995318\\
4271.99999999999	24.6717071229606\\
4272.09999999999	24.6734703431487\\
4272.50000000002	24.6711604175444\\
4272.60000000001	24.6729234441922\\
4273.90000000003	24.6654188114179\\
4273.99999999998	24.6671813949135\\
4274.4	24.6695519943853\\
4276.20000000002	24.6615064424853\\
4276.50000000002	24.666791376327\\
4277.90000000002	24.6633941093969\\
4278.09999999999	24.6669159927072\\
4278.40000000003	24.6698609325698\\
4279.5	24.663519955136\\
4279.59999999997	24.6652802766549\\
4280.10000000002	24.6623989533199\\
4280.2	24.6641590542719\\
4281.39999999998	24.6665888123321\\
4281.70000000004	24.6648605726564\\
4281.80000000002	24.6666199584297\\
4282.59999999997	24.6620122819717\\
4282.70000000004	24.6637713645279\\
4283.20000000001	24.6655615997012\\
4283.70000000003	24.6626826649237\\
4283.80000000002	24.6644412801419\\
4284.20000000001	24.6668067128819\\
4284.59999999998	24.6691717039699\\
4285.29999999998	24.6651421104214\\
4285.40000000002	24.6669000116673\\
4285.90000000001	24.6640223985068\\
4286.00000000002	24.6657800797928\\
4286.69999999997	24.6640850984417\\
4286.8	24.6658424502554\\
4287.5	24.6618154678608\\
4287.60000000002	24.6635725447209\\
4288.20000000002	24.6601217265583\\
4288.3	24.6618785561048\\
4288.69999999998	24.6642417459429\\
4289.00000000001	24.6671795947868\\
4289.49999999998	24.6643043640433\\
4289.60000000003	24.6660605636758\\
4290.39999999998	24.6614613681389\\
4290.49999999999	24.6632172656505\\
4291.30000000002	24.6609498065899\\
4291.40000000003	24.6627053477805\\
4291.79999999997	24.6650667536522\\
4292.3	24.6621936445811\\
4292.4	24.663948747816\\
4292.90000000001	24.6610761705101\\
4293.10000000002	24.6645858567036\\
4293.49999999998	24.6622880566424\\
4293.6	24.6640426671635\\
4294.10000000002	24.66582832658\\
4295.09999999999	24.6600856770348\\
4295.20000000001	24.6618396852374\\
4295.60000000002	24.6641990828037\\
4296.50000000001	24.6683424102779\\
4297.59999999999	24.6620285268865\\
4298.10000000003	24.6707924247359\\
4298.50000000002	24.6684967198623\\
4298.6	24.6702491450904\\
4298.89999999997	24.673179809258\\
4299.70000000001	24.6685892367087\\
4299.80000000001	24.6703411707249\\
4300.19999999997	24.6680464153664\\
4300.30000000002	24.6697981583109\\
4300.70000000002	24.6675037202381\\
4300.80000000001	24.6692552721523\\
4301.59999999997	24.6646674570519\\
4301.70000000001	24.6664187084476\\
4302.09999999999	24.6687741155688\\
4303.9	24.6631040892193\\
4304.00000000003	24.6648544411143\\
4305.20000000002	24.6719160104987\\
4307.00000000003	24.661605256437\\
4307.09999999999	24.6633543833581\\
4307.49999999997	24.6657071222955\\
4308.19999999999	24.6709839147692\\
4308.59999999997	24.6733353447676\\
4309.1	24.67047247749\\
4309.19999999997	24.6722205462604\\
4309.69999999998	24.6693582068773\\
4309.79999999997	24.6711060581452\\
4310.59999999999	24.6758067135268\\
4310.99999999999	24.6735171997866\\
4311.10000000002	24.6752644275376\\
4311.99999999997	24.6724333851256\\
4312.09999999997	24.6741802328278\\
4312.99999999998	24.6713500730333\\
4313.09999999999	24.6730965408513\\
4314.4	24.6749333642369\\
4314.90000000001	24.6720741599073\\
4315.00000000003	24.6738198419504\\
4315.30000000003	24.672104555777\\
4315.4	24.6738500753099\\
4315.70000000002	24.676769081051\\
4316.9	24.672226082928\\
4317.00000000001	24.6739709527229\\
4318.29999999998	24.6781215264913\\
4319.1	24.68049638822\\
4319.4	24.678782266466\\
4319.49999999999	24.6805259746273\\
4319.89999999998	24.6782407407407\\
4320.00000000001	24.6799842596236\\
4321.20000000004	24.6731307708329\\
4321.40000000002	24.6766169154229\\
4322.6	24.669766581072\\
4322.90000000002	24.674994216979\\
4323.7	24.6727415699154\\
4323.90000000001	24.6762257169288\\
4325	24.6791981688285\\
4326.59999999997	24.6770055700649\\
4326.70000000001	24.6787464176759\\
4328.90000000002	24.6685146685147\\
4328.99999999997	24.6702547873692\\
4329.3	24.6731648727306\\
4330.30000000002	24.667467208572\\
4330.4	24.6692067890544\\
4330.69999999998	24.667497921862\\
4330.79999999997	24.6692373409684\\
4331.69999999998	24.6641119165243\\
4331.8	24.6658510122579\\
4332.3	24.6630043393962\\
4332.5	24.6664820200342\\
4332.80000000001	24.6647741697247\\
4332.90000000003	24.6665128086776\\
4333.19999999998	24.6694205340041\\
4333.6	24.667143549392\\
4333.69999999998	24.6688818127279\\
4334.30000000003	24.6746954595792\\
4334.69999999998	24.6724185660238\\
4334.80000000002	24.6741562665805\\
4335.20000000003	24.6718796853736\\
4335.30000000003	24.6736171979517\\
4335.69999999997	24.6782600673463\\
4336.59999999998	24.6731385615791\\
4336.69999999999	24.674875484228\\
4337.90000000002	24.6680497925311\\
4337.99999999998	24.6697863119799\\
4338.49999999997	24.6715530355414\\
4339.40000000001	24.673349464224\\
4339.90000000002	24.6705069124424\\
4339.99999999999	24.6722425750559\\
4341.10000000001	24.6682944807887\\
4341.19999999999	24.6700297146016\\
4342.49999999998	24.6649472666145\\
4342.60000000002	24.6666820180993\\
4344.6	24.6576288351325\\
4344.70000000004	24.6593629165899\\
4346.50000000003	24.656053006948\\
4346.59999999999	24.657786366669\\
4346.90000000003	24.6560846560847\\
4347.00000000001	24.6578178555819\\
4347.30000000003	24.6561162993973\\
4347.40000000003	24.6578493387004\\
4348.00000000002	24.6544467698535\\
4348.10000000003	24.6561795685571\\
4348.89999999997	24.6516440561049\\
4349	24.6533765606677\\
4349.7	24.6563060370592\\
4350.30000000001	24.6529054799559\\
4350.4	24.6546373980002\\
4350.89999999999	24.6518041829464\\
4350.99999999998	24.6535358874767\\
4353.00000000001	24.6422090004824\\
4353.1	24.6439400900487\\
4353.49999999998	24.6416758544653\\
4353.59999999998	24.6434067574707\\
4353.99999999997	24.6457362026596\\
4354.79999999999	24.6480975452938\\
4355.3	24.652155944345\\
4356.20000000001	24.6562449785368\\
4357.50000000002	24.6488892968607\\
4357.59999999998	24.6506184455102\\
4357.89999999999	24.6535107847637\\
4358.40000000002	24.6506825742801\\
4358.50000000002	24.6524113247373\\
4358.79999999999	24.6507146298378\\
4359.10000000004	24.6559001651679\\
4359.90000000001	24.651376146789\\
4359.99999999999	24.6531042865989\\
4360.79999999997	24.650874819418\\
4360.89999999999	24.6526026140793\\
4361.40000000002	24.6497764530551\\
4361.49999999997	24.6515040352164\\
4362.60000000002	24.6452884681505\\
4362.69999999998	24.647015678005\\
4363.20000000003	24.6487750097403\\
4363.59999999998	24.6465155716479\\
4363.80000000003	24.649969064369\\
4364.2	24.6522924638545\\
4365.20000000001	24.6489359265113\\
4365.40000000003	24.6523880426068\\
4366.19999999999	24.6501614639397\\
4366.40000000001	24.6536127333104\\
4367.3	24.6485323075514\\
4367.50000000001	24.6519827823061\\
4368.79999999998	24.6492252054293\\
4368.89999999998	24.6509498741131\\
4369.70000000002	24.6464369078676\\
4369.79999999997	24.6481612851553\\
4370.69999999999	24.6430859339251\\
4370.80000000003	24.6448099933652\\
4371.40000000002	24.6482900606199\\
4371.89999999997	24.6454711802379\\
4372.09999999998	24.6489181647683\\
4372.70000000004	24.6546834979876\\
4373.09999999998	24.652428427696\\
4373.19999999999	24.6541513273729\\
4373.70000000003	24.6559056198272\\
4374.6	24.6508331999909\\
4374.70000000002	24.6525555453964\\
4376.10000000001	24.6560943284128\\
4377.1	24.6504614822261\\
4377.19999999999	24.6521828524433\\
4377.80000000003	24.6488042212019\\
4377.90000000002	24.6505253540429\\
4378.30000000003	24.64827334186\\
4378.40000000003	24.6499942902821\\
4379.09999999998	24.6460540738034\\
4379.20000000003	24.6477747585231\\
4379.99999999999	24.6501221433301\\
4381.09999999999	24.6462156486807\\
4381.29999999999	24.6496553613\\
4381.80000000003	24.6514069239371\\
4382.20000000003	24.6491568354517\\
4382.29999999999	24.6508762322015\\
4383.1	24.6463770761088\\
4383.20000000003	24.6480961832409\\
4384.00000000003	24.6504413676695\\
4384.79999999999	24.6527856963671\\
4385.79999999998	24.6471647780387\\
4385.90000000004	24.6488828089375\\
4386.39999999998	24.6460731790722\\
4386.49999999999	24.6477909998632\\
4387.00000000004	24.6449818786898\\
4387.20000000002	24.6484170218585\\
4387.60000000002	24.6461699751578\\
4387.69999999998	24.6478873239437\\
4388.60000000001	24.645111308588\\
4388.70000000002	24.6468282901932\\
4389.09999999999	24.6445821562016\\
4389.19999999998	24.6462989542752\\
4390.19999999999	24.6429628954741\\
4390.60000000003	24.6498280456419\\
4391.09999999997	24.647021315358\\
4391.19999999997	24.6487372759775\\
4392.09999999999	24.6505168252812\\
4393.30000000001	24.6437838576046\\
4393.40000000001	24.6454990326619\\
4393.99999999997	24.6489611069388\\
4394.39999999998	24.6467174877688\\
4394.49999999999	24.6484321667501\\
4396.09999999999	24.6440107365452\\
4396.29999999999	24.6474388135747\\
4396.69999999999	24.6451965065502\\
4396.80000000002	24.6469103231822\\
4397.29999999998	24.6486560240142\\
4400.19999999997	24.6460468604413\\
4400.40000000001	24.6494716509488\\
4400.89999999998	24.6512156328107\\
4401.90000000004	24.6478873239437\\
4402.00000000002	24.6495990549965\\
4402.89999999997	24.6513740631388\\
4403.60000000001	24.6588096373504\\
4404.39999999997	24.6543307980475\\
4404.50000000003	24.6560414112519\\
4406.90000000002	24.6516904923985\\
4406.99999999999	24.6534001951397\\
4407.40000000001	24.6557005104935\\
4407.80000000001	24.6534631003426\\
4407.89999999998	24.6551724137931\\
4408.30000000002	24.6574720987206\\
4408.99999999997	24.6535574153455\\
4409.10000000004	24.6552662614533\\
4409.60000000003	24.6570061455428\\
4410.20000000001	24.6536516790241\\
4410.39999999999	24.6570683595964\\
4411	24.6605155176713\\
4411.59999999998	24.6662284380171\\
4412.59999999998	24.6606386112811\\
4412.70000000002	24.6623459028281\\
4413.80000000003	24.6561997326627\\
4413.9	24.6579066606253\\
4414.90000000001	24.6545866364666\\
4415	24.6562931756925\\
4415.5	24.6535012229369\\
4415.59999999998	24.6552075548611\\
4416.60000000002	24.6496252858469\\
4416.69999999999	24.6513312805651\\
4417.89999999998	24.6446355817112\\
4418.00000000001	24.6463411873882\\
4418.90000000003	24.6481104322245\\
4419.99999999999	24.6532883871406\\
4420.4	24.6510575726728\\
4420.49999999999	24.6527620684975\\
4420.90000000001	24.6505315539471\\
4421.29999999999	24.657348351201\\
4422.50000000001	24.6551802107358\\
4422.59999999998	24.6568838040111\\
4423.39999999998	24.654685204024\\
4423.5	24.6563884618863\\
4424.60000000004	24.6525188148349\\
4424.70000000003	24.654221659736\\
4425.10000000001	24.656512699991\\
4426.1	24.653201391713\\
4426.19999999998	24.6549036441272\\
4427.2	24.6515935220112\\
4427.30000000003	24.6532953878123\\
4428.00000000002	24.6493981617398\\
4428.10000000003	24.6510997696581\\
4428.49999999997	24.653389332972\\
4429.00000000003	24.6551218080423\\
4429.40000000003	24.6528953606502\\
4429.49999999999	24.6545963518151\\
4430.29999999999	24.656915854099\\
4430.99999999999	24.6597910225452\\
4431.40000000002	24.6620783030577\\
4432.4	24.6655386350818\\
4432.99999999999	24.6689675396449\\
4435	24.6713715587022\\
4435.80000000001	24.6669221578485\\
4435.90000000003	24.6686203787196\\
4436.7	24.673187883159\\
4437.20000000001	24.6749149257431\\
4437.59999999998	24.6726908083016\\
4437.70000000002	24.6743882103745\\
4441.80000000003	24.6516130484703\\
4442.19999999998	24.65614659073\\
4443.09999999998	24.6511523226503\\
4443.99999999999	24.6529106005715\\
4444.4	24.6506918663517\\
4444.89999999997	24.6546681664792\\
4445.49999999998	24.6580888968868\\
4445.89999999997	24.6603688708952\\
4447.50000000003	24.6537458404533\\
4447.99999999998	24.6554708752051\\
4448.39999999998	24.6577498033045\\
4448.99999999998	24.6544244903464\\
4449.50000000001	24.6561488673139\\
4450.79999999999	24.6579343503561\\
4451.3	24.6551646672957\\
4451.80000000003	24.6568880702621\\
4452.70000000003	24.6586417535034\\
4453.7	24.6553504872244\\
4454.19999999997	24.6593179624183\\
4455.10000000003	24.6610702100916\\
4456.3	24.658917511893\\
4457.20000000002	24.6561819935836\\
4457.6	24.6584561545192\\
4458.09999999998	24.655690637477\\
4458.6	24.6618969654832\\
4459.90000000001	24.6547085201794\\
4460.1	24.6580870812968\\
4460.49999999997	24.6558758911357\\
4460.70000000004	24.6592539454806\\
4461.59999999997	24.6565210569962\\
4461.79999999997	24.6598982496246\\
4462.69999999998	24.6616473962535\\
4463.40000000002	24.6645009521676\\
4463.69999999999	24.66732380483\\
4464.90000000003	24.6651735722284\\
4467.30000000003	24.667591887899\\
4467.79999999997	24.669307728463\\
4468.40000000003	24.6659953004364\\
4469	24.6693965227898\\
4469.90000000001	24.6711409395973\\
4470.30000000003	24.6734073013601\\
4472.29999999998	24.6668455415437\\
4474.30000000003	24.6580547112462\\
4474.60000000003	24.6608711198516\\
4475.49999999999	24.6581463937796\\
4475.70000000002	24.6615130256044\\
4476.49999999999	24.6593396774338\\
4476.89999999997	24.6616037525128\\
4477.80000000003	24.6633466580317\\
4478.7	24.6606233812628\\
4479.50000000003	24.6651486739888\\
4479.90000000001	24.6629464285714\\
4481.60000000004	24.6602851596492\\
4482.00000000002	24.6647776711809\\
4482.79999999997	24.6626067947088\\
4483.59999999999	24.6648972946451\\
4484.00000000001	24.6626970852568\\
4484.90000000002	24.659977703456\\
4485.40000000003	24.6616876602385\\
4485.89999999997	24.6589389210878\\
4487.10000000001	24.6545730076663\\
4487.4	24.6573816155989\\
4487.79999999999	24.6551839390361\\
4488.80000000003	24.6586023301923\\
4489.70000000004	24.6603412178716\\
4490.09999999998	24.6581444033673\\
4491.30000000002	24.6560092621454\\
4492.39999999999	24.6521981079577\\
4494.39999999999	24.6456780509512\\
4494.59999999999	24.649031081051\\
4496.40000000001	24.6436116979873\\
4496.69999999999	24.6486390322007\\
4498.39999999998	24.650439035234\\
4499.19999999997	24.6460560531638\\
4499.50000000001	24.6488576762379\\
4500.80000000002	24.6506254304695\\
4501.59999999997	24.6462447519826\\
4502.19999999997	24.6496235257535\\
4502.59999999999	24.647433761965\\
4502.79999999997	24.6507806080526\\
4503.69999999997	24.6525156534482\\
4505.00000000001	24.6565004106457\\
4505.29999999998	24.659297731611\\
4506.4	24.6554976145568\\
4506.59999999998	24.6588412807598\\
4507	24.6610902797808\\
4507.39999999997	24.663338879645\\
4508.00000000001	24.6667110312549\\
4508.40000000001	24.6645225684818\\
4508.90000000002	24.6662231093369\\
4509.59999999999	24.6690467215114\\
4510.10000000004	24.6707463083677\\
4512.99999999997	24.6615408477543\\
4513.60000000003	24.6649090546558\\
4514.40000000001	24.6605382655887\\
4516.00000000004	24.6628728327539\\
4516.40000000001	24.6606885862947\\
4516.80000000002	24.6629325422303\\
4517.20000000001	24.6607486773072\\
4517.50000000002	24.6635381618559\\
4517.99999999999	24.6608087470397\\
4518.3	24.6635977337111\\
4518.7	24.6658404886253\\
4519.9	24.6681415929204\\
4520.29999999997	24.6659587647111\\
4520.50000000004	24.6692916869442\\
4521.20000000001	24.665472319908\\
4521.50000000001	24.6704706298655\\
4522.20000000004	24.6666519249055\\
4522.39999999997	24.6699834162521\\
4523.20000000002	24.6678310083346\\
4524.09999999998	24.6651341673666\\
4524.70000000003	24.6684936350778\\
4528.00000000001	24.6703915549568\\
4529.5	24.6644295302013\\
4529.89999999999	24.6666666666667\\
4530.9	24.6634297064666\\
4531.79999999998	24.6651514817185\\
4533.3	24.6614020382053\\
4533.70000000002	24.6636375667211\\
4534.10000000001	24.6614617793657\\
4534.69999999997	24.6670194936932\\
4535.40000000002	24.6632124352332\\
4535.79999999999	24.6654467691087\\
4536.80000000001	24.6600101390818\\
4537.10000000002	24.6627876223221\\
4537.50000000001	24.6672249647391\\
4538.79999999997	24.6689726585737\\
4539.19999999998	24.666798845637\\
4540.89999999997	24.6619687293548\\
4541.8	24.6636870032365\\
4542.80000000002	24.6670628893438\\
4543.60000000001	24.6649206593745\\
4545.09999999998	24.6589809029306\\
4546.2	24.6552141301718\\
4546.60000000001	24.6596432577474\\
4547.50000000004	24.6547629518867\\
4547.69999999999	24.6580764325608\\
4549.00000000003	24.6554263480689\\
4549.89999999997	24.6527472527473\\
4550.10000000001	24.6560590743264\\
4550.50000000002	24.6538917944886\\
4553.59999999999	24.6458923512748\\
4553.79999999998	24.649201783087\\
4554.2	24.6514283204883\\
4554.60000000001	24.649263398248\\
4555.4	24.6515201404895\\
4555.8	24.6537456923989\\
4556.69999999998	24.6488764044944\\
4556.89999999998	24.6521834540268\\
4557.50000000004	24.6489380375636\\
4557.80000000004	24.6517036354461\\
4558.99999999997	24.6496018951109\\
4559.29999999999	24.6523665394569\\
4560.59999999999	24.6497248229438\\
4561.70000000002	24.6459730808014\\
4561.90000000004	24.6492766330557\\
4562.4	24.6465753424658\\
4562.90000000002	24.6504492658339\\
4563.89999999997	24.6450482033304\\
4564.20000000003	24.6478101789979\\
4564.70000000003	24.6451104100946\\
4564.99999999998	24.6478718976583\\
4565.70000000003	24.644093039555\\
4565.99999999998	24.6468539891811\\
4567.70000000002	24.6398703971277\\
4568.00000000002	24.6426304152711\\
4568.40000000002	24.6404728028894\\
4569.19999999998	24.6383472304292\\
4569.90000000003	24.6411378555799\\
4571.10000000001	24.6368568428421\\
4572.60000000001	24.6309620136899\\
4572.99999999998	24.6331809931994\\
4574.3	24.6349247988807\\
4574.8	24.6322323985224\\
4575.60000000001	24.6301112398103\\
4576.00000000001	24.6323288389677\\
4577	24.6291319831334\\
4578.19999999998	24.6248607561759\\
4578.40000000001	24.6281533253249\\
4579.49999999998	24.6244213468425\\
4580.8	24.6261651640507\\
4581.69999999997	24.6278755074425\\
4583.00000000001	24.6230717200148\\
4583.20000000003	24.6263609189885\\
4584.20000000002	24.6209890277687\\
4584.39999999997	24.6242774566474\\
4585.20000000002	24.6265238915665\\
4586.5	24.6239044172154\\
4586.69999999998	24.627191070027\\
4587.49999999997	24.6316156596042\\
4588.69999999999	24.6273535564854\\
4589.49999999999	24.6295973505316\\
4590.00000000003	24.6269144463084\\
4591.59999999997	24.6292222923972\\
4592.49999999998	24.6331054304751\\
4592.89999999999	24.6353146091879\\
4593.80000000004	24.6304882561658\\
4594.50000000002	24.6332651373351\\
4596.39999999997	24.6274339171108\\
4598.20000000001	24.6243176826218\\
4599.39999999998	24.6200673986303\\
4599.79999999997	24.6222743972695\\
4600.39999999998	24.6255841756331\\
4600.99999999997	24.6223729108257\\
4602.29999999998	24.6175908221797\\
4603.20000000003	24.6192948536919\\
4604.00000000002	24.6171890271714\\
4604.4	24.6193940710175\\
4605.10000000002	24.6221662468514\\
4606.19999999999	24.6249701495778\\
4607	24.6272058344729\\
4607.80000000003	24.6294407430717\\
4608.99999999998	24.623028356946\\
4610.10000000003	24.6193223721314\\
4610.3	24.6225923997918\\
4611.3	24.6259270503535\\
4612.00000000002	24.6221894581644\\
4612.20000000003	24.6254580144397\\
4612.90000000001	24.6282245827011\\
4613.30000000001	24.6304244158321\\
4614.09999999999	24.6261540462052\\
4614.50000000003	24.6326875568847\\
4615.79999999998	24.6365822483156\\
4616.20000000002	24.6387799753049\\
4616.6	24.6366452227782\\
4617.09999999999	24.6404747465997\\
4619.19999999999	24.6357673240534\\
4619.99999999997	24.6379948485963\\
4621.00000000001	24.6326632187142\\
4621.3	24.6353918725927\\
4621.89999999997	24.6321938554738\\
4623.50000000001	24.6258326844883\\
4623.79999999998	24.6307229827635\\
4625.59999999997	24.6254620922239\\
4625.80000000004	24.6287208975551\\
4626.69999999999	24.6239301461053\\
4627.60000000004	24.62778486073\\
4628.59999999997	24.6224641908095\\
4629.50000000004	24.6263176084327\\
4630.69999999997	24.6242549883389\\
4632.10000000002	24.6211303484306\\
4632.4	24.6238532110092\\
4634.00000000002	24.6153514166721\\
4634.2	24.6186047515267\\
4635.39999999999	24.6122316902168\\
4636.69999999999	24.6074879227053\\
4637.3	24.6129296588606\\
4637.79999999998	24.6102762025917\\
4639.5	24.6034140874213\\
4639.8	24.6061337528826\\
4640.60000000001	24.6018919559549\\
4640.80000000004	24.6051412441552\\
4641.30000000003	24.6024906278278\\
4641.8	24.6063034533273\\
4642.99999999998	24.6085589369171\\
4643.49999999998	24.6059092083728\\
4644.29999999997	24.6081302213418\\
4644.69999999998	24.6103169135377\\
4645.19999999997	24.6076679654705\\
4645.4	24.6109137875363\\
4646.09999999998	24.6136627781843\\
4646.90000000003	24.609425435765\\
4648.19999999999	24.6111481616935\\
4648.80000000003	24.6079717782701\\
4648.99999999999	24.6112150738853\\
4649.80000000004	24.6134325469365\\
4650.20000000003	24.6113153990065\\
4650.60000000001	24.6134990431548\\
4652.60000000001	24.6050680250177\\
4652.89999999997	24.6077799269289\\
4653.50000000003	24.6110538078047\\
4654.30000000002	24.6089721553799\\
4654.5	24.6122115756456\\
4655.79999999997	24.6053394617582\\
4657.29999999999	24.5995619873749\\
4658.49999999997	24.5975185678101\\
4658.90000000001	24.6018458896759\\
4659.59999999998	24.5981500954997\\
4659.9	24.6008583690987\\
4660.50000000003	24.597691284384\\
4660.70000000003	24.6009268795057\\
4661.09999999997	24.5988157555994\\
4661.30000000003	24.6020508859999\\
4661.90000000003	24.5988845988846\\
4662.70000000003	24.6053873209231\\
4663.89999999997	24.6012006861064\\
4664.70000000002	24.5991253644315\\
4664.89999999999	24.6023579849946\\
4665.70000000001	24.6002829096832\\
4666.09999999999	24.6024602460246\\
4667.3	24.6047049749325\\
4668.49999999998	24.6069485498865\\
4668.9	24.6048404369244\\
4669.29999999998	24.6070158906926\\
4670.10000000004	24.6028007365852\\
4670.70000000003	24.6060632011647\\
4671.5	24.6039900676428\\
4671.89999999997	24.6083047945205\\
4672.69999999998	24.6040917651087\\
4674.20000000002	24.6004749374238\\
4674.6	24.6026482982865\\
4675.30000000002	24.5989647944561\\
4676.20000000003	24.6027842525073\\
4677.79999999998	24.5986446909938\\
4678.9	24.6014105578115\\
4679.30000000003	24.5993076035389\\
4679.9	24.6025641025641\\
4680.40000000002	24.5999359042837\\
4680.90000000001	24.6058534501175\\
4681.69999999999	24.6080567303174\\
4682.20000000004	24.605428955855\\
4682.49999999996	24.6102592576774\\
4682.9	24.608157164211\\
4683.19999999999	24.6129865692994\\
4683.60000000003	24.6151546853983\\
4684.4	24.6130857081866\\
4685.19999999998	24.6110174375173\\
4685.70000000003	24.6147936318238\\
4686.80000000003	24.6111502272291\\
4687.19999999999	24.6133168348516\\
4687.99999999999	24.615515880634\\
4688.40000000003	24.6134158046283\\
4688.70000000001	24.6161064664733\\
4689.30000000002	24.6129568814774\\
4690.70000000001	24.6098746482476\\
4692.10000000004	24.6131878436554\\
4693.30000000002	24.6090254399795\\
4694.10000000002	24.6154829363896\\
4694.59999999999	24.6192514963682\\
4695.20000000003	24.6224948352608\\
4696.90000000003	24.6178411752182\\
4697.10000000003	24.6210508387976\\
4697.80000000002	24.6173822346155\\
4699.6	24.6207204715195\\
4700.19999999998	24.6175776014297\\
4700.6	24.6197374859064\\
4701.79999999998	24.6219613347796\\
4702.8	24.6167258500074\\
4703.29999999997	24.6226134285836\\
4704.6	24.6243118583544\\
4705.20000000002	24.6211718700189\\
4705.60000000002	24.6233291540047\\
4706.89999999999	24.6186530698959\\
4707.19999999999	24.6234571835235\\
4708.09999999999	24.6208742194469\\
4708.89999999997	24.6230622212784\\
4709.40000000003	24.6268181335598\\
4709.7	24.6294959446261\\
4710.69999999997	24.626390422009\\
4711.50000000004	24.6243314373037\\
4712.1	24.6275624973473\\
4713.40000000001	24.6292563912167\\
4714.19999999998	24.6271980993997\\
4715.49999999997	24.6288913393842\\
4716.40000000001	24.6263118838121\\
4716.69999999997	24.6289857530529\\
4717.59999999998	24.6348856434279\\
4718.4	24.6370668644696\\
4718.70000000003	24.6397389166737\\
4719.69999999998	24.6429933471757\\
4720.59999999997	24.6467684877243\\
4723	24.6448307255828\\
4725.20000000003	24.64817048653\\
4725.60000000004	24.6460841780054\\
4726.10000000002	24.6498243832254\\
4727.70000000003	24.6520580396802\\
4728.79999999998	24.6547822960942\\
4729.19999999998	24.6569259721312\\
4730.09999999997	24.6543486533339\\
4730.4	24.6591269421837\\
4731.19999999998	24.6570709952867\\
4732.8	24.6508483170995\\
4733.1	24.6535113665174\\
4733.79999999998	24.6562031306111\\
4734.59999999999	24.6541491541175\\
4735.6	24.651054754313\\
4736.70000000004	24.6474413105894\\
4738.00000000001	24.6512315063\\
4738.60000000004	24.6481102412054\\
4739.40000000001	24.650279565355\\
4739.69999999997	24.6529389425714\\
4740.20000000002	24.6503385861654\\
4741.00000000002	24.6525068022189\\
4742.20000000002	24.6483773696308\\
4742.49999999998	24.6531438451482\\
4745.39999999997	24.6570435149089\\
4746.39999999998	24.6602759928368\\
4746.79999999999	24.6581979818408\\
4747.20000000003	24.6603332420534\\
4747.7	24.6577362146678\\
4748.50000000003	24.6598997599292\\
4749.40000000003	24.6573323507738\\
4749.60000000003	24.660504873992\\
4750.00000000004	24.6584282436159\\
4750.69999999997	24.6611097078387\\
4750.99999999998	24.6637620761508\\
4751.40000000001	24.6679995790803\\
4752.69999999997	24.6633563373169\\
4752.99999999997	24.6660074477709\\
4753.50000000002	24.671827667452\\
4754.10000000004	24.675024189138\\
4754.59999999999	24.6724293856605\\
4755	24.6745599461631\\
4755.50000000003	24.6719656825637\\
4756.79999999997	24.6673253589523\\
4758.2	24.6642708530357\\
4759.80000000002	24.6601819365953\\
4760.2	24.662311198874\\
4761.79999999997	24.6603246603247\\
4762.80000000001	24.6572466354532\\
4762.99999999997	24.6604102370305\\
4764.30000000001	24.6578792712619\\
4764.59999999996	24.6626230402754\\
4765.40000000001	24.6605812611478\\
4765.80000000001	24.6669044671521\\
4766.89999999999	24.6633102580239\\
4767.99999999999	24.6597177072628\\
4768.79999999998	24.6576778712072\\
4769.50000000004	24.660348876216\\
4770.09999999999	24.6572470755943\\
4771.29999999997	24.6594290983778\\
4771.6	24.6620701217595\\
4772.39999999999	24.6579360921949\\
4772.80000000001	24.6600599216409\\
4773.90000000003	24.6627565982405\\
4774.19999999998	24.665395974279\\
4774.60000000004	24.6633296332754\\
4776.20000000003	24.6592550719176\\
4777.10000000003	24.6629825002093\\
4778.30000000002	24.6651598861544\\
4779.49999999997	24.6610595028873\\
4780.30000000003	24.6673918500544\\
4781.80000000001	24.6617453313537\\
4782.10000000004	24.6643804106896\\
4782.70000000001	24.6612862758217\\
4783	24.663920888127\\
4783.59999999999	24.6608273930221\\
4784.70000000001	24.6635178063869\\
4785.70000000002	24.6583643278031\\
4786.10000000002	24.6625715599014\\
4786.79999999997	24.6652321962021\\
4788.80000000001	24.6674601683059\\
4789.59999999997	24.6654278973631\\
4789.80000000001	24.6685734566484\\
4791.59999999998	24.6613936598702\\
4791.79999999997	24.6645380746677\\
4792.20000000001	24.6624793940279\\
4792.50000000004	24.6651087092601\\
4792.90000000004	24.6672230335907\\
4793.3	24.6693370050486\\
4793.79999999997	24.6667640125993\\
4794.60000000002	24.6647339770997\\
4794.89999999997	24.667361835245\\
4795.40000000001	24.6647899072047\\
4797.49999999998	24.6581624145406\\
4798.60000000002	24.6608456456957\\
4798.99999999999	24.6587901898273\\
4799.8	24.6567636825767\\
4800.89999999999	24.6531972505728\\
4801.30000000004	24.65947431999\\
4801.80000000003	24.6569066411212\\
4802.10000000004	24.6595310482695\\
4802.80000000003	24.6559370380395\\
4803.30000000004	24.6616979639422\\
4804.29999999998	24.6586462409458\\
4805.20000000001	24.6561088797786\\
4805.59999999999	24.6582183656907\\
4806.50000000002	24.6556817708984\\
4806.70000000004	24.6588166763751\\
4807.20000000001	24.6562519501591\\
4807.49999999996	24.6588734503703\\
4807.90000000002	24.6609816971714\\
4808.30000000001	24.6589302054738\\
4808.49999999997	24.6620638023541\\
4809.89999999999	24.6548856548857\\
4810.60000000001	24.6616916457064\\
4811.60000000004	24.6586445538999\\
4813.10000000003	24.6530374802626\\
4814.39999999999	24.6484577837782\\
4815.4	24.6516457273388\\
4817.89999999997	24.6409298464093\\
4818.40000000004	24.6445989415793\\
4818.79999999999	24.6467036045571\\
4819.10000000001	24.6493193891102\\
4819.79999999997	24.6457395381647\\
4820.50000000003	24.6483840185869\\
4821.30000000002	24.64429418841\\
4821.60000000002	24.6469087666176\\
4822.5	24.6423091278563\\
4822.70000000003	24.6454341876089\\
4823.39999999997	24.6480771224215\\
4824.20000000002	24.6460626412122\\
4826.09999999996	24.6425759396627\\
4826.50000000004	24.6446774126714\\
4827.89999999997	24.6375310687655\\
4828.49999999998	24.6406825995113\\
4828.90000000001	24.6386415406917\\
4829.19999999998	24.6412523554138\\
4830.69999999997	24.6377411608843\\
4830.90000000002	24.6408611053612\\
4832.49999999997	24.6389107312834\\
4832.80000000003	24.6415195845145\\
4833.50000000002	24.6379510095995\\
4834.70000000003	24.6421775461239\\
4835.49999999997	24.6443047398461\\
4836.60000000001	24.6407674654206\\
4836.79999999997	24.6438834790878\\
4837.29999999996	24.641336255013\\
4838.60000000002	24.6388492776986\\
4841.69999999999	24.6334008013549\\
4842.09999999997	24.6354962620297\\
4842.90000000003	24.6314268015693\\
4843.20000000002	24.6360952243305\\
4843.49999999997	24.6386984887274\\
4843.90000000003	24.6428571428571\\
4844.2	24.6454596123279\\
4846.00000000001	24.6404325127422\\
4846.19999999998	24.6435424963374\\
4846.70000000004	24.6471898984897\\
4846.99999999997	24.6497905964391\\
4848.59999999999	24.6478437519335\\
4850.30000000002	24.645390070922\\
4850.49999999997	24.6484970931431\\
4850.89999999997	24.6464646464647\\
4853.50000000004	24.6415032141091\\
4853.80000000002	24.6441006201199\\
4854.19999999998	24.6420699173928\\
4854.40000000002	24.6451745802863\\
4855.10000000001	24.6478002965892\\
4856.20000000002	24.6504540493792\\
4857.39999999999	24.6443643849717\\
4858.40000000004	24.6475249562622\\
4858.99999999997	24.6444814883415\\
4859.40000000003	24.6486264018932\\
4859.90000000001	24.6460905349794\\
4861.50000000001	24.6441500740497\\
4862.29999999997	24.642152023692\\
4863.4	24.6386347280765\\
4864.09999999998	24.641256527281\\
4864.90000000002	24.639260020555\\
4865.29999999999	24.6413450076047\\
4866.60000000004	24.6388723364908\\
4867.39999999997	24.6368772470467\\
4867.70000000003	24.6394675212622\\
4868.10000000002	24.6374429974118\\
4869.69999999999	24.6314017002752\\
4870.90000000003	24.6335454732088\\
4871.6	24.6300059527475\\
4872.69999999999	24.6264981119685\\
4874.10000000003	24.631734438472\\
4874.90000000001	24.6297435897436\\
4875.69999999997	24.6318552852865\\
4877.90000000004	24.6227962279623\\
4878.19999999998	24.6253817928377\\
4878.7	24.6228580798557\\
4879.60000000002	24.6203660061069\\
4880.10000000001	24.6280890127454\\
4881.10000000001	24.625092190445\\
4883.99999999997	24.6268503920886\\
4885.19999999997	24.6289890078398\\
4885.60000000002	24.6269725934871\\
4887.20000000001	24.6291408344075\\
4888.10000000001	24.6266519373184\\
4888.59999999999	24.6323153394563\\
4889.60000000003	24.63545820807\\
4890.30000000003	24.6380664158351\\
4890.80000000003	24.6355476497168\\
4891.00000000004	24.6386293471816\\
4893.19999999998	24.6418572333599\\
4893.60000000001	24.6398430635307\\
4895.20000000004	24.6420035544298\\
4896.40000000003	24.6380067395078\\
4899.40000000004	24.6331258291662\\
4899.59999999998	24.6362022164622\\
4900	24.6382726883125\\
4901.89999999997	24.6348429212566\\
4903.29999999997	24.6318880776604\\
4904.50000000002	24.6360559474779\\
4905.19999999997	24.6325403135384\\
4907.09999999998	24.6250407564395\\
4907.49999999999	24.6271089738365\\
4909.10000000003	24.6211195306771\\
4909.29999999996	24.6241903287571\\
4913.60000000002	24.6128172253088\\
4917.30000000001	24.6024321796071\\
4917.89999999997	24.607564050427\\
4918.7	24.6055948605351\\
4919.09999999999	24.6076597820784\\
4920.40000000001	24.6113199878061\\
4920.89999999997	24.6149156675473\\
4921.79999999997	24.6124464129706\\
4922.7	24.6181035183229\\
4923.60000000002	24.6136035907955\\
4925.9	24.6183516037353\\
4926.39999999999	24.6158530396833\\
4926.70000000004	24.6184135747341\\
4927.09999999998	24.6225036531905\\
4927.90000000004	24.6245941558442\\
4928.30000000001	24.6225955685415\\
4930.3	24.6186921953594\\
4930.59999999998	24.6212505323788\\
4931.39999999998	24.6233397546386\\
4931.90000000001	24.6208434712084\\
4932.10000000001	24.6239000851547\\
4933.6	24.6204673976934\\
4933.8	24.6235229737125\\
4934.30000000003	24.6210278858625\\
4935.00000000001	24.6236145164232\\
4935.90000000001	24.6272285251216\\
4936.29999999998	24.6252329632931\\
4937.1	24.6273191282508\\
4937.50000000004	24.62532404407\\
4937.79999999998	24.6299034002309\\
4939.59999999998	24.6229528108994\\
4940.4	24.6250379516243\\
4940.99999999997	24.6220477221671\\
4941.20000000003	24.6250986582478\\
4942.00000000001	24.6211124825479\\
4942.70000000001	24.6236950716193\\
4943.30000000002	24.6207063964073\\
4944.10000000001	24.6187451963917\\
4944.70000000002	24.6218249474195\\
4945.79999999998	24.6183707717503\\
4945.99999999998	24.6214188956956\\
4946.50000000002	24.6189301742611\\
4946.70000000002	24.621977844263\\
4949.20000000001	24.6236841573556\\
4949.49999999999	24.626232422822\\
4949.89999999998	24.6242424242424\\
4950.90000000004	24.6293678044839\\
4951.39999999997	24.6268807432091\\
4951.90000000004	24.6304523424879\\
4953.39999999998	24.6250126173413\\
4954.40000000003	24.6281158542739\\
4955.59999999998	24.6241701475069\\
4956.29999999997	24.6267452183036\\
4956.70000000004	24.624757908328\\
4957.70000000004	24.6278591310662\\
4958.49999999999	24.6319525672569\\
4959.60000000001	24.6365707603282\\
4960.20000000003	24.6335907102393\\
4961.20000000002	24.6306411625985\\
4962.40000000003	24.6267002518892\\
4962.60000000003	24.6297378443186\\
4963.59999999999	24.6247758728368\\
4964.50000000003	24.6223260685654\\
4966.29999999999	24.6154155927835\\
4967.20000000001	24.619008314376\\
4967.80000000004	24.622073713239\\
4968.69999999999	24.6276767026244\\
4969.19999999997	24.6312357877367\\
4970.00000000003	24.6292831130158\\
4971.20000000004	24.627361052441\\
4971.60000000003	24.6294024176841\\
4971.99999999999	24.631443454476\\
4972.49999999999	24.634999798898\\
4972.89999999998	24.6330182988136\\
4973.70000000003	24.6310667899795\\
4973.90000000003	24.6340973059912\\
4974.60000000003	24.6366615072266\\
4975.20000000003	24.6336904307278\\
4976.19999999997	24.6367783292808\\
4976.60000000001	24.6408262503265\\
4977.50000000001	24.6383799421408\\
4979.60000000003	24.6299977910316\\
4980.00000000002	24.6360514849099\\
4982.09999999999	24.6296816667336\\
4982.4	24.6322127446061\\
4983.99999999999	24.6343371922714\\
4984.59999999997	24.6313719983148\\
4984.99999999999	24.6374195101402\\
4985.49999999997	24.6349486521181\\
4986	24.6384950161449\\
4986.70000000003	24.6350364963504\\
4986.99999999997	24.6395700908344\\
4988.10000000003	24.6421554869492\\
4989.70000000002	24.6382620545914\\
4989.90000000004	24.6412825651303\\
4990.3	24.643315165117\\
4991.29999999999	24.6403814561045\\
4991.9	24.6434294871795\\
4992.2	24.6459547703463\\
4993.10000000002	24.6415124569414\\
4993.8	24.6440657602275\\
4994.40000000002	24.6411052157373\\
4994.60000000001	24.6441227701363\\
4995	24.6461532301656\\
4995.79999999999	24.6422066094197\\
4997.00000000001	24.6462948510136\\
4997.90000000001	24.6438575430172\\
4998.20000000002	24.6483804493528\\
4998.79999999999	24.6454219928384\\
4999.20000000002	24.6494509231292\\
5000.10000000003	24.645014199432\\
5000.40000000002	24.6475352464754\\
5000.89999999997	24.6450709858028\\
5001.70000000004	24.6471270342677\\
5002.20000000002	24.6446634548108\\
5002.49999999998	24.6471834645984\\
5003.59999999997	24.6417650938306\\
5004.89999999999	24.6393606393606\\
5005.10000000001	24.6423719331895\\
5006.49999999996	24.6374785283426\\
5007.09999999997	24.6405176545774\\
5008.40000000001	24.6341219926126\\
5008.60000000004	24.637131391379\\
5009.40000000001	24.6411817546661\\
5010.00000000002	24.6462146464142\\
5012.60000000004	24.6493905480081\\
5013.10000000001	24.646932099258\\
5013.79999999998	24.6494744609984\\
5014.49999999997	24.6460335819407\\
5015.30000000003	24.6480839015831\\
5016.09999999998	24.64415294446\\
5016.29999999999	24.6471573239774\\
5018.39999999997	24.6408289329481\\
5019.19999999999	24.6428784890323\\
5020.30000000004	24.6394709584894\\
5020.5	24.6424730111939\\
5021.3	24.6385470187597\\
5021.89999999997	24.6415770609319\\
5023.19999999997	24.6351999681484\\
5023.39999999997	24.6382004578481\\
5024.50000000001	24.6347967997453\\
5024.79999999999	24.6373062150491\\
5025.59999999996	24.6353741767316\\
5027.20000000001	24.6374793626798\\
5028.40000000001	24.6355772099036\\
5028.60000000002	24.6385745818999\\
5029.00000000003	24.6366149012746\\
5031.9	24.6323529411765\\
5032.30000000001	24.6343692870201\\
5033.09999999996	24.6364142096479\\
5033.50000000002	24.6384297520661\\
5033.80000000004	24.6409344643318\\
5034.60000000004	24.6449639501857\\
5035.2	24.6420272873513\\
5035.40000000003	24.6450203554761\\
5035.80000000003	24.6430628090312\\
5036.89999999997	24.6396664681358\\
5037.09999999999	24.6426586198682\\
5038.3	24.6367894569705\\
5038.7	24.6388028895769\\
5039.50000000002	24.6348916580681\\
5043.00000000004	24.6237433324741\\
5043.50000000002	24.627250376715\\
5043.89999999997	24.6252973830293\\
5044.19999999998	24.6277977122693\\
5044.90000000003	24.6243805748266\\
5046.69999999999	24.6175794562891\\
5047.59999999999	24.6151712661212\\
5047.90000000001	24.6176703645008\\
5048.70000000001	24.6137696086199\\
5049.80000000004	24.6103883245213\\
5050.60000000003	24.6124299602035\\
5052.19999999997	24.6085941056549\\
5053.39999999999	24.6067082220243\\
5053.60000000004	24.6096919088984\\
5054	24.607744207673\\
5054.39999999997	24.6097536848353\\
5054.90000000003	24.6073194856578\\
5055.10000000003	24.6103022630163\\
5055.39999999998	24.6127979428345\\
5056.39999999997	24.6079303866311\\
5056.60000000002	24.610912255028\\
5057.00000000003	24.6129204484784\\
5057.49999999996	24.6104871875989\\
5058.60000000003	24.6071124992587\\
5059.10000000002	24.6106103731815\\
5059.40000000004	24.6131040616662\\
5059.89999999998	24.6166007905138\\
5061.50000000002	24.6127706654022\\
5061.7	24.6157493381801\\
5062.09999999997	24.6138042748212\\
5062.30000000002	24.6167825537295\\
5063.40000000002	24.6193344524538\\
5064.89999999999	24.614017769003\\
5065.09999999996	24.6169943931138\\
5066.29999999999	24.6190588978367\\
5067.09999999998	24.6171455636249\\
5067.29999999999	24.6201207719935\\
5068.19999999997	24.6157488704299\\
5070.70000000002	24.6115011438037\\
5071.29999999999	24.61450487045\\
5072	24.6111078251612\\
5074.50000000002	24.6068655657589\\
5074.79999999997	24.609351908412\\
5075.40000000003	24.6064427150035\\
5076.00000000001	24.6114142747385\\
5076.60000000001	24.6085055252428\\
5076.89999999998	24.6109907425645\\
5077.39999999998	24.6085672082718\\
5077.60000000004	24.6115367193808\\
5078	24.6095980780213\\
5080.70000000001	24.5984884270194\\
5081.90000000001	24.594647776466\\
5083.20000000001	24.5903251824602\\
5085.19999999997	24.5885198513362\\
5085.39999999999	24.5914855963032\\
5088.10000000003	24.5784363822177\\
5088.30000000002	24.5814008332678\\
5088.60000000002	24.5838819344823\\
5090.10000000001	24.578602019567\\
5091.40000000001	24.5742904841402\\
5091.7	24.576770493735\\
5092.59999999999	24.5724271997172\\
5094.00000000001	24.5696001256355\\
5094.50000000001	24.5730773760452\\
5095.60000000002	24.5697352669898\\
5095.80000000003	24.5726956965404\\
5096.20000000003	24.5766536506878\\
5096.69999999997	24.5742426620625\\
5097.69999999997	24.5713837341598\\
5098.49999999997	24.5694896638293\\
5098.70000000003	24.5724484192359\\
5099.19999999997	24.5700390249642\\
5099.79999999998	24.573030843742\\
5100.19999999998	24.571103660569\\
5100.39999999999	24.5740613665327\\
5101.10000000004	24.5706892495883\\
5101.99999999996	24.5683150075459\\
5103.80000000004	24.5714061795882\\
5104.50000000004	24.5680366728049\\
5104.79999999997	24.5705106858117\\
5105.29999999997	24.5681043600893\\
5106.69999999996	24.565285501684\\
5107.00000000001	24.5677586105618\\
5107.40000000002	24.565834557024\\
5107.70000000002	24.5683072947257\\
5108.50000000004	24.5644599303136\\
5109.99999999999	24.5592062777636\\
5111.5	24.5559120431959\\
5112.20000000003	24.5584179332199\\
5112.70000000002	24.5560162728837\\
5113.00000000001	24.5584870235278\\
5113.40000000002	24.5565659528699\\
5114.60000000001	24.5527596926506\\
5114.90000000001	24.5571847507331\\
5115.30000000001	24.5552644954451\\
5116.3	24.5582831678524\\
5117.30000000001	24.5613006604917\\
5118.09999999997	24.5574616075964\\
5118.30000000002	24.5604095029697\\
5118.7	24.5623974368993\\
5119.4	24.5648989159098\\
5120.60000000004	24.5630480207784\\
5121.89999999998	24.5587661069895\\
5124.49999999999	24.5541115404129\\
5125.00000000002	24.5575696083979\\
5126.59999999998	24.5557571147132\\
5126.79999999997	24.5587001891982\\
5127.29999999999	24.5563053399384\\
5128.50000000002	24.5603088562181\\
5131.59999999998	24.5532669485745\\
5132.00000000004	24.5552502874067\\
5133.00000000002	24.5504665796497\\
5134.20000000003	24.5525193307754\\
5135.49999999998	24.5463042293013\\
5136.40000000001	24.5439501606152\\
5137.2	24.547914274035\\
5138.20000000004	24.5509215110056\\
5139.09999999999	24.5485678704857\\
5139.40000000001	24.552972078996\\
5139.79999999999	24.5510613046946\\
5140.09999999999	24.5554647679079\\
5140.49999999999	24.5535540598374\\
5142.30000000002	24.5507934038581\\
5142.6	24.5532502382017\\
5143.1	24.5508632757816\\
5143.50000000003	24.5528423672136\\
5144.10000000001	24.5499786166945\\
5144.50000000003	24.5519573922171\\
5144.89999999998	24.5539358600583\\
5145.79999999998	24.5515847567967\\
5146	24.5545170128835\\
5146.40000000003	24.5526085689303\\
5146.60000000001	24.5555404433909\\
5147.29999999998	24.5522011112406\\
5148.10000000003	24.5542131230333\\
5148.49999999999	24.5523054811017\\
5151.4	24.5443074832573\\
5151.6	24.547236834443\\
5152.00000000001	24.5453310300654\\
5152.70000000003	24.5478186616985\\
5153.49999999997	24.5498292455759\\
5155.80000000002	24.5524544696367\\
5157.80000000004	24.5487504604587\\
5158.29999999998	24.5521867245658\\
5158.80000000004	24.5556223225881\\
5160.30000000003	24.5504224478723\\
5160.50000000001	24.5533465100957\\
5161.10000000002	24.5563047353329\\
5162.00000000004	24.5597721857384\\
5162.59999999998	24.5569178918008\\
5163.50000000004	24.5545743279882\\
5165	24.557123773015\\
5165.6	24.5542714443347\\
5166.00000000003	24.5562416523103\\
5166.60000000003	24.5591963922813\\
5167.39999999997	24.5553942912434\\
5167.70000000001	24.5597739850613\\
5168.40000000004	24.5622521040921\\
5168.89999999999	24.5598761849487\\
5169.19999999996	24.562319849883\\
5170.79999999996	24.560521379257\\
5171.99999999998	24.5625567951122\\
5172.59999999999	24.5597076961742\\
5173.30000000002	24.5621834770171\\
5174.30000000003	24.557436611008\\
5174.69999999998	24.5594032619618\\
5175.90000000005	24.5614374034003\\
5176.5	24.5643858903527\\
5177.89999999999	24.5616067979915\\
5178.80000000003	24.5650620788198\\
5180	24.5632323700315\\
5180.30000000002	24.567600957455\\
5180.70000000004	24.5695645460161\\
5181.29999999999	24.5667194194619\\
5181.49999999999	24.5696310020071\\
5182.19999999996	24.5663122551763\\
5182.50000000002	24.568749276425\\
5182.90000000004	24.5707119428902\\
5183.29999999999	24.5688158351661\\
5183.70000000003	24.5707781936032\\
5186.39999999999	24.559915164369\\
5187.40000000003	24.5628915662651\\
5187.89999999998	24.5605242868157\\
5189.00000000004	24.557244994315\\
5190	24.5544401841968\\
5190.20000000004	24.5573473594975\\
5191.90000000004	24.5512326656394\\
5192.20000000001	24.5536660054311\\
5193.19999999996	24.5508636127318\\
5193.79999999997	24.5538035002599\\
5194.69999999999	24.5495495495496\\
5194.89999999998	24.5524542829644\\
5196.49999999997	24.5545164145788\\
5197.29999999999	24.550736906915\\
5198.30000000003	24.5537088334872\\
5199.49999999996	24.5518886068159\\
5200.39999999998	24.5495625420633\\
5200.90000000004	24.5529705825803\\
5201.7	24.5491945095928\\
5202.40000000003	24.5516578567996\\
5203.00000000004	24.548826661029\\
5203.89999999999	24.5465026902383\\
5204.19999999997	24.5489306919278\\
5205.99999999999	24.5442845892319\\
5206.99999999996	24.5472527894605\\
5207.79999999996	24.5434820177039\\
5208.60000000003	24.5454719987713\\
5210.49999999998	24.5384408705331\\
5210.70000000001	24.5413372226913\\
5211.19999999998	24.5389825955136\\
5212.20000000002	24.536193235232\\
5212.60000000003	24.538147217373\\
5213.09999999999	24.535793754316\\
5214.7	24.5321009434686\\
5215.09999999999	24.5340543028072\\
5215.69999999998	24.5312320257679\\
5215.90000000005	24.5341257668712\\
5216.30000000001	24.5379955524883\\
5217.00000000001	24.5404535086542\\
5217.49999999998	24.538101809261\\
5217.69999999998	24.5409942887807\\
5219.00000000003	24.5348814929777\\
5223.19999999996	24.5323837420788\\
5223.89999999996	24.5348392036753\\
5224.40000000001	24.5324911474782\\
5224.60000000003	24.5353800218194\\
5225.39999999998	24.5335374605301\\
5226.10000000003	24.5359917339558\\
5227.90000000004	24.5275439938791\\
5228.09999999997	24.5304311235224\\
5229.1	24.5333894285933\\
5230.20000000002	24.5282297382559\\
5231.30000000003	24.5307183545514\\
5232.70000000004	24.5279773734903\\
5233.39999999997	24.5304289672303\\
5234.80000000003	24.5276891631168\\
5235.00000000004	24.530572481901\\
5235.5	24.5339598135839\\
5237.30000000003	24.5293466223699\\
5238.6	24.5251684578235\\
5238.90000000003	24.5275815995419\\
5239.39999999999	24.5252409581067\\
5239.90000000001	24.5286259541985\\
5240.79999999997	24.5263218149554\\
5241.00000000002	24.5292018851005\\
5242.10000000001	24.5240547861585\\
5243.3	24.5203493916161\\
5243.69999999998	24.5222929936306\\
5245.89999999999	24.5158215783454\\
5246.30000000002	24.5177645623666\\
5247.00000000004	24.5202111642622\\
5247.39999999999	24.522153406384\\
5249.00000000003	24.5203939723\\
5249.6	24.5233060936815\\
5250.20000000001	24.520503590271\\
5250.49999999999	24.5248162114806\\
5251.49999999999	24.5220504227283\\
5251.8	24.5244578152669\\
5252.59999999998	24.528337807223\\
5253.20000000003	24.5255363295452\\
5253.5	24.5279427440232\\
5254.50000000001	24.5251779393294\\
5254.69999999996	24.5280505442643\\
5255.29999999999	24.5252502188225\\
5256	24.5276916344818\\
5256.39999999999	24.5296299819271\\
5256.90000000002	24.5272969374168\\
5257.69999999998	24.5311727338431\\
5259	24.5251088589302\\
5259.29999999998	24.5275126440278\\
5259.80000000001	24.5251810870929\\
5260.60000000003	24.5233524055734\\
5260.80000000004	24.5262217491304\\
5261.19999999998	24.5281584399293\\
5261.59999999999	24.526293783378\\
5261.89999999997	24.5286963131889\\
5262.20000000002	24.5310985690668\\
5263.09999999999	24.5345037239702\\
5263.79999999999	24.5369402914189\\
5264.19999999996	24.5350758885322\\
5265.50000000003	24.5385141294439\\
5266.00000000001	24.5361842729914\\
5267.00000000002	24.5334244650757\\
5267.2	24.5362899398174\\
5269.00000000003	24.5336015638344\\
5269.60000000003	24.5383987703285\\
5270.00000000001	24.5365363086089\\
5270.3	24.5389344262295\\
5271.89999999999	24.5314871016692\\
5272.20000000001	24.5338846423762\\
5272.50000000004	24.536281910253\\
5272.89999999997	24.5344206334155\\
5273.30000000002	24.5363522585049\\
5275.30000000004	24.5327368540774\\
5276.90000000003	24.5309835133599\\
5277.19999999999	24.5333788111345\\
5277.60000000003	24.531519411865\\
5278.00000000002	24.5353441579356\\
5278.50000000003	24.5330201189709\\
5278.80000000001	24.5354145750062\\
5280.50000000004	24.5294095367951\\
5282.00000000002	24.5243369114557\\
5282.19999999997	24.5271945932643\\
5283.10000000003	24.5249091459721\\
5283.49999999997	24.5287304110834\\
5284.19999999998	24.5311583369604\\
5284.60000000002	24.5293015686794\\
5285.59999999996	24.526552774467\\
5285.99999999999	24.5284803541363\\
5286.39999999997	24.5304076421073\\
5288.50000000003	24.5376848315244\\
5290	24.5401788245969\\
5290.40000000001	24.5383234098856\\
5290.79999999996	24.5402483509422\\
5292.00000000001	24.538462992007\\
5292.4	24.5422768068021\\
5293.00000000001	24.5394948140031\\
5293.29999999997	24.5418823440511\\
5295.2	24.5330765018035\\
5295.40000000004	24.5359267302427\\
5296.09999999996	24.5383482496885\\
5298.39999999999	24.5276965178824\\
5298.89999999997	24.5310435931308\\
5299.8	24.5363120058869\\
5300.30000000004	24.5339974341559\\
5302.10000000003	24.5313266191392\\
5303	24.529049046784\\
5303.20000000001	24.5318952350423\\
5305.60000000001	24.5283374484045\\
5305.8	24.5311822687951\\
5306.29999999996	24.5288707975275\\
5309.19999999998	24.5211233119244\\
5309.50000000002	24.5235045954498\\
5310.70000000004	24.5217293063192\\
5312.69999999998	24.5162626110526\\
5312.90000000004	24.5191040843215\\
5314.29999999999	24.5145265693211\\
5317.19999999999	24.5067985631806\\
5317.7	24.5138967242093\\
5318.60000000003	24.5172692575253\\
5319.89999999998	24.5131578947368\\
5320.90000000002	24.516068408194\\
5321.39999999997	24.5194024241285\\
5322.99999999998	24.5157896714321\\
5324.20000000002	24.5140206224292\\
5325.00000000004	24.5122157330379\\
5325.19999999996	24.5150507952604\\
5325.90000000003	24.5174615095757\\
5326.79999999997	24.5151964557247\\
5328.10000000003	24.5110919259788\\
5328.29999999999	24.5139253809774\\
5329.10000000003	24.5158748029723\\
5329.70000000003	24.5206199106908\\
5329.99999999998	24.5229920639388\\
5331.20000000003	24.5174722863092\\
5332.30000000001	24.5142900007501\\
5333.5	24.5125243737813\\
5335.30000000004	24.5080031487798\\
5335.59999999997	24.5103735217497\\
5336.20000000003	24.507617637689\\
5336.50000000003	24.509987632575\\
5337.00000000004	24.5076914429184\\
5337.30000000002	24.5100610784277\\
5337.59999999998	24.5124304475711\\
5337.89999999997	24.5147995503934\\
5338.30000000003	24.51670912633\\
5338.70000000001	24.5186184161235\\
5339.09999999996	24.5167815403057\\
5339.59999999997	24.5201041256999\\
5340.19999999997	24.5229668745202\\
5341.50000000004	24.5169986520893\\
5342.4	24.5147402901263\\
5343.09999999998	24.5171432849229\\
5343.69999999996	24.5143905086268\\
5344.39999999996	24.5167929647301\\
5344.8	24.5149581844375\\
5345.10000000001	24.5173239542019\\
5346.49999999996	24.5109041259866\\
5346.89999999998	24.5146811296054\\
5347.90000000002	24.5100972326103\\
5348.70000000001	24.5083009273108\\
5349.10000000004	24.510207133777\\
5350.00000000002	24.5060839984299\\
5350.79999999996	24.5080266870994\\
5351.39999999997	24.5052788937681\\
5351.60000000003	24.5081002298335\\
5351.89999999998	24.5104633781764\\
5352.49999999997	24.5077158763965\\
5354.60000000002	24.5111770967561\\
5355.00000000003	24.5130809882168\\
5356.10000000003	24.5155147305926\\
5356.79999999999	24.5197782299464\\
5358.99999999999	24.5227743464388\\
5359.69999999998	24.5195716258069\\
5359.99999999996	24.5237961978321\\
5361.30000000001	24.5215801842802\\
5361.8	24.5248885656204\\
5362.20000000003	24.5230591350726\\
5362.50000000001	24.5254167754447\\
5363.09999999999	24.5226730310263\\
5363.3	24.5254875638587\\
5363.69999999998	24.5236586002461\\
5364.10000000001	24.5255583311584\\
5364.99999999997	24.5233080464483\\
5365.2	24.5261215589063\\
5365.59999999997	24.5280205751347\\
5366.70000000004	24.5304464485354\\
5367.69999999998	24.5258765229703\\
5368.00000000001	24.5282315903206\\
5368.80000000003	24.532027044646\\
5369.19999999999	24.5301994673421\\
5369.50000000002	24.5325536352801\\
5370.29999999999	24.5344853269775\\
5371.19999999998	24.5303743972595\\
5372.69999999998	24.5328320428827\\
5373.19999999999	24.5305491969553\\
5373.40000000001	24.5333581464595\\
5373.80000000004	24.535253726344\\
5374.70000000001	24.5311453449431\\
5375.09999999998	24.5330406310463\\
5375.79999999999	24.5298461652933\\
5377.7	24.5230391609952\\
5378.1	24.5249339927857\\
5378.99999999997	24.5208306222231\\
5379.49999999996	24.5259870622351\\
5379.89999999998	24.5278810408922\\
5382.09999999998	24.5215711047527\\
5382.40000000004	24.5239201114724\\
5383.2	24.5277060539075\\
5383.90000000003	24.5245170876672\\
5384.19999999999	24.5287223965975\\
5384.99999999997	24.5250784572246\\
5385.69999999997	24.5274611014148\\
5386.59999999999	24.5252195221564\\
5386.80000000001	24.5280216822291\\
5387.60000000003	24.5262356849862\\
5388.29999999997	24.5286170291738\\
5389.20000000001	24.5245208097527\\
5390.19999999997	24.5218262434373\\
5391.10000000002	24.5195874758866\\
5391.29999999997	24.5223875060281\\
5392.09999999997	24.5243128964059\\
5392.89999999997	24.520674949008\\
5393.80000000004	24.5184374942064\\
5395.19999999998	24.5157822549256\\
5395.80000000003	24.5186159862118\\
5396.09999999998	24.5209591935065\\
5397.60000000005	24.5234081182726\\
5398.20000000001	24.5206824370635\\
5399.50000000005	24.5166308615453\\
5399.80000000002	24.5208244597122\\
5400.70000000001	24.5167382609984\\
5401.89999999999	24.5186967789708\\
5402.30000000002	24.5224344735673\\
5402.69999999998	24.5206189383283\\
5403.49999999996	24.5188392923236\\
5404.30000000003	24.5170601731922\\
5404.69999999996	24.5189461219657\\
5405.09999999996	24.5171316510027\\
5406.59999999998	24.5121793330497\\
5407.89999999999	24.5155325443787\\
5408.30000000003	24.5174173507877\\
5408.90000000001	24.5202440377149\\
5409.99999999999	24.5226520766714\\
5410.90000000003	24.5204213638884\\
5411.30000000001	24.5241527146395\\
5411.99999999999	24.5209807653222\\
5412.2	24.5237699314524\\
5412.90000000001	24.5205985590246\\
5413.20000000004	24.5229342545213\\
5413.69999999999	24.5262107946359\\
5414.49999999998	24.5281276548591\\
5415.19999999997	24.5249570660905\\
5415.40000000002	24.5277444372634\\
5416.00000000001	24.5250272336183\\
5416.39999999999	24.5287547309148\\
5417.2	24.5251324460525\\
5417.40000000004	24.5279187817259\\
5417.79999999998	24.5261079015855\\
5418.00000000003	24.528893892693\\
5418.80000000003	24.5308088357416\\
5419.70000000003	24.5285803904203\\
5419.99999999997	24.5309127137876\\
5420.80000000001	24.5328266524009\\
5421.59999999999	24.534740026191\\
5422.20000000004	24.5320251553769\\
5422.49999999997	24.5343562128868\\
5422.89999999999	24.5380785543057\\
5423.60000000004	24.5404428711028\\
5424.90000000003	24.5364055299539\\
5425.2	24.5387351851511\\
5425.70000000001	24.5364738840355\\
5426	24.5388031919795\\
5426.3	24.5411322423706\\
5427.49999999998	24.5430761294126\\
5428.2	24.5399112060866\\
5428.6	24.543629229834\\
5430.19999999998	24.5419221774119\\
5430.59999999998	24.5437973005322\\
5432.50000000001	24.535213341678\\
5433	24.538477112514\\
5434.59999999998	24.5422930428543\\
5435.70000000003	24.5373266124582\\
5436.00000000004	24.539651588455\\
5436.49999999999	24.5373946952139\\
5437.80000000004	24.5352066054911\\
5438	24.5379820157776\\
5438.70000000001	24.534823858204\\
5439.70000000005	24.5321519173499\\
5440.10000000005	24.534024484394\\
5441.09999999999	24.5295155480409\\
5442.8	24.5273659262525\\
5444.09999999997	24.5233459461445\\
5444.69999999997	24.5279900088157\\
5445.39999999999	24.5303461573777\\
5446.40000000001	24.5258422840356\\
5446.69999999997	24.5281633252552\\
5447.10000000003	24.5263621677192\\
5447.39999999998	24.5305185865076\\
5449.60000000002	24.5334605574619\\
5450.1	24.5312098638582\\
5450.59999999996	24.5362980901536\\
5451.40000000003	24.5326974227277\\
5452.40000000003	24.5300320953691\\
5452.80000000004	24.5319004566378\\
5453.40000000004	24.5292014302741\\
5454.59999999997	24.5311382844153\\
5455.39999999996	24.5330400513243\\
5455.9	24.5307917888563\\
5456.80000000001	24.5340761238066\\
5457.40000000002	24.5313788364636\\
5457.8	24.5332453874201\\
5458.10000000004	24.5355611740134\\
5458.39999999999	24.5378767060548\\
5460.09999999996	24.5320684224021\\
5461.29999999999	24.530340205808\\
5461.49999999998	24.533103852351\\
5463.7	24.5287162780483\\
5463.99999999998	24.531029812778\\
5464.59999999999	24.5338261935696\\
5466.10000000003	24.527093776298\\
5467.70000000001	24.5217454917883\\
5468.49999999999	24.5236440770947\\
5469.09999999998	24.5209537043809\\
5469.9	24.5191956124314\\
5470.69999999999	24.5174380346567\\
5470.99999999998	24.5197492277604\\
5472.50000000004	24.5221649672916\\
5473.20000000003	24.5245098934829\\
5473.8	24.5218217358739\\
5475.49999999999	24.5196873402002\\
5476.10000000004	24.5224790913407\\
5477.00000000001	24.5275784630553\\
5478.40000000003	24.5213105777129\\
5480.10000000002	24.519178132185\\
5480.49999999998	24.521037842572\\
5481.69999999996	24.5229669086796\\
5485.70000000002	24.5196689635058\\
5486.00000000003	24.5219737153898\\
5486.70000000004	24.5243128964059\\
5487.59999999998	24.5275798604151\\
5488.40000000001	24.5258267286144\\
5488.7	24.5281300102026\\
5489.19999999999	24.5258958337129\\
5490.79999999999	24.5223915933636\\
5491	24.5251406821948\\
5491.59999999998	24.5224611686727\\
5492.4	24.5207100591716\\
5492.69999999999	24.5230119429071\\
5493.10000000001	24.524867108425\\
5493.69999999999	24.5221886490225\\
5494.09999999997	24.5240435368206\\
5495.3	24.5205080612876\\
5495.49999999999	24.5232549676105\\
5496.20000000003	24.5201317249786\\
5496.60000000004	24.521985918824\\
5498.70000000004	24.5144395140758\\
5498.89999999998	24.5171849427169\\
5501.39999999999	24.5151322366627\\
5503.50000000002	24.5130460062505\\
5503.69999999999	24.5157890911734\\
5504.99999999997	24.5118163157799\\
5505.20000000004	24.5145586979819\\
5505.70000000004	24.512332449417\\
5506.70000000002	24.5096971017651\\
5506.99999999996	24.5119936082512\\
5509.70000000004	24.5090565900759\\
5510.00000000002	24.5131667301864\\
5511.20000000004	24.5114582766317\\
5512.10000000003	24.5092703457785\\
5514.00000000003	24.5026386899041\\
5514.39999999998	24.5044881675583\\
5514.90000000002	24.5022665457842\\
5515.29999999997	24.5077419588788\\
5515.79999999997	24.5127721677333\\
5516.70000000004	24.5087732018562\\
5517.10000000004	24.5124338432538\\
5517.7	24.5097683859509\\
5517.99999999997	24.5138725285877\\
5518.49999999996	24.5116515058167\\
5518.8	24.5139429958869\\
5519.70000000003	24.5099460125367\\
5522.99999999997	24.5043544386305\\
5523.30000000003	24.5084549371764\\
5524.59999999998	24.5026879287563\\
5524.89999999997	24.5067873303167\\
5526.69999999998	24.502424549468\\
5526.89999999999	24.5051565044328\\
5527.80000000003	24.5084028292842\\
5529.39999999999	24.5049281128493\\
5530.40000000002	24.502305397342\\
5532.40000000001	24.4970628106643\\
5532.70000000001	24.4993493348757\\
5533.09999999997	24.4975782548977\\
5533.50000000001	24.4994217146162\\
5534.79999999997	24.4936674555999\\
5535.09999999996	24.4959531724238\\
5535.49999999996	24.4941831057157\\
5536.19999999997	24.4965048859347\\
5536.60000000003	24.4947351310347\\
5538.10000000002	24.4917121086274\\
5538.49999999999	24.4971653486441\\
5538.90000000001	24.4990070409821\\
5539.39999999999	24.4967957396877\\
5540.60000000005	24.5005143754399\\
5540.99999999997	24.4987457364061\\
5541.60000000004	24.5051157587022\\
5542.19999999999	24.5024628764231\\
5542.60000000003	24.5061071319032\\
5542.99999999997	24.5043387274269\\
5543.20000000003	24.5070625800516\\
5543.99999999997	24.5053299904403\\
5544.19999999999	24.5080533160183\\
5544.50000000003	24.5103343793962\\
5545.10000000003	24.5076823198442\\
5545.40000000001	24.5099630330899\\
5546.19999999998	24.5118367199755\\
5547.20000000003	24.5092207019631\\
5549.70000000004	24.5071894482684\\
5550.4	24.50950364832\\
5551	24.5068544973068\\
5551.29999999997	24.5091328313579\\
5551.69999999997	24.5073669800785\\
5552	24.5096449991895\\
5553.10000000002	24.5047900309731\\
5553.30000000002	24.5075089134584\\
5553.7	24.5093449530052\\
5554.60000000004	24.5071741048121\\
5554.80000000001	24.5098921672757\\
5555.60000000001	24.5063628345663\\
5556.99999999998	24.509186446168\\
5557.39999999997	24.5110211426001\\
5558.19999999999	24.5092924095497\\
5559.40000000005	24.5111970500944\\
5559.79999999999	24.5094336229069\\
5561.90000000001	24.5055735346997\\
5563.20000000001	24.5088346844499\\
5563.59999999998	24.5070726315222\\
5563.80000000003	24.5097863009759\\
5564.39999999997	24.5071434989667\\
5564.70000000003	24.5094163312248\\
5564.99999999999	24.5116889184381\\
5565.79999999999	24.5099624499183\\
5566.00000000001	24.5126749429583\\
5566.60000000001	24.5100328740547\\
5566.90000000004	24.512304652416\\
5567.70000000005	24.5105786845792\\
5567.90000000004	24.5132902298851\\
5569.40000000004	24.5156656791453\\
5569.89999999997	24.5134649910233\\
5570.1	24.5161753617464\\
5571.39999999998	24.5104549941667\\
5571.79999999998	24.5122848579479\\
5572.69999999998	24.508326155613\\
5573.80000000003	24.5124598575504\\
5574.80000000002	24.5080629248955\\
5575.00000000003	24.5107711072447\\
5575.59999999996	24.5081335079004\\
5575.89999999999	24.5104017216643\\
5577.89999999996	24.5069917533166\\
5578.20000000001	24.5092590932721\\
5579.99999999999	24.5049371875056\\
5580.20000000004	24.5076429582639\\
5581.5	24.5019349290526\\
5582.40000000003	24.499776085983\\
5584.29999999999	24.4968125492443\\
5584.50000000002	24.4995165275937\\
5585.10000000003	24.4968846236482\\
5586.89999999996	24.4907821729014\\
5587.59999999997	24.4930830216368\\
5588.40000000002	24.4913661984432\\
5588.59999999999	24.4940683879972\\
5589.50000000003	24.4901245169601\\
5589.80000000003	24.4923880570314\\
5590.8	24.4880072975728\\
5591	24.4907084473538\\
5592.10000000002	24.4930438825507\\
5592.69999999997	24.4904162494636\\
5594.79999999999	24.493735366137\\
5595.30000000001	24.496908174572\\
5597.00000000003	24.4948276786193\\
5597.19999999999	24.4975255926965\\
5598.10000000001	24.493587224465\\
5598.39999999998	24.4976332946325\\
5600	24.4906340958197\\
5600.40000000001	24.4924560307115\\
5601.00000000002	24.489832354359\\
5602.60000000004	24.4917629000303\\
5606.10000000005	24.494309871214\\
5606.79999999999	24.4912518503986\\
5608.4	24.4878309708478\\
5608.69999999997	24.4900870061332\\
5609.69999999998	24.4857214160933\\
5609.90000000003	24.4884135472371\\
5611.10000000002	24.4920872540633\\
5611.50000000001	24.4903414355977\\
5611.80000000001	24.4925960904507\\
5613.09999999997	24.4904867098981\\
5613.40000000001	24.4927407143493\\
5615.49999999998	24.4835814516703\\
5615.89999999999	24.4853988603989\\
5616.40000000003	24.4832190866198\\
5617.20000000002	24.4815124704039\\
5617.50000000003	24.4837653090288\\
5618.90000000005	24.4776650649582\\
5619.10000000003	24.4803530751708\\
5620.39999999999	24.4746908638022\\
5621.60000000001	24.4712453528292\\
5621.89999999999	24.4734969761651\\
5622.30000000005	24.47531303358\\
5622.9	24.472701404944\\
5623.2	24.4749524300678\\
5623.60000000004	24.4732115866778\\
5624.00000000005	24.4750271154496\\
5624.50000000001	24.4799630195925\\
5624.89999999997	24.4817777777778\\
5625.80000000002	24.4849712934819\\
5627.50000000003	24.4829056791528\\
5628.99999999996	24.4799346254286\\
5629.20000000001	24.4826177322225\\
5631.29999999998	24.4805909720496\\
5631.50000000002	24.4832729597273\\
5631.89999999998	24.4815340909091\\
5634	24.4759588931684\\
5634.20000000005	24.4786397600412\\
5635.40000000001	24.4752018454441\\
5635.69999999998	24.4774477447745\\
5636.20000000004	24.4752763337651\\
5638.49999999999	24.4777072322917\\
5638.90000000003	24.4759709168292\\
5640.40000000005	24.4712348196082\\
5640.89999999996	24.4743839744726\\
5642.29999999999	24.4683113568694\\
5643.50000000001	24.4648805726841\\
5643.89999999998	24.4684620836286\\
5645.60000000005	24.4664080627734\\
5646.00000000003	24.4699881334018\\
5646.39999999998	24.4717966882139\\
5646.79999999996	24.4700632205281\\
5647.79999999997	24.4745834735034\\
5648.20000000002	24.4728502381247\\
5648.59999999997	24.4746578858853\\
5649.30000000005	24.4769356037809\\
5649.90000000002	24.4743362831858\\
5650.10000000002	24.4770096633748\\
5650.70000000004	24.4744107029093\\
5650.90000000005	24.4770837019996\\
5652.99999999996	24.4803736003255\\
5653.30000000002	24.4826122333463\\
5654.09999999996	24.4844540341693\\
5654.5	24.4827220316203\\
5654.80000000004	24.4849599462413\\
5656.40000000004	24.4780341200389\\
5657.09999999998	24.4803082797144\\
5657.49999999997	24.4821125565611\\
5658.00000000003	24.479949099521\\
5658.29999999998	24.4821857768981\\
5659.40000000004	24.4791942751126\\
5659.7	24.4814304392381\\
5660.49999999997	24.4832703247006\\
5661.10000000004	24.4859747050095\\
5661.60000000002	24.48381228253\\
5664.19999999997	24.4866973853786\\
5664.99999999997	24.4832394838573\\
5666	24.4859780095657\\
5666.40000000001	24.4877790523251\\
5668.49999999997	24.4910559926613\\
5669.40000000001	24.4871681806156\\
5669.7	24.4893999788352\\
5670.1	24.4876723925082\\
5670.50000000004	24.491235495362\\
5672.89999999998	24.4879252600035\\
5673.09999999999	24.4905873228513\\
5674.19999999997	24.4928889907125\\
5674.59999999999	24.4911625284156\\
5675.00000000002	24.4964846434424\\
5675.6	24.4938950261642\\
5675.80000000001	24.4965556123258\\
5676.20000000001	24.4983528002396\\
5676.6	24.4966265612063\\
5677.30000000001	24.5024130764082\\
5678.20000000001	24.5002905799271\\
5678.7	24.5034162146932\\
5679.1	24.5052120016904\\
5679.50000000001	24.5034861609973\\
5679.80000000002	24.5057131287523\\
5680.20000000001	24.5075084062462\\
5680.59999999996	24.5057827380428\\
5680.79999999998	24.5084405639951\\
5681.39999999998	24.5111326234269\\
5681.8	24.5094070645383\\
5682.20000000003	24.511201450117\\
5683.20000000001	24.5068886034522\\
5683.60000000003	24.5086827242817\\
5685.20000000004	24.5017853059645\\
5685.40000000004	24.5044411221528\\
5687.60000000002	24.5002373542908\\
5689	24.4977237172839\\
5689.29999999997	24.4999472703624\\
5689.80000000005	24.4977943373346\\
5690.20000000005	24.4995870165018\\
5690.59999999998	24.5013794436537\\
5691.60000000002	24.5041024649929\\
5692.49999999996	24.5002283666514\\
5694.80000000001	24.497357284588\\
5696.29999999999	24.4996840109543\\
5697.30000000001	24.4953838593042\\
5698.10000000005	24.4972096451511\\
5699	24.4950957168676\\
5700.10000000001	24.4921230834006\\
5700.29999999999	24.4947722966809\\
5701.10000000003	24.4965972076054\\
5701.60000000002	24.4944490239753\\
5702.00000000002	24.4962382280213\\
5703.50000000003	24.4985623115225\\
5704.60000000005	24.4938384139394\\
5706.30000000002	24.4917986821814\\
5706.60000000004	24.4957681321955\\
5707.2	24.4984493543357\\
5708.30000000002	24.5024875621891\\
5708.80000000004	24.5003415719315\\
5711.00000000003	24.4961566073086\\
5711.19999999997	24.4988006233257\\
5712.79999999997	24.5006914176688\\
5713.99999999999	24.5025463327558\\
5714.40000000003	24.5008312188293\\
5715.30000000003	24.4987227490639\\
5717.7	24.4936863828745\\
5717.90000000001	24.4963273871983\\
5718.29999999999	24.4981113598209\\
5719.50000000001	24.4929715364711\\
5719.90000000001	24.4947552447552\\
5720.29999999999	24.4965387035872\\
5720.69999999997	24.4983219130192\\
5721.9	24.4949318420133\\
5723.59999999997	24.4911508290092\\
5723.89999999998	24.4933612858141\\
5724.90000000005	24.4978165938865\\
5725.79999999996	24.4939660140764\\
5727.90000000005	24.4902234636872\\
5728.30000000002	24.4954961245723\\
5730.30000000002	24.4991623621388\\
5731.50000000002	24.4940330797683\\
5731.70000000001	24.4966677134583\\
5732.50000000002	24.4932491365175\\
5732.80000000003	24.4954560519109\\
5733.19999999997	24.4972354490433\\
5734.70000000005	24.4925716677129\\
5734.89999999997	24.4952048823017\\
5737.19999999998	24.4923570320534\\
5739.80000000003	24.4830049303995\\
5740.00000000003	24.4856361387432\\
5741.2	24.4822601153049\\
5741.50000000004	24.4844642608332\\
5741.79999999998	24.4866681760393\\
5742.19999999996	24.4884453964439\\
5742.60000000001	24.4867396869069\\
5742.79999999999	24.4893694823173\\
5743.19999999999	24.4911462051434\\
5743.70000000001	24.489014241443\\
5746.89999999997	24.4823386114495\\
5747.29999999998	24.4858544733271\\
5749.39999999996	24.4803895990956\\
5749.70000000001	24.4825906988069\\
5750	24.4847915688423\\
5750.49999999997	24.482662678677\\
5752.69999999998	24.4802530941455\\
5752.89999999997	24.4828784981749\\
5753.90000000001	24.4786235662148\\
5755.20000000004	24.4748318940802\\
5756.69999999999	24.477140077821\\
5757.09999999998	24.4789133606614\\
5757.7	24.4763624995658\\
5758.10000000005	24.4781355284638\\
5758.40000000001	24.4803334201615\\
5758.69999999996	24.4825310828645\\
5759.09999999997	24.4808306709265\\
5759.4	24.4830280406285\\
5760.40000000002	24.4857217255447\\
5761.30000000003	24.4818967612039\\
5761.7	24.4836682981013\\
5762.69999999997	24.4811549941001\\
5762.99999999996	24.483350974302\\
5763.29999999997	24.485546725891\\
5764.00000000002	24.4825731684044\\
5765.20000000001	24.4844153816801\\
5765.80000000005	24.4818675315215\\
5768.69999999999	24.4782277076688\\
5769.30000000001	24.4808818941311\\
5770.09999999998	24.4774877820526\\
5771.79999999998	24.4754760130979\\
5772.10000000002	24.4794012681473\\
5772.59999999996	24.4772809950283\\
5775.19999999996	24.4749190518242\\
5775.50000000001	24.478842025071\\
5776.3	24.4754518385153\\
5776.60000000003	24.477642944934\\
5778.69999999998	24.4704782999931\\
5779.00000000001	24.4743991278919\\
5779.50000000001	24.472281818811\\
5779.70000000001	24.4748953250978\\
5781.00000000003	24.4711214128799\\
5781.40000000002	24.4746173138459\\
5782.90000000005	24.471727477088\\
5783.19999999996	24.4739162761745\\
5784.10000000002	24.4701082258566\\
5784.40000000004	24.4722966548535\\
5786.59999999997	24.4699051272746\\
5787	24.4716697482331\\
5788.10000000001	24.4739297190837\\
5789.40000000001	24.4770705587702\\
5790.39999999997	24.4728434504792\\
5790.60000000002	24.4754520178908\\
5791.1	24.4733388589584\\
5791.39999999996	24.4755244755245\\
5792.30000000003	24.4734479663007\\
5793.79999999996	24.468837915739\\
5794.09999999999	24.4710227468848\\
5795	24.4689479042639\\
5795.29999999999	24.4711322773234\\
5795.89999999999	24.4685990338164\\
5796.09999999997	24.4712052724199\\
5796.89999999999	24.4747283077454\\
5797.39999999996	24.4726175075464\\
5799.4	24.4745236658333\\
5800.80000000002	24.4772362909204\\
5801.70000000001	24.4734392774656\\
5802.39999999996	24.4756570443774\\
5802.79999999999	24.4774164641817\\
5804.70000000004	24.4694046306505\\
5805.30000000005	24.4720432700589\\
5806.09999999997	24.4755606076263\\
5806.89999999998	24.4773549164801\\
5807.59999999998	24.4744046696627\\
5808.69999999997	24.4783776339347\\
5809.79999999999	24.4806278937675\\
5811.6	24.4833697541167\\
5812.49999999998	24.4812992464646\\
5813.99999999998	24.4835830137081\\
5815.20000000002	24.4854091792341\\
5816.39999999997	24.4820768503396\\
5817.40000000005	24.4795874516545\\
5818.1	24.4817984943797\\
5819.10000000003	24.4793098707726\\
5820.49999999997	24.4768580558705\\
5820.90000000004	24.4786119223501\\
5822.60000000001	24.4748999604994\\
5823.99999999996	24.4707336755894\\
5824.19999999996	24.473327266796\\
5825.49999999998	24.4712990936556\\
5825.80000000001	24.4734719099195\\
5826.70000000004	24.4714079769342\\
5826.99999999997	24.4735803401349\\
5828.10000000005	24.4758244397927\\
5828.90000000002	24.4776119402985\\
5829.69999999998	24.4793989502213\\
5830.30000000002	24.4768798024149\\
5830.60000000004	24.4790505428165\\
5831.5	24.4769874476987\\
5832.59999999996	24.4792291734531\\
5834.20000000001	24.4725159830657\\
5835.49999999998	24.4704914661731\\
5837.40000000002	24.4728051391863\\
5838.10000000003	24.4750094207119\\
5839.99999999999	24.4721836954847\\
5840.7	24.476099164498\\
5841.89999999998	24.4779185210544\\
5843.19999999998	24.474184108295\\
5843.5	24.4763501950852\\
5845.39999999996	24.4718159267813\\
5845.70000000001	24.4756919497759\\
5846.99999999996	24.4788014571326\\
5849.00000000003	24.4806893368211\\
5851.40000000002	24.4757754421943\\
5852.80000000004	24.4733380033829\\
5852.99999999997	24.4759187439135\\
5853.79999999996	24.4776986282649\\
5854.40000000001	24.4751900247673\\
5854.70000000003	24.4773519163763\\
5854.99999999999	24.479513586446\\
5856.10000000003	24.4817458420136\\
5856.70000000003	24.4792378090425\\
5857.00000000002	24.4813986443803\\
5858.40000000001	24.4772552701203\\
5859.99999999998	24.4825173631849\\
5860.50000000003	24.4855475548579\\
5863.50000000002	24.4883689201173\\
5864.3	24.4901439192415\\
5865.00000000003	24.4923360215512\\
5865.69999999998	24.4945276006683\\
5866.30000000003	24.4920223646529\\
5867.20000000002	24.4899698328021\\
5867.79999999997	24.4925782647966\\
5868.60000000002	24.4892395249374\\
5868.90000000005	24.4930993354916\\
5870.00000000004	24.4953237593908\\
5871.20000000002	24.4903173062184\\
5873.50000000005	24.4824298556252\\
5874.80000000001	24.4804166879436\\
5875.10000000003	24.484272875817\\
5876.39999999998	24.4822598485493\\
5878.39999999999	24.4756315386578\\
5878.70000000003	24.4777845818875\\
5879.60000000001	24.4757385580897\\
5879.79999999997	24.4783074542084\\
5880.79999999999	24.4741451138431\\
5881.40000000004	24.4784493751594\\
5882.20000000005	24.4751202760825\\
5882.39999999999	24.4776880577986\\
5882.99999999999	24.4751916506604\\
5883.80000000002	24.476962558847\\
5885.50000000004	24.4749898056273\\
5885.69999999999	24.4775561520949\\
5886.50000000001	24.4742296062243\\
5886.89999999997	24.4759639884491\\
5887.69999999997	24.4777336186691\\
5888.40000000003	24.4748238091195\\
5890.4	24.4682115270351\\
5891.89999999996	24.4704684317719\\
5893.29999999996	24.4646553772016\\
5894.90000000004	24.459711620017\\
5895.19999999998	24.4635557138738\\
5895.89999999997	24.4606512890095\\
5897.30000000003	24.4633228202259\\
5898	24.465505840864\\
5899.40000000003	24.4596999745741\\
5899.70000000002	24.4635411369877\\
5900.30000000004	24.4610534878991\\
5900.99999999998	24.4632356679263\\
5902.09999999998	24.4586764257396\\
5903.40000000001	24.456678241721\\
5905.10000000003	24.4530244530245\\
5905.50000000001	24.4581414250881\\
5905.99999999999	24.4611503360932\\
5906.69999999998	24.458251506738\\
5907.00000000004	24.4603951177397\\
5907.80000000005	24.4621608354915\\
5908.80000000003	24.4580209514461\\
5909.09999999999	24.4618560888107\\
5909.79999999999	24.4640349244488\\
5910.39999999997	24.4683190931393\\
5911.39999999997	24.4658716061913\\
5911.89999999998	24.4688768606225\\
5912.60000000003	24.4659800091329\\
5916.29999999998	24.4642012034345\\
5917.09999999998	24.4659636314473\\
5918.60000000001	24.4631422440739\\
5918.89999999999	24.4652812975165\\
5920.20000000001	24.4683546441903\\
5920.70000000005	24.4662883394136\\
5921.09999999997	24.4697020874147\\
5921.89999999996	24.4714623438028\\
5923.09999999998	24.4665045921124\\
5924.00000000001	24.469539676913\\
5924.7	24.4717121253038\\
5926.40000000002	24.466379819455\\
5926.80000000001	24.469790278223\\
5927.29999999997	24.4677261531194\\
5927.7	24.4694490367421\\
5928.9	24.4661831674819\\
5929.29999999998	24.4695922015718\\
5930.00000000004	24.4667037655352\\
5931.7	24.4630634883172\\
5932.10000000004	24.4647854084488\\
5933.70000000004	24.4666149853382\\
5934.50000000003	24.4633168200047\\
5934.69999999997	24.4658623710993\\
5935.19999999998	24.4638013242802\\
5936.39999999999	24.4655942053398\\
5937.8	24.4598258643628\\
5939.09999999998	24.4578394396552\\
5939.49999999998	24.4595595663008\\
5939.99999999996	24.4575007154762\\
5940.20000000005	24.4600441055166\\
5941.50000000001	24.454692338764\\
5943.50000000005	24.456558314826\\
5945.90000000001	24.4584594685503\\
5946.79999999999	24.4614841345911\\
5947.5	24.4586051516578\\
5947.69999999996	24.4611452974209\\
5948.60000000004	24.4574444836687\\
5948.99999999997	24.4591618900338\\
5951.10000000002	24.4572523188601\\
5952.50000000003	24.4548600611498\\
5952.69999999996	24.4573981991668\\
5953.60000000001	24.4553806876396\\
5955.29999999998	24.4534372166437\\
5956.19999999996	24.4514211842923\\
5956.40000000004	24.4539578611601\\
5957.69999999997	24.4486219745544\\
5958.00000000003	24.4507477215891\\
5959.19999999999	24.4475022234155\\
5959.39999999998	24.4500377548452\\
5959.99999999999	24.4475763829466\\
5960.30000000003	24.4497013623247\\
5960.9	24.4472403959067\\
5962.59999999998	24.4453016251027\\
5962.90000000005	24.4474257923864\\
5963.20000000002	24.449549745946\\
5966.50000000002	24.4444072000804\\
5966.90000000004	24.4461203284733\\
5967.70000000002	24.4428432588224\\
5969.79999999996	24.4359201996683\\
5970.20000000004	24.437632949768\\
5971.6	24.435252943048\\
5972.39999999997	24.4370029300963\\
5973.00000000002	24.43454822454\\
5973.70000000001	24.4400549064247\\
5974.30000000002	24.4376004284949\\
5975.20000000001	24.4406138603919\\
5975.69999999997	24.4385688945413\\
5976.00000000003	24.4406887434949\\
5976.49999999999	24.4386440451093\\
5977.70000000003	24.4354110207769\\
5977.99999999996	24.4375303189977\\
5978.30000000003	24.439649404523\\
5980.10000000003	24.4373097889703\\
5981.00000000002	24.4403203424119\\
5982.20000000005	24.4354178158902\\
5982.39999999996	24.4379440033431\\
5983.79999999998	24.4405822289811\\
5984.39999999997	24.4381318405882\\
5985.50000000003	24.4353114140604\\
5986.49999999996	24.439581732536\\
5987.20000000004	24.43672439998\\
5987.6	24.4384321191776\\
5988.09999999996	24.4363915700878\\
5989.19999999999	24.4385821381464\\
5990.39999999997	24.4353559803021\\
5990.70000000003	24.4374707885424\\
5992.09999999999	24.4317612896766\\
5993.29999999997	24.4335435645877\\
5994.19999999999	24.4315432994678\\
5994.59999999997	24.4332493702771\\
5995.00000000003	24.4349552134243\\
5995.59999999998	24.4325099654753\\
5996.80000000002	24.434291050376\\
5997.20000000003	24.4359961982892\\
5998.40000000001	24.4327748603818\\
5998.59999999999	24.435294313768\\
6000.2	24.4371114777595\\
6001.00000000004	24.4338537934712\\
6001.40000000001	24.4355577772224\\
6001.99999999998	24.4331150763899\\
6002.39999999998	24.4348188254894\\
6004.10000000004	24.4378934745678\\
6004.79999999998	24.435044713484\\
6005.79999999996	24.4326412361178\\
6006.50000000001	24.4347883994273\\
6007.40000000005	24.4311277569705\\
6009	24.4262867983558\\
6009.19999999999	24.4288020235302\\
6010.00000000002	24.4255503236219\\
6010.29999999996	24.4293225076534\\
6010.7	24.4310241565183\\
6011.10000000002	24.4327255789194\\
6011.50000000004	24.4344267749019\\
6012.39999999997	24.4307692307692\\
6013.80000000003	24.4284075225727\\
6015.00000000005	24.4301840368406\\
6016.30000000001	24.433215876604\\
6016.99999999997	24.4303734357082\\
6017.99999999999	24.4279756069191\\
6018.39999999996	24.4296751682313\\
6019.50000000003	24.4268722174231\\
6020.39999999997	24.4248816543477\\
6020.80000000003	24.4265807437426\\
6021.69999999996	24.422930020924\\
6023.30000000001	24.4197629245941\\
6023.49999999999	24.4222723952454\\
6023.89999999999	24.4239707835325\\
6025.09999999999	24.4191064197039\\
6025.39999999996	24.4228694714132\\
6026.79999999995	24.4171962368714\\
6028.89999999996	24.4103499751203\\
6030.39999999996	24.4125694386867\\
6030.69999999998	24.4146713537176\\
6031.19999999997	24.4126473562914\\
6031.80000000005	24.4151925595583\\
6032.80000000002	24.4111455518905\\
6034.30000000005	24.4083918865173\\
6037.4	24.4024844720497\\
6038.00000000003	24.4050280717444\\
6040.29999999999	24.3973909012648\\
6041.29999999997	24.3950077796537\\
6042.00000000004	24.3988017411165\\
6042.80000000003	24.4005361664102\\
6043.80000000002	24.4031171925412\\
6044.10000000005	24.4052149167797\\
6045.39999999999	24.4082375320486\\
6046.29999999996	24.4046043926965\\
6047.69999999998	24.4022619795628\\
6049.19999999998	24.3995172995223\\
6050.49999999995	24.3975804052491\\
6050.89999999996	24.4025780862667\\
6052.60000000002	24.405637153667\\
6055.20000000004	24.4034151899988\\
6055.50000000002	24.4071603144197\\
6056.30000000002	24.4088897695\\
6057.50000000001	24.405705229794\\
6058.90000000002	24.4033668922264\\
6060.00000000003	24.400587449052\\
6061.39999999996	24.404850284583\\
6062.4	24.400824742268\\
6062.6	24.4033186534053\\
6063.59999999999	24.3992941603312\\
6064.89999999997	24.3973619126134\\
6065.80000000002	24.3953906262879\\
6066.29999999998	24.4016220493208\\
6067.00000000004	24.3988066786438\\
6068.20000000004	24.3956297480349\\
6069.39999999997	24.3924540736469\\
6069.59999999998	24.3949453844506\\
6070.60000000002	24.3909269112294\\
6073.19999999997	24.388717830504\\
6073.40000000005	24.3912077056063\\
6075.00000000001	24.3880759164458\\
6075.30000000002	24.391809592784\\
6077.3	24.3936551814921\\
6077.99999999997	24.3957815764795\\
6078.59999999997	24.3933735831675\\
6079.70000000002	24.3955393269515\\
6080.49999999999	24.3972634279512\\
6081.59999999997	24.3928506831971\\
6082.80000000004	24.3946144108895\\
6084.10000000003	24.3976200650866\\
6086.50000000001	24.4011434955476\\
6086.80000000003	24.4032266013899\\
6087.99999999998	24.4049867774839\\
6088.69999999999	24.402181053738\\
6089.00000000002	24.4042633558326\\
6090.2	24.3994548708602\\
6090.40000000003	24.4019374435596\\
6090.90000000003	24.3999343293384\\
6091.20000000002	24.4020159900186\\
6092.00000000002	24.3988115756472\\
6092.60000000003	24.4013327424623\\
6093.6	24.3973283883355\\
6095.4	24.3950455253876\\
6095.69999999999	24.397125889957\\
6096.70000000004	24.3947644666054\\
6098.29999999996	24.3965630329267\\
6098.99999999998	24.3937630142152\\
6099.20000000003	24.3962421917269\\
6100.1	24.3926428641684\\
6100.39999999998	24.3947217441193\\
6101.10000000001	24.391922900413\\
6102.00000000004	24.389964110716\\
6103.1	24.3872067112335\\
6103.39999999999	24.3892848365692\\
6104.90000000004	24.3865683865684\\
6105.49999999997	24.3890854297694\\
6105.8	24.3911626459654\\
6106.39999999998	24.3936788667813\\
6107.30000000005	24.3900841601991\\
6109.40000000003	24.3866110156314\\
6110.19999999997	24.3883279053402\\
6111.7	24.3905232501064\\
6112.29999999997	24.3930371049015\\
6112.80000000002	24.3910418950089\\
6114.00000000003	24.3927969774783\\
6114.89999999996	24.3908421913328\\
6115.59999999997	24.3929558349821\\
6117.10000000003	24.386974432747\\
6118.09999999996	24.3846229283123\\
6118.50000000001	24.3879318798418\\
6119.69999999998	24.3913199777771\\
6120.40000000003	24.3885303488277\\
6121.10000000002	24.3906423577076\\
6121.69999999998	24.3882518213597\\
6122.89999999999	24.390004899559\\
6125.50000000004	24.3878150711767\\
6128.40000000001	24.3860651056539\\
6132.10000000002	24.3827663807443\\
6132.3	24.3852325353858\\
6134.3	24.3821726656234\\
6134.70000000002	24.3871030840451\\
6136.00000000001	24.3900849073516\\
6136.70000000005	24.3873028288359\\
6137.10000000005	24.3922309848139\\
6138.8	24.3903630943654\\
6139.49999999998	24.3924685647273\\
6140.30000000002	24.3941762751612\\
6142.49999999996	24.3968352163579\\
6144.60000000001	24.3901248230182\\
6145.29999999999	24.3922283333876\\
6146.59999999997	24.3952039305644\\
6147.19999999996	24.399329787061\\
6149.70000000001	24.3942892451787\\
6150.50000000001	24.3959938867753\\
6151.20000000005	24.3932176938208\\
6151.39999999996	24.3956758514184\\
6152.70000000005	24.39377194123\\
6153.70000000005	24.391432935747\\
6153.89999999996	24.3938901527462\\
6154.89999999997	24.391551584078\\
6155.09999999997	24.39400831817\\
6155.49999999996	24.3972967704204\\
6158.10000000001	24.395115455815\\
6158.29999999999	24.3975707976098\\
6159.00000000002	24.3947979412577\\
6159.20000000002	24.3972529345867\\
6159.49999999998	24.3993116436132\\
6161.19999999997	24.3958255562949\\
6163.29999999998	24.3923808287633\\
6164.5	24.3892547772767\\
6166.80000000004	24.3866448296551\\
6167.2	24.3915489760511\\
6172.20000000004	24.3960922184599\\
6173.29999999996	24.3917452295332\\
6174.09999999997	24.3950633280425\\
6175.59999999997	24.3972343216154\\
6177.50000000001	24.3994431494432\\
6178.00000000005	24.3974684773636\\
6178.30000000002	24.4011394535802\\
6179.50000000001	24.3980192892744\\
6180.70000000005	24.3997540771421\\
6181.79999999997	24.4018829162555\\
6183.1	24.3983697761677\\
6183.80000000002	24.4004592571031\\
6184.29999999997	24.3984865144557\\
6184.70000000001	24.4017591514681\\
6186.79999999999	24.3983254941893\\
6187.00000000005	24.4007693426646\\
6188.10000000004	24.4028958340067\\
6189	24.4058102147324\\
6190.00000000001	24.4018674981018\\
6193.29999999998	24.4001679206898\\
6195.40000000003	24.3983536437737\\
6196.80000000004	24.4009101324856\\
6199.60000000002	24.3979547397455\\
6200.30000000003	24.4000387071802\\
6202.10000000001	24.3977943310438\\
6203.00000000003	24.3958665828376\\
6203.90000000004	24.3939393939394\\
6204.60000000005	24.3960223701388\\
6205.30000000002	24.3932703774132\\
6205.80000000001	24.3961391578981\\
6206.29999999997	24.3941737561227\\
6206.70000000003	24.3974350712122\\
6209.40000000002	24.3916579434737\\
6212.09999999998	24.3891053089083\\
6213.2	24.3912252748137\\
6213.80000000001	24.393697999646\\
6214.79999999995	24.3897729649713\\
6215.10000000005	24.3918136182263\\
6216.90000000002	24.3847514878559\\
6218.10000000002	24.3864784021099\\
6218.69999999996	24.3889496365858\\
6220.49999999998	24.394752917725\\
6221.09999999997	24.3924001800296\\
6222.50000000001	24.3901263137595\\
6223.99999999998	24.3858549830498\\
6224.29999999999	24.3894993894994\\
6226.60000000004	24.3869144169464\\
6226.80000000005	24.3893430117715\\
6227.39999999998	24.3869931754316\\
6227.90000000001	24.3898522800257\\
6228.59999999997	24.387111275226\\
6230.40000000002	24.3848808281839\\
6231.30000000005	24.3893828032224\\
6232.29999999996	24.3918875553559\\
6233.40000000001	24.3940001604235\\
6234.50000000001	24.3913001636031\\
6234.69999999998	24.3937255405145\\
6235.59999999999	24.3982231345318\\
6237.5	24.3939976914198\\
6237.70000000001	24.3964218153836\\
6239.90000000001	24.3990384615385\\
6240.39999999996	24.3970835670219\\
6241.89999999996	24.3992310157001\\
6242.80000000004	24.3957135305707\\
6243.5	24.3977833301301\\
6244.09999999999	24.3954389673617\\
6244.40000000005	24.3990711826407\\
6245.59999999998	24.4007877419665\\
6246.50000000001	24.3988729869049\\
6247.19999999999	24.4009412066013\\
6248.80000000002	24.3962937476996\\
6250.90000000001	24.3912973924172\\
6251.20000000003	24.3933261881529\\
6252.7	24.3890736949847\\
6254.99999999998	24.3817045291043\\
6255.20000000005	24.3841222643199\\
6257.10000000005	24.3783161797609\\
6258.69999999998	24.3752796063143\\
6260	24.3734125652945\\
6260.30000000004	24.3754392690563\\
6261.79999999996	24.3775850780115\\
6263.19999999996	24.3801191065413\\
6263.89999999998	24.382183908046\\
6264.50000000004	24.3846374868308\\
6265.29999999996	24.3879081942095\\
6266.00000000002	24.3899714335871\\
6267.40000000001	24.3925009972078\\
6268.40000000001	24.3902049932201\\
6268.90000000005	24.3946402935077\\
6269.49999999999	24.3923057292331\\
6270.59999999996	24.3896215733491\\
6272.10000000004	24.3853831191607\\
6273.80000000003	24.3819633720652\\
6273.99999999998	24.3843738544174\\
6276.30000000004	24.3865910394494\\
6277.80000000005	24.3903216043581\\
6280.10000000003	24.3829814337123\\
6280.69999999995	24.3870207616864\\
6283.19999999999	24.3852752534496\\
6285.59999999997	24.3807372289482\\
6286.00000000005	24.383958257107\\
6287.20000000004	24.3856663432634\\
6287.90000000002	24.3877226463104\\
6290.00000000003	24.3811704106453\\
6291.10000000003	24.378496948118\\
6291.99999999996	24.3813671111394\\
6292.89999999998	24.3778801843318\\
6295.40000000002	24.3761416885077\\
6295.70000000002	24.3781568664824\\
6296.40000000005	24.3754466767252\\
6296.59999999998	24.377848714406\\
6298.19999999998	24.3795944937523\\
6299.50000000002	24.3777382690965\\
6299.79999999997	24.3797520595565\\
6300.50000000001	24.3833920578993\\
6302.19999999997	24.3863351474858\\
6302.7	24.3844005838675\\
6302.99999999999	24.3864130348559\\
6303.50000000003	24.3844787105781\\
6304.10000000004	24.3869166587355\\
6305.30000000001	24.3886192787135\\
6306.39999999998	24.3859510029335\\
6306.69999999999	24.3895477896873\\
6307.79999999999	24.3916358851599\\
6309.89999999998	24.3882725832013\\
6310.20000000001	24.3902825539198\\
6312.19999999998	24.3920599464537\\
6314.20000000003	24.3890850925677\\
6316.19999999998	24.3908617386761\\
6316.90000000001	24.3929080259617\\
6318.29999999996	24.3906685236769\\
6319.29999999997	24.3883913029718\\
6319.49999999999	24.3907842268498\\
6320.69999999997	24.3924819643083\\
6322.99999999999	24.3883538137939\\
6325.79999999997	24.3902053462748\\
6326.89999999996	24.392287023866\\
6328.39999999996	24.3896657975824\\
6330.10000000001	24.3862753151559\\
6330.30000000002	24.3886642234298\\
6331.70000000001	24.3911683881361\\
6332.60000000005	24.3877019280875\\
6333.30000000001	24.3897432658604\\
6333.8	24.3925543504002\\
6334.50000000004	24.3898588703312\\
6336.20000000002	24.3943626406578\\
6336.70000000002	24.3987501578084\\
6337.00000000004	24.4007511322214\\
6339.20000000001	24.3970154433455\\
6340.60000000002	24.3995142492154\\
6341.89999999995	24.3960895616525\\
6344.29999999998	24.4010465922703\\
6345.4	24.3968166417146\\
6345.90000000001	24.4011976047904\\
6348.59999999997	24.3971206703735\\
6348.89999999995	24.3991179713341\\
6349.49999999995	24.3968123976314\\
6349.69999999997	24.3991936753914\\
6350.70000000002	24.3953517667066\\
6350.90000000004	24.3977326405291\\
6356.59999999996	24.3994525461324\\
6360.20000000002	24.3966479568574\\
6361.10000000003	24.3994843740175\\
6362.59999999998	24.3937322205982\\
6362.89999999999	24.3957252868144\\
6363.90000000003	24.3934632306725\\
6365	24.3908186831315\\
6365.49999999997	24.3936156843031\\
6367.20000000003	24.3918144268371\\
6368.60000000005	24.3943033900168\\
6369.1	24.3970985367079\\
6371.40000000003	24.3945695676057\\
6371.69999999995	24.3965598418029\\
6371.99999999996	24.398549928595\\
6372.60000000004	24.3962527657037\\
6373.89999999998	24.3928459366175\\
6374.20000000004	24.3948355113503\\
6375.00000000003	24.3917742466785\\
6375.3	24.3937635285629\\
6377.60000000004	24.3896702573028\\
6377.80000000005	24.3920412675018\\
6378.10000000001	24.3940296635414\\
6379.30000000003	24.3972787409474\\
6379.70000000003	24.4004514248096\\
6382.79999999995	24.3933008507105\\
6384.09999999995	24.3961655336612\\
6384.60000000002	24.3942550158974\\
6384.90000000002	24.3962411902897\\
6385.49999999999	24.3939488849912\\
6386.60000000003	24.3913131977391\\
6386.80000000002	24.3936808154191\\
6389.29999999997	24.3966569630951\\
6391.10000000001	24.4023031668544\\
6391.80000000005	24.3996307827094\\
6392.89999999999	24.3969967151572\\
6393.19999999996	24.3989801823784\\
6394.49999999997	24.3940199543365\\
6394.79999999998	24.3960030649424\\
6395.49999999997	24.3980236412534\\
6399.09999999998	24.3952369046131\\
6400.00000000001	24.3933688536117\\
6400.70000000002	24.3953880764904\\
6401.40000000003	24.3974068577677\\
6402.19999999997	24.394358277494\\
6402.80000000005	24.3967577191585\\
6403.40000000003	24.394471773249\\
6403.70000000004	24.3964521065617\\
6404.69999999999	24.4020109917562\\
6405.20000000005	24.4001061620845\\
6407.29999999999	24.3983519056091\\
6408.70000000004	24.4008238671826\\
6410.10000000004	24.3970546940813\\
6410.39999999995	24.3990328367522\\
6412.1	24.395683228845\\
6413.49999999996	24.3981539229138\\
6414.19999999999	24.3954913240728\\
6415.3	24.3991021604265\\
6416.20000000001	24.3972382837461\\
6416.39999999996	24.39959479467\\
6417.10000000001	24.4016081780216\\
6417.60000000004	24.3997070601617\\
6418.70000000003	24.3970835670219\\
6421.19999999999	24.3891423854983\\
6422.49999999997	24.3873197770373\\
6423.39999999996	24.3901299914377\\
6423.70000000005	24.3921043619042\\
6424.39999999999	24.3894466495447\\
6426.99999999997	24.3842479500864\\
6428.40000000002	24.386715407949\\
6429.90000000003	24.3825816485226\\
6430.60000000001	24.384592657098\\
6431.89999999997	24.3827736318408\\
6433.79999999999	24.3802359377671\\
6435.50000000003	24.3784573310958\\
6435.90000000001	24.3816034804226\\
6436.8	24.3781944724945\\
6437.59999999996	24.3813784426115\\
6438.70000000002	24.3834254830093\\
6439.40000000004	24.385433651681\\
6440.6	24.3886534072383\\
6441.10000000003	24.3867602310129\\
6441.39999999996	24.3887293332298\\
6443.60000000005	24.3912658876112\\
6444.69999999999	24.3948609731877\\
6446.39999999999	24.3930815171023\\
6447.90000000001	24.3905086848635\\
6449.00000000001	24.392550898575\\
6449.60000000005	24.3902817185295\\
6450.00000000004	24.3934202570503\\
6450.70000000005	24.3954238234017\\
6451.40000000001	24.3974269549717\\
};
\addplot [color=mycolor1, forget plot]
  table[row sep=crcr]{%
6451.40000000001	24.3974269549717\\
6452.00000000002	24.3951581655585\\
6452.29999999995	24.3986733618499\\
6453.40000000003	24.3960641512358\\
6454.89999999998	24.3919442292796\\
6455.20000000003	24.3939088810745\\
6455.49999999999	24.3958733502695\\
6456.4	24.3924727019283\\
6457.90000000006	24.3945493960979\\
6459.30000000003	24.3908102919776\\
6460.40000000002	24.3959445863323\\
6461.09999999999	24.3979446542438\\
6462.19999999996	24.39998143076\\
6462.70000000004	24.4027356563719\\
6462.99999999997	24.4046974362148\\
6463.60000000004	24.4024320435664\\
6466.00000000004	24.3995607862545\\
6466.30000000004	24.4030681677595\\
6467.00000000001	24.4050656399313\\
6468.80000000001	24.4029123962343\\
6469.09999999999	24.404872318061\\
6469.79999999999	24.4022318737538\\
6474.30000000004	24.3991721240578\\
6475.19999999999	24.4035025404228\\
6476.20000000001	24.4012785077899\\
6476.50000000002	24.403236265942\\
6476.99999999996	24.4013524571182\\
6477.30000000001	24.4033099700497\\
6477.90000000005	24.4010497066996\\
6478.10000000002	24.4033836559538\\
6478.99999999998	24.4061675232671\\
6479.7	24.4035309731782\\
6481.00000000003	24.4017219299193\\
6481.30000000005	24.405221094208\\
6482.70000000001	24.4030357253039\\
6482.90000000001	24.4053678852383\\
6483.49999999997	24.4031093836757\\
6484.99999999997	24.4051749394766\\
6486.00000000001	24.4075792849324\\
6487.10000000003	24.4034406215316\\
6487.79999999995	24.4054316496863\\
6488.69999999998	24.4082110713845\\
6489.39999999995	24.410201094075\\
6490.09999999998	24.4075683337956\\
6490.4	24.411062321855\\
6491.60000000003	24.4065499021828\\
6492.90000000003	24.4047435699985\\
6493.99999999996	24.4067692212932\\
6495.70000000005	24.4050001539456\\
6495.99999999997	24.4069518634257\\
6497.09999999996	24.408976174352\\
6497.60000000005	24.4070978961786\\
6498	24.4117511272526\\
6499.20000000002	24.4072438570308\\
6499.70000000004	24.4115203544725\\
6501.69999999995	24.4132394106248\\
6503.00000000004	24.4160477310821\\
6503.59999999996	24.413795224257\\
6505.50000000005	24.4158878504673\\
6506.80000000002	24.4140835113495\\
6507.20000000003	24.4171929986323\\
6508.59999999998	24.4150137508258\\
6508.89999999997	24.4169611307421\\
6510.1	24.4201407022826\\
6511.30000000004	24.4171760297325\\
6513.20000000001	24.4146592357177\\
6513.40000000006	24.416980118216\\
6514.80000000004	24.4194078189995\\
6515.29999999998	24.4221383184455\\
6515.80000000005	24.4202642766157\\
6515.99999999997	24.4225840610181\\
6516.80000000002	24.4257238871242\\
6517.89999999997	24.4277385701135\\
6518.7	24.4247407498313\\
6518.90000000005	24.4270593649333\\
6521.39999999998	24.4238288737254\\
6521.69999999997	24.4273053451501\\
6523.6	24.4293882306053\\
6524.10000000002	24.4321142822108\\
6524.60000000001	24.4302420034638\\
6525.80000000002	24.4272820607119\\
6526.1	24.4292237442922\\
6526.59999999997	24.4273522607137\\
6527.20000000002	24.4297029399599\\
6529.29999999996	24.4264404080007\\
6529.69999999995	24.4295384238415\\
6530.80000000005	24.4269549372981\\
6531.10000000003	24.4304262616365\\
6532.19999999999	24.4324357423878\\
6532.70000000006	24.4305657604702\\
6534.70000000005	24.433800575381\\
6537.50000000005	24.4294542339697\\
6537.90000000003	24.4325481798715\\
6538.39999999996	24.4352680278351\\
6540.39999999997	24.4308539102515\\
6540.60000000001	24.4331646459859\\
6541.09999999999	24.4358833241607\\
6541.8	24.4393830538529\\
6542.39999999998	24.4432556362247\\
6543.90000000003	24.4407090464548\\
6546.8	24.4436298095282\\
6547.99999999995	24.4391502878698\\
6548.90000000005	24.4373186746068\\
6549.60000000002	24.4392872956013\\
6550.29999999999	24.4412554958476\\
6552.20000000003	24.4433252445706\\
6552.80000000004	24.4487173617788\\
6553.40000000003	24.4464789806973\\
6554.90000000005	24.4439359267735\\
6555.30000000004	24.4470207767642\\
6555.9	24.444783404515\\
6557.39999999998	24.4468166221883\\
6559.29999999999	24.4427844010123\\
6561.69999999996	24.4368923161328\\
6561.99999999999	24.4403468401884\\
6564.49999999997	24.4386558206136\\
6565.39999999995	24.4368288782271\\
6565.70000000001	24.4387584148162\\
6566.30000000005	24.4410940545809\\
6567.99999999999	24.4393355764985\\
6568.20000000001	24.4416363442595\\
6569.90000000004	24.4398782343988\\
6570.10000000005	24.4421783202947\\
6570.79999999999	24.4395744875131\\
6571.09999999999	24.4430241051863\\
6571.60000000005	24.4411643866884\\
6571.80000000001	24.4434638384638\\
6573.30000000004	24.4409285909879\\
6573.59999999998	24.4428556216438\\
6575.10000000002	24.4448838058158\\
6576.20000000005	24.4423155877925\\
6577.3	24.4443093015477\\
6579.59999999997	24.4464033314589\\
6580.29999999996	24.4483618017142\\
6581.60000000006	24.4526490116535\\
6583.50000000001	24.4501488547299\\
6584.09999999997	24.4524771422496\\
6584.59999999996	24.4506203775419\\
6584.8	24.4529150025057\\
6585.09999999996	24.454838121849\\
6585.60000000001	24.4529814598296\\
6586.20000000001	24.4553087469444\\
6586.60000000005	24.4583782470736\\
6589.50000000004	24.4506495083161\\
6589.99999999996	24.4533466866967\\
6590.3	24.4552682689973\\
6592.79999999998	24.4490285003564\\
6593.20000000005	24.4520953088742\\
6595.10000000002	24.4480834546337\\
6596.20000000005	24.4455224898807\\
6596.70000000004	24.4497332039777\\
6597.39999999999	24.4516862447897\\
6601.6	24.4543072238969\\
6602.10000000003	24.452455242192\\
6602.59999999996	24.4551471367774\\
6603.30000000003	24.4525547445255\\
6603.89999999998	24.4548758328286\\
6605.80000000003	24.4523834753781\\
6606.09999999998	24.4543005055857\\
6606.59999999996	24.4524497858235\\
6606.80000000002	24.4547367146468\\
6607.30000000004	24.4528861579441\\
6608.7	24.4492192228544\\
6609.29999999998	24.4515387175841\\
6610.09999999999	24.4485794680948\\
6611.50000000005	24.4509649706576\\
6613.40000000001	24.4439404248885\\
6614.40000000005	24.4462922367526\\
6616.19999999997	24.4487100040808\\
6617.59999999996	24.4510932801427\\
6618.89999999997	24.4538449916906\\
6620.70000000005	24.4517278878685\\
6621	24.4536406337316\\
6621.70000000006	24.4510556042164\\
6622.29999999997	24.4548804058951\\
6624.09999999999	24.4588025723861\\
6624.60000000002	24.4569565414283\\
6624.99999999999	24.4600081508204\\
6626.59999999998	24.4631566239606\\
6627.20000000003	24.4609418616933\\
6627.39999999999	24.463221425877\\
6628.00000000003	24.4655331090358\\
6628.50000000001	24.4636876565187\\
6630.89999999996	24.4593575629618\\
6631.09999999999	24.461635903004\\
6631.60000000003	24.4597916069786\\
6633.89999999997	24.4573409707567\\
6634.70000000006	24.4604208114789\\
6635.39999999998	24.4578404038882\\
6635.59999999999	24.4601172446012\\
6636.60000000001	24.4624587520906\\
6637.29999999999	24.4598788682315\\
6637.49999999996	24.4621549957816\\
6639.09999999996	24.4577659959031\\
6640.89999999998	24.4556542689354\\
6642.5	24.4512690813838\\
6643.79999999996	24.4494950255121\\
6644.10000000004	24.4529062942115\\
6645.40000000003	24.451132345196\\
6645.6	24.4534059617497\\
6647.20000000005	24.4580506371008\\
6648.89999999996	24.4563092194315\\
6649.19999999997	24.4597175642549\\
6650.50000000001	24.4579436441825\\
6650.99999999998	24.4606155372796\\
6652.20000000002	24.4577063571998\\
6652.50000000001	24.4596097766287\\
6653.09999999998	24.4574039559911\\
6654.40000000002	24.4556315275378\\
6655.69999999995	24.4538597914601\\
6656.29999999997	24.456162490235\\
6657.19999999996	24.4543583735148\\
6657.90000000004	24.4562931811355\\
6658.60000000005	24.4582275819604\\
6659.90000000004	24.4534534534535\\
6660.79999999997	24.4516506778363\\
6661.00000000004	24.4539190223837\\
6661.3	24.4558200978773\\
6661.60000000003	24.4577210021466\\
6662.4	24.4547842401501\\
6662.79999999996	24.4578186675472\\
6663.70000000003	24.4560160869174\\
6663.89999999997	24.4582833133253\\
6664.79999999998	24.4564809674564\\
6665.00000000001	24.4587478057343\\
6665.60000000002	24.4565461991989\\
6666.70000000001	24.4540109197816\\
6666.99999999998	24.4574102683326\\
6668.00000000004	24.4552421229436\\
6669.79999999996	24.4531402269899\\
6669.99999999997	24.4554054661849\\
6671.20000000003	24.4600002997916\\
6672.70000000004	24.4619949646325\\
6673.30000000003	24.4597956064375\\
6673.49999999997	24.4620594581635\\
6675.90000000001	24.4592570401438\\
6676.99999999999	24.4612181935271\\
6678.79999999996	24.4651065295183\\
6682.29999999998	24.4672572728361\\
6684.10000000006	24.4696448340864\\
6687.10000000001	24.4646488814452\\
6688.39999999998	24.4628840547208\\
6688.6	24.4651427033654\\
6689.30000000006	24.4670673005053\\
6691.50000000002	24.4635064857433\\
6691.69999999996	24.4657640694581\\
6692.50000000001	24.4628395541344\\
6694	24.4603456775369\\
6695.49999999999	24.4623334727284\\
6697.40000000006	24.4598730869727\\
6697.59999999996	24.4621287904803\\
6697.99999999997	24.4651468326839\\
6698.70000000003	24.4625903146832\\
6699.40000000001	24.4645122770356\\
6699.80000000004	24.4675293661099\\
6701.30000000002	24.4695138329304\\
6701.89999999995	24.4673231871083\\
6702.19999999999	24.4692120615311\\
6703.90000000002	24.4659904534606\\
6704.09999999999	24.4682437874765\\
6705.49999999999	24.4661178716297\\
6706.99999999995	24.469591924975\\
6710.90000000001	24.4717627775294\\
6712.80000000005	24.4737743747114\\
6713.09999999998	24.4756598939403\\
6714.19999999999	24.4776074944521\\
6714.90000000002	24.4750558451229\\
6715.79999999998	24.4777319495526\\
6716.70000000004	24.4759409242496\\
6717.50000000001	24.4789805883053\\
6718.69999999998	24.4760969220694\\
6721.00000000002	24.4706967609469\\
6723.79999999996	24.467942711819\\
6724.49999999999	24.4698569431639\\
6725.49999999999	24.4677054835256\\
6725.90000000001	24.4721974427594\\
6727.29999999997	24.4700775931266\\
6729.20000000001	24.4720847636455\\
6730.09999999995	24.470298059493\\
6731.30000000004	24.4733636390647\\
6732.30000000003	24.4697284772147\\
6732.89999999999	24.4720035645329\\
6733.40000000004	24.4701863815252\\
6735.40000000004	24.4673743597357\\
6737.10000000005	24.464169091017\\
6738.99999999998	24.4661750085323\\
6740.30000000001	24.4644234763516\\
6743.09999999995	24.4616799145806\\
6743.50000000001	24.4661605077407\\
6744.20000000001	24.4680693326216\\
6744.9	24.4714603409933\\
6746.10000000003	24.4685897245857\\
6746.29999999998	24.4708288865173\\
6747.50000000004	24.4664769695892\\
6747.69999999996	24.4687157295711\\
6748.60000000003	24.4669343725458\\
6749.59999999996	24.4692356697335\\
6750.09999999995	24.4674231874611\\
6751.79999999999	24.4657059494365\\
6751.99999999998	24.4679433065269\\
6754.70000000001	24.4655652276899\\
6754.99999999996	24.4674394161448\\
6755.30000000005	24.4693134381384\\
6757.99999999995	24.4728547964665\\
6758.69999999996	24.470320175179\\
6759.40000000002	24.4722242769436\\
6759.89999999997	24.4704142011834\\
6760.99999999997	24.472349173951\\
6761.89999999996	24.4705708370305\\
6763.60000000002	24.4673773230628\\
6763.89999999999	24.4707273802484\\
6764.7	24.4678334910123\\
6765.29999999996	24.4700978508292\\
6766.2	24.4668430309032\\
6766.59999999998	24.4713080231132\\
6767.70000000004	24.4688081799108\\
6770.99999999995	24.4657441183855\\
6771.20000000001	24.4679751303295\\
6772.20000000002	24.4702685941261\\
6772.70000000003	24.4728915662651\\
6774.80000000003	24.477114053344\\
6775.69999999997	24.4753387053927\\
6776.00000000001	24.4772066527944\\
6777.90000000005	24.4791974033638\\
6778.50000000002	24.4814563479185\\
6779.1	24.4837148926127\\
6780.99999999996	24.4857029095574\\
6783.99999999999	24.4837192847983\\
6784.69999999999	24.4856149039028\\
6785.29999999998	24.487871017184\\
6787.89999999998	24.4829110194461\\
6789.10000000003	24.4800565604195\\
6789.30000000005	24.4822812030518\\
6790.79999999996	24.4783460218822\\
6791.10000000002	24.4802096831193\\
6792.00000000001	24.4784381855391\\
6792.80000000003	24.4814438605014\\
6793.50000000002	24.4833372585963\\
6794.89999999995	24.4797645327447\\
6796.30000000004	24.4820787475722\\
6796.89999999998	24.4799176107106\\
6797.09999999998	24.4821397045842\\
6800.89999999999	24.4802234965446\\
6802.00000000004	24.4777348171888\\
6803.69999999998	24.4819071695229\\
6805.5	24.4754319971788\\
6805.7	24.4776514149696\\
6806.49999999995	24.4747744835895\\
6807.60000000004	24.4766955065587\\
6808.09999999998	24.4748979172175\\
6809.89999999998	24.4713656387665\\
6811.2	24.4755039419788\\
6812.99999999997	24.4719731106251\\
6814.69999999997	24.4688031930504\\
6814.99999999997	24.4706607386539\\
6815.29999999995	24.4725181207266\\
6816.20000000004	24.4692868565057\\
6816.5	24.4711439720682\\
6817.79999999995	24.469411402338\\
6818.10000000004	24.472734739374\\
6820.70000000002	24.4663382594417\\
6821.00000000001	24.468194279515\\
6821.49999999995	24.4664008443767\\
6822.00000000003	24.4690051450433\\
6822.99999999998	24.4668845539418\\
6826.00000000003	24.4693163006695\\
6829.10000000001	24.4713875710186\\
6829.59999999998	24.4739886085772\\
6831.90000000002	24.4759953161593\\
6832.59999999997	24.4778784375137\\
6833.50000000006	24.4746546476235\\
6834.10000000003	24.4768956132393\\
6834.89999999996	24.4740307242136\\
6835.60000000001	24.4759132202993\\
6836.09999999997	24.4741230508177\\
6837.50000000001	24.4764244764245\\
6838.09999999995	24.4742768564827\\
6839.39999999999	24.4725491629505\\
6840.49999999998	24.4744613045639\\
6841.19999999999	24.4763422156608\\
6843.10000000003	24.4783142389525\\
6844.20000000004	24.4758412109346\\
6846.40000000004	24.4738187395019\\
6846.89999999997	24.4764130276033\\
6848.39999999996	24.4739724027159\\
6853.20000000004	24.4714225263742\\
6853.50000000002	24.4732695225867\\
6854.59999999996	24.4708010562096\\
6854.90000000006	24.474106491612\\
6855.90000000004	24.470536756126\\
6857.70000000005	24.4670302429351\\
6858.60000000005	24.4711096855089\\
6860.10000000004	24.4730474330194\\
6861.09999999999	24.4709380283332\\
6864.00000000001	24.4737110473332\\
6864.60000000002	24.4715719550745\\
6864.90000000004	24.4734158776402\\
6866.39999999996	24.4709823053958\\
6867.29999999997	24.4765122171419\\
6867.79999999996	24.4747302668938\\
6868.10000000001	24.4765731923939\\
6868.60000000003	24.4791590839606\\
6870.70000000002	24.4745881120102\\
6871.7	24.4768474053378\\
6872.40000000004	24.4743543106584\\
6872.69999999996	24.4776510301478\\
6873.6	24.4759008976243\\
6873.89999999997	24.4777422170498\\
6876.10000000001	24.4800907477968\\
6876.80000000001	24.4775989181172\\
6877.09999999999	24.4794393066946\\
6877.40000000002	24.4812795347147\\
6877.89999999999	24.4794998546089\\
6879.70000000005	24.4774557399924\\
6883.10000000004	24.4740818224082\\
6884.40000000003	24.472365458639\\
6884.90000000005	24.4749455337691\\
6886.20000000001	24.4732294555857\\
6886.50000000005	24.4750675224349\\
6886.79999999999	24.4769054291481\\
6888.80000000004	24.474154073945\\
6889.09999999996	24.4759914068397\\
6889.90000000001	24.4731494920174\\
6890.90000000005	24.4710491946017\\
6892.59999999997	24.4679153307122\\
6893.10000000001	24.4704926594325\\
6894.50000000004	24.4669741536855\\
6894.69999999998	24.4691651679527\\
6895.39999999996	24.4710318323544\\
6895.99999999997	24.4689027131277\\
6896.99999999995	24.4711545432138\\
6898.10000000003	24.4672523266939\\
6898.60000000005	24.4727267456187\\
6903.00000000002	24.4745114513769\\
6903.60000000001	24.4723843735968\\
6904.09999999996	24.4764056661163\\
6905.40000000005	24.4746940844255\\
6905.79999999997	24.4776205852966\\
6906.90000000002	24.4751701172723\\
6907.20000000002	24.4770025914612\\
6908.60000000002	24.4749373977738\\
6908.79999999996	24.4771237099973\\
6909.60000000006	24.4742897665601\\
6909.79999999997	24.4764757811256\\
6911.2	24.4744114710691\\
6911.50000000002	24.4776896811158\\
6912.29999999996	24.48064348128\\
6913.69999999999	24.482918221528\\
6916.59999999995	24.4798820246649\\
6917.09999999999	24.4824495460591\\
6917.60000000004	24.480679994796\\
6917.90000000003	24.4825093957791\\
6918.9	24.484752131811\\
6919.79999999999	24.4873480830648\\
6920.10000000005	24.4891766133927\\
6921.89999999997	24.4856977752095\\
6923.00000000004	24.4832517224943\\
6923.20000000003	24.4854332471509\\
6924.30000000002	24.481543527237\\
6925.69999999998	24.4838141442144\\
6926.00000000001	24.4856412699788\\
6927.30000000003	24.4824898230216\\
6927.70000000003	24.485406622593\\
6928.20000000006	24.4836395652613\\
6929.29999999998	24.4855254423182\\
6930.20000000002	24.4881173975153\\
6931.09999999998	24.4863804247461\\
6934.70000000001	24.4880890580839\\
6937.1	24.4839416479271\\
6939.5	24.4870021326878\\
6941.19999999995	24.4838863037183\\
6942.49999999997	24.4865036153602\\
6943.30000000004	24.4894432122591\\
6944.20000000004	24.4877093443544\\
6944.60000000001	24.4906187452302\\
6946.80000000003	24.4929392966647\\
6948.10000000003	24.4883566966984\\
6951.00000000003	24.4838946353815\\
6951.90000000004	24.482163406214\\
6953.50000000001	24.4779682466636\\
6954.2	24.479818241951\\
6955.30000000006	24.4816976737499\\
6957.50000000001	24.4797056456249\\
6958.1	24.4819062401196\\
6958.79999999996	24.4794435902226\\
6959.00000000004	24.4816140018106\\
6960.79999999998	24.4752833685299\\
6960.99999999998	24.4774532760627\\
6962.10000000002	24.4793312458706\\
6962.59999999996	24.4775733551639\\
6964.19999999998	24.4748215901096\\
6964.39999999997	24.4769904515759\\
6964.70000000005	24.4788077188146\\
6966.30000000004	24.4746210381259\\
6967.89999999999	24.4776119402985\\
6968.40000000006	24.4758556360766\\
6972.49999999996	24.4700685540544\\
6973.60000000003	24.4719445918236\\
6974.00000000006	24.4748426320242\\
6975.69999999996	24.4717451761805\\
6975.90000000002	24.4739105504587\\
6977.20000000004	24.4765167041692\\
6978.30000000002	24.478390461997\\
6980.20000000005	24.4803231952781\\
6980.70000000005	24.4785697914279\\
6981.39999999999	24.481844875743\\
6983.10000000006	24.4758849810975\\
6983.30000000005	24.4780479422631\\
6983.99999999996	24.4798900359388\\
6984.69999999996	24.481731760394\\
6985.20000000003	24.4799793852805\\
6986.79999999995	24.4758047202622\\
6987.29999999997	24.4797778859089\\
6987.8	24.4823194378855\\
6989.39999999997	24.4852993776379\\
6992.29999999999	24.4822950632115\\
6992.59999999996	24.484104852203\\
6995.7	24.4804025272306\\
6995.90000000001	24.4825614636935\\
6997.1	24.4797919167667\\
6997.30000000005	24.4819504387344\\
6998.69999999995	24.4784820254901\\
7000.19999999998	24.4818079225176\\
7000.70000000003	24.4800594217804\\
7000.90000000006	24.4822168261677\\
7002.99999999999	24.4862989247619\\
7003.3	24.4881057771939\\
7004.40000000001	24.4856877721465\\
7007.49999999999	24.4876990695816\\
7007.80000000001	24.4895047018365\\
7008.49999999999	24.487058756385\\
7009.20000000002	24.4888933274364\\
7009.7	24.4871465662358\\
7009.90000000002	24.4893009985735\\
7011.50000000002	24.485138912659\\
7012.69999999996	24.489504905316\\
7013.20000000003	24.4920365591091\\
7014.30000000004	24.489621350365\\
7014.60000000001	24.4914251500421\\
7015.99999999998	24.4879633984692\\
7016.30000000006	24.4897668319936\\
7016.60000000001	24.4915701113059\\
7017.20000000002	24.4894760092913\\
7017.39999999999	24.4916280726755\\
7018.09999999999	24.4934598615029\\
7019.10000000002	24.4913950307727\\
7019.29999999995	24.4935464569621\\
7020.60000000004	24.4890110672725\\
7023.19999999999	24.4842168211525\\
7025.10000000001	24.486135626032\\
7025.80000000002	24.4836960389416\\
7027.70000000005	24.4870371951393\\
7029.50000000005	24.4836121543189\\
7031.90000000006	24.4866325369738\\
7033.40000000004	24.4814103931187\\
7034.00000000004	24.4835870971411\\
7035.5	24.4854738757178\\
7039.20000000003	24.4825479806231\\
7041.00000000006	24.4848106119782\\
7041.70000000004	24.4823766650572\\
7042.00000000006	24.4841737549879\\
7043.09999999995	24.486029077692\\
7044.09999999996	24.4882314528264\\
7044.80000000004	24.4928955698449\\
7047.20000000005	24.4888113178097\\
7049.50000000006	24.4907512482978\\
7052.10000000004	24.4888120019285\\
7053	24.4871049609392\\
7054.30000000006	24.4840099795872\\
7055.69999999998	24.4876555457921\\
7056.40000000002	24.4908949195777\\
7057.10000000001	24.4927166581647\\
7057.80000000003	24.4959548874311\\
7058.49999999995	24.4977757628992\\
7059.80000000004	24.5003470304112\\
7061.00000000001	24.4961833142145\\
7061.30000000005	24.497974905826\\
7063.50000000003	24.4960077014554\\
7065.90000000002	24.4990093405038\\
7066.4	24.4972758791481\\
7067.00000000004	24.4994410720097\\
7067.60000000004	24.5016058972509\\
7068.50000000004	24.4999009704892\\
7068.70000000003	24.5020371208692\\
7070.40000000001	24.498974612828\\
7072.3	24.5008766472485\\
7073.00000000006	24.5026933027951\\
7074.89999999999	24.504593639576\\
7075.20000000002	24.5063813548542\\
7077.09999999999	24.5082801107783\\
7077.6	24.5065487375842\\
7077.80000000004	24.5086819536868\\
7078.09999999998	24.5104687632449\\
7079.19999999996	24.5080728320597\\
7080.59999999998	24.5046393718135\\
7083.29999999999	24.5080046305447\\
7084.29999999999	24.5101914064706\\
7085.90000000004	24.5131244707875\\
7088.19999999997	24.5150459207426\\
7089.19999999998	24.511587885969\\
7089.59999999999	24.5158469328745\\
7090.10000000004	24.5141180784745\\
7090.49999999997	24.51837644205\\
7091.49999999995	24.5163291781826\\
7091.79999999997	24.5181122125242\\
7092.29999999995	24.5163837347019\\
7092.49999999996	24.5185122522065\\
7093.39999999999	24.5154014238387\\
7093.79999999998	24.5182480722875\\
7094.39999999998	24.5161745013743\\
7094.69999999998	24.5179568134408\\
7095.69999999997	24.5201386735816\\
7096.4	24.5219474388783\\
7099.89999999995	24.5253521126761\\
7101.20000000003	24.5208623773112\\
7101.4	24.5229881011054\\
7102.20000000004	24.5202258423328\\
7104.39999999999	24.5182630727004\\
7104.60000000004	24.5203879122271\\
7105.89999999997	24.5229383619477\\
7107.29999999998	24.5209218560937\\
7110.09999999998	24.5182976568873\\
7110.39999999999	24.5200759440264\\
7111.80000000002	24.518061277577\\
7112.00000000004	24.5201839119247\\
7112.50000000004	24.5184601973962\\
7114.99999999998	24.5210889516662\\
7117.19999999996	24.5163193907802\\
7117.40000000001	24.5184404636459\\
7118.09999999995	24.5160293332584\\
7119.49999999997	24.5126130681499\\
7120.80000000005	24.5151596006123\\
7121.80000000005	24.5131214984765\\
7122.00000000005	24.5152412911922\\
7123.50000000001	24.5185018810714\\
7123.80000000003	24.5202768146661\\
7128.09999999999	24.5181111641088\\
7129.19999999997	24.5157308571669\\
7130.30000000005	24.5133512846404\\
7130.49999999995	24.5154685440215\\
7132.40000000001	24.5131440588854\\
7132.60000000006	24.5152607007164\\
7134.59999999996	24.5181997841535\\
7136.40000000006	24.5162194352974\\
7136.59999999999	24.5183348046016\\
7137.09999999996	24.5166171607913\\
7138.99999999996	24.514294518917\\
7139.70000000003	24.516092887756\\
7140.00000000004	24.5178638954636\\
7142.20000000004	24.5215126779889\\
7142.69999999995	24.5197961583693\\
7142.89999999996	24.5219095618088\\
7143.89999999995	24.5240761478164\\
7144.79999999997	24.5265853965766\\
7145.70000000005	24.5234963195164\\
7146.39999999995	24.5252921010285\\
7147.89999999996	24.5215444879687\\
7150.00000000001	24.5157410385869\\
7150.30000000004	24.5175095099575\\
7151.40000000001	24.5151366846116\\
7151.69999999995	24.5169048351464\\
7152.90000000001	24.5141898504124\\
7153.19999999998	24.5159576698866\\
7157.59999999998	24.5134610279839\\
7159.20000000002	24.5107761932032\\
7159.50000000004	24.5125426001453\\
7162.99999999996	24.5089416593374\\
7163.20000000001	24.5110493766839\\
7165.09999999998	24.5157148439681\\
7165.6	24.514004214522\\
7168.09999999996	24.511034848358\\
7173.00000000002	24.5082321450977\\
7173.59999999995	24.5103642471807\\
7174.3	24.5121543264942\\
7175.30000000006	24.5143127909245\\
7175.99999999995	24.5119215172587\\
7177.50000000003	24.508192153366\\
7177.69999999997	24.5102956337597\\
7177.99999999995	24.5120575082543\\
7179.80000000004	24.5100906697865\\
7180.10000000002	24.5118520375477\\
7181.90000000001	24.515455304929\\
7182.30000000003	24.5182668745823\\
7183.60000000004	24.5207901220819\\
7185.10000000003	24.5156711017091\\
7185.29999999997	24.5177721490801\\
7186.80000000002	24.5154378104607\\
7188.39999999996	24.5113723308062\\
7190.10000000003	24.5069678173069\\
7191.59999999995	24.5046372902096\\
7191.79999999999	24.5067367455054\\
7192.30000000001	24.5050330904844\\
7193.90000000003	24.5023630803447\\
7194.20000000006	24.5041213182659\\
7194.89999999999	24.5017373175817\\
7195.10000000002	24.5038358905937\\
7196.80000000006	24.4994372577082\\
7197.00000000001	24.5015353406233\\
7197.50000000006	24.5040013337779\\
7198.00000000001	24.5022992178492\\
7199.90000000002	24.5041666666667\\
7201.49999999998	24.5014996667407\\
7201.79999999997	24.5032560851997\\
7203.20000000005	24.5054350089542\\
7203.90000000002	24.5030538589672\\
7204.09999999998	24.5051497737431\\
7209.2	24.5030724203459\\
7209.70000000001	24.508308136148\\
7210.2	24.5066086015838\\
7210.40000000002	24.5087025865058\\
7211.6	24.5060110653521\\
7212.09999999996	24.5098582956657\\
7215.20000000003	24.5048715923108\\
7216.70000000004	24.5025496064738\\
7219.39999999999	24.5003116559319\\
7220.69999999997	24.497285619322\\
7220.99999999999	24.4990375427566\\
7221.99999999996	24.4970299497376\\
7223.50000000002	24.4933274267678\\
7224.40000000002	24.4958128590214\\
7225.39999999996	24.4979586187807\\
7225.99999999999	24.5000761129793\\
7227.09999999998	24.4977307947753\\
7232.30000000006	24.493943919031\\
7233.00000000003	24.4957210601264\\
7234.40000000002	24.4937452484622\\
7234.70000000004	24.4954940012163\\
7235.69999999995	24.4934906990243\\
7236.39999999997	24.4952670489878\\
7237.39999999998	24.4974093264249\\
7238.30000000001	24.4943633952255\\
7241	24.4893731615362\\
7242.90000000003	24.4912329145382\\
7244.19999999997	24.4882183233715\\
7244.49999999998	24.4899649394031\\
7245.09999999998	24.487936840943\\
7247.49999999999	24.4812075721618\\
7247.69999999998	24.483291481553\\
7248.80000000001	24.4850942901682\\
7251.40000000002	24.4832103702682\\
7251.60000000001	24.485293103686\\
7252.29999999999	24.4870663504495\\
7253.90000000004	24.4899365867108\\
7254.59999999995	24.4917088232456\\
7255.50000000003	24.4941838028557\\
7256.99999999998	24.491876920533\\
7259.79999999999	24.4893180346837\\
7263.00000000006	24.4867893874516\\
7264.40000000005	24.4848234565352\\
7264.69999999996	24.4865653562383\\
7265.29999999998	24.4845431772511\\
7265.49999999996	24.4866218894517\\
7266.8	24.4891219089295\\
7267.49999999998	24.4867631680335\\
7267.79999999995	24.4898801579548\\
7269.09999999998	24.4868761349254\\
7269.70000000003	24.4917329225013\\
7274.00000000004	24.4896275827938\\
7275.70000000003	24.4921520657522\\
7276.49999999996	24.494956435698\\
7278.40000000006	24.496805660507\\
7280.40000000005	24.4928232951034\\
7282.10000000005	24.4898519678119\\
7283.60000000005	24.4861814737016\\
7284.70000000003	24.4893476828465\\
7287.30000000006	24.486099294673\\
7288.39999999999	24.4878918844755\\
7289.20000000004	24.4852043406088\\
7291.79999999998	24.4874449731894\\
7292.69999999999	24.4844229925406\\
7293.10000000006	24.4871935501563\\
7294.80000000006	24.4814870663066\\
7296.29999999996	24.4791952195603\\
7296.6	24.4809297353598\\
7297.30000000003	24.4826924658097\\
7298.2	24.479673348588\\
7298.4	24.4817428238679\\
7302.20000000002	24.4785889377319\\
7302.40000000001	24.4806573091407\\
7303.00000000004	24.4786460544153\\
7303.20000000003	24.4807141976915\\
7307.99999999997	24.4837919568698\\
7309.80000000006	24.4818670570049\\
7310.10000000002	24.4835982599655\\
7310.70000000002	24.4815888822017\\
7310.99999999996	24.4833198834649\\
7312.09999999999	24.4851070813162\\
7315.89999999999	24.4874248223073\\
7317.39999999997	24.4892381277759\\
7318.40000000005	24.4858919177427\\
7319.89999999998	24.4836065573771\\
7323.29999999996	24.4804325859573\\
7324.60000000002	24.4774530014881\\
7325.90000000003	24.4744744744745\\
7326.19999999999	24.4775671211935\\
7326.80000000002	24.4796571537758\\
7328.70000000002	24.4814976530946\\
7329.79999999997	24.479187983465\\
7330.00000000003	24.4812485504973\\
7331.00000000001	24.4792732332119\\
7332.19999999997	24.4766307979761\\
7332.50000000003	24.4783569266018\\
7333.59999999999	24.4746853566413\\
7334.10000000006	24.4771072509613\\
7334.69999999997	24.4791950700769\\
7337.3	24.4773352958814\\
7338.40000000001	24.4750289568713\\
7338.59999999994	24.4770872225326\\
7339.30000000002	24.4747527045808\\
7339.79999999998	24.4771727135247\\
7340.90000000002	24.4748671843073\\
7344.3	24.4771526605305\\
7350.50000000002	24.4714717165946\\
7352.09999999999	24.4743070101466\\
7353.09999999997	24.4723385736822\\
7354.69999999999	24.4765323326263\\
7356.60000000002	24.4783666589639\\
7357.79999999996	24.4757335652836\\
7358.10000000005	24.4774537250958\\
7359.90000000001	24.4741847826087\\
7360.50000000004	24.4762655218325\\
7361.20000000006	24.4780133943733\\
7363.80000000003	24.4720868018305\\
7364.10000000002	24.4738057086988\\
7364.70000000004	24.4718118618292\\
7367.39999999998	24.469630132338\\
7368.70000000006	24.4720985777874\\
7371.70000000004	24.4702786293714\\
7373.49999999995	24.4724422263209\\
7376.80000000006	24.4682725806233\\
7377.70000000002	24.470709425574\\
7381.70000000004	24.4642228182828\\
7381.99999999995	24.4659378767559\\
7384.69999999999	24.4637634059149\\
7385.1	24.466500568705\\
7386.20000000001	24.4682723420386\\
7386.79999999995	24.4703461533255\\
7387.50000000005	24.4680275055498\\
7389.4	24.4644427904459\\
7390.49999999997	24.4662138392012\\
7391.59999999995	24.4625728858043\\
7391.9	24.4642857142857\\
7394.5	24.4624455683877\\
7396.49999999998	24.4598869750967\\
7396.79999999996	24.4615987778664\\
7398.10000000004	24.458652104566\\
7398.30000000003	24.4606942041523\\
7399.29999999997	24.4573884369003\\
7399.50000000006	24.4594302394724\\
7400.4	24.4564556448889\\
7400.7	24.4581666846827\\
7401.19999999996	24.4605677381001\\
7404.19999999995	24.4547087503208\\
7405.30000000003	24.4564777054582\\
7405.90000000005	24.4585471239536\\
7406.80000000001	24.4555752069017\\
7407.79999999998	24.453623834015\\
7408.80000000002	24.4516729878929\\
7412.20000000001	24.4499008404949\\
7412.90000000005	24.4516390125455\\
7413.4	24.4540365549336\\
7414.20000000006	24.451397974185\\
7414.50000000006	24.4531060340409\\
7415.49999999996	24.4511570203355\\
7417.50000000004	24.448608714409\\
7417.79999999994	24.4503161272058\\
7418.09999999995	24.4520234019034\\
7419.30000000004	24.4548076663881\\
7421.89999999996	24.4529776340609\\
7422.50000000005	24.4550427074071\\
7423.29999999997	24.4524072527413\\
7423.50000000004	24.4544425885015\\
7425.30000000003	24.4565949309128\\
7425.99999999996	24.4596760075948\\
7427.20000000005	24.4570705370727\\
7427.50000000001	24.4587753783187\\
7429.10000000006	24.4535077801109\\
7429.40000000002	24.456558314826\\
7430.00000000001	24.458621014522\\
7430.30000000003	24.4603251507321\\
7430.99999999999	24.4634038029363\\
7431.60000000004	24.4654655058735\\
7434.59999999998	24.4623185871656\\
7436.30000000004	24.4647947931795\\
7437.79999999998	24.462549913282\\
7439.79999999997	24.4586620787914\\
7440.6	24.4614082008413\\
7440.89999999997	24.4631097970703\\
7444.30000000005	24.4653699425071\\
7444.79999999996	24.4677564507247\\
7445.90000000002	24.4654848240666\\
7446.10000000005	24.4675136311139\\
7446.60000000006	24.4712422952449\\
7448.19999999996	24.4740410563484\\
7455.90000000003	24.4702253218884\\
7456.50000000004	24.4722795912346\\
7457.10000000006	24.4703105723328\\
7457.30000000002	24.4723362029662\\
7459.20000000001	24.4741463676216\\
7459.79999999995	24.4761994128608\\
7461.20000000002	24.4742873225845\\
7462.70000000002	24.476067963767\\
7464.19999999996	24.4711493375132\\
7466.50000000002	24.4689684729328\\
7467.09999999996	24.4710199271481\\
7469.40000000005	24.4634848383426\\
7474.10000000001	24.4601428915469\\
7474.60000000003	24.4625202349258\\
7477.70000000005	24.4604027922651\\
7481.40000000002	24.4563256031544\\
7481.60000000004	24.4583450285363\\
7483.00000000006	24.4564418489663\\
7483.20000000003	24.4584608394692\\
7483.9	24.4601817210048\\
7486.09999999998	24.4623440463787\\
7486.8	24.4640638982757\\
7489.40000000001	24.4622471460044\\
7492.89999999998	24.45749366075\\
7494.30000000003	24.459596498719\\
7497.09999999999	24.4571306621139\\
7497.30000000003	24.4591458372236\\
7498.19999999998	24.4615446167798\\
7499.70000000001	24.4646523907304\\
7500.50000000003	24.4620430365571\\
7502.69999999995	24.4588686890228\\
7506.09999999999	24.4557832192055\\
7506.3	24.457796014068\\
7509.09999999998	24.4526713897619\\
7510.20000000006	24.4544159354487\\
7511.69999999999	24.4561889294177\\
7512.50000000002	24.453584644464\\
7512.69999999996	24.4555957831967\\
7513.30000000002	24.4589666462587\\
7514.00000000001	24.4606805871628\\
7515.49999999995	24.4584597370802\\
7517.40000000005	24.4602593947456\\
7519.20000000002	24.4570638224303\\
7521.79999999997	24.455257315306\\
7521.99999999997	24.45726592308\\
7522.99999999994	24.4540149672343\\
7524.00000000001	24.4520939381454\\
7525.10000000002	24.4564928506884\\
7525.69999999997	24.4598580881767\\
7531.19999999995	24.4579289100155\\
7532.79999999998	24.4553890268024\\
7534.99999999999	24.4535573515945\\
7535.20000000006	24.4555624858997\\
7535.89999999997	24.4532908704883\\
7536.10000000002	24.4552957724052\\
7536.79999999995	24.457004869376\\
7537.29999999997	24.4593626449439\\
7538.09999999996	24.4567668674219\\
7538.79999999995	24.4584753743915\\
7540.89999999999	24.4556424877337\\
7542.10000000002	24.4583808437856\\
7543.10000000006	24.460441192067\\
7543.79999999999	24.4581715028036\\
7544.00000000004	24.460174175846\\
7545.09999999998	24.4579335206489\\
7546.10000000005	24.4560175982614\\
7546.29999999999	24.4580197180112\\
7547.39999999996	24.4544551175886\\
7550.89999999995	24.4510660839624\\
7551.90000000006	24.4491525423729\\
7553.79999999996	24.4443267716014\\
7553.99999999994	24.4463271600853\\
7554.59999999999	24.4443856142534\\
7557.49999999995	24.4469143643485\\
7558.10000000003	24.4449736709799\\
7561.9	24.4406241734991\\
7565.30000000002	24.4428582758347\\
7567.99999999999	24.4447087115657\\
7568.70000000003	24.4424479441919\\
7570.80000000005	24.4396306911992\\
7572.10000000004	24.436755500383\\
7572.50000000005	24.4394263529039\\
7574.49999999997	24.4435349721437\\
7576.7	24.4390771829796\\
7577.39999999998	24.4407786209172\\
7577.99999999999	24.4388435095868\\
7581	24.4344488266874\\
7582.80000000003	24.4365612100911\\
7584.29999999998	24.4383207636728\\
7585.00000000003	24.4360654440943\\
7587.39999999995	24.433607907743\\
7587.6	24.4355997206005\\
7589.10000000005	24.4320877035788\\
7589.89999999999	24.4347826086957\\
7591.00000000002	24.4365111775632\\
7593	24.4406105543191\\
7594.00000000006	24.4387089977746\\
7594.7	24.4404065939854\\
7595.29999999999	24.4384759196356\\
7596.70000000001	24.4366048862679\\
7597.69999999998	24.4386532943747\\
7598.40000000001	24.4364019214319\\
7598.70000000005	24.4393851660789\\
7600.39999999998	24.4418130386159\\
7601.49999999996	24.4435382024837\\
7602.70000000005	24.440995422739\\
7603.69999999998	24.4430416370762\\
7604.69999999996	24.4450873132758\\
7607.10000000006	24.4426332947734\\
7607.29999999999	24.4446197123853\\
7608.70000000002	24.4401219640416\\
7609.6	24.4451161018174\\
7610.40000000006	24.4478023782931\\
7611.8	24.4459333412157\\
7614.00000000002	24.4441234026346\\
7614.40000000001	24.4467791713179\\
7617.50000000006	24.4433942449065\\
7618.00000000004	24.4457279374122\\
7619.7	24.4428987637471\\
7619.89999999996	24.4448818897638\\
7620.69999999995	24.4423157673735\\
7622.7	24.4398383795981\\
7622.90000000006	24.4418208054572\\
7623.6	24.4395765835487\\
7624.89999999998	24.4432786885246\\
7626.00000000001	24.4397529536723\\
7629.50000000005	24.4416483170808\\
7630.80000000001	24.4387948996842\\
7632.30000000002	24.4366123368796\\
7634.39999999995	24.4338201584911\\
7634.90000000005	24.4361493123772\\
7635.9	24.4329491880566\\
7641.20000000002	24.4290892910892\\
7642.19999999998	24.4311267550345\\
7643.30000000005	24.4328440222937\\
7645.00000000003	24.4274110214386\\
7645.2	24.4293879900069\\
7646.09999999998	24.426512516021\\
7647.09999999994	24.429856679569\\
7647.6	24.4321822247212\\
7648.20000000005	24.4341879894879\\
7649.39999999995	24.436891300085\\
7651.20000000002	24.4324493876858\\
7651.70000000003	24.4347735173423\\
7653.09999999999	24.4329169497726\\
7654.09999999996	24.4349507459957\\
7655.50000000004	24.4304822613512\\
7655.79999999994	24.4334434880288\\
7656.40000000006	24.4315287664076\\
7658.30000000003	24.4293847278805\\
7658.50000000005	24.4313582116836\\
7659.50000000001	24.4333907775863\\
7660.59999999997	24.4364092054251\\
7662.6	24.4339462591515\\
7664.20000000001	24.4314549273906\\
7667.00000000001	24.4290540099907\\
7668.39999999996	24.4311142987546\\
7669.69999999998	24.4334923987588\\
7672.00000000005	24.4352915107989\\
7673.79999999997	24.4295599369291\\
7674.69999999998	24.4319070203784\\
7675.99999999998	24.4342830343534\\
7677.70000000005	24.4366875928\\
7680.50000000001	24.4394969143036\\
7681.29999999996	24.4369515973651\\
7683	24.4393539066262\\
7684.09999999999	24.4358553915827\\
7684.30000000003	24.4378220810994\\
7684.90000000005	24.4359141184125\\
7688.20000000004	24.4306283573742\\
7691.89999999999	24.4331773270931\\
7692.80000000003	24.4368183649859\\
7694.89999999998	24.4392462638077\\
7696.70000000004	24.441326265461\\
7697.49999999997	24.4387861151528\\
7700.50000000007	24.44095265304\\
7701.59999999998	24.4374618590701\\
7702.60000000005	24.440780505537\\
7704.00000000004	24.4428291429239\\
7706.19999999996	24.4449346638465\\
7709.29999999999	24.4467792564921\\
7709.89999999999	24.4448767833982\\
7710.19999999995	24.447816557073\\
7710.80000000004	24.4459142253174\\
7712.30000000002	24.4437529173798\\
7718.89999999997	24.4461717839098\\
7720.80000000003	24.4440414977529\\
7722.29999999996	24.4457681549777\\
7723.79999999995	24.4474941415606\\
7724.39999999999	24.4494789306751\\
7726.19999999995	24.451548606707\\
7726.80000000006	24.4496499242905\\
7727.1	24.4525830831349\\
7728.50000000006	24.4507414020651\\
7730.40000000001	24.4524933704159\\
7731.20000000006	24.4499631368593\\
7732.49999999999	24.4523187543646\\
7733.40000000004	24.449473071701\\
7733.60000000004	24.4514268720018\\
7734.59999999995	24.4547299830633\\
7736.40000000002	24.4490402636851\\
7738.69999999995	24.4508192484623\\
7739.80000000001	24.4473442809339\\
7740.09999999995	24.4502726027751\\
7740.69999999995	24.4522529971062\\
7742.10000000001	24.4504146108341\\
7745.30000000002	24.445477315568\\
7746.49999999996	24.4429814370175\\
7747.89999999994	24.4411461022199\\
7749.20000000006	24.4434981224111\\
7750.59999999997	24.4390829215426\\
7752.70000000005	24.4350428232381\\
7752.89999999994	24.4369921320779\\
7754.39999999996	24.4348442839642\\
7757.89999999999	24.4302655323537\\
7758.1	24.4322136578072\\
7759.59999999997	24.4339343015838\\
7762.20000000001	24.4270383778004\\
7764.49999999996	24.4288179687299\\
7765.10000000001	24.4320815948076\\
7766.90000000001	24.4354319557101\\
7768.69999999995	24.4336319637524\\
7769.20000000006	24.4359208680319\\
7771.00000000006	24.437981752905\\
7771.69999999995	24.4357806428369\\
7772.89999999996	24.4384407564647\\
7774.00000000001	24.4362691501267\\
7774.60000000003	24.4382419900446\\
7779	24.441130722063\\
7779.99999999998	24.4431305510212\\
7782.99999999999	24.4452724492811\\
7784.50000000005	24.4431313105362\\
7785	24.4454149593454\\
7786.49999999995	24.4471271158144\\
7788.49999999998	24.4498369411704\\
7789.1	24.4530888923124\\
7790.79999999998	24.4554544404369\\
7792.19999999995	24.4574772531858\\
7793.19999999996	24.4594715973978\\
7794.29999999995	24.4560197064559\\
7795.8	24.4590105055221\\
7797.70000000002	24.4607453384288\\
7801.20000000001	24.456180380193\\
7804.10000000002	24.4534993977602\\
7804.29999999998	24.455435395418\\
7805.69999999996	24.4587358118322\\
7806.50000000003	24.4562293444009\\
7806.7	24.4581646769483\\
7808.29999999994	24.4608370472824\\
7808.99999999998	24.4586444020438\\
7809.59999999995	24.4606066814346\\
7810.09999999995	24.4628818724232\\
7811.40000000001	24.4600908916341\\
7811.59999999998	24.4620249113509\\
7814.00000000004	24.4596306676393\\
7815.20000000004	24.4571545558072\\
7816.30000000005	24.4549920679597\\
7822.00000000005	24.4525127523299\\
7822.60000000001	24.4557505720531\\
7824.30000000004	24.4581054138336\\
7827.4	24.462472053657\\
7828.90000000001	24.4641716694342\\
7830.50000000002	24.460450029372\\
7831.39999999996	24.4627466002681\\
7833.80000000003	24.4603581868546\\
7835.19999999998	24.457263920973\\
7837.99999999999	24.4549061634835\\
7841.50000000005	24.4516425219343\\
7842.09999999998	24.453597204866\\
7844.80000000004	24.4553786536476\\
7845.40000000001	24.4535083806003\\
7847.10000000002	24.4558568661433\\
7847.89999999995	24.4597349643221\\
7849.00000000004	24.4575811239505\\
7850.20000000003	24.4551163649797\\
7851.80000000001	24.4526802429985\\
7853.99999999998	24.4496505010122\\
7855.60000000001	24.4523085148364\\
7856.59999999996	24.4491962274237\\
7857.10000000004	24.451458534847\\
7857.59999999995	24.4537205543607\\
7857.99999999997	24.4562935060638\\
7859.80000000002	24.4583264418122\\
7861.69999999998	24.4600473174082\\
7863.70000000003	24.4563696940411\\
7867.49999999999	24.4585388174284\\
7869.79999999997	24.4552027344693\\
7870.80000000002	24.457177705218\\
7871.40000000001	24.4591246903386\\
7873.00000000001	24.4566943135487\\
7875.70000000005	24.4546585743671\\
7876.40000000003	24.457563638672\\
7878.10000000002	24.4599020080729\\
7879.60000000006	24.4628602611775\\
7880.50000000007	24.4651422480522\\
7881.59999999999	24.4617277998401\\
7881.79999999996	24.4636445527094\\
7882.30000000003	24.4658987110525\\
7882.99999999998	24.4688003450419\\
7883.79999999996	24.466317431728\\
7886.10000000004	24.470594202531\\
7887.90000000001	24.4688133874239\\
7889.20000000002	24.4711191106942\\
7890.39999999996	24.4686648501362\\
7890.90000000004	24.4709162336839\\
7891.70000000001	24.4684355913733\\
7892.19999999994	24.4719536763681\\
7892.99999999998	24.469473337472\\
7895.29999999997	24.4712110849355\\
7897.20000000004	24.4691223582743\\
7897.79999999996	24.4710619278542\\
7899.00000000001	24.4686103480143\\
7900.80000000001	24.4668328924553\\
7901.40000000003	24.4700373346833\\
7902.89999999999	24.4653928887764\\
7906.80000000003	24.4609138853407\\
7907.20000000002	24.4647351181819\\
7907.99999999995	24.4622602142108\\
7908.90000000003	24.4645340751043\\
7911.50000000006	24.4615501289246\\
7912.49999999997	24.4635138892399\\
7914.29999999998	24.4655311836652\\
7916.79999999994	24.459068574821\\
7917.10000000007	24.461930985702\\
7918.10000000004	24.4588416559319\\
7919.70000000005	24.4564256673148\\
7920.19999999999	24.4586694948424\\
7923.09999999999	24.453503634895\\
7926.20000000003	24.4477246634621\\
7926.39999999999	24.4496309846717\\
7931.40000000002	24.4468259471727\\
7931.60000000005	24.4487310412648\\
7933.79999999994	24.4507745245088\\
7935.20000000004	24.4489811349287\\
7935.60000000002	24.4515291656691\\
7936.70000000002	24.4494002620704\\
7938.09999999996	24.4476077700234\\
7938.9	24.4514422471344\\
7939.60000000006	24.4543244706979\\
7940.39999999997	24.4518607140608\\
7942.60000000001	24.4501240132449\\
7942.8	24.4520263379874\\
7943.59999999996	24.4495638052797\\
7943.89999999997	24.452416918429\\
7945.70000000005	24.4506531752624\\
7947.59999999999	24.4485826088051\\
7953.00000000002	24.4458135821252\\
7953.69999999994	24.4486911916317\\
7955.00000000004	24.4459529107114\\
7957.10000000001	24.4432715025386\\
7957.70000000006	24.4451984216743\\
7960	24.4469290586802\\
7961.40000000005	24.445142247064\\
7961.59999999995	24.4470402049813\\
7962.19999999994	24.4451979955541\\
7964.70000000001	24.4400361591\\
7965.30000000005	24.4419614834158\\
7967.20000000001	24.4386429530707\\
7967.79999999997	24.4405677782101\\
7969.09999999997	24.4378356673192\\
7969.39999999995	24.440680092854\\
7970.80000000006	24.4388964859677\\
7971.2	24.441433643195\\
7972.69999999997	24.4443608268112\\
7974.59999999995	24.4473146325254\\
7976.50000000003	24.44901336409\\
7978.50000000007	24.445391422054\\
7978.89999999996	24.4491790951247\\
7979.90000000006	24.4461152882206\\
7980.49999999996	24.4480364884846\\
7983.89999999998	24.4451402805611\\
7985.20000000006	24.4424129337658\\
7985.49999999995	24.4452514526147\\
7987.70000000001	24.443526377726\\
7989.49999999998	24.4480324421748\\
7990.89999999997	24.4500062570392\\
7991.80000000004	24.452257911135\\
7993.20000000006	24.4492262269651\\
7994.50000000003	24.4465013884372\\
7997.19999999999	24.4432495967389\\
7999.40000000001	24.4415275954747\\
7999.59999999999	24.4434166281236\\
8000.69999999995	24.4475552444756\\
8001.19999999998	24.4497769112519\\
8002.20000000007	24.4517201304625\\
8003.00000000004	24.4492759055866\\
8003.30000000001	24.452107854162\\
8004.59999999997	24.4543830499582\\
8005.20000000004	24.4562977027719\\
8007.50000000001	24.4505220040961\\
8008.50000000003	24.4537122593212\\
8011.90000000004	24.4558162755866\\
8013.70000000006	24.4540667348823\\
8013.89999999999	24.4559520838533\\
8014.49999999999	24.4578643974746\\
8017.50000000006	24.4599381361006\\
8018.29999999997	24.4574977551631\\
8019.70000000004	24.459462829497\\
8020.40000000004	24.4573280967521\\
8021.70000000002	24.4608441995562\\
8022.29999999999	24.4590147586757\\
8023.10000000006	24.461561471732\\
8026.79999999996	24.4652356451432\\
8028.30000000001	24.4631557969209\\
8032.99999999995	24.4600465573689\\
8033.60000000002	24.4619540186962\\
8035.29999999995	24.4592677402494\\
8035.9	24.461174713788\\
8036.39999999996	24.4633858022771\\
8038.20000000004	24.4616398989836\\
8038.9	24.4644856325414\\
8039.80000000002	24.4617470366547\\
8042.10000000002	24.4634552734326\\
8043.1	24.4616570519196\\
8043.89999999994	24.4654400795624\\
8044.60000000001	24.4633112483996\\
8045.19999999999	24.4664586777373\\
8046.20000000005	24.4646607757603\\
8047.69999999997	24.462586048361\\
8050.39999999995	24.460592509782\\
8054.3	24.4549562971792\\
8055.29999999995	24.4568860640068\\
8055.89999999999	24.4550645481629\\
8056.79999999997	24.4585386438953\\
8057.39999999994	24.4616816630469\\
8059.69999999994	24.4633861882429\\
8061.7	24.4660001488501\\
8062.70000000007	24.462965719105\\
8064.40000000005	24.4652489304979\\
8065.10000000004	24.4631255269553\\
8066.79999999999	24.4654080253877\\
8067.89999999997	24.4633118492811\\
8069.69999999998	24.4652903417681\\
8072.39999999996	24.467017652524\\
8073.30000000004	24.464290137984\\
8073.50000000002	24.4661613158938\\
8076.59999999995	24.4641994874144\\
8078.39999999999	24.4661756514204\\
8079.19999999999	24.4637530479126\\
8080.09999999998	24.4672161580159\\
8081.30000000003	24.4635830425421\\
8081.50000000005	24.4654523856662\\
8082.99999999999	24.4633865719835\\
8083.19999999999	24.4652555268269\\
8084.10000000005	24.4674797753643\\
8085.29999999996	24.4700324040864\\
8086.19999999995	24.4722555433264\\
8087.50000000002	24.4695583362184\\
8087.69999999998	24.4714260985682\\
8089.20000000005	24.4693607605108\\
8089.70000000002	24.4715567751984\\
8091.60000000001	24.4744614852256\\
8092.20000000002	24.4763540649753\\
8093.10000000001	24.4736321850442\\
8094.90000000006	24.4706609017912\\
8097.09999999998	24.4677172355876\\
8097.49999999997	24.4702133965619\\
8100.00000000001	24.4750064814015\\
8103.60000000006	24.4727716968792\\
8110.29999999997	24.4747484710988\\
8111.39999999994	24.4714294520126\\
8113.70000000007	24.4743523379921\\
8117.19999999999	24.471191159622\\
8117.80000000001	24.4743098584609\\
8120.10000000007	24.4784611216472\\
8120.80000000004	24.4825080963933\\
8122.90000000003	24.4786408962206\\
8124.29999999999	24.4805770272266\\
8125.50000000001	24.4831150930393\\
8126.60000000002	24.479801149298\\
8129.80000000007	24.4775458492724\\
8130.09999999998	24.4803325871442\\
8131.40000000006	24.4764188649081\\
8131.6	24.4782763751737\\
8133.20000000001	24.4808380362215\\
8134.29999999999	24.478756884343\\
8134.50000000007	24.4806136749195\\
8135.90000000004	24.482546705998\\
8137.30000000001	24.4844790719394\\
8138.30000000005	24.4826993020741\\
8140.10000000002	24.4858848676937\\
8141.40000000006	24.4881164404594\\
8145.49999999996	24.4843842074249\\
8148.09999999997	24.4814805723964\\
8150.09999999996	24.4791538857942\\
8150.69999999997	24.4810325366835\\
8152.79999999999	24.4845392437047\\
8153.79999999998	24.4827628496793\\
8155.80000000007	24.485341899729\\
8156.3	24.4875190034819\\
8158.39999999995	24.4897959183673\\
8160.90000000005	24.4920965567945\\
8161.90000000002	24.490320999755\\
8164.49999999999	24.4923205055973\\
8167.09999999997	24.4894210990303\\
8167.40000000006	24.4921946740129\\
8170.49999999997	24.493917215382\\
8171.40000000002	24.4912194823472\\
8172.7	24.4885473766641\\
8173.89999999996	24.4910692439442\\
8177.50000000007	24.4864018783017\\
8178.09999999995	24.4907192291702\\
8181.39999999999	24.488174540121\\
8184.30000000005	24.4904941107473\\
8187.29999999997	24.4864059408359\\
8187.49999999995	24.4882505251844\\
8189.90000000003	24.4847374847375\\
8190.39999999995	24.486905561321\\
8192.20000000002	24.4888492853045\\
8193.20000000001	24.4858603981302\\
8194.19999999997	24.4840926009543\\
8194.90000000003	24.4868822452715\\
8195.50000000001	24.4887500610083\\
8196.10000000005	24.4869573704888\\
8197.80000000005	24.4916381024409\\
8199.60000000001	24.4887008061271\\
8201.00000000004	24.4857397178427\\
8202.40000000005	24.4839987808595\\
8202.79999999994	24.4864621048654\\
8204.20000000001	24.484721426569\\
8204.70000000004	24.486885725429\\
8205.29999999997	24.4850951812221\\
8206.19999999996	24.4872841597309\\
8207.2	24.4855189891925\\
8208.90000000003	24.4877573395054\\
8210.60000000006	24.4899947629313\\
8213.29999999998	24.4855967078189\\
8213.49999999999	24.4874354728743\\
8214.30000000002	24.4899201402415\\
8214.90000000003	24.4917833231893\\
8216.60000000006	24.4879331118332\\
8218.99999999996	24.4905159932353\\
8219.60000000002	24.4923780673261\\
8220.49999999996	24.4896966158188\\
8221.00000000005	24.4918563209303\\
8221.70000000001	24.4946362110487\\
8222.40000000005	24.4925509273335\\
8223.29999999999	24.4947345380256\\
8225.00000000001	24.4908876487824\\
8226.50000000001	24.488853232198\\
8226.7	24.4906889677639\\
8227.80000000001	24.4934916564372\\
8228.90000000005	24.4902175233929\\
8231.59999999994	24.4882588043782\\
8233.00000000001	24.4901677375472\\
8233.79999999998	24.4877882898748\\
8234.99999999994	24.4854342995228\\
8238.5	24.4811011579637\\
8242.70000000002	24.4831853253749\\
8246.00000000004	24.4806635864227\\
8246.2	24.4824951796563\\
8252.70000000002	24.4801764249709\\
8254.00000000002	24.4835899734677\\
8255.79999999996	24.4855194466987\\
8256.99999999995	24.4831720579865\\
8258.50000000002	24.481146925629\\
8259.99999999995	24.4791225287829\\
8260.49999999994	24.4812725467884\\
8264.00000000001	24.4830048039109\\
8264.59999999999	24.4812273887739\\
8267.09999999998	24.477453067544\\
8268.5	24.4793556345693\\
8269.20000000006	24.4772834459991\\
8269.70000000006	24.4794311833418\\
8272.40000000006	24.4750679963735\\
8272.60000000001	24.4768938798699\\
8274.40000000001	24.4788204725361\\
8275.39999999997	24.4806960304513\\
8276.90000000004	24.4762595143168\\
8278.49999999999	24.4787766047399\\
8279.30000000006	24.4764113341546\\
8279.49999999999	24.4782356635586\\
8285.4	24.4728742984732\\
8287.3	24.4708835099066\\
8289.40000000002	24.4683032752277\\
8289.69999999998	24.4710366956983\\
8290.60000000002	24.4683802332734\\
8292.80000000004	24.4715358921487\\
8294.2	24.4686109738013\\
8297.6	24.4658158284826\\
8300.79999999997	24.4636123793805\\
8301	24.4654322921059\\
8302.90000000005	24.4682644827171\\
8303.89999999996	24.4665221579961\\
8304.10000000001	24.4683413212591\\
8306.30000000003	24.4654724068188\\
8306.59999999999	24.4682003683773\\
8308.00000000005	24.4700954490196\\
8309.20000000005	24.4725789175984\\
8312.70000000004	24.4743046867481\\
8313.60000000001	24.4716552196976\\
8315.1	24.4696459495863\\
8315.29999999998	24.4714625874883\\
8315.80000000003	24.4748012842867\\
8316.50000000006	24.472741264459\\
8316.89999999995	24.4751713358182\\
8317.59999999997	24.4779205790062\\
8318.59999999996	24.474978061476\\
8319.99999999994	24.4732635425055\\
8323.40000000001	24.4668709076711\\
8327.10000000005	24.4644058026708\\
8328.29999999996	24.462081552279\\
8329.19999999996	24.4642406924952\\
8330.09999999997	24.4663993661617\\
8332.20000000002	24.4638335153559\\
8332.39999999997	24.4656465646565\\
8334.50000000005	24.4618817939673\\
8338.2	24.4594221843781\\
8339.29999999999	24.457395016428\\
8340.09999999998	24.4598450876478\\
8341.60000000002	24.4578443242984\\
8343.30000000001	24.4552580482777\\
8343.90000000003	24.4570949185043\\
8345.70000000005	24.4590093220542\\
8346.59999999996	24.4611642924749\\
8348.79999999997	24.4571141108409\\
8349.90000000003	24.4550898203593\\
8352.10000000001	24.4570292856972\\
8353.79999999998	24.459234608985\\
8354.40000000003	24.4574780058651\\
8357.79999999998	24.4535110494263\\
8359.60000000001	24.450638180796\\
8361.00000000005	24.4489361447656\\
8365.69999999997	24.4507399172823\\
8366.30000000001	24.452572193536\\
8367.30000000002	24.4544302889787\\
8370.20000000006	24.4519312330502\\
8372.90000000006	24.4500179147259\\
8374.29999999998	24.453095147115\\
8376.89999999995	24.4490867852453\\
8377.20000000007	24.4517923435952\\
8378.60000000006	24.4548677002399\\
8379.79999999999	24.4525590997506\\
8380.79999999995	24.4544142037251\\
8382.20000000007	24.4527158417141\\
8384.90000000005	24.4555754323196\\
8385.80000000007	24.457720697838\\
8386.99999999997	24.4554136709947\\
8388.69999999995	24.4528418844173\\
8390.20000000007	24.4508539623136\\
8390.60000000005	24.4532637324657\\
8391.30000000005	24.4512238720595\\
8391.70000000001	24.4536333087061\\
8393.39999999999	24.455828915232\\
8397.79999999997	24.4525417068553\\
8398.19999999996	24.4549492159127\\
8400.19999999994	24.4574598526243\\
8401.10000000004	24.4596010093796\\
8401.69999999994	24.4578542693232\\
8402.69999999996	24.4561336697291\\
8405.09999999995	24.4539094846048\\
8408.70000000001	24.4517648178099\\
8410.29999999998	24.4494911062494\\
8410.59999999995	24.4521859060482\\
8411.59999999997	24.454034261802\\
8417.99999999994	24.4520735082738\\
8418.50000000002	24.4541847813176\\
8419.70000000006	24.4566379248913\\
8420.30000000007	24.4548952543822\\
8423.79999999996	24.456605610228\\
8425.29999999997	24.454625299689\\
8428.89999999997	24.4524854668407\\
8430.19999999997	24.4546457421444\\
8431.39999999999	24.4511652730831\\
8431.59999999996	24.4529572921238\\
8432.20000000006	24.4512173428365\\
8432.49999999998	24.4539050826554\\
8434.8	24.4567214786186\\
8436.8	24.4544797259657\\
8438.50000000004	24.4566634275828\\
8441.99999999995	24.4583693630732\\
8444.19999999995	24.455549897564\\
8446.40000000003	24.4574675901261\\
8450.50000000007	24.4598016708873\\
8452.00000000002	24.4578270488991\\
8457.09999999998	24.4560847561841\\
8457.29999999997	24.4578712133753\\
8460.00000000002	24.4559756976868\\
8460.40000000001	24.4595473080787\\
8462.10000000006	24.4617239015859\\
8462.60000000006	24.4650052583691\\
8463.30000000001	24.462981780372\\
8463.50000000001	24.4647667659152\\
8464.10000000003	24.4630325370384\\
8465.89999999996	24.4649184975195\\
8466.90000000005	24.467934333294\\
8468.39999999994	24.4718663281573\\
8469.19999999994	24.4695547447841\\
8471.69999999995	24.4717769541302\\
8472.5	24.4694662795364\\
8475.19999999999	24.4734699656649\\
8476.90000000001	24.4756399669695\\
8480.19999999994	24.4779076211926\\
8481.7	24.4747577165224\\
8483.00000000004	24.4780799471891\\
8483.59999999994	24.4763487629218\\
8484.60000000007	24.4781783681214\\
8487.70000000001	24.4810198166781\\
8488.20000000006	24.4831120483489\\
8490.59999999997	24.4785471162566\\
8492.49999999998	24.4742481689942\\
8493.59999999999	24.4769652801488\\
8494.90000000006	24.4743967039435\\
8496.9	24.4721666470519\\
8497.20000000006	24.474833182305\\
8500.09999999998	24.4770711277382\\
8500.59999999997	24.4791605397203\\
8503.49999999999	24.4755162519404\\
8504.40000000003	24.4776294902699\\
8505.00000000002	24.4806057541945\\
8505.60000000003	24.4824059160328\\
8508.69999999998	24.4781872884543\\
8510.10000000002	24.4753354797772\\
8510.30000000001	24.4771103590901\\
8511.99999999994	24.4792706852598\\
8512.7	24.4772577765248\\
8513.09999999999	24.4796316308791\\
8514.49999999998	24.4814788715853\\
8515.30000000007	24.4791788994058\\
8516.79999999997	24.4772158884101\\
8517.49999999996	24.4799004414389\\
8518.39999999996	24.4820097434994\\
8519.39999999999	24.4850049885557\\
8520.59999999999	24.4874247420987\\
8521.30000000001	24.4854131950149\\
8521.60000000004	24.4880716289003\\
8522.70000000001	24.4849110620923\\
8522.90000000005	24.4866830928077\\
8527.50000000004	24.4887189830667\\
8528.29999999998	24.4911120491534\\
8529.99999999998	24.4885757494051\\
8530.20000000003	24.4903461777429\\
8530.90000000005	24.488336654554\\
8534.19999999995	24.4905850509122\\
8536.90000000001	24.4875248916481\\
8541.60000000005	24.48458737722\\
8542.60000000003	24.4864035960528\\
8547.30000000006	24.4881484428013\\
8550.90000000001	24.4848555724477\\
8553.49999999994	24.482089412645\\
8557.40000000004	24.4802804557406\\
8559.70000000004	24.4783756629828\\
8562.99999999995	24.4806203360932\\
8564.29999999994	24.4839101396478\\
8567.40000000006	24.4808870732419\\
8569.19999999996	24.4827465487263\\
8571.10000000001	24.4808194885197\\
8571.60000000005	24.4828913751065\\
8572.30000000001	24.4808921655546\\
8574.39999999999	24.4830602367485\\
8575.1	24.4810616662002\\
8575.70000000006	24.4828470813219\\
8578.00000000003	24.4856087012275\\
8579.69999999998	24.4877502972097\\
8580.20000000006	24.4898197032738\\
8580.79999999997	24.4916034448601\\
8581.29999999997	24.4936723611532\\
8582.60000000005	24.4957880387291\\
8583.20000000006	24.4975708643529\\
8584.99999999994	24.4994234196457\\
8586.99999999997	24.4972109326781\\
8587.99999999997	24.4990160803903\\
8590.19999999998	24.5020546430276\\
8595.49999999999	24.5043975987715\\
8596.39999999995	24.5018321409876\\
8598.40000000002	24.5054369948247\\
8601.40000000005	24.5015404289949\\
8603.50000000005	24.5036961272026\\
8605.09999999995	24.4991400548506\\
8608.00000000007	24.4966949733391\\
8610.40000000004	24.4945125137913\\
8610.80000000002	24.4980199514569\\
8611.39999999993	24.4963130697323\\
8611.60000000007	24.4980665838336\\
8613.80000000002	24.4999361497115\\
8614.99999999995	24.4976842984992\\
8617.39999999995	24.5001450536699\\
8619.70000000006	24.4959279797675\\
8623.80000000006	24.4935585987778\\
8624.30000000006	24.4967765873568\\
8625.10000000007	24.4945044752585\\
8626.39999999999	24.4966092853417\\
8628.60000000005	24.4938403235713\\
8628.90000000007	24.4964654073473\\
8632.29999999996	24.4984013715769\\
8633.30000000006	24.5001969096764\\
8638.30000000004	24.5022226338211\\
8642.1	24.5041771771077\\
8642.70000000003	24.5024760494284\\
8645.50000000006	24.5003238641621\\
8645.69999999999	24.5020703694279\\
8646.7	24.5038627006523\\
8648.40000000005	24.5059836965948\\
8650.29999999996	24.5029131600851\\
8650.79999999999	24.5049648013501\\
8651.90000000003	24.5030050855294\\
8652.29999999997	24.5053395589663\\
8654.10000000007	24.5071757065933\\
8655.1	24.5089657084758\\
8656.59999999994	24.5070292374692\\
8657.99999999994	24.5088414317229\\
8660.39999999999	24.5066682062237\\
8661.50000000007	24.504710446107\\
8661.80000000007	24.5073251826966\\
8663.89999999997	24.5036934441367\\
8665.89999999997	24.5015001153935\\
8667.39999999995	24.4995673492933\\
8670.49999999998	24.4977279542362\\
8672.90000000005	24.4955609362389\\
8676.49999999994	24.4980752829449\\
8676.99999999994	24.5001210081709\\
8678.90000000006	24.5028229058647\\
8689.29999999996	24.5045687849564\\
8691.00000000004	24.5066792465856\\
8695.79999999999	24.5046516174289\\
8696.30000000007	24.5066924244515\\
8697.09999999996	24.5090373913443\\
8698.20000000004	24.5128358414863\\
8699.2	24.5100180474291\\
8700.49999999998	24.5075052295244\\
8701.39999999996	24.50956731598\\
8703.59999999995	24.5114146857084\\
8704.59999999999	24.5131940216205\\
8705.09999999997	24.5152322749621\\
8705.69999999995	24.5169886742172\\
8709.19999999998	24.5151734352933\\
8709.69999999996	24.5172104985189\\
8714.30000000006	24.5191866336179\\
8717.29999999993	24.5164842728336\\
8717.90000000002	24.518238128011\\
8718.29999999996	24.520554230134\\
8719.80000000003	24.5174829986582\\
8719.99999999997	24.5192142291946\\
8721.50000000005	24.5218767198679\\
8723.69999999996	24.5237167289484\\
8724.70000000006	24.5254905556574\\
8725.5	24.523241954708\\
8728.59999999998	24.5259889788857\\
8730.49999999997	24.5240876915676\\
8731.79999999995	24.5261626908233\\
8733.20000000007	24.52337604342\\
8734.09999999996	24.5254287742438\\
8734.6	24.5274594433696\\
8735.99999999994	24.52352880576\\
8736.20000000001	24.5252566876138\\
8740.59999999994	24.5277838159415\\
8741.90000000001	24.5252802562343\\
8743.19999999997	24.522777441012\\
8746.60000000005	24.5201047252107\\
8747.10000000002	24.5232760197549\\
8747.89999999997	24.5210333790581\\
8748.40000000007	24.5230610961879\\
8751.80000000006	24.5192472491688\\
8754.20000000001	24.5216636395828\\
8755.1	24.5191429093567\\
8755.29999999999	24.5208671220047\\
8756.40000000001	24.5189287957517\\
8760.39999999996	24.5145825009988\\
8760.60000000007	24.5163057746527\\
8761.79999999998	24.5186546297036\\
8763.20000000002	24.520443212032\\
8765.49999999996	24.5174317787716\\
8770.30000000006	24.5142752896105\\
8771.29999999994	24.5171808377226\\
8773.90000000002	24.5190335080921\\
8774.49999999997	24.5207758758234\\
8778.70000000005	24.5227138105436\\
8780.80000000002	24.5248209181291\\
8782.60000000005	24.5220718002437\\
8785.09999999996	24.5241997905568\\
8787.4	24.5189189189189\\
8788.49999999999	24.5169879161641\\
8789.80000000007	24.5190502736095\\
8790.90000000002	24.5171197815948\\
8791.09999999995	24.518837018837\\
8792.10000000004	24.5205978025978\\
8795.69999999998	24.5230678278269\\
8796.50000000007	24.5253848077666\\
8797.60000000002	24.5280016367914\\
8798.59999999999	24.5308966097264\\
8804.29999999997	24.5331879514788\\
8805.79999999994	24.5290089599019\\
8806.00000000004	24.5307230215419\\
8807.50000000003	24.5276806394478\\
8808.70000000002	24.5254745254745\\
8809.19999999996	24.527487995641\\
8810.20000000008	24.52470403959\\
8810.40000000002	24.5264173429431\\
8811.90000000007	24.5290512936904\\
8814.50000000001	24.5320264107277\\
8815.09999999996	24.5348942735275\\
8820.60000000002	24.5332003129003\\
8821.90000000001	24.5307186579007\\
8822.80000000006	24.5327500028335\\
8825.00000000001	24.5345661805532\\
8825.90000000001	24.5320643553138\\
8826.5	24.5337955724741\\
8827.29999999996	24.5315721503501\\
8827.49999999997	24.5332819792469\\
8831.39999999998	24.5303742286135\\
8831.59999999999	24.5320832908727\\
8832.99999999999	24.5338556112803\\
8835.90000000001	24.5303304662743\\
8838.30000000005	24.5281951484432\\
8841.39999999998	24.5263812701465\\
8842.29999999995	24.528408576857\\
8843.70000000004	24.5256563920487\\
8844.70000000004	24.5274059334298\\
8845.89999999999	24.5252091340719\\
8846.09999999995	24.5269155117452\\
8851.49999999997	24.5243797731484\\
8851.70000000002	24.5260850900382\\
8854.19999999998	24.5281953401172\\
8854.90000000001	24.5262563523433\\
8855.39999999999	24.5282592738976\\
8856.20000000006	24.5316893059178\\
8863.20000000005	24.5337515372378\\
8864.59999999997	24.5355172763884\\
8865.69999999994	24.533601028672\\
8867.69999999998	24.5314508671824\\
8868.30000000006	24.5331739659916\\
8869.5	24.5354920176784\\
8870.10000000005	24.5383418637686\\
8872.69999999996	24.5356595437742\\
8873.29999999996	24.537381387067\\
8873.69999999995	24.5396560661723\\
8876	24.5378037651671\\
8876.3	24.5403541976477\\
8878.39999999996	24.5424339697021\\
8878.89999999997	24.5444306791305\\
8880.30000000002	24.5416873113824\\
8880.49999999998	24.5433867081053\\
8882.50000000004	24.5401121293315\\
8883.00000000002	24.5421080478662\\
8890.00000000005	24.5374067783265\\
8890.20000000002	24.5391044171738\\
8892.39999999999	24.5420298003936\\
8895.09999999996	24.5368288515154\\
8895.29999999998	24.5385255300492\\
8897.10000000003	24.5358090185676\\
8897.59999999997	24.5378019038628\\
8900.10000000001	24.5342801285364\\
8900.90000000006	24.537692394113\\
8902.79999999998	24.5335789461861\\
8905.69999999994	24.527835792405\\
8906.39999999994	24.5303991466906\\
8906.89999999997	24.5323902548557\\
8907.90000000006	24.5341266277503\\
8910.50000000003	24.5370682108949\\
8912.29999999997	24.5388447556214\\
8914.40000000004	24.5364294127545\\
8917.39999999999	24.533781889543\\
8919.19999999998	24.5355577231397\\
8920.40000000007	24.5333781738692\\
8922.59999999995	24.5351743306398\\
8925.10000000007	24.5283018867925\\
8926.40000000002	24.5303310368005\\
8927.80000000007	24.5276044758566\\
8929.10000000007	24.5296331138288\\
8932.99999999994	24.5278794595381\\
8935.10000000007	24.5254722893724\\
8936.69999999998	24.5222003401665\\
8938.89999999999	24.5239959727039\\
8943.3	24.5219938725764\\
8946.00000000007	24.5201819787393\\
8946.39999999996	24.5224389426033\\
8948.00000000004	24.5202892234106\\
8951.80000000001	24.5165830717501\\
8955.09999999997	24.5131320350188\\
8956.40000000006	24.5151565901859\\
8958.90000000005	24.512780444246\\
8959.89999999998	24.5145089285714\\
8962.60000000001	24.5171655862631\\
8965.20000000007	24.5145170825293\\
8965.59999999994	24.5167694658532\\
8966.60000000007	24.5184962137687\\
8967.19999999998	24.5202011753817\\
8967.90000000003	24.5238626226583\\
8969.69999999994	24.5256304488394\\
8972.89999999999	24.5235707121364\\
8973.19999999999	24.5260940791013\\
8975.19999999999	24.5239713435763\\
8975.69999999998	24.5259475478509\\
8978.2	24.5202321152111\\
8980.39999999996	24.5164523133456\\
8981.59999999998	24.5142901677856\\
8982.19999999996	24.5182191643566\\
8985.09999999997	24.520322307795\\
8987.19999999995	24.5223815829003\\
8988.00000000005	24.5201989296959\\
8989.69999999997	24.5177868250684\\
8992.10000000002	24.5145793020618\\
8992.80000000001	24.5171190605923\\
8995.20000000003	24.5105777461563\\
8996.49999999993	24.5125936464887\\
8999.29999999993	24.514967664511\\
8999.89999999998	24.5166666666667\\
9005.00000000004	24.5138865753851\\
9006.00000000001	24.516716447741\\
9007.00000000005	24.5184354564732\\
9008.20000000001	24.5162794311912\\
9009.40000000002	24.5185637382763\\
9011.19999999995	24.5158856102893\\
9011.69999999994	24.5189640249451\\
9013.50000000001	24.5162865003994\\
9018.90000000001	24.5182392726466\\
9020.20000000008	24.5158143299003\\
9021.30000000006	24.5205843882324\\
9023.39999999993	24.5182024713249\\
9024.90000000001	24.5163434903047\\
9030.30000000002	24.5182937632884\\
9032.49999999998	24.5200717401413\\
9033.30000000001	24.5223282485\\
9034.09999999995	24.52015673773\\
9035.19999999993	24.5227053888637\\
9038.60000000002	24.524544458827\\
9040.70000000002	24.5210600831785\\
9046.00000000006	24.523275223577\\
9047.40000000005	24.5250069079856\\
9049.70000000001	24.5231938827377\\
9051.30000000002	24.5210685639791\\
9051.70000000003	24.5232992333017\\
9059.90000000007	24.5253863134658\\
9060.79999999994	24.5273648313081\\
9061.50000000006	24.5254701156529\\
9062.90000000005	24.5271984993931\\
9065.29999999994	24.5240143843625\\
9066.89999999994	24.5218925774788\\
9068.49999999998	24.5186688132678\\
9072.89999999996	24.5166978948529\\
9074.49999999994	24.5189870627907\\
9075.40000000007	24.5209630323398\\
9075.89999999997	24.5229175848391\\
9078.00000000002	24.5172447979203\\
9079.29999999994	24.5192413595612\\
9080.80000000006	24.5217985001487\\
9086.20000000006	24.5193313009696\\
9087.20000000001	24.5210348508358\\
9088.19999999999	24.5183367626509\\
9089.00000000006	24.5205795953395\\
9089.70000000001	24.5186912803362\\
9091.59999999995	24.5168670325682\\
9092.30000000001	24.5204786415028\\
9093.69999999998	24.5222019397832\\
9094.6	24.5197752537192\\
9095.00000000008	24.521995360139\\
9099.40000000002	24.5189296115171\\
9100.10000000007	24.5214390892508\\
9102.30000000006	24.5188082264018\\
9102.69999999996	24.5210264973415\\
9104.90000000004	24.5227896760022\\
9105.80000000007	24.5203659166035\\
9107.70000000007	24.5185445442368\\
9108.59999999998	24.5216112068682\\
9110.79999999999	24.5233731025475\\
9112.99999999997	24.5207448617924\\
9115.19999999999	24.5181178897019\\
9116.29999999997	24.5162564170067\\
9116.79999999994	24.5192993232349\\
9117.89999999995	24.5174380346567\\
9118.29999999996	24.5196525706264\\
9119.89999999997	24.5230263157895\\
9121.00000000002	24.5211652103365\\
9122.30000000002	24.5231518021573\\
9126.09999999993	24.5249939733953\\
9127.69999999997	24.5228861280922\\
9131.40000000006	24.5249958933363\\
9133.50000000003	24.5270211088727\\
9134.2	24.5251414996223\\
9136.29999999997	24.5271660610306\\
9139	24.5199199045858\\
9140.09999999997	24.5224393339314\\
9142.49999999999	24.5203771356069\\
9144.00000000006	24.5185420106954\\
9144.59999999996	24.5213074239724\\
9148.30000000004	24.5179484937257\\
9150.00000000008	24.5199506016328\\
9150.99999999995	24.5172711477309\\
9152.29999999999	24.5192517809536\\
9153.70000000005	24.5165942013153\\
9158.20000000003	24.5132830328773\\
9161.10000000005	24.5153473344103\\
9165.69999999997	24.5117720220821\\
9168.10000000003	24.5086276477389\\
9168.40000000004	24.5110977804439\\
9171.59999999995	24.5134489789243\\
9172.00000000004	24.5156507233894\\
9173.69999999998	24.5121977806362\\
9176.29999999994	24.5063423564797\\
9177.50000000001	24.5042276847978\\
9179.4	24.5002451113895\\
9180.69999999997	24.5033112582781\\
9181.60000000001	24.500909417646\\
9182.00000000007	24.5031093105063\\
9182.49999999994	24.5050421449263\\
9184.80000000003	24.503260786726\\
9186.9	24.5009252204202\\
9189.60000000003	24.4991675462746\\
9192.60000000002	24.5009627204195\\
9193.39999999997	24.5031815956926\\
9196.40000000003	24.504974718643\\
9197.49999999993	24.5020440114813\\
9199.89999999997	24.5\\
9202.20000000001	24.4960499005683\\
9204.20000000001	24.498332301207\\
9207.39999999995	24.500678794461\\
9207.79999999998	24.5028725333681\\
9210.99999999999	24.5008739455657\\
9211.79999999998	24.5030883965306\\
9216.8	24.504985407241\\
9218.19999999997	24.5066877840817\\
9221.00000000007	24.5100909869755\\
9222.40000000004	24.5063702900515\\
9226.1	24.5019618044265\\
9227.99999999997	24.5001679652366\\
9230.7	24.5038349872167\\
9231.70000000006	24.5065967633614\\
9234.29999999999	24.5083600450489\\
9235.60000000003	24.51032406856\\
9236.99999999997	24.507691808035\\
9237.90000000002	24.5096341199394\\
9239.60000000007	24.5072891977012\\
9239.99999999994	24.5094750056818\\
9241.40000000005	24.5111724287183\\
9244.30000000003	24.5067283977327\\
9245.5	24.5046292290387\\
9246.30000000005	24.5068350925766\\
9249.30000000004	24.5086167751422\\
9251.60000000008	24.5046856253445\\
9253.19999999998	24.5069326618612\\
9256.60000000006	24.5044130197587\\
9257.79999999996	24.5087978915305\\
9259.10000000001	24.510756868844\\
9260.60000000002	24.5078665759608\\
9261.90000000001	24.5055063701144\\
9262.40000000003	24.5074224021592\\
9263.30000000007	24.5093594144698\\
9266.6	24.5071060895464\\
9267.19999999996	24.509835658714\\
9268.50000000004	24.5074768573463\\
9269.60000000003	24.5056474319557\\
9271.50000000001	24.5081755036887\\
9273.29999999998	24.5098884982854\\
9276.49999999993	24.5068236207231\\
9277.80000000003	24.5087789262656\\
9279.3	24.5069724335625\\
9280.09999999996	24.5091700609901\\
9283.00000000005	24.5058224084627\\
9283.4	24.5079980610761\\
9284.80000000002	24.5043026850047\\
9286.20000000003	24.5016852783132\\
9289.59999999997	24.5034823514215\\
9291.39999999994	24.5051929182586\\
9293.00000000005	24.5031259751859\\
9293.90000000004	24.5061329890252\\
9296.49999999993	24.5078846029731\\
9298.70000000005	24.5053125134426\\
9299.60000000007	24.5072421691022\\
9300.80000000006	24.5040802502984\\
9303.20000000003	24.5063579590038\\
9303.89999999996	24.5045141874463\\
9305.30000000008	24.5019021213489\\
9305.79999999994	24.5038094112337\\
9308.00000000007	24.499092188524\\
9309.50000000005	24.5015897568102\\
9310.00000000008	24.5034962030483\\
9310.80000000008	24.5013908429905\\
9311.20000000006	24.5035601903064\\
9311.99999999994	24.5014550960578\\
9312.39999999993	24.5036241610738\\
9313.30000000001	24.5012562544291\\
9316.10000000001	24.5035529507739\\
9319.40000000003	24.5013144482\\
9319.89999999994	24.5042918454936\\
9323.80000000006	24.500477268096\\
9325.30000000003	24.5051150620885\\
9329.29999999998	24.5031834844685\\
9329.79999999997	24.5050857994191\\
9331.10000000004	24.502743484225\\
9334.6	24.5053402894576\\
9336.09999999995	24.5035453396457\\
9337.19999999994	24.5017296220535\\
9339.50000000008	24.4999785858067\\
9340.79999999997	24.497639413761\\
9341.30000000005	24.4995396835592\\
9342.60000000003	24.5014824408362\\
9343.10000000005	24.5033821388818\\
9344.79999999996	24.505345161532\\
9346.39999999994	24.5075696784893\\
9348.10000000008	24.5095312466571\\
9349.09999999994	24.512257733282\\
9351.39999999997	24.5094369887184\\
9353.4	24.5074036456941\\
9355.80000000005	24.5043234750265\\
9357.89999999994	24.5063047659756\\
9363.80000000005	24.510086609212\\
9364.79999999996	24.5074693803458\\
9365.69999999993	24.5093852100194\\
9367.70000000006	24.5116249279447\\
9368.50000000006	24.5138014217706\\
9369.59999999994	24.5162598589069\\
9370.50000000007	24.518173862933\\
9371.30000000006	24.5203491474059\\
9374.30000000002	24.5221027479092\\
9377.80000000001	24.5193486814745\\
9378.2	24.5225680560443\\
9382.60000000006	24.5206603642875\\
9383.00000000006	24.5228122901813\\
9384.00000000005	24.5201990601123\\
9384.49999999996	24.5220893804744\\
9388.69999999995	24.5239008179959\\
9390.40000000007	24.5205260635749\\
9390.99999999993	24.5232187922608\\
9396.19999999999	24.5170971552632\\
9396.60000000003	24.5192461183181\\
9396.99999999996	24.5213948984261\\
9400.39999999995	24.5231636615074\\
9401.59999999998	24.5253517980791\\
9404.30000000005	24.523627238314\\
9404.80000000005	24.5255132962605\\
9406.59999999994	24.5229464105372\\
9408.69999999998	24.5259756823399\\
9411.40000000003	24.5242522445944\\
9413.09999999998	24.5261972549186\\
9414.80000000002	24.5238929781517\\
9417.00000000003	24.5255970521711\\
9418.00000000004	24.5229929603636\\
9418.30000000001	24.525397095048\\
9418.80000000006	24.5272802556562\\
9421.99999999999	24.525318135023\\
9424.79999999998	24.52015405999\\
9427.00000000001	24.5218571989265\\
9427.90000000003	24.5237590156979\\
9428.79999999995	24.5267210385093\\
9431.10000000002	24.5239206039528\\
9431.79999999995	24.5263414582428\\
9434.10000000005	24.5246019800301\\
9435.40000000005	24.5222828678925\\
9436.10000000006	24.5257624891376\\
9437.89999999996	24.5221445221445\\
9438.80000000005	24.52404411531\\
9442.99999999998	24.5258442672428\\
9445.20000000007	24.5307189819275\\
9445.90000000002	24.5289011221681\\
9446.29999999998	24.5310382791328\\
9449.30000000008	24.5338328359473\\
9450.20000000003	24.5357290244754\\
9452.29999999996	24.5334518217596\\
9455.49999999998	24.5314945640679\\
9456.19999999997	24.5339086111904\\
9456.69999999997	24.5357837746384\\
9460.8	24.5378346668922\\
9461.90000000005	24.5360388924118\\
9467.90000000002	24.5384452893959\\
9468.69999999998	24.5363720851639\\
9471.30000000004	24.5317482104018\\
9472.99999999999	24.5294570942986\\
9473.79999999998	24.5316078911536\\
9474.89999999993	24.5298153034301\\
9476.59999999995	24.5317462829888\\
9477.70000000004	24.5299542087826\\
9481.29999999995	24.5280232877001\\
9484.60000000004	24.5247609307622\\
9487.90000000005	24.5267706576728\\
9488.30000000001	24.5288984444163\\
9489.10000000001	24.5268305020444\\
9493.09999999997	24.5249231028526\\
9494.29999999994	24.5228766430738\\
9494.7	24.526056367696\\
9497.39999999995	24.5243485127665\\
9497.89999999995	24.5262160454833\\
9498.80000000006	24.5238922401541\\
9502.10000000004	24.5258992654333\\
9502.90000000007	24.5280437756498\\
9503.80000000004	24.5299298182851\\
9505.09999999995	24.527626983125\\
9505.50000000006	24.5308028951355\\
9506.80000000006	24.5327078227393\\
9508.69999999995	24.5309607942117\\
9509.89999999995	24.5289169295478\\
9512.59999999998	24.5261597653663\\
9514.89999999994	24.5244351024698\\
9518.3	24.5209278870398\\
9519.09999999995	24.5230691654761\\
9520.3	24.5210285282131\\
9521.20000000002	24.5229117872559\\
9522.29999999993	24.5211291271108\\
9522.70000000002	24.524299575755\\
9524.2	24.5225370893399\\
9525.80000000003	24.5205177463547\\
9526.20000000002	24.5226373303381\\
9528.69999999998	24.5204013097137\\
9530.00000000003	24.518105791125\\
9531.29999999998	24.5200075539795\\
9532.50000000001	24.5179699137696\\
9535.70000000008	24.5139369533757\\
9536.30000000003	24.5176376829831\\
9539.89999999994	24.5157232704403\\
9540.80000000006	24.5176031611274\\
9541.50000000007	24.5158044772365\\
9542.29999999999	24.5179409792086\\
9543.2	24.5198201879853\\
9545.70000000003	24.5175888872593\\
9546.09999999994	24.5197041754834\\
9547.69999999994	24.5218793858271\\
9549.70000000008	24.5240738025927\\
9551.79999999998	24.5260105319361\\
9553.80000000002	24.5282031421723\\
9563.30000000005	24.529978877805\\
9564.40000000005	24.5282032516075\\
9564.79999999993	24.5303139604178\\
9565.70000000002	24.5342783666813\\
9566.99999999996	24.5361708354674\\
9572.40000000006	24.5317315225908\\
9575.2	24.5297797458043\\
9579.19999999994	24.5268443414446\\
9579.90000000005	24.5313152400835\\
9582.30000000002	24.5293454666889\\
9582.59999999994	24.5317081824538\\
9583.09999999994	24.5335587277736\\
9583.80000000008	24.5317668172665\\
9584.10000000003	24.5341290874564\\
9588.59999999997	24.5320012097573\\
9590.59999999999	24.5341841575693\\
9591.30000000001	24.5323936026023\\
9592.00000000005	24.5347734072831\\
9595.79999999994	24.537562917496\\
9596.70000000004	24.5394298099366\\
9597.69999999998	24.5368730334035\\
9599.30000000002	24.5390336896056\\
9600.19999999997	24.5367332270867\\
9600.89999999999	24.540152067493\\
9603.90000000003	24.5376926280716\\
9605.99999999994	24.5406564578757\\
9609.99999999994	24.5366853622751\\
9611.30000000002	24.5385687829036\\
9612.79999999996	24.5357800455638\\
9615.09999999999	24.5340710541642\\
9616.39999999997	24.5359538293558\\
9617.29999999993	24.5378168735833\\
9618.00000000003	24.5360310248386\\
9619.20000000006	24.5340097512293\\
9619.59999999993	24.5361081946422\\
9620.29999999998	24.5405596440896\\
9622.29999999994	24.5364981709345\\
9623.10000000008	24.5386150137168\\
9624.19999999996	24.5368494332055\\
9625.10000000005	24.5387108839297\\
9626.40000000007	24.5364358801226\\
9626.79999999995	24.5385326532944\\
9630.69999999995	24.5358641026706\\
9631.60000000001	24.5377243892563\\
9632.90000000003	24.5354510536697\\
9634.9	24.5334717176959\\
9635.40000000004	24.5353121270303\\
9636.69999999998	24.5382284575793\\
9637.89999999997	24.5362108321228\\
9640.39999999993	24.5339971993154\\
9642.1	24.5307087594118\\
9642.89999999996	24.5328217359743\\
9643.59999999994	24.5351887760922\\
9645.3	24.5370850353536\\
9647.8	24.5348728738897\\
9649.99999999994	24.5323882654066\\
9650.69999999999	24.5368259626145\\
9651.40000000001	24.5350463658499\\
9651.80000000006	24.5371377656213\\
9658.69999999997	24.5330682900567\\
9662.09999999994	24.5285752727122\\
9663.00000000001	24.5314650577972\\
9663.90000000006	24.5333195364238\\
9664.59999999999	24.5315426241891\\
9667.79999999999	24.5296289783717\\
9668.7	24.5314827072646\\
9672.40000000007	24.534505040062\\
9673.59999999994	24.5366302448908\\
9679.19999999997	24.5348320643022\\
9685.20000000006	24.5320227561356\\
9686.30000000005	24.5343987446317\\
9687.49999999996	24.5365209133325\\
9688.40000000001	24.53940238427\\
9689.59999999996	24.541523473379\\
9691.49999999999	24.5439349539808\\
9692.30000000006	24.5470678057034\\
9693.50000000006	24.5491870925147\\
9698.80000000007	24.5512377692316\\
9700.29999999999	24.5536266545709\\
9701.39999999993	24.5518734216358\\
9701.7	24.5542064359191\\
9702.20000000003	24.5560331055523\\
9703.50000000003	24.5578960385836\\
9705.09999999997	24.5538474220006\\
9705.39999999997	24.5561794858585\\
9706.69999999998	24.5539209626241\\
9707.49999999994	24.5560179653055\\
9708.80000000005	24.557879883406\\
9713.10000000008	24.5614215706461\\
9714	24.5632637094533\\
9717.30000000002	24.5610965896227\\
9719.49999999996	24.5586238116795\\
9720.10000000004	24.5612230201025\\
9721.20000000004	24.5594724985341\\
9727.69999999994	24.5625938033265\\
9729.40000000003	24.5583020710211\\
9731.30000000008	24.5565900076042\\
9732.09999999995	24.5586814903105\\
9732.99999999994	24.5564105988842\\
9737.99999999998	24.5581787001571\\
9748.59999999996	24.5601977699591\\
9750.50000000005	24.5554119746477\\
9751.80000000008	24.5572657635948\\
9752.99999999993	24.5552696065866\\
9754.60000000003	24.5584179933776\\
9756.19999999998	24.5554154751289\\
9758.10000000005	24.5537086757804\\
9759.79999999995	24.5555794629043\\
9760.80000000005	24.5530637543669\\
9763.69999999995	24.5549888363137\\
9766.89999999997	24.5571823487253\\
9772.99999999997	24.5541332842189\\
9773.40000000007	24.5561978820279\\
9777.10000000006	24.558155709201\\
9778.10000000001	24.5607576036489\\
9778.90000000005	24.5628387360671\\
9781.69999999993	24.5609192582142\\
9782.60000000005	24.5627485254582\\
9783.29999999996	24.5609910665004\\
9783.79999999997	24.56280215456\\
9790.49999999997	24.5592711376218\\
9792.2	24.5611347691554\\
9793.09999999993	24.5629620553037\\
9795.69999999998	24.5605259396884\\
9797.60000000005	24.5588250303643\\
9801.30000000001	24.5607770318526\\
9804.1	24.5588625283042\\
9806.09999999996	24.5609920254533\\
9807.29999999998	24.5630850174358\\
9809.90000000005	24.5606523955148\\
9811.59999999998	24.5625121028976\\
9814.59999999997	24.5651930267864\\
9815.69999999995	24.5624401475173\\
9820.09999999999	24.5605995804566\\
9824.20000000007	24.5625642539418\\
9825.50000000008	24.5644031916626\\
9827.39999999996	24.5667769015518\\
9829.79999999995	24.561796152555\\
9832.79999999999	24.5583703688637\\
9834.60000000002	24.5559091787243\\
9836.70000000007	24.5587996096291\\
9837.59999999997	24.556552852801\\
9839.69999999999	24.558425984268\\
9842.90000000008	24.5555216905415\\
9844.90000000009	24.5525647536821\\
9845.50000000004	24.5551312261315\\
9846.50000000005	24.557715353523\\
9847.40000000004	24.5554709317086\\
9847.90000000006	24.557270511779\\
9849.50000000007	24.5553118908382\\
9851.50000000004	24.5523569775468\\
9852.90000000005	24.5498832842789\\
9853.80000000003	24.5517003419966\\
9857.90000000003	24.5536620004058\\
9861.69999999996	24.550284937841\\
9863.70000000002	24.55341754699\\
9867.7	24.5505583818075\\
9870.49999999997	24.5476465463092\\
9875.59999999996	24.5430703646324\\
9876.39999999999	24.5451323849542\\
9877.6	24.5431628820474\\
9878.49999999993	24.5449760087462\\
9882.70000000008	24.5466871736755\\
9884.69999999993	24.5488022013597\\
9887.00000000002	24.5461257598285\\
9887.79999999996	24.5481851555942\\
9889.09999999999	24.550014156858\\
9890.19999999995	24.5523391605917\\
9891.30000000004	24.5506197302707\\
9895.00000000005	24.5485139109256\\
9899.70000000001	24.5459504232409\\
9900.49999999997	24.5480071914834\\
9901.40000000002	24.5498156844923\\
9902.50000000002	24.5480984791873\\
9905.79999999998	24.5500156472405\\
9909.39999999996	24.5481608557445\\
9910.20000000004	24.5502154324289\\
9911.50000000005	24.5480043585294\\
9914.89999999999	24.5456379223399\\
9915.29999999998	24.5476733162555\\
9918.10000000001	24.5457845173519\\
9919.29999999997	24.5478557170797\\
9921.69999999997	24.5429256788083\\
9923.90000000007	24.5455461507457\\
9924.40000000005	24.5473323593128\\
9925.80000000005	24.5448775425906\\
9933.89999999993	24.5470102677673\\
9934.79999999999	24.5488127711401\\
9936.49999999992	24.5466256063442\\
9939.29999999998	24.5447411312554\\
9940.89999999998	24.5428025349562\\
9942.10000000002	24.544869344813\\
9943.20000000006	24.5431597155874\\
9944.39999999993	24.5412036804264\\
9944.89999999993	24.5429864253394\\
9946.99999999996	24.5388103065215\\
9947.40000000005	24.5408394068862\\
9949.80000000003	24.5379350546237\\
9953.30000000003	24.5403580685997\\
9955.00000000003	24.5381764120903\\
9956.60000000004	24.5362419275463\\
9958.5	24.5335689755588\\
9965.90000000002	24.5314067830624\\
9966.40000000008	24.5341895349421\\
9968.70000000007	24.5295321402777\\
9971.10000000006	24.5316511553273\\
9972	24.534451118621\\
9972.99999999997	24.5319910559405\\
9973.40000000006	24.5340151401213\\
9974.29999999997	24.5318014116137\\
9976.60000000007	24.5281505908767\\
9977.90000000006	24.5299659250351\\
9978.59999999993	24.5282451621955\\
9980.40000000005	24.5258253594509\\
9981.10000000001	24.5281128521621\\
9984.70000000003	24.5302860347729\\
9985.5	24.528320781926\\
9987.30000000008	24.525902637323\\
9988.50000000007	24.5279618765393\\
9989.39999999997	24.5257520396416\\
9996.30000000002	24.5278300188068\\
9998.00000000004	24.5246596853402\\
10003.2000000001	24.5219077704358\\
10004.4	24.5239642161028\\
10005.2000000001	24.5260012193537\\
10006.3	24.5233050847458\\
10007.4	24.521608793405\\
10013.1	24.5176367195302\\
10014.5999999999	24.5199556651722\\
10018.3000000001	24.5218797412761\\
10020.8	24.5177578860182\\
10021.9	24.5210536818998\\
10024.9	24.5177057356608\\
10027	24.5195520140419\\
10029.0999999999	24.5213975192438\\
10034.1	24.5231308923482\\
10037.1	24.5207826883992\\
10043.9	24.5181202708084\\
10047.2	24.516039134892\\
10050.1	24.5179200413922\\
10052.6	24.5158017249097\\
10054.0999999999	24.5181118338605\\
10057.8	24.5200290319053\\
10063.5	24.5180651059263\\
10064.8	24.5198660692108\\
10069.5000000001	24.5233177087471\\
10070.2	24.5216130601869\\
10072.2999999999	24.519478972241\\
10072.7	24.5214835993964\\
10074.7	24.5176082899909\\
10076	24.5194073103681\\
10076.9999999999	24.5169741294619\\
10080.9	24.5144330919552\\
10082.1	24.5174664259785\\
10083.3	24.5145486641411\\
10083.7000000001	24.5165513001051\\
10084.6	24.5183297470426\\
10085.6999999999	24.5206131392651\\
10090.2999999999	24.5223182430825\\
10091	24.5206171775129\\
10091.4	24.5226180448893\\
10095	24.5247694426009\\
10098.7999999999	24.5204923308479\\
10105.5999999999	24.5178463639332\\
10108.5	24.5157588587935\\
10112.1000000001	24.5129645378849\\
10115.0999999999	24.5106374565011\\
10118.8000000001	24.5125458300803\\
10120.9000000001	24.5143760497975\\
10121.7000000001	24.5124384990812\\
10121.9999999999	24.5146758083797\\
10124.1999999999	24.5123119623085\\
10128.3000000001	24.5102879033214\\
10129.9999999999	24.5120976100927\\
10131.6000000001	24.5092136561485\\
10133	24.5117486257907\\
10135.8	24.5059639499206\\
10137.9	24.503846912606\\
10139.9000000001	24.5059171597633\\
10141.9000000001	24.5079865904161\\
10145.2	24.5098715661439\\
10146.8	24.5138909420611\\
10148.2000000001	24.5105091493157\\
10148.6	24.5124991378206\\
10150.6	24.514565497946\\
10152.5000000001	24.5109627090597\\
10152.8	24.5131932748279\\
10153.7000000001	24.5149599164845\\
10159.4000000001	24.5120330724937\\
10166.7000000001	24.5101703584215\\
10170.5000000001	24.5069120799166\\
10179.1000000001	24.5088022634392\\
10185.2	24.5108146053626\\
10191.4000000001	24.5125840160918\\
10191.9000000001	24.5143249607535\\
10196.4000000001	24.5123326631687\\
10199.4	24.5100250012256\\
10202.4000000001	24.5126194560157\\
10204.5	24.5154146169375\\
10210	24.512982243073\\
10213.3	24.5148530362073\\
10213.7000000001	24.5168301709452\\
10214.6000000001	24.5185859594506\\
10215.7000000001	24.5159458877425\\
10218.6	24.5187744037891\\
10219	24.5207503596207\\
10220.5	24.5181300510733\\
10226.4	24.5206082237325\\
10231.0999999999	24.5230276018453\\
10231.5	24.5250009773642\\
10232	24.5267344924307\\
10233.4	24.5243562808423\\
10234.4	24.5268454736431\\
10239.2	24.5290205385134\\
10240.6000000001	24.5266436864667\\
10242.2000000001	24.5286703181903\\
10246.3000000001	24.5256870705809\\
10246.8	24.5274180483854\\
10248.4000000001	24.5294433331707\\
10248.9	24.531173773051\\
10250.2000000001	24.5329404993025\\
10254.0000000001	24.5306755346642\\
10254.4000000001	24.5336193866108\\
10257.6	24.5318151242481\\
10260.2	24.5294971881914\\
10260.9000000001	24.5317220543807\\
10261.8	24.5334684610063\\
10264.7999999999	24.5301951309803\\
10268.1	24.5281548859586\\
10270.3000000001	24.5258217790933\\
10270.8	24.5285223300782\\
10274.5	24.5303953438577\\
10276	24.5277877794105\\
10278.9	24.5305963615138\\
10279.4	24.5323216109733\\
10280.6000000001	24.5304308072408\\
10281.1000000001	24.532155779481\\
10282.2	24.5343940558046\\
10287.8	24.5365915298555\\
10289.6000000001	24.5391022090051\\
10290.8	24.5372124887036\\
10291.8999999999	24.5394481150408\\
10293.3	24.5419394952105\\
10294.1	24.5400322511706\\
10300.6	24.5371673769744\\
10305.1	24.5351861196289\\
10307.2	24.5379488323809\\
10309.6999999999	24.5397582882306\\
10312.4	24.5372121212121\\
10313	24.5406327874257\\
10315.4999999999	24.5385629531971\\
10315.9	24.540519581233\\
10317.6000000001	24.5422914021536\\
10320.8	24.5443711304247\\
10324.1	24.5462118130218\\
10325.3	24.5482015224592\\
10327.4	24.5461147421932\\
10327.9	24.5478311386522\\
10329.0000000001	24.5500576042443\\
10331.2	24.5525732482843\\
10336.4	24.5547332269143\\
10337.5	24.5579244698963\\
10342.5000000001	24.5547541237213\\
10344.1	24.5519228166509\\
10344.8999999999	24.5538907684872\\
10348.5	24.5559785864755\\
10349.3	24.5540804297834\\
10349.8	24.5557928095924\\
10352.7	24.5537439146897\\
10356.1999999999	24.5560673213406\\
10358.2	24.5542222179315\\
10359.0999999999	24.5559502664298\\
10362.1	24.5536662098782\\
10363	24.5563586185601\\
10365.9	24.5591356357322\\
10368.8000000001	24.5609466770824\\
10370	24.5581045505829\\
10371.3	24.5598472723065\\
10373.9000000001	24.5623674571043\\
10375.1	24.5604904001851\\
10377.2	24.5622657145886\\
10379.5999999999	24.5604400897907\\
10380.5	24.5621640367609\\
10382.5999999999	24.5639380893217\\
10384.8000000001	24.5664378087415\\
10387	24.5621973409325\\
10388.6000000001	24.5603395997574\\
10388.9	24.5625180479353\\
10394.5000000001	24.560829661555\\
10395.4	24.562551103843\\
10396.1999999999	24.5606610043958\\
10397.3	24.5628714871025\\
10399.4	24.5607961921246\\
10400.2000000001	24.5627529975097\\
10405.4999999999	24.5646574921196\\
10408.9	24.5672014602748\\
10411.3000000001	24.564419770636\\
10411.6	24.5665933517101\\
10413.0000000001	24.5642507994737\\
10413.4	24.5661881211888\\
10414.4	24.5638292764895\\
10417.5999999999	24.5658830644\\
10418.7	24.5690482589166\\
10419.5	24.5671618872126\\
10427.7000000001	24.5689407161626\\
10429.7	24.5671057930162\\
10430.7999999999	24.5702671869158\\
10432.1	24.5681639539119\\
10433.3000000001	24.5701305422969\\
10440.1	24.5665791843068\\
10443.3	24.5647969052224\\
10444.6	24.5665265637117\\
10447.9	24.5683384379786\\
10449.1	24.5664739884393\\
10457.1999999999	24.559876832452\\
10459.6	24.5580657189021\\
10462.8	24.5562893652811\\
10466.5999999999	24.5540619297391\\
10467.7	24.5572135501251\\
10470.9000000001	24.5554388310572\\
10473.3	24.5526763037791\\
10476.1	24.5508867719211\\
10478.6	24.5526639754931\\
10481.3000000001	24.5501555135764\\
10482.4999999999	24.5521149333181\\
10484.4999999999	24.5493390305782\\
10486.2	24.5472664333464\\
10487.4	24.5492252681764\\
10490	24.5450472350121\\
10492.3	24.5472913728032\\
10494.9	24.5450214387804\\
10495.3	24.5469443756312\\
10497	24.5486848748702\\
10499.3999999999	24.5468831849136\\
10503.7999999999	24.5442169099097\\
10505.8	24.5414481386649\\
10506.1000000001	24.5436028249986\\
10507.3000000001	24.5417515274949\\
10507.8	24.5443904110241\\
10508.5	24.5465618636165\\
10514.1	24.5487055600997\\
10515.4	24.5466216537492\\
10517.2	24.5490762838371\\
10519.7	24.5508469742771\\
10521.6000000001	24.5473640191224\\
10525.3	24.5491857791628\\
10526.1	24.5473200205202\\
10527.7	24.5492885503144\\
10529.2000000001	24.5467409989268\\
10531.8999999999	24.5432966198253\\
10535.8000000001	24.54560123008\\
10540.2	24.539149739571\\
10540.6	24.541064635176\\
10542	24.5387541381698\\
10543.4000000001	24.5364442547541\\
10544.3	24.5381434695194\\
10546.7	24.5363522585049\\
10549	24.5319505929416\\
10550.2	24.5338995099665\\
10551.3	24.5360805201206\\
10556.6	24.5389184120038\\
10561.4	24.5362874591677\\
10567.9000000001	24.5344436033308\\
10572.2999999999	24.5374749347357\\
10573.1	24.5394015056936\\
10574.4	24.5363846990401\\
10576.4	24.5383633527159\\
10585.1	24.5352000906927\\
10586.7000000001	24.5333811916727\\
10588.4	24.5351088445011\\
10592.5	24.537884938542\\
10594.2000000001	24.5405548266521\\
10596.6000000001	24.5434899544198\\
10597.7999999999	24.5416544787175\\
10599.1000000001	24.5433617631519\\
10601.1	24.5406180432404\\
10601.4000000001	24.5427533839551\\
10604.9000000001	24.5403111739745\\
10607.3	24.5423006580312\\
10611.3	24.5396460410502\\
10613.0000000001	24.5366575270185\\
10615.1	24.533687542392\\
10615.8	24.5358377528048\\
10619.9	24.5376647834275\\
10623.8	24.535246001939\\
10625.3000000001	24.5327234739398\\
10626.5000000001	24.534658310278\\
10628.3999999999	24.5321541139389\\
10630.1000000001	24.5301123214991\\
10630.4	24.5322421334838\\
10632.7999999999	24.5304667588334\\
10634.8	24.5277341582902\\
10637.3	24.5294902889804\\
10639.8	24.5312455944135\\
10643.4	24.5285855216799\\
10644	24.5309608139721\\
10646.4	24.529187996055\\
10650.6	24.5270263926315\\
10651.3000000001	24.5291698743827\\
10653.0000000001	24.5261942533159\\
10654	24.5285852394853\\
10654.4000000001	24.5304800788399\\
10656	24.5333658655606\\
10658.8	24.5316120800458\\
10661.7000000001	24.5333808550151\\
10662.5	24.5352915799148\\
10663.3000000001	24.5334508693287\\
10666	24.5356784579181\\
10667	24.5333783315053\\
10668.0000000001	24.5357655065101\\
10668.8000000001	24.5395495318168\\
10671.6000000001	24.5377962274052\\
10672.2	24.5401647255043\\
10676.7	24.5429342124981\\
10678.4000000001	24.5455822446973\\
10680.1000000001	24.5472931218517\\
10681.7999999999	24.5490034544416\\
10686.3	24.550830962719\\
10688.7999999999	24.5525732301734\\
10698.2000000001	24.5487600833777\\
10698.5000000001	24.5508758155273\\
10700.7	24.5476973684211\\
10701.5999999999	24.5503050917144\\
10703.3000000001	24.5529457929256\\
10706.7999999999	24.5495895170404\\
10708.2	24.5519830411924\\
10711.1	24.5490701321981\\
10712.5000000001	24.5467953624704\\
10714.5000000001	24.5440800403188\\
10714.8000000001	24.5461926849527\\
10716.5	24.5441651269992\\
10720.9	24.540621210708\\
10728.2	24.5425649916576\\
10729.6	24.5402946960306\\
10731.1999999999	24.5422269436135\\
10732.6000000001	24.5446159866576\\
10734.2	24.5418890845234\\
10736.7	24.5436256612771\\
10739.5	24.541882379232\\
10743.5	24.5401913697457\\
10745.6	24.5372567631704\\
10746.3	24.5393806297923\\
10748.2000000001	24.5369035103226\\
10749.4	24.5397460346993\\
10750.4999999999	24.5428162149089\\
10751.6	24.5449556814271\\
10753.1000000001	24.5471115574899\\
10755.5	24.5444233701514\\
10758.9999999999	24.5466628249575\\
10759.9	24.5446096654275\\
10761.9	24.5465526853745\\
10763.9999999999	24.5445508681636\\
10764.4000000001	24.5464257513122\\
10765.6	24.5483340609528\\
10766.4	24.5465100078949\\
10767.6	24.549346657132\\
10771	24.5518099358469\\
10774.6000000001	24.5538158835049\\
10777.4	24.5520760844352\\
10781.1	24.5501428412422\\
10782.4999999999	24.5525198004192\\
10783.7999999999	24.5504873005128\\
10786	24.5473340688479\\
10791.3000000001	24.5454713938877\\
10795.3000000001	24.5474924504882\\
10797.1000000001	24.5452524728633\\
10797.4000000001	24.5473489233619\\
10799.5	24.5453535316123\\
10799.9	24.5472222222222\\
10801.9999999999	24.5489303005897\\
10803.2	24.5462034748642\\
10805.1000000001	24.5492910820716\\
10809.3999999999	24.5524769878348\\
10810.8	24.5492974682959\\
10813.3999999999	24.5517177602071\\
10815	24.5499348133628\\
10816.2	24.5518338063848\\
10817.3	24.5493371789894\\
10818	24.5514461874081\\
10828.9	24.5535137131776\\
10830.8000000001	24.5510530057521\\
10831.1999999999	24.5529160857884\\
10832.5	24.5555083728745\\
10835.2	24.5530811329636\\
10840.5000000001	24.5493791856539\\
10842.0000000001	24.551516772581\\
10843.3	24.5485733257097\\
10843.9000000001	24.5518258945039\\
10845.4000000001	24.553962472915\\
10847.0000000001	24.5558720764075\\
10847.8	24.5540611546935\\
10849.4000000001	24.555970321213\\
10853.5	24.5577504238225\\
10855.8999999999	24.5560058953574\\
10856.5000000001	24.5583331798169\\
10857.8	24.5563138360088\\
10859.5000000001	24.5543113926848\\
10860.1999999999	24.5564118854912\\
10862.7	24.5544426851272\\
10863.8	24.5565588784875\\
10869.4999999999	24.5538014278354\\
10872.8	24.5509477692244\\
10874.9000000001	24.5489655172414\\
10876.2000000001	24.5469507093405\\
10878.8	24.5493570121979\\
10880.0000000001	24.5475684966131\\
10886.9	24.5494626618903\\
10890	24.5461474182974\\
10894.1	24.5488425033504\\
10896.9000000001	24.5507937964577\\
10898.7	24.5476566227475\\
10899.1	24.5504257193189\\
10900.1	24.5481734280105\\
10901.5	24.5459382109048\\
10904.4000000001	24.5485808611124\\
10906.3000000001	24.5507225115528\\
10911.1	24.552753134394\\
10911.9	24.5546187683284\\
10915.5	24.5565978965884\\
10918.6	24.5542051709453\\
10919.0000000001	24.556053154564\\
10920.6000000001	24.5579495819865\\
10922.2000000001	24.5552676634042\\
10922.5	24.5573398275136\\
10926.1	24.5547399827937\\
10927.3	24.5529586177865\\
10927.8999999999	24.555270863836\\
10933	24.5529630205523\\
10933.8000000001	24.5548249023679\\
10934.5999999999	24.5530284324216\\
10935.9	24.5510241404535\\
10937.8000000001	24.5531591987493\\
10938.5999999999	24.5550202492069\\
10942.4	24.5510623714873\\
10946.0000000001	24.5484693178392\\
10948.4	24.5503950312828\\
10949.3000000001	24.5538568323378\\
10951.6999999999	24.5557807848938\\
10953.3	24.5594975076232\\
10959.0000000001	24.5567610478963\\
10959.6	24.5590663978029\\
10961.4	24.5568580942389\\
10965.7999999999	24.5543001486426\\
10968.7000000001	24.5560134198819\\
10970.3000000001	24.5533435426238\\
10970.7999999999	24.5558705302209\\
10971.7	24.5538562496582\\
10973.3	24.5520987114295\\
10974.9	24.5558086560364\\
10982.1999999999	24.5540551614871\\
10982.9	24.5561322043158\\
10984.5	24.5580175882599\\
10985.3000000001	24.5598703734047\\
10988.0999999999	24.5581623923846\\
10989.8000000001	24.5561834047626\\
10992.6000000001	24.5581158405123\\
10995.4	24.560047292074\\
10997.4000000001	24.556490111389\\
11000.2	24.5584211339691\\
11002.6000000001	24.5603351904533\\
11006.3000000001	24.5575301642681\\
11006.8999999999	24.5598255655492\\
11008.1	24.5580567213532\\
11008.5	24.5598895409044\\
11009.1	24.5630926861171\\
11012.3000000001	24.5604954415023\\
11016.0000000001	24.5622316427774\\
11018.6	24.5646038098868\\
11020.7	24.5626451800232\\
11021.9	24.560878243513\\
11025.6000000001	24.5626128046292\\
11026.4000000001	24.5644583503378\\
11028.0000000001	24.5618012168914\\
11031.8	24.5596860015047\\
11033.9000000001	24.5622620989668\\
11035.1	24.5641220820647\\
11036.6999999999	24.5605610321832\\
11037.1	24.5623890117059\\
11038.3	24.5606247282215\\
11039.1	24.5633741575476\\
11042.8000000001	24.5651051807043\\
11045.6	24.562499434169\\
11048.5	24.5605778107634\\
11054.9	24.5635459068295\\
11057.7	24.5618477454828\\
11060.8	24.5640047374084\\
11063.6999999999	24.5657007538097\\
11066	24.5678242560613\\
11067.1999999999	24.5696782413055\\
11068.8000000001	24.5715473082239\\
11070.0000000001	24.5697870841275\\
11073.8999999999	24.5728733971465\\
11074.5000000001	24.5751539558991\\
11076.9	24.5770515482531\\
11081.8	24.5752082224167\\
11082.6	24.5779458074296\\
11089.1999999999	24.5804514261495\\
11090.8999999999	24.5829952213507\\
11093	24.5855531817076\\
11097.8	24.5884356499878\\
11102.8000000001	24.5854686613407\\
11103.2	24.5872848612575\\
11104.8	24.5855433187152\\
11105.7000000001	24.5880530893767\\
11107.7000000001	24.5899277984839\\
11110	24.5875374659094\\
11111.4000000001	24.585339513117\\
11112.1	24.5873904357373\\
11112.9000000001	24.5892198326285\\
11115.8000000001	24.5864032601949\\
11117	24.5882469349021\\
11119.6	24.5860949486047\\
11123.2	24.5817338379797\\
11124.3	24.5837977778577\\
11128.2999999999	24.5857445814313\\
11130.6000000001	24.5878516175982\\
11132.1000000001	24.585436840876\\
11134.6	24.5835092099473\\
11136.7	24.5806694921342\\
11138.3000000001	24.5825253178194\\
11140.7	24.5844104552635\\
11141.8	24.5864708891661\\
11143.7999999999	24.584750401565\\
11144.3	24.5872366390295\\
11145	24.5901786435294\\
11145.8	24.5928996312545\\
11152.5	24.5906784068289\\
11155.6000000001	24.5928090572532\\
11158	24.5911042202526\\
11160.4000000001	24.5894001164822\\
11161.2	24.5921174056786\\
11162.8000000001	24.5885925700311\\
11164.5	24.586639915447\\
11165.6	24.5895913377576\\
11166.4	24.5878296690995\\
11166.8	24.5896354404535\\
11168.9	24.5876980929358\\
11170.1	24.589532864228\\
11171.2999999999	24.5913672413484\\
11174.7999999999	24.5935086667442\\
11176.3999999999	24.5953563280097\\
11179.6999999999	24.5934632104331\\
11182.4000000001	24.5955734406439\\
11185.2000000001	24.5920985579287\\
11187.4	24.5899441340782\\
11191	24.5873953409406\\
11192.3000000001	24.5899002894822\\
11197	24.5920818783435\\
11197.4000000001	24.5938825630721\\
11198.2	24.5921255904914\\
11200.5	24.5942181668839\\
11202.7999999999	24.5918467539655\\
11205.2000000001	24.5901493043471\\
11205.5	24.5921682016135\\
11207.7999999999	24.5897982672936\\
11209.4	24.5916410187787\\
11212.4	24.5895206243032\\
11214.4000000001	24.5913772348299\\
11215.2999999999	24.5894038554131\\
11216.1	24.5912162764573\\
11217.2000000001	24.5888047925972\\
11218.5	24.5868468436347\\
11221.6000000001	24.5836192377269\\
11223.4	24.5814585467991\\
11225.2000000001	24.5792985488138\\
11230.1	24.5748072162562\\
11231	24.5790706164134\\
11232.8000000001	24.5760222204417\\
11234.2000000001	24.5791905147628\\
11236.7000000001	24.5817314537947\\
11241.1999999999	24.5790077660057\\
11242.4999999999	24.5770551295964\\
11246	24.5711846773548\\
11246.4	24.5738674254212\\
11247.8	24.571697828039\\
11248.5000000001	24.5737247301886\\
11250.2	24.5717891967325\\
11250.9	24.5738156608301\\
11251.9	24.5716317099182\\
11253.3999999999	24.5692451237393\\
11254.4000000001	24.5715047314408\\
11255.9	24.569118692253\\
11259	24.5712357115578\\
11259.8	24.5739304967184\\
11267.1	24.5722095995456\\
11268.3	24.5704802811402\\
11268.6	24.5724883970644\\
11270.6000000001	24.5707897468658\\
11271.2000000001	24.5730306175863\\
11276.4	24.5750011085\\
11281.2999999999	24.573191270587\\
11284.2	24.5757379722269\\
11285.8000000001	24.5740259970406\\
11290.9	24.571782835887\\
11294.4	24.5677099473195\\
11295.7000000001	24.5657678075037\\
11298.0000000001	24.5634221683292\\
11304	24.5654231650463\\
11314.2	24.5636053489831\\
11315.8000000001	24.5610159156585\\
11318.7	24.5556066014065\\
11319.1	24.5582726694466\\
11321.3000000001	24.560566714364\\
11322.5999999999	24.5586300087435\\
11323.4	24.5604274296816\\
11326.9000000001	24.5581354286219\\
11330.6999999999	24.556077240795\\
11336.9999999999	24.5583085621543\\
11338.2999999999	24.5563747971495\\
11339.6	24.554441475524\\
11340.8	24.5562521493003\\
11341.6000000001	24.5545200454958\\
11345.4	24.5568727689392\\
11352	24.5549281630712\\
11358.2999999999	24.5527539090013\\
11358.9000000001	24.5558587903865\\
11362.5999999999	24.5531431789979\\
11365.4	24.5550129778716\\
11369	24.5525151507155\\
11370.6000000001	24.5552164774376\\
11372	24.5521935262616\\
11373.3	24.5502664110996\\
11375.6000000001	24.5523352409083\\
11376.8	24.5541404073166\\
11383.9	24.5520028109628\\
11387.6000000001	24.5501725545984\\
11387.8999999999	24.5521601685985\\
11388.3000000001	24.5539320712304\\
11389.8000000001	24.5559662508011\\
11392.4	24.552117621242\\
11392.8000000001	24.5538888254966\\
11397.3	24.5556004000913\\
11398.3	24.5578326782706\\
11400.8	24.5603417274075\\
11412.2000000001	24.5620952831594\\
11414.8000000001	24.5600049058686\\
11420.1	24.5538607029649\\
11424.9000000001	24.5557986870897\\
11426.1	24.5593460643083\\
11430.1	24.5612500218719\\
11437.1	24.5593327038086\\
11438.9	24.5572165399073\\
11440.9	24.560790140722\\
11442.1999999999	24.5579997028569\\
11442.5999999999	24.5597629929999\\
11444.9	24.5618173875055\\
11446.4	24.5594723277858\\
11451.5	24.5572671067798\\
11452.4000000001	24.5597031215892\\
11453.2000000001	24.5614800974392\\
11453.8999999999	24.5652173913044\\
11455.2000000001	24.5633025761001\\
11456.4	24.5616025836861\\
11456.8000000001	24.5642363990259\\
11461.3	24.561571884761\\
11462.1	24.5633473504214\\
11469.0999999999	24.5657936037387\\
11472.6	24.5678872453738\\
11476.5	24.5699945977032\\
11478.8000000001	24.5720408749967\\
11482.0000000001	24.574772907395\\
11485.1	24.5716226099676\\
11486.4	24.5758063814043\\
11492.3	24.572761129094\\
11494.6	24.5748040401228\\
11496.4000000001	24.5726960379246\\
11496.7	24.5746642544012\\
11501.4000000001	24.5715776203104\\
11502.4999999999	24.5735746700746\\
11504.1000000001	24.5753724726622\\
11505.6000000001	24.5730377117429\\
11510.0000000001	24.5705945213334\\
11510.6	24.5727887965111\\
11511.7000000001	24.570440765128\\
11514.6000000001	24.5729372020113\\
11515.3000000001	24.5749170675791\\
11516.0999999999	24.5732099129921\\
11518.3	24.5754618697041\\
11520.7	24.5772862995625\\
11524	24.5754549162191\\
11526.8000000001	24.5772931143673\\
11535.8999999999	24.5752427184466\\
11540.3000000001	24.5780042286229\\
11541.2000000001	24.5804198833754\\
11542.7999999999	24.5822107096137\\
11548.2	24.5845708892218\\
11551.1000000001	24.5827273356881\\
11552.8	24.5808411740775\\
11555.8000000001	24.5839787467874\\
11562.3	24.5822666574414\\
11568.2	24.5844246777833\\
11570.3	24.5868768581899\\
11572	24.584129069054\\
11572.3	24.5860841312087\\
11582.4	24.5879559680553\\
11586.0000000001	24.589810203606\\
11588.9	24.5879713521443\\
11589.2999999999	24.589711287901\\
11591.1	24.5919318103389\\
11591.8	24.5938974628836\\
11592.9000000001	24.5958768222203\\
11594.3	24.5937694059201\\
11595.4999999999	24.5955362378833\\
11598.5999999999	24.5975842120238\\
11605.1999999999	24.5956588799945\\
11606.4000000001	24.5974238573213\\
11608.6000000001	24.5996537079949\\
11610.1999999999	24.6014314875585\\
11615.6000000001	24.6037690367348\\
11625.4	24.6019526041891\\
11631.1000000001	24.6053717587179\\
11636.6999999999	24.6072803519868\\
11637.2	24.6096603163964\\
11640.7000000001	24.6074152979177\\
11641.1	24.6091468233515\\
11646.9	24.6063363956384\\
11647.4	24.6087143163769\\
11649.5999999999	24.6066422311304\\
11650.2	24.6088083568663\\
11651	24.6114100814515\\
11652.9	24.6073972367631\\
11654.5999999999	24.6098140664281\\
11655.7000000001	24.6074915492716\\
11657.1	24.6096832858662\\
11662.3	24.6115722321306\\
11665.7000000001	24.6138284558281\\
11670.9	24.6157141633108\\
11673.5999999999	24.6134473217575\\
11675.6	24.6109440975702\\
11679.5999999999	24.6127897120645\\
11680.9999999999	24.6106959104879\\
11686.3	24.608091456736\\
11687.1	24.6098295571223\\
11695.7000000001	24.6071239248277\\
11696.7999999999	24.6090844582753\\
11698.4000000001	24.610847544557\\
11701.4	24.6130837926762\\
11703.3999999999	24.6105865766651\\
11710	24.6086711471294\\
11711.9999999999	24.6104456075341\\
11713.7999999999	24.6083712512485\\
11716.8	24.6046309177342\\
11718	24.606378167109\\
11720.5000000001	24.6036892309267\\
11721.5999999999	24.605645938729\\
11727.6999999999	24.6073432357305\\
11729.6	24.6050623630613\\
11730.2000000001	24.6072137967486\\
11739	24.6092119498088\\
11741.1	24.6073655163016\\
11754.6000000001	24.6097305758548\\
11755.3999999999	24.6114584662498\\
11761.2999999999	24.608464978659\\
11767.4	24.605056299129\\
11772	24.6030869598457\\
11775.8999999999	24.605129076087\\
11779.1	24.6069342569954\\
11779.9	24.6095076400679\\
11786.9	24.6059217782303\\
11788.1999999999	24.6040565645598\\
11789.2	24.6062107164972\\
11791.7000000001	24.6086263335538\\
11792.4	24.6105575577698\\
11794.5	24.6129584725213\\
11797.8	24.6103120046788\\
11798.2	24.6120203758169\\
11800.6000000001	24.6137940969604\\
11802.2	24.6155410386111\\
11803.1	24.617900230446\\
11804	24.6160232461602\\
11805.1999999999	24.6177564314333\\
11807.3	24.6159188305639\\
11810.0000000001	24.6196052531308\\
11814.4999999999	24.6169993059435\\
11819.1	24.6141870854203\\
11823.7	24.6113770530625\\
11828.2	24.6087772545505\\
11829.0999999999	24.6111317756061\\
11831.7	24.6074139184232\\
11842.7	24.6098895531462\\
11843.7999999999	24.6076039142512\\
11844.5000000001	24.6095267041521\\
11851	24.6112175241117\\
11853	24.6138141077018\\
11854.1000000001	24.6115300906008\\
11854.5999999999	24.6138662302715\\
11857.5	24.6120631493726\\
11857.8	24.6139704332133\\
11858.7000000001	24.6121024049651\\
11860.0999999999	24.6142560833713\\
11863.7	24.6185876363391\\
11867.3000000001	24.615332760335\\
11870.4999999999	24.6129092042525\\
11873.0999999999	24.6150995519321\\
11875.3	24.6172760496489\\
11876.9	24.6148017176055\\
11877.6000000001	24.6167187250057\\
11881.0999999999	24.6187253812746\\
11888.6000000001	24.6158116530823\\
11894.5000000001	24.613690245994\\
11895.2999999999	24.6153975486322\\
11896.2	24.6135353008919\\
11897.2000000001	24.6156691013928\\
11902.0000000001	24.6175044740004\\
11903.7999999999	24.6154621594603\\
11904.5000000001	24.6173747962972\\
11905.1999999999	24.6192872082182\\
11911.7000000001	24.6218035897178\\
11914.2	24.6199944604383\\
11917.3999999999	24.6226138032305\\
11918.5000000001	24.6203413152552\\
11919.2	24.6222513067042\\
11922.7000000001	24.6192169624585\\
11925.5999999999	24.6165843514427\\
11928.7000000001	24.6135403393468\\
11932.4	24.6117745652629\\
11932.7000000001	24.6136698846876\\
11935.3000000001	24.6158486519095\\
11938.2	24.6140572778369\\
11940.2	24.6157969230254\\
11943.7000000001	24.6177933321054\\
11946.6000000001	24.6151656942922\\
11946.9	24.6170586758182\\
11947.7000000001	24.6187582651199\\
11948.9000000001	24.6213072223617\\
11950.7000000001	24.6192723499682\\
11952.9000000001	24.6172508993558\\
11956.2000000001	24.6154746869851\\
11956.9000000001	24.617378941206\\
11962.1000000001	24.6192172008493\\
11962.7	24.6213261109439\\
11966.0000000001	24.6195502293981\\
11969.6	24.6213355389024\\
11971.6	24.6230694053476\\
11977.2	24.625750377798\\
11979.1999999999	24.6274824071523\\
11983.2000000001	24.6251032687156\\
11985.0000000001	24.6272454964915\\
11987.4	24.6248175182482\\
11988.6	24.6265233094497\\
11991.0000000001	24.6232622528375\\
11995.7	24.6194501408826\\
11996	24.6213352672952\\
11998.7000000001	24.6191285795246\\
12000.9000000001	24.6171152403966\\
12002.6999999999	24.6192555070484\\
12005	24.6170377589524\\
12006.2	24.6187418272074\\
12007.2	24.6166915126631\\
12008.3000000001	24.6186003131142\\
12009.4	24.6163453932304\\
12012.4999999999	24.614155137106\\
12012.8	24.6160377594086\\
12015.3000000001	24.614245052183\\
12021.0999999999	24.6123515123282\\
12023.6000000001	24.6105608090688\\
12027.4000000001	24.6086052795677\\
12031.4999999999	24.6068685794076\\
12039.7000000001	24.6092127776209\\
12044.4	24.6070820706547\\
12048.2999999999	24.6099067096046\\
12049.9999999999	24.6122438817935\\
12050.9999999999	24.610201558364\\
12059.2000000001	24.6075642864843\\
12066.4999999999	24.6092519848176\\
12069	24.6074686596349\\
12071.8000000001	24.6042462247037\\
12072.5999999999	24.6067573947833\\
12075.8	24.6043773134922\\
12080.6	24.6020512056421\\
12084.9000000001	24.5999172527927\\
12088.4	24.6018943624106\\
12089.7	24.6000760972059\\
12091.2999999999	24.5976479150471\\
12096.1	24.5994609877482\\
12100.7000000001	24.597547269602\\
12102.2	24.6002825909125\\
12105.7000000001	24.6022567694824\\
12108.8	24.6000875389177\\
12111.4	24.5981092350246\\
12112.1	24.5999900926339\\
12114.2000000001	24.5982021247616\\
12118.5999999999	24.5958724945745\\
12122.2000000001	24.5918678798578\\
12123.8000000001	24.5935713755475\\
12126.2000000001	24.5953011223539\\
12128.3000000001	24.5926915339204\\
12132.3	24.5903531040849\\
12138.4	24.5878815339622\\
12141.3000000001	24.5844795493106\\
12145.9000000001	24.5825786267084\\
12148.3	24.5843073985052\\
12152.4000000001	24.5825961736268\\
12156.0999999999	24.5808723120712\\
12170.4000000001	24.5774618955672\\
12173.2	24.5750946744104\\
12173.5000000001	24.5769534073733\\
12178.2000000001	24.5789642232495\\
12180.1	24.5767721383885\\
12184.1999999999	24.575067915268\\
12186.8	24.5714660824328\\
12189.6999999999	24.5664407947628\\
12191.2999999999	24.5640369440753\\
12193.1999999999	24.5659501529529\\
12197.7	24.5683647870928\\
12200.5000000001	24.5701031096831\\
12202.9	24.5726460706384\\
12204.4	24.5753615469704\\
12206.8	24.5779026616094\\
12208.1	24.5761045854426\\
12209.2000000001	24.577985633902\\
12210.4	24.5755702059703\\
12213.9000000001	24.5726215817914\\
12217.0999999999	24.5702779687653\\
12218.6	24.5721721623413\\
12223.0000000001	24.5739624154265\\
12228.6	24.5717042694645\\
12233.6	24.5682009531049\\
12236.9	24.5632099370761\\
12239.2	24.5651303587623\\
12242.6	24.567293162456\\
12245.3000000001	24.565142829144\\
12245.8	24.5674062339232\\
12251.6999999999	24.5645537798528\\
12254.8000000001	24.56649993064\\
12259.4999999999	24.5685014192959\\
12260.7000000001	24.5660968289182\\
12263.5000000001	24.5678267392935\\
12265.9	24.5654655144301\\
12266.2000000001	24.5673104359098\\
12267.7	24.5691974111088\\
12274.4	24.5655627520469\\
12276.8	24.5672767555328\\
12278.1999999999	24.565289983141\\
12279.8000000001	24.5677896399808\\
12281.7	24.5696884821443\\
12283.6000000001	24.5675163020914\\
12285.3	24.5657447050971\\
12290.5000000001	24.5634875433258\\
12291.7000000001	24.5659708098082\\
12292.8	24.5678399726672\\
12295.2	24.5695509666295\\
12297.3	24.5677948184169\\
12302.8000000001	24.5698168724447\\
12305.1	24.5717257744693\\
12306	24.56992873453\\
12313.3000000001	24.5651079311969\\
12313.9999999999	24.5669598265403\\
12314.9000000001	24.5700365408039\\
12316.6999999999	24.5664458300857\\
12318.5000000001	24.5685386326368\\
12322.9999999999	24.5660588650583\\
12326	24.5681926967979\\
12328.4000000001	24.5698990144787\\
12329.3	24.5681055039175\\
12331.9	24.5661693156017\\
12335.5	24.5638639385194\\
12338.2000000001	24.5657829684803\\
12341.2000000001	24.5679142391806\\
12345.1	24.5698733110845\\
12348.9	24.5679812130537\\
12353.5	24.5653088978112\\
12356.3000000001	24.5670259946263\\
12357.2000000001	24.5692829339743\\
12358.0999999999	24.5715395445939\\
12359.5	24.5695653581022\\
12368.5000000001	24.5662403182252\\
12371.0999999999	24.5626939989653\\
12372.6000000001	24.5605243802889\\
12373.6	24.5625803114671\\
12376.6000000001	24.564706262574\\
12379.5999999999	24.5668311833082\\
12380.7000000001	24.5686869992246\\
12382.1	24.5707547931708\\
12383.4000000001	24.5681753946784\\
12388.9	24.5717975623537\\
12398	24.569087198845\\
12401.4000000001	24.5671894528888\\
12410	24.5654748954481\\
12411.2000000001	24.5679340600904\\
12414.8999999999	24.565445026178\\
12417	24.5637065015181\\
12417.6000000001	24.5657408376753\\
12419.9	24.5636070853462\\
12420.4	24.5666438549173\\
12422.6	24.5630981992643\\
12426.1	24.5650319486247\\
12427.7000000001	24.5683065385668\\
12429.7	24.5651579269176\\
12432.7	24.5672736632134\\
12436.3000000001	24.5690071081664\\
12439.4000000001	24.5709232686201\\
12444	24.5730908623364\\
12450.3	24.5702949302834\\
12452.3000000001	24.5679547717709\\
12455.7000000001	24.5700798021805\\
12468.8000000001	24.5723359718981\\
12472.8000000001	24.5740765980646\\
12482.0999999999	24.5717902292865\\
12482.4	24.573603044262\\
12483.7000000001	24.5718451112642\\
12484.3999999999	24.5736713524771\\
12488.3	24.5700009608917\\
12492.7999999999	24.5675543708827\\
12497.6	24.5653200188835\\
12500.8	24.5670311737555\\
12505.4	24.5643916676662\\
12507.5	24.5666634686111\\
12510	24.5649515191725\\
12510.6999999999	24.5667743069988\\
12516.2000000001	24.5631696268066\\
12517.3	24.5658044002748\\
12519.6000000001	24.5684800754012\\
12522.1	24.5667694175145\\
12523.5999999999	24.5686179004607\\
12526.3000000001	24.570507089028\\
12527.0000000001	24.572327194642\\
12534.3	24.5699834056676\\
12536.4	24.5682606788179\\
12538	24.5707084805513\\
12540.7000000001	24.5725950497576\\
12542.7000000001	24.5750550116401\\
12546.2999999999	24.5767710259517\\
12548.2	24.5738466565192\\
12549.2000000001	24.5766696150383\\
12549.8000000001	24.5786819018478\\
12554.1000000001	24.5806184384509\\
12557.3000000001	24.5823179957635\\
12562.1000000001	24.5840696693254\\
12566.2	24.5816190923343\\
12578.9999999999	24.578864942643\\
12582.0999999999	24.5767830745021\\
12585.7	24.5737259451127\\
12587.5000000001	24.5765674155518\\
12592.9000000001	24.5795283093782\\
12593.8	24.5777717784007\\
12594.9	24.5795950774117\\
12600.7000000001	24.5778045838359\\
12601.2	24.5807972193345\\
12603.3	24.577494961677\\
12605.5	24.5795519451672\\
12608.4999999999	24.5768761004394\\
12612.7000000001	24.575034885196\\
12616.3	24.5767413842301\\
12621.3999999999	24.57473358951\\
12624.4	24.5712701493128\\
12626.5000000001	24.5695595013701\\
12629.0999999999	24.571627656542\\
12631.1	24.5693204129457\\
12638.8	24.5670113696603\\
12639.5000000001	24.568815468844\\
12642.3	24.5712839334304\\
12644.4	24.5695757048519\\
12653.2	24.5675041293576\\
12657.6	24.5692345370802\\
12663.1	24.5711984332554\\
12665.2000000001	24.5734408186146\\
12667.1000000001	24.5752810407983\\
12669.4000000001	24.5771340621177\\
12670.6000000001	24.5803309998658\\
12672.8999999999	24.5821825929141\\
12677.2	24.584099137829\\
12680.1999999999	24.5861690969457\\
12681.1	24.5844241869855\\
12682.6	24.5862474078863\\
12683.5	24.5845028225425\\
12690.3	24.5823614700876\\
12692.1	24.5843904130096\\
12696.7	24.5825719866423\\
12700.4000000001	24.5801346403685\\
12701.4	24.5821359681927\\
12707.1	24.579765802065\\
12709.0000000001	24.5776648228435\\
12710.7000000001	24.5759511596438\\
12718.8000000001	24.5783833507615\\
12720.5	24.5750986588683\\
12720.9999999999	24.5772771222614\\
12724.8	24.5746528459949\\
12727.8000000001	24.5727889125465\\
12732.9000000001	24.5747270870965\\
12734.6	24.5730170322034\\
12740.2	24.575559445225\\
12742.1999999999	24.5732717013412\\
12744.1000000001	24.571177476813\\
12749.6000000001	24.5731272108363\\
12751.3000000001	24.5753407468984\\
12751.8000000001	24.5775139390993\\
12754.3000000001	24.5742645675218\\
12754.6000000001	24.5760386367378\\
12755.9000000001	24.5735340232048\\
12760.1	24.570931490102\\
12761.0000000001	24.5739003690904\\
12764	24.5759591353875\\
12767.1999999999	24.573715664236\\
12770.2	24.5702920056694\\
12771.3	24.5720907653037\\
12786.5	24.5702532338542\\
12788.2000000001	24.5685509410946\\
12790.1000000001	24.5703741927413\\
12798.8	24.5724241927041\\
12800.5	24.5707232473478\\
12801.8	24.5729149579359\\
12810.5000000001	24.5749613601237\\
12811.0000000001	24.5771245248261\\
12813.6999999999	24.5797499570775\\
12815.5000000001	24.5770779362652\\
12816.2	24.5788566122828\\
12820.3	24.5811363139996\\
12821.6000000001	24.5794239453427\\
12826.3	24.5828915362066\\
12827.3	24.5809751001762\\
12831.5999999999	24.5789723886936\\
12834.5	24.5812101662693\\
12835.8	24.5794996844787\\
12836.4000000001	24.5814669107623\\
12837.4000000001	24.5795520934761\\
12838.1000000001	24.582106525837\\
12839.9	24.5841121495327\\
12841.2	24.5816233558908\\
12845.7	24.5846891591026\\
12849.5	24.5828663927282\\
12852.2000000001	24.5847046832085\\
12852.9000000001	24.5872558935657\\
12853.7999999999	24.585534351442\\
12855.6	24.587537045824\\
12860.6999999999	24.5855623289375\\
12861.3000000001	24.5875254637909\\
12869	24.5899091622569\\
12871.7000000001	24.5878587299368\\
12880.4000000001	24.5860020961919\\
12885.5	24.5809275470292\\
12888.6	24.5827740578957\\
12889.9	24.5849495733126\\
12891.2999999999	24.5830553702468\\
12895	24.5853075974595\\
12897.6999999999	24.5879142179286\\
12898.2000000001	24.5900622562663\\
12900.9000000001	24.5880164328347\\
12903.2999999999	24.5904180293566\\
12906	24.5875981125204\\
12911.5	24.5848694197466\\
12913.3999999999	24.5828009447477\\
12915.7000000001	24.5846172904505\\
12920.8	24.5826529111749\\
12925.4000000001	24.5785462844764\\
12927.5999999999	24.5805518383007\\
12929.6999999999	24.5842936472335\\
12931	24.5825954481831\\
12935.9999999999	24.5808242051314\\
12946.7000000001	24.5790465597677\\
12947.9000000001	24.5814025332098\\
12949	24.579314392506\\
12952.9000000001	24.5773179958311\\
12957.2	24.5791947396448\\
12958.6000000001	24.5811694074251\\
12961.3000000001	24.5791349699878\\
12962.6000000001	24.5812986492012\\
12965.3	24.579264812501\\
12967.2000000001	24.5772057405936\\
12968.6	24.5753236638985\\
12971.8000000001	24.5731157347806\\
12974.7	24.5761013657243\\
12975.7	24.5742073706438\\
12976.2	24.577113661059\\
12977.7000000001	24.5750435358844\\
12984.4999999999	24.5768063706237\\
12989.1000000001	24.5734918239768\\
12991.3000000001	24.5762581400003\\
12997.5	24.5783837016065\\
13000.0000000001	24.5744263505665\\
13004.2000000001	24.5711034042586\\
13008.6000000001	24.5681736068939\\
13011.3	24.5699924681433\\
13021.4	24.5716699304996\\
13024.4999999999	24.5734993780999\\
13025.7000000001	24.5712355479126\\
13031.9999999999	24.5731693280438\\
13035.4000000001	24.5705956810249\\
13043	24.5685458211621\\
13043.3999999999	24.5708590485683\\
13048.9000000001	24.5727641964902\\
13052.3	24.5709601299378\\
13055.7	24.5691570030178\\
13056.3	24.5710915719494\\
13059.4000000001	24.572916267851\\
13064.4	24.5711661372421\\
13069.0000000001	24.5686390034509\\
13069.8999999999	24.5707727620505\\
13078.1	24.5676010460155\\
13080.6000000001	24.5644346250583\\
13081.3	24.5669423761983\\
13082.2999999999	24.5650645141564\\
13090	24.5674211808924\\
13091.1	24.5653568809582\\
13093.6	24.5629577583112\\
13097.5	24.5648057659419\\
13104.3	24.566557797381\\
13106.3999999999	24.5626215999695\\
13111.5000000001	24.5599316635651\\
13112.2000000001	24.5616711027051\\
13114.3999999999	24.5644134355103\\
13118	24.5622460569747\\
13119.3999999999	24.5641983307291\\
13121.2000000001	24.5661634136861\\
13122.7	24.5679275764319\\
13124.4	24.5700788601471\\
13125.1	24.5725779416695\\
13128.6	24.5705972411587\\
13131.6999999999	24.5724120074933\\
13133.3	24.5701798468028\\
13136.9000000001	24.5680140062419\\
13141.9000000001	24.5655151422919\\
13142.3	24.5678110542975\\
13144.2	24.5703460815715\\
13151.3	24.5684870051858\\
13154.7000000001	24.5712591601545\\
13156.7	24.5675240179983\\
13162.9999999999	24.5656418320912\\
13165.2	24.5676133472082\\
13168.1	24.5652405036376\\
13170.5000000001	24.5630419267156\\
13176.3	24.5651315989193\\
13178.6	24.5669147943272\\
13179.9999999999	24.5643052784122\\
13183.6000000001	24.5621487139422\\
13186.1000000001	24.5650756093492\\
13187.1	24.5670043678719\\
13189.7000000001	24.5651943168206\\
13190.4000000001	24.5676812857739\\
13193.4000000001	24.5651267669686\\
13194.9	24.5668813944676\\
13197.2000000001	24.5641153872383\\
13206.7000000001	24.5623466698973\\
13210.3	24.5579240598316\\
13212.9	24.5553621433437\\
13216.7000000001	24.553598450457\\
13219.3	24.5555774089596\\
13221.6	24.5573564670201\\
13222.2999999999	24.5598378509197\\
13229.6	24.5576241335782\\
13231.9	24.5594014510278\\
13235.1999999999	24.5570557524197\\
13237.7999999999	24.5590312662885\\
13239.7999999999	24.556076707528\\
13242.6	24.5584359684958\\
13245.4	24.5562643916802\\
13247.4999999999	24.5584105800296\\
13249.8000000001	24.5564117465037\\
13250.1	24.5581198774358\\
13256.1	24.5545480605302\\
13258.4	24.5563223592412\\
13261.2	24.5541538159909\\
13264	24.5512322735806\\
13267.7999999999	24.5532450500833\\
13273.4000000001	24.5549402945719\\
13278	24.5532116793818\\
13281.5	24.5550234911456\\
13282.9000000001	24.5531882857788\\
13285.6	24.5497038168858\\
13286.7000000001	24.5521871330945\\
13287.7000000001	24.550339409082\\
13290.6999999999	24.552321906883\\
13291.7999999999	24.5540517157066\\
13295.2	24.5522853940866\\
13303.6000000001	24.549561400212\\
13308.1000000001	24.5465201905592\\
13311.1	24.5447442754973\\
13317.9	24.5419732692596\\
13319	24.5437004001772\\
13320.9000000001	24.5462052398469\\
13323.7000000001	24.5485522148336\\
13325.5999999999	24.5503050496409\\
13328.5	24.5479645274072\\
13329.2999999999	24.5502423214848\\
13331.6	24.5467569777298\\
13333.7	24.5443909463169\\
13338.6	24.5421217959771\\
13339.2999999999	24.5438325561869\\
13341.1000000001	24.5457679968818\\
13342.7	24.5435740624157\\
13344.5999999999	24.5453251103434\\
13347.7	24.5426212559373\\
13348.2999999999	24.5445146983908\\
13349.6999999999	24.542689778124\\
13355.2000000001	24.5385727014743\\
13357	24.5405065470798\\
13358.4	24.5386832353932\\
13359.1	24.540391640218\\
13368.2000000001	24.542387588549\\
13371.1000000001	24.5400562402776\\
13378.8	24.5378917549275\\
13381.9000000001	24.5404274398446\\
13386.5999999999	24.5422695660618\\
13389.1999999999	24.5389975577513\\
13389.7	24.5418154117313\\
13390.4	24.5442664575632\\
13394.2000000001	24.5417826986106\\
13397.8	24.5441449779443\\
13399.2	24.5460583761838\\
13404.5999999999	24.5481062612367\\
13406.1000000001	24.5498351509003\\
13409.1	24.5518002565403\\
13413	24.5483892612446\\
13414.1	24.5501036215354\\
13416.7999999999	24.5473991756665\\
13421.5	24.5447636645407\\
13424.5999999999	24.546544801746\\
13434.8999999999	24.544845552661\\
13437.8999999999	24.5423426105075\\
13441.6000000001	24.5385628305943\\
13446.1	24.5407624458955\\
13450.3999999999	24.5388647262184\\
13451	24.5407438796827\\
13451.7000000001	24.5431838118319\\
13454.4999999999	24.5410491579088\\
13455.2999999999	24.5440492293057\\
13459.5	24.5393622395911\\
13460.3999999999	24.5414360536384\\
13464.3000000001	24.5432399512789\\
13466.2	24.5449752344742\\
13471.1	24.5427281905101\\
13472.1	24.5446178055551\\
13473.5000000001	24.5465206032538\\
13474.8999999999	24.5484230055659\\
13475.9000000001	24.5466013653903\\
13480.7	24.543795620438\\
13481.4999999999	24.5460479468312\\
13482.9000000001	24.544240895943\\
13488.8999999999	24.5459263103269\\
13497.2000000001	24.5478725374705\\
13505.3	24.5457372606513\\
13507.9	24.5439739413681\\
13509	24.5456766179834\\
13511.4999999999	24.5477959679091\\
13513.6999999999	24.546019624384\\
13515.1	24.5479164200308\\
13517.8000000001	24.5459723773663\\
13523.1000000001	24.5437470421202\\
13526.6	24.5455284733157\\
13529.0000000001	24.5433916520685\\
13530	24.5452731317581\\
13532.7	24.5470264838023\\
13538.1	24.5490537885391\\
13540.4000000001	24.5507920682397\\
13543.7000000001	24.5485018975472\\
13544.8000000001	24.5509379914211\\
13556.6	24.5539106124647\\
13560.7000000001	24.5560733880007\\
13568	24.5583390452606\\
13570.6	24.5558445769194\\
13575.6000000001	24.5578496873089\\
13581.2	24.5558230802648\\
13585.6999999999	24.5535780005594\\
13590.2	24.555749321207\\
13595.8	24.5522547238506\\
13603	24.5539619645522\\
13606.3999999999	24.555910777937\\
13610.5000000001	24.5536567087417\\
13616.3	24.5556828530302\\
13617.3	24.5538795952238\\
13625.7	24.5519529128565\\
13627.9000000001	24.5553272673907\\
13635.4	24.5572219573906\\
13638.8999999999	24.5553193049344\\
13640.8	24.5570306944556\\
13652.7000000001	24.559064807219\\
13656.2	24.5608254065889\\
13667.3	24.5569749915858\\
13672.0000000001	24.5587729756219\\
13674.2000000001	24.5614035087719\\
13676.1	24.5594536494055\\
13686.1	24.5575835513145\\
13687.3000000001	24.5598141356284\\
13688.2000000001	24.5618520926631\\
13690.1	24.5635564126163\\
13695.1	24.5655412115194\\
13697.6000000001	24.5625177949583\\
13698.6	24.5643747216889\\
13702.7	24.5665119537613\\
13707.2	24.5642832651215\\
13712.3000000001	24.5617105685365\\
13713.2	24.563744685816\\
13713.8	24.5655867404604\\
13718.9999999999	24.5621068437434\\
13721.2	24.5596262744784\\
13725.0000000001	24.5564695339196\\
13730.1	24.5582730040349\\
13737.4	24.5561419472248\\
13740.5000000001	24.5535129470329\\
13745.3	24.5507587993074\\
13749.2000000001	24.5525226738816\\
13755.2000000001	24.554898838993\\
13757.8	24.5524389623416\\
13759.3000000001	24.5548497754263\\
13760.5999999999	24.5525300311757\\
13766.1	24.5543432465023\\
13768.2000000001	24.5505981130568\\
13772.5	24.54728954591\\
13785.1000000001	24.5495168731683\\
13786.9	24.5477623848553\\
13788.1999999999	24.5497994676646\\
13791.1000000001	24.5475375601833\\
13804.8	24.5492542503024\\
13807.5999999999	24.5464487206414\\
13809.6999999999	24.5485090298194\\
13817.5999999999	24.5467769599861\\
13818.9	24.5495332513206\\
13820.0000000001	24.5475792505119\\
13824.1000000001	24.5497026952735\\
13826.7000000001	24.5479792866028\\
13827.3	24.5505301068892\\
13830.9	24.5484780565397\\
13841.2	24.5504396263357\\
13845.5	24.5464263014965\\
13847.7	24.5446930198299\\
13849.9	24.5429602888087\\
13857.9	24.5403377110694\\
13859.7	24.5429226972972\\
13863.6	24.5410676803451\\
13866.2000000001	24.5393508001413\\
13872.4999999999	24.5375776710927\\
13876	24.5393158019905\\
13877.1	24.53737065114\\
13881.1	24.5396651586318\\
13883	24.5377473330884\\
13884.6	24.5399612523137\\
13894.2	24.5366805092736\\
13895.5	24.5394225510233\\
13898.6	24.5375466770274\\
13899.1	24.5395418441349\\
13907.4	24.5371202588531\\
13910.1	24.5388276228954\\
13915.6	24.537033710126\\
13921.1000000001	24.5388328592363\\
13930.3	24.5405731350141\\
13934.0999999999	24.5424925722323\\
13937.5999999999	24.5442217869519\\
13939.3999999999	24.542487176728\\
13941.2	24.5443394805363\\
13944.4000000001	24.5480296891247\\
13945.7999999999	24.5498677030525\\
13948.9000000001	24.5479962721342\\
13949.6	24.5503487530198\\
13951.8	24.5522115267455\\
13954.0000000001	24.5540737131022\\
13960.0000000001	24.5521163888511\\
13962.6999999999	24.5538144211763\\
13964.5000000001	24.5556621743552\\
13971.3	24.5537311937243\\
13979.8	24.5488165151396\\
13982.4000000001	24.5471124620061\\
13990.3000000001	24.5454025617566\\
13992.2	24.5434989244084\\
13994.8	24.5453700991075\\
13998.3	24.5470910961253\\
14004.1000000001	24.5490638522729\\
14006.4000000001	24.5464605718773\\
14007.9	24.5445459737293\\
14008.9	24.5463630523235\\
14009.8	24.5483550917565\\
14010.9999999999	24.5462526139989\\
14013.6	24.5481207675346\\
14019.7	24.5502788912823\\
14025.4999999999	24.5522473191878\\
14026.5000000001	24.5504969130081\\
14027.0000000001	24.5524734264388\\
14029.4	24.5496988488542\\
14032.3	24.5474758416237\\
14035.0999999999	24.5511285909713\\
14038.2	24.5535428078898\\
14039.5999999999	24.5510943966039\\
14044.3000000001	24.5535587137934\\
14046.3000000001	24.5557580590044\\
14050.2	24.5581944869505\\
14053.7000000001	24.5599055059841\\
14055.5	24.5567602948291\\
14060.2000000001	24.5549525970285\\
14062.8999999999	24.5530825570646\\
14066.0000000001	24.5547806428221\\
14071.6000000001	24.5570897617203\\
14077.9	24.5546242363972\\
14079.9999999999	24.5566437738368\\
14085.9	24.5548771830186\\
14086.4999999999	24.5573807732171\\
14088.6	24.5593986670168\\
14089.1999999999	24.5611918264215\\
14092.6	24.5630716612147\\
14095.8999999999	24.5651248581158\\
14108.9999999999	24.5607444840564\\
14109.5999999999	24.5625349936568\\
14112.6000000001	24.5643994416377\\
14117.7	24.5626089050702\\
14127.0000000001	24.559180582002\\
14129.4	24.5571322410559\\
14134.4	24.5548126923485\\
14135.6	24.5590950571956\\
14136.8	24.5570103770982\\
14143.7000000001	24.5542216377494\\
14149.6	24.5524640098377\\
14150.9999999999	24.5507416384592\\
14153.3000000001	24.5538174572894\\
14154.2	24.5557887002536\\
14165	24.5575393043466\\
14172.1	24.5593485838473\\
14176.1999999999	24.5564780655037\\
14179.3	24.5532251012032\\
14181.0999999999	24.555044707077\\
14184.8	24.5528696007727\\
14198.0999999999	24.5502951078306\\
14201.9000000001	24.5479509928179\\
14206.4000000001	24.5507338190265\\
14209.3000000001	24.546427013104\\
14210.6000000001	24.5484036676589\\
14215.6999999999	24.5501484263988\\
14228.8000000001	24.5472243110852\\
14231.5999999999	24.5445027649543\\
14233.8	24.5463295372315\\
14242	24.5483461006453\\
14247.0999999999	24.5500870346454\\
14250.9999999999	24.55178898471\\
14253.2000000001	24.5536121459592\\
14257.1999999999	24.5516331984317\\
14258.3000000001	24.553947146945\\
14264.4000000001	24.5560657576501\\
14268	24.5540751746904\\
14269	24.5558584633929\\
14272.0000000001	24.5577034914273\\
14273.5999999999	24.5556513027456\\
14277.9	24.5538590839053\\
14287.1000000001	24.5520465871549\\
14289.5	24.5556208711231\\
14299.3	24.553477768298\\
14305.4	24.5479011568977\\
14326.4	24.5461208250445\\
14329	24.5479478822815\\
14330.6	24.5452071427076\\
14334.7	24.5472556296565\\
14336.3	24.5445160570297\\
14341	24.5427477669077\\
14343.7000000001	24.5409166329703\\
14349.5000000001	24.538663098623\\
14351.6	24.5357692816879\\
14357.2999999999	24.5329934389235\\
14362.2	24.5350675031158\\
14368.4000000001	24.5377040052893\\
14371.3999999999	24.5395400619281\\
14372.6000000001	24.5416657969623\\
14383.4999999999	24.5439250257237\\
14386.2	24.5420990803751\\
14392.4	24.5384749001216\\
14397.1000000001	24.5401883699608\\
14400	24.5380240414997\\
14401	24.5397920992146\\
14415.5000000001	24.5414689641777\\
14419.5000000001	24.5388221587284\\
14426.6000000001	24.5406087324197\\
14428.1000000001	24.5380574153394\\
14435.6000000001	24.5398560513172\\
14437.7999999999	24.5416577203056\\
14446.4000000001	24.5436610943827\\
14450.1	24.5456810286363\\
14454.8000000001	24.5473853157061\\
14455.5999999999	24.5494856700125\\
14461.3	24.5474158795137\\
14471.2	24.5444431391789\\
14484.4	24.5427871172633\\
14485.1000000001	24.5450528815619\\
14488.3999999999	24.5429133450668\\
14491.3	24.5449024938929\\
14502.8	24.548193809514\\
14504.4	24.5461753248992\\
14510.9000000001	24.5434497967059\\
14513.1	24.5411074056721\\
14516.8	24.5431187099174\\
14521.3999999999	24.5449850222085\\
14524	24.5419681770299\\
14528.4000000001	24.544860102557\\
14531.7000000001	24.5468558609395\\
14533.6	24.5491512828805\\
14546.0000000001	24.546785736383\\
14555.5000000001	24.5486273324356\\
14556.4	24.5505444303232\\
14559.3	24.5525227687954\\
14563.9000000001	24.5502609173304\\
14567.2000000001	24.5522505886472\\
14569.4999999999	24.5504337799253\\
14586.7999999999	24.5521666700944\\
14590.6	24.5539967239406\\
14592.5999999999	24.5520020284115\\
14597.5	24.5492409711185\\
14598.9	24.5509966436057\\
14600.2	24.5529201454765\\
14602.9000000001	24.5511196329521\\
14604.3000000001	24.5528744761853\\
14609.2999999999	24.5547387298589\\
14612.6	24.5526151908956\\
14625.3999999999	24.5509555228881\\
14628	24.5534279913318\\
14629.4999999999	24.5509104828567\\
};
\addplot [color=mycolor1]
  table[row sep=crcr]{%
14629.4999999999	24.5509104828567\\
14637.1000000001	24.5490940890334\\
14644.6000000001	24.5508614037843\\
14646.7	24.5487068847803\\
14647.6999999999	24.5504444353418\\
14651.5999999999	24.5486871830573\\
14659.9	24.5457025920873\\
14660.7	24.5477736549165\\
14662.8000000001	24.5442579571572\\
14663.4	24.545981518737\\
14665.8	24.5440102550815\\
14666.2	24.5460681971595\\
14672.4000000001	24.5479638780031\\
14675.4000000001	24.5497598037546\\
14679.6	24.5515916537804\\
14681	24.553337283991\\
14685.6000000001	24.5551795283848\\
14692.7000000001	24.5582870521616\\
14693.7999999999	24.5564485943147\\
14697.8999999999	24.5584433256225\\
14700.7	24.5605681323465\\
14701.7000000001	24.562298494062\\
14704.4	24.5598286238906\\
14711.9	24.5629418162045\\
14714.7	24.5609862179574\\
14715.8999999999	24.563740146779\\
14728.5	24.5617370286382\\
14730.4000000001	24.5639998642273\\
14732.2000000001	24.5616774027138\\
14737.7000000001	24.5633676668159\\
14740	24.5656406672953\\
14744.6999999999	24.561879442244\\
14747.6999999999	24.5595953294729\\
14754.5000000001	24.5577650359888\\
14761.5000000001	24.5596683286365\\
14765.1999999999	24.557577563612\\
14766.2	24.5599777872588\\
14779.9999999999	24.5566674109106\\
14783.3999999999	24.5584604457672\\
14802.2000000001	24.5610479452518\\
14805	24.5631572903932\\
14809.3000000001	24.5654786824584\\
14810.2999999999	24.5671960244153\\
14811.4	24.5653715018735\\
14814.8	24.5671587388373\\
14820.9	24.5651440523581\\
14825.8000000001	24.567142635523\\
14830.4000000001	24.5649168942382\\
14838.1	24.566995996819\\
14840.8999999999	24.5690991173102\\
14849.9999999999	24.5668379337513\\
14858.3000000001	24.5685941958757\\
14865.5	24.5661123668066\\
14867.8	24.5643298650112\\
14870.9	24.5666061461906\\
14880.5	24.5648696961144\\
14885.5999999999	24.5631713658075\\
14889.5999999999	24.5653035319718\\
14897.4	24.562510488337\\
14902.3	24.5644996779042\\
14907.1	24.5626274551894\\
14916.1999999999	24.5644027004016\\
14917.1000000001	24.566272490816\\
14919.0000000001	24.5631438893767\\
14929.1000000001	24.561262492297\\
14933.7	24.563071689724\\
14935.2000000001	24.5606047417862\\
14945.3000000001	24.5627417131693\\
14951.9000000001	24.5605939004815\\
14952.8	24.563128222619\\
14954.8	24.5651926793225\\
14956.6000000001	24.5669164989603\\
14967.2	24.5688935212096\\
14968.3000000001	24.5670879987173\\
14969.3	24.5687869921306\\
14972.6999999999	24.570554605685\\
14985.8	24.5684276553293\\
14987.5999999999	24.5701475209672\\
14991.8999999999	24.5684364994664\\
14994.6000000001	24.5706816408464\\
14997.7000000001	24.568936777394\\
15000.5	24.5710171593136\\
15007.9000000001	24.5675639658849\\
15010.7999999999	24.565482416111\\
15013.7000000001	24.5673979938457\\
15015.4000000001	24.5692784123073\\
15017.7000000001	24.5668473411552\\
15018.3999999999	24.5690315277824\\
15025.7000000001	24.5670779592434\\
15030.8000000001	24.5693870626509\\
15038.7	24.5711093970264\\
15046.0999999999	24.5689941646396\\
15047.3000000001	24.5710222363996\\
15048.8000000001	24.5692376186964\\
15053.3	24.5665431065407\\
15056.0000000001	24.5647943358506\\
15059.7999999999	24.5665641870132\\
15070.4000000001	24.5685279187817\\
15081.2000000001	24.5668476855444\\
15086.2000000001	24.5686483763415\\
15103.5	24.5669906512355\\
15105.6999999999	24.5687087079135\\
15108.0000000001	24.5669541504226\\
15110.6999999999	24.563226301718\\
15116.3000000001	24.5600804424334\\
15118.9999999999	24.5623086030253\\
15123.4	24.5597910536582\\
15132.2000000001	24.5580645374464\\
15133.6000000001	24.5604181396486\\
15138.0000000001	24.5585641527008\\
15140.6000000001	24.5556678357011\\
15148.9999999999	24.5578945283878\\
15155.3	24.5602227588846\\
15171.2000000001	24.5582118869181\\
15182.1	24.5563884022079\\
15183.5000000001	24.5587344239838\\
15184.1	24.561056888081\\
15185.3000000001	24.563067156611\\
15201.7000000001	24.5648541620071\\
15203.4999999999	24.566550027625\\
15205.7999999999	24.5648070814618\\
15206.3999999999	24.5671259001085\\
15208.4	24.5652102442713\\
15212.1999999999	24.5623607212585\\
15213.6999999999	24.560596300727\\
15216.9	24.5633173424459\\
15219.0999999999	24.5610807401177\\
15229.4000000001	24.555632161266\\
15233.8	24.5537912156441\\
15236.6	24.5558421442963\\
15240.4	24.5575932548145\\
15243.8	24.5593319294931\\
15253.2	24.5612424852327\\
15259.9999999999	24.5594720873389\\
15262.5000000001	24.5574148572327\\
15265.1999999999	24.5556916667213\\
15268.3	24.5539807707422\\
15269.6	24.5571294786407\\
15270.7	24.5553605574037\\
15274.1	24.5570962800016\\
15281.5000000001	24.5589467071511\\
15288.8000000001	24.5616100569694\\
15290.7	24.5598660632537\\
15295.4999999999	24.5580428358482\\
15298.3000000001	24.5561627359724\\
15300.1000000001	24.5539273996418\\
15302.4	24.5561182813266\\
15307.8	24.5579080082833\\
15319.6000000001	24.554005626742\\
15322.1000000001	24.5571784730652\\
15325.0000000001	24.5603617594665\\
15327.7000000001	24.5586450762667\\
15328.4	24.561437844538\\
15329.7	24.5632689271876\\
15336.3000000001	24.5611747215774\\
15337.5	24.5631650323388\\
15340.9	24.5648914673098\\
15350.2999999999	24.5674379820721\\
15355.9	24.5649908830425\\
15357.3999999999	24.5632427152857\\
15359.8999999999	24.5657552083333\\
15362.5999999999	24.5627396225924\\
15371.9999999999	24.5646333292133\\
15376.5999999999	24.5670397419472\\
15381.6000000001	24.568805788697\\
15387.0000000001	24.5705818510311\\
15388.3	24.5724051883237\\
15391.2	24.5697244547244\\
15399.8	24.5715881271956\\
15401.9999999999	24.5693769031496\\
15409.9999999999	24.5721961570658\\
15413.3999999999	24.5739124793201\\
15421.4000000001	24.5721881788412\\
15428.7000000001	24.5702841439386\\
15433.3000000001	24.5720320862545\\
15446.5000000001	24.5691608509316\\
15452.9999999999	24.5672389358769\\
15455.6999999999	24.5655352683135\\
15459.6	24.5632192086522\\
15465.9000000001	24.5609724557093\\
15466.8000000001	24.5627759926036\\
15470.6	24.5651457270841\\
15476.4000000001	24.5669240461345\\
15483.1999999999	24.5651766742232\\
15487.4999999999	24.5680415299982\\
15490.8	24.5699087851577\\
15492.8	24.5719006770843\\
15494.6999999999	24.5701783824251\\
15504	24.5676949968073\\
15505.3000000001	24.5695048176764\\
15507.5999999999	24.571664398976\\
15515.5999999999	24.5693072178503\\
15524.4999999999	24.5668165363359\\
15526.5999999999	24.5686462673975\\
15531.5999999999	24.5665316739314\\
15532.9	24.56833837636\\
15534.9	24.5658191181204\\
15537.1999999999	24.5641134560059\\
15540.1	24.5659644020025\\
15542.8999999999	24.5641124622016\\
15543.8000000001	24.5665502222737\\
15549.2000000001	24.5683085412205\\
15557.5000000001	24.5661284516892\\
15574.3000000001	24.5685227039244\\
15587.7999999999	24.5652076289943\\
15591.1999999999	24.5669059026508\\
15596.0999999999	24.5649581308267\\
15596.8999999999	24.5669038917741\\
15602.8999999999	24.5689931423444\\
15607.6	24.5660795632925\\
15608.7999999999	24.5686755633004\\
15615.4	24.5704588389741\\
15631.2	24.56865391874\\
15641.7000000001	24.5668657059929\\
15653.8000000001	24.5651243460096\\
15656.9	24.5672861978668\\
15659.0000000001	24.5652687574637\\
15664.0000000001	24.5670035303656\\
15665.7000000001	24.5649759348389\\
15673.2999999999	24.5626347824977\\
15674.8000000001	24.5647500143542\\
15683.0000000001	24.5665716599397\\
15684.9	24.5648708957603\\
15685.4000000001	24.5666379777502\\
15695.2	24.5621300644142\\
15697.2000000001	24.5640970103139\\
15701.3	24.560867183818\\
15711.5	24.5589246162071\\
15716.1	24.5561904277115\\
15719.3999999999	24.5580330163173\\
15722.4999999999	24.5557350565428\\
15724	24.5578443281332\\
15726	24.5559929035171\\
15730.2000000001	24.5577007431518\\
15733.8000000001	24.5597086545612\\
15739.9000000001	24.5622617534943\\
15746.5000000001	24.558952408774\\
15755.9000000001	24.5608022340696\\
15760.7000000001	24.5634739353332\\
15762.0000000001	24.5652546297765\\
15764.6000000001	24.5624718516686\\
15766.9000000001	24.5645969429822\\
15770.6	24.5613701357581\\
15771.0999999999	24.5631277264888\\
15772.6999999999	24.5650740515318\\
15773.6000000001	24.5668422754332\\
15774.7999999999	24.5687769811536\\
15779.7000000001	24.5706536204515\\
15784.5000000001	24.5726847686986\\
15785.4	24.5744512368946\\
15786.2999999999	24.5762175036741\\
15788.4	24.5742154099503\\
15792.8	24.5768668199001\\
15793.9	24.5751551221983\\
15797.3000000001	24.5730310051021\\
15801.4000000001	24.5710850235737\\
15807.9000000001	24.5673076923077\\
15812	24.5691590617312\\
15813.7999999999	24.5669948589532\\
15815.1000000001	24.5687692852446\\
15820.5	24.5667041705119\\
15825.4	24.5692079239203\\
15827.1000000001	24.5665689446017\\
15845.5000000001	24.5645478871106\\
15853.6000000001	24.5665049799101\\
15856.7000000001	24.5686393219313\\
15857.9	24.5667801740447\\
15858.8	24.5685388015562\\
15861.2999999999	24.5709710366046\\
15871.4999999999	24.569041558507\\
15872.5999999999	24.5711189652674\\
15882.5	24.5690252225706\\
15886.5	24.5640980448932\\
15890.2000000001	24.5659301586502\\
15896.8000000001	24.5676830073788\\
15899.2	24.5658613901241\\
15905.1000000001	24.5630359882303\\
15908.2999999999	24.5650096804204\\
15911.0999999999	24.5632007642415\\
15913.3	24.5604333454824\\
15930.1000000001	24.555875004708\\
15944.4000000001	24.553921414908\\
15947.4999999999	24.5516566756126\\
15951.0999999999	24.5536385977231\\
15955.6000000001	24.5517275957808\\
15962.4	24.5494126859828\\
15965.3000000001	24.5512170067772\\
15967.8	24.549251936698\\
15970.1	24.5469687292582\\
15973.5	24.5423699103521\\
15974.3999999999	24.5441171867664\\
15980.2	24.5458470742101\\
15984.1999999999	24.5478375656113\\
15987.5	24.5502764642598\\
15990.7000000001	24.5522425394602\\
15994.3	24.5492172260291\\
15997.0000000001	24.5513249276431\\
16002.2999999999	24.5538169274609\\
16012.5000000001	24.5519153666488\\
16019.9	24.5536828963795\\
16021.4000000001	24.5513840776457\\
16025.0000000001	24.548988773861\\
16029.5000000001	24.550830962719\\
16030.7000000001	24.5527359832323\\
16032.1	24.5505919337334\\
16035.8	24.5524105288758\\
16043.1	24.5499650942455\\
16047.6999999999	24.5522750782039\\
16058.3000000001	24.5541274348628\\
16064.3	24.5561614501631\\
16077.0999999999	24.5527828228796\\
16080.8	24.5552176806025\\
16082.7	24.5529385430398\\
16084.6999999999	24.5548592459962\\
16088.6	24.5520147681292\\
16091.8	24.5539681454645\\
16095.8000000001	24.5522151603825\\
16107.4	24.5544001241658\\
16108.7000000001	24.5524185538339\\
16128.6999999999	24.5542135806756\\
16139.2000000001	24.5562075182939\\
16141.7000000001	24.5585994127049\\
16142.9	24.5567738338599\\
16144.6000000001	24.5585238499322\\
16150.0999999999	24.5563522433159\\
16153.8	24.5581562347173\\
16154.9999999999	24.5563320561309\\
16159.1	24.5531957027576\\
16162.7000000001	24.5551513351647\\
16166	24.5575618114449\\
16170.8	24.5595483244594\\
16174.4	24.5621193854524\\
16193.8999999999	24.5640360627393\\
16196.9000000001	24.566277705748\\
16199.8	24.5680528892154\\
16201.8999999999	24.570423404518\\
16207.1000000001	24.5668591737006\\
16211.4999999999	24.5651262059266\\
16225.6000000001	24.5628848062025\\
16227.0000000001	24.5656956572647\\
16229.7	24.5622250428225\\
16231.0000000001	24.563954383868\\
16232.8000000001	24.5618466201357\\
16234.8000000001	24.5600527259176\\
16242.2	24.5580982988862\\
16243.8	24.560604288379\\
16251.2999999999	24.557884243819\\
16252.0000000001	24.5599030279164\\
16262.7	24.5615761123546\\
16265.9999999999	24.5590522620665\\
16268.8	24.5572841433656\\
16270.8000000001	24.5554947790227\\
16276.7999999999	24.5581161031892\\
16278.3000000001	24.5558531551012\\
16279.1	24.5577178239717\\
16281.8	24.5554879958727\\
16291.8999999999	24.5531549226614\\
16306.1	24.5513976278961\\
16307.7999999999	24.5531306912601\\
16308.6999999999	24.5554547238301\\
16310.8000000001	24.5584241212931\\
16312.3	24.5561658615532\\
16314.6999999999	24.5580699732758\\
16316.7999999999	24.5561350501627\\
16325	24.5578893850574\\
16327.4999999999	24.5602538033759\\
16331.2000000001	24.5620373148494\\
16337.3	24.558987354169\\
16339	24.5607163185243\\
16341.4999999999	24.5587947324619\\
16344.0000000001	24.5605447837446\\
16360.5000000001	24.5583902790851\\
16364.2	24.5558930110057\\
16365	24.5577478903276\\
16371.3	24.5599032459044\\
16374.5999999999	24.558007169597\\
16379.1999999999	24.5602681433272\\
16380.4	24.5621318030585\\
16388.5999999999	24.5638763294221\\
16391.8	24.562131296555\\
16397.6	24.559541886972\\
16398.9	24.5612537349838\\
16399.7	24.5631044281028\\
16403.2000000001	24.5651789578926\\
16417.5000000001	24.5669281746419\\
16421.4	24.5641384769966\\
16427.3999999999	24.5618627301781\\
16432.5999999999	24.5589586616928\\
16443.9000000001	24.5609340792994\\
16444.8000000001	24.5626303595644\\
16449.2	24.5597077079268\\
16466	24.5613715451746\\
16468.9	24.5588681765742\\
16477.6	24.561073450785\\
16479.7	24.5634048956905\\
16482.2	24.5608925938734\\
16483.4999999999	24.5625955495159\\
16489.2	24.5644144990994\\
16490.3999999999	24.5626269670416\\
16496.5000000001	24.5644557060243\\
16500.1000000001	24.5627325729385\\
16513.6	24.5602136407952\\
16517.3999999999	24.5624337823521\\
16520.6999999999	24.5605539683308\\
16524.0000000001	24.5623059652265\\
16527.3	24.5604269274054\\
16530.0000000001	24.5576251807309\\
16532.8999999999	24.5557370108268\\
16542.6	24.5540328966855\\
16549.8000000001	24.5512057474668\\
16551.5000000001	24.5492882863288\\
16552.7	24.5511333429994\\
16553.9999999999	24.552829812554\\
16562.9000000001	24.5547304232325\\
16568.2	24.5571362179584\\
16575.9000000001	24.559000965251\\
16579.9	24.5572979493366\\
16585.2	24.5590975140637\\
16586.9	24.5565804545729\\
16590.0999999999	24.5584742799966\\
16592.5999999999	24.5553767620701\\
16601.9000000002	24.55728225515\\
16603.1	24.5555073720729\\
16606.7999999999	24.5572623427611\\
16609.5999999999	24.5555308042891\\
16616.1	24.5579615074445\\
16617.5000000002	24.5558925476603\\
16621.8999999999	24.5536036578029\\
16628.7999999999	24.5554426330064\\
16631.6000000001	24.55792252145\\
16636.8	24.5550553288173\\
16644.0000000001	24.5516429245198\\
16652.1000000001	24.5535124488056\\
16659.5	24.5552114096377\\
16664	24.553381220708\\
16664.7999999999	24.5552028515023\\
16670.0999999999	24.5509951890199\\
16671	24.5532688304911\\
16673.8000000001	24.5515446296307\\
16677.7000000001	24.5494010001319\\
16679.1999999999	24.5513900463449\\
16681.8	24.5493618832387\\
16687.4999999999	24.5511637383446\\
16690.3999999999	24.5492945088523\\
16694.2000000001	24.5461025619523\\
16698.6999999999	24.5442786307998\\
16701.5	24.5461512669445\\
16714.2	24.5478422668015\\
16716.3	24.5459548706659\\
16718.0999999999	24.5439102295702\\
16720.1000000001	24.5457590220213\\
16722.8000000001	24.5477758044358\\
16730.6000000001	24.5494808944037\\
16736.3	24.5470949547095\\
16751.7000000001	24.5448250337277\\
16757	24.546610093632\\
16765.2000000001	24.5483230243419\\
16767.2999999999	24.54644130873\\
16770.9000000001	24.5483274700376\\
16773.4	24.5500342802635\\
16778.4	24.5522543731561\\
16788.2000000001	24.5504309548912\\
16790.6	24.5487085112592\\
16793.1000000001	24.5462449086535\\
16796.5999999999	24.5441068781368\\
16810.3999999999	24.5418042294994\\
16818.5	24.5400925166185\\
16819.9000000001	24.5422116527943\\
16821.4	24.5441845257557\\
16824.5	24.54620020684\\
16826.9000000001	24.5444820823676\\
16828.0000000001	24.5464431516333\\
16830.2	24.5438286899223\\
16832.4000000001	24.5459676221595\\
16834.3000000002	24.5479494368674\\
16855.8000000001	24.5451147669362\\
16858.9	24.5429740791269\\
16863.9000000001	24.5410341555977\\
16865.4000000001	24.5430019863034\\
16871.1	24.5406372990659\\
16875.2	24.5423785058636\\
16880.5000000001	24.5441512742438\\
16882.5000000001	24.5459822539182\\
16887.2000000001	24.5427036885707\\
16888.3000000001	24.5452499940788\\
16890.9000000001	24.5432478834882\\
16892.9000000001	24.5450778428935\\
16901.6000000001	24.5472348935314\\
16906.2	24.5452878512744\\
16920	24.5435901679068\\
16924.0000000001	24.5401527998535\\
16931.6999999999	24.5384424573879\\
16933.1	24.5411381191978\\
16938.4000000002	24.539362989639\\
16964.3	24.5366768055457\\
16968.6	24.5339949436315\\
16973.8999999999	24.5357605749971\\
16975.1000000001	24.537560676752\\
16986.0000000001	24.5353553788097\\
16986.9000000001	24.5381762524283\\
17001.0000000001	24.5360594314486\\
17009.8999999999	24.5379188712522\\
17011.9	24.5397366564778\\
17022.5999999999	24.5372355736752\\
17023.6999999999	24.5391745673704\\
17035.2	24.5372843448604\\
17045.8000000001	24.5355188051086\\
17047.7	24.5374769765014\\
17050.7000000001	24.534332699932\\
17060.5	24.5302040959872\\
17062.9000000001	24.5273398581727\\
17072.1	24.5240800834105\\
17073.2999999999	24.5258706525941\\
17074.1	24.528235583512\\
17084.5999999999	24.5301351501636\\
17087.3000000001	24.5321113803153\\
17090.8	24.5341087947387\\
17100.4	24.5320312271571\\
17108.1	24.5338492652646\\
17115.4999999999	24.5314216270537\\
17122.8000000001	24.5297233529367\\
17124.6999999999	24.5316733626086\\
17132.0999999999	24.5292490164719\\
17135.4000000001	24.5262758600566\\
17139.1	24.5279826362957\\
17148.4000000001	24.529842260256\\
17151.7	24.528037873576\\
17157.1000000001	24.5261464574639\\
17160	24.5243326087843\\
17161.2000000001	24.5261139890335\\
17169.7999999999	24.5283897984263\\
17173.6000000001	24.526456150975\\
17176.4000000001	24.5282799173289\\
17178.2	24.5257097617343\\
17179.7	24.527642929487\\
17181.6000000001	24.529586711443\\
17185	24.5276431327138\\
17186.1999999999	24.5294216905326\\
17200.0000000001	24.5312527252749\\
17201.5999999999	24.5330403390363\\
17206.8	24.535506105109\\
17209.5000000001	24.5380485310524\\
17211.1000000001	24.5363484242819\\
17211.9000000001	24.5381129444574\\
17218.4999999999	24.5403226743173\\
17228.0000000001	24.5383994752759\\
17237.3	24.5361829510251\\
17248.3	24.5344495721342\\
17255.6000000002	24.5362401988908\\
17264.2999999999	24.5331433470031\\
17271.9000000001	24.5350856878184\\
17280.1	24.532702167799\\
17283.6	24.5346771813906\\
17285.8	24.5327116320238\\
17302.4999999999	24.5344630286778\\
17313.8	24.5323121884729\\
17314.1999999999	24.5340556649706\\
17326	24.5323529242011\\
17333.5	24.5344302395348\\
17335.7000000001	24.5318935382273\\
17340.1	24.5297055397285\\
17352.0999999999	24.5271492951902\\
17353.5000000001	24.5292043149548\\
17354.6999999999	24.5309654965773\\
17358.7	24.5328018065765\\
17364.2000000001	24.5307901844589\\
17365.9999999999	24.5328542390059\\
17368.2999999999	24.5307570069782\\
17369.9000000001	24.5325273459989\\
17375.2999999999	24.5306582870035\\
17376.7	24.5327102803738\\
17379.9	24.5304948216341\\
17381.5000000001	24.5322640033138\\
17394.9000000001	24.5288876113826\\
17403.3000000001	24.5308388016135\\
17409.4000000001	24.5325827852609\\
17415.9000000002	24.5343362425356\\
17428.6000000001	24.5365403042109\\
17436.2	24.5390363781307\\
17439.0000000001	24.536816693522\\
17446.3999999999	24.5350070214656\\
17448.3000000001	24.5369202906857\\
17453.5999999999	24.5386365068725\\
17460.5	24.5403938009003\\
17473.5999999999	24.5425983048811\\
17475.9000000001	24.5399404898146\\
17477	24.5418290219773\\
17487.7	24.5445396219078\\
17493.8000000001	24.5422690194868\\
17494.9999999999	24.5440151813936\\
17498.9999999999	24.5458337857376\\
17502.4000000001	24.5479217254678\\
17510.2999999999	24.5448419225146\\
17517.1999999999	24.5465910842424\\
17520.6	24.5441106805093\\
17522.4	24.5461549436439\\
17524.4	24.5439242203772\\
17525.8999999999	24.5458176423599\\
17528.4	24.5440282967738\\
17530.6	24.5460820161203\\
17535.5000000001	24.5443554825612\\
17537.9000000001	24.5461284068879\\
17540.0999999999	24.5441899180169\\
17541.3000000001	24.5459313395738\\
17548.9	24.5438486523449\\
17554.6	24.5404364643087\\
17562.2999999999	24.5422037990252\\
17567.6000000001	24.5439072843912\\
17572.3	24.545878764426\\
17575.9999999999	24.5424183977105\\
17580.4	24.5448081681408\\
17601.0999999999	24.5426448196714\\
17601.8000000001	24.5445093995535\\
17604.4000000001	24.5425885427022\\
17607.3	24.5447936662994\\
17613.7999999999	24.547090649998\\
17615.9	24.5441643960036\\
17617.4999999999	24.5459086368177\\
17622.3999999999	24.5441906653426\\
17627.7000000001	24.5458877454929\\
17630.7000000001	24.5439798534383\\
17640.6000000002	24.5421100069725\\
17654.3999999999	24.5438839955819\\
17656.6000000001	24.541958576631\\
17668.6000000001	24.5400057729205\\
17673.1999999999	24.537579286268\\
17677.4000000002	24.5396690708528\\
17682.2999999999	24.5351309776953\\
17684.5999999999	24.5370291834184\\
17686.9999999999	24.5387881563399\\
17688.2000000001	24.5405154819853\\
17694	24.5386880372554\\
17695.2000000001	24.5409798081977\\
17697.2000000001	24.5427268566391\\
17699.8000000001	24.5402516398398\\
17703.3	24.538224295898\\
17706.2999999999	24.5402792210726\\
17707.9000000001	24.5420149085159\\
17709.5000000001	24.5437502823327\\
17711.5000000001	24.541543395289\\
17727.8000000002	24.5432341112032\\
17738.2999999999	24.545054796374\\
17741.3	24.5471045126089\\
17745.9999999999	24.5451113202337\\
17751.2	24.5474979297291\\
17756.4999999999	24.5452395165741\\
17760.7	24.5473176883924\\
17762.9	24.5454033665484\\
17771.3	24.547869048021\\
17777.1000000001	24.5499853745247\\
17782.9	24.5481639768318\\
17803.9000000001	24.5501011008762\\
17820.2	24.5523363804201\\
17822.5	24.5502900811329\\
17829	24.5469485279683\\
17837.1000000001	24.5486959836746\\
17849.2999999999	24.5504050556321\\
17855.1	24.5485908866885\\
17862.1	24.546808343877\\
17864.1000000001	24.5490981963928\\
17873.6999999999	24.5471024628227\\
17874.8000000001	24.5489485255862\\
17881.4	24.5471576769287\\
17892.7999999999	24.5488433959839\\
17898.5999999999	24.5470341421444\\
17918.6	24.5447493400749\\
17924.4	24.542386119557\\
17926.9	24.5406370279467\\
17938.3000000001	24.5384203719395\\
17940.6999999999	24.5401542852047\\
17954.4	24.5381380712356\\
17957	24.5362558542304\\
17960.3999999999	24.5382923637983\\
17962.9000000001	24.5365473473251\\
17966.4	24.5384465533075\\
17970.3	24.5364599563727\\
17975.5000000001	24.5382629787045\\
17988.8	24.5401330820673\\
17995.4	24.5383568114251\\
18006.3999999999	24.53558437231\\
18007.9000000001	24.5374278098623\\
18010.7	24.5391653896551\\
18012.2000000001	24.541008088917\\
18018.8	24.5392338045053\\
18022.6999999999	24.5372528131034\\
18023.9999999999	24.539366736758\\
18038.4	24.5375169775757\\
18046.8999999999	24.5392586025378\\
18066.1000000002	24.5375341798497\\
18074.0999999999	24.5393987009107\\
18075.8000000001	24.5376440453864\\
18095.7000000001	24.5393958819174\\
18099.7	24.5411551508857\\
18101.8000000001	24.5394129897967\\
18104.1	24.5412666674031\\
18108.8999999999	24.5430448948037\\
18114.9999999999	24.5408526588316\\
18115.7999999999	24.5430809399478\\
18118.7	24.5413603549904\\
18133.5000000001	24.5395288304584\\
18134.4000000002	24.5416195649177\\
18151.3000000001	24.5435613781857\\
18153.3	24.5408573600538\\
18155.9000000001	24.5389953734303\\
18157.7	24.5409686195464\\
18160.0000000001	24.538961789858\\
18164.9	24.5367464905037\\
18179.1000000001	24.53848354163\\
18183.5000000001	24.5358454871423\\
18186.8000000001	24.5341427071134\\
18191.2	24.5315068191938\\
18193.9999999999	24.533227804618\\
18197.0000000001	24.530831835842\\
18203.7	24.5327898570628\\
18208.6	24.5349750394042\\
18219.0000000001	24.5368871129748\\
18222.3999999999	24.5345040471944\\
18230.7999999999	24.5314274117021\\
18238.4	24.5332675384489\\
18241.7999999999	24.530887681656\\
18242.8000000001	24.5328319510604\\
18243.8999999999	24.5346415259811\\
18245.2	24.5328934026845\\
18245.6999999999	24.534961470585\\
18256.1000000001	24.5368696661956\\
18262.8000000001	24.5349862289122\\
18270.1	24.5328458363893\\
18274.5000000001	24.5307694833266\\
18276.1	24.5329992011469\\
18277.6	24.5348156496715\\
18306.2	24.5319917186979\\
18309.7999999999	24.5337221940043\\
18319	24.5355939975217\\
18326.4000000001	24.5338717158214\\
18333.2	24.5356809739654\\
18338.8000000001	24.537458626199\\
18343	24.5356564593771\\
18352.7000000001	24.5374002876945\\
18355.3999999999	24.5348805535126\\
18360.6	24.5366462062993\\
18368.7	24.5345368233091\\
18376.6	24.5326962947646\\
18378.6	24.535467688139\\
18380.5	24.5372838753904\\
18385.7	24.5390464380119\\
18393.9000000002	24.5373491355877\\
18398.6000000001	24.5392337502106\\
18404.1000000001	24.5411373490834\\
18406.4	24.5391573628881\\
18410.1999999999	24.5411535933689\\
18414.5000000002	24.5430256426966\\
18416.0000000001	24.5410266017235\\
18427.2	24.5391348705453\\
18438.5000000001	24.5409087457833\\
18444.8000000001	24.53903246968\\
18446.2999999999	24.540831815422\\
18461.2000000001	24.5437753570984\\
18463.1999999999	24.5416583167689\\
18467.6000000001	24.5433919762613\\
18474.6	24.5416705007389\\
18479.8000000001	24.5396349547346\\
18484.4	24.5373150477427\\
18487.1999999999	24.5395487713187\\
18489.7999999999	24.5415064440586\\
18495.9999999999	24.5397678429507\\
18499.3000000001	24.5429581499941\\
18504.9000000001	24.54471764388\\
18509.4	24.5409114238634\\
18511.4999999999	24.5429892607878\\
18513.7	24.5449340491957\\
18525.6	24.5469806808919\\
18528.8000000001	24.5492177085526\\
18530.9	24.5469753386218\\
18534.1	24.5486721843943\\
18558.8	24.5504852119468\\
18559.8	24.5529340136531\\
18561.8	24.550827232126\\
18568.3000000001	24.5486956334418\\
18572.1000000001	24.5469034363188\\
18573.6000000001	24.5492282097805\\
18582.8999999999	24.5509336490341\\
18589.4	24.5488044326098\\
18598.5	24.545933564892\\
18601	24.5485482041385\\
18604	24.5467396971635\\
18607.7999999999	24.5444139317172\\
18617.4999999999	24.5461283946373\\
18620.7999999999	24.5439264482383\\
18624.5000000001	24.5417351245127\\
18627.8	24.5395347838457\\
18632.5000000001	24.5413951890772\\
18648	24.5392291976126\\
18654.1	24.5413901427025\\
18662.0999999999	24.5394433668056\\
18665.2000000002	24.5412610566131\\
18671.8000000002	24.5395487336586\\
18676.6	24.541273351288\\
18691.8	24.5395064172182\\
18694.3000000001	24.5372945908935\\
18696.1	24.5392111766027\\
18704.0999999999	24.5410121790828\\
18708.9	24.5384574269068\\
18711.8000000001	24.5410674490565\\
18723.6	24.5432259649535\\
18738.1000000002	24.5450470162556\\
18743.6	24.5431798417602\\
18750.3	24.5450763717041\\
18754.8999999999	24.5427885897094\\
18765.7000000001	24.5409201845911\\
18774.7999999999	24.5428737303528\\
18785.7000000002	24.5387473517231\\
18789.4	24.5413661885628\\
18796.4000000001	24.5433990370548\\
18808.8000000001	24.5452950464939\\
18822.8999999999	24.5470966370929\\
18825.2000000001	24.5488783711282\\
18832.3000000001	24.5507741976594\\
18834.9999999999	24.5525640957574\\
18837.7000000001	24.5495758528066\\
18850.6	24.5513429209525\\
18852.4000000001	24.5495292401538\\
18859.0000000001	24.5515427565472\\
18861.9999999999	24.5534696560828\\
18868.3	24.5516312988913\\
18869.8000000001	24.5533892601445\\
18884.9000000002	24.5507016150384\\
18890.1000000001	24.5524134207155\\
18896.9	24.5541620363021\\
18898.2999999999	24.5523430554968\\
18902.2000000002	24.5541547853965\\
18907	24.5563835807712\\
18909.9000000001	24.5542041248017\\
18914.8999999999	24.5524715833994\\
18924.9999999999	24.5541635182905\\
18931.6000000001	24.5514137663284\\
18934.6000000001	24.5533332981246\\
18939.5	24.5554288369343\\
18942.2000000002	24.5535125090406\\
18943.6000000001	24.5564488457904\\
18946.7	24.5582367471024\\
18950.8999999999	24.5564877842858\\
18957.0000000001	24.554388593192\\
18962.5999999999	24.5561022428241\\
18971.1999999999	24.553931464897\\
18975.5000000001	24.5520563249647\\
18977.3000000001	24.5502545132631\\
18988.9999999999	24.551979820002\\
18990.5000000001	24.5500405463756\\
18994.2999999999	24.554079096997\\
18997.7000000001	24.556001221194\\
18999.8	24.5532871225638\\
19001.8000000001	24.5554391929228\\
19007.4999999999	24.5528104547655\\
19010.1999999999	24.5545835678553\\
19019.7000000001	24.5565147898506\\
19023.2	24.5583048156734\\
19025.6999999999	24.5603338624394\\
19030.6000000001	24.5629430341501\\
19037.4000000001	24.564674983585\\
19045.9	24.5668381812454\\
19050.3000000001	24.5632637634905\\
19057.6000000001	24.561200984379\\
19066.6999999999	24.5594436402543\\
19070.2000000002	24.5612287169053\\
19086.2	24.5631683458816\\
19102.5	24.5610545161392\\
19104.4	24.5633227773561\\
19109.9999999999	24.5613576067106\\
19117.2999999999	24.5593019971335\\
19121.9999999999	24.5616328750503\\
19131.0000000001	24.5594869087507\\
19135.2	24.5614126770942\\
19151.2000000001	24.5596904648771\\
19155.2999999999	24.5570439667144\\
19156.4000000001	24.5587659541148\\
19160.3000000001	24.5605519717751\\
19170.5000000001	24.558438442198\\
19174.8	24.5565817813913\\
19176.9999999999	24.5584577438716\\
19179.1999999999	24.5566835077401\\
19189.6999999999	24.554190246902\\
19203.3999999999	24.5559403233786\\
19208.5000000001	24.5541059733661\\
19211.5999999999	24.552225987289\\
19215.4000000001	24.5499726783066\\
19219.6999999999	24.5481222489308\\
19228.1000000001	24.5462393775808\\
19242.5000000001	24.5445002234625\\
19248.6999999999	24.5464652341964\\
19253.4	24.5482639520087\\
19255.1999999999	24.5464884992703\\
19262.4000000001	24.5482154445166\\
19268.3999999999	24.546280198251\\
19275.9	24.5445113094003\\
19298.1	24.5463307458727\\
19301.1999999999	24.5480874345251\\
19314.4000000001	24.5499495197908\\
19321.3000000001	24.546357924374\\
19335.8000000001	24.5481203357485\\
19340.3999999999	24.546418138104\\
19343.6000000001	24.5444253167698\\
19349.6000000002	24.5461169940619\\
19351.1000000002	24.5488651866551\\
19356.9999999999	24.547065417857\\
19363.9	24.5450320181781\\
19374.7	24.5468340318352\\
19378.5999999999	24.5444740875291\\
19385.9999999999	24.5464533867049\\
19396.6999999999	24.544770271385\\
19398.9999999999	24.5464995798774\\
19401.2999999999	24.5482284783572\\
19412.7000000001	24.5461757191132\\
19421.0999999999	24.5479167095751\\
19423.4999999999	24.5500319199325\\
19426.6	24.5517766784889\\
19436.1000000001	24.5490373632706\\
19460.7000000002	24.5467812217381\\
19465.0000000001	24.5485509963987\\
19468.5	24.546705977831\\
19486.6000000001	24.5449460401197\\
19489.3	24.5471897544306\\
19499.3999999999	24.549860252827\\
19515.4999999999	24.5480538646006\\
19518.5000000001	24.5447931716414\\
19525.2	24.5466138804525\\
19528.2999999999	24.5488621699678\\
19536.3000000001	24.5469994471858\\
19539.8000000001	24.5487438523227\\
19548.1999999999	24.5468915455564\\
19561.5	24.5450269916571\\
19569.5000000001	24.5426580001635\\
19589.0000000001	24.5447723478873\\
19600.8999999999	24.5467068006734\\
19603.0999999999	24.5444621286321\\
19608.2	24.542668155832\\
19614.7999999999	24.544606396158\\
19620.2000000002	24.5465155986402\\
19623.3	24.5487530193544\\
19625.4	24.5512216249268\\
19629.1999999999	24.5530915519147\\
19643.8999999999	24.5510079413561\\
19646.9	24.5528579426885\\
19649.5	24.5546983144695\\
19653.9999999999	24.5567082695214\\
19656.6000000001	24.5539688757523\\
19666.3000000002	24.5520278241061\\
19670.9999999999	24.55378702767\\
19673.7	24.5519421769053\\
19675.9999999999	24.5536463018586\\
19680.0999999999	24.5515797603683\\
19681.9000000001	24.5533990448125\\
19692.6	24.5552920625409\\
19698.6000000001	24.5533969246702\\
19710.6000000002	24.5551908354346\\
19713.7	24.553358561008\\
19720.8	24.5516178267726\\
19728.4	24.549256152267\\
19735.6000000001	24.5473938091884\\
19748.4	24.545661695825\\
19753.5000000001	24.5479305038069\\
19756.9999999999	24.5461125367589\\
19761.5000000001	24.5440652578739\\
19772.9000000001	24.546603954888\\
19792.2	24.5484355026955\\
19795.0000000001	24.5464786740153\\
19796.3	24.5484027398921\\
19799.9999999999	24.5503810586815\\
19813.5999999999	24.5522037781939\\
19820.6999999999	24.5540038747175\\
19833.5000000001	24.5522749273959\\
19847.7	24.5498241618718\\
19851.3000000001	24.5478908288584\\
19862.9000000001	24.5496652066657\\
19865.5	24.547962306701\\
19868.0000000001	24.5499066342529\\
19871.1	24.5480896976529\\
19876.5999999999	24.5463281128155\\
19883.7000000002	24.5446041501121\\
19887.6	24.5428078661685\\
19891.9000000001	24.5445405188015\\
19899.6000000001	24.5420785237968\\
19903.9000000002	24.5438102893891\\
19906.7	24.5408604095083\\
19917.8000000001	24.5377273708574\\
19920.0999999999	24.5358982339535\\
19924.4000000002	24.5376295515571\\
19926.8	24.5356779027345\\
19927.6999999999	24.5380824777447\\
19933.4	24.5400958185968\\
19936.4999999999	24.5382863677859\\
19940.1	24.5413787223799\\
19944.6999999999	24.54323934058\\
19952.3000000001	24.5414085523546\\
19953.3	24.5431856225004\\
19963.1000000001	24.5451631001042\\
19970.7999999999	24.5472161995704\\
19973.7000000001	24.5491593988124\\
19975.0999999999	24.5509431695302\\
19980.2000000001	24.5491809432291\\
19993.6	24.5512336385962\\
19995.3999999999	24.5495236428196\\
19996.6000000001	24.551551005916\\
20003.4999999999	24.5490811653902\\
20015.6	24.55122728658\\
20017.9000000001	24.5534019382556\\
20025.8000000002	24.551705541324\\
20028.3	24.5496395118931\\
20037.0000000001	24.5514570471775\\
20043.4000000002	24.5481078653928\\
20046.0000000001	24.5459216505954\\
20047.0000000002	24.5476901896035\\
20049.3	24.5498618412521\\
20063.0000000001	24.5515398916419\\
20071.7999999999	24.553231134073\\
20075.7999999999	24.5513277113355\\
20088.4999999999	24.5532291946676\\
20092.4999999999	24.5553089197018\\
20097.0000000001	24.5527961745724\\
20098.3000000001	24.5546909206703\\
20103.9999999999	24.5571798787312\\
20116.0000000001	24.555455580356\\
20121.0999999999	24.5571834681828\\
20124.7	24.5552750834791\\
20129.2999999999	24.5571154629547\\
20141	24.5592345998977\\
20152.5	24.5610988160337\\
20155.1	24.5589227593872\\
20158.6999999999	24.5570172827748\\
20171.2000000001	24.5591508727747\\
20173.3	24.5570900294447\\
20176.5999999999	24.5550560795373\\
20179.3000000002	24.5532572821789\\
20183.3000000001	24.5513639921916\\
20187.7000000001	24.5534431686464\\
20190.7	24.555738257028\\
20194	24.5576678336742\\
20207.6	24.5549963627726\\
20209	24.5572539103671\\
20217.4999999999	24.5553379233935\\
20220.4999999999	24.5576293482884\\
20223.6000000002	24.5597986520765\\
20226.3000000002	24.5580034014951\\
20238.3000000001	24.5597478061507\\
20245.7000000002	24.5626253346373\\
20252.4000000001	24.5589433403284\\
20258.5999999999	24.560805974717\\
20261	24.5628322252987\\
20266.3999999999	24.5602348703526\\
20273.8999999999	24.5620005918911\\
20283.2999999999	24.5599850123747\\
20287.4999999999	24.5618012973442\\
20292.9000000001	24.559207608535\\
20304.4	24.5566253786107\\
20309.0999999999	24.5583282453272\\
20316.6000000001	24.5600909596539\\
20326.9999999999	24.5617918935805\\
20329.6	24.5645533382195\\
20334.0999999999	24.5664938871458\\
20339.6000000001	24.5647674252816\\
20342.6999999999	24.562498771064\\
20353.2000000001	24.5601450379054\\
20366.0999999999	24.5578458426216\\
20376.1	24.5595351439424\\
20380.7	24.5613518605747\\
20384.2999999999	24.5594670434254\\
20405.4000000001	24.5615152777437\\
20434.6000000002	24.559205664874\\
20437.4999999999	24.5571887110032\\
20449.2999999999	24.555243674631\\
20454.6000000001	24.5532811529868\\
20457.9	24.5507869781992\\
20476.3	24.5526557402668\\
20481.6	24.5506964753902\\
20484.2	24.5524621295334\\
20494.7000000001	24.550617717665\\
20508.3999999999	24.5488456006046\\
20516.0000000001	24.5509624148839\\
20516.9999999999	24.5526901950081\\
20525.7	24.5500784378684\\
20534.6999999999	24.5475972495471\\
20537.7	24.5493675077175\\
20550.0000000001	24.5512187288626\\
20556.6	24.5491737487048\\
20562.1000000001	24.5508749063816\\
20570.8999999999	24.5491225511643\\
20578.3999999999	24.5508661953009\\
20589.3999999999	24.5479491974064\\
20604.4000000001	24.5499769467835\\
20611.3	24.5475804651795\\
20613.5000000001	24.5493266581286\\
20616.1	24.5472007450452\\
20618.8999999999	24.5492021921529\\
20620.5000000001	24.5511769783614\\
20633.2999999999	24.553394011651\\
20635.5	24.550776328287\\
20655.5999999999	24.5525448181374\\
20658.5	24.5544228553726\\
20665.8999999999	24.5562760089035\\
20671.5000000001	24.5588149925502\\
20675.4000000001	24.5566008077193\\
20686.3999999999	24.5585285089309\\
20698.6000000001	24.5566146666216\\
20702.7	24.5536835597117\\
20708.6000000001	24.5553801059458\\
20712.1	24.5536447118124\\
20715.8000000002	24.5555346376455\\
20718.0000000001	24.5572711783416\\
20719.7	24.555256324868\\
20729.2999999999	24.5520854438623\\
20744.7	24.5502487370329\\
20746.5	24.55197478141\\
20752.3999999999	24.553668232743\\
20760.3000000002	24.5515500664727\\
20780.6999999999	24.5495842315984\\
20795.7	24.5477452177844\\
20802.6	24.5458522211059\\
20804.2000000001	24.547809827776\\
20810.1	24.549980298123\\
20811.0999999999	24.5516837087722\\
20814.0999999999	24.5534298699926\\
20822.3	24.5514445981251\\
20825.4999999999	24.553434234788\\
20827.6	24.5514387090269\\
20830.6	24.5541436437566\\
20832.4	24.5558622344894\\
20845.8	24.5539890338148\\
20848.9000000001	24.5522567029594\\
20862.4999999999	24.5544658863229\\
20865.4000000001	24.551532433922\\
20870.5999999999	24.5497276085613\\
20872.7	24.5515695067265\\
20888.5	24.5497544114972\\
20890.8999999999	24.5478914365038\\
20897.9000000002	24.5497176763327\\
20903.3	24.5515083670599\\
20913.5000000001	24.5495753959146\\
20914.3999999999	24.5513877931579\\
20922.1000000001	24.5495215608301\\
20932.8	24.5513044059829\\
20935.2999999999	24.5493279325926\\
20938.1999999999	24.551181327997\\
20960.4999999999	24.5484384988979\\
20961.3999999999	24.5502468811869\\
20967.8000000001	24.5484764807158\\
20977.1000000002	24.5466506492764\\
20981.7000000002	24.5484181528753\\
20995.3999999999	24.5462122835846\\
20998.7999999999	24.5431903575902\\
21004.6000000002	24.5449827895661\\
21007.5999999999	24.5429056964827\\
21011.0999999999	24.5411970758453\\
21013.0999999999	24.5431443092913\\
21018.9000000001	24.5449355345164\\
21022.9	24.5431194406127\\
21024.7	24.5448232563449\\
21027.8000000001	24.5431070149658\\
21042.2	24.5448453828716\\
21057.6000000002	24.543041262816\\
21061.8000000001	24.5447941543735\\
21080.1	24.541987267673\\
21090.1999999999	24.5401914624259\\
21099.1	24.5378971714567\\
21109.7	24.535997498792\\
21117.4000000002	24.5341541375636\\
21131.8000000001	24.5358912355254\\
21134.1	24.5341673685306\\
21145.4	24.5361897330401\\
21153.4	24.5344742004869\\
21175.2000000001	24.536134080745\\
21180.2	24.5383682006393\\
21182.8	24.5400771376913\\
21184.2999999999	24.5421158966032\\
21189.9999999999	24.5402334108853\\
21221.2	24.5423230433574\\
21233.1000000002	24.5403424825274\\
21236.1000000001	24.5420555466609\\
21246.3999999999	24.5438072153061\\
21253.4	24.5456042534171\\
21258.2	24.5428844263182\\
21261.3000000002	24.5449500032923\\
21269.1000000001	24.5430011471988\\
21272.1999999999	24.5413048894572\\
21291.3000000001	24.5385460796378\\
21294.5000000001	24.5404938341176\\
21297.6999999999	24.5386847467814\\
21299.1	24.5408278245192\\
21309.2000000002	24.5381124673265\\
21316.6	24.5399147147542\\
21321.6000000002	24.5421331319735\\
21326.1	24.5402368917107\\
21330.1	24.5421983853879\\
21339.2000000002	24.5439166233194\\
21344.4999999999	24.5457867563693\\
21353.6999999999	24.5483239517088\\
21371.6000000001	24.5502229583983\\
21378.7999999999	24.5485034309529\\
21386.6999999999	24.5464492116633\\
21410.8000000001	24.5440406521912\\
21411.6000000002	24.5459258256\\
21423.1999999999	24.548505599044\\
21430.3	24.5455054502016\\
21434.7000000002	24.543732621718\\
21444.1	24.5418341556225\\
21448.3000000001	24.5435556964622\\
21452.5	24.5452765632138\\
21460.0999999999	24.5477674951771\\
21468.4	24.5494561799846\\
21471.6	24.5513862432877\\
21489.6999999999	24.5530437696023\\
21492.1	24.5512325401774\\
21530.7000000001	24.5531982090772\\
21535.2	24.555032899472\\
21561.3	24.5526728320053\\
21564.1999999999	24.5507621392765\\
21573.1	24.548513896872\\
21577.7000000002	24.550232183077\\
21582.9999999999	24.5520800997076\\
21592.8999999999	24.5473069976381\\
21595.4000000002	24.5490958764557\\
21599.4999999999	24.5509176095854\\
21608.3000000002	24.5534144129135\\
21613.4	24.5513220903602\\
21623.4999999999	24.5495662146914\\
21642.7	24.5513519507642\\
21650.1000000001	24.5494267951335\\
21666.1000000001	24.5516057268926\\
21685.9000000001	24.5499400534907\\
21694.0000000001	24.5518366744875\\
21710.3000000001	24.5536701304444\\
21712.7	24.5518772337055\\
21716.6000000001	24.5539147292176\\
21728.0999999999	24.5556465791\\
21746.3000000001	24.5576279292205\\
21751.9000000001	24.55590290548\\
21756.9	24.5525577974905\\
21766.8000000001	24.5542543954353\\
21772.7000000001	24.5521935626102\\
21784.9	24.550378700941\\
21790.0999999999	24.5481913887895\\
21799.1000000001	24.5458548937576\\
21801.5000000002	24.5440701599883\\
21803.9000000001	24.5459548706659\\
21806.3000000002	24.5441705187468\\
21816.9	24.5459962414631\\
21821.9000000001	24.5440381266612\\
21826.3000000001	24.5459626873877\\
21835.8999999999	24.5479941381205\\
21844.7000000001	24.5500073243976\\
21849.6000000002	24.5481631326746\\
21860.7000000002	24.5498792358926\\
21864.0000000002	24.5516623140216\\
21866.6999999999	24.549545429601\\
21874.8	24.5468550713375\\
21876.3000000001	24.548828874952\\
21882.2	24.5463228271251\\
21887.0000000002	24.5482498823508\\
21897.7999999999	24.5461893606236\\
21903.8000000001	24.5444875113564\\
21908.1999999999	24.5427532031239\\
21911.0000000001	24.5446371930209\\
21922.1000000002	24.5427010062859\\
21929.6000000001	24.5406913911271\\
21938.3000000001	24.5373409182073\\
21942.2999999999	24.5392482135045\\
21945.6000000002	24.5373808992194\\
21951.8	24.5354616229119\\
21956.6999999999	24.5372731909932\\
21971.9	24.5389586746769\\
21974.3	24.5408293286734\\
21997.3999999999	24.5427889532901\\
22017.9000000001	24.544463620674\\
22026.9000000002	24.5462387070414\\
22033.8000000002	24.5480827270706\\
22045.8000000001	24.5460607187731\\
22057.9999999999	24.544271718779\\
22059.2999999999	24.5459985312384\\
22091.7000000001	24.5439484333554\\
22092.6000000002	24.5456644049845\\
22117.7000000001	24.5440324082865\\
22120.2000000001	24.5462312898107\\
22124.2000000002	24.548121296493\\
22136.5000000002	24.5498405355836\\
22153.4000000002	24.5473627192092\\
22157.3999999999	24.5492496897213\\
22186.8	24.547368041502\\
22194.5000000002	24.5492146738396\\
22197.7	24.5465766877799\\
22204.6000000001	24.548406418461\\
22207.3999999999	24.5466621636834\\
22222.6999999999	24.5446118400922\\
22227.7	24.542689784864\\
22235.8	24.5395958787366\\
22240.9999999999	24.5415019940561\\
22247.2000000001	24.5396070534402\\
22253	24.5417492394318\\
22262.6999999999	24.5436333255476\\
22273.3000000001	24.5418301651297\\
22276.9999999999	24.5435896054693\\
22280.0999999999	24.541072342259\\
22285.2000000001	24.5390459181613\\
22288.6999999999	24.5410250888339\\
22310.2999999999	24.5392283419392\\
22316.3	24.5353193167357\\
22324.6000000001	24.533364390115\\
22328.6	24.5352394004129\\
22344.5999999999	24.5328869933362\\
22347.7999999999	24.5347437566841\\
22353.9999999999	24.5328597438501\\
22360.9000000002	24.5306560529493\\
22363.3000000002	24.5324950588909\\
22368.6	24.5342822783622\\
22373.0000000002	24.5325860073034\\
22378.1	24.5301230661984\\
22400.0999999999	24.5252274533263\\
22408.2000000002	24.5230561890014\\
22409.5000000001	24.5247572468942\\
22416.8000000002	24.5221239332825\\
22430.4	24.5241969639553\\
22437.4	24.5223398328691\\
22443.5000000001	24.520130460354\\
22446.7	24.5184168790206\\
22449.8000000001	24.5163675562029\\
22458.6000000001	24.5183381050551\\
22469.7999999999	24.5203583460541\\
22476.2000000001	24.518270355886\\
22484.0000000002	24.5199941291846\\
22488.9000000001	24.5217661968073\\
22503.5999999999	24.5235227984731\\
22516.5000000001	24.5214641642166\\
22519.3000000002	24.5232999102996\\
22525.8999999999	24.5249933410281\\
22527.5999999999	24.5266938036284\\
22530.4	24.528527995384\\
22534.9000000002	24.5267361881518\\
22537.3000000002	24.5285614134727\\
22549.2	24.5302514933945\\
22552.1000000001	24.5319747075673\\
22564.9000000001	24.5295812098382\\
22568.4999999999	24.5314286220678\\
22573.6000000002	24.5334172067494\\
22582.0000000001	24.5353620788146\\
22589.8	24.5370718772549\\
22599.1000000002	24.5389217317427\\
22603.8999999999	24.5363652450894\\
22605.1	24.5381593615628\\
22610.2999999999	24.5400346743092\\
22612.2000000002	24.5379726962759\\
22620.4999999999	24.5360423684606\\
22630.6000000001	24.5379064721816\\
22638.4000000001	24.5360779203569\\
22640.5000000002	24.5377772673869\\
22655.3000000001	24.535430846509\\
22661.7	24.5373271320019\\
22664.1999999999	24.5394739744885\\
22668.5000000001	24.5414361716207\\
22670.7	24.5434656033312\\
22672.7999999999	24.5416334037551\\
22685.1999999999	24.5396798808039\\
22703.4999999999	24.5379587378213\\
22742.9	24.5398584179748\\
22754.3999999998	24.5380034718407\\
22761.6999999999	24.5362844766231\\
22767.6999999999	24.5342105956658\\
22771.2000000002	24.5361485729844\\
22782.1000000001	24.5380165216704\\
22790.9	24.5351235136677\\
22813.7000000001	24.5329581218385\\
22851.6	24.5347173295641\\
22856.3	24.5322972996622\\
22859.8999999999	24.5341207349081\\
22865.8	24.5317262823681\\
22875.1999999999	24.5334487416558\\
22899.0999999998	24.5353549468977\\
22906.4999999999	24.5335405516314\\
22912.4999999998	24.5354084652113\\
22915.5000000001	24.5335055595315\\
22928.0000000002	24.5353954318064\\
22941.0999999999	24.5375132948582\\
22953.9000000001	24.5399494641457\\
22971.8000000001	24.5382401978069\\
22982.0000000002	24.5399680621005\\
22987.5999999999	24.5374700383248\\
22992.8000000002	24.5397492269353\\
22997.5999999998	24.535931854055\\
23007.8999999999	24.5340751043115\\
23013.1999999999	24.5362464314114\\
23018.9000000001	24.537990355793\\
23023.3	24.5359069468454\\
23029.2999999999	24.5334224947241\\
23031.2	24.536174684017\\
23061.0000000001	24.5378581247209\\
23065.5	24.5361057158713\\
23078.0000000002	24.5340820951465\\
23122.1999999999	24.535621456343\\
23125.7	24.5336377552344\\
23129.1	24.5356519032219\\
23142.9999999999	24.5334462539591\\
23146.1000000001	24.5353448946263\\
23169.3000000002	24.5371049746649\\
23175.4000000001	24.5345299993528\\
23183.1999999999	24.5366276587026\\
23199.2000000001	24.5347919980344\\
23201.5	24.5366698848355\\
23209.5999999999	24.534569598056\\
23217.6999999999	24.5363471129909\\
23222.9999999999	24.5380676998334\\
23227.5000000002	24.53589695018\\
23231.4	24.5382347244044\\
23233.4000000001	24.5399961262832\\
23253.7999999999	24.5373894271499\\
23261.5000000002	24.5395845513636\\
23266.9000000001	24.5377573387201\\
23273.3999999999	24.53949771199\\
23292.4999999999	24.5412706181362\\
23296.6000000002	24.5395270574802\\
23303.2000000001	24.5377264164303\\
23314.9999999999	24.5360302979614\\
23321.8999999998	24.5339164737158\\
23333.2999999999	24.5356441838738\\
23336.8	24.533678423441\\
23353.6	24.5310164984564\\
23365.3	24.5328562746625\\
23369.5	24.5310146515131\\
23377.7000000002	24.5292542497583\\
23380.2	24.5313362103994\\
23394.0999999999	24.5295842559267\\
23412.3000000001	24.5314448753652\\
23426.7000000002	24.5295985794048\\
23445.1	24.5274085953628\\
23451.0999999999	24.5292351777308\\
23462.3999999999	24.5310602024507\\
23472	24.5291218084449\\
23484.1	24.5309612420266\\
23486.5999999999	24.5292016332648\\
23490.6000000002	24.5309846024171\\
23494.8000000002	24.5287275110769\\
23501.7000000002	24.5304614965662\\
23509.2000000002	24.5281654494179\\
23513.8	24.5254934315447\\
23540.5999999998	24.5226352657313\\
23541.7999999999	24.5243586966218\\
23569.0999999998	24.5226821444937\\
23571.5000000001	24.5244277011319\\
23575.1000000001	24.5261970205979\\
23578.9	24.5243649009712\\
23586.1	24.5262060018146\\
23597.1	24.5283338701202\\
23601.8999999998	24.5301245657148\\
23610	24.5280621428965\\
23615.1999999999	24.5298598789768\\
23626.6999999999	24.5280782839826\\
23631.6000000001	24.526377704524\\
23634.4	24.5281262561087\\
23646.4999999999	24.5299535662632\\
23647.6999999999	24.53166890789\\
23652.5999999999	24.5299690944374\\
23681.7999999998	24.527170539526\\
23716.0000000001	24.5255332875135\\
23719.1	24.5278086950656\\
23740.7999999999	24.5247652784857\\
23746.1999999999	24.5267683807583\\
23751.5000000001	24.5242425773422\\
23768.2000000002	24.5208954784314\\
23774.3000000002	24.5188101487314\\
23783.4999999998	24.516053078592\\
23797.7000000001	24.5178125709099\\
23819.0999999999	24.5205548465104\\
23835.4	24.5222462293638\\
23850.3999999999	24.5240141716107\\
23856.3999999999	24.5220380189885\\
23880.4999999999	24.5199031850121\\
23892.1000000001	24.5180435455923\\
23900.8000000001	24.5162316063412\\
23915.7999999999	24.5179984863626\\
23916.8999999998	24.5197976334825\\
23918.8000000002	24.5216126159648\\
23954.1999999999	24.5200235448333\\
23957.7	24.5181110118625\\
23968.5999999999	24.5198946959994\\
23975.7	24.521809491237\\
23979.8999999999	24.5237698081735\\
23993.6999999998	24.5213346781252\\
24001.5000000001	24.5196153589761\\
24002.2000000002	24.5218166592368\\
24008.3000000002	24.5197514203362\\
24009.0000000001	24.5215355844242\\
24015.1999999999	24.5197853035357\\
24057.7000000001	24.5213610554581\\
24060.0999999999	24.5193306788805\\
24065.3000000002	24.5215122125541\\
24071.4	24.5194524645327\\
24080.5999999999	24.5212971383722\\
24099.7000000002	24.5230250873451\\
24107.1	24.520890024557\\
24110.8999999999	24.5190991663556\\
24116.9	24.5208773893934\\
24123.8999999999	24.5183220029846\\
24125.6999999999	24.5202231635842\\
24139.9	24.5223695111848\\
24141.9	24.5240659431696\\
24147.8000000002	24.5222151822726\\
24172.7000000002	24.5192944135557\\
24194.3	24.521376847535\\
24201.0000000001	24.5232654714042\\
24203.3000000002	24.5209350752374\\
24227.3999999998	24.5192446599938\\
24249.6	24.5215404726656\\
24258.8999999998	24.5191475328744\\
24266.9	24.5209543824947\\
24270.9000000001	24.5226813893123\\
24278.8000000001	24.5200565099737\\
24313.7000000001	24.5181748636577\\
24317.3000000002	24.5198911067795\\
24320.3	24.5176888538018\\
24322.5000000001	24.5195826104117\\
24334.8999999999	24.5177727552907\\
24340.5000000001	24.5158295194038\\
24345.6999999999	24.5138791906612\\
24347.2000000001	24.5156547132536\\
24357.3000000001	24.5132895957697\\
24369.9000000001	24.514977431268\\
24378.0000000001	24.5121646067577\\
24383.3999999998	24.5141181536695\\
24386.5000000001	24.5159226788482\\
24395.7	24.514055698112\\
24407.2	24.5123385216718\\
24418.2000000001	24.5103057952437\\
24440.1999999999	24.512383235885\\
24448.2999999999	24.5140786309125\\
24453.2000000001	24.5116201085334\\
24455.9	24.5134118416748\\
24468.9	24.5097878948874\\
24480.0000000001	24.5117462755463\\
24486.4	24.5135074428767\\
24490.3	24.511645379414\\
24496.5999999999	24.5098319365466\\
24505.5	24.5078675894489\\
24534.4	24.5095681591229\\
24540	24.5076425931435\\
24543.3999999999	24.5095442785259\\
24551.7000000002	24.5114411163336\\
24567.8	24.5096243472173\\
24586.3999999999	24.511418868078\\
24596.7999999998	24.5132516699259\\
24598.7000000001	24.5150169926988\\
24604.5000000002	24.5133023906099\\
24622.3000000002	24.5150757034245\\
24628.5000000002	24.5133706341408\\
24631.2000000001	24.5151494237007\\
24649.5999999999	24.512671553812\\
24667.6000000001	24.5110002148559\\
24674.5	24.5130620151897\\
24683.1999999999	24.514955455713\\
24689.7999999999	24.5169077234011\\
24693.9	24.5148619097757\\
24702.3000000001	24.5130027851545\\
24712.4999999998	24.5150247242297\\
24718.7999999998	24.5168676599687\\
24735.9999999999	24.5151822639786\\
24752.9000000001	24.5133923160829\\
24783.9000000002	24.5117817947063\\
24785.3000000002	24.5136249566277\\
24796.9000000001	24.5118361092068\\
24799.7000000001	24.5139073702207\\
24814.2999999999	24.5156038429299\\
24819.1	24.5177120938628\\
24820.2000000002	24.5194457762396\\
24841.3000000001	24.517136715322\\
24880.9000000001	24.5187090551023\\
24883.9000000001	24.5205754701816\\
24908.0999999999	24.5224464232662\\
24917.8000000001	24.5201240875034\\
24922.7000000001	24.5221243198999\\
24930.3	24.5238744665148\\
24932.2	24.5256153664122\\
24935.3	24.5237694201818\\
24946.0000000002	24.5220695820188\\
24981.3000000002	24.5202430608373\\
25007.3	24.5219415053144\\
25011.4999999998	24.5198228022198\\
25014.8	24.5221847778724\\
25018.7999999999	24.5246593575257\\
25024.0000000001	24.5263565922451\\
25045.0000000001	24.5241584182135\\
25046	24.5259741037527\\
25059.7999999999	24.5236413553127\\
25085.2000000002	24.5215325310042\\
25092.2999999999	24.5193763848815\\
25094.8999999998	24.5216178521618\\
25115.8000000001	24.5199256248034\\
25120.2000000001	24.5176212067531\\
25126.1000000001	24.5194259378657\\
25133.9000000002	24.517386806716\\
25135.4	24.5191064430785\\
25161.1000000002	24.5171136511772\\
25205.2000000002	24.5150821454218\\
25219	24.5127700829927\\
25229.8999999999	24.5144669044788\\
25269	24.5121512044355\\
25284.2000000001	24.5140264907472\\
25318.1	24.51240609522\\
25328.9	24.510639977891\\
25329.9	24.512435846822\\
25332.7	24.5105160108634\\
25343.8999999999	24.5123106060606\\
25353.6000000001	24.510426486075\\
25362.6000000001	24.5123744711722\\
25363.7000000002	24.5144654980721\\
25386.8000000001	24.5161874825205\\
25388.8000000001	24.5142562300848\\
25394.3999999999	24.5123944161137\\
25415.8000000002	24.5141820671312\\
25422	24.5160706629271\\
25434.4000000002	24.5143407576324\\
25450.9000000001	24.5118070016895\\
25456.7999999999	24.5135896358158\\
25467.2000000002	24.5118249677037\\
25475.5000000002	24.5101194868816\\
25481.1000000002	24.5074800244886\\
25493.5999999999	24.5095847209311\\
25503	24.5076088789206\\
25510.1000000001	24.5094119215059\\
25517.1	24.5073910930666\\
25522.7999999999	24.5093621806299\\
25539.2	24.5112434561636\\
25552.4999999999	24.5094432660473\\
25558.7999999998	24.5073144775401\\
25562.1999999999	24.5091404138125\\
25583.3000000002	24.5072977008529\\
25590.8000000001	24.509102845152\\
25599.2	24.5069201110968\\
25605.1	24.5051786355896\\
25615.1000000002	24.5069333833037\\
25632.5000000002	24.5086335369803\\
25645.1000000002	24.5063403677881\\
25654.7	24.5080842571371\\
25658.7000000002	24.5062122936381\\
25660.7999999999	24.5081037687688\\
25699.1000000001	24.5062103100486\\
25713	24.5042410288919\\
25717.3000000001	24.5063653401977\\
25719.6000000001	24.5080619136304\\
25736.6999999999	24.5100400982251\\
25743.6	24.5069667530308\\
25776.9999999999	24.5085754409922\\
25804.9999999999	24.5063960224917\\
25820.8000000001	24.5041807218184\\
25833.1	24.5060619667714\\
25848.9999999999	24.503754482748\\
25853.1999999999	24.5055756905308\\
25866.5000000001	24.5034136685919\\
25873.4999999999	24.5052872425948\\
25876.4	24.5071783278264\\
25885.8000000002	24.5090956852959\\
25897.8999999999	24.5072978608387\\
25913.1000000002	24.5044996372505\\
25915.2000000002	24.5063726833183\\
25919.9000000002	24.5081018518519\\
25928.2000000002	24.5098984507276\\
25933.6000000002	24.5117356952537\\
25943	24.5140326329544\\
25959.0999999999	24.512311627477\\
25980.5999999999	24.5101171254046\\
25987.5999999999	24.5081326935435\\
25991.9000000001	24.5098491843644\\
26001.9000000001	24.508114760403\\
26024.1999999999	24.5097850854778\\
26048.7999999999	24.5081366199724\\
26056.2000000001	24.505397926797\\
26061.2999999998	24.5071254805958\\
26069.1	24.5089991254047\\
26074.1000000002	24.5108191238849\\
26078.1999999999	24.5127174700805\\
26090.7999999999	24.5108447772978\\
26132.6999999999	24.5090461029817\\
26148.6	24.5113523808067\\
26155.2000000001	24.5131961782124\\
26166.7000000001	24.5150343182965\\
26192.1999999998	24.5167472883252\\
26197.2	24.5185572559004\\
26231.6	24.5207134878792\\
26237.8999999999	24.5224483573443\\
26243.6999999999	24.5242685891525\\
26248.8	24.5263611046558\\
26254.3	24.5246511061003\\
26266.0000000001	24.5224833530673\\
26271.6	24.5206819505399\\
26276.6999999999	24.5185867381112\\
26291.3000000002	24.5167621351469\\
26299.8000000001	24.5187244057962\\
26327.7999999998	24.5207555482967\\
26342.4000000001	24.518933282718\\
26355.9	24.5169980270147\\
26376.3999999998	24.5195533903285\\
26382.9999999998	24.5213792162407\\
26389.0999999998	24.5195004016795\\
26393.7000000001	24.5175003220453\\
26394.6999999999	24.5192234834134\\
26403.5000000001	24.5212774015665\\
26419.0000000001	24.5190032968572\\
26426.2000000001	24.517242292716\\
26433.3000000002	24.5189797755869\\
26460.4000000002	24.5169214489522\\
26474.0999999999	24.5151883720755\\
26489.8	24.5168913434932\\
26499.3000000001	24.5145173098259\\
26516.3999999999	24.5168102879339\\
26534.4000000002	24.5148768584296\\
26542.8000000001	24.5169141276952\\
26546.7000000002	24.5151958051441\\
26554.0000000001	24.5171178838673\\
26570.0999999999	24.5146818616345\\
26596.0999999999	24.5166602747761\\
26611.1000000002	24.5189995189995\\
26638.7000000001	24.5209994444194\\
26642.4000000001	24.5190954302337\\
26652.8000000001	24.520783854665\\
26670.3000000002	24.5185674005639\\
26672.3000000002	24.5204780972091\\
26693.7	24.5188021188441\\
26699.9999999999	24.5205074138299\\
26721.6999999999	24.5223001444513\\
26729.1000000002	24.5241159481017\\
26745.6	24.5216988151366\\
26761.7000000002	24.5233878139736\\
26764.3000000001	24.5251154518689\\
26800.1999999999	24.5224866885818\\
26812.0000000002	24.5202725635068\\
26833.0000000001	24.5219523648032\\
26840	24.5237536372815\\
26846.5000000002	24.521913389405\\
26859.4000000001	24.5201883877213\\
26863.8000000002	24.5184057415342\\
26890.5	24.5200925230378\\
26892.4	24.5220786464628\\
26901.3999999999	24.523911306061\\
26906.2	24.5221379379551\\
26907.9999999998	24.5238422631104\\
26922.3999999999	24.5218683257498\\
26925.7000000001	24.5236910323927\\
26930.9000000002	24.5256395974899\\
26937.9000000001	24.527433365506\\
26948.9999999999	24.5247522180704\\
26974.8	24.522797118803\\
26985.4000000002	24.5246521279947\\
27006.2000000002	24.5264993723687\\
27022.4999999999	24.5246571388393\\
27043.6	24.5225320499784\\
27052.5000000002	24.5248146204062\\
27064.4	24.5265938775887\\
27081.9999999999	24.5283785230835\\
27087.5	24.5263515409265\\
27109.9999999998	24.5281278932944\\
27115.6000000002	24.5263813952802\\
27117.8	24.5280792391741\\
27126.8999999999	24.5261178899252\\
27138.6	24.5280724574132\\
27144.7000000002	24.525876042557\\
27157.9000000001	24.5238971941969\\
27165.2000000002	24.5220925224459\\
27172.7000000002	24.5197403285639\\
27174.0000000001	24.5215112920023\\
27187.8999999998	24.5192732087686\\
27202.5000000002	24.5175093557233\\
27204.2	24.519285554122\\
27214.8999999999	24.5166268601874\\
27227.2000000002	24.5147333742237\\
27241.2000000001	24.5124131374053\\
27258.6000000002	24.5143752269918\\
27271.5000000002	24.5119464937884\\
27273.9000000002	24.510156192711\\
27280.5000000002	24.5119242245405\\
27291.9999999999	24.5136871109222\\
27296.6000000001	24.5117541680863\\
27305.7000000002	24.5134733280109\\
27313.1	24.5115914649327\\
27317.1000000001	24.509466563191\\
27331.3	24.5077090818619\\
27336.2999999999	24.509811094365\\
27342.8000000002	24.507641837552\\
27356.4999999999	24.5056037665499\\
27371.2999999999	24.5073324711195\\
27376.8000000002	24.5053311368343\\
27423.6999999998	24.5035334271691\\
27432.0000000002	24.5052329205566\\
27435.6999999999	24.5070309595492\\
27445.9000000001	24.5088537491802\\
27453.4000000002	24.5065292221393\\
27458.6	24.5044375735195\\
27466.9	24.5061346342884\\
27475.5000000001	24.5079270334406\\
27484.1	24.5049883205624\\
27492.6000000002	24.5068690961601\\
27508.7000000002	24.5088844297098\\
27514	24.51070542013\\
27525.6000000001	24.5087318396989\\
27528.9999999999	24.5107903999768\\
27559.8000000002	24.5091600477505\\
27567.6000000002	24.5109312710164\\
27576.8	24.5132701645218\\
27582.2000000002	24.5149969364411\\
27592.2000000001	24.5169848109799\\
27607.2999999999	24.5187884407804\\
27642.2999999999	24.5170462767343\\
27653.5999999999	24.5153451436878\\
27685.7000000001	24.5136495965441\\
27712.7000000002	24.5153863918478\\
27728.8999999998	24.5172923653936\\
27734.3000000001	24.5190088842737\\
27748.4	24.5166405391282\\
27752.2000000001	24.5147249056835\\
27756.8000000002	24.5164265461921\\
27762.3999999998	24.5147230977037\\
27764.2	24.5167355200743\\
27770.0999999999	24.5144075303743\\
27810.7000000002	24.5127792080774\\
27813.2999999998	24.5148022176361\\
27817.3000000001	24.5130745504612\\
27827.4999999999	24.5148701289367\\
27832.8	24.5166691217947\\
27841.5000000002	24.5147548991437\\
27876.0000000002	24.5131133838665\\
27878.1000000001	24.5148539001801\\
27905.1999999999	24.5172064088184\\
27915.5	24.5153247646477\\
27921.6	24.5171318365286\\
27930.6000000001	24.5188985596494\\
27943.1	24.5218872570071\\
27960	24.519940915805\\
27966.9999999999	24.5216701052308\\
27985.6999999999	24.5235083506636\\
27989	24.525261619702\\
28007.8000000001	24.527008451187\\
28009.6	24.5290024527217\\
28021.0000000002	24.527231265011\\
28058.0000000001	24.5255380799128\\
28069.9	24.5276095475597\\
28082.5	24.5258629898941\\
28093.0000000002	24.5238154564644\\
28097.8999999999	24.5220300377251\\
28106.9	24.5198704948945\\
28161.2000000001	24.5215952388562\\
28165.3999999999	24.5197138343008\\
28190.2000000001	24.5172985033859\\
28197.4999999999	24.5191080091923\\
28219.0000000001	24.5209804706741\\
28234.1	24.5227419229162\\
28236.2	24.5244596494583\\
28251.6000000002	24.5227720809721\\
28260.9000000002	24.520717596688\\
28275.2000000001	24.5189264128055\\
28299.1999999999	24.5172142067118\\
28303.4999999998	24.5191424412442\\
28314.1999999999	24.5211783445114\\
28333.8000000003	24.5229213062798\\
28339.7999999999	24.5247866082802\\
28369.6999999999	24.5267855254531\\
28374.5999999998	24.5285412709209\\
28379.1	24.530994531206\\
28387.6999999998	24.5327218030281\\
28401.5000000002	24.5345332657315\\
28416.3	24.5326642361453\\
28422.4	24.5344357463277\\
28445.0999999999	24.5324342947141\\
28460.5	24.5342684272292\\
28481.0000000001	24.5320581016885\\
28500.2999999998	24.5301820325329\\
28504.1999999998	24.5320881410875\\
28523.8000000001	24.5299555811092\\
28547.1000000002	24.5316528416097\\
28553.7	24.5287842598884\\
28555.9999999998	24.5306606994653\\
28569.4	24.532455940776\\
28576.1999999998	24.5343168989687\\
28585.7	24.5324601725332\\
28601.7999999999	24.5305381810299\\
28618.5	24.5288029463356\\
28622.5999999998	24.5305299639796\\
28626.5000000002	24.5285853017823\\
28642.3999999999	24.530679933665\\
28650.8999999999	24.5282887159262\\
28654.9999999999	24.5300138544273\\
28684.3999999999	24.5317157349788\\
28687.7	24.5295909759549\\
28691.3000000002	24.5275587806799\\
28702.8	24.5257447853701\\
28713.7000000001	24.5237481629042\\
28731.0000000001	24.5218595876942\\
28755.3000000001	24.519916259207\\
28757.3000000001	24.5216883306558\\
28764.4	24.5239792104852\\
28791.8	24.5256478384546\\
28803.3	24.5238409354451\\
28809.8	24.5214318689062\\
28814.5000000002	24.5191673665433\\
28830.8000000001	24.5209133256333\\
28834.9	24.5226287497833\\
28841.5000000001	24.5246449572839\\
28844.8	24.5263460785095\\
28879.4999999998	24.5280405545783\\
28890.8999999999	24.529784361912\\
28894.4	24.531658274066\\
28901.3999999998	24.5298686919364\\
28911.8000000001	24.5272707777766\\
28932.2999999998	24.5250998880148\\
28939.2000000002	24.5230534256184\\
28955.5	24.5247896779898\\
28965.8999999998	24.5228889042326\\
28976.9000000001	24.5208268626842\\
28989.5000000001	24.5191378977288\\
29019.9000000001	24.5210199862164\\
29030.1000000001	24.522738389677\\
29052.1000000001	24.5244766317181\\
29058.0999999999	24.5225099971781\\
29066.7999999999	24.5206747193543\\
29074.8000000002	24.522526302756\\
29079.3	24.5207947894386\\
29094.2000000001	24.5188920166493\\
29106.7	24.5166078029876\\
29110.7	24.518391799607\\
29118.8999999999	24.5204162230846\\
29128.6999999998	24.5221224355277\\
29130.5	24.5240400129074\\
29140.6999999998	24.5223192225334\\
29142.7999999999	24.5243266799117\\
29154.4000000001	24.5262309420501\\
29161.3000000002	24.5279719080703\\
29178.1000000001	24.5258446374348\\
29203.8	24.5230945181979\\
29209.8	24.5259312767247\\
29223.4999999999	24.5277789184084\\
29234.8000000002	24.5299282706628\\
29251.5000000002	24.5316495507938\\
29268.6	24.5333752438612\\
29281.4999999999	24.5307633462652\\
29290	24.5287656921622\\
29310.7999999999	24.5256883957845\\
29321.9000000001	24.5232248823409\\
29326.9	24.5214307634603\\
29331	24.5197077504765\\
29343.4999999999	24.5177824125193\\
29347	24.5196288560028\\
29414.1999999999	24.5210662840863\\
29432.6000000002	24.5193271429397\\
29448.5000000001	24.5223881610671\\
29454.2999999999	24.519935900918\\
29469.9000000002	24.5181540549712\\
29474.7000000001	24.5199288884064\\
29494.4000000001	24.5181304989066\\
29510.8000000001	24.5200925759635\\
29513.4	24.5182712995748\\
29523.9000000002	24.5200514835388\\
29537.6999999999	24.5217991861276\\
29559.9000000001	24.5196211096076\\
29566.7000000001	24.5214226767861\\
29575.1	24.5232492087966\\
29592.3000000002	24.521498763196\\
29596.7000000002	24.5235971456374\\
29601.1000000002	24.5253570801184\\
29622	24.5232444695008\\
29623.6000000002	24.5249580572312\\
29627.7999999999	24.5268817567226\\
29652.3000000002	24.5288745599007\\
29667.5999999999	24.5307185929479\\
29675.8999999999	24.5289122523251\\
29683.9999999999	24.5269352953265\\
29699.0000000002	24.528689421565\\
29705.0999999998	24.5266821970564\\
29713.8000000001	24.5289241735349\\
29727.3	24.5268674690689\\
29739.5000000003	24.5285746950194\\
29755.3000000002	24.5306734239836\\
29768.7000000002	24.5323963344172\\
29779.7000000002	24.5347517444711\\
29784.2999999998	24.5326412484388\\
29798.5000000001	24.5306826495205\\
29800.9000000003	24.5323982416697\\
29815.4999999999	24.5304471484726\\
29821.4000000001	24.5286119075164\\
29825.3999999999	24.530351544819\\
29837.9	24.5284536497084\\
29851.8999999999	24.526664880075\\
29866.1000000002	24.5283966490548\\
29894.6000000001	24.5301006532931\\
29897.1999999999	24.5319811487994\\
29915.8999999999	24.5336943441637\\
29932.6999999999	24.5319515715202\\
29944.5999999999	24.5302173673471\\
29949.0000000002	24.5279490869509\\
29961.0000000002	24.5258017896539\\
29968.6999999999	24.5275086089533\\
30005.2999999999	24.5299179481027\\
30039.1000000002	24.5316120269514\\
30049.0000000002	24.5298528075716\\
30076	24.5281136849525\\
30080.2	24.5300080118882\\
30108.0999999998	24.5321872446709\\
30119.5999999999	24.5297927934209\\
30128.7	24.5280263402459\\
30147.5999999998	24.5298978031492\\
30155.6000000003	24.5273696183474\\
30177.2000000001	24.525388288548\\
30202.2000000002	24.5232978945312\\
30229.9999999998	24.5212553051429\\
30243.2000000002	24.5184883924704\\
30254.1999999999	24.520481386117\\
30259.3000000001	24.5222972035136\\
30291.1	24.5206528628777\\
30295.9000000002	24.5223791919725\\
30301.5000000002	24.5241175383478\\
30310.7999999999	24.5258306417823\\
30325.3000000002	24.5276237081786\\
30390.4000000002	24.5290469061055\\
30406.9999999999	24.5307839287535\\
30417.7999999999	24.5289780030837\\
30422.5999999999	24.5306958291013\\
30431.5999999999	24.5286986924819\\
30455.5999999999	24.5303834750145\\
30458.0000000001	24.5323903986132\\
30481.9999999999	24.5304621400757\\
30500.7000000002	24.532143419189\\
30510.2000000002	24.5304044863538\\
30511.6999999999	24.5321482180665\\
30543.6	24.530426896545\\
30545.0000000001	24.5322490350334\\
30549.7999999999	24.5339591946291\\
30558.7999999998	24.5358962528101\\
30561.7000000001	24.5378217251602\\
30597	24.5359854365283\\
30606.1999999999	24.5377585660469\\
30617.0999999999	24.5401277713181\\
30624.3000000001	24.5418685753843\\
30637.9000000003	24.5437038971212\\
30655.8999999999	24.5456028183716\\
30676.4999999999	24.5434630956495\\
30679.9	24.5453063885267\\
30703.9999999999	24.5429763451787\\
30719.0000000003	24.5446643944647\\
30766.9	24.5422043098125\\
30774.9000000002	24.5439480097482\\
30779.0999999998	24.5418984249103\\
30797.6000000001	24.5437159268387\\
30816.6000000002	24.5415635029059\\
30837.7000000002	24.5432553554404\\
30852.8000000003	24.5406428569113\\
30868.6	24.5423357640587\\
30927.7999999998	24.5441817905516\\
30948.5999999999	24.5461037135647\\
30951.6000000001	24.5443707453872\\
30959.2	24.5425445665761\\
31008.1	24.5409278835921\\
31009.6000000003	24.5426431084467\\
31015.0000000001	24.5403045613266\\
31037.3000000002	24.5384600514218\\
31090.6999999998	24.5365188415866\\
31146.3000000002	24.5344566306218\\
31167.9999999998	24.5324546571655\\
31170.2999999998	24.5341734466032\\
31181.6	24.5358655878288\\
31191.5999999998	24.5376173789822\\
31199.6000000001	24.5358128443543\\
31206.5000000001	24.5339126979549\\
31212.8999999998	24.5356101624323\\
31229.9	24.533781620237\\
31269.0999999999	24.5318076573753\\
31302.5	24.5299112533783\\
31318.3000000001	24.5280729539185\\
31329.4000000002	24.5299158939658\\
31337.4999999999	24.5318722556928\\
31340.8999999999	24.5301681503462\\
31370.3000000001	24.5282176829113\\
31380.8000000001	24.5302078652939\\
31407.8000000003	24.532044485623\\
31433.6000000001	24.5303607274995\\
31452.2999999999	24.532309140161\\
31458.2000000002	24.530251157882\\
31496.2000000003	24.5285954223194\\
31540.5999999998	24.5302101728877\\
31560.6999999998	24.5282122126182\\
31563.2	24.5300713170042\\
31569.8000000002	24.5281106370308\\
31589.9000000002	24.5264324153213\\
31607.0999999998	24.5244754359766\\
31626.8000000002	24.5227954684146\\
31637.7999999998	24.524699806245\\
31657.1999999999	24.5229378374024\\
31667.4999999999	24.5247508494487\\
31674.5999999998	24.5227263399495\\
31696.5000000001	24.5243969384729\\
31707.0999999999	24.5262905586113\\
31736.2	24.5280640780431\\
31746.5999999998	24.5297936478437\\
31765.1000000003	24.5315628423558\\
31813.4	24.5298379618715\\
31822.6000000001	24.5277741989209\\
31839.4999999999	24.5295167024711\\
31849.8000000002	24.5272355643189\\
31860.4000000003	24.5250388411984\\
31886.2999999998	24.5267574890863\\
31902.3	24.5285621144491\\
31940.3000000001	24.5269314097507\\
31965.4000000003	24.525191221786\\
31978.5999999999	24.5231982538377\\
31985.2	24.5250161793074\\
32000.8999999999	24.5232961469954\\
32014.9999999998	24.5212415391487\\
32029.2999999999	24.5190356360094\\
32036.8000000001	24.5167291467027\\
32043.9000000002	24.5147297465984\\
32067.9000000002	24.5166521142572\\
32074.0999999998	24.5187720971996\\
32083.2999999999	24.5207802165606\\
32093.0999999999	24.5189012002543\\
32105.4000000001	24.5207207487814\\
32112.3000000001	24.5226143172108\\
32119.7000000002	24.5207006270276\\
32123.6000000002	24.5189688609967\\
32132.1000000002	24.5168398055533\\
32163.5	24.5143578455148\\
32216.4000000001	24.5160088774386\\
32225.8999999998	24.5177806739899\\
32231.9000000001	24.5200421940928\\
32244.6999999998	24.5217833573165\\
32266.7000000001	24.5196300841732\\
32287.1999999999	24.5173798985979\\
32295.3	24.5149464010354\\
32297.4000000002	24.5167582630235\\
32301.7000000001	24.5144233448291\\
32320.9000000003	24.5125460227097\\
32342.4	24.5107830254309\\
32356.6000000002	24.5126975247785\\
32381.4999999998	24.5145391209823\\
32402.6999999998	24.5163998172997\\
32407.7999999999	24.5143930955107\\
32413.9000000002	24.5162584068612\\
32442.5999999998	24.5142975153116\\
32454.3999999998	24.5124713059822\\
32464.3000000001	24.5105407769742\\
32485.8	24.5087868890811\\
32507.2000000001	24.5068030873065\\
32529.9	24.5041500153704\\
32567.5999999998	24.5024978736601\\
32592.8000000001	24.5007348226147\\
32616.9	24.5025600147163\\
32621.0000000002	24.5007065978768\\
32637.6000000002	24.5026457133928\\
32650.4000000002	24.5006967734032\\
32658.7999999999	24.4989880247038\\
32661.3999999998	24.500711847282\\
32681.5000000003	24.4988005483208\\
32692.5999999999	24.4969060371276\\
32725.6999999999	24.4944355829346\\
32749.9000000002	24.4961832061069\\
32758.9999999998	24.4979257671914\\
32789.6999999998	24.4954223569525\\
32808.1	24.4932669271706\\
32822.3	24.491201130935\\
32851.8000000001	24.4889945482605\\
32870.4999999999	24.4872317511697\\
32898.4000000001	24.4853108804353\\
32905.5999999999	24.4832962070401\\
32919.1000000002	24.4851029186615\\
32923.3000000001	24.4868391478401\\
32935.4000000002	24.4848264031213\\
32985.0999999998	24.4864363411469\\
33005.4999999999	24.4843299318903\\
33075.3000000002	24.4828482799906\\
33139.3999999999	24.484376650221\\
33168.1999999998	24.4827139166011\\
33188.5000000001	24.4806951784649\\
33222.9999999998	24.4823631750198\\
33241.9000000002	24.4804765056254\\
33244.4999999998	24.4824723413727\\
33255.2000000001	24.4842175532923\\
33271.2000000001	24.4859683871685\\
33309.3000000001	24.4840795691306\\
33327.3000000003	24.4858584828099\\
33334.9999999999	24.4841023425758\\
33346.5	24.4813564201448\\
33360.1	24.4791697891499\\
33375.1999999999	24.4809784481338\\
33381.5999999999	24.4828753478702\\
33395.9000000003	24.4846688226135\\
33407.6999999999	24.4864971653327\\
33454.0000000001	24.484891239041\\
33479.1999999998	24.4831881192259\\
33484.5999999998	24.484914005501\\
33489.9999999998	24.4830561867537\\
33499.2999999999	24.4813339940417\\
33515.6999999999	24.4833779888888\\
33545.7000000002	24.4853305033715\\
33561.9	24.4836422144092\\
33570.9999999999	24.4853460267909\\
33578.4000000002	24.4870973986331\\
33582.6000000002	24.4896926095877\\
33600.1000000001	24.4915208838043\\
33606.1000000002	24.4895287179151\\
33620.5999999999	24.4872950295502\\
33627.8000000002	24.4891890364846\\
33658.6999999999	24.4875040108382\\
33673.7000000002	24.4858020181862\\
33694.7000000001	24.4835998432993\\
33712.2000000001	24.4854252008911\\
33743.6000000002	24.4833850466902\\
33770.1	24.4816435792503\\
33789.2	24.483490335698\\
33806.4999999998	24.481610099803\\
33870.2000000001	24.4831017144814\\
33875.2000000001	24.4850968109508\\
33880.7999999998	24.4869528259285\\
33892.9000000002	24.4849969020152\\
33906.2000000002	24.483060670141\\
33933.4999999998	24.4804559492656\\
33935.9000000003	24.4822607260726\\
33957.1999999998	24.4839842979271\\
33975.7	24.4818370722691\\
33978.4999999999	24.4836455886941\\
33988.7000000001	24.4854187261686\\
33999	24.4835892714807\\
34028.6	24.4813936471273\\
34062.2	24.4795565772129\\
34068.5	24.4817808774062\\
34087.7999999999	24.4796540707993\\
34093.2000000002	24.4816430207695\\
34125.3999999998	24.4796413239366\\
34131.7	24.4818614898716\\
34139.6	24.4800042179632\\
34173.2000000002	24.4816859946215\\
34220.4000000002	24.4832775675399\\
34239.9000000002	24.4851051401869\\
34262.4999999999	24.4870500195549\\
34266.9	24.4888668398167\\
34305.8000000001	24.4870415875986\\
34314.2000000002	24.4851271918704\\
34322.0000000002	24.4830590202814\\
34344.4999999999	24.4847807224425\\
34351.8999999998	24.4864927806241\\
34354.9000000001	24.4884296317858\\
34367.8999999998	24.4867318435754\\
34370.4000000003	24.4884421233325\\
34375.3999999999	24.4904074122558\\
34387.7000000003	24.4886267804279\\
34407.8000000002	24.4903060053069\\
34413.3999999998	24.4921324480219\\
34437.8999999999	24.4903885243045\\
34452.2999999999	24.4879892257143\\
34467.7	24.4860420450391\\
34488.3000000001	24.4841743890699\\
34525.2000000001	24.4858118539158\\
34534.1000000002	24.483845000029\\
34547.1000000002	24.4815788254909\\
34590.4000000001	24.4786863445166\\
34605.4999999998	24.4804309129158\\
34623.6999999999	24.4785378843454\\
34640	24.4765459684008\\
34642.9000000002	24.478538232832\\
34649.4999999998	24.4764730328777\\
34653.2000000002	24.4781882244981\\
34675.2000000003	24.4765005638019\\
34686.8	24.4746575796626\\
34689.6000000003	24.4764296030234\\
34693.8999999998	24.4782959589554\\
34717.5000000001	24.4800907896859\\
34733.1999999998	24.481981268696\\
34747.7000000001	24.4789598190389\\
34816.9	24.4805698365741\\
34908.6999999998	24.4792144101201\\
34916.8999999999	24.4809118767363\\
34919.9000000002	24.4790950744559\\
34979.2999999998	24.4806943515326\\
35006.6000000002	24.4824562155245\\
35033.7000000002	24.4806444062591\\
35038.2	24.4823521689123\\
35058.5999999999	24.4840795579984\\
35076.9	24.4858454257776\\
35083.3000000002	24.4876494296448\\
35097.6000000001	24.4859349757961\\
35117.1999999998	24.4876456903008\\
35161.5000000003	24.4858026938478\\
35170.4999999998	24.4874980807834\\
35187.7	24.4894537311227\\
35191.3999999999	24.4877314124149\\
35203.2000000001	24.489465476248\\
35209.1000000003	24.4913261306704\\
35234.3999999999	24.4930394925428\\
35254.8	24.4947510842464\\
35280.1000000001	24.493058429374\\
35355.7000000001	24.4913140135423\\
35375.3999999998	24.4892651693969\\
35414.0999999999	24.4876349035133\\
35419.1999999999	24.4894732532828\\
35437.4000000003	24.4912874779541\\
35454.6	24.4895599173029\\
35471.3999999999	24.4917750870417\\
35498.1999999999	24.4893417431257\\
35525.5999999999	24.4873429657966\\
35562.7000000002	24.4856985389227\\
35574.4999999999	24.4874151782452\\
35594.6999999998	24.4855990200816\\
35604.9000000003	24.487291110799\\
35623.4	24.4855221974259\\
35640.6	24.4838064347782\\
35645.9999999998	24.4857081139311\\
35662.3000000001	24.4874153169725\\
35670.9000000001	24.485156009083\\
35727.5999999999	24.48352398839\\
35743.1999999998	24.4817909929973\\
35757.1000000003	24.4795453782735\\
35773.3	24.4813185215831\\
35792.9999999999	24.4795784662407\\
35810.6999999999	24.4778111631128\\
35872.6	24.4793394419712\\
35891.6999999998	24.481079243727\\
35897.5000000001	24.4793523801034\\
35932.5999999999	24.4810437289711\\
35939.7000000001	24.4792681094497\\
35951.6999999999	24.477494868129\\
35973.5000000001	24.4746147174595\\
35987.4	24.4768322334144\\
36005.7	24.478556232607\\
36009.6	24.4767382122039\\
36014.8000000002	24.4784797403297\\
36021.6000000001	24.4802438530108\\
36041.9	24.4783863270629\\
36055.5999999998	24.4765737456214\\
36087.7999999999	24.4746854208751\\
36091.4000000003	24.4764002604491\\
36098.2000000003	24.4781610214331\\
36109.9999999998	24.4759776350661\\
36149.7999999999	24.4742032481418\\
36160.7999999997	24.4761607150265\\
36184.1999999999	24.4780194725337\\
36217.4000000003	24.4762890867675\\
36234.5000000001	24.4779851302346\\
36257.8999999999	24.4762535164653\\
36274.3	24.4745605716428\\
36282.2000000002	24.4763975822922\\
36290.1999999998	24.474584117519\\
36332.7000000001	24.4726528095825\\
36345.1000000001	24.4744835631665\\
36386.8000000001	24.4763912287114\\
36389.9000000002	24.4781533388294\\
36401.6000000002	24.4763294021983\\
36409.2000000003	24.4780866426984\\
36433.2000000003	24.4759601792865\\
36483.8000000003	24.4779203977645\\
36542.9999999998	24.4763033240201\\
36546.9000000001	24.4780693353764\\
36552.8000000002	24.4760333653417\\
36568.8000000001	24.4779033550367\\
36599.2999999999	24.476084307393\\
36611.7000000002	24.4779005675766\\
36622.2	24.4796203406121\\
36638.3000000001	24.4814184025503\\
36666.6	24.47970501845\\
36714.2999999999	24.4778615475127\\
36750.9999999999	24.4760020788493\\
36781.2000000003	24.4741213605827\\
36787.1999999999	24.4758381289195\\
36803.8000000003	24.4740367189347\\
36814.7000000002	24.4757543161989\\
36864.4000000003	24.4774783328134\\
36874.8000000002	24.4757273918031\\
36905.2999999999	24.4774477447745\\
36916.5999999998	24.4756438143171\\
36925.2999999999	24.4774599598109\\
36937.5000000003	24.4756020965087\\
36947.6000000003	24.4773016994292\\
36963.5999999998	24.4745520605351\\
37023.4999999998	24.4765500923735\\
37047.5000000003	24.474729807059\\
37106.4000000001	24.4730707557975\\
37130.1000000002	24.4749556964412\\
37139.4999999999	24.4730691768355\\
37154.4999999999	24.4753005011493\\
37193.6	24.4772636226028\\
37271.4000000001	24.4758059106824\\
37297.0000000003	24.4737526510104\\
37359.2000000001	24.4755656556734\\
37386.8999999999	24.4737475593121\\
37447.3999999999	24.4718606048468\\
37488.9999999998	24.4735136346298\\
37528.0000000001	24.475259871936\\
37572.9000000003	24.4768849972054\\
37604.7000000003	24.4750670127218\\
37629.7999999999	24.4768123221162\\
37642.7000000003	24.4750656167979\\
37658.4000000002	24.4768113440525\\
37672.0999999999	24.4785279330647\\
37735.9000000001	24.4803370786517\\
37770.8999999998	24.4825395144423\\
37791.5999999998	24.4807722330565\\
37796.3	24.4824903959107\\
37821.0000000002	24.4805148448882\\
37877.8999999998	24.4788531601457\\
37881.5999999999	24.4806859248661\\
37891.2000000002	24.4789701065944\\
37971.9000000003	24.4804066154008\\
37992	24.4821949826411\\
38020	24.4838914153303\\
38026.7999999997	24.4818799323637\\
38039.8999999997	24.4837013669821\\
38093.9999999998	24.4820063999412\\
38131.5000000003	24.4802211289324\\
38211.9	24.4784360933738\\
38251.8	24.480091185013\\
38265.1	24.4783772200328\\
38286.5000000003	24.4801053109965\\
38304.2000000002	24.4779306761904\\
38321.5	24.479666819757\\
38358.2000000001	24.4778835349846\\
38377.6000000003	24.4761932059503\\
38381.8000000002	24.4742964782879\\
38388.0000000001	24.4763351142672\\
38424.1000000001	24.4780633038554\\
38443.8999999999	24.4797627718239\\
38452.2000000002	24.4778595818716\\
38469.7999999998	24.4796581223242\\
38491.1999999998	24.4813763110105\\
38542.5000000002	24.4796666545589\\
38586.0000000002	24.4813546847181\\
38654.3000000001	24.4797487478786\\
38682.6999999997	24.4814232682226\\
38703.3000000002	24.4831203460161\\
38757.4000000003	24.4848093917306\\
38780.9999999997	24.4830600472911\\
38801.4000000001	24.4812700539928\\
38820.6000000001	24.4838449590038\\
38855.0000000003	24.4855887644093\\
38897	24.4874296541387\\
38908.3	24.4895703755487\\
38921.0999999999	24.4912798166552\\
38932.7000000002	24.4929725064727\\
38957.9999999999	24.491184118322\\
39002.3000000003	24.4895185937276\\
39006.4999999999	24.4912399440095\\
39078.7999999997	24.4889185724266\\
39110.7	24.4870981928265\\
39179.2000000003	24.4886968373605\\
39278.6999999999	24.4872552114627\\
39298.2999999998	24.485475235633\\
39323.4999999998	24.4873307632058\\
39342.8	24.4890437664737\\
39356.1999999997	24.4908693144427\\
39389.6000000002	24.4926973295049\\
39397.4	24.4944476172346\\
39424.8	24.4923893275569\\
39466.1999999997	24.4940620225357\\
39488.2000000002	24.4923179777301\\
39510.5999999998	24.4893155524959\\
39561.5000000001	24.4909204885546\\
39596.5000000003	24.4889712753115\\
39619.8000000002	24.4869371199826\\
39630.0999999999	24.488647546568\\
39638.2999999999	24.4848429805441\\
39645.1999999998	24.4866352379727\\
39665.6999999999	24.4845685704057\\
39725.0000000001	24.4827577526551\\
39733.9999999999	24.4809873634989\\
39752.9000000001	24.4831836590949\\
39777.7	24.481243306568\\
39802.9000000002	24.4795618420722\\
39827.1	24.4777438534469\\
39833.3000000002	24.4759422996782\\
39859.5999999997	24.4778560802013\\
39862.4000000002	24.479648792725\\
39882.4000000002	24.4774023694603\\
39905.8	24.4790870522905\\
39916.0999999999	24.4810377741368\\
39969.8999999999	24.479109331999\\
40017.2999999999	24.4771024604297\\
40045.8000000002	24.475414461905\\
40049.3000000001	24.477270570845\\
40086.4999999999	24.4752610598055\\
40116.3000000001	24.4730334725948\\
40150.1	24.4713600430384\\
40217.8	24.4729336937036\\
40228.9000000002	24.4746327276343\\
40235.8	24.4764004284731\\
40246.7000000003	24.4742439150442\\
40313.1000000002	24.4726293124833\\
40324.4999999999	24.4709184964017\\
40344.5	24.4689499957863\\
40350.9999999998	24.470708357393\\
40393.5000000003	24.4687277192427\\
40408.5999999998	24.4704729427079\\
40453.0000000001	24.4685821358561\\
40458.9000000002	24.4704515682543\\
40488.8000000001	24.4721392776786\\
40504.4999999997	24.4703070762333\\
40566.4	24.4687118681671\\
40578.7000000003	24.4706595562215\\
40610.3000000001	24.4688552686011\\
40617.7000000001	24.4705523194265\\
40688.2000000001	24.4721455553562\\
40707.4000000003	24.4704292820733\\
40727.3000000004	24.4682940722953\\
40739.4999999999	24.47004879773\\
40745.3000000002	24.468283536301\\
40767.4000000003	24.4699822162262\\
40784.5	24.468059022278\\
40860.6000000002	24.4697227409222\\
40870.8999999997	24.4716302512784\\
40889.1000000003	24.4736996566331\\
40907.8000000003	24.4754680636259\\
40927.8000000003	24.477190376247\\
40979.3999999998	24.4788247782428\\
40995.4999999998	24.4806759749827\\
41025.3000000001	24.4823938340638\\
41035.9000000002	24.4804561848133\\
41046.1999999997	24.4786984454141\\
41077.1999999998	24.4804307975451\\
41109.5	24.4787105688209\\
41179.5999999999	24.4771088667475\\
41226.2000000002	24.4787914510883\\
41239.4000000002	24.4806556820524\\
41250.8000000003	24.4787386457023\\
41285.1000000003	24.4804433550037\\
41300.5000000003	24.4822109121901\\
41342.1000000001	24.4803130941266\\
41353.4999999998	24.4820281668343\\
41365.7000000002	24.4837522784523\\
41401.6000000002	24.4820864843714\\
41414.5000000004	24.4800142944759\\
41430.0000000002	24.4819587691075\\
41435.9	24.4801621778164\\
41445.1999999999	24.4819074780494\\
41483.6	24.4835923507305\\
41503.0000000001	24.4817856979358\\
41515.9999999999	24.479900568695\\
41524.5000000003	24.4780202578712\\
41535.6999999999	24.479846301263\\
41591.3	24.4815033877196\\
41617.9000000002	24.4833966072373\\
41630.5000000002	24.485114314951\\
41651.9000000001	24.4871794871795\\
41682.1999999999	24.4854530580126\\
41718.0000000002	24.4872129842922\\
41753.1000000003	24.4889014494697\\
41776.8000000001	24.4908071206815\\
41803.5000000002	24.4926274292166\\
41836.1000000002	24.4943374398248\\
41860.9000000002	24.4924870404434\\
41893.8000000003	24.4942581139498\\
41905.5999999999	24.4961902557409\\
41917.5000000001	24.4942458537703\\
41929.3000000002	24.4921224725372\\
41945.4000000003	24.4901121693626\\
41975.5000000003	24.4918476448222\\
41996.3000000004	24.489956281967\\
42011.9999999999	24.4879451396145\\
42021.3000000003	24.4899027638299\\
42032.6000000002	24.4916457900635\\
42073.5000000003	24.489466078491\\
42076.1999999998	24.491221899264\\
42081.2999999999	24.4894418911918\\
42101.2000000001	24.4911677311627\\
42114.7000000002	24.4892531841538\\
42143.3999999999	24.4872874820554\\
42150.6000000001	24.4892730132596\\
42172.4999999998	24.4872263033344\\
42213.3000000001	24.4889063662249\\
42228.3000000002	24.490627160868\\
42274.2000000002	24.4888738548007\\
42313.3999999998	24.490529027379\\
42406.1000000001	24.4921733142795\\
42441.2000000001	24.4940659216376\\
42480.0999999999	24.4958827877458\\
42491.8	24.4940800481974\\
42498.6000000002	24.4958081070715\\
42526.8000000003	24.4939085614047\\
42544.3000000001	24.4920600596083\\
42569.0999999997	24.4902417710457\\
42595.0999999997	24.4919615355721\\
42625.5000000002	24.4937314665365\\
42641.9000000003	24.4918155808827\\
42660.8000000003	24.493622966229\\
42694.4000000002	24.4919134783169\\
42711.3999999999	24.4936375449235\\
42723.1000000001	24.4953561530971\\
42777.3	24.4923721404293\\
42791.6999999998	24.4906734467819\\
42797.6999999999	24.4923804494624\\
42809.2999999998	24.4941531532794\\
42816.4000000001	24.492426984924\\
42837.0000000003	24.4941884487979\\
42850.6000000003	24.4924820364661\\
42878.8000000002	24.4901338420529\\
42894.2000000002	24.4920653793161\\
42909.8000000002	24.4941144118257\\
42941.8000000001	24.4923955390889\\
42967.1999999998	24.4907173594803\\
42995.5000000002	24.4925062099378\\
43043.0000000003	24.4942859598867\\
43101.5000000003	24.4925942424411\\
43135.6999999999	24.4942715795233\\
43152.0000000004	24.4922031604487\\
43176.7	24.4904671027033\\
43212.9999999998	24.4886388618266\\
43231.8000000003	24.4867794383314\\
43240.9999999998	24.4850385397226\\
43263.3999999999	24.4869231569337\\
43282.8000000003	24.4893479873114\\
43319.4000000003	24.4875864218193\\
43324.5000000003	24.4895509710419\\
43354.1999999998	24.4912269371204\\
43392.5000000004	24.4894290731599\\
43396.2999999999	24.4912020351919\\
43413.1000000002	24.4930113421724\\
43423.9999999997	24.4949233259872\\
43451.2000000001	24.4931682136093\\
43470.2000000004	24.494885013446\\
43481.4000000002	24.4966250014374\\
43520.6	24.4947806446128\\
43650.7000000003	24.4934800736756\\
43657.6999999998	24.4955082482397\\
43711.9000000001	24.4937316983895\\
43772.3999999998	24.4918613284596\\
43815.5000000004	24.4894512456751\\
43849.1	24.4912107860577\\
43908.6999999999	24.4928123747404\\
44016.6999999999	24.4913305828684\\
44037.2999999998	24.4932716282069\\
44045.6000000004	24.4911535064717\\
44115.5000000002	24.489522980533\\
44142.7000000001	24.487798689707\\
44168.6	24.4895140676542\\
44177.7000000003	24.4912603162675\\
44183.8999999997	24.4930291508238\\
44191.8999999998	24.49108435916\\
44211.5999999999	24.492837868709\\
44227.9000000003	24.4910463959483\\
44264.6000000002	24.4892657128592\\
44305.7000000002	24.4909244387868\\
44327.3999999997	24.492696407422\\
44331.9000000001	24.4944960750699\\
44348.8999999999	24.4927732305125\\
44371.3999999999	24.4909457647364\\
44377.5000000001	24.4927621142198\\
44415.6999999998	24.4944366644302\\
44428.2999999997	24.4962681528032\\
44437.3	24.4980579421838\\
44509.4999999998	24.4965131117781\\
44524.7000000001	24.4982571510709\\
44570.3	24.500116669359\\
44586.6999999999	24.4982820027452\\
44619.3000000004	24.4965194511804\\
44628.7	24.4983060266017\\
44654.8000000001	24.4965278166562\\
44675.6999999998	24.4946928762328\\
44688.8000000001	24.4922117125282\\
44727.5	24.4904712079342\\
44749.7999999997	24.492345234291\\
44784.2000000004	24.4905022519052\\
44800.1000000001	24.4878371078701\\
44812.4999999999	24.4897640395782\\
44830.4000000002	24.4880159712696\\
44860.6000000003	24.4862429699046\\
44899.1000000001	24.4879641508089\\
44911.8000000003	24.4861606834714\\
44926.8000000001	24.4844402796543\\
44935.1000000003	24.4861489433673\\
44951.6999999998	24.4880071543298\\
45047.2000000001	24.4898140399092\\
45067.5999999998	24.4915981956035\\
45132.4000000001	24.4934359940176\\
45160.3000000003	24.4951329040487\\
45208.7999999997	24.493186076193\\
45256.6000000002	24.4911803114235\\
45308.6000000002	24.4928678156734\\
45368.7000000003	24.4910599354623\\
45403.7999999999	24.4893059847282\\
45413.1000000003	24.4875939154255\\
45426.6000000004	24.4855998784856\\
45436.0000000001	24.4873569694582\\
45446.2000000004	24.4891223267901\\
45460.7999999999	24.4874166591511\\
45487.7000000001	24.4854664327578\\
45539.9999999997	24.4836967859096\\
45629.8999999999	24.4817006355468\\
45650.1000000001	24.4796298811396\\
45661.0999999997	24.4813977731641\\
45675.3000000002	24.4832010228701\\
45707.0000000003	24.4850362416342\\
45716.5000000001	24.4832292865174\\
45786.9000000004	24.4848974599777\\
45792.9000000003	24.4827812111021\\
45798.1999999997	24.4845332687021\\
45806.1999999998	24.4824401883584\\
45827.2000000004	24.4843139351435\\
45846.3	24.4826202275424\\
45881.5999999999	24.4807842778275\\
45886.4999999998	24.4827465970458\\
45902.2000000002	24.480254802047\\
45938.4999999999	24.4785430988319\\
45959.5000000002	24.476496749319\\
46009.5000000001	24.4781523855891\\
46044.2999999999	24.479849884025\\
46113.5	24.4814978661393\\
46131.2000000002	24.4794748901505\\
46201.2000000004	24.481345762998\\
46219.2999999997	24.4832256584897\\
46247.8000000002	24.4813278008299\\
46262.9000000003	24.4830642197869\\
46290.2999999998	24.4813179406529\\
46325	24.4796017709622\\
46329.3999999997	24.4813779557302\\
46356.5000000003	24.4793621620222\\
46387.5999999999	24.4810585564708\\
46394.9000000003	24.4793620002155\\
46445.7999999998	24.4772950895558\\
46494.6000000001	24.4791341808851\\
46520.4000000002	24.4773809395858\\
46528.9999999997	24.4793043493212\\
46540.5999999997	24.4811530552828\\
46559.9000000002	24.4830326460481\\
46639.2999999997	24.4814470168999\\
46652.3000000002	24.4831991494543\\
46714.6000000002	24.4847981470501\\
46732.3999999999	24.4868132456\\
46783.5999999998	24.4884436245957\\
46818.4	24.4864743637665\\
46828.5999999997	24.4881877993624\\
46848.4000000001	24.4859493900552\\
46888.4000000002	24.4876675517451\\
46917.8	24.4895871298588\\
46928.0999999997	24.4878346069101\\
47029.1000000002	24.4862766111267\\
47143.9999999997	24.4878998644581\\
47164.7	24.4858453762128\\
47206.7000000001	24.4841844818967\\
47225.9000000002	24.4822767119807\\
47245.1000000001	24.4805821543776\\
47312.3000000003	24.4789949357885\\
47320.8000000004	24.4807262752822\\
47348.1	24.4824512864269\\
47360.1000000002	24.4842716035828\\
47400.9	24.4859813084112\\
47408.9000000002	24.487755489464\\
47533.0999999997	24.4862538183838\\
47571.7000000003	24.4844634846695\\
47623.2000000003	24.4827636892026\\
47637.4	24.4844922592495\\
47656.4999999999	24.4862201667765\\
47679.3000000002	24.487933992458\\
47750.1999999997	24.4863383057279\\
47776.8000000001	24.4846358805197\\
47785.8999999999	24.482693675972\\
47795.5	24.480914561173\\
47809.1000000002	24.4791797394644\\
47826.9000000004	24.4773872498798\\
47830.8000000001	24.4791546887054\\
47912.3000000001	24.480718978803\\
47958.3999999998	24.4824170897755\\
48058.6	24.4840164215844\\
48080.7000000003	24.4819137784729\\
48104.3999999998	24.4796224885406\\
48140.7000000004	24.4777818399362\\
48150.7999999998	24.4795009023715\\
48190.0999999999	24.4777983905441\\
48198.4999999999	24.4799641483363\\
48205.5999999997	24.4782256040261\\
48219.3	24.476040763676\\
48231.6000000001	24.4778848765439\\
48252.7	24.4760925790835\\
48292.4000000003	24.4744007868717\\
48295.9000000001	24.476147092927\\
48327.0000000004	24.4744667070857\\
48347.1000000001	24.4727719495648\\
48391.9999999998	24.474449341938\\
48404.3999999998	24.4727246433699\\
48447.3000000003	24.4743783980152\\
48468.5000000002	24.4723387925378\\
48524.9000000001	24.4706852138073\\
48563.6999999998	24.4725495121881\\
48583.0999999998	24.4708047226202\\
48640.2	24.4725053093834\\
48680	24.4701633727129\\
48696.6999999998	24.4720392305039\\
48714.6000000001	24.4702317780875\\
48755.9000000004	24.4720649766183\\
48774.0000000003	24.4691342331278\\
48784.3000000003	24.4672477267323\\
48789.9000000002	24.4691535150646\\
48841.2000000001	24.4708883670173\\
48886.7999999998	24.4728137803788\\
48967	24.4743919897237\\
49038.3000000002	24.4761248327841\\
49083.5000000003	24.477829662046\\
49119.6000000002	24.4759231021362\\
49275.9999999999	24.4769776828929\\
49316.3	24.4786724091783\\
49323.4	24.4769734507892\\
49356.5000000003	24.4747409667603\\
49367.7000000003	24.4764806209716\\
49377.4999999999	24.4783059524967\\
49404.7000000003	24.4761642593432\\
49415.7000000001	24.4743584035875\\
49471.1000000003	24.476058797846\\
49483.1999999997	24.4743175980583\\
49612.2000000002	24.4755836758223\\
49658.5	24.4739078427503\\
49682.0000000001	24.4717916513191\\
49712.0999999999	24.4738716049581\\
49727.0000000002	24.4755877579831\\
49830.4999999998	24.4741183128439\\
49852.8	24.4758078266259\\
49894.1000000001	24.473586108205\\
49914.4000000004	24.4718468581274\\
49995.8000000003	24.4702065569377\\
50018.3000000003	24.4719943060954\\
50044.3000000001	24.4702703998849\\
50072.2999999999	24.4685695113476\\
50183.9999999997	24.4671120932726\\
50223.4000000004	24.4654394854998\\
50237.9000000004	24.4671364305904\\
50352.3000000004	24.4687442902424\\
50375.6000000001	24.4707269576403\\
50472.2	24.4722352656804\\
50572.5	24.4737268797728\\
50641.0000000001	24.4753767197\\
50673.6000000004	24.4736421457285\\
50679.8	24.4753837320121\\
50697.5000000002	24.4770955627091\\
50735.2999999998	24.4787663051834\\
50830.9999999997	24.4802886421895\\
50961.3999999999	24.4816184766932\\
50969.3	24.4833174414453\\
50994.4000000004	24.481267587681\\
51071.9999999997	24.4828389668723\\
51084.5000000004	24.4809590365785\\
51105.5	24.4828355405279\\
51199.7999999998	24.4844228211383\\
51228.8000000004	24.4824698558821\\
51254.3000000003	24.4845320596866\\
51267.4000000001	24.4862729799581\\
51327.1999999999	24.4844361577562\\
51400.0999999999	24.4825895619083\\
51409.5999999999	24.4842899297214\\
51417.8000000001	24.4823300834923\\
51431.6999999997	24.4805742750594\\
51462.3000000002	24.4825348215396\\
51510.5999999997	24.484233372872\\
51542.7999999997	24.4860106823636\\
51572.0999999998	24.4878054455695\\
51584.7000000003	24.4860889254199\\
51636.7999999998	24.4879146501823\\
51654.2999999999	24.4896852930244\\
51658.7999999998	24.4914235494755\\
51727.1999999999	24.4893895486523\\
51781.3999999997	24.4911792821761\\
51792.5000000004	24.4932673779652\\
51808.1000000003	24.4951571372872\\
51827.3000000003	24.4970806947676\\
51865.5999999998	24.4953794125983\\
51881.2000000004	24.4970731265408\\
51900.2000000001	24.4952341315946\\
51926.7999999997	24.4934706289033\\
51937.9	24.4917016442682\\
51962.5	24.4933856273551\\
52046.9999999999	24.4949286319507\\
52061.5000000005	24.4967500038416\\
52077.4999999998	24.4949844078836\\
52085.3	24.4966919712626\\
52162.8000000004	24.494995485297\\
52182.0999999999	24.4932179938753\\
52240.8999999996	24.4949369269348\\
52275.2000000003	24.49321189931\\
52297.5000000002	24.4950055069449\\
52326.4	24.4933255616179\\
52345.6000000003	24.4916010293109\\
52399.2000000005	24.4936478159059\\
52407.9000000002	24.4918714700046\\
52435.6999999999	24.4935711860981\\
52497.1999999999	24.4917357654584\\
52524.0999999999	24.493471580719\\
52554.0000000004	24.4917142525512\\
52586.4000000003	24.489935629867\\
52690.6999999997	24.4883357246426\\
52763.9999999998	24.4867248754361\\
52790.9999999997	24.4884080839384\\
52825.3000000004	24.4867052592124\\
52852.8999999996	24.4848920591073\\
52965.6999999998	24.4831570560626\\
52984.9	24.4850429366802\\
53056.4000000004	24.4867264142942\\
53173.6999999998	24.488187791732\\
53227.8000000002	24.4858429507833\\
53272.3	24.4841606535467\\
53285.6999999999	24.4858855454924\\
53311.7000000001	24.4876368833917\\
53386.1999999999	24.4858699703857\\
53406.7	24.4841480860115\\
53465.0999999997	24.4860208135385\\
53549.0000000003	24.4876197732548\\
53568.4000000005	24.485658549334\\
53629.6999999998	24.4839622747055\\
53636.1999999999	24.4858426103217\\
53682.3999999998	24.4841428771015\\
53716.7000000004	24.4858219402496\\
53802.6000000005	24.4874327868304\\
53834.6000000004	24.4856941712316\\
53844.9999999999	24.4874649689572\\
53867.2000000003	24.4855413209869\\
53897.7	24.4872703524077\\
53917.3999999999	24.4855566374554\\
53936.3000000001	24.4838365185663\\
53967.9	24.4821005040024\\
54004.6999999999	24.4804165555654\\
54089.8999999999	24.4819744869662\\
54123.9999999997	24.4800375433495\\
54203.2999999996	24.4783537564064\\
54246.1999999997	24.4765080752051\\
54287.6000000002	24.4784730242762\\
54355.8999999999	24.4765987195526\\
54394.6999999998	24.4747659702766\\
54447.3999999997	24.4764222416089\\
54460.1999999998	24.4781978799235\\
54487.3000000004	24.4799715163506\\
54503.5999999998	24.478154694085\\
54532.3999999998	24.4764131481227\\
54570.9000000005	24.4741712631251\\
54594.6999999997	24.4759574171899\\
54662.2000000001	24.4742720302658\\
54717.5999999998	24.4725198610322\\
54766.9999999998	24.4741824927739\\
54801.1999999997	24.4724121508066\\
54814.1999999997	24.4742704002423\\
54875.8999999997	24.472629200379\\
54918.8999999998	24.4709481236002\\
54952.7	24.469180824271\\
55001.1999999999	24.467421679124\\
55023.8999999997	24.4691407385868\\
55039.3000000005	24.4670181724365\\
55084.5000000004	24.4687262864757\\
55120.2	24.4670293884467\\
55223.7000000003	24.4655384091642\\
55264.3	24.4672881638089\\
55330.7000000001	24.4690118342767\\
55403.6999999997	24.4708846685606\\
55441.4	24.4688545584084\\
55457.0999999998	24.4705827196469\\
55488.7999999998	24.4722818437561\\
55502.6000000003	24.4741246822227\\
55595.6000000001	24.4725761164982\\
55635.9	24.4744410094184\\
55646.6999999996	24.472745962032\\
55705.4000000004	24.4744235308901\\
55718.7999999997	24.4724859966726\\
55760.2999999998	24.4745374853839\\
55770.9000000001	24.4727546574385\\
55812.6999999996	24.474493306195\\
55906.6000000002	24.4729164840708\\
55978.3	24.4712246152087\\
55997.6	24.47296942553\\
56020.7999999999	24.4747942285825\\
56041.0999999997	24.4730662441204\\
56108.0000000003	24.4750757911959\\
56139.3000000001	24.4733645176115\\
56154.9999999998	24.4750699402191\\
56181.7000000001	24.4767878565657\\
56218.2999999997	24.4784625674157\\
56224.7999999999	24.4802569679982\\
56266.2000000001	24.4819723351278\\
56359	24.4803412403676\\
56413.6999999999	24.4785850270678\\
56431.7000000002	24.4803461877877\\
56566	24.4789723880557\\
56662.2999999999	24.4806079516575\\
56773.4000000002	24.4790263062873\\
56826.8000000003	24.4769642546048\\
56887.7999999997	24.4753277937839\\
57018.0000000001	24.4767889494739\\
57067.3999999999	24.4785560958514\\
57099	24.4802457481817\\
57110.3999999998	24.4785109568293\\
57124.0000000004	24.480210629139\\
57149.4000000001	24.4819289757566\\
57178.6999999997	24.4800520472623\\
57228.9	24.4818186583725\\
57241.8999999997	24.4801020229901\\
57323.2000000003	24.4818424619657\\
57365.2999999999	24.4835737221391\\
57409.1999999998	24.4855798624961\\
57450.8999999996	24.4873022227637\\
57489.8000000005	24.4889971977687\\
57507.4999999999	24.4870243237416\\
57518.5000000001	24.4887740661282\\
57538.7000000003	24.4869548895702\\
57687.9000000001	24.4856815975593\\
57728.7000000003	24.4839664084478\\
57790.4000000004	24.4856853635113\\
57845.8000000003	24.4840170176279\\
57867.8000000003	24.4857684484835\\
57892.1999999997	24.4875397937204\\
57936.3999999998	24.4857732172292\\
57999.0000000005	24.4840006138026\\
58023.8999999998	24.4859023852199\\
58054	24.4876761503494\\
58109.8	24.4858449248751\\
58196.3000000005	24.4874253390244\\
58209.4000000001	24.4891297812213\\
58286.2999999998	24.4875305388564\\
58327.1000000004	24.4894320317108\\
58378.1000000002	24.4913340938912\\
58387.6000000005	24.4895757154332\\
58434.0999999996	24.491308172269\\
58476.7999999999	24.4932614416975\\
58492.3	24.4915578776046\\
58526.1999999997	24.4932620035779\\
58539.2999999999	24.4913682067121\\
58577.6000000002	24.4896948838893\\
58602.3999999998	24.487692504586\\
58651.3999999998	24.4893992481011\\
58765.8999999997	24.487799067488\\
58868.3000000001	24.4893695089386\\
58945.3000000003	24.4877462872421\\
59040.9000000005	24.4894226046307\\
59067.1000000002	24.4877021426443\\
59090.2999999998	24.4894263704426\\
59138.4000000005	24.4876011396975\\
59150.4999999996	24.4893542922642\\
59157.7999999997	24.4910654367379\\
59183.7000000002	24.4887959205053\\
59206.2	24.4870900562947\\
59271.8999999997	24.4852544202996\\
59295.2999999999	24.48351811439\\
59325.5999999999	24.4818013103933\\
59454.6	24.4801504338597\\
59485.2999999998	24.4782753415124\\
59609.1	24.4797782892574\\
59627.6	24.4780529854413\\
59662.9999999996	24.4760999679869\\
59683.7999999997	24.4777904929135\\
59714.4999999999	24.4760912741608\\
59801.7000000004	24.4776913069506\\
59838.2000000003	24.4759627195291\\
59923.5999999997	24.4742898052023\\
59999.8999999999	24.4726666666667\\
60081.5000000002	24.4707198210434\\
60127.6	24.4727471697737\\
60163.8999999998	24.4744365401237\\
60272.4000000005	24.4728524617363\\
60315.9000000001	24.4746004376948\\
60338.7000000005	24.47629054605\\
60457.1000000002	24.474669683677\\
60526.9	24.4763493977894\\
60544.6000000003	24.4744791864523\\
60607.9000000003	24.4761417634636\\
60633.8000000003	24.474427671649\\
60668.3000000002	24.4727073731959\\
60734.1000000002	24.4743488841542\\
60816.5999999999	24.4760074124377\\
60840.6999999997	24.4740371592747\\
60929.3999999999	24.47238201528\\
61000.5000000005	24.474021567001\\
61008.4000000005	24.4757697697862\\
61061.6000000005	24.4740975112059\\
61117.7999999998	24.4758082329399\\
61175.6000000005	24.4775294765732\\
61224.6999999998	24.4758659889457\\
61243.5000000004	24.4776597064836\\
61346.5000000005	24.4792702448057\\
61373.7000000001	24.477545793156\\
61457.4999999998	24.4791531071828\\
61507.4000000001	24.4773401617689\\
61618.3000000005	24.4758059280994\\
61635.7999999999	24.4740484036089\\
61769.3000000004	24.4755493820565\\
61792.2999999996	24.4772819958442\\
61837.9	24.4755651864549\\
61873.3999999997	24.477361067339\\
61949.7000000004	24.4756238115377\\
62021.1000000003	24.4772755122442\\
62068.9999999998	24.4790080732603\\
62120.4000000004	24.4808074629148\\
62204.9999999998	24.4823977455225\\
62328	24.4806114737975\\
62388.8999999999	24.4823606725545\\
62408.6000000003	24.4805612038065\\
62460.9999999999	24.4822777696839\\
62486.8999999997	24.4842927328884\\
62493.9000000003	24.4823503056293\\
62589.3000000005	24.4805030883824\\
62648.8999999996	24.4822742581685\\
62693.5000000001	24.4839983666594\\
62736.2999999999	24.485784966941\\
62792.6999999999	24.4879986240461\\
62855.7000000005	24.4862049325599\\
62863.9999999998	24.4879032707062\\
62925.4999999997	24.4895559200071\\
63009.2999999999	24.4874574269871\\
63165.3	24.4858735953544\\
63180.9	24.4875832924455\\
63365.0000000002	24.4861919258393\\
63417.5000000001	24.484370269452\\
63439.2999999997	24.4825770735537\\
63489.3000000002	24.484402120669\\
63519.5000000003	24.4863002915636\\
63546.1000000002	24.4845482499347\\
63594.4999999998	24.4827391004897\\
63635.0999999998	24.4850334406115\\
63670.5999999997	24.4869304091207\\
63694.5000000002	24.4887321688181\\
63720.8999999997	24.4865899781862\\
63814.4999999998	24.4849924625399\\
63866.8999999996	24.4868241815022\\
63873.9000000002	24.4849234430285\\
63913.7000000001	24.4829755076369\\
63982.0999999999	24.4813401227216\\
63993.3999999999	24.4831115660184\\
64015.7999999998	24.4848545439492\\
64057.3999999996	24.4865940756352\\
64093.9000000005	24.4884076512622\\
64153.6	24.4866624995908\\
64213.0999999997	24.4849968542294\\
64372.3999999998	24.4863878208863\\
64402.1000000004	24.4881386039607\\
64457.7000000005	24.4898212473898\\
64493.5000000003	24.4917325129935\\
64548.5000000005	24.4900741456825\\
64584.7999999999	24.4919478082338\\
64647.2000000003	24.4902726022587\\
64740.5	24.4918953485139\\
64763.8999999996	24.4901488481255\\
64847.7000000005	24.4882632872665\\
64872.2000000005	24.4899595050584\\
65028.7	24.4913023152818\\
65155.6000000003	24.4897990505819\\
65215.3999999997	24.4915702555374\\
65335.4999999998	24.4932318674658\\
65348.5	24.4949394478229\\
65409.4000000002	24.4965945313754\\
65469.7000000002	24.4983183085942\\
65488.1999999998	24.4965894671262\\
65572.9000000003	24.498192853766\\
65595.3000000003	24.4998887117084\\
65725.5999999997	24.5013746525332\\
65766.5000000003	24.4995179924156\\
65784.2000000005	24.5012867811925\\
65806.9999999998	24.502979161823\\
65831.7000000001	24.5012288893817\\
65869.1999999999	24.4994253772243\\
65987.9999999997	24.4979018944325\\
66035.2000000002	24.4997751202766\\
66077.1999999998	24.5014551139347\\
66147.5999999999	24.4997180552007\\
66193.4999999998	24.5014623770274\\
66317.3000000004	24.4999050023071\\
66374.4000000002	24.4981129801354\\
66592.2999999999	24.496939590704\\
66785.1999999996	24.4982054434134\\
66869.6999999999	24.4965589847734\\
66937.8000000002	24.4947929349442\\
67013.6	24.4964835548552\\
67047.1000000002	24.4945351931177\\
67102.1999999996	24.4927521113285\\
67153.8999999999	24.4944456026447\\
67280.6000000005	24.4960293219304\\
67304.2000000005	24.4978404054421\\
67488.9000000002	24.4965846286061\\
67570.5000000002	24.4983765128621\\
67591.8999999997	24.5000887679015\\
67717.3999999996	24.5016428545059\\
67841.8000000006	24.5034410887667\\
67982.8	24.5017497047052\\
68116.7000000002	24.5033530641486\\
68163.0999999998	24.5016372470776\\
68331.3000000003	24.5003029354001\\
68352.0999999998	24.4985530824172\\
68536.0000000002	24.4997891622079\\
68632.1999999998	24.4980861780823\\
68720.8000000002	24.4964777818684\\
68765.4	24.4981858635508\\
68885.7000000002	24.4960499841767\\
68923.7999999997	24.4977431631118\\
68963.5000000002	24.4998810966945\\
69095.5000000001	24.4983761628816\\
69152.8000000006	24.5000571198026\\
69167.2000000006	24.5017515502268\\
69188.5000000001	24.4999898827263\\
69201.5000000001	24.4981329911447\\
69272.6999999999	24.4998036747468\\
69321.4000000002	24.5016336922888\\
69412.8999999999	24.5000216097849\\
69464.7000000003	24.5017620435098\\
69505.7999999996	24.5000784106097\\
69632.5999999998	24.4982601565069\\
69757.9999999996	24.4998071908495\\
69782.8000000006	24.4979787311791\\
69874.8000000004	24.4996415021703\\
69941.5000000006	24.5012982259485\\
69958.5999999997	24.4995976197385\\
70054.6999999996	24.4979644506872\\
70090.9000000002	24.4960123268323\\
70220.5000000005	24.4975121260713\\
70260.3000000001	24.4994335358182\\
70417.1999999995	24.4979571781366\\
70458.2000000006	24.4961913642538\\
70498.2	24.4978956939387\\
70523.3000000005	24.4961246905283\\
70561.1000000005	24.4978826890699\\
70594.8000000006	24.4961038261971\\
70618.7999999998	24.4978327331635\\
70695.2999999998	24.4994723843418\\
70749.9000000005	24.5013427561837\\
70847.5999999998	24.4997367592738\\
70892.4	24.5014634834432\\
70963.1000000004	24.4997407106782\\
71110.5999999999	24.4981416298813\\
71260.6	24.4965317489163\\
71321.2000000001	24.4984317447943\\
71400.9000000005	24.5002170837943\\
71418.4000000001	24.5023348292109\\
71511.0000000006	24.5006719236594\\
71670.1000000004	24.5022338433547\\
71785.1000000006	24.5037974401409\\
71826.7000000001	24.5054770642713\\
71872.2000000004	24.5072162710808\\
71958.7999999997	24.508851580555\\
72002.4000000001	24.5105378285476\\
72100.2999999996	24.508879284997\\
72154.1000000003	24.5108392858628\\
72196.0000000002	24.5089416187301\\
72228.0999999997	24.5071869436037\\
72264.3000000006	24.5088868101029\\
72297.6000000005	24.5105999222659\\
72337.0999999995	24.5122841359632\\
72393.9999999998	24.5105609435023\\
72441.7000000003	24.5126432529286\\
72659.9000000004	24.5115606936416\\
72767.9000000004	24.5095371591909\\
72828.7000000001	24.511319697702\\
72850.8000000006	24.5130808267297\\
72883.0999999998	24.5112728310502\\
72908.2000000005	24.5129841184063\\
72951.7	24.5112526353017\\
72978.4999999996	24.509376721395\\
73129.9000000001	24.5075892246684\\
73148.0000000001	24.5094541074888\\
73179.5999999996	24.511169081043\\
73327.8999999999	24.5096006982326\\
73391.9000000003	24.5077120122084\\
73481.9000000006	24.5094036634822\\
73497.7000000002	24.5076723384918\\
73721.2999999997	24.5064526718158\\
73730.4000000005	24.5046486867714\\
73746.8999999998	24.5063528007919\\
73764.8	24.5081332720576\\
73812.4000000004	24.5099407281964\\
73981.8999999999	24.5085291016734\\
74050.5999999996	24.5068581390858\\
74164.0000000006	24.5052525413239\\
74309.1000000005	24.5035607973172\\
74328.4000000002	24.5052705220743\\
74398.4999999999	24.5071815867503\\
74445.1000000005	24.5089005066814\\
74548.3999999996	24.5104864618336\\
74625.3000000004	24.5088133530943\\
74723.9000000006	24.5069857074032\\
74740.7000000005	24.508702074369\\
74790.9999999995	24.5069266262964\\
74891.0000000002	24.5051815235722\\
74906.6000000005	24.5068865668892\\
74932.1000000003	24.5086891883596\\
74977.3000000002	24.5069847714111\\
75019.3000000001	24.5049947080355\\
75138.8999999998	24.5033870559896\\
75226.0000000006	24.5017354349089\\
75271.7000000007	24.4998791047909\\
75402.3000000002	24.4982918315597\\
75425.9000000006	24.5000397740832\\
75455.9999999997	24.4983506966302\\
75498.0999999998	24.4966105152176\\
75582.8999999999	24.4983660346903\\
75622.6000000002	24.4964805541193\\
75661.2000000002	24.4946888303532\\
75702.0999999998	24.4963818752956\\
75737.7999999998	24.4946057390025\\
75774.4999999998	24.4929039546233\\
75814.4000000004	24.4910933924249\\
75976.8000000006	24.4895487970686\\
76112.9000000005	24.4879324162758\\
76241.0000000002	24.4863990682191\\
76398.6999999996	24.4849395540244\\
76429.7999999996	24.4867519125368\\
76501.8	24.4846206434089\\
76694.4000000004	24.4832419534647\\
76791.6999999997	24.481520162309\\
76813.8999999997	24.4834275001953\\
76856.9	24.4852127977933\\
76973.8999999999	24.4868397121106\\
77070.0999999995	24.4852095881417\\
77106.1000000004	24.4870062329618\\
77332.4000000002	24.4858242006918\\
77455.6999999996	24.4840283103396\\
77509.1000000005	24.4857384671755\\
77688.0999999998	24.4872451672197\\
77734.2000000006	24.488932170226\\
77918.2000000003	24.4875722391274\\
78118.3999999999	24.4892055018978\\
78212.0000000003	24.490839652688\\
78256.7000000003	24.4891178785741\\
78287.6000000004	24.4908203970739\\
78392.3	24.4924252861247\\
78421.8999999997	24.4941470505725\\
78467.9000000006	24.4922771065912\\
78502.0000000003	24.4905550297381\\
78600.7999999999	24.4922131934876\\
78621.0000000006	24.4939335623643\\
78649.7999999997	24.4956446225615\\
78764.2999999995	24.4973617522637\\
79035.1999999999	24.496522439973\\
79118.6000000003	24.4982538894092\\
79128.9000000004	24.4999936812041\\
79292.5000000006	24.4983768977181\\
79405.5000000007	24.5000352619966\\
79502.7000000002	24.4983824469075\\
79514.7000000001	24.5000930644358\\
79564.2000000006	24.5019437109357\\
79644.0000000003	24.503635548647\\
79721.2	24.5018583490234\\
79743.8000000006	24.5035670439996\\
79808.9000000004	24.5018732223183\\
79839.4999999998	24.4999974949774\\
79926.6000000003	24.5016996823339\\
80058.6000000005	24.500148016393\\
80076.6000000007	24.5018838188886\\
80107.4999999998	24.5001722682991\\
80206.0000000005	24.4985107117788\\
80258.7999999995	24.4967224818681\\
80312.9000000004	24.498400009961\\
80386.8999999995	24.4966225882294\\
80456.5999999996	24.4949146559578\\
80484.2000000005	24.496703083707\\
80609.7000000002	24.4949125292458\\
80682.4999999996	24.493261248398\\
80719.7	24.4915126152443\\
80819.1999999998	24.4931594309775\\
81027.0999999999	24.4944907389124\\
81038.3999999999	24.4962579514675\\
81093.4000000001	24.4945649158071\\
81248.9999999997	24.4929482295804\\
81327.4000000003	24.4912237558022\\
81387.8999999999	24.4894824789895\\
81588.8999999996	24.4908014560786\\
81635.7999999997	24.4925333094876\\
81781.2999999998	24.4940781155617\\
81905.1000000007	24.4924620170636\\
81990.2999999998	24.4907940441808\\
82102.4000000002	24.4890228677568\\
82182.0999999999	24.4907047998228\\
82250.8999999995	24.4889423836792\\
82353.0999999996	24.4873301826766\\
82481.2000000005	24.4890660064766\\
82573.9999999995	24.4873392504429\\
82656.2000000002	24.4889984187533\\
82679.5	24.4872979549974\\
82741.4000000005	24.4890411703921\\
82813.8999999995	24.4872847586157\\
82883.7000000002	24.4854844975737\\
82925.2999999998	24.4871896909753\\
82985.8000000006	24.4888589507374\\
83072.4000000005	24.4868036955671\\
83097.8000000007	24.488585150768\\
83122.6999999997	24.4903925276819\\
83373.7999999995	24.4914775487293\\
83405.0000000006	24.4897494277928\\
83444.3000000005	24.4916375454794\\
83749.1999999995	24.4922644129563\\
83801.4000000005	24.4939529722022\\
83819.2000000001	24.4956710447355\\
83968.8000000001	24.4940686373169\\
84042.2999999996	24.4959687015126\\
84161.7999999997	24.4976646202141\\
84235.5999999995	24.4993512251931\\
84275.2999999994	24.4976588660511\\
84582.0999999997	24.4968799581945\\
84816.3999999995	24.4982992695997\\
84851.0000000004	24.4964414132522\\
84870.7000000007	24.4981784076502\\
85069.8999999997	24.4995885741154\\
85094.0999999999	24.4976743420821\\
85126.4000000003	24.4994214492549\\
85220.6999999998	24.5010607738956\\
85267.8999999996	24.502744288596\\
85326.5000000001	24.5010348472809\\
85527.2999999997	24.5024401536817\\
85554.3000000008	24.5041751213263\\
85791.5999999998	24.5054008721123\\
85848.6	24.5035743115504\\
85881.9000000001	24.5018746652384\\
86086.0000000006	24.5004710400401\\
86219.2999999994	24.4987786971378\\
86273.9999999994	24.5005163774528\\
86320.5000000005	24.5022624958585\\
86398.3000000002	24.5004537121058\\
86489.2999999997	24.5023089534671\\
86569.0000000002	24.5006590111252\\
86824.9	24.5020443420674\\
86867.4999999994	24.5038426294729\\
86896.4000000006	24.5021376004787\\
87028.8999999998	24.5005687759253\\
87162.2999999998	24.5023083347866\\
87262.7999999995	24.5005609485818\\
87461.2000000001	24.4991784938024\\
87504.1000000001	24.5008811005643\\
87672.9999999994	24.499304803868\\
87907.4000000004	24.4980234906009\\
88001.8999999997	24.499670462035\\
88177.6999999998	24.5011783011143\\
88286.5000000003	24.5028124313316\\
88574.4999999999	24.5037516398606\\
88657.6000000002	24.5054857051333\\
88680.4000000006	24.5071915471834\\
88818.7000000004	24.5087751692209\\
89089.8000000007	24.5099612862962\\
89118.6999999996	24.5116630834347\\
89155.7999999996	24.513352453399\\
89196.9999999995	24.5115592323069\\
89359.5000000005	24.5132028343905\\
89384.9999999999	24.5113559194989\\
89537.4999999995	24.5098148710709\\
89691.9999999999	24.5082900277728\\
89747.2000000004	24.5100409705919\\
89898.9000000005	24.5085039878085\\
89977.6000000007	24.5067388919699\\
90076.1000000001	24.5049191684374\\
90159.0999999998	24.503212095937\\
90196.4000000006	24.5049419877711\\
90389.2000000004	24.5033427629155\\
90613.6999999994	24.5046560236962\\
90680.0999999998	24.506342068059\\
90874.3	24.5077821696759\\
91022.1000000006	24.5061094985619\\
91087.3000000001	24.5043771147272\\
91164.1999999997	24.502683616284\\
91294.9000000005	24.5042992496851\\
91509.0000000003	24.5029182890008\\
91676.0999999999	24.5013427694429\\
91731.8000000007	24.4996560629399\\
91824.8000000005	24.5014151934824\\
91904.1000000008	24.5032327140653\\
92025.0000000001	24.5048361805638\\
92171.3999999997	24.5032358158433\\
92201.7000000008	24.5049445889343\\
92378.2999999997	24.5065946151914\\
92465.4999999999	24.5048969562735\\
92539.2000000004	24.506560996247\\
92641.1000000002	24.5082101699892\\
92696.7000000006	24.5099075696248\\
92784.7999999998	24.5080826729349\\
92809.8000000007	24.5097775129593\\
93101.8999999999	24.5109664668858\\
93335.8000000004	24.5096474132676\\
93499.6000000007	24.511308592434\\
93520.6000000005	24.5094401560296\\
93605.5000000006	24.5111403591238\\
93674.1000000003	24.513046281687\\
93758.8000000005	24.5112730631439\\
93903.6999999995	24.5128525150207\\
93951	24.5111552711996\\
94078.5000000005	24.5095058812525\\
94430.7999999996	24.5100915060642\\
94504.2999999999	24.5117687642057\\
94604.2999999998	24.5097479609828\\
94789.7999999994	24.5114722138118\\
94811.3000000003	24.5131914516609\\
94875.2999999997	24.5114118095945\\
94917.4000000005	24.5131825005926\\
95112.2999999999	24.5117355886299\\
95189.8000000007	24.5134200161992\\
95228.0000000006	24.5151378637188\\
95269.4999999993	24.516844827731\\
95289.7000000007	24.518573866248\\
95321.5999999996	24.5168728631571\\
95433.7999999997	24.5185411054143\\
95637.5999999995	24.5202467227882\\
95687.2000000002	24.5185097708891\\
95834.9999999997	24.5169045579334\\
95887.6999999999	24.5152146571305\\
96037.8000000003	24.5136555464041\\
96082.0000000005	24.5154924798688\\
96193.1000000005	24.5171176340947\\
96251.0999999997	24.5153307179547\\
96399.6999999999	24.5136400697927\\
96452.1000000008	24.5153557928176\\
96487.6000000004	24.5171146166817\\
96629.7000000007	24.5187302467769\\
96710.8999999996	24.5170663109677\\
96855.0999999997	24.51546225706\\
97078.1999999995	24.5140263065999\\
97129.6999999995	24.5157510877197\\
97185.6000000008	24.5137916380702\\
97380.2000000002	24.515328048897\\
97543.5999999994	24.5171138679382\\
97577.3000000003	24.5154103306708\\
97682.9000000002	24.5170602868462\\
97791.5000000004	24.5153980505483\\
97889.0999999995	24.5171070965949\\
97954.0000000003	24.5153597450234\\
98114.8000000001	24.5138098290881\\
98203.3	24.5155463049141\\
98237.3000000009	24.5172408878899\\
98304.5000000005	24.5154346795572\\
98457.2999999994	24.5138506602856\\
98794.7	24.5130310502172\\
98876.8999999994	24.51470008192\\
98894.0999999995	24.5164023774903\\
99175.1000000009	24.5152013809904\\
99288.2000000008	24.5135630280708\\
99323.7999999998	24.5118244450731\\
99361.0000000003	24.5135168592135\\
99676.9999999998	24.512450703321\\
99743.2000000007	24.5141277659753\\
99864.9000000002	24.5124918640164\\
100000	24.5132754867245\\
};
\addlegendentry{PSK 4}

\addplot [color=mycolor2, forget plot]
  table[row sep=crcr]{%
0.1	100\\
0.2	66.6666666666667\\
0.3	50\\
0.4	60\\
0.5	66.6666666666667\\
0.6	71.4285714285714\\
0.699999999999989	62.5\\
0.8	66.6666666666667\\
0.900000000000001	70\\
1	63.6363636363636\\
1.10000000000001	66.6666666666667\\
1.2	69.2307692307692\\
1.29999999999999	64.2857142857143\\
1.39999999999998	66.6666666666667\\
1.5	68.75\\
1.6	64.7058823529412\\
1.8	68.4210526315789\\
2	71.4285714285714\\
2.20000000000002	65.2173913043478\\
2.3	62.5\\
2.4	64\\
2.5	61.5384615384615\\
2.7	64.2857142857143\\
2.79999999999996	62.0689655172414\\
3.10000000000004	65.625\\
3.39999999999996	68.5714285714286\\
3.50000000000006	69.4444444444444\\
3.6	67.5675675675676\\
3.79999999999997	69.2307692307692\\
3.89999999999997	67.5\\
4	68.2926829268293\\
4.09999999999995	66.6666666666667\\
4.5	69.5652173913043\\
4.8	71.4285714285714\\
4.9	70\\
5.19999999999996	71.6981132075472\\
5.40000000000001	69.0909090909091\\
5.90000000000006	71.6666666666667\\
6.49999999999995	74.2424242424242\\
6.90000000000001	75.7142857142857\\
6.99999999999989	74.6478873239437\\
7.5	76.3157894736842\\
7.59999999999994	75.3246753246753\\
7.69999999999994	75.6410256410256\\
7.79999999999994	74.6835443037975\\
8	75.3086419753086\\
8.10000000000001	74.390243902439\\
8.80000000000008	76.4044943820225\\
8.90000000000012	75.5555555555556\\
9.10000000000008	76.0869565217391\\
9.20000000000001	75.2688172043011\\
9.80000000000001	76.7676767676768\\
9.9000000000001	76\\
10	76.2376237623762\\
10.1000000000001	75.4901960784314\\
10.1999999999999	75.7281553398058\\
10.2999999999998	75\\
10.4999999999998	75.4716981132076\\
10.6	74.7663551401869\\
10.9000000000001	75.4545454545455\\
11.0000000000001	74.7747747747748\\
11.1999999999998	75.2212389380531\\
11.3999999999999	73.9130434782609\\
12.0999999999998	75.4098360655738\\
12.2000000000001	74.7967479674797\\
12.4000000000002	75.2\\
12.5	74.6031746031746\\
12.5999999999998	74.8031496062992\\
12.9000000000001	73.0769230769231\\
13.2000000000001	73.6842105263158\\
13.5999999999998	71.5328467153285\\
13.6999999999999	71.7391304347826\\
13.8999999999999	70.7142857142857\\
14.6000000000002	72.108843537415\\
14.7	71.6216216216216\\
14.9000000000002	72\\
15	71.523178807947\\
15.2999999999998	72.0779220779221\\
15.3999999999999	71.6129032258065\\
16.8999999999997	74.1176470588235\\
16.9999999999998	73.6842105263158\\
17.2000000000001	73.9884393063584\\
17.3	73.5632183908046\\
18.4	75.1351351351351\\
18.5	74.7311827956989\\
19.1000000000001	75.5208333333333\\
19.2	75.1295336787565\\
19.6999999999998	75.7575757575758\\
19.8000000000002	75.3768844221106\\
20.5999999999997	76.3285024154589\\
20.7	75.9615384615385\\
22.5	77.8761061946903\\
22.5999999999997	77.5330396475771\\
22.6999999999997	77.6315789473684\\
22.7999999999998	77.292576419214\\
23.2000000000003	77.6824034334764\\
23.3999999999998	77.0212765957447\\
23.6000000000002	77.2151898734177\\
23.7000000000002	76.890756302521\\
24.8999999999997	78\\
25	77.6892430278884\\
25.4999999999997	78.125\\
25.6	77.8210116731518\\
25.7000000000003	77.9069767441861\\
25.8000000000002	77.6061776061776\\
26.4000000000003	78.1132075471698\\
26.5	77.8195488721805\\
26.8999999999998	78.1481481481481\\
27	77.859778597786\\
27.6	78.3393501805054\\
27.7	78.0575539568345\\
27.9999999999996	78.2918149466192\\
28.1999999999996	77.7385159010601\\
28.7000000000001	78.125\\
28.8	77.8546712802768\\
29.0999999999998	78.0821917808219\\
29.2000000000004	77.8156996587031\\
29.2999999999999	77.891156462585\\
29.4	77.6271186440678\\
29.8000000000004	77.9264214046823\\
29.8999999999998	77.6666666666667\\
30.7999999999998	78.3171521035599\\
31.1999999999998	77.3162939297125\\
31.8999999999996	77.8125\\
32	77.5700934579439\\
32.4999999999997	77.9141104294479\\
32.7999999999996	77.2036474164134\\
32.9	77.2727272727273\\
33.0000000000003	77.0392749244713\\
33.5000000000003	77.3809523809524\\
33.5999999999995	77.1513353115727\\
34.1999999999997	77.5510204081633\\
34.2999999999995	77.3255813953488\\
36.5000000000005	78.6885245901639\\
36.6000000000003	78.4741144414169\\
36.8999999999996	78.6486486486486\\
37	78.4366576819407\\
37.7999999999994	78.8918205804749\\
37.9000000000001	78.6842105263158\\
38.2000000000002	78.8511749347258\\
38.2999999999996	78.6458333333333\\
38.7000000000002	78.8659793814433\\
38.7999999999997	78.6632390745501\\
38.9999999999997	78.772378516624\\
39.0999999999996	78.5714285714286\\
39.3000000000004	78.6802030456853\\
39.3999999999996	78.4810126582278\\
39.6999999999994	78.643216080402\\
39.8000000000006	78.4461152882206\\
40.4000000000002	78.7654320987654\\
40.5000000000001	78.5714285714286\\
40.5999999999998	78.6240786240786\\
40.7999999999995	78.239608801956\\
40.9999999999995	78.345498783455\\
41.0999999999996	78.1553398058252\\
41.5999999999997	78.4172661870504\\
41.8999999999995	77.8571428571429\\
42.0999999999999	77.9620853080569\\
42.1999999999999	77.7777777777778\\
42.6999999999997	78.0373831775701\\
42.8000000000002	77.8554778554778\\
43.0000000000002	77.9582366589327\\
43.0999999999998	77.7777777777778\\
43.5000000000005	77.9816513761468\\
43.8000000000007	77.4487471526196\\
44.3000000000001	77.7027027027027\\
44.4	77.5280898876404\\
45	77.8270509977827\\
45.1999999999995	77.4834437086093\\
45.2999999999996	77.5330396475771\\
45.3999999999994	77.3626373626374\\
45.8000000000001	77.5599128540305\\
45.8999999999995	77.3913043478261\\
46.1999999999997	77.5377969762419\\
46.4000000000005	77.2043010752688\\
46.7999999999997	77.3987206823028\\
46.8999999999997	77.2340425531915\\
48.2999999999993	77.8925619834711\\
48.3999999999993	77.7319587628866\\
49.6000000000007	78.2696177062374\\
49.7000000000003	78.1124497991968\\
49.7999999999994	78.1563126252505\\
50	77.8443113772455\\
50.3999999999992	78.019801980198\\
50.5000000000003	77.8656126482213\\
51.0999999999999	78.125\\
51.2000000000001	77.9727095516569\\
51.4999999999992	78.1007751937985\\
51.6999999999997	77.7992277992278\\
53.2000000000005	78.4240150093809\\
53.3000000000007	78.2771535580524\\
53.4000000000007	78.3177570093458\\
53.5000000000002	78.1716417910448\\
53.7999999999995	78.2931354359926\\
53.9000000000005	78.1481481481481\\
54.0999999999999	78.2287822878229\\
54.5000000000007	77.6556776556777\\
55.0000000000005	77.8584392014519\\
55.2999999999996	77.4368231046931\\
55.5999999999998	77.5583482944345\\
55.7000000000004	77.4193548387097\\
56.2000000000007	77.6198934280639\\
56.3000000000007	77.4822695035461\\
57.4000000000002	77.9130434782609\\
57.5	77.7777777777778\\
57.6999999999999	77.8546712802768\\
57.8000000000005	77.720207253886\\
58.0000000000007	77.7969018932874\\
58.1000000000002	77.6632302405498\\
59.5000000000003	78.1879194630873\\
59.6000000000009	78.0569514237856\\
59.7999999999996	78.1302170283806\\
59.9000000000006	78\\
60.1999999999994	78.1094527363184\\
60.3000000000006	77.9801324503311\\
60.4999999999991	78.052805280528\\
60.6000000000003	77.9242174629325\\
60.6999999999994	77.9605263157895\\
60.7999999999995	77.8325123152709\\
61.0000000000005	77.9050736497545\\
61.1000000000006	77.7777777777778\\
61.2999999999995	77.8501628664495\\
61.4000000000001	77.7235772357724\\
61.7999999999991	77.8675282714055\\
62.2000000000006	77.3675762439807\\
62.7000000000001	77.5477707006369\\
63.1000000000001	77.0569620253165\\
64.1000000000006	77.4143302180685\\
64.2000000000003	77.293934681182\\
64.6999999999991	77.4691358024691\\
64.8000000000001	77.3497688751926\\
64.8999999999991	77.3846153846154\\
64.9999999999995	77.2657450076805\\
66.2000000000003	77.6772247360483\\
66.3000000000009	77.5602409638554\\
66.3999999999992	77.593984962406\\
66.5000000000005	77.4774774774775\\
66.7999999999992	77.5784753363229\\
66.8999999999993	77.4626865671642\\
67.0000000000006	77.4962742175857\\
67.099999999999	77.3809523809524\\
67.2999999999999	77.4480712166172\\
67.399999999999	77.3333333333333\\
67.9999999999992	77.5330396475771\\
68.0999999999992	77.4193548387097\\
68.8999999999993	77.6811594202899\\
69.1000000000003	77.4566473988439\\
69.2	77.4891774891775\\
69.400000000001	77.2661870503597\\
69.4999999999997	77.2988505747126\\
69.6000000000008	77.1879483500717\\
69.7000000000006	77.2206303724928\\
69.8000000000006	77.1101573676681\\
70.2999999999994	77.2727272727273\\
70.4999999999989	77.0538243626062\\
70.6000000000001	77.0862800565771\\
70.6999999999992	76.9774011299435\\
71.5000000000001	77.2346368715084\\
71.5999999999999	77.1269177126918\\
71.7999999999995	77.1905424200278\\
72.0000000000001	76.9764216366158\\
72.1000000000001	77.0083102493075\\
72.2000000000012	76.9017980636238\\
72.6999999999998	77.0604395604396\\
72.8000000000007	76.9547325102881\\
72.9000000000002	76.986301369863\\
73.0000000000011	76.8809849521204\\
73.3000000000003	76.9754768392371\\
73.5000000000001	76.7663043478261\\
74.400000000001	77.0469798657718\\
74.5000000000011	76.9436997319035\\
75.0999999999996	77.1276595744681\\
75.2999999999998	76.9230769230769\\
75.6999999999996	77.0448548812665\\
75.8000000000003	76.9433465085639\\
76.7000000000001	77.2135416666667\\
76.8000000000001	77.1131339401821\\
77.1999999999996	77.2315653298836\\
77.3000000000008	77.1317829457364\\
78.300000000001	77.4234693877551\\
78.4000000000001	77.3248407643312\\
78.7999999999993	77.4397972116603\\
78.8999999999996	77.3417721518987\\
79.8000000000007	77.5969962453066\\
80	77.4032459425718\\
80.2999999999992	77.4875621890547\\
80.4000000000008	77.3913043478261\\
80.4999999999987	77.4193548387097\\
80.6000000000004	77.3234200743494\\
80.8000000000005	77.3794808405439\\
80.8999999999994	77.2839506172839\\
81.0000000000001	77.3119605425401\\
81.0999999999988	77.2167487684729\\
81.1999999999996	77.2447724477245\\
81.300000000001	77.1498771498772\\
81.4000000000007	77.1779141104294\\
81.4999999999987	77.0833333333333\\
81.5999999999991	77.1113831089351\\
81.6999999999998	77.0171149144254\\
81.9000000000008	77.0731707317073\\
81.999999999999	76.979293544458\\
82.3	77.0631067961165\\
82.5000000000007	76.8765133171913\\
82.5999999999994	76.9044740024184\\
82.6999999999989	76.8115942028985\\
82.8000000000001	76.8395657418577\\
82.8999999999988	76.7469879518072\\
83.299999999999	76.8585131894484\\
83.3999999999997	76.7664670658683\\
83.6000000000001	76.8219832735962\\
83.7000000000012	76.7303102625298\\
84.300000000001	76.8957345971564\\
84.4999999999987	76.7139479905437\\
86.1999999999996	77.1726535341831\\
86.2999999999991	77.0833333333333\\
86.4000000000001	77.1098265895954\\
86.5999999999986	76.9319492502884\\
86.7000000000008	76.9585253456221\\
86.7999999999998	76.8699654775604\\
86.8999999999987	76.8965517241379\\
87.000000000001	76.8082663605052\\
87.2000000000011	76.8613974799542\\
87.2999999999994	76.7734553775744\\
87.6000000000013	76.8529076396807\\
87.6999999999998	76.7653758542141\\
88.5000000000009	76.9751693002257\\
88.7000000000001	76.8018018018018\\
89.4999999999998	77.0089285714286\\
89.5999999999986	76.9230769230769\\
90.399999999999	77.1270718232044\\
90.5000000000009	77.0419426048565\\
90.5999999999992	77.0672546857773\\
90.7999999999989	76.8976897689769\\
91.0000000000008	76.9484083424808\\
91.0999999999992	76.8640350877193\\
91.2999999999997	76.9146608315098\\
91.3999999999987	76.8306010928962\\
92.1000000000001	77.0065075921909\\
92.3999999999994	76.7567567567568\\
92.899999999999	76.8817204301075\\
93.0000000000013	76.7991407089151\\
93.1999999999998	76.8488745980707\\
93.4000000000005	76.6844919786096\\
94.2000000000011	76.8822905620361\\
94.2999999999989	76.8008474576271\\
94.4000000000009	76.8253968253968\\
94.6000000000014	76.6631467793031\\
94.7000000000001	76.6877637130802\\
94.9000000000006	76.5263157894737\\
95.1	76.5756302521009\\
95.2000000000005	76.495278069255\\
95.5000000000005	76.5690376569038\\
95.6000000000015	76.4890282131661\\
95.7999999999997	76.5380604796663\\
95.9000000000006	76.4583333333333\\
96.3000000000005	76.5560165975104\\
96.400000000001	76.4766839378238\\
97.4999999999992	76.7418032786885\\
97.6000000000009	76.6632548618219\\
98.1999999999985	76.8056968463886\\
98.4999999999991	76.5720081135903\\
99.4000000000007	76.78391959799\\
99.5999999999988	76.629889669007\\
99.8000000000001	76.6766766766767\\
99.9000000000001	76.6\\
100	76.6233766233766\\
100.099999999999	76.5469061876248\\
100.4	76.6169154228856\\
100.500000000001	76.5407554671968\\
100.600000000001	76.5640516385303\\
100.799999999999	76.412289395441\\
100.9	76.4356435643564\\
101.000000000001	76.3600395647873\\
101.499999999999	76.4763779527559\\
101.699999999999	76.3261296660118\\
101.899999999999	76.3725490196078\\
102.099999999999	76.2230919765166\\
102.4	76.2926829268293\\
102.499999999999	76.2183235867446\\
102.800000000001	76.287657920311\\
102.899999999998	76.2135922330097\\
103.3	76.3056092843327\\
103.399999999999	76.231884057971\\
103.5	76.2548262548263\\
103.600000000002	76.1812921890068\\
104.199999999999	76.3183125599233\\
104.5	76.0994263862333\\
105.000000000002	76.2131303520457\\
105.1	76.1406844106464\\
105.4	76.2085308056872\\
105.5	76.1363636363636\\
105.600000000001	76.158940397351\\
105.700000000001	76.0869565217391\\
105.9	76.1320754716981\\
106	76.0603204524034\\
106.299999999999	76.1278195488722\\
106.499999999999	75.984990619137\\
107.100000000001	76.1194029850746\\
107.200000000001	76.0484622553588\\
107.800000000001	76.1816496756256\\
107.900000000001	76.1111111111111\\
108.2	76.1772853185596\\
108.300000000002	76.1070110701107\\
108.400000000001	76.1290322580645\\
108.5	76.0589318600368\\
108.600000000001	76.0809567617295\\
108.700000000001	76.0110294117647\\
108.899999999998	76.0550458715596\\
109.000000000001	75.9853345554537\\
109.500000000002	76.0948905109489\\
109.699999999998	75.9562841530055\\
110.000000000001	76.0217983651226\\
110.100000000001	75.9528130671506\\
110.500000000001	76.0397830018083\\
110.699999999999	75.9025270758123\\
110.8	75.9242560865645\\
110.899999999999	75.8558558558559\\
111.100000000002	75.8992805755396\\
111.2	75.8310871518419\\
111.900000000001	75.9821428571428\\
111.999999999998	75.914362176628\\
112.300000000002	75.9786476868327\\
112.400000000001	75.9111111111111\\
112.7	75.9751773049645\\
112.799999999998	75.9078830823738\\
112.900000000001	75.929203539823\\
112.999999999999	75.8620689655172\\
113.399999999998	75.9471365638767\\
113.499999999999	75.8802816901408\\
114.299999999999	76.0489510489511\\
114.4	75.9825327510917\\
114.5	76.0034904013962\\
114.600000000001	75.9372275501308\\
115.4	76.1038961038961\\
115.499999999999	76.038062283737\\
116.000000000001	76.1412575366064\\
116.2	76.0103181427343\\
116.399999999999	76.0515021459227\\
116.600000000001	75.9211653813196\\
117.1	76.0238907849829\\
117.2	75.9590792838875\\
117.400000000001	76\\
117.499999999998	75.9353741496599\\
117.700000000002	75.9762308998302\\
117.800000000001	75.9117896522477\\
117.900000000001	75.9322033898305\\
118.000000000001	75.8679085520745\\
118.700000000001	76.010101010101\\
118.800000000001	75.946173254836\\
120.099999999999	76.2063227953411\\
120.2	76.1429758935993\\
120.299999999999	76.1627906976744\\
120.399999999999	76.0995850622407\\
120.500000000001	76.1194029850746\\
120.600000000001	76.056338028169\\
121.100000000002	76.1551155115512\\
121.200000000001	76.0923330585326\\
121.799999999999	76.2100082034454\\
121.9	76.1475409836066\\
122.3	76.2254901960784\\
122.399999999999	76.1632653061225\\
122.5	76.1827079934747\\
122.599999999999	76.120619396903\\
122.8	76.1594792514239\\
122.999999999999	76.0357432981316\\
123.599999999998	76.1519805982215\\
123.699999999999	76.0904684975767\\
124.400000000001	76.2248995983936\\
124.499999999998	76.1637239165329\\
124.799999999999	76.2209767814251\\
125	76.0991207034373\\
125.4	76.1752988047809\\
125.5	76.1146496815287\\
126.6	76.3220205209155\\
126.699999999999	76.2618296529969\\
126.8	76.2805358550039\\
126.899999999998	76.2204724409449\\
126.999999999999	76.2391817466562\\
127.1	76.1792452830189\\
127.399999999999	76.2352941176471\\
127.499999999999	76.1755485893417\\
127.700000000001	76.2128325508607\\
127.799999999999	76.1532447224394\\
127.9	76.171875\\
128	76.1124121779859\\
128.500000000001	76.2052877138414\\
128.600000000002	76.1460761460761\\
129.000000000001	76.2199845081332\\
129.099999999999	76.1609907120743\\
129.500000000002	76.2345679012346\\
129.6	76.1757902852737\\
130.3	76.3036809815951\\
130.399999999998	76.2452107279694\\
130.500000000002	76.2633996937213\\
130.6	76.2050497322112\\
131.3	76.3318112633181\\
131.400000000002	76.2737642585551\\
131.499999999999	76.2917933130699\\
131.799999999999	76.1182714177407\\
131.900000000001	76.1363636363636\\
132.000000000001	76.0787282361847\\
132.200000000001	76.1148904006047\\
132.400000000001	76\\
132.799999999998	76.0722347629797\\
133.099999999999	75.9009009009009\\
133.599999999998	75.9910246821241\\
133.699999999999	75.9342301943199\\
134.799999999998	76.1304670126019\\
134.900000000002	76.0740740740741\\
135.3	76.1447562776957\\
135.399999999998	76.0885608856089\\
135.599999999998	76.1238025055269\\
135.700000000001	76.0677466863034\\
135.999999999998	76.1204996326231\\
136.299999999999	75.9530791788856\\
136.700000000002	76.0233918128655\\
136.799999999999	75.9678597516435\\
137.4	76.0727272727273\\
137.600000000001	75.9622367465505\\
138.099999999998	76.0492040520984\\
138.3	75.9393063583815\\
138.800000000002	76.0259179265659\\
138.899999999998	75.9712230215827\\
139.1	76.0057471264368\\
139.200000000002	75.9511844938981\\
140.000000000002	76.0885082084226\\
140.100000000001	76.0342368045649\\
140.300000000001	76.0683760683761\\
140.399999999999	76.0142348754448\\
140.599999999999	76.0483297796731\\
140.800000000001	75.9403832505323\\
141.499999999998	76.0593220338983\\
141.700000000001	75.9520451339915\\
142.300000000002	76.0533707865169\\
142.600000000002	75.8934828311142\\
142.900000000001	75.9440559440559\\
143	75.8909853249476\\
143.599999999999	75.9916492693111\\
143.7	75.9388038942976\\
144.099999999998	76.005547850208\\
144.2	75.952875952876\\
144.600000000002	76.0193503800968\\
144.7	75.9668508287293\\
144.800000000002	75.9834368530021\\
144.899999999998	75.9310344827586\\
145.000000000002	75.947622329428\\
145.1	75.8953168044077\\
145.3	75.9284731774415\\
145.4	75.8762886597938\\
145.8	75.9424263193968\\
145.899999999999	75.8904109589041\\
146.000000000002	75.9069130732375\\
146.100000000002	75.8549931600547\\
146.900000000002	75.9863945578231\\
147.1	75.883152173913\\
147.2	75.8995247793619\\
147.299999999998	75.8480325644505\\
148.199999999998	75.9946055293324\\
148.299999999999	75.9433962264151\\
148.400000000002	75.959595959596\\
148.6	75.857431069267\\
148.7	75.8736559139785\\
148.800000000002	75.8226997985225\\
149.300000000001	75.9036144578313\\
149.399999999998	75.8528428093645\\
149.6	75.8851035404142\\
149.7	75.8344459279039\\
150.6	75.9787657597877\\
150.7	75.9283819628647\\
150.900000000001	75.9602649006623\\
150.999999999999	75.9099933818663\\
151.600000000001	76.0052735662492\\
151.699999999998	75.9552042160738\\
151.900000000002	75.9868421052632\\
151.999999999999	75.9368836291913\\
152.900000000001	76.078431372549\\
152.999999999998	76.0287393860222\\
153.100000000002	76.0443864229765\\
153.199999999999	75.9947814742335\\
153.5	76.0416666666667\\
153.6	75.9921925829538\\
153.700000000001	76.0078023407022\\
153.799999999999	75.9584145549058\\
154.499999999998	76.0672703751617\\
154.600000000002	76.0180995475113\\
155.500000000001	76.1568123393316\\
155.600000000002	76.1078998073218\\
155.9	76.1538461538461\\
156.100000000001	76.056338028169\\
156.299999999998	76.0869565217391\\
156.399999999998	76.038338658147\\
157.499999999998	76.2055837563452\\
157.599999999999	76.1572606214331\\
158.799999999998	76.3373190685966\\
159	76.2413576367065\\
159.100000000001	76.2562814070352\\
159.200000000003	76.2084118016321\\
159.900000000002	76.3125\\
160	76.264834478451\\
160.099999999999	76.2796504369538\\
160.299999999998	76.1845386533666\\
161.399999999999	76.3467492260062\\
161.500000000002	76.299504950495\\
162.699999999999	76.4742014742015\\
162.800000000001	76.427255985267\\
163.199999999998	76.4849969381506\\
163.299999999998	76.438188494492\\
163.500000000002	76.4669926650367\\
163.6	76.4202810018326\\
163.700000000001	76.4346764346764\\
163.800000000002	76.3880414887126\\
165.100000000002	76.5738498789346\\
165.199999999999	76.5275257108288\\
165.300000000001	76.5417170495768\\
165.399999999998	76.4954682779456\\
165.700000000001	76.5379975874548\\
165.799999999998	76.4918625678119\\
166.799999999999	76.6327142001198\\
166.899999999999	76.5868263473054\\
166.999999999998	76.6008378216637\\
167.100000000001	76.555023923445\\
167.3	76.5830346475508\\
167.500000000002	76.4916467780429\\
168.100000000001	76.5755053507729\\
168.199999999998	76.5300059417707\\
168.4	76.5578635014837\\
168.499999999998	76.5124555160142\\
168.600000000002	76.5263781861292\\
168.699999999999	76.4810426540284\\
168.900000000002	76.508875739645\\
168.999999999997	76.4636309875813\\
169.100000000001	76.4775413711584\\
169.499999999998	76.2971698113208\\
169.799999999997	76.3390229546792\\
169.9	76.2941176470588\\
170.899999999998	76.4327485380117\\
170.999999999999	76.3880771478667\\
171.200000000001	76.4156450671337\\
171.299999999998	76.3710618436406\\
171.500000000003	76.3986013986014\\
171.699999999999	76.3096623981374\\
171.800000000002	76.3234438627109\\
171.900000000001	76.2790697674419\\
172.1	76.3066202090592\\
172.200000000001	76.2623331398723\\
172.5	76.3035921205098\\
172.699999999998	76.2152777777778\\
173	76.2564991334489\\
173.200000000003	76.1684939411425\\
174.2	76.3052208835341\\
174.300000000001	76.2614678899082\\
174.7	76.3157894736842\\
174.799999999999	76.2721555174385\\
175.200000000003	76.3262977752425\\
175.4	76.2393162393162\\
176.100000000001	76.3337116912599\\
176.299999999999	76.2471655328798\\
176.6	76.287492925863\\
176.800000000002	76.2012436404748\\
177.000000000002	76.2281197063806\\
177.099999999999	76.1851015801354\\
177.299999999998	76.2119503945885\\
177.4	76.169014084507\\
177.800000000001	76.2225969645868\\
177.899999999998	76.1797752808989\\
178.299999999999	76.2331838565022\\
178.399999999998	76.1904761904762\\
178.600000000002	76.2171236709569\\
178.700000000002	76.1744966442953\\
178.800000000003	76.1878144214645\\
178.9	76.145251396648\\
179.299999999999	76.1984392419175\\
179.399999999999	76.1559888579387\\
180.799999999998	76.3405196241017\\
181.000000000002	76.2562120375483\\
181.100000000003	76.2693156732892\\
181.199999999998	76.2272476558191\\
181.700000000002	76.2926292629263\\
181.899999999999	76.2087912087912\\
182.199999999998	76.2479429511794\\
182.299999999998	76.2061403508772\\
182.899999999998	76.2841530054645\\
183.000000000002	76.2424904423812\\
183.200000000001	76.2684124386252\\
183.300000000002	76.226826608506\\
184.900000000002	76.4324324324324\\
185	76.3911399243652\\
185.3	76.4293419633226\\
185.399999999997	76.3881401617251\\
185.500000000003	76.4008620689655\\
185.600000000002	76.3597199784599\\
185.799999999998	76.3851533082302\\
185.899999999999	76.3440860215054\\
186.600000000002	76.4327798607391\\
186.700000000001	76.3918629550321\\
186.9	76.4171122994652\\
186.999999999999	76.376269374666\\
188.599999999998	76.5765765765766\\
188.800000000002	76.4955002646903\\
188.900000000002	76.5079365079365\\
188.999999999997	76.467477525119\\
189.3	76.5047518479409\\
189.4	76.4643799472296\\
189.500000000001	76.4767932489452\\
189.699999999999	76.3962065331928\\
189.800000000001	76.4086361242759\\
189.9	76.3684210526316\\
190.100000000003	76.3932702418507\\
190.499999999998	76.2329485834208\\
190.599999999998	76.2454116413215\\
190.8	76.1655316919853\\
191.099999999999	76.2029288702929\\
191.3	76.1233019853709\\
191.699999999998	76.1730969760167\\
191.899999999999	76.09375\\
192.900000000003	76.2176165803109\\
192.999999999999	76.1781460383221\\
193.099999999998	76.1904761904762\\
193.199999999997	76.1510605276772\\
193.399999999999	76.1757105943152\\
193.500000000001	76.1363636363636\\
194.200000000001	76.2223365928976\\
194.299999999998	76.1831275720165\\
194.500000000003	76.207605344296\\
194.600000000002	76.1684643040575\\
194.800000000002	76.1929194458697\\
194.899999999999	76.1538461538461\\
195.700000000002	76.2512768130746\\
195.9	76.1734693877551\\
196.399999999997	76.234096692112\\
196.500000000002	76.1953204476094\\
196.699999999999	76.219512195122\\
196.799999999998	76.1808024377857\\
196.900000000001	76.1928934010152\\
196.999999999998	76.154236428209\\
197.4	76.2025316455696\\
197.500000000002	76.163967611336\\
197.600000000003	76.1760242792109\\
197.699999999999	76.1375126390293\\
198.300000000002	76.2096774193548\\
198.400000000003	76.1712846347607\\
198.900000000003	76.2311557788945\\
199.000000000003	76.1928679055751\\
199.199999999998	76.2167586552935\\
199.300000000001	76.1785356068205\\
199.8	76.2381190595298\\
199.899999999999	76.2\\
200.699999999998	76.2948207171315\\
200.799999999999	76.2568442010951\\
201.800000000001	76.3744427934621\\
201.9	76.3366336633663\\
202.000000000001	76.3483424047501\\
202.199999999997	76.2728620860109\\
202.299999999999	76.2845849802372\\
202.400000000002	76.2469135802469\\
202.899999999997	76.3054187192118\\
202.999999999999	76.2678483505662\\
203.299999999999	76.3028515240905\\
203.399999999998	76.2653562653563\\
203.899999999997	76.3235294117647\\
203.999999999998	76.2861342479177\\
204.100000000001	76.2977473065622\\
204.199999999998	76.2604013705335\\
204.4	76.2836185819071\\
204.5	76.2463343108504\\
204.8	76.2811127379209\\
204.899999999999	76.2439024390244\\
205.4	76.301703163017\\
205.499999999998	76.2645914396887\\
205.700000000001	76.287657920311\\
205.799999999997	76.2506070908208\\
206.499999999999	76.3310745401742\\
206.6	76.2941461054669\\
206.699999999998	76.3056092843327\\
206.799999999999	76.2687288545191\\
207.100000000001	76.3030888030888\\
207.499999999997	76.1560693641619\\
207.899999999999	76.2019230769231\\
207.999999999998	76.1653051417588\\
208.1	76.1767531219981\\
208.200000000003	76.1401824291887\\
208.399999999998	76.1630695443645\\
208.499999999999	76.1265580057526\\
208.6	76.1379971250599\\
208.700000000003	76.1015325670498\\
208.800000000003	76.1129727142173\\
208.899999999998	76.0765550239234\\
209.200000000002	76.1108456760631\\
209.300000000002	76.0744985673352\\
209.499999999997	76.0973282442748\\
209.600000000002	76.0610395803529\\
209.999999999997	76.1066158971918\\
210.100000000003	76.0704091341579\\
210.700000000001	76.1385199240987\\
210.8	76.1024182076814\\
212.400000000002	76.2823529411765\\
212.499999999997	76.2464722483537\\
213.200000000003	76.3244256915143\\
213.4	76.2529274004684\\
214.400000000002	76.3636363636364\\
214.8	76.2214983713355\\
215.499999999999	76.2987012987013\\
215.7	76.2279888785913\\
215.900000000002	76.25\\
216	76.2147154095326\\
216.199999999997	76.2367082755432\\
216.3	76.2014787430684\\
216.600000000004	76.2344254730042\\
216.7	76.1992619926199\\
217.400000000001	76.2758620689655\\
217.500000000002	76.2408088235294\\
217.699999999998	76.2626262626263\\
218.000000000003	76.1577258138469\\
218.899999999998	76.2557077625571\\
219.100000000004	76.1861313868613\\
219.299999999999	76.207839562443\\
219.399999999997	76.1731207289294\\
219.700000000002	76.2056414922657\\
219.799999999999	76.1709868121874\\
219.900000000001	76.1818181818182\\
220.000000000002	76.1472058155384\\
220.100000000001	76.158038147139\\
220.200000000003	76.1234679981843\\
220.5	76.1559383499547\\
220.600000000002	76.121431807884\\
220.700000000004	76.1322463768116\\
220.900000000001	76.0633484162896\\
221.000000000003	76.0741745816373\\
221.099999999997	76.0397830018083\\
221.399999999998	76.0722347629797\\
221.6	76.0036084799278\\
222.7	76.1220825852783\\
222.800000000002	76.0879318079856\\
223.099999999998	76.1200716845878\\
223.200000000003	76.0859829825347\\
223.6	76.1287438533751\\
223.700000000001	76.09472743521\\
223.999999999997	76.1267291387773\\
224.099999999997	76.092774308653\\
224.300000000003	76.1140819964349\\
224.399999999999	76.0801781737194\\
224.499999999998	76.0908281389136\\
224.600000000003	76.0569648420116\\
225.099999999998	76.1101243339254\\
225.200000000003	76.0763426542388\\
225.800000000002	76.1398849048251\\
225.999999999997	76.0725342768686\\
226.200000000001	76.0936809544852\\
226.599999999999	75.9594177326864\\
226.799999999997	75.9806081974438\\
226.899999999999	75.9471365638767\\
227.199999999998	75.9788825340959\\
227.300000000001	75.9454705364996\\
228.099999999997	76.0297984224365\\
228.2	75.9964958388086\\
228.300000000003	76.0070052539404\\
228.499999999997	75.9405074365704\\
228.700000000001	75.9615384615385\\
228.8	75.9283529925732\\
229.300000000001	75.9808195292066\\
229.400000000003	75.9477124183007\\
230.1	76.0208514335361\\
230.200000000001	75.9878419452888\\
230.5	76.01908065915\\
230.699999999999	75.9532062391681\\
230.999999999998	75.9844223279965\\
231.300000000003	75.885911840968\\
232.000000000003	75.9586385178802\\
232.200000000002	75.8932414980628\\
232.400000000001	75.9139784946237\\
232.500000000003	75.8813413585555\\
232.599999999997	75.8917060593038\\
232.799999999998	75.8265349935595\\
233.099999999996	75.8576329331046\\
233.300000000003	75.7926306769494\\
233.700000000003	75.8340461933276\\
233.800000000001	75.8016246259085\\
234.999999999996	75.9251382390472\\
235.1	75.8928571428571\\
235.800000000003	75.9643916913947\\
235.9	75.9322033898305\\
236.8	76.0236386661038\\
237.199999999998	75.8954909397387\\
237.8	75.9562841530055\\
238.099999999997	75.8606213266163\\
238.199999999997	75.8707511540076\\
238.400000000004	75.8071278825996\\
238.700000000001	75.8375209380234\\
238.899999999998	75.7740585774059\\
239.300000000003	75.8145363408521\\
239.400000000002	75.7828810020877\\
239.900000000004	75.8333333333333\\
240	75.801749271137\\
240.1	75.811823480433\\
240.199999999999	75.7802746566792\\
240.599999999998	75.8205234732032\\
240.899999999998	75.7261410788382\\
242.400000000001	75.8762886597938\\
242.600000000001	75.8137618459003\\
243.099999999997	75.8634868421053\\
243.399999999996	75.7700205338809\\
244.000000000002	75.8295780417861\\
244.099999999996	75.7985257985258\\
245	75.8873929008568\\
245.099999999999	75.8564437194127\\
245.4	75.8859470468432\\
245.6	75.8241758241758\\
245.999999999997	75.8634701340918\\
246.100000000001	75.8326563769293\\
246.9	75.9109311740891\\
247.000000000004	75.880210441117\\
247.700000000004	75.9483454398709\\
247.799999999998	75.9177087535296\\
248.299999999999	75.9661835748792\\
248.4	75.9356136820926\\
248.599999999999	75.9549658222758\\
248.699999999996	75.9244372990354\\
248.800000000002	75.9341100843712\\
248.900000000001	75.9036144578313\\
248.999999999997	75.9132878362104\\
249.3	75.8219727345629\\
249.5	75.8413461538462\\
249.599999999998	75.8109731678014\\
249.899999999997	75.84\\
250	75.8096761295482\\
250.899999999997	75.8964143426295\\
250.999999999999	75.8661887694146\\
251.500000000002	75.9141494435612\\
251.599999999997	75.8839888756456\\
252.099999999997	75.9318001586043\\
252.299999999998	75.8716323296355\\
252.400000000001	75.8811881188119\\
252.500000000001	75.8511480601742\\
252.6	75.8607043925604\\
252.7	75.8306962025316\\
252.800000000002	75.8402530644524\\
252.900000000003	75.8102766798419\\
253.300000000001	75.8484609313339\\
253.399999999999	75.8185404339251\\
253.999999999997	75.8756395120031\\
254.100000000001	75.8457907159717\\
254.699999999996	75.9026687598116\\
254.999999999997	75.8134065072521\\
255.400000000001	75.8512720156556\\
255.499999999999	75.8215962441315\\
255.8	75.8499413833529\\
255.900000000004	75.8203125\\
256.100000000004	75.8391881342701\\
256.199999999998	75.8095981271947\\
257.000000000003	75.8848697005056\\
257.099999999998	75.855365474339\\
257.599999999996	75.9022118742724\\
257.700000000003	75.8727695888285\\
258.000000000002	75.9008136381248\\
258.099999999998	75.8714175058094\\
258.199999999997	75.8807588075881\\
258.400000000003	75.8220502901354\\
258.599999999999	75.8407421724005\\
258.700000000002	75.8114374034003\\
258.899999999998	75.8301158301158\\
259.000000000004	75.8008490930143\\
259.5	75.8474576271186\\
259.7	75.7890685142417\\
259.899999999997	75.8076923076923\\
259.999999999998	75.7785467128028\\
260.199999999997	75.7971571263926\\
260.399999999999	75.7389635316699\\
260.799999999996	75.7761594480644\\
260.900000000003	75.7471264367816\\
261.699999999996	75.8212375859435\\
261.800000000004	75.7922871324933\\
262.500000000004	75.8568164508759\\
262.599999999999	75.827940616673\\
262.899999999998	75.8555133079848\\
263.2	75.7690846942651\\
263.5	75.7966616084977\\
263.599999999998	75.7679180887372\\
263.699999999999	75.7771038665656\\
264.000000000003	75.6910261264672\\
264.3	75.7186081694402\\
264.400000000003	75.6899810964083\\
265.200000000004	75.763286845081\\
265.4	75.7062146892655\\
266.800000000003	75.8336455601349\\
267.000000000004	75.7768625982778\\
267.099999999998	75.7859281437126\\
267.199999999997	75.7575757575758\\
267.300000000003	75.7666417352281\\
267.399999999997	75.7383177570093\\
267.500000000001	75.7473841554559\\
267.599999999997	75.7190885319387\\
267.800000000003	75.7372153788727\\
267.900000000003	75.7089552238806\\
268.000000000003	75.7180156657963\\
268.200000000004	75.6615728661946\\
268.5	75.6887565152643\\
268.799999999996	75.6043138713276\\
268.999999999998	75.6224451876626\\
269.2	75.5662829558114\\
269.399999999998	75.5844155844156\\
269.500000000003	75.5563798219585\\
269.599999999996	75.5654430849092\\
269.800000000004	75.5094479436828\\
270.700000000001	75.5908419497784\\
271.000000000003	75.5071929177425\\
271.699999999998	75.570272259014\\
271.900000000004	75.5147058823529\\
272.299999999999	75.5506607929515\\
272.399999999997	75.5229357798165\\
272.799999999997	75.5588127519238\\
272.9	75.5311355311355\\
273.199999999999	75.5579948774241\\
273.299999999998	75.5303584491587\\
274.299999999997	75.6195335276968\\
274.399999999996	75.591985428051\\
274.699999999999	75.6186317321689\\
274.800000000001	75.5911240451073\\
274.899999999998	75.6\\
275.000000000002	75.5725190839695\\
275.500000000001	75.6168359941945\\
275.599999999997	75.5894087776569\\
275.700000000002	75.5982596084119\\
275.800000000002	75.5708590068866\\
276	75.5885548714234\\
276.100000000002	75.5611875452571\\
277.299999999998	75.6669069935112\\
277.400000000002	75.6396396396396\\
278.099999999996	75.7009345794392\\
278.199999999999	75.6737333812433\\
278.299999999996	75.6824712643678\\
278.400000000003	75.6552962298025\\
279.499999999999	75.7510729613734\\
279.6	75.7239899892742\\
279.799999999999	75.741336191497\\
280.000000000005	75.6872545519457\\
280.299999999999	75.7132667617689\\
280.400000000001	75.6862745098039\\
280.799999999998	75.7208971164115\\
280.899999999999	75.6939501779359\\
281.100000000001	75.7112375533428\\
281.399999999998	75.6305506216696\\
281.500000000003	75.6392045454545\\
281.600000000003	75.6123535676251\\
282.400000000001	75.6814159292035\\
282.499999999997	75.654635527247\\
283.300000000005	75.7233592095977\\
283.499999999996	75.6699576868829\\
284.400000000003	75.7469244288225\\
284.500000000002	75.720309205903\\
285.499999999997	75.8053221288515\\
285.600000000002	75.778788939447\\
285.700000000004	75.7872638208538\\
285.800000000001	75.7607555089192\\
286.100000000003	75.7861635220126\\
286.300000000005	75.7332402234637\\
286.599999999997	75.7586327171259\\
286.699999999998	75.7322175732218\\
287.700000000002	75.8165392633774\\
287.799999999996	75.7902049322681\\
288	75.8070114543561\\
288.199999999996	75.7544224765869\\
288.400000000001	75.7712305025996\\
288.5	75.7449757449757\\
288.599999999998	75.7533772081746\\
288.700000000004	75.7271468144044\\
289.399999999999	75.7858376511226\\
289.499999999998	75.7596685082873\\
289.699999999997	75.7763975155279\\
289.900000000004	75.7241379310345\\
290.399999999996	75.7659208261618\\
290.500000000001	75.7398485891259\\
290.899999999998	75.7731958762887\\
290.999999999998	75.7471659223635\\
291.399999999999	75.7804459691252\\
291.500000000002	75.7544581618656\\
291.600000000001	75.7627699691464\\
291.700000000002	75.7368060315284\\
292.599999999995	75.8114110010249\\
292.699999999999	75.7855191256831\\
293.100000000003	75.818553888131\\
293.200000000001	75.7927037163314\\
293.300000000001	75.8009543285617\\
293.399999999996	75.7751277683135\\
294.099999999996	75.8327668252889\\
294.199999999999	75.8069996602107\\
294.500000000001	75.8316361167685\\
294.599999999995	75.8059043094673\\
295.100000000004	75.8468834688347\\
295.199999999997	75.8211987809008\\
296	75.886524822695\\
296.100000000001	75.8609047940581\\
296.199999999996	75.8690516368545\\
296.499999999997	75.7923128792987\\
296.599999999998	75.8004718570947\\
296.700000000002	75.7749326145553\\
296.800000000005	75.7830919501516\\
296.899999999997	75.7575757575758\\
297.1	75.7738896366083\\
297.200000000001	75.7484022872519\\
300.299999999996	75.9986684420772\\
300.399999999999	75.973377703827\\
301.100000000004	76.0292164674635\\
301.300000000003	75.9787657597877\\
301.4	75.9867330016584\\
301.600000000001	75.9363606231356\\
302.899999999997	76.039603960396\\
303.000000000002	76.0145166611679\\
303.1	76.0224274406333\\
303.299999999996	75.9723137771918\\
304.199999999996	76.0433782451528\\
304.400000000004	75.9934318555008\\
304.499999999999	76.0013131976363\\
304.600000000001	75.9763702001969\\
304.700000000002	75.9842519685039\\
304.899999999998	75.9344262295082\\
305.199999999999	75.9580740255486\\
305.299999999999	75.9332023575639\\
305.399999999996	75.9410801963993\\
305.500000000003	75.9162303664921\\
306.399999999997	75.9869494290375\\
306.6	75.9373981089012\\
307	75.9687398241615\\
307.099999999996	75.9440104166667\\
307.200000000001	75.9518385942076\\
307.3	75.9271307742355\\
307.599999999999	75.9506012349691\\
307.7	75.9259259259259\\
307.799999999998	75.9337447223124\\
307.900000000004	75.9090909090909\\
308.400000000004	75.9481361426256\\
308.499999999996	75.9235255994815\\
308.699999999995	75.9391191709845\\
308.799999999998	75.9145354483652\\
308.899999999997	75.9223300970874\\
308.999999999996	75.8977677127143\\
309.099999999996	75.9055627425614\\
309.200000000003	75.881021661817\\
310.099999999996	75.9509993552547\\
310.199999999998	75.9265227199484\\
310.600000000005	75.9575152880592\\
310.700000000002	75.9330759330759\\
310.900000000005	75.9485530546624\\
311.000000000003	75.9241401478624\\
311.499999999999	75.9627727856226\\
311.700000000002	75.9140474663246\\
313.700000000004	76.0675589547483\\
314.100000000003	75.9707192870783\\
314.2	75.97836461979\\
314.400000000004	75.9300476947536\\
314.899999999998	75.968253968254\\
314.999999999995	75.9441447159632\\
316.400000000002	76.0505529225908\\
316.7	75.9785353535353\\
316.800000000003	75.9861154938466\\
317.100000000002	75.9142496847415\\
317.4	75.9370078740157\\
317.499999999996	75.9130982367758\\
318.600000000003	75.9962347034829\\
318.700000000003	75.9723964868256\\
318.999999999996	75.9949858978377\\
319.100000000001	75.9711779448622\\
319.299999999999	75.9862241703193\\
319.399999999997	75.962441314554\\
319.499999999997	75.9699624530663\\
319.600000000002	75.9461995620895\\
319.800000000004	75.9612378868396\\
319.900000000001	75.9375\\
320	75.9450171821306\\
320.100000000001	75.9212991880075\\
320.299999999995	75.936329588015\\
320.699999999997	75.8416458852868\\
321.100000000003	75.8717310087173\\
321.199999999997	75.8481170245876\\
321.500000000005	75.8706467661692\\
321.700000000001	75.8234928527035\\
321.999999999995	75.846010555728\\
322.1	75.8224705152079\\
322.300000000002	75.8374689826303\\
322.400000000002	75.8139534883721\\
322.500000000002	75.8214507129572\\
322.600000000001	75.79795475674\\
323.000000000004	75.8279170535438\\
323.099999999998	75.8044554455445\\
323.200000000002	75.8119393751933\\
323.300000000002	75.7884972170687\\
323.499999999995	75.8034610630408\\
323.700000000004	75.7566399011736\\
323.800000000005	75.7641247298549\\
323.899999999999	75.7407407407407\\
324.000000000001	75.7482258562172\\
324.100000000001	75.7248611967921\\
324.199999999998	75.7323465926611\\
324.299999999995	75.7090012330456\\
324.399999999995	75.7164869029276\\
324.499999999995	75.6931608133087\\
324.599999999999	75.7006467508469\\
324.699999999998	75.6773399014778\\
325.100000000003	75.7072570725707\\
325.200000000004	75.6839840147556\\
325.600000000003	75.713847098557\\
325.700000000004	75.6906077348066\\
325.800000000004	75.6980668916846\\
325.899999999999	75.6748466257669\\
326.599999999997	75.7269666360575\\
326.699999999995	75.703794369645\\
327.2	75.7409104796822\\
327.400000000003	75.6946564885496\\
327.500000000003	75.7020757020757\\
327.600000000003	75.6789746719561\\
328.300000000003	75.7308160779537\\
328.500000000005	75.6847230675593\\
328.900000000001	75.7142857142857\\
329	75.6912792464297\\
329.499999999998	75.7281553398058\\
329.599999999995	75.705186533212\\
330.200000000005	75.7493188010899\\
330.300000000004	75.726392251816\\
331.900000000004	75.8433734939759\\
331.999999999996	75.8205359831376\\
332.4	75.8496240601504\\
332.599999999998	75.8040276525398\\
333.599999999999	75.8765358106083\\
333.799999999998	75.8310871518419\\
333.900000000005	75.8383233532934\\
333.999999999996	75.8156240646513\\
334.200000000003	75.8300927310799\\
334.299999999998	75.8074162679426\\
334.700000000006	75.8363201911589\\
334.800000000005	75.8136757240967\\
334.900000000004	75.8208955223881\\
335.000000000003	75.7982691733811\\
335.399999999999	75.8271236959762\\
335.599999999999	75.7819481680071\\
336.599999999999	75.8538758538758\\
336.799999999999	75.8088453547047\\
337.299999999999	75.8446947243628\\
337.5	75.7997630331754\\
337.800000000004	75.8212488902042\\
337.899999999998	75.7988165680473\\
338.600000000006	75.8488337762031\\
338.700000000003	75.8264462809917\\
339.2	75.8620689655172\\
339.500000000003	75.7950530035336\\
339.800000000001	75.8164165931156\\
339.899999999998	75.7941176470588\\
340.4	75.8296622613803\\
340.499999999996	75.8073987081621\\
340.799999999997	75.8286887650337\\
340.899999999998	75.8064516129032\\
341.100000000002	75.820633059789\\
341.200000000005	75.7984178142397\\
341.499999999998	75.8196721311475\\
341.699999999999	75.7753071971913\\
341.799999999996	75.7823925124305\\
341.899999999995	75.7602339181286\\
341.999999999997	75.767319497223\\
342.199999999996	75.7230499561788\\
342.599999999996	75.7513860519405\\
342.700000000005	75.7292882147024\\
342.899999999996	75.7434402332361\\
343.1	75.6993006993007\\
343.200000000001	75.7063792601223\\
343.300000000001	75.6843331391963\\
343.7	75.7126236183828\\
343.800000000002	75.6906077348066\\
343.900000000001	75.6976744186046\\
344.000000000002	75.6756756756757\\
344.199999999999	75.6898054022655\\
344.400000000002	75.6458635703919\\
344.600000000004	75.6599941978532\\
344.699999999997	75.6380510440835\\
345	75.659229208925\\
345.199999999997	75.6154068925572\\
345.300000000002	75.6224667052692\\
345.399999999996	75.6005788712012\\
345.600000000001	75.6146948221001\\
345.8	75.5709742700202\\
346.100000000003	75.5921432697862\\
346.399999999995	75.5266955266955\\
347.000000000005	75.5690002881014\\
347.100000000002	75.5472350230415\\
347.7	75.5894192064405\\
347.799999999995	75.5676918654786\\
348.000000000004	75.5817293881069\\
348.099999999995	75.5600229753015\\
348.199999999996	75.5670399081252\\
348.300000000002	75.5453501722158\\
348.800000000005	75.5803955288048\\
348.899999999996	75.5587392550143\\
349.199999999998	75.5797308903521\\
349.299999999995	75.5580995993131\\
350.000000000006	75.6069694373036\\
350.100000000005	75.5853797829811\\
350.6	75.6201881950385\\
350.700000000002	75.5986316989738\\
351.799999999999	75.6749076442171\\
352.000000000003	75.631922749219\\
352.300000000001	75.6526674233825\\
352.4	75.6312056737589\\
352.499999999995	75.6381168462847\\
352.699999999998	75.5952380952381\\
353.400000000002	75.6435643564356\\
353.499999999996	75.6221719457014\\
354.099999999998	75.6634669678148\\
354.199999999998	75.6421112051933\\
354.999999999996	75.6969867642918\\
355.100000000003	75.6756756756757\\
355.600000000002	75.7098678661794\\
355.799999999997	75.6673222815398\\
356.799999999997	75.7355001400953\\
357.000000000002	75.6930831699804\\
357.600000000005	75.73385518591\\
357.799999999999	75.6915339480302\\
358.1	75.7118927973199\\
358.200000000006	75.6907619313425\\
358.399999999995	75.7043235704324\\
358.500000000005	75.6832124930284\\
358.599999999998	75.689991636465\\
358.699999999995	75.6688963210702\\
358.899999999997	75.6824512534819\\
358.999999999998	75.6613756613757\\
359.799999999998	75.7154765212559\\
359.899999999995	75.6944444444444\\
360.299999999999	75.7214206437292\\
360.399999999995	75.7004160887656\\
360.500000000001	75.7071547420965\\
360.600000000001	75.6861657887441\\
360.799999999999	75.6996397894154\\
361.000000000006	75.6577125450014\\
361.199999999996	75.6711873789095\\
361.399999999996	75.6293222683264\\
361.599999999996	75.6427978988112\\
361.700000000005	75.6218905472637\\
361.900000000004	75.6353591160221\\
362.100000000004	75.5935946990613\\
362.700000000002	75.6339581036384\\
362.999999999995	75.5714679151749\\
363.600000000002	75.6117679406104\\
363.799999999997	75.5702115965925\\
364.300000000005	75.6037321624588\\
364.500000000001	75.5622600109709\\
364.599999999997	75.5689607896902\\
364.700000000002	75.5482456140351\\
365.599999999995	75.6084222039923\\
365.7	75.587752870421\\
365.799999999996	75.5944247062039\\
365.899999999994	75.5737704918033\\
366.199999999998	75.5937755937756\\
366.300000000004	75.5731441048035\\
366.600000000004	75.5931278974639\\
366.699999999995	75.5725190839695\\
367.299999999998	75.6124115405553\\
367.399999999999	75.5918367346939\\
367.5	75.5984766050054\\
367.600000000002	75.5779167799837\\
367.799999999997	75.5911932590378\\
367.900000000004	75.570652173913\\
368	75.5772887802228\\
368.1	75.556762629006\\
368.899999999995	75.609756097561\\
368.999999999996	75.5892712002167\\
369.100000000004	75.5958829902492\\
369.200000000006	75.5754129434064\\
369.300000000002	75.5820249052518\\
369.400000000004	75.5615696887686\\
369.899999999997	75.5945945945946\\
370	75.5741691434747\\
370.299999999994	75.5939524838013\\
370.399999999995	75.5735492577598\\
370.6	75.5867278122471\\
370.699999999998	75.5663430420712\\
371.200000000004	75.5992458928091\\
371.299999999998	75.5788906838988\\
372.399999999997	75.6510067114094\\
372.699999999995	75.5901287553648\\
372.899999999999	75.6032171581769\\
373.000000000004	75.5829536317341\\
373.099999999999	75.5894962486602\\
373.200000000004	75.5692472542191\\
374.199999999996	75.6345177664975\\
374.300000000006	75.6143162393162\\
374.7	75.6403415154749\\
374.800000000001	75.620165377434\\
374.899999999994	75.6266666666667\\
375.099999999999	75.5863539445629\\
375.299999999999	75.5993606819393\\
375.399999999994	75.5792276964048\\
375.499999999998	75.5857294994675\\
375.6	75.5656108597285\\
375.900000000006	75.5851063829787\\
375.999999999994	75.5650093060356\\
376.499999999999	75.5974508762613\\
376.600000000003	75.5773825325192\\
376.700000000004	75.583864118896\\
376.800000000004	75.5638100291855\\
377.000000000001	75.5767700875099\\
377.099999999995	75.5567338282078\\
377.199999999996	75.5632122979062\\
377.399999999996	75.523178807947\\
377.999999999994	75.5620206294631\\
378.100000000003	75.5420412480169\\
379.600000000003	75.6386621016592\\
379.699999999997	75.6187467087941\\
379.899999999996	75.6315789473684\\
380.099999999998	75.5917937927407\\
380.4	75.611038107753\\
380.500000000005	75.5911718339464\\
380.699999999995	75.6039915966387\\
380.800000000002	75.5841428196377\\
381.9	75.6544502617801\\
382.000000000002	75.6346506150223\\
382.199999999997	75.6473973319383\\
382.3	75.6276150627615\\
382.799999999995	75.6594411073387\\
383.199999999999	75.5804852595878\\
384.099999999995	75.6376887038001\\
384.200000000004	75.6180067655477\\
384.499999999998	75.6370254810192\\
384.600000000004	75.6173641798804\\
384.899999999996	75.6363636363636\\
385.400000000002	75.538261997406\\
385.999999999998	75.5762755762756\\
386.100000000001	75.5567063697566\\
386.700000000003	75.5946225439504\\
386.799999999998	75.5750840010339\\
387.899999999999	75.6443298969072\\
387.999999999997	75.6248389590312\\
388.899999999998	75.6812339331619\\
389.000000000005	75.6617836031868\\
389.600000000004	75.699255837824\\
389.699999999994	75.6798358132376\\
389.799999999997	75.6860733521416\\
389.899999999996	75.6666666666667\\
390.100000000002	75.6791389031266\\
390.199999999998	75.659748911094\\
391.100000000002	75.7157464212679\\
391.199999999994	75.6963966266292\\
391.500000000005	75.7150153217569\\
391.599999999996	75.6956854735767\\
391.899999999999	75.7142857142857\\
392	75.6949757714869\\
392.3	75.7135575942915\\
392.400000000005	75.6942675159236\\
393.299999999997	75.7498729028978\\
393.399999999998	75.7306226175349\\
393.5	75.7367886178862\\
393.599999999996	75.7175514351029\\
393.999999999996	75.7421974118244\\
394.099999999998	75.7229832572298\\
394.200000000006	75.7291402485417\\
394.4	75.6907477820025\\
394.700000000002	75.709219858156\\
394.900000000001	75.6708860759494\\
395.700000000004	75.7200606366852\\
395.799999999996	75.7009345794392\\
395.900000000002	75.7070707070707\\
396.000000000004	75.6879575864681\\
396.099999999994	75.6940938919738\\
396.200000000001	75.6749936916477\\
396.600000000005	75.6995210486514\\
396.800000000006	75.6613756613757\\
397.200000000002	75.6858796878933\\
397.400000000005	75.6477987421384\\
398.299999999996	75.7028112449799\\
398.499999999994	75.6648268941295\\
399.000000000003	75.6953144575294\\
399.099999999997	75.6763527054108\\
399.2	75.6824442774856\\
399.299999999998	75.6634952428643\\
399.500000000002	75.6756756756757\\
399.600000000001	75.6567425569177\\
400	75.6810797300675\\
400.099999999998	75.6621689155422\\
400.600000000007	75.6925380583978\\
400.799999999995	75.6547767523073\\
400.999999999997	75.6669159810521\\
401.299999999995	75.6103637269557\\
402.700000000005	75.695134061569\\
402.799999999996	75.6763464879623\\
403.100000000003	75.6944444444444\\
403.300000000005	75.6569162121963\\
403.399999999995	75.6629491945477\\
403.499999999996	75.6442021803766\\
403.900000000006	75.6683168316832\\
404.000000000002	75.6495916852264\\
404.099999999997	75.6556160316675\\
404.199999999996	75.6369032896364\\
404.499999999997	75.6549678695008\\
404.800000000004	75.5989133119289\\
405.300000000005	75.6290083867785\\
405.400000000006	75.6103575832306\\
406.699999999995	75.6882989183874\\
406.799999999996	75.6696977144261\\
407.100000000004	75.6876227897839\\
407.299999999994	75.6504663721159\\
407.400000000004	75.6564417177914\\
407.500000000007	75.6378802747792\\
408.099999999996	75.6736893679569\\
408.200000000001	75.6551555228998\\
408.499999999999	75.6730298580519\\
408.699999999994	75.6360078277887\\
409	75.6538743583476\\
409.099999999995	75.6353861192571\\
409.199999999996	75.6413388712436\\
409.600000000001	75.5674884061508\\
409.799999999998	75.5794096121005\\
409.900000000005	75.5609756097561\\
409.999999999995	75.5669348939283\\
410.100000000006	75.5485129205266\\
410.399999999997	75.5663824604141\\
410.500000000006	75.5479785679493\\
410.599999999995	75.5539323106891\\
410.7	75.5355404089581\\
410.800000000001	75.5414942808469\\
410.999999999996	75.5047433714425\\
411.200000000005	75.5166545100899\\
411.400000000002	75.4799513973268\\
411.699999999999	75.4978144730452\\
411.800000000001	75.4794853119689\\
412.600000000006	75.52701720378\\
412.700000000001	75.5087209302326\\
412.899999999995	75.5205811138015\\
412.999999999997	75.5022996853062\\
413.700000000003	75.543740937651\\
413.800000000004	75.5254892486108\\
413.9	75.5314009661836\\
414.000000000001	75.5131610722048\\
414.099999999997	75.5190729116369\\
414.399999999993	75.4644149577805\\
414.9	75.4939759036145\\
414.999999999995	75.4757889665141\\
415.399999999996	75.4993983152828\\
415.5	75.4812319538017\\
415.600000000002	75.4871301419293\\
415.899999999994	75.4326923076923\\
416.400000000007	75.4621848739496\\
416.499999999995	75.4440710513682\\
416.699999999995	75.4558541266795\\
416.799999999996	75.4377548572799\\
417.199999999999	75.4612988257848\\
417.299999999999	75.4432199329181\\
417.499999999995	75.4549808429119\\
417.600000000005	75.4369164472109\\
418.700000000003	75.5014326647565\\
418.800000000003	75.4834089281452\\
418.999999999995	75.4951085659747\\
419.200000000005	75.4590984974958\\
419.299999999998	75.4649499284692\\
419.399999999999	75.4469606674613\\
419.800000000005	75.4703500833532\\
419.900000000002	75.4523809523809\\
420.100000000003	75.4640647310804\\
420.300000000003	75.4281636536632\\
420.699999999995	75.4515209125475\\
420.900000000004	75.4156769596199\\
420.999999999999	75.4215150795535\\
421.100000000005	75.4036087369421\\
421.400000000002	75.4211150652432\\
421.500000000005	75.4032258064516\\
421.600000000001	75.4090585724449\\
421.700000000004	75.3911806543386\\
422.499999999993	75.4377662091813\\
422.600000000001	75.4199195647031\\
423.300000000004	75.4605573925366\\
423.399999999997	75.4427390791027\\
423.500000000001	75.4485363550519\\
423.999999999999	75.3595850035369\\
425.100000000002	75.4233301975541\\
425.199999999998	75.4055960498472\\
425.300000000005	75.4113775270334\\
425.399999999997	75.3936545240893\\
425.5	75.3994360902256\\
425.700000000006	75.3640206669798\\
429.400000000005	75.5762514551804\\
429.500000000004	75.5586592178771\\
430.000000000002	75.5870727737735\\
430.099999999999	75.5695025569503\\
430.500000000002	75.59219693451\\
430.599999999996	75.574645925238\\
433.000000000007	75.7099976910644\\
433.300000000001	75.6575911398246\\
433.899999999996	75.6912442396313\\
433.999999999999	75.673807878369\\
434.599999999994	75.7073844030366\\
434.800000000002	75.6725684065302\\
434.899999999993	75.6781609195402\\
435.000000000005	75.6607676396231\\
435.699999999995	75.6998623221661\\
435.900000000001	75.6651376146789\\
436.499999999997	75.6985799358681\\
436.600000000004	75.6812457064346\\
436.800000000004	75.6923781185626\\
436.9	75.675057208238\\
437.699999999996	75.7195066240292\\
437.799999999997	75.7022151176068\\
438.300000000005	75.7299270072993\\
438.399999999996	75.7126567844926\\
439.099999999998	75.7513661202186\\
439.300000000003	75.7168866636322\\
439.700000000005	75.7389722601182\\
439.800000000002	75.7217549443055\\
440.000000000004	75.7327880027267\\
440.099999999994	75.7155838255339\\
440.400000000005	75.7321225879682\\
440.499999999999	75.7149341806627\\
440.800000000001	75.7314583805852\\
440.899999999997	75.7142857142857\\
441.000000000001	75.7197914305146\\
441.1	75.7026291931097\\
441.400000000007	75.7191392978482\\
441.5	75.7019927536232\\
441.600000000001	75.7074937740548\\
441.699999999994	75.6903576278859\\
442.000000000006	75.7068536530197\\
442.099999999994	75.6897331524197\\
442.399999999997	75.7062146892655\\
442.500000000004	75.6891098056936\\
442.699999999995	75.7000903342367\\
442.799999999995	75.6829984195078\\
443.2	75.7049402210693\\
443.300000000006	75.6878664862427\\
443.399999999999	75.6933483652762\\
443.5	75.6762849413886\\
443.599999999998	75.6817669596574\\
443.7	75.6647138350608\\
444	75.6811528934925\\
444.100000000007	75.6641152633949\\
444.299999999994	75.6750675067507\\
444.400000000006	75.6580427446569\\
444.899999999998	75.685393258427\\
445.099999999996	75.6513926325247\\
445.299999999996	75.6623259991019\\
445.500000000005	75.6283662477558\\
445.600000000003	75.6338344177698\\
445.699999999995	75.6168685509197\\
445.900000000004	75.627802690583\\
445.999999999995	75.6108495852948\\
446.199999999997	75.6217790723728\\
446.300000000007	75.6048387096774\\
446.400000000006	75.6103023516237\\
446.500000000004	75.5933721450963\\
446.9	75.6152125279642\\
447.199999999999	75.5644980997094\\
448.199999999995	75.619005130493\\
448.299999999999	75.6021409455843\\
448.699999999997	75.6238859180036\\
448.799999999999	75.6070394297171\\
449.300000000005	75.6341789052069\\
449.399999999995	75.6173526140156\\
449.799999999996	75.6390308957546\\
449.900000000004	75.6222222222222\\
450	75.6276383025994\\
450.1	75.6108396268325\\
450.199999999995	75.616255829447\\
450.3	75.5994671403197\\
450.7	75.6211180124224\\
450.899999999995	75.5875831485588\\
451.400000000004	75.6146179401993\\
451.499999999995	75.5978742249779\\
452.599999999998	75.6571681024961\\
452.700000000004	75.6404593639576\\
453.199999999997	75.6673284800353\\
453.400000000005	75.6339581036384\\
453.599999999994	75.6446991404011\\
453.7	75.6280299691494\\
454.199999999998	75.6548536209553\\
454.300000000001	75.6382042253521\\
455.000000000004	75.6756756756757\\
455.099999999995	75.6590509666081\\
455.299999999995	75.6697408871322\\
455.499999999996	75.6365232660228\\
455.900000000005	75.6578947368421\\
455.999999999997	75.64130673098\\
456.700000000004	75.6786339754816\\
456.799999999995	75.6620704749398\\
457.400000000003	75.6939890710382\\
457.500000000004	75.6774475524475\\
457.700000000007	75.6880733944954\\
457.799999999999	75.6715440052413\\
458.600000000002	75.7139742751254\\
458.700000000002	75.6974716652136\\
458.800000000006	75.70276748747\\
458.899999999997	75.6862745098039\\
459.099999999996	75.6968641114983\\
459.200000000002	75.6803831918136\\
460.200000000001	75.7332174668694\\
460.3	75.7167680278019\\
460.800000000001	75.7431113039705\\
460.899999999998	75.7266811279826\\
461.899999999998	75.7792207792208\\
462.099999999993	75.7464301168325\\
463.399999999995	75.8144552319309\\
463.499999999993	75.7981018119068\\
463.700000000005	75.8085381630013\\
463.900000000002	75.7758620689655\\
464.299999999998	75.7967269595177\\
464.499999999995	75.7640981489453\\
465.000000000006	75.7901526553429\\
465.100000000001	75.7738607050731\\
465.399999999996	75.7894736842105\\
465.499999999996	75.7731958762887\\
465.599999999997	75.7783981103715\\
465.800000000005	75.7458682120627\\
466.199999999993	75.7666738151405\\
466.400000000004	75.7341907824223\\
466.500000000004	75.7393913416202\\
466.600000000005	75.7231626312406\\
466.800000000006	75.7335617905333\\
467.000000000003	75.7011346606722\\
467.1	75.7063356164384\\
467.200000000007	75.690134817034\\
468.699999999993	75.7679180887372\\
468.799999999999	75.7517594369802\\
468.999999999997	75.7620976337668\\
469.099999999993	75.7459505541347\\
469.700000000001	75.776926351639\\
469.800000000006	75.760800170249\\
470.2	75.7814161173719\\
470.400000000001	75.7492029755579\\
471.400000000002	75.8006362672322\\
471.499999999994	75.7845631891433\\
472.099999999993	75.8153324862346\\
472.499999999993	75.7511637748625\\
472.600000000008	75.7562936323249\\
472.699999999998	75.7402707275804\\
472.799999999996	75.7454007189681\\
472.899999999994	75.7293868921776\\
473.100000000006	75.7396449704142\\
473.200000000001	75.7236425100359\\
473.6	75.7441418619379\\
473.700000000004	75.7281553398058\\
473.900000000001	75.7383966244726\\
474.100000000002	75.7064529734289\\
475.6	75.7830565482447\\
475.7	75.7671290458176\\
477.799999999997	75.8736137267211\\
478.000000000007	75.8418740849195\\
478.299999999999	75.8570234113712\\
478.499999999994	75.825323861262\\
478.700000000006	75.8354218880535\\
478.800000000004	75.8195865525162\\
478.999999999999	75.829680651221\\
479.099999999995	75.813856427379\\
479.700000000006	75.8441017090454\\
479.800000000008	75.8282975619921\\
480.299999999997	75.8534554537885\\
480.399999999998	75.837669094693\\
480.899999999994	75.8627858627859\\
480.999999999996	75.8470172521305\\
482.000000000005	75.8971167807509\\
482.100000000001	75.8813770219826\\
483.3	75.94124948283\\
483.400000000002	75.9255429162358\\
483.600000000003	75.9354972090138\\
483.799999999999	75.9041124199215\\
484.700000000005	75.9488448844885\\
484.900000000001	75.9175257731959\\
485.399999999997	75.9423274974253\\
485.500000000002	75.9266886326194\\
485.599999999995	75.9316450483838\\
485.700000000008	75.916014820914\\
486.100000000002	75.9358288770053\\
486.299999999996	75.9046052631579\\
486.399999999996	75.9095580678315\\
486.500000000006	75.8939580764488\\
487.699999999992	75.9532595325953\\
487.899999999996	75.922131147541\\
488.400000000005	75.9467758444217\\
488.800000000005	75.8846389854776\\
488.999999999993	75.8945001022286\\
489.200000000002	75.8634784385857\\
489.300000000007	75.8684102983245\\
489.399999999997	75.85291113381\\
489.499999999995	75.8578431372549\\
489.599999999995	75.8423524606902\\
489.699999999998	75.8472846059616\\
489.799999999995	75.8318024086548\\
490.199999999999	75.8515194778707\\
490.300000000002	75.8360522022839\\
491.000000000008	75.8704948075748\\
491.100000000005	75.8550488599349\\
491.200000000001	75.8599633625076\\
491.299999999999	75.8445258445258\\
491.799999999998	75.869079081114\\
491.999999999994	75.8382442592969\\
492.299999999993	75.8529650690495\\
492.400000000002	75.8375634517766\\
492.499999999995	75.8424685343078\\
492.600000000007	75.8270752993708\\
492.900000000006	75.841784989858\\
493	75.8264043804502\\
493.100000000007	75.8313057583131\\
493.199999999995	75.8159335090209\\
493.300000000004	75.8208350222943\\
493.399999999995	75.8054711246201\\
493.599999999993	75.8152724326514\\
493.699999999999	75.7999189955448\\
493.9	75.8097165991903\\
494.000000000008	75.7943736085813\\
494.100000000005	75.7992715499798\\
494.199999999993	75.783936880437\\
494.500000000003	75.7986251516377\\
494.700000000007	75.767987065481\\
494.900000000003	75.7777777777778\\
495.000000000005	75.7624722278328\\
495.1	75.767366720517\\
495.199999999993	75.7520694528569\\
495.599999999997	75.7716360702038\\
495.700000000005	75.7563533682937\\
496.400000000001	75.7905337361531\\
496.500000000002	75.7752718485703\\
496.800000000001	75.7898973636547\\
496.899999999997	75.7746478873239\\
497.199999999999	75.7892620148804\\
497.300000000002	75.7740249296341\\
497.500000000008	75.7837620578778\\
497.600000000005	75.7685352622061\\
497.800000000001	75.7782687286604\\
497.900000000007	75.7630522088353\\
498.200000000008	75.7776439895645\\
498.3	75.7624398073836\\
499.500000000001	75.8206565252202\\
499.699999999993	75.7903161264506\\
499.899999999994	75.8\\
500.099999999995	75.7696921231507\\
500.699999999997	75.7987220447284\\
500.900000000002	75.7684630738523\\
501.100000000006	75.7781324820431\\
501.200000000008	75.7630161579892\\
501.600000000001	75.7823400438509\\
501.700000000005	75.7672379434037\\
501.799999999993	75.7720661486352\\
501.9	75.7569721115538\\
501.999999999998	75.7618004381597\\
502.100000000001	75.7467144563919\\
502.600000000005	75.7708374776208\\
502.799999999994	75.7407039172798\\
503.300000000008	75.7647993643226\\
503.499999999995	75.7347100873709\\
503.599999999996	75.7395274965257\\
503.700000000008	75.7244938467646\\
504.100000000007	75.7437524791749\\
504.199999999994	75.7287328970851\\
505.199999999999	75.7767662774589\\
505.299999999999	75.7617728531856\\
505.4	75.7665677546983\\
505.600000000005	75.7366027288906\\
506.099999999997	75.760568945081\\
506.200000000008	75.7456053723089\\
506.499999999997	75.759968416897\\
506.700000000007	75.7300710339384\\
506.899999999995	75.7396449704142\\
506.999999999992	75.724709130349\\
507.999999999994	75.7724857311553\\
508.099999999992	75.7575757575758\\
508.400000000001	75.771878072763\\
508.499999999994	75.7569799449469\\
508.600000000005	75.7617456261058\\
508.700000000004	75.7468553459119\\
508.900000000007	75.7563850687623\\
508.999999999993	75.741504615989\\
509.200000000001	75.7510308266248\\
509.299999999997	75.736160188457\\
509.700000000001	75.7551981169086\\
509.799999999995	75.7403412433811\\
510.6	75.7783434501664\\
510.699999999992	75.7635082223962\\
510.900000000002	75.7729941291585\\
510.999999999999	75.7581686558403\\
511.199999999996	75.7676510854684\\
511.300000000001	75.7528353539304\\
511.800000000007	75.7765188513382\\
511.900000000005	75.76171875\\
512.000000000001	75.7664518648701\\
512.299999999999	75.7220921155347\\
512.800000000005	75.7457594072919\\
512.899999999995	75.7309941520468\\
513.700000000007	75.7687816270923\\
513.799999999998	75.7540377505351\\
513.899999999994	75.7587548638132\\
514.000000000006	75.7440186734098\\
514.199999999997	75.7534512930196\\
514.299999999998	75.7387247278383\\
514.900000000002	75.7669902912621\\
514.999999999992	75.752281110464\\
515.099999999997	75.7569875776398\\
515.199999999992	75.7422860469629\\
515.499999999999	75.7564003103181\\
515.700000000003	75.7270259790617\\
515.8	75.7317309556116\\
516.000000000003	75.7023832590583\\
516.100000000006	75.7070902750872\\
516.299999999999	75.6777691711851\\
516.399999999995	75.6824782187802\\
516.5	75.6678281068525\\
517.199999999998	75.7007539145563\\
517.300000000007	75.6861229223038\\
518.099999999994	75.7236588189888\\
518.300000000002	75.6944444444444\\
518.499999999998	75.7038179714616\\
518.6	75.6892230576441\\
518.7	75.6939090208173\\
518.799999999998	75.6793216419349\\
519.799999999994	75.7261011733026\\
519.899999999992	75.7115384615385\\
520.299999999995	75.7302075326672\\
520.600000000007	75.6865757633954\\
521.300000000009	75.7192174913694\\
521.399999999993	75.7046979865772\\
522.299999999994	75.7465543644717\\
522.4	75.7320574162679\\
522.600000000001	75.7413430265927\\
522.800000000002	75.7123733027347\\
522.900000000002	75.717017208413\\
523.000000000004	75.7025425348882\\
523.100000000001	75.7071865443425\\
523.200000000007	75.6927192814829\\
523.300000000008	75.6973633931983\\
523.399999999992	75.6829035339064\\
523.500000000004	75.6875477463713\\
523.600000000007	75.6730952835593\\
523.899999999999	75.6870229007634\\
524.000000000006	75.672581568403\\
524.500000000008	75.6957682043462\\
524.600000000007	75.6813417190776\\
525.000000000009	75.6998666920587\\
525.099999999995	75.6854531607007\\
526.100000000002	75.7316609654124\\
526.199999999999	75.7172715181456\\
526.299999999995	75.7218844984802\\
526.400000000001	75.707502374169\\
526.799999999992	75.7259442019358\\
526.900000000003	75.7115749525617\\
527.3	75.7299962078119\\
527.399999999999	75.7156398104265\\
527.699999999998	75.7294429708223\\
527.799999999993	75.7150975563554\\
527.899999999993	75.719696969697\\
528.000000000005	75.7053588335542\\
528.100000000006	75.7099583491102\\
528.300000000004	75.6813020439061\\
528.399999999993	75.685903500473\\
528.500000000004	75.6715853197124\\
528.899999999996	75.6899810964083\\
529	75.6756756756757\\
529.299999999992	75.6894597657726\\
529.400000000004	75.6751652502361\\
529.800000000001	75.6935270805812\\
529.999999999999	75.6649688737974\\
530.100000000003	75.6695586571105\\
530.199999999993	75.6552894587969\\
530.3	75.6598793363499\\
530.400000000007	75.6456173421301\\
530.500000000002	75.6502073124764\\
530.699999999993	75.621703089676\\
531.000000000005	75.6354735454717\\
531.199999999994	75.6070016939582\\
531.299999999997	75.6115920210764\\
531.4	75.5973659454374\\
531.499999999997	75.601956358164\\
531.600000000001	75.5877374459282\\
531.699999999996	75.5923279428357\\
531.800000000003	75.5781161872532\\
532.299999999998	75.6010518407213\\
532.399999999997	75.5868544600939\\
532.599999999993	75.5960202740755\\
532.699999999998	75.5818318318318\\
533.099999999996	75.6001500375094\\
533.300000000007	75.5718035245594\\
533.400000000003	75.5763823805061\\
533.500000000001	75.5622188905547\\
533.600000000006	75.5667978264943\\
533.699999999995	75.5526414387411\\
534.100000000007	75.5709472107825\\
534.199999999996	75.5568032940296\\
534.399999999994	75.5659494855005\\
534.499999999995	75.5518144407033\\
534.600000000006	75.5563867589302\\
534.700000000005	75.5422587883321\\
534.899999999996	75.5514018691589\\
535.000000000002	75.5372827508877\\
535.300000000002	75.5509899140829\\
535.400000000001	75.5368814192344\\
535.899999999999	75.5597014925373\\
536.100000000006	75.5315180902648\\
536.200000000004	75.5360805519299\\
536.300000000007	75.5219985085757\\
536.899999999992	75.5493482309125\\
537.199999999999	75.5071654569142\\
537.899999999997	75.5390334572491\\
537.999999999995	75.5249953540234\\
538.300000000007	75.538632986627\\
538.399999999999	75.5246053853296\\
538.499999999997	75.5291496472336\\
538.600000000007	75.5151290142937\\
539.699999999998	75.5650240829937\\
539.799999999995	75.5510279681423\\
540.599999999993	75.5872017754762\\
540.799999999992	75.5592530966907\\
540.999999999999	75.5682868231381\\
541.300000000009	75.5264130033247\\
541.500000000009	75.5354505169867\\
541.799999999995	75.493633511718\\
542.099999999994	75.5071929177425\\
542.199999999996	75.4932694080767\\
542.6	75.5113322277501\\
542.700000000006	75.4974207811349\\
542.800000000005	75.5019340578375\\
542.899999999994	75.48802946593\\
543.599999999995	75.5195880080927\\
543.7	75.5057006252299\\
543.9	75.5147058823529\\
543.999999999994	75.5008270538504\\
544.099999999993	75.50532892319\\
544.200000000006	75.4914569171413\\
544.300000000008	75.495958853784\\
544.499999999992	75.4682335659199\\
544.599999999999	75.4727372865798\\
544.700000000007	75.4588839941263\\
544.899999999993	75.4678899082569\\
545.199999999998	75.4263708050614\\
545.699999999996	75.4488823744962\\
545.800000000001	75.4350613665507\\
546.000000000005	75.4440578648599\\
546.099999999997	75.4302453313804\\
546.199999999994	75.4347428153029\\
546.299999999995	75.4209370424597\\
548.399999999993	75.5150410209663\\
548.499999999992	75.5012759752096\\
548.799999999992	75.5146656950264\\
548.900000000005	75.5009107468124\\
549.000000000005	75.5053724276088\\
549.1	75.4916241806264\\
549.399999999998	75.5050045495905\\
549.499999999997	75.4912663755458\\
549.600000000002	75.4957249408768\\
549.700000000008	75.4819934521644\\
549.900000000006	75.4909090909091\\
550.000000000005	75.477185966188\\
550.300000000002	75.4905523255814\\
550.400000000004	75.4768392370572\\
550.600000000002	75.4857454149265\\
550.799999999994	75.4583408967145\\
551.099999999998	75.4716981132076\\
551.199999999995	75.4580083439144\\
552.100000000001	75.4980079681275\\
552.200000000003	75.4843382219808\\
552.800000000003	75.5109423042141\\
552.999999999996	75.4836376785391\\
553.499999999994	75.5057803468208\\
553.6	75.4921437601589\\
553.699999999994	75.496569158541\\
553.900000000009	75.4693140794224\\
554.700000000006	75.5046863734679\\
554.800000000003	75.4910794737791\\
554.900000000006	75.4954954954955\\
555	75.4818951540263\\
555.099999999991	75.4863112391931\\
555.200000000009	75.472717450027\\
556.099999999998	75.512405609493\\
556.199999999993	75.498831565702\\
556.699999999994	75.5208333333333\\
556.800000000007	75.5072724007901\\
557.000000000004	75.5160653383594\\
557.099999999992	75.5025125628141\\
557.300000000002	75.5113024757804\\
557.399999999994	75.4977578475336\\
557.699999999997	75.5109358192901\\
557.799999999997	75.4974009679154\\
557.900000000001	75.5017921146953\\
558.000000000008	75.4882637520158\\
558.499999999995	75.5102040816327\\
558.599999999996	75.4966887417219\\
559.100000000009	75.5185979971388\\
559.199999999999	75.5050956552834\\
559.299999999994	75.5094744368967\\
559.400000000004	75.4959785522788\\
560.5	75.5440599357831\\
560.599999999999	75.5305867665418\\
560.699999999999	75.5349500713267\\
560.800000000003	75.5214833303619\\
560.999999999999	75.5302085189806\\
561.099999999995	75.5167498218104\\
561.299999999994	75.5254720342002\\
561.400000000005	75.5120213713268\\
561.599999999997	75.5207406088659\\
561.700000000006	75.5072979708081\\
562.000000000007	75.520370040918\\
562.200000000001	75.49350880313\\
563.000000000007	75.5283253418576\\
563.200000000006	75.5015089650275\\
563.899999999997	75.531914893617\\
563.999999999992	75.5185250842049\\
564.199999999994	75.5272018429913\\
564.300000000001	75.5138199858257\\
564.400000000005	75.5181576616475\\
564.500000000005	75.5047821466525\\
565.300000000005	75.5394411036434\\
565.399999999993	75.52608311229\\
565.500000000002	75.5304101838755\\
565.599999999994	75.5170585115786\\
565.799999999994	75.5257112564057\\
565.899999999996	75.5123674911661\\
566.099999999994	75.5210173083716\\
566.200000000003	75.5076814409324\\
566.300000000005	75.5120056497175\\
566.400000000006	75.4986760812004\\
566.999999999992	75.5245988361841\\
567.100000000001	75.5112834978843\\
568.400000000007	75.5672823218997\\
568.500000000002	75.5539922616954\\
568.599999999992	75.5582908387551\\
568.700000000002	75.5450070323488\\
569.300000000006	75.5707762557078\\
569.400000000004	75.5575065847234\\
569.699999999999	75.5703755703756\\
569.799999999996	75.557115283383\\
569.999999999995	75.5656902297842\\
570.100000000008	75.5524377411435\\
570.5	75.5695758850333\\
570.6	75.556334326266\\
570.700000000005	75.5606166783462\\
570.799999999999	75.5473813277281\\
570.999999999993	75.5559446681842\\
571.099999999994	75.5427170868347\\
571.900000000008	75.5769230769231\\
572.000000000001	75.5637126376508\\
573.099999999999	75.6106071179344\\
573.199999999994	75.5974184545613\\
573.600000000009	75.6144326302946\\
573.700000000002	75.6012547926107\\
573.800000000008	75.6055061857466\\
573.899999999999	75.5923344947735\\
574.299999999999	75.6093314763231\\
574.399999999996	75.5961705831158\\
575	75.621631020692\\
575.099999999995	75.6084840055633\\
575.500000000003	75.6254343293954\\
575.599999999992	75.6122980719124\\
576.000000000001	75.6292310362784\\
576.10000000001	75.616105518917\\
576.500000000003	75.6330211585154\\
576.599999999997	75.619906363794\\
577.199999999996	75.6452451065304\\
577.299999999998	75.6321440942154\\
577.800000000003	75.6532272019381\\
578.000000000005	75.6270541428819\\
578.200000000006	75.6354833131593\\
578.300000000007	75.6224066390041\\
578.400000000006	75.6266205704408\\
578.500000000003	75.6135499481507\\
578.70000000001	75.6219765031099\\
578.799999999999	75.6089134565555\\
579.299999999993	75.6299620296859\\
579.399999999993	75.6169111302847\\
579.599999999991	75.6253234431603\\
579.900000000006	75.5862068965517\\
580.500000000004	75.6114364450568\\
580.599999999998	75.5984157051834\\
580.699999999995	75.6026170798898\\
580.799999999992	75.5896023411947\\
580.900000000006	75.5938037865749\\
581.000000000002	75.5807950438823\\
581.100000000004	75.5849965588438\\
581.200000000001	75.5719938069844\\
581.599999999998	75.588791473268\\
581.799999999996	75.5628114796357\\
582.499999999999	75.5921730175077\\
582.699999999997	75.5662319835278\\
583.100000000003	75.5829903978052\\
583.299999999996	75.5570791909496\\
583.70000000001	75.5738266529633\\
583.800000000007	75.5608837129645\\
584.199999999994	75.5776142392607\\
584.300000000002	75.564681724846\\
584.400000000007	75.5688622754491\\
584.500000000003	75.555935682518\\
586.099999999993	75.6226543841692\\
586.200000000007	75.609756097561\\
586.799999999991	75.634690747998\\
587.100000000006	75.5960490463215\\
587.300000000008	75.6043581886279\\
587.399999999994	75.5914893617021\\
587.800000000003	75.6080966150706\\
588.000000000001	75.5823839483081\\
588.299999999999	75.594833446635\\
588.399999999998	75.5819881053526\\
588.500000000008	75.5861365953109\\
588.600000000008	75.5732970952947\\
588.800000000001	75.5815928001358\\
588.900000000009	75.5687606112054\\
589.099999999997	75.5770536320434\\
589.199999999991	75.5642287459698\\
589.400000000009	75.5725190839695\\
589.500000000007	75.5597014925373\\
590.300000000007	75.5928184281843\\
590.399999999993	75.580016934801\\
591.899999999999	75.6418918918919\\
592	75.6291167032596\\
592.399999999992	75.6455696202532\\
592.500000000005	75.6328045899426\\
592.899999999991	75.6492411467116\\
592.999999999995	75.636486258641\\
593.400000000004	75.6529064869419\\
593.600000000009	75.6274212565269\\
593.699999999994	75.6315257662513\\
593.899999999991	75.6060606060606\\
594.400000000001	75.6265769554247\\
594.499999999999	75.6138580558359\\
595.699999999991	75.6629741524001\\
595.800000000003	75.650276892096\\
595.900000000009	75.6543624161074\\
596.000000000009	75.6416708605939\\
596.100000000008	75.6457564575646\\
596.199999999993	75.6330706020459\\
596.500000000001	75.6453234998324\\
596.600000000002	75.6326462208815\\
596.700000000008	75.6367292225201\\
596.800000000007	75.624057631094\\
597.500000000009	75.6526104417671\\
597.599999999993	75.6399531537561\\
597.999999999996	75.6562447751212\\
598.300000000004	75.6183155080214\\
598.899999999992	75.6427378964942\\
599.099999999996	75.6174899866489\\
599.299999999995	75.6256256256256\\
599.400000000001	75.6130108423686\\
600.099999999994	75.6414528490503\\
600.200000000008	75.6288522405464\\
600.300000000007	75.6329113924051\\
600.400000000001	75.6203164029975\\
600.799999999997	75.6365451822267\\
600.899999999991	75.6239600665557\\
601.299999999996	75.6401729298304\\
601.499999999995	75.6150265957447\\
601.799999999999	75.6271805947832\\
601.900000000007	75.6146179401993\\
602.099999999994	75.6227167054135\\
602.200000000008	75.610161049311\\
602.299999999994	75.6142098273572\\
602.399999999998	75.6016597510373\\
603.999999999994	75.6662804171495\\
604.100000000004	75.6537570340947\\
604.500000000003	75.6698643731393\\
604.600000000006	75.6573507524392\\
604.899999999997	75.6694214876033\\
604.999999999991	75.6569162121963\\
605.400000000002	75.6729975227085\\
605.500000000009	75.6605019815059\\
605.599999999997	75.6645203896318\\
605.699999999999	75.6520303730604\\
606.299999999994	75.6761213720317\\
606.400000000003	75.6636438582028\\
606.599999999992	75.6716663919565\\
606.699999999994	75.659195781147\\
606.999999999994	75.6712238510954\\
607.299999999995	75.6338491932828\\
607.400000000005	75.6378600823045\\
607.500000000001	75.6254114549046\\
607.900000000004	75.6414473684211\\
607.999999999995	75.6290083867785\\
608.300000000001	75.6410256410256\\
608.399999999991	75.6285949055053\\
608.50000000001	75.6325994084785\\
608.699999999991	75.607752956636\\
608.800000000008	75.6117589095089\\
608.900000000002	75.5993431855501\\
609.299999999992	75.615359369872\\
609.400000000005	75.6029532403609\\
609.699999999991	75.6149557231879\\
609.899999999998	75.5901639344262\\
610.500000000006	75.6141500163773\\
610.700000000002	75.5893909626719\\
612.400000000009	75.6571428571429\\
612.5	75.6447926869083\\
612.599999999993	75.6487677493063\\
612.799999999994	75.6240822320117\\
612.899999999993	75.6280587275693\\
612.999999999995	75.6157233730223\\
613.199999999999	75.6236751997391\\
613.299999999993	75.6113465927616\\
613.5	75.619295958279\\
613.600000000001	75.6069740915757\\
613.900000000009	75.6188925081433\\
614.000000000001	75.6065787331054\\
614.299999999995	75.6184895833333\\
614.400000000001	75.6061838893409\\
614.699999999998	75.6180871828237\\
614.900000000004	75.5934959349593\\
615.399999999999	75.6133225020309\\
615.500000000003	75.6010396361274\\
615.999999999996	75.6208407726019\\
616.100000000008	75.6085686465433\\
616.200000000001	75.612526367029\\
616.300000000008	75.6002595717067\\
616.899999999998	75.6239870340357\\
617.199999999991	75.587234731897\\
617.599999999997	75.6030435486482\\
617.799999999995	75.5785725845606\\
617.999999999991	75.5864746804724\\
618.099999999991	75.5742478162407\\
618.300000000002	75.5821474773609\\
618.400000000007	75.5699272433306\\
618.499999999995	75.573876495312\\
618.699999999995	75.5494505494505\\
618.90000000001	75.5573505654281\\
618.999999999991	75.5451461799386\\
619.100000000007	75.5490956072351\\
619.299999999993	75.5247013238618\\
619.400000000005	75.5286521388216\\
619.499999999996	75.5164622336992\\
619.600000000002	75.5204131031144\\
619.700000000003	75.5082284607938\\
619.800000000001	75.5121793837716\\
619.9	75.5\\
620.500000000001	75.5236867547535\\
620.60000000001	75.5115192524569\\
621.300000000003	75.5391052462182\\
621.400000000005	75.526950925181\\
622.000000000006	75.5505545732197\\
622.099999999997	75.5384120861459\\
622.299999999993	75.5462724935733\\
622.400000000009	75.5341365461847\\
623.799999999995	75.5890367046001\\
623.999999999996	75.5648133311969\\
624.099999999991	75.5687279718039\\
624.300000000002	75.5445227418322\\
624.899999999991	75.568\\
625	75.5559110542313\\
625.099999999996	75.5598208573257\\
625.199999999994	75.5477370861986\\
625.400000000005	75.5555555555556\\
625.499999999998	75.5434782608696\\
625.799999999999	75.5552005112638\\
625.900000000009	75.5431309904153\\
626.300000000011	75.558748403576\\
626.400000000008	75.5466879489226\\
626.699999999995	75.5583918315252\\
626.800000000003	75.5463391290477\\
627.300000000006	75.5658272234619\\
627.400000000009	75.5537848605578\\
627.700000000005	75.565466709143\\
627.799999999992	75.5534320751712\\
627.900000000006	75.5573248407643\\
628.000000000007	75.5452953351377\\
628.400000000001	75.5608591885441\\
628.499999999992	75.5488386891505\\
628.600000000005	75.5527278511214\\
628.699999999997	75.5407124681934\\
628.800000000007	75.5446016854826\\
628.999999999992	75.520584962645\\
629.200000000007	75.5283648498332\\
629.300000000005	75.5163647918653\\
629.499999999995	75.5241423125794\\
629.700000000007	75.5001587805653\\
630.099999999993	75.5157092986354\\
630.300000000009	75.4917512690355\\
630.900000000004	75.5150554675119\\
631.099999999994	75.4911280101394\\
631.400000000008	75.5027711797308\\
631.499999999999	75.4908169727676\\
632.199999999994	75.5179503400285\\
632.400000000001	75.4940711462451\\
633.4	75.5327545382794\\
633.600000000007	75.5089158908001\\
633.900000000002	75.5205047318612\\
634	75.5085948588551\\
634.099999999999	75.5124566382844\\
634.200000000005	75.5005517893741\\
635.400000000002	75.5468135326515\\
635.5	75.5349276274386\\
635.999999999999	75.5541581512341\\
636.099999999994	75.5422823011632\\
636.80000000001	75.56916313393\\
636.999999999995	75.5454402762518\\
637.1	75.549278091651\\
637.200000000007	75.5374235054135\\
637.400000000006	75.5450980392157\\
637.499999999992	75.5332496863237\\
638.7	75.5792110206637\\
638.799999999993	75.5673814368446\\
638.900000000008	75.5712050078247\\
638.999999999994	75.5593803786575\\
639.200000000004	75.5670264351635\\
639.399999999998	75.5433932759969\\
639.699999999992	75.5548608940294\\
639.800000000002	75.5430536021253\\
640.300000000006	75.5621486570893\\
640.399999999996	75.5503512880562\\
641.099999999998	75.5770430442919\\
641.199999999992	75.5652580695462\\
641.30000000001	75.569067664484\\
641.499999999994	75.5455112219451\\
642.500000000007	75.5835667600373\\
642.600000000004	75.5718064415746\\
643.299999999993	75.598383587193\\
643.400000000002	75.5866355866356\\
643.500000000001	75.5904288377874\\
643.600000000002	75.578685723163\\
644.200000000001	75.6014279062548\\
644.299999999995	75.5896958410925\\
644.4	75.5934833204034\\
644.500000000005	75.5817561278312\\
644.69999999999	75.5893300248139\\
645.000000000004	75.5541776468765\\
645.099999999995	75.5579665220087\\
645.30000000001	75.5345522156802\\
645.899999999995	75.5572755417957\\
646.400000000004	75.4988399071926\\
647.799999999998	75.5517826825127\\
647.900000000009	75.5401234567901\\
648.200000000002	75.5514422335339\\
648.399999999995	75.5281418658443\\
648.699999999995	75.539457459926\\
648.900000000002	75.5161787365177\\
648.999999999991	75.5199507009706\\
649.199999999999	75.4966887417219\\
649.49999999999	75.5080049261084\\
649.699999999997	75.4847645429363\\
649.900000000001	75.4923076923077\\
649.999999999995	75.4806952776496\\
650.400000000008	75.4957724827056\\
650.499999999993	75.4841684598832\\
650.799999999995	75.4954678137963\\
650.89999999999	75.4838709677419\\
651.100000000007	75.4914004914005\\
651.300000000009	75.4682222904513\\
651.500000000001	75.475751995089\\
651.700000000006	75.4525928198834\\
652.40000000001	75.4789272030651\\
652.500000000008	75.467361323935\\
652.599999999993	75.4711199632297\\
652.700000000008	75.4595588235294\\
653	75.4708314193845\\
653.099999999996	75.4592774035518\\
653.500000000002	75.4742962056304\\
653.700000000001	75.4512083205873\\
654.299999999998	75.4737163814181\\
654.4	75.4621848739496\\
654.500000000009	75.4659333944394\\
654.699999999999	75.4428833231521\\
654.900000000006	75.4503816793893\\
655.000000000007	75.4388642955274\\
655.099999999996	75.4426129426129\\
655.200000000007	75.4311002594232\\
655.499999999995	75.4423428920073\\
655.599999999994	75.4308372731432\\
655.699999999998	75.4345837145471\\
655.800000000009	75.4230827870102\\
656.600000000007	75.4530226892036\\
656.699999999995	75.4415347137637\\
657.09999999999	75.4564820450396\\
657.200000000008	75.445002282063\\
657.599999999994	75.4599361410978\\
657.800000000001	75.436996504028\\
658.299999999999	75.4556500607533\\
658.4	75.4441913439636\\
658.699999999992	75.4553734061931\\
658.800000000007	75.4439216876613\\
659.199999999991	75.4588199605642\\
659.399999999997	75.4359363153904\\
660.400000000009	75.4731264193793\\
660.499999999991	75.4617014834998\\
660.600000000008	75.4654154684426\\
660.699999999995	75.453995157385\\
661.100000000008	75.4688445251059\\
661.200000000002	75.4574323302586\\
661.399999999995	75.4648526077098\\
661.500000000001	75.453446191052\\
661.599999999991	75.4571558107904\\
661.800000000006	75.4343556428464\\
662.100000000011	75.4454847478103\\
662.199999999999	75.4340933111883\\
662.299999999999	75.4378019323672\\
662.400000000001	75.4264150943396\\
663.099999999999	75.4523522316044\\
663.19999999999	75.4409769335142\\
665.599999999996	75.5295178008112\\
665.700000000011	75.5181736257134\\
666.100000000004	75.5328730111078\\
666.199999999997	75.5215368452649\\
666.399999999993	75.5288822205551\\
666.49999999999	75.5175517551755\\
666.60000000001	75.5212239388031\\
666.799999999993	75.4985754985755\\
668.599999999995	75.5645281890235\\
668.700000000001	75.5532296650718\\
668.800000000001	75.5568844371356\\
668.900000000007	75.5455904334828\\
669.099999999999	75.5528989838613\\
669.400000000011	75.5190440627334\\
669.800000000008	75.5336617405583\\
669.900000000003	75.5223880597015\\
670.000000000006	75.5260408894195\\
670.09999999999	75.5147717099373\\
670.299999999994	75.522076372315\\
670.50000000001	75.4995526394274\\
670.99999999999	75.5178065862018\\
671.199999999998	75.4953076120959\\
671.700000000004	75.5135456981244\\
671.800000000007	75.5023068909064\\
671.900000000007	75.5059523809524\\
672.099999999991	75.4834870574234\\
672.600000000001	75.5017095287647\\
672.799999999993	75.4792688363799\\
672.90000000001	75.482912332838\\
672.999999999999	75.4716981132076\\
673.499999999995	75.4899049881235\\
673.599999999999	75.4786997179754\\
673.800000000009	75.4859771479448\\
673.899999999998	75.4747774480712\\
674.099999999993	75.4820528033224\\
674.199999999992	75.4708586682486\\
674.400000000009	75.4781319495923\\
674.50000000001	75.4669433738512\\
674.599999999999	75.4705795168223\\
674.699999999994	75.4593953764078\\
674.799999999996	75.4630315602311\\
675.000000000001	75.4406754554881\\
675.100000000011	75.4443127962085\\
675.199999999998	75.4331408262994\\
675.299999999993	75.4367782055078\\
675.500000000007	75.4144464179988\\
676.400000000004	75.4471544715447\\
676.499999999998	75.4360035471475\\
676.799999999991	75.4468902348944\\
676.900000000007	75.4357459379616\\
676.99999999999	75.4393738000295\\
677.30000000001	75.4059639799232\\
677.699999999998	75.4204780171142\\
677.900000000011	75.3982300884956\\
678.199999999997	75.4091110128262\\
678.299999999998	75.3979952830189\\
678.399999999999	75.4016212232867\\
678.699999999997	75.3682969946965\\
678.899999999997	75.3755522827688\\
679.40000000001	75.3200883002208\\
680.699999999998	75.3672150411281\\
680.800000000001	75.3561462769864\\
680.899999999996	75.359765051395\\
680.999999999992	75.3487006313317\\
681.099999999997	75.3523194362889\\
681.299999999997	75.3302025242149\\
681.499999999997	75.337441314554\\
681.599999999994	75.326389907584\\
681.999999999993	75.3408591115672\\
682.200000000003	75.3187747325224\\
682.600000000003	75.3332356818515\\
682.699999999996	75.3222026947862\\
683.100000000007	75.3366510538642\\
683.199999999995	75.3256256402751\\
683.50000000001	75.3364540667057\\
683.700000000003	75.3144194208833\\
683.899999999997	75.3216374269006\\
683.999999999995	75.310627101301\\
684.19999999999	75.3178430512933\\
684.299999999991	75.3068381063705\\
684.600000000001	75.3176573681905\\
684.700000000006	75.3066588785047\\
684.800000000003	75.3102642721565\\
684.999999999993	75.2882790833455\\
686.400000000001	75.3386744355426\\
686.699999999998	75.3057658707047\\
686.799999999996	75.3093608967826\\
686.900000000002	75.2983988355167\\
687.000000000002	75.3019938873526\\
687.100000000007	75.291036088475\\
687.299999999992	75.2982251963922\\
687.600000000004	75.2653773447724\\
688.300000000009	75.2905287623475\\
688.399999999999	75.279593318809\\
689.399999999994	75.3154459753445\\
689.500000000005	75.3045243619489\\
689.700000000007	75.3116845462453\\
689.800000000003	75.3007682272793\\
690.699999999992	75.3329473074696\\
690.799999999991	75.3220437111015\\
690.89999999999	75.3256150506512\\
691.200000000002	75.2929263706061\\
691.300000000002	75.2964998553659\\
691.5	75.2747252747253\\
691.800000000004	75.2854458736812\\
691.899999999998	75.2745664739884\\
693.8	75.3422683383773\\
693.900000000009	75.3314121037464\\
694.399999999998	75.3491720662347\\
694.49999999999	75.3383242153757\\
694.599999999994	75.341874190298\\
694.699999999998	75.3310305123777\\
694.799999999997	75.334580515182\\
694.900000000005	75.3237410071942\\
695.19999999999	75.3343880339422\\
695.299999999992	75.3235547886109\\
696.700000000011	75.3731343283582\\
696.800000000001	75.3623188405797\\
696.900000000004	75.3658536585366\\
697.000000000006	75.3550423181753\\
697.300000000002	75.3656438199025\\
697.50000000001	75.3440366972477\\
698.599999999989	75.3828538714756\\
698.800000000002	75.3612820145944\\
698.899999999992	75.3648068669528\\
698.999999999999	75.3540266056358\\
699.099999999999	75.3575514874142\\
699.199999999995	75.3467753467753\\
699.900000000008	75.3714285714286\\
700.000000000012	75.3606627624625\\
701.2	75.4028233281049\\
701.400000000003	75.3813257305773\\
701.599999999998	75.3883425965512\\
701.799999999998	75.3668613762644\\
702.599999999999	75.3949053650206\\
702.700000000001	75.3841775754126\\
702.800000000009	75.3876796130317\\
702.9	75.3769559032717\\
704.000000000007	75.4154239454623\\
704.099999999994	75.4047145697245\\
704.400000000005	75.4151880766501\\
704.499999999996	75.4044848140789\\
705.600000000001	75.442822729205\\
705.700000000006	75.4321337489374\\
706.000000000001	75.4425718736723\\
706.100000000001	75.4318889832908\\
706.400000000001	75.4423213021939\\
706.500000000008	75.4316444947637\\
706.999999999992	75.4490171121482\\
707.100000000002	75.4383484162896\\
707.599999999991	75.4557015684612\\
707.700000000002	75.445040972026\\
707.80000000001	75.448509676508\\
708.000000000007	75.4271995480864\\
708.800000000002	75.4549301735082\\
708.900000000012	75.444287729196\\
709.800000000001	75.4754190731089\\
709.900000000011	75.4647887323944\\
710.200000000005	75.4751513445023\\
710.300000000004	75.464527027027\\
710.599999999992	75.4748839172647\\
710.799999999995	75.4536503024335\\
711.299999999997	75.4709024458814\\
711.50000000001	75.4496908375492\\
711.800000000004	75.4600365219834\\
711.900000000004	75.4494382022472\\
712.099999999989	75.4563324908734\\
712.200000000005	75.4457391548505\\
712.300000000006	75.4491858506457\\
712.400000000011	75.4385964912281\\
712.69999999999	75.4489337822671\\
713.100000000009	75.4066180594504\\
713.199999999995	75.4100658909295\\
713.300000000003	75.3994953742641\\
713.499999999999	75.4063901345291\\
713.599999999993	75.3958245761524\\
713.899999999996	75.406162464986\\
714.000000000004	75.3956028567428\\
714.099999999996	75.3990478857463\\
714.199999999998	75.3884922301554\\
714.400000000007	75.3953813855843\\
714.500000000003	75.3848306745032\\
714.599999999991	75.3882748006156\\
714.700000000003	75.3777280358142\\
714.800000000007	75.381172191915\\
714.899999999993	75.3706293706294\\
715.400000000011	75.3878406708595\\
715.499999999999	75.3773057574064\\
716.199999999999	75.40136814184\\
716.299999999996	75.3908431044109\\
716.400000000012	75.3942777390091\\
716.600000000004	75.3732384540254\\
716.799999999989	75.3801088017855\\
716.90000000001	75.3695955369596\\
717.099999999997	75.3764640267708\\
717.199999999995	75.3659556670849\\
717.399999999989	75.3728222996516\\
717.599999999995	75.3518183084854\\
717.999999999995	75.3655479738198\\
718.099999999995	75.3550543024227\\
718.699999999998	75.3756260434057\\
718.800000000008	75.365141187926\\
720.099999999997	75.4096084420994\\
720.199999999994	75.3991392475358\\
720.599999999997	75.4127931178021\\
720.79999999999	75.3918712720211\\
721.000000000001	75.3986964360006\\
721.100000000007	75.3882418191902\\
721.599999999998	75.4052930580574\\
721.700000000011	75.3948462177889\\
722.200000000008	75.4118787207532\\
722.300000000008	75.4014396456257\\
723.100000000012	75.4286504424779\\
723.499999999998	75.3869541182974\\
723.600000000008	75.3903551195247\\
723.700000000005	75.3799392097264\\
724.000000000008	75.3901394834968\\
724.099999999999	75.3797293565313\\
724.499999999989	75.3933204526635\\
724.699999999996	75.3725165562914\\
725.200000000001	75.3894940024817\\
725.299999999995	75.3791011855528\\
725.600000000007	75.389279316522\\
725.699999999998	75.3788922568201\\
726.399999999991	75.4026152787336\\
726.800000000007	75.3611225753199\\
726.999999999997	75.3678998762206\\
727.200000000005	75.347174480957\\
727.599999999995	75.3607255737254\\
727.700000000011	75.3503709810388\\
728.499999999998	75.3774361789734\\
728.600000000009	75.3670920817895\\
728.899999999993	75.3772290809328\\
729.000000000002	75.3668906871485\\
729.400000000004	75.3803975325565\\
729.599999999995	75.3597368781691\\
729.89999999999	75.3698630136986\\
730.299999999992	75.328587075575\\
731.599999999993	75.3724203908706\\
731.799999999989	75.3518240196748\\
732.000000000007	75.3585575741019\\
732.100000000002	75.3482655012292\\
732.499999999998	75.3617253617254\\
732.600000000007	75.3514398798963\\
732.899999999994	75.3615279672578\\
733.000000000003	75.3512481244032\\
733.499999999989	75.3680479825518\\
733.599999999996	75.3577756576257\\
734.199999999989	75.3779109355849\\
734.299999999999	75.3676470588235\\
734.50000000001	75.3743533895998\\
734.599999999995	75.364094188104\\
734.900000000009	75.3741496598639\\
735	75.3638960685621\\
735.299999999998	75.3739461517541\\
735.40000000001	75.3636981645139\\
735.499999999998	75.3670473083197\\
735.599999999994	75.3568030447193\\
735.800000000007	75.3635004756081\\
735.899999999999	75.3532608695652\\
736.100000000004	75.3599565335507\\
736.200000000001	75.3497215808774\\
736.699999999995	75.3664495114006\\
736.800000000001	75.3562220111277\\
736.999999999989	75.362908696242\\
737.100000000008	75.3526858383071\\
737.199999999989	75.3560287535603\\
737.500000000007	75.3253796095445\\
738.600000000004	75.3621226478949\\
738.900000000011	75.3315290933694\\
739.299999999988	75.3448742223424\\
739.4	75.3346855983773\\
739.600000000008	75.3413546032175\\
739.699999999998	75.331170586645\\
739.799999999993	75.3345046627923\\
739.899999999993	75.3243243243243\\
740	75.3276584245372\\
740.099999999993	75.317481761686\\
740.700000000001	75.3374730021598\\
740.899999999997	75.3171390013495\\
742.699999999991	75.3769520732364\\
742.799999999989	75.3668057612061\\
743.199999999992	75.3800618861832\\
743.299999999994	75.3699219800915\\
743.700000000007	75.38316751815\\
743.799999999998	75.3730340099476\\
743.900000000003	75.3763440860215\\
744.199999999988	75.3459626494693\\
744.400000000009	75.3525856279382\\
744.600000000002	75.3323485967504\\
744.699999999996	75.3356605800215\\
744.899999999998	75.3154362416107\\
745.200000000002	75.3253723332886\\
745.299999999992	75.315266970754\\
745.399999999991	75.3185781354795\\
745.500000000005	75.3084763948498\\
745.700000000008	75.3150978814696\\
745.799999999998	75.3050006703311\\
746.000000000008	75.3116204262163\\
746.09999999999	75.3015277405521\\
747.400000000004	75.3444816053512\\
747.499999999994	75.3344034242911\\
747.999999999998	75.3508889185938\\
748.10000000001	75.3408179631115\\
748.600000000011	75.3572859623347\\
748.799999999995	75.3371611697156\\
749.099999999997	75.3470368392952\\
749.199999999995	75.3369811824369\\
749.600000000002	75.3501400560224\\
749.699999999992	75.3400906908509\\
750	75.3499533395547\\
750.099999999993	75.3399093575047\\
750.199999999999	75.3431960549114\\
750.299999999997	75.3331556503198\\
750.5	75.3397282174261\\
750.699999999998	75.3196590303676\\
751.099999999996	75.3328008519702\\
751.300000000003	75.3127495342028\\
752.9	75.3652058432935\\
752.999999999998	75.3551985128137\\
753.200000000006	75.3617416699854\\
753.499999999999	75.3317409766454\\
754.000000000003	75.3480970693542\\
754.100000000002	75.3381066030231\\
754.199999999991	75.3413761103009\\
754.300000000008	75.331389183457\\
755.600000000006	75.3738255921662\\
755.900000000002	75.3439153439153\\
755.999999999989	75.3471762994313\\
756.099999999994	75.3372123776779\\
756.299999999991	75.3437334743522\\
756.399999999992	75.3337739590218\\
757.200000000002	75.3598309784762\\
757.300000000012	75.349881172432\\
757.400000000011	75.3531353135314\\
757.500000000004	75.3431890179514\\
758.000000000003	75.3594512597283\\
758.1	75.3495120021103\\
758.599999999991	75.365757216291\\
758.70000000001	75.3558249868213\\
760.299999999997	75.4076801683325\\
760.600000000002	75.377941369791\\
761.100000000012	75.3941145559643\\
761.200000000006	75.384211217654\\
761.500000000002	75.3939075630252\\
761.600000000004	75.3840094525404\\
762.199999999989	75.4033844942936\\
762.300000000002	75.3934942287513\\
762.600000000001	75.4031729382457\\
762.700000000004	75.3932878867331\\
763.30000000001	75.4126277181032\\
763.49999999999	75.392875851231\\
763.599999999991	75.3960979442189\\
763.800000000001	75.3763581620631\\
764.500000000003	75.3989013863458\\
764.600000000001	75.389041454165\\
766.300000000001	75.4436325678497\\
766.399999999998	75.4337899543379\\
766.499999999999	75.4369945212627\\
766.60000000001	75.4271553410721\\
767.100000000003	75.4431699687174\\
767.299999999998	75.4235079489184\\
767.599999999998	75.4331118926664\\
767.700000000011	75.4232873144048\\
767.900000000001	75.4296875\\
768.000000000001	75.419867204791\\
768.199999999991	75.4262657815957\\
768.300000000012	75.416449765747\\
768.999999999997	75.438824600182\\
769.100000000009	75.4290171606864\\
769.200000000007	75.4322111010009\\
769.300000000009	75.4224070704445\\
769.400000000006	75.4256010396361\\
769.499999999995	75.4158004158004\\
770.100000000009	75.4349519605297\\
770.200000000012	75.4251590289498\\
770.600000000001	75.4379135850526\\
770.800000000004	75.4183421974316\\
770.900000000012	75.4215304798962\\
771.000000000009	75.4117494488393\\
771.39999999999	75.4244977316915\\
771.600000000013	75.4049501101464\\
772.099999999995	75.4208754208754\\
772.200000000002	75.4111096724071\\
772.89999999999	75.4333764553687\\
773.000000000008	75.4236191954469\\
773.100000000008	75.4267977237455\\
773.200000000008	75.4170438380965\\
774.699999999998	75.4646360351058\\
774.800000000005	75.4548974061169\\
775.099999999989	75.4643962848297\\
775.499999999995	75.4254770500258\\
775.600000000012	75.4286450947531\\
775.799999999997	75.4092022167805\\
776.100000000007	75.4187065189384\\
776.200000000009	75.408991369316\\
776.600000000008	75.4216557229304\\
776.699999999993	75.4119464469619\\
777.299999999999	75.4309235914587\\
777.399999999988	75.4212218649518\\
778.100000000008	75.4433307632999\\
778.199999999997	75.4336374148786\\
778.899999999995	75.4557124518614\\
779.200000000009	75.4266649557295\\
779.699999999991	75.442421133624\\
780.100000000002	75.4037426300949\\
780.300000000008	75.4100461301896\\
780.399999999996	75.4003843689942\\
780.699999999998	75.4098360655738\\
780.800000000007	75.4001792803176\\
781.600000000009	75.4253549955226\\
781.700000000004	75.4157073420312\\
781.800000000002	75.4188515155391\\
781.900000000005	75.4092071611253\\
782.70000000001	75.4343382728666\\
782.899999999994	75.4150702426564\\
783.299999999996	75.4276231810059\\
783.400000000001	75.4179961710274\\
783.6	75.4242694908766\\
783.799999999999	75.4050261512948\\
784.599999999999	75.4301006754173\\
784.699999999988	75.4204892966361\\
784.800000000011	75.4236208434195\\
784.900000000004	75.4140127388535\\
785.299999999996	75.4265342500637\\
785.400000000011	75.4169318905156\\
785.899999999992	75.4325699745547\\
786.000000000009	75.4229741763134\\
786.900000000011	75.4510800508259\\
787	75.4414940922373\\
787.499999999987	75.4570848146267\\
787.600000000006	75.4475053954551\\
787.700000000002	75.4506219852754\\
787.799999999998	75.4410458179972\\
787.899999999995	75.4441624365482\\
787.999999999993	75.4345895190966\\
789.900000000001	75.4936708860759\\
790.099999999994	75.4745634016705\\
790.399999999988	75.4838709677419\\
790.500000000001	75.4743232987604\\
791.200000000005	75.4960192088967\\
791.400000000008	75.4769425142135\\
792.099999999995	75.4986114617521\\
792.199999999989	75.4890824182759\\
792.499999999999	75.4983598284128\\
792.700000000003	75.4793138244198\\
793.600000000011	75.5071185586494\\
793.699999999988	75.4976064499874\\
793.999999999989	75.5068631154766\\
794.100000000007	75.4973558297658\\
794.199999999994	75.5004406395568\\
794.400000000004	75.4814348646948\\
794.900000000007	75.496855345912\\
794.999999999999	75.487360080493\\
795.500000000004	75.5027652086476\\
795.600000000011	75.4932763604373\\
795.8	75.499434602337\\
796.000000000013	75.4804672779802\\
796.100000000009	75.4835468475257\\
796.200000000002	75.4740675624764\\
796.500000000008	75.4833040421793\\
796.599999999992	75.4738295468809\\
796.699999999994	75.4769076305221\\
797.100000000001	75.4390366281987\\
797.300000000011	75.4451968898921\\
797.49999999999	75.4262788365095\\
797.600000000012	75.4293594082989\\
797.700000000004	75.4199047380296\\
798.199999999995	75.4353000125266\\
798.300000000001	75.4258517034068\\
798.599999999996	75.4350820082634\\
798.699999999988	75.4256384576865\\
798.799999999994	75.4287144824133\\
798.89999999999	75.4192740926158\\
799.099999999989	75.4254254254254\\
799.299999999995	75.4065549161871\\
799.400000000004	75.4096310193871\\
799.50000000001	75.40020010005\\
799.799999999989	75.4094261782723\\
799.899999999992	75.4\\
800.100000000005	75.4061484628843\\
800.299999999991	75.3873063468266\\
800.40000000001	75.3903810118676\\
800.499999999994	75.3809642767924\\
801.199999999987	75.4024709846499\\
801.299999999994	75.3930621412528\\
802.399999999998	75.4267912772586\\
802.500000000004	75.4173934712185\\
803.100000000001	75.4357569721116\\
803.199999999998	75.426366239263\\
803.300000000008	75.4294249439881\\
803.400000000009	75.4200373366521\\
803.500000000001	75.4230960676954\\
803.599999999991	75.4137115839244\\
803.700000000008	75.4167703408808\\
803.800000000007	75.4073889787287\\
803.899999999995	75.410447761194\\
804.000000000008	75.4010695187166\\
804.500000000003	75.4163559532687\\
804.600000000013	75.4069839691811\\
805.499999999999	75.4344587884806\\
805.599999999992	75.4250961896488\\
806.399999999988	75.4494730316181\\
806.500000000002	75.4401190181007\\
807.099999999991	75.4583746283449\\
807.200000000002	75.4490276229407\\
807.599999999999	75.4611860839421\\
807.700000000009	75.4518445159693\\
808.099999999998	75.463994060876\\
808.199999999994	75.4546579240381\\
808.299999999999	75.4576942107867\\
808.399999999992	75.4483611626469\\
808.900000000006	75.4635352286774\\
808.999999999994	75.4542083796811\\
809.300000000012	75.4633061527057\\
809.400000000012	75.4539839407041\\
809.899999999991	75.4691358024691\\
810.000000000001	75.4598197753364\\
810.300000000012	75.4689042448174\\
810.499999999994	75.4502837404392\\
810.89999999999	75.462392108508\\
810.999999999988	75.4530883984712\\
812.600000000003	75.5014150362988\\
812.699999999992	75.492125984252\\
812.799999999991	75.4951408537335\\
812.900000000004	75.4858548585486\\
813.699999999995	75.5099533054805\\
813.900000000001	75.4914004914005\\
814.39999999999	75.5064456721915\\
814.500000000009	75.4971765283575\\
814.700000000007	75.5031909671085\\
814.800000000008	75.4939256350472\\
814.999999999987	75.4999386578334\\
815.099999999994	75.4906771344455\\
815.300000000006	75.4966887417219\\
815.500000000012	75.4781755762629\\
816.199999999993	75.499203724121\\
816.300000000009	75.4899559039686\\
816.799999999991	75.5049577671686\\
816.999999999997	75.4864765634561\\
817.19999999999	75.4924752232962\\
817.300000000008	75.4832395400049\\
818.700000000002	75.5251587689301\\
818.800000000012	75.515936011723\\
818.89999999999	75.5189255189255\\
819.000000000008	75.5097057746307\\
819.200000000001	75.5156841205907\\
819.299999999992	75.5064681474249\\
819.599999999997	75.5154324752958\\
819.700000000012	75.5062210295194\\
819.899999999993	75.5121951219512\\
819.99999999999	75.502987440556\\
820.299999999995	75.5119453924915\\
820.400000000011	75.5027422303474\\
820.899999999995	75.5176613885506\\
821.000000000012	75.5084642552673\\
821.100000000004	75.5114466634194\\
821.199999999989	75.5022525264824\\
822.500000000001	75.540967663506\\
822.700000000007	75.5226057365095\\
822.800000000005	75.5255802649168\\
823	75.5072287692868\\
823.399999999999	75.5191256830601\\
823.500000000008	75.5099562894609\\
824.400000000003	75.5366889023651\\
824.500000000011	75.527528498666\\
825.000000000008	75.5423585019998\\
825.299999999996	75.5149018657621\\
825.400000000002	75.5178679588128\\
825.500000000011	75.5087209302326\\
825.900000000004	75.5205811138015\\
825.999999999994	75.5114392930638\\
826.199999999992	75.5173665738836\\
826.400000000001	75.4990925589837\\
826.999999999989	75.5168661588683\\
827.100000000005	75.5077369439072\\
827.500000000003	75.5195746737554\\
827.600000000007	75.5104506463694\\
829.599999999998	75.569482945643\\
829.699999999991	75.5603759942155\\
830.4	75.5809753160747\\
830.499999999999	75.5718757524681\\
830.599999999992	75.5748164198868\\
830.700000000013	75.565719788156\\
831.200000000005	75.580416215566\\
831.300000000005	75.5713254751022\\
831.700000000004	75.5830728540515\\
831.899999999995	75.5649038461538\\
831.999999999994	75.5678404037976\\
832.100000000008	75.5587599134823\\
832.899999999989	75.5822328931573\\
832.99999999999	75.5731604849358\\
833.300000000001	75.5819534437245\\
833.399999999989	75.5728854229154\\
833.599999999991	75.5787453520451\\
833.700000000002	75.569680978652\\
834.700000000004	75.5989458552947\\
834.800000000011	75.5898910049108\\
835.300000000005	75.6045008379219\\
835.400000000003	75.5954518252543\\
835.699999999998	75.6042115338598\\
835.900000000007	75.5861244019139\\
836.099999999991	75.591963645061\\
836.19999999999	75.5829247877556\\
836.400000000008	75.5887627017334\\
836.599999999999	75.5706943946456\\
836.800000000007	75.5765324411519\\
836.899999999997	75.5675029868578\\
837.000000000012	75.5704216939434\\
837.09999999999	75.5613951266125\\
837.300000000005	75.5672319082876\\
837.500000000007	75.549188156638\\
837.700000000004	75.5550250656481\\
837.899999999994	75.5369928400955\\
837.99999999999	75.5399117050471\\
838.09999999999	75.5308995466476\\
838.299999999994	75.5367366412214\\
838.40000000001	75.5277280858676\\
838.499999999998	75.5306463152874\\
838.599999999995	75.521640634315\\
838.9	75.5303933253874\\
839.100000000006	75.5123927550048\\
839.899999999989	75.5357142857143\\
839.999999999987	75.5267230091656\\
840.299999999993	75.5354593050928\\
840.600000000005	75.5085048174141\\
840.70000000001	75.511417697431\\
840.899999999999	75.4934601664685\\
841.100000000009	75.4992867332382\\
841.200000000005	75.4903126114347\\
841.300000000013	75.4932255764203\\
841.399999999989	75.4842543077837\\
841.499999999998	75.4871673003802\\
841.599999999996	75.4781988832125\\
841.699999999995	75.4811119030649\\
841.800000000008	75.4721463356693\\
842.100000000007	75.4808834006174\\
842.200000000011	75.4719221180102\\
842.699999999997	75.4864736592311\\
842.800000000005	75.4775180923004\\
842.899999999992	75.4804270462633\\
843.00000000001	75.4714743209584\\
843.100000000005	75.4743833017078\\
843.200000000002	75.4654334163406\\
843.599999999993	75.4770653075738\\
843.700000000009	75.4681204076795\\
844.10000000001	75.4797441364606\\
844.199999999995	75.4708042165107\\
844.400000000011	75.4766133806986\\
844.500000000011	75.4676770068671\\
844.600000000005	75.4705812714573\\
844.700000000008	75.4616477272727\\
844.800000000009	75.4645520179903\\
844.899999999996	75.4556213017751\\
845.699999999999	75.4788366043982\\
845.800000000008	75.4699137013831\\
846.299999999991	75.484404536862\\
846.400000000001	75.4754873006497\\
847.099999999997	75.4957507082153\\
847.200000000002	75.4868405523427\\
847.400000000012	75.4926253687316\\
847.600000000008	75.4748142031379\\
847.999999999999	75.4863813229572\\
848.09999999999	75.477481726008\\
848.199999999998	75.4803725097253\\
848.399999999991	75.4625810253388\\
849.400000000002	75.4914655679812\\
849.500000000001	75.4825800376648\\
849.800000000005	75.4912342628545\\
849.900000000014	75.4823529411765\\
851.300000000005	75.5226685459244\\
851.400000000013	75.5137991779213\\
852.700000000013	75.5511257035647\\
852.899999999988	75.5334114888628\\
853.000000000012	75.5362794514125\\
853.100000000014	75.5274261603376\\
853.999999999994	75.5532139093783\\
854.199999999992	75.5355261617699\\
854.300000000003	75.5383895131086\\
854.400000000012	75.5295494441194\\
855.499999999989	75.5610098176718\\
855.600000000012	75.552179502162\\
856.000000000004	75.5636023828992\\
856.099999999989	75.5547769212801\\
856.800000000005	75.5747461780838\\
856.89999999999	75.5659276546091\\
857.19999999999	75.5744780123644\\
857.300000000006	75.5656636342431\\
857.500000000014	75.5713619402985\\
857.600000000009	75.5625510085111\\
858.300000000008	75.5824790307549\\
858.40000000001	75.5736750145603\\
858.800000000011	75.5850506461753\\
858.899999999987	75.5762514551804\\
859.000000000008	75.5790944011175\\
859.100000000007	75.5702979515829\\
859.599999999987	75.5845062231011\\
859.700000000003	75.5757152826239\\
860.400000000014	75.5955839628123\\
860.499999999998	75.5867999070416\\
860.599999999997	75.5896363425119\\
860.700000000013	75.5808550185874\\
862.5	75.6318108045444\\
862.60000000001	75.6230439318419\\
863.100000000006	75.6371640407785\\
863.200000000011	75.6284026410286\\
863.300000000005	75.6312253880009\\
863.399999999988	75.6224667052692\\
864.000000000002	75.639393588705\\
864.199999999994	75.6218905472637\\
865.39999999999	75.6556903523975\\
865.499999999999	75.6469500924214\\
865.599999999998	75.6497631974125\\
865.700000000008	75.6410256410256\\
865.799999999994	75.6438387804596\\
865.900000000007	75.635103926097\\
866.600000000001	75.6547825083651\\
866.700000000005	75.646054453161\\
867.000000000008	75.6544804520817\\
867.100000000003	75.6457564575646\\
867.799999999992	75.6653992395437\\
867.900000000003	75.6566820276498\\
868.300000000009	75.6678949792722\\
868.400000000011	75.6591824985607\\
868.600000000009	75.6647864625302\\
868.699999999988	75.6560773480663\\
868.900000000011	75.6616800920598\\
868.999999999987	75.6529743412726\\
869.199999999988	75.658575865639\\
869.300000000007	75.6498734759604\\
870.70000000001	75.6890215893431\\
870.799999999994	75.6803306923872\\
871.199999999988	75.6914954665442\\
871.300000000001	75.6828092724352\\
872.000000000011	75.7023277147116\\
872.100000000012	75.6936482458152\\
872.600000000005	75.7075741950269\\
872.699999999994	75.698900091659\\
872.999999999994	75.7072500286336\\
873.200000000008	75.6899118286958\\
873.600000000009	75.7010415474419\\
873.699999999995	75.6923781185626\\
874.100000000006	75.7035003431709\\
874.199999999999	75.6948415875558\\
874.299999999986	75.6976212259835\\
874.400000000012	75.6889651229274\\
874.699999999998	75.6973022405121\\
874.899999999992	75.68\\
875.700000000011	75.7022151176068\\
875.800000000002	75.6935723256079\\
876.200000000009	75.7046673513637\\
876.400000000014	75.687393040502\\
877.799999999987	75.7261647112427\\
877.899999999999	75.7175398633257\\
878.099999999998	75.7230699157367\\
878.199999999996	75.7144483661619\\
879.700000000009	75.7558536030916\\
879.799999999987	75.7472440050006\\
880.100000000009	75.7555101113383\\
880.199999999987	75.7469044643871\\
880.30000000001	75.7496592457974\\
880.400000000003	75.7410562180579\\
881.199999999988	75.7630772722115\\
881.300000000007	75.7544815066939\\
881.400000000012	75.7572319909246\\
881.499999999994	75.7486388384755\\
882.500000000001	75.7761160208475\\
882.599999999998	75.7675314376345\\
882.800000000014	75.7730207271492\\
882.900000000012	75.7644394110985\\
883.000000000001	75.7671837843959\\
883.299999999999	75.7414534752094\\
883.500000000004	75.7469443186962\\
883.600000000002	75.7383727509336\\
884.300000000013	75.7575757575758\\
884.399999999988	75.7490107405314\\
885.099999999988	75.7681879801175\\
885.300000000011	75.7510729613734\\
885.799999999992	75.7647590021447\\
885.899999999998	75.7562076749436\\
886.100000000005	75.7616790792146\\
886.299999999999	75.7445848375451\\
886.90000000001	75.76099210823\\
887.000000000001	75.7524518092661\\
887.4	75.7633802816901\\
887.600000000014	75.7463106905486\\
888.599999999988	75.7736018904017\\
888.800000000013	75.756553043087\\
889.399999999996	75.7729061270377\\
889.499999999992	75.7643884892086\\
889.799999999995	75.7725587144623\\
889.900000000001	75.7640449438202\\
890.100000000006	75.7694900022467\\
890.400000000014	75.7439640651319\\
890.599999999991	75.7494105759515\\
890.699999999992	75.7409070498428\\
890.9	75.7463524130191\\
891.000000000009	75.7378520929189\\
891.099999999986	75.7405745062837\\
891.200000000007	75.7320767418378\\
891.700000000009	75.74568288854\\
891.800000000008	75.7371902679673\\
892.099999999995	75.7453485765523\\
892.200000000003	75.7368598005155\\
893.500000000009	75.772157564906\\
893.700000000005	75.7552025061535\\
893.8	75.7579147555655\\
893.899999999989	75.7494407158837\\
894.399999999998	75.7629960871996\\
894.499999999998	75.7545271629779\\
894.999999999998	75.7680706066361\\
895.100000000013	75.7596067917784\\
895.20000000001	75.7623143080532\\
895.300000000013	75.7538530265803\\
895.700000000014	75.7646796159857\\
895.800000000009	75.7562227927224\\
896.700000000008	75.7805530776093\\
896.800000000001	75.7721039134798\\
896.900000000013	75.7748049052397\\
896.999999999994	75.7663582655222\\
897.699999999991	75.7852528402762\\
897.999999999993	75.7599376461419\\
898.400000000012	75.7707289927657\\
898.50000000001	75.7622969062987\\
899.800000000008	75.7973108123125\\
899.900000000002	75.7888888888889\\
900.000000000001	75.7915787134763\\
900.100000000008	75.7831592979338\\
900.69999999999	75.7992895204263\\
900.899999999987	75.7824639289678\\
901.399999999999	75.79589572934\\
901.500000000003	75.7874889086069\\
901.79999999999	75.7955427430979\\
901.899999999989	75.7871396895787\\
901.999999999997	75.7898237445959\\
902.100000000006	75.7814231877632\\
902.800000000008	75.8001993576254\\
902.900000000006	75.7918050941307\\
903.200000000008	75.7998450127311\\
903.400000000005	75.7830658550083\\
904.7	75.8178603006189\\
904.800000000004	75.8094817106863\\
905.000000000009	75.8148270909292\\
905.100000000012	75.8064516129032\\
905.199999999995	75.8091240472771\\
905.300000000008	75.80075104926\\
905.599999999987	75.8087666997902\\
905.700000000005	75.8003974387282\\
906.100000000001	75.8110792319576\\
906.199999999992	75.8027143330023\\
907.100000000011	75.8267195767196\\
907.300000000002	75.8100066122989\\
907.599999999998	75.8180015423598\\
907.70000000001	75.8096497025777\\
907.800000000002	75.8123141315123\\
907.900000000011	75.8039647577093\\
907.999999999989	75.8066292258562\\
908.200000000006	75.7899372454035\\
909.499999999993	75.8245382585752\\
909.600000000004	75.8162031438936\\
910.000000000008	75.82683221624\\
910.09999999999	75.8185014282575\\
910.80000000001	75.8370842024372\\
910.900000000004	75.8287596048299\\
911.299999999997	75.8393680052666\\
911.400000000013	75.8310477235326\\
912.19999999999	75.8522415871972\\
912.300000000015	75.8439281017098\\
912.599999999987	75.8518680837077\\
912.699999999987	75.8435582822086\\
914.100000000008	75.8805513016845\\
914.200000000015	75.8722519960626\\
914.699999999995	75.8854394403148\\
914.800000000006	75.8771450431741\\
915.400000000015	75.8929546695795\\
915.499999999994	75.8846657929227\\
916.399999999989	75.9083469721768\\
916.599999999986	75.8917857532453\\
916.999999999995	75.9023007305637\\
917.09999999999	75.8940252943742\\
917.400000000005	75.9019073569482\\
917.50000000001	75.8936355710549\\
917.700000000008	75.898888646764\\
917.799999999994	75.8906198932346\\
917.899999999999	75.8932461873638\\
918.099999999998	75.8767153125681\\
918.89999999999	75.8977149075082\\
919.000000000005	75.8894570775759\\
919.499999999993	75.9025663331883\\
919.600000000009	75.8943133630532\\
919.799999999999	75.8995542993804\\
920.000000000001	75.8830561895446\\
920.200000000009	75.8882972943605\\
920.299999999988	75.8800521512386\\
920.50000000001	75.8852922007386\\
920.6	75.8770500705985\\
921.299999999989	75.8953766008248\\
921.500000000015	75.87890625\\
921.799999999997	75.8867556134071\\
921.900000000001	75.8785249457701\\
922	75.8811408740917\\
922.099999999992	75.8729126003036\\
922.299999999993	75.8781439722463\\
922.400000000005	75.869918699187\\
923.299999999994	75.8934372969461\\
923.399999999994	75.8852192744992\\
923.700000000012	75.893050443819\\
923.799999999995	75.8848360212144\\
924.899999999991	75.9135135135135\\
925	75.9053075343206\\
925.600000000005	75.9209247056282\\
925.699999999986	75.9127241304817\\
925.999999999987	75.9205269409351\\
926.200000000006	75.9041347295693\\
926.600000000008	75.9145354483652\\
926.900000000005	75.8899676375405\\
927.2	75.8977677127143\\
927.40000000001	75.8814016172507\\
927.60000000001	75.8866012719629\\
927.700000000013	75.8784220737228\\
928.000000000011	75.8862191574184\\
928.099999999995	75.8780435251023\\
928.300000000005	75.883239982766\\
928.400000000006	75.8750673128702\\
928.599999999997	75.8802627328524\\
928.699999999997	75.8720930232558\\
929.300000000006	75.8876694641704\\
929.600000000004	75.8631816715069\\
930.000000000013	75.8735619825825\\
930.200000000002	75.8572503493497\\
930.700000000007	75.8702191663085\\
930.900000000015	75.8539205155746\\
932.100000000007	75.8850032181935\\
932.199999999987	75.8768636704923\\
933.600000000013	75.9130341651494\\
933.700000000012	75.9049046905119\\
934.599999999999	75.9281052744196\\
934.800000000012	75.9118622312547\\
935.49999999999	75.9298845660539\\
935.699999999992	75.9136567642659\\
936.299999999993	75.929090132422\\
936.399999999995	75.9209823812066\\
936.600000000006	75.9261236254938\\
936.900000000015	75.9018143009605\\
937.299999999994	75.9120972903776\\
937.399999999986	75.904\\
937.5	75.9065699658703\\
937.599999999998	75.8984749920017\\
937.700000000007	75.9010449989337\\
937.799999999991	75.8929523403348\\
938.600000000008	75.9134973900074\\
938.699999999999	75.905411163187\\
939.200000000007	75.9182369849888\\
939.299999999987	75.9101554183521\\
939.600000000012	75.9178461211025\\
939.699999999996	75.9097680357523\\
940.100000000013	75.9200170176558\\
940.200000000005	75.9119429969159\\
941.000000000004	75.9324195090851\\
941.400000000008	75.9001593202337\\
941.500000000007	75.9027187765505\\
941.600000000013	75.8946585961559\\
941.699999999988	75.8972180930134\\
941.799999999988	75.88916020809\\
942.600000000002	75.9096213005198\\
942.700000000002	75.9015697921086\\
942.800000000003	75.9041255700498\\
942.999999999989	75.8880288410561\\
943.100000000009	75.8905852417303\\
943.200000000011	75.8825400190819\\
944.500000000008	75.9157315265721\\
944.699999999995	75.8996613039797\\
944.999999999986	75.9073113956195\\
945.200000000016	75.8912514545647\\
945.499999999989	75.8989001692047\\
945.599999999992	75.8908744845088\\
945.699999999995	75.8934235567773\\
945.799999999989	75.8854001480072\\
946.700000000001	75.9083227714406\\
946.799999999996	75.9003062625409\\
946.899999999996	75.9028511087645\\
947.000000000001	75.8948368704466\\
947.299999999999	75.9024699176694\\
947.500000000003	75.886449978894\\
947.699999999995	75.8915382992192\\
947.800000000002	75.8835320181454\\
948.000000000009	75.8886193439511\\
948.099999999993	75.8806159038178\\
948.399999999996	75.8882445967317\\
948.499999999995	75.8802445709467\\
948.600000000002	75.8827869716454\\
948.699999999997	75.8747892074199\\
949.700000000007	75.9001895135818\\
949.899999999994	75.8842105263158\\
949.999999999992	75.8867487632881\\
950.100000000001	75.8787623658177\\
950.299999999986	75.8838383838384\\
950.40000000001	75.8758548132562\\
950.700000000004	75.8834665544804\\
950.900000000011	75.8675078864353\\
951.300000000007	75.8776539836031\\
951.4	75.8696794534945\\
951.800000000016	75.8798193087509\\
951.900000000012	75.8718487394958\\
952.1	75.8769166141567\\
952.299999999986	75.8609827803444\\
952.799999999987	75.8736488613706\\
952.899999999997	75.8656873032529\\
953.100000000003	75.8707511540076\\
953.199999999991	75.8627924053289\\
953.299999999997	75.8653241032096\\
953.400000000004	75.8573675930781\\
954.499999999988	75.8851875130945\\
954.700000000014	75.8692919983243\\
955.200000000016	75.8819219093479\\
955.299999999986	75.8739794850325\\
955.7	75.884076166562\\
955.80000000001	75.8761376713045\\
956.200000000012	75.886228171076\\
956.299999999993	75.8782936010038\\
956.599999999999	75.8858576356225\\
956.699999999996	75.8779264214047\\
956.799999999994	75.8804472776675\\
956.999999999988	75.8645909518337\\
957.699999999987	75.8822301106703\\
957.8	75.874308382921\\
957.999999999997	75.879344536061\\
958.300000000016	75.855592654424\\
958.800000000006	75.8681822922098\\
958.899999999995	75.8602711157456\\
959.800000000014	75.8829044692155\\
960.000000000001	75.8670971773773\\
960.799999999996	75.8871890935581\\
960.999999999994	75.8713973571949\\
961.500000000009	75.8839434276206\\
961.600000000004	75.8760528231257\\
961.699999999994	75.8785610313995\\
961.799999999988	75.8706726270922\\
961.999999999993	75.8756885978589\\
962.099999999993	75.8678029515693\\
962.300000000012	75.8728179551122\\
962.499999999993	75.8570538125909\\
963.099999999995	75.8720930232558\\
963.19999999999	75.864216754905\\
963.599999999991	75.8742347203487\\
963.699999999988	75.8663623158332\\
964.300000000009	75.8813770219826\\
964.400000000005	75.8735095904614\\
964.80000000001	75.8835112446886\\
964.899999999986	75.8756476683938\\
965.200000000013	75.8831451362271\\
965.299999999991	75.8752848560182\\
965.700000000012	75.8852764547525\\
965.799999999994	75.8774200227767\\
966.100000000006	75.8849099565307\\
966.300000000001	75.8692052980132\\
966.6	75.8766939071067\\
966.700000000005	75.8688456764584\\
967.200000000006	75.8813191357387\\
967.299999999994	75.8734752946041\\
967.500000000006	75.8784621744523\\
967.800000000003	75.8549436925302\\
967.899999999988	75.8574380165289\\
967.999999999986	75.8496023138106\\
968.500000000006	75.8620689655172\\
968.800000000016	75.8385798328001\\
969.500000000015	75.8560231023102\\
969.800000000003	75.8325600577379\\
969.900000000007	75.8350515463918\\
969.999999999993	75.8272343057417\\
970.099999999993	75.8297258297258\\
970.20000000001	75.8219107492528\\
971.000000000005	75.841828853877\\
971.19999999999	75.8262122928035\\
971.599999999999	75.8361634249254\\
971.699999999997	75.8283597448034\\
971.799999999994	75.8308467949378\\
971.900000000002	75.8230452674897\\
972.300000000003	75.8329905388729\\
972.500000000013	75.817396668723\\
972.899999999986	75.8273381294964\\
973.099999999994	75.8117550349363\\
973.599999999985	75.8241758241758\\
973.700000000013	75.8163894023413\\
973.9	75.8213552361396\\
974.099999999994	75.8057893656333\\
974.399999999996	75.8132375577219\\
974.499999999993	75.8054586497024\\
975.100000000016	75.8203445447088\\
975.199999999999	75.8125704911309\\
975.399999999985	75.8175294720656\\
975.600000000012	75.8019883160808\\
976.099999999993	75.8143822987093\\
976.299999999988	75.7988529291274\\
977.800000000003	75.8359750485735\\
977.900000000014	75.8282208588957\\
978.700000000012	75.8479771148345\\
978.799999999995	75.8402288282766\\
979.5	75.8574928542262\\
979.59999999999	75.849749923446\\
979.900000000003	75.8571428571429\\
980.000000000001	75.8494031221304\\
980.399999999998	75.859255481897\\
980.499999999998	75.8515194778707\\
980.600000000004	75.8539818496992\\
980.69999999999	75.8462479608483\\
980.899999999987	75.8511722731906\\
981.100000000014	75.835711373828\\
981.199999999999	75.838173851014\\
981.300000000009	75.8304463012024\\
981.399999999989	75.8329088130413\\
981.600000000001	75.817459509015\\
982.200000000009	75.8322304794869\\
982.299999999986	75.8245114006515\\
982.500000000011	75.8294321188683\\
982.700000000012	75.8140008140008\\
982.90000000001	75.8189216683622\\
983	75.811209439528\\
983.899999999985	75.8333333333333\\
983.999999999989	75.8256274768824\\
984.500000000007	75.8379037172456\\
984.599999999987	75.8302020920077\\
984.999999999991	75.8400162420059\\
985.100000000013	75.8323183110029\\
986	75.8543758239529\\
986.099999999991	75.8466842425471\\
987.399999999998	75.8784810126582\\
987.500000000009	75.8707978938842\\
988.000000000017	75.8830077927335\\
988.099999999998	75.8753288807934\\
988.20000000001	75.8777699079227\\
988.30000000001	75.8700930797248\\
988.900000000001	75.8847320525784\\
989.099999999995	75.8693894055803\\
989.199999999997	75.8718285656525\\
989.300000000004	75.8641600970285\\
989.400000000014	75.866599292572\\
989.499999999989	75.8589329021827\\
989.699999999987	75.863810870883\\
989.800000000006	75.8561470855642\\
990.800000000015	75.8805126652538\\
990.900000000012	75.8728557013118\\
990.999999999993	75.8752900817274\\
991.099999999995	75.8676351896691\\
991.199999999994	75.8700696055684\\
991.299999999996	75.8624167843454\\
991.600000000009	75.8697186649188\\
991.700000000014	75.8620689655172\\
992.099999999993	75.8718000403145\\
992.200000000006	75.8641539856898\\
992.800000000003	75.8787390472354\\
992.900000000006	75.8710976837865\\
993.099999999985	75.8759565042288\\
993.199999999998	75.8683177287828\\
993.600000000002	75.8780315990742\\
994.000000000007	75.8475002514838\\
994.200000000005	75.8523584431258\\
994.500000000014	75.8294791876131\\
994.600000000004	75.8319091183271\\
994.699999999995	75.8242862887012\\
994.799999999986	75.8267162528897\\
994.900000000004	75.8190954773869\\
995.399999999994	75.8312405826218\\
995.499999999986	75.8236239453596\\
995.600000000003	75.8260520237019\\
995.700000000011	75.8184374372364\\
995.900000000017	75.8232931726908\\
996.099999999998	75.8080706685405\\
996.29999999999	75.8129265355279\\
996.400000000016	75.805318615153\\
997.099999999994	75.8223024468512\\
997.400000000011	75.7994987468672\\
997.599999999984	75.8043500050115\\
997.799999999996	75.7891572301834\\
998.600000000006	75.8085511164514\\
998.69999999999	75.8009611533841\\
998.900000000015	75.8058058058058\\
999.000000000001	75.7982183965569\\
999.100000000011	75.8006405124099\\
999.200000000001	75.793055138597\\
999.499999999997	75.8003201280512\\
999.599999999989	75.7927378213464\\
999.900000000015	75.8\\
1000	75.7924207579242\\
1000.29999999999	75.7996801279488\\
1000.4	75.792103948026\\
1000.6	75.7969421405016\\
1000.7	75.7893685051958\\
1000.8	75.7917873913478\\
1000.9	75.7842157842158\\
1001.20000000001	75.7914710875861\\
1001.29999999999	75.783902536449\\
1001.6	75.7911550364381\\
1001.70000000002	75.7835895388301\\
1001.99999999999	75.790839237601\\
1002.1	75.7832767910597\\
1003.10000000001	75.8074162679426\\
1003.2	75.7998604604804\\
1003.40000000001	75.8046836073742\\
1003.59999999999	75.7895785593305\\
1004.80000000001	75.8184894019305\\
1004.9	75.8109452736318\\
1005.09999999999	75.8157580580979\\
1005.20000000001	75.8082164528002\\
1005.29999999999	75.8106226377561\\
1005.59999999999	75.7880083523914\\
1005.90000000001	75.7952286282306\\
1006.00000000001	75.7876950601332\\
1006.49999999998	75.7997218358832\\
1006.60000000002	75.7921923115129\\
1006.89999999999	75.7994041708044\\
1006.99999999999	75.7918776685533\\
1007.40000000002	75.8014888337469\\
1007.50000000001	75.793965859468\\
1008.39999999999	75.8155676747645\\
1008.49999999999	75.8080507634345\\
1008.60000000001	75.8104490928918\\
1008.79999999999	75.795420755278\\
1009.30000000001	75.8074103427779\\
1009.6	75.7848865999802\\
1009.8	75.7896821467472\\
1010.00000000001	75.7746757746758\\
1010.2	75.7794714441255\\
1010.3	75.771971496437\\
1010.70000000001	75.7815591610605\\
1010.8	75.7740627163913\\
1010.90000000001	75.7764589515331\\
1010.99999999999	75.7689644941153\\
1011.20000000001	75.773756550974\\
1011.29999999999	75.7662645837453\\
1011.9	75.7806324110672\\
1012.00000000001	75.7731449461516\\
1012.19999999999	75.7779314432481\\
1012.40000000002	75.762962962963\\
1013.10000000001	75.7797078562969\\
1013.2	75.7722293496496\\
1013.40000000001	75.7770103601381\\
1013.50000000002	75.7695343330702\\
1013.69999999999	75.7743144604459\\
1013.79999999999	75.7668409113325\\
1014.3	75.778785488959\\
1014.49999999999	75.7638478218017\\
1015.49999999999	75.7877116975187\\
1015.60000000002	75.7802500738407\\
1015.80000000002	75.7850182104538\\
1015.9	75.7775590551181\\
1015.99999999999	75.779942919004\\
1016.19999999998	75.7650300108236\\
1016.5	75.772181782412\\
1016.60000000001	75.7647290252779\\
1016.8	75.7694955256171\\
1016.99999999999	75.7545964015338\\
1017.9	75.7760314341847\\
1017.99999999999	75.768588547294\\
1018.09999999999	75.7709683755647\\
1018.2	75.763527447707\\
1019.49999999999	75.7944291879168\\
1019.59999999999	75.7869961753457\\
1019.80000000001	75.7917442886557\\
1019.90000000001	75.7843137254902\\
1020.10000000001	75.7890609684376\\
1020.29999999999	75.7742061936495\\
1020.40000000001	75.7765801077903\\
1020.5	75.769155398785\\
1020.70000000001	75.7739028213166\\
1020.79999999999	75.7664805563718\\
1020.99999999999	75.7712271080208\\
1021.10000000001	75.7638072855464\\
1022.30000000002	75.7922535211268\\
1022.39999999999	75.7848410757946\\
1022.99999999999	75.7990421268693\\
1023.09999999999	75.7916340891321\\
1023.3	75.796365057651\\
1023.40000000001	75.7889594528578\\
1023.50000000001	75.7913247362251\\
1023.60000000001	75.7839210706262\\
1023.70000000001	75.7862863840594\\
1023.80000000001	75.7788846567048\\
1024	75.7836148813592\\
1024.1	75.776215582894\\
1024.5	75.7856724575444\\
1024.6	75.7782765687518\\
1024.9	75.7853658536585\\
1024.99999999999	75.7779728806946\\
1025.20000000001	75.7826977470009\\
1025.29999999999	75.7753071971913\\
1025.99999999999	75.7918331546633\\
1026.19999999999	75.7770632368703\\
1026.59999999999	75.7865004382975\\
1026.70000000001	75.7791195948578\\
1026.9	75.7838364167478\\
1027	75.7764579885114\\
1027.1	75.7788161993769\\
1027.20000000001	75.7714396962913\\
1028.3	75.7973551147414\\
1028.39999999999	75.7899854156539\\
1028.50000000001	75.7923391016916\\
1028.69999999999	75.7776049766719\\
1028.80000000002	75.7799591797065\\
1028.90000000001	75.7725947521866\\
1028.99999999998	75.7749489845496\\
1029.10000000002	75.767586474932\\
1029.2	75.7699407364228\\
1029.3	75.7625801437731\\
1029.70000000001	75.7719945620509\\
1029.8	75.7646373434314\\
1029.9	75.7669902912621\\
1030.10000000001	75.752281110464\\
1030.5	75.7616922181254\\
1030.59999999999	75.7543417095178\\
1030.89999999999	75.7613967022308\\
1031	75.7540490738047\\
1031.20000000001	75.7587510908562\\
1031.50000000002	75.7367196587825\\
1032.1	75.750823483821\\
1032.20000000001	75.7434854209048\\
1032.39999999999	75.7481840193705\\
1032.49999999999	75.7408483439861\\
1032.79999999999	75.7478942782457\\
1032.90000000001	75.7405614714424\\
1033	75.7429096892847\\
1033.20000000001	75.7282492983645\\
1033.29999999999	75.7305980259338\\
1033.49999999999	75.7159442724458\\
1033.9	75.7253384912959\\
1034.09999999999	75.7106942564301\\
1034.7	75.724777734828\\
1034.80000000001	75.7174606242149\\
1035	75.7221524490387\\
1035.10000000001	75.7148377125193\\
1035.4	75.7218734910671\\
1035.5	75.714561606798\\
1036.10000000001	75.7286238177958\\
1036.19999999999	75.7213162211715\\
1036.30000000002	75.7236588189888\\
1036.4	75.7163531114327\\
1036.6	75.7210379087489\\
1036.7	75.7137345679012\\
1036.90000000001	75.718418514947\\
1037	75.7111175392923\\
1037.70000000001	75.7275004817884\\
1037.79999999999	75.7202042585991\\
1038.30000000001	75.7318952234206\\
1038.39999999999	75.7246027924892\\
1038.7	75.731613400077\\
1039.00000000001	75.7097488210952\\
1039.10000000001	75.7120862201694\\
1039.19999999998	75.7048013085731\\
1039.40000000001	75.7094757094757\\
1039.49999999999	75.702193151212\\
1039.70000000001	75.7068667051356\\
1039.79999999998	75.6995864987018\\
1040.5	75.7159331155103\\
1040.59999999999	75.7086576342846\\
1041.79999999999	75.7366349937614\\
1041.99999999999	75.7220996065637\\
1042.2	75.7267581310563\\
1042.4	75.7122302158273\\
1042.5	75.71455975446\\
1042.60000000002	75.7072983600269\\
1042.90000000001	75.7142857142857\\
1043	75.7070271306682\\
1043.19999999998	75.7116840793636\\
1043.29999999999	75.7044278320874\\
1043.69999999999	75.7137382640353\\
1043.79999999999	75.7064852955264\\
1044.19999999999	75.7157904816624\\
1044.39999999999	75.7012924844423\\
1044.90000000001	75.7129186602871\\
1045	75.7056740981724\\
1045.2	75.7103223954846\\
1045.39999999999	75.6958393113343\\
1046.00000000001	75.7097791798107\\
1046.10000000001	75.7025425348882\\
1046.40000000001	75.7095078834209\\
1046.50000000001	75.702274030193\\
1047.09999999998	75.7161955691367\\
1047.20000000001	75.708965912346\\
1047.29999999999	75.7112850868818\\
1047.40000000001	75.7040572792363\\
1047.49999999999	75.7063764795724\\
1047.59999999999	75.6991505201871\\
1048.00000000001	75.7084247686289\\
1048.10000000001	75.7012020606754\\
1048.2	75.7035199847372\\
1048.3	75.6962991224723\\
1048.5	75.7009345794392\\
1048.7	75.6864988558352\\
1050.30000000002	75.7235338918507\\
1050.50000000001	75.7091185988959\\
1050.60000000001	75.7114304749215\\
1050.69999999998	75.7042253521127\\
1051.40000000001	75.7203994293866\\
1051.50000000001	75.7131989349563\\
1051.59999999999	75.7155082247789\\
1051.69999999998	75.708309564556\\
1052.4	75.7244655581948\\
1052.5	75.7172715181456\\
1053.2	75.733409285104\\
1053.29999999999	75.7262198595026\\
1054.10000000001	75.7446404856763\\
1054.20000000001	75.7374561320307\\
1054.6	75.7466578173888\\
1054.70000000001	75.7394766780432\\
1055.4	75.7555660824254\\
1055.50000000001	75.748389541493\\
1055.59999999999	75.7506867481292\\
1055.69999999999	75.7435120287933\\
1056.10000000001	75.7526983525847\\
1056.20000000001	75.7455268389662\\
1056.29999999999	75.7478227943961\\
1056.39999999999	75.740653099858\\
1056.60000000001	75.745244629507\\
1056.70000000001	75.7380772142317\\
1057.30000000001	75.7518441460185\\
1057.60000000001	75.7303583246667\\
1057.79999999999	75.7349465923055\\
1058	75.7206313202911\\
1058.4	75.7298063297119\\
1058.49999999999	75.7226525599849\\
1058.7	75.7272383830752\\
1058.80000000001	75.720086882614\\
1058.9	75.7223796033994\\
1059.09999999998	75.708081570997\\
1059.20000000001	75.7103747757953\\
1059.4	75.6960830580462\\
1060.10000000001	75.7121297868327\\
1060.20000000001	75.704989154013\\
1060.3	75.7072802715956\\
1060.39999999999	75.7001414427157\\
1060.6	75.7047232959366\\
1060.80000000001	75.6904515034405\\
1061.10000000001	75.697323784395\\
1061.19999999999	75.6901912748516\\
1061.5	75.6970610399397\\
1061.70000000002	75.682802787719\\
1062.29999999999	75.6965361445783\\
1062.39999999999	75.6894117647059\\
1063.60000000001	75.7168374541694\\
1063.7	75.7097198721564\\
1063.90000000001	75.7142857142857\\
1064.00000000001	75.7071703787238\\
1064.09999999999	75.7094531103176\\
1064.19999999999	75.7023395659119\\
1064.4	75.7069046500705\\
1064.5	75.6997933496149\\
1065.30000000001	75.7180401727051\\
1065.4	75.7109338338808\\
1065.5	75.7132132132132\\
1065.6	75.706108660974\\
1065.8	75.7106670419364\\
1066.00000000001	75.6964637463653\\
1066.29999999999	75.7033008252063\\
1066.50000000001	75.6891055690981\\
1067.6	75.714151915332\\
1067.70000000001	75.7070612474246\\
1068.00000000002	75.7138844677465\\
1068.10000000001	75.7067964800599\\
1068.20000000001	75.7090704858186\\
1068.30000000001	75.7019842755522\\
1068.39999999999	75.7042583060365\\
1068.5	75.6971738723563\\
1068.79999999999	75.7039947609692\\
1068.9	75.6969130028064\\
1069.20000000001	75.703731413074\\
1069.59999999999	75.6754230157988\\
1069.7	75.6776967657506\\
1069.79999999999	75.6706234227498\\
1070	75.6751705448089\\
1070.1	75.668099420669\\
1070.20000000002	75.670372792675\\
1070.39999999999	75.6562354040168\\
1071.49999999999	75.6812243374393\\
1071.60000000001	75.6741625454885\\
1072.40000000001	75.6923076923077\\
1072.5	75.6852507924669\\
1073.00000000001	75.6965800018638\\
1073.10000000002	75.6895266492732\\
1073.2	75.6917916705488\\
1073.29999999999	75.684740078256\\
1073.50000000001	75.6892697466468\\
1073.59999999998	75.6822203595045\\
1073.79999999998	75.686749231772\\
1073.89999999999	75.6797020484171\\
1074.10000000001	75.684230124744\\
1074.3	75.6701414743113\\
1074.60000000002	75.6769330976086\\
1074.79999999998	75.6628523583589\\
1075.10000000001	75.6696428571429\\
1075.19999999998	75.6626057844322\\
1075.50000000002	75.6693938267014\\
1075.6	75.6623593938831\\
1076.20000000001	75.675926786212\\
1076.29999999999	75.6688963210702\\
1076.60000000001	75.6756756756757\\
1076.69999999999	75.6686478454681\\
1077.09999999999	75.6776828815448\\
1077.30000000001	75.663634676072\\
1077.49999999999	75.6681514476615\\
1077.59999999999	75.6611301846525\\
1078.00000000001	75.6701604674891\\
1078.1	75.6631422741606\\
1078.30000000002	75.6676557863502\\
1078.39999999999	75.6606397774687\\
1078.79999999999	75.6696635462045\\
1078.90000000001	75.6626506024096\\
1079.20000000002	75.6694153618086\\
1079.3	75.6624050398369\\
1079.69999999998	75.6714206334506\\
1079.80000000001	75.6644133716085\\
1080.40000000002	75.6779268857011\\
1080.50000000001	75.6709235609846\\
1080.70000000001	75.6754256106588\\
1080.80000000001	75.6684244610972\\
1081.30000000001	75.6796744960237\\
1081.39999999998	75.6726768377254\\
1081.8	75.6816711341159\\
1081.89999999999	75.6746765249538\\
1082	75.67692449866\\
1082.3	75.6559497413156\\
1082.8	75.6671899529042\\
1082.90000000001	75.6602031394275\\
1083.19999999998	75.6669435982646\\
1083.40000000002	75.6529764651592\\
};
\addplot [color=mycolor2, forget plot]
  table[row sep=crcr]{%
1083.40000000002	75.6529764651592\\
1084.00000000001	75.6664514343695\\
1084.1	75.6594724220624\\
1084.19999999999	75.661717236927\\
1084.29999999999	75.6547399483585\\
1084.39999999999	75.6569847856155\\
1084.50000000001	75.6500092199889\\
1084.79999999999	75.6567425569177\\
1085	75.6427978988112\\
1085.10000000001	75.6450423884998\\
1085.3	75.6311037405565\\
1085.40000000001	75.6333486872409\\
1085.60000000001	75.619416044948\\
1086.20000000002	75.6328822608856\\
1086.30000000001	75.6259204712813\\
1087.1	75.6438557763061\\
1087.19999999999	75.6368987399982\\
1087.4	75.6413793103448\\
1087.50000000001	75.6344244207429\\
1087.7	75.6389042103328\\
1087.8	75.6319514661274\\
1088.7	75.6520940484938\\
1088.79999999999	75.6451464780972\\
1088.89999999999	75.6473829201102\\
1088.99999999998	75.6404370581214\\
1089.3	75.6471452175509\\
1089.5	75.6332599118943\\
1089.59999999999	75.6354960080756\\
1089.79999999999	75.6216166620791\\
1090.4	75.6350298028427\\
1090.5	75.6280946268109\\
1091.19999999999	75.6437276642536\\
1091.29999999999	75.6367967747847\\
1091.6	75.643491801777\\
1091.7	75.6365634731636\\
1091.90000000001	75.6410256410256\\
1092.00000000001	75.6340994414431\\
1092.49999999999	75.6452498627128\\
1092.59999999999	75.6383270797108\\
1092.69999999999	75.6405563689605\\
1092.8	75.6336352822765\\
1093.19999999999	75.6425500777463\\
1093.30000000002	75.6356319736601\\
1093.79999999999	75.6467684431849\\
1093.90000000001	75.6398537477148\\
1094.59999999999	75.6554307116105\\
1094.7	75.6485202776763\\
1094.80000000001	75.6507443602156\\
1095.00000000002	75.6369281344169\\
1095.50000000002	75.6480467323841\\
1095.70000000001	75.6342398247855\\
1095.90000000001	75.6386861313869\\
1095.99999999999	75.6317854210382\\
1097.39999999999	75.6628701594533\\
1097.49999999998	75.6559766763848\\
1097.80000000001	75.6626286547044\\
1097.89999999999	75.655737704918\\
1098.10000000001	75.6601711892187\\
1098.2	75.653282345443\\
1098.99999999999	75.6710035483577\\
1099.2	75.6572364231784\\
1099.80000000001	75.6705155014092\\
1099.90000000001	75.6636363636364\\
1100.6	75.6791132915417\\
1100.7	75.672238372093\\
1100.80000000001	75.6744481787628\\
1100.89999999998	75.6675749318801\\
1101.20000000001	75.674203214383\\
1101.29999999998	75.667332485927\\
1101.49999999998	75.6717501815541\\
1101.59999999999	75.6648815467006\\
1101.79999999998	75.669298484436\\
1101.89999999999	75.6624319419238\\
1102.09999999999	75.6668481219379\\
1102.2	75.6599836705071\\
1102.80000000001	75.6732251337383\\
1102.90000000001	75.6663644605621\\
1103.39999999999	75.677390122338\\
1103.50000000002	75.6705328017398\\
1103.80000000002	75.6771446689012\\
1104	75.6634362829454\\
1104.09999999998	75.6656402825575\\
1104.2	75.6587883727248\\
1104.40000000001	75.663196016297\\
1104.5	75.6563461886656\\
1105.10000000002	75.6695620702135\\
1105.2	75.6627160047046\\
1105.7	75.6737203834328\\
1105.79999999998	75.6668776562076\\
1106.1	75.6734767673115\\
1106.20000000001	75.6666365362018\\
1106.4	75.6710347943967\\
1106.59999999998	75.6573597180808\\
1106.8	75.661758063059\\
1106.89999999999	75.6549232158988\\
1107.09999999999	75.6593208092486\\
1107.2	75.6524880339565\\
1108	75.6700658785308\\
1108.1	75.6632376827287\\
1108.20000000001	75.6654335468736\\
1108.3	75.6586070010826\\
1108.40000000001	75.6608028867839\\
1108.60000000001	75.647154324885\\
1108.90000000001	75.653742110009\\
1108.99999999999	75.6469209268776\\
1109.40000000001	75.6557007661109\\
1109.60000000001	75.6420654230873\\
1110	75.6508422664625\\
1110.09999999999	75.6440281030445\\
1110.3	75.6484149855908\\
1110.40000000002	75.6416028815849\\
1110.7	75.6481814908174\\
1110.8	75.6413718606535\\
1111.00000000002	75.6457564575646\\
1111.1	75.6389488840893\\
1111.29999999999	75.6433327334893\\
1111.49999999998	75.6297229219144\\
1113.69999999999	75.6778595798168\\
1113.79999999998	75.6710656252805\\
1113.89999999999	75.673249551167\\
1114.09999999998	75.6596661281637\\
1114.5	75.6684012201687\\
1114.6	75.6616129900422\\
1114.79999999999	75.6659790115706\\
1114.9	75.6591928251121\\
1115.49999999999	75.6722839727501\\
1115.59999999999	75.6655014788922\\
1115.90000000001	75.6720430107527\\
1116.09999999999	75.6584841426268\\
1116.40000000001	75.6650246305419\\
1116.5	75.6582482536271\\
1116.99999999999	75.6691433175186\\
1117.10000000001	75.6623702112424\\
1117.19999999999	75.6645484650497\\
1117.29999999998	75.6577769822803\\
1117.39999999999	75.6599552572707\\
1117.50000000002	75.6531853972799\\
1117.90000000001	75.6618962432916\\
1118	75.6551292370986\\
1118.29999999999	75.6616595135908\\
1118.5	75.6481315930628\\
1119.3	75.6655351080936\\
1119.40000000001	75.6587762393926\\
1120.30000000002	75.6783291681542\\
1120.49999999999	75.6648224165626\\
1121.60000000001	75.6886868146563\\
1121.69999999999	75.681939739704\\
1121.8	75.6841073179428\\
1121.90000000002	75.6773618538324\\
1122.7	75.694691841824\\
1122.80000000001	75.6879508415709\\
1123.3	75.698771586256\\
1123.6	75.6785618937439\\
1123.79999999999	75.6828899368271\\
1123.90000000001	75.6761565836299\\
1124.3	75.6848096762718\\
1124.4	75.6780791462872\\
1124.49999999999	75.6802418637738\\
1124.69999999998	75.6667852062589\\
1125.2	75.6775970852217\\
1125.3	75.6708725786387\\
1125.39999999999	75.6730342070191\\
1125.49999999999	75.6663113006396\\
1126.00000000001	75.6771157090844\\
1126.09999999999	75.670396022021\\
1126.90000000002	75.6876663708962\\
1127	75.6809511134771\\
1127.29999999999	75.6874223877949\\
1127.5	75.6739978715857\\
1127.79999999999	75.680468126607\\
1127.90000000001	75.6737588652482\\
1128.39999999999	75.6845369960124\\
1128.50000000001	75.6778309409888\\
1129.09999999998	75.6907545164718\\
1129.29999999998	75.6773508057376\\
1129.70000000001	75.6859621171889\\
1129.80000000001	75.6792636516506\\
1130.5	75.6943215991509\\
1130.60000000001	75.687627133634\\
1130.9	75.6940760389036\\
1131.19999999999	75.6740033589676\\
1131.40000000002	75.6783031374282\\
1131.49999999999	75.6716154118063\\
1131.69999999999	75.6759144725216\\
1131.79999999999	75.6692287304532\\
1131.99999999998	75.6735270735801\\
1132.1	75.6668433139021\\
1132.80000000001	75.6818783652573\\
1132.89999999999	75.6751985878199\\
1133.09999999999	75.6794917049065\\
1133.20000000002	75.6728139062914\\
1133.3	75.6749602964532\\
1133.69999999999	75.6482624801552\\
1133.80000000002	75.6504100890731\\
1133.99999999998	75.6370690415307\\
1134.59999999999	75.6499515290385\\
1134.79999999999	75.6366199665169\\
1135.29999999999	75.6473489519112\\
1135.49999999999	75.634026065516\\
1136.19999999998	75.6490363460354\\
1136.4	75.6357237131544\\
1137.60000000001	75.661422167531\\
1137.8	75.648123736708\\
1138.30000000001	75.658819395643\\
1138.40000000002	75.6521739130435\\
1138.60000000001	75.6564503381048\\
1138.7	75.6498068141904\\
1140.09999999999	75.679705314857\\
1140.20000000002	75.6730684907481\\
1140.39999999999	75.6773345024112\\
1140.60000000002	75.6640659244324\\
1140.70000000001	75.6661991584853\\
1140.80000000002	75.6595670085021\\
1141	75.6638331434581\\
1141.2	75.6505739069482\\
1141.50000000001	75.6569726699369\\
1141.6	75.6503459753\\
1142.09999999998	75.6610050779198\\
1142.19999999999	75.6543815109866\\
1142.3	75.656512605042\\
1142.40000000001	75.6498905908096\\
1142.69999999999	75.6562828141407\\
1142.80000000002	75.6496631376323\\
1143.10000000001	75.6560531840448\\
1143.20000000001	75.6494358436106\\
1144.19999999999	75.6707157214017\\
1144.3	75.6641034603286\\
1144.7	75.672606568833\\
1144.8	75.6659970303083\\
1144.90000000001	75.6681222707424\\
1145	75.661514278229\\
1145.29999999999	75.6678889470927\\
1145.39999999999	75.6612832824094\\
1145.49999999999	75.663407821229\\
1145.7	75.6502007331122\\
1146.09999999999	75.6586983074507\\
1146.2	75.6520980546105\\
1146.3	75.6542219120726\\
1146.39999999999	75.6476232010467\\
1146.59999999999	75.6518705851574\\
1146.70000000001	75.6452738053715\\
1147.10000000001	75.6537656903766\\
1147.20000000002	75.647171620326\\
1147.4	75.6514161220043\\
1147.49999999999	75.644823980481\\
1148.09999999998	75.657550949312\\
1148.19999999999	75.6509622920839\\
1148.30000000001	75.6530825496343\\
1148.39999999999	75.6464954288202\\
1148.99999999999	75.6592115568706\\
1149.20000000001	75.6460454189507\\
1149.6	75.6545185700618\\
1149.69999999999	75.6479387719603\\
1150.29999999998	75.6606397774687\\
1150.59999999998	75.6409142261232\\
1150.70000000001	75.6430309350017\\
1150.80000000001	75.6364584238422\\
1150.90000000001	75.6385751520417\\
1151.00000000001	75.6320041699244\\
1151.19999999998	75.6362372969686\\
1151.4	75.6231003039514\\
1152	75.6357955038625\\
1152.1	75.6292310362784\\
1152.5	75.6376887038001\\
1152.70000000002	75.6245662734212\\
1152.79999999999	75.6266805447133\\
1152.89999999999	75.6201214223764\\
1153.00000000001	75.6222357124274\\
1153.09999999999	75.6156781130767\\
1153.39999999999	75.6220199393151\\
1153.5	75.6154646324549\\
1153.80000000001	75.6218043157986\\
1154	75.6086994194611\\
1154.3	75.6150381150381\\
1154.39999999999	75.6084885231702\\
1154.80000000002	75.6169365313014\\
1154.99999999999	75.6038438230456\\
1155.1	75.6059556786704\\
1155.40000000001	75.5863262656859\\
1155.9	75.5968858131488\\
1156.00000000001	75.5903468558083\\
1156.29999999999	75.596679349706\\
1156.40000000001	75.5901426718547\\
1157.8	75.6196562742897\\
1157.9	75.6131260794473\\
1158.59999999999	75.627858807284\\
1158.7	75.6213324128409\\
1158.79999999999	75.6234360169126\\
1158.9	75.6169111302847\\
1159	75.6190147528255\\
1159.10000000001	75.6124913733609\\
1159.19999999998	75.6145950142327\\
1159.29999999999	75.60807314128\\
1159.60000000002	75.614383030094\\
1159.70000000002	75.6078634247284\\
1160.39999999999	75.622576475657\\
1160.6	75.6095459636426\\
1160.70000000002	75.6116471399035\\
1160.8	75.6051339477991\\
1161.80000000001	75.6261296152853\\
1162	75.6131141898287\\
1162.4	75.6215053763441\\
1162.50000000002	75.6150008601411\\
1163.10000000001	75.6275790921596\\
1163.2	75.6210779678501\\
1163.40000000002	75.6252685861624\\
1163.7	75.6057741880048\\
1164.40000000001	75.6204379562044\\
1164.50000000001	75.6139447020436\\
1164.70000000002	75.6181318681319\\
1164.80000000001	75.6116404841617\\
1165.29999999998	75.6221039986271\\
1165.39999999999	75.6156156156156\\
1165.49999999998	75.617707618394\\
1165.6	75.6112207257442\\
1165.69999999998	75.6133127466118\\
1165.79999999999	75.606827343683\\
1166.29999999998	75.6172839506173\\
1166.50000000001	75.6043202468713\\
1166.59999999999	75.606411245393\\
1166.80000000001	75.5934527380238\\
1167.69999999999	75.6122623736941\\
1167.8	75.6057881667951\\
1169.3	75.6370788438515\\
1169.39999999999	75.6306113723814\\
1169.59999999999	75.6347781482431\\
1169.69999999999	75.6283125320568\\
1170.30000000001	75.6408065618592\\
1170.39999999998	75.6343442973088\\
1170.8	75.6426680331369\\
1171	75.6297498078729\\
1172.30000000001	75.6567724326169\\
1172.40000000001	75.6503198294243\\
1172.49999999999	75.6523963841037\\
1172.59999999999	75.6459452545408\\
1172.80000000001	75.6500980475744\\
1172.9	75.6436487638534\\
1173.10000000001	75.6478008864644\\
1173.2	75.6413534475411\\
1173.90000000002	75.6558773424191\\
1174.00000000001	75.6494336087216\\
1174.40000000001	75.6577266922095\\
1174.60000000002	75.6448454924662\\
1176.20000000001	75.6779733061294\\
1176.30000000001	75.6715402924175\\
1176.5	75.6756756756757\\
1176.69999999999	75.6628144119646\\
1176.9	75.6669498725574\\
1177.00000000002	75.6605216209328\\
1177.09999999999	75.6625891946993\\
1177.40000000001	75.6433121019108\\
1178.00000000001	75.6557168321874\\
1178.10000000002	75.6492955355627\\
1178.60000000001	75.6596250106049\\
1178.7	75.6532066508314\\
1179.00000000001	75.6594012382325\\
1179.19999999998	75.646569999152\\
1179.29999999998	75.6486348991012\\
1179.39999999999	75.6422212802035\\
1179.5	75.6442861987114\\
1179.6	75.6378740357718\\
1179.8	75.6420035596237\\
1180.00000000001	75.6291839674604\\
1180.2	75.633313564348\\
1180.29999999999	75.6269061335141\\
1180.60000000001	75.6330990090624\\
1180.70000000001	75.6266937669377\\
1181.00000000001	75.6328845991025\\
1181.1	75.6264815441923\\
1181.29999999999	75.6306077535128\\
1181.40000000001	75.6242065171392\\
1182.4	75.6448202959831\\
1182.50000000002	75.6384238119398\\
1183.6	75.6610627692828\\
1183.69999999999	75.6546713971955\\
1183.9	75.6587837837838\\
1184	75.6523942234609\\
1184.60000000001	75.6647252468979\\
1184.7	75.6583389601621\\
1184.79999999998	75.6603932821335\\
1184.9	75.6540084388186\\
1185.29999999998	75.6622237219504\\
1185.4	75.6558414171236\\
1185.89999999998	75.6661045531197\\
1185.99999999999	75.6597251496501\\
1186.29999999998	75.6658799730276\\
1186.39999999999	75.6595027391487\\
1186.50000000001	75.6615540198888\\
1186.6	75.6551782253307\\
1186.70000000002	75.6572295247725\\
1186.80000000001	75.6508551689274\\
1186.89999999999	75.6529064869419\\
1187.00000000001	75.6465335692023\\
1187.20000000002	75.6506358965721\\
1187.3	75.644264780192\\
1187.39999999999	75.6463157894737\\
1187.49999999999	75.6399461098013\\
1188.29999999998	75.6563446650959\\
1188.4	75.649978965082\\
1188.6	75.6540758812148\\
1188.69999999999	75.6477119784657\\
1188.8	75.6497602826142\\
1188.89999999998	75.6433978132885\\
1189.29999999998	75.651589036489\\
1189.39999999999	75.645229087852\\
1189.9	75.6554621848739\\
1190.00000000001	75.649105117217\\
1190.49999999999	75.6593314295313\\
1190.6	75.6529772402788\\
1190.7	75.6550218340611\\
1190.99999999998	75.6359667534212\\
1191.19999999998	75.6400570805003\\
1191.29999999999	75.6337082424039\\
1191.9	75.6459731543624\\
1192.00000000002	75.6396275480245\\
1193	75.6600452602464\\
1193.10000000002	75.6537043245055\\
1193.40000000002	75.6598240469208\\
1193.50000000001	75.6534852546917\\
1193.80000000001	75.6596029818243\\
1193.9	75.6532663316583\\
1194.09999999999	75.6573438285044\\
1194.2	75.651008959223\\
1194.40000000001	75.6550858099623\\
1194.59999999998	75.642420691387\\
1194.8	75.6464976148632\\
1194.89999999999	75.6401673640167\\
1195.49999999998	75.6523921043827\\
1195.60000000001	75.6460650664883\\
1195.99999999999	75.6542095142547\\
1196.1	75.6478849690687\\
1196.40000000002	75.653990806519\\
1196.50000000002	75.6476683937824\\
1196.70000000001	75.6517379679144\\
1196.8	75.6454173280976\\
1197.00000000001	75.6494862584579\\
1197.09999999999	75.643167390578\\
1197.5	75.6513026052104\\
1197.6	75.6449862235952\\
1198.29999999999	75.6592122830441\\
1198.39999999999	75.6528994576554\\
1198.7	75.658992325659\\
1198.8	75.6526816248228\\
1199.80000000001	75.672972747729\\
1199.90000000001	75.6666666666667\\
1200.70000000001	75.6828780812791\\
1200.8	75.6765759014073\\
1201.29999999998	75.6866988513401\\
1201.4	75.6803995006242\\
1201.59999999999	75.6844470333694\\
1201.7	75.6781494425029\\
1201.90000000001	75.6821963394343\\
1202.09999999999	75.6696057228415\\
1202.2	75.6716293770274\\
1202.29999999999	75.6653359946773\\
1202.69999999999	75.673428666445\\
1202.80000000001	75.6671377504364\\
1202.90000000001	75.6691604322527\\
1203.09999999998	75.6565824468085\\
1203.19999999998	75.6586055015374\\
1203.29999999999	75.6523184311119\\
1203.5	75.6563642406115\\
1203.6	75.6500789233198\\
1203.99999999999	75.658167926252\\
1204.10000000001	75.6518850689254\\
1204.8	75.6660303759648\\
1204.90000000001	75.6597510373444\\
1205.30000000002	75.6678281068525\\
1205.39999999999	75.6615512235587\\
1205.80000000001	75.6696243469608\\
1205.89999999999	75.6633499170813\\
1206.09999999999	75.6673851765876\\
1206.19999999998	75.6611124927464\\
1206.29999999998	75.6631299734748\\
1206.40000000001	75.6568586821384\\
1207.9	75.6870860927152\\
1207.99999999999	75.6808211240791\\
1208.39999999999	75.6888705006206\\
1208.6	75.6763464879623\\
1208.7	75.6783587028458\\
1208.80000000001	75.6720986020349\\
1209.4	75.6841670111616\\
1209.59999999998	75.6716541291229\\
1209.79999999999	75.6756756756757\\
1209.90000000002	75.6694214876033\\
1210.19999999998	75.6754523671817\\
1210.4	75.6629491945477\\
1210.89999999999	75.6729975227085\\
1211.00000000002	75.6667492362315\\
1212.59999999999	75.6988537973118\\
1212.69999999998	75.6926121372032\\
1212.80000000001	75.6946162090857\\
1212.9	75.6883759274526\\
1213.19999999998	75.6943872084398\\
1213.59999999999	75.6694405536788\\
1213.89999999999	75.6754530477759\\
1213.99999999999	75.6692199983527\\
1214.3	75.6752305665349\\
1214.40000000001	75.668999588308\\
1214.59999999999	75.6730056804149\\
1214.69999999998	75.6667764241027\\
1215.1	75.6747860434496\\
1215.20000000002	75.6685592034889\\
1215.39999999999	75.6725627313863\\
1215.49999999999	75.6663376110563\\
1216.29999999999	75.6823413350871\\
1216.40000000002	75.6761200164406\\
1216.49999999998	75.6781193490054\\
1216.7	75.6656804733728\\
1216.99999999998	75.6716785802317\\
1217.19999999998	75.6592458720118\\
1217.8	75.6712373758108\\
1217.9	75.6650246305419\\
1218.4	75.6750102585146\\
1218.50000000001	75.6688002625964\\
1219.1	75.6807742782152\\
1219.19999999999	75.6745673747232\\
1219.59999999999	75.6825448880872\\
1219.70000000001	75.6763403836695\\
1220.30000000002	75.6882989183874\\
1220.49999999998	75.675897099787\\
1220.8	75.6818740273569\\
1220.90000000001	75.6756756756757\\
1221.00000000001	75.677667676685\\
1221.10000000001	75.6714706845726\\
1221.19999999999	75.6734627036764\\
1221.30000000001	75.6672670705748\\
1221.40000000001	75.6692591076545\\
1221.70000000001	75.6506793255852\\
1222.09999999999	75.6586483390607\\
1222.20000000001	75.6524584799149\\
1222.40000000001	75.6564417177914\\
1222.49999999998	75.6502535579912\\
1222.69999999999	75.6542361792607\\
1222.79999999999	75.6480497178837\\
1223.09999999999	75.654022236756\\
1223.20000000001	75.6478378157443\\
1223.3	75.649828347229\\
1223.39999999998	75.6436452799346\\
1224.09999999999	75.6575722921091\\
1224.19999999999	75.6513926325247\\
1224.29999999999	75.6533812479582\\
1224.49999999999	75.6410256410256\\
1225	75.6509672679781\\
1225.09999999999	75.6447926869083\\
1225.59999999999	75.6547279105817\\
1225.69999999999	75.6485560450318\\
1226.29999999999	75.660469667319\\
1226.4	75.6543008560946\\
1226.7	75.6602543201826\\
1226.79999999998	75.6540875376966\\
1227.2	75.6620223254298\\
1227.29999999998	75.6558579110314\\
1227.40000000001	75.6578411405295\\
1227.50000000002	75.6516780710329\\
1227.69999999999	75.6556442417332\\
1227.80000000002	75.6494828569102\\
1227.9	75.6514657980456\\
1228	75.6453057568602\\
1228.50000000001	75.6552173205274\\
1228.59999999999	75.6490599820949\\
1228.7	75.6510416666667\\
1228.8	75.6448856701115\\
1228.9	75.646867371847\\
1229	75.6407127166219\\
1229.2	75.6446758317742\\
1229.30000000001	75.6385228566781\\
1229.4	75.6405042700285\\
1229.60000000001	75.6282020004879\\
1229.80000000001	75.6321652166843\\
1229.9	75.6260162601626\\
1230.09999999998	75.6299788652252\\
1230.19999999999	75.6238315857921\\
1231.00000000001	75.6396718381935\\
1231.19999999999	75.6273856899212\\
1231.30000000002	75.6293649504629\\
1231.39999999999	75.6232237109216\\
1231.60000000002	75.6271819436551\\
1231.90000000002	75.6087662337662\\
1232.20000000002	75.6147042116368\\
1232.4	75.6024340770791\\
1232.5	75.6044134350154\\
1232.60000000002	75.5982801979395\\
1232.90000000002	75.6042173560422\\
1232.99999999999	75.5980861244019\\
1234.20000000001	75.6218099327554\\
1234.7	75.591188856495\\
1235.1	75.5990932642487\\
1235.19999999999	75.5929733667935\\
1235.29999999999	75.5949490043711\\
1235.49999999998	75.5827128520557\\
1235.70000000001	75.5866645088202\\
1235.79999999998	75.5805485880735\\
1235.99999999998	75.5844996359518\\
1236.10000000001	75.5783853745349\\
1236.19999999998	75.5803607538623\\
1236.29999999999	75.5742478162407\\
1236.5	75.5781982856219\\
1236.60000000001	75.5720870057411\\
1236.7	75.5740620957309\\
1236.80000000001	75.5679521384105\\
1237.2	75.5758506425281\\
1237.39999999999	75.5636363636364\\
1237.70000000002	75.5695588948134\\
1237.9	75.5573505654281\\
1238.40000000001	75.5672184093662\\
1238.50000000002	75.5611173906023\\
1238.80000000001	75.5670352732263\\
1238.90000000002	75.5609362389023\\
1239.10000000002	75.5648805681085\\
1239.2	75.5587831840555\\
1239.6	75.5666693554892\\
1239.70000000001	75.5605742861752\\
1240.00000000002	75.5664865736634\\
1240.1	75.5603934849218\\
1240.59999999999	75.5702426049811\\
1240.69999999999	75.5641521598968\\
1241.20000000002	75.5739950052365\\
1241.29999999999	75.5679072015466\\
1242	75.5816761935432\\
1242.10000000001	75.5755916921591\\
1242.29999999999	75.5795235028976\\
1242.50000000001	75.5673587638822\\
1242.60000000001	75.5693248571659\\
1242.70000000002	75.5632442870937\\
1243.19999999998	75.5730716641197\\
1243.30000000001	75.5669937268779\\
1243.69999999999	75.5748512622608\\
1243.80000000001	75.5687756250503\\
1243.90000000001	75.5707395498392\\
1244.00000000001	75.5646652198376\\
1244.4	75.5725190839695\\
1244.5	75.5664470512614\\
1244.99999999998	75.5762589350253\\
1245.19999999999	75.5641210953184\\
1245.89999999999	75.5778491171749\\
1246.09999999999	75.565719788156\\
1247.09999999998	75.585311096857\\
1247.2	75.5792511825543\\
1247.8	75.5909928680183\\
1248.10000000001	75.5728248678096\\
1248.39999999999	75.57869443332\\
1248.49999999999	75.5726413583213\\
1249	75.5824193419262\\
1249.09999999999	75.5763688760807\\
1249.6	75.5861406737617\\
1250	75.5619550435965\\
1250.09999999998	75.5639097744361\\
1250.19999999999	75.5578661121331\\
1251.19999999999	75.5773995045153\\
1251.3	75.5713600767141\\
1252	75.5850171711525\\
1252.09999999998	75.5789809934515\\
1252.70000000001	75.5906768837803\\
1252.80000000002	75.5846436267858\\
1253.00000000002	75.588540419759\\
1253.1	75.5825087775295\\
1253.2	75.5844570334317\\
1253.30000000001	75.5784266794319\\
1254.20000000002	75.5959499322331\\
1254.29999999999	75.5899234693878\\
1254.49999999998	75.593814761677\\
1254.60000000001	75.5877899099386\\
1255.59999999998	75.6072310265191\\
1255.69999999999	75.6012103838191\\
1255.9	75.6050955414013\\
1256.00000000001	75.5990765066475\\
1256.2	75.6029610761761\\
1256.29999999998	75.5969436485196\\
1256.59999999999	75.6027691573168\\
1256.7	75.5967536600891\\
1256.99999999998	75.6025773605918\\
1257.09999999998	75.5965637925549\\
1257.3	75.6004453634484\\
1257.39999999999	75.5944333996024\\
1257.50000000001	75.5963740458015\\
1257.60000000001	75.590363361692\\
1257.8	75.5942443755466\\
1257.89999999999	75.5882352941177\\
1257.99999999999	75.5901756617121\\
1258.09999999998	75.584167858846\\
1258.9	75.5996822875298\\
1258.99999999999	75.5936780239854\\
1259.1	75.5956162642948\\
1259.2	75.5896132772175\\
1259.70000000001	75.5993014764248\\
1259.79999999999	75.5933010556393\\
1260.19999999999	75.6010473696739\\
1260.30000000001	75.5950491907331\\
1262.3	75.6337135614702\\
1262.50000000001	75.621732932045\\
1262.70000000001	75.6255939182768\\
1262.80000000002	75.6196056694909\\
1263	75.6234660755285\\
1263.20000000001	75.611493706958\\
1264.30000000001	75.6327111673521\\
1264.39999999999	75.6267299327798\\
1264.50000000002	75.6286572829353\\
1264.6	75.6226773147782\\
1264.8	75.6265317416397\\
1264.90000000001	75.6205533596838\\
1265.00000000001	75.6224804363291\\
1265.10000000001	75.6165033196333\\
1265.30000000001	75.6203571993046\\
1265.39999999999	75.6143816673252\\
1265.8	75.6220870526898\\
1265.89999999999	75.6161137440758\\
1266.10000000002	75.6199652503554\\
1266.2	75.6139935244413\\
1268.30000000002	75.6543677073478\\
1268.40000000001	75.6484036263303\\
1268.6	75.6522424529045\\
1268.70000000001	75.6462799495586\\
1268.8	75.6481992276775\\
1268.90000000002	75.6422379826635\\
1269.10000000002	75.6460762685156\\
1269.20000000002	75.6401165997006\\
1269.30000000001	75.6420356073736\\
1269.40000000001	75.6360771957464\\
1269.50000000002	75.6379962192817\\
1269.70000000001	75.6260828476926\\
1269.99999999999	75.6318400125974\\
1270.1	75.6258856872933\\
1270.39999999998	75.6316410861865\\
1270.5	75.625688651031\\
1270.8	75.6314422849949\\
1270.89999999999	75.6254917387884\\
1272.39999999999	75.6542239685658\\
1272.49999999998	75.6482791136256\\
1272.70000000002	75.6521055939661\\
1272.80000000001	75.6461623065441\\
1273.7	75.6633694457529\\
1273.8	75.6574299395557\\
1273.99999999999	75.6612510791932\\
1274.2	75.6493761280703\\
1274.3	75.6512868801004\\
1274.49999999999	75.6394162874627\\
1274.60000000002	75.6413273711461\\
1274.70000000001	75.6353937872608\\
1275.50000000001	75.6506741925369\\
1275.59999999999	75.6447440620836\\
1275.69999999998	75.6466530804201\\
1275.79999999999	75.6407241946861\\
1276.3	75.6502663741774\\
1276.4	75.6443399921661\\
1276.60000000001	75.6481554006423\\
1276.70000000002	75.6422305764411\\
1276.90000000001	75.6460454189507\\
1277.00000000001	75.64012215175\\
1277.2	75.6439364284037\\
1277.30000000001	75.6380147173947\\
1277.59999999999	75.6437348360335\\
1277.7	75.6378149945218\\
1278.29999999998	75.6492490613267\\
1278.40000000001	75.6433320297223\\
1280.50000000002	75.683273465563\\
1280.70000000002	75.6714553404122\\
1281.50000000001	75.6866416978777\\
1281.60000000002	75.680736521807\\
1281.79999999999	75.6845307746314\\
1281.89999999999	75.6786271450858\\
1282.3	75.6862133499688\\
1282.39999999998	75.6803118908382\\
1282.60000000002	75.6841038434552\\
1282.90000000002	75.6664068589244\\
1283.80000000001	75.6834644442714\\
1283.90000000002	75.6775700934579\\
1285.1	75.7002801120448\\
1285.20000000001	75.6943904146892\\
1285.3	75.6962813132099\\
1285.39999999999	75.6903928432517\\
1285.49999999999	75.6922837585563\\
1285.6	75.6863965155168\\
1285.90000000002	75.6920684292379\\
1286.00000000002	75.6861830339787\\
1286.19999999999	75.6899634610899\\
1286.30000000001	75.68407960199\\
1286.59999999999	75.6897489702339\\
1286.8	75.677985857487\\
1287.30000000001	75.687432033556\\
1287.4	75.6815533980583\\
1287.60000000002	75.6853304341073\\
1287.69999999999	75.6794533312626\\
1288.4	75.6926658905704\\
1288.50000000001	75.6867918671426\\
1288.59999999999	75.6886785132304\\
1288.8	75.676933819536\\
1289.39999999998	75.6882512601784\\
1289.5	75.682382133995\\
1290.00000000001	75.6918068366793\\
1290.1	75.6859401643156\\
1290.4	75.6915924060442\\
1290.60000000002	75.6798636398853\\
1290.9	75.6855151045701\\
1290.99999999999	75.679653009062\\
1291.30000000001	75.6853027721852\\
1291.40000000002	75.6794425087108\\
1291.59999999999	75.6832081752729\\
1291.70000000001	75.677349434897\\
1292.2	75.6867600402383\\
1292.29999999999	75.6809037449706\\
1292.49999999999	75.6846665635154\\
1292.60000000002	75.6788117892783\\
1292.80000000001	75.6825740583185\\
1292.89999999999	75.676720804331\\
1293.1	75.6804825239715\\
1293.50000000001	75.6570810142239\\
1293.8	75.662725094675\\
1293.90000000002	75.6568778979907\\
1294.09999999998	75.6606397774687\\
1294.29999999999	75.6489493201483\\
1294.80000000002	75.6583519962931\\
1295.09999999998	75.6408276714021\\
1295.30000000001	75.6445885440791\\
1295.40000000001	75.6387495175608\\
1296.4	75.6575395295025\\
1296.79999999999	75.6342046418382\\
1296.89999999998	75.6360832690825\\
1296.99999999999	75.6302521008403\\
1297.39999999999	75.6377649325626\\
1297.5	75.6319358816276\\
1297.8	75.6375683796903\\
1298.20000000002	75.6142648078256\\
1298.69999999999	75.6236526024022\\
1298.9	75.6120092378753\\
1298.99999999998	75.6138865368332\\
1299.10000000002	75.6080665024631\\
1299.3	75.6118208403879\\
1299.39999999999	75.6060023085802\\
1299.60000000002	75.609756097561\\
1299.69999999998	75.6039390675489\\
1299.8	75.6058158319871\\
1299.9	75.6\\
1300.50000000001	75.6112563432262\\
1300.60000000001	75.60544322288\\
1300.70000000001	75.6073185731857\\
1300.80000000002	75.6015066492428\\
1301.69999999999	75.6183745583039\\
1301.79999999998	75.6125662493279\\
1302.40000000001	75.6238003838772\\
1302.49999999999	75.6179947796714\\
1302.80000000002	75.6236088725152\\
1302.90000000001	75.6178050652341\\
1303.6	75.6308966786837\\
1303.69999999999	75.6250958736002\\
1303.99999999998	75.630703166935\\
1304.09999999998	75.6249041558043\\
1304.6	75.6342454204032\\
1304.69999999998	75.6284488044145\\
1305.70000000002	75.6471128809925\\
1305.79999999999	75.6413201623402\\
1306.39999999999	75.6525066972828\\
1306.5	75.6467166692178\\
1306.79999999998	75.6523069859974\\
1306.89999999998	75.6465187452181\\
1307	75.6483819141611\\
1307.30000000002	75.6310234052318\\
1308.49999999998	75.6533700137552\\
1308.6	75.6475892106671\\
1308.70000000001	75.6494498777506\\
1308.8	75.6436702574681\\
1309.00000000002	75.6473913375602\\
1309.2	75.6358359428702\\
1309.80000000001	75.6469959538896\\
1309.90000000002	75.6412213740458\\
1310.50000000002	75.6523729589501\\
1310.60000000002	75.6466010528725\\
1311.4	75.6614563476935\\
1311.50000000002	75.6556877096676\\
1311.99999999998	75.664964560628\\
1312.09999999999	75.6591982929432\\
1312.2	75.6610531128553\\
1312.30000000001	75.6552880219445\\
1312.7	75.6627056672761\\
1312.79999999998	75.6569426460507\\
1313.9	75.6773211567732\\
1314.00000000002	75.6715622859752\\
1314.40000000002	75.6789653860784\\
1314.49999999999	75.6732085805568\\
1314.99999999999	75.6824576077865\\
1315.10000000001	75.676703163017\\
1315.5	75.6840985101855\\
1315.6	75.6783461275367\\
1315.89999999999	75.6838905775076\\
1316.09999999999	75.6723902142532\\
1316.20000000001	75.6742383955025\\
1316.39999999998	75.6627421192556\\
1317.1	75.6756756756757\\
1317.29999999999	75.6641870350691\\
1318.09999999999	75.6789561523289\\
1318.20000000001	75.6732155048168\\
1318.60000000001	75.6805945249109\\
1318.69999999999	75.674855929633\\
1319.8	75.6951284188196\\
1319.9	75.6893939393939\\
1320.10000000001	75.6930768065445\\
1320.20000000001	75.6873437855033\\
1320.29999999998	75.6891850954256\\
1320.49999999998	75.6777222474633\\
1321.29999999999	75.6924474042682\\
1321.39999999999	75.6867196367764\\
1321.69999999999	75.6922378574671\\
1321.79999999998	75.6865118390196\\
1322.29999999999	75.6957047791894\\
1322.4	75.6899810964083\\
1322.69999999999	75.6954944058059\\
1322.79999999999	75.6897724695744\\
1322.89999999999	75.6916099773243\\
1323	75.685889199607\\
1323.40000000002	75.6932376275028\\
1323.60000000001	75.681801012314\\
1323.70000000001	75.6836380117843\\
1323.80000000002	75.677921293149\\
1323.9	75.6797583081571\\
1324.00000000001	75.6740427460162\\
1324.3	75.6795530051344\\
1324.4	75.673839184598\\
1325.20000000001	75.6885233532031\\
1325.3	75.6828127357779\\
1326.2	75.6993138807208\\
1326.29999999998	75.6936067551267\\
1326.79999999998	75.7027658452031\\
1326.90000000001	75.6970610399397\\
1327	75.6988923216035\\
1327.1	75.6931886678722\\
1327.30000000002	75.6968509868917\\
1327.39999999998	75.6911487758945\\
1329.09999999999	75.7222389407162\\
1329.19999999998	75.7165425411871\\
1329.50000000001	75.7220216606498\\
1329.6	75.7163269910506\\
1329.79999999998	75.7199789457854\\
1329.90000000002	75.7142857142857\\
1330.2	75.7197624595956\\
1330.39999999999	75.7083803081548\\
1331.1	75.7211538461538\\
1331.30000000001	75.7097791798107\\
1331.80000000001	75.7188978151513\\
1331.9	75.7132132132132\\
1332	75.715036408678\\
1332.09999999999	75.7093529500075\\
1332.20000000001	75.7111761615252\\
1332.30000000002	75.705493845692\\
1332.60000000001	75.710962707286\\
1332.69999999998	75.7052821128451\\
1333.20000000002	75.7143928598215\\
1333.29999999999	75.7087145642718\\
1333.50000000001	75.7123575284943\\
1333.79999999998	75.6953294849689\\
1334.00000000002	75.698973090473\\
1334.20000000001	75.6876264708087\\
1334.5	75.6930915630151\\
1334.59999999999	75.6874203940961\\
1335.19999999999	75.6983449412117\\
1335.29999999999	75.6926763516549\\
1335.60000000002	75.6981358089391\\
1335.70000000002	75.6924689324749\\
1335.89999999999	75.6961077844311\\
1336.10000000001	75.684777727885\\
1336.2	75.6865973209609\\
1336.50000000001	75.6696094568308\\
1336.80000000001	75.675069189917\\
1336.99999999999	75.6637499065141\\
1337.09999999999	75.6655698474424\\
1337.4	75.6485981308411\\
1337.50000000001	75.6504186602871\\
1337.6	75.6447633998654\\
1338.90000000002	75.6684092606423\\
1339.00000000001	75.6627585691883\\
1339.50000000001	75.6718423409973\\
1339.60000000002	75.6661939240128\\
1339.80000000001	75.6698261064259\\
1340.00000000001	75.6585329453026\\
1340.19999999998	75.6621651868984\\
1340.29999999999	75.6565204416592\\
1340.59999999999	75.6619676288506\\
1340.79999999998	75.6506823775076\\
1340.90000000001	75.6524981357196\\
1341.30000000002	75.6299388698375\\
1341.39999999999	75.6317554975773\\
1341.5	75.6261180679785\\
1341.70000000001	75.629751080638\\
1341.80000000002	75.6241150607348\\
1341.99999999998	75.6277475597944\\
1342.3	75.6108462455304\\
1342.5	75.6144793683897\\
1342.7	75.6032171581769\\
1343.09999999998	75.6104824300179\\
1343.20000000002	75.6048537184546\\
1343.30000000002	75.6066696441864\\
1343.40000000001	75.6010420543357\\
1343.80000000001	75.6083041892998\\
1343.90000000001	75.6026785714286\\
1344.80000000001	75.619005130493\\
1344.9	75.6133828996283\\
1346.30000000001	75.6387403446227\\
1346.4	75.6331229112514\\
1347.99999999998	75.6620428751576\\
1348.09999999998	75.6564307966177\\
1348.19999999999	75.6582362975599\\
1348.29999999998	75.6526253337289\\
1349.29999999999	75.6706684452349\\
1349.59999999999	75.6538490034823\\
1350.10000000001	75.6628647607762\\
1350.20000000002	75.6572613493298\\
1350.50000000002	75.6626684436547\\
1350.7	75.6514657980456\\
1351.10000000002	75.6586737714624\\
1351.29999999998	75.6474766908391\\
1351.70000000001	75.6546826453617\\
1351.79999999999	75.6490864708928\\
1351.99999999998	75.6526884106205\\
1352.1	75.6470936252034\\
1352.30000000002	75.6506950606329\\
1352.4	75.645101663586\\
1353.99999999998	75.6738793294439\\
1354.10000000002	75.6682912420617\\
1354.70000000001	75.6790670209625\\
1354.80000000001	75.6734814377445\\
1355.10000000001	75.6788665879575\\
1355.2	75.673282668044\\
1355.49999999999	75.6786662732369\\
1355.59999999998	75.6730840156377\\
1355.80000000002	75.6766723209676\\
1355.89999999999	75.6710914454277\\
1356.3	75.6782659982306\\
1356.39999999999	75.6726870622927\\
1357.20000000001	75.6870257128122\\
1357.39999999999	75.6758747697974\\
1357.70000000001	75.6812490793931\\
1357.79999999999	75.6756756756757\\
1358.6	75.6899977920071\\
1358.80000000002	75.6788578997719\\
1358.99999999999	75.6824369067765\\
1359.1	75.6768687463214\\
1359.99999999998	75.6929637526652\\
1360.10000000001	75.6873989119247\\
1360.3	75.6909732431638\\
1360.39999999999	75.6854097758177\\
1360.9	75.6943423952976\\
1360.99999999999	75.6887811329072\\
1361.1	75.6905671466353\\
1361.4	75.6738890929122\\
1362.19999999999	75.6881744109227\\
1362.30000000001	75.6826189078098\\
1362.59999999999	75.68797240772\\
1362.70000000002	75.682418550044\\
1363.10000000001	75.6895539906103\\
1363.19999999999	75.6840020538399\\
1363.29999999998	75.6857855361596\\
1363.40000000002	75.6802346901357\\
1363.50000000001	75.6820181871517\\
1363.70000000001	75.6709194896612\\
1363.80000000001	75.6727032773664\\
1363.89999999999	75.6671554252199\\
1364.30000000001	75.6742890647904\\
1364.40000000001	75.6687431293514\\
1365.99999999999	75.6972403191567\\
1366.09999999999	75.6916996047431\\
1366.70000000001	75.702370500439\\
1366.8	75.6968322481528\\
1366.89999999998	75.6986100950988\\
1367.00000000002	75.693072928096\\
1367.19999999998	75.6966283917209\\
1367.30000000001	75.6910925844669\\
1367.49999999998	75.6946475577654\\
1367.59999999998	75.6891131096001\\
1367.90000000001	75.6944444444444\\
1367.99999999999	75.6889116292669\\
1368.20000000001	75.6924651026822\\
1368.29999999999	75.686933645133\\
1368.7	75.6940385739334\\
1368.79999999998	75.6885090218423\\
1369.1	75.6938358165352\\
1369.5	75.6717289719626\\
1369.60000000001	75.6735051471125\\
1369.70000000001	75.6679807271134\\
1370.19999999999	75.6768590819529\\
1370.30000000002	75.6713368359603\\
1370.39999999998	75.6731120029186\\
1370.50000000002	75.6675908361302\\
1370.7	75.6711409395973\\
1370.80000000002	75.6656211248085\\
1370.89999999998	75.6673960612691\\
1370.99999999998	75.6618773247757\\
1371.59999999999	75.6725231464606\\
1371.69999999999	75.6670068523108\\
1372.1	75.6740999854249\\
1372.20000000001	75.6685855862421\\
1372.30000000002	75.6703584960653\\
1372.4	75.6648451730419\\
1373.10000000002	75.6772502184678\\
1373.20000000001	75.6717396053302\\
1373.5	75.6770529994176\\
1373.59999999999	75.6715440052413\\
1373.8	75.6750855229638\\
1374	75.6640710283094\\
1374.49999999999	75.6729230321548\\
1374.59999999998	75.6674183458209\\
1375.39999999999	75.6815703380589\\
1375.49999999999	75.6760686246002\\
1375.9	75.6831395348837\\
1376.00000000001	75.6776397064167\\
1376.20000000002	75.6811741626099\\
1376.29999999999	75.6756756756757\\
1376.50000000001	75.6792096469563\\
1376.60000000002	75.673712500908\\
1377.10000000002	75.6825442927679\\
1377.20000000001	75.6770492993538\\
1378.09999999998	75.6929328109128\\
1378.19999999999	75.6874410505695\\
1378.29999999999	75.6892048752176\\
1378.40000000002	75.683714182082\\
1378.6	75.6872416044099\\
1378.70000000001	75.6817522483319\\
1379.7	75.699376721264\\
1379.8	75.6938908616566\\
1380.9	75.7132512671977\\
1380.99999999998	75.7077691695026\\
1381.20000000001	75.7112864692681\\
1381.39999999998	75.700325732899\\
1381.70000000001	75.705601389492\\
1381.79999999998	75.7001230190318\\
1382.49999999999	75.7124258643136\\
1382.6	75.7069501699573\\
1383	75.7139758513484\\
1383.2	75.7030289886503\\
1383.5	75.7082971957213\\
1383.69999999999	75.6973551091198\\
1383.8	75.6991112074572\\
1383.90000000001	75.6936416184971\\
1385.19999999998	75.7164513101855\\
1385.30000000001	75.710985996824\\
1385.80000000001	75.719748899632\\
1385.90000000002	75.7142857142857\\
1386.10000000001	75.7177896407445\\
1386.49999999999	75.695946920525\\
1387.09999999999	75.7064590542099\\
1387.20000000001	75.7010019462265\\
1387.80000000002	75.711506592694\\
1387.9	75.7060518731988\\
1388.60000000001	75.7182976884856\\
1388.69999999998	75.7128456221198\\
1388.99999999998	75.7180908501908\\
1389.10000000001	75.7126403685574\\
1389.30000000001	75.7161364617821\\
1389.5	75.7052389176742\\
1389.70000000002	75.7087350697942\\
1389.80000000001	75.7032880063314\\
1390.49999999998	75.7155184812311\\
1390.59999999998	75.7100740634213\\
1391.49999999998	75.7257832710549\\
1391.59999999999	75.7203420277359\\
1391.70000000001	75.722086506682\\
1391.80000000001	75.7166463107982\\
1391.90000000001	75.7183908045977\\
1392.00000000002	75.7129516557719\\
1392.2	75.7164404223228\\
1392.39999999998	75.705565529623\\
1393.10000000001	75.7177720356015\\
1393.30000000001	75.7069039758863\\
1393.40000000002	75.7086472909939\\
1393.6	75.6977828801033\\
1393.7	75.6995264743866\\
1393.80000000001	75.6940957027046\\
1393.89999999999	75.6958393113343\\
1394.19999999999	75.6795524636018\\
1394.29999999998	75.6812966150316\\
1394.6	75.6650175665018\\
1395.00000000002	75.6719948390796\\
1395.09999999999	75.6665711009174\\
1395.20000000002	75.6683150576937\\
1395.3	75.6628923606134\\
1395.40000000001	75.6646363310641\\
1395.50000000001	75.6592146746919\\
1395.59999999998	75.6609586587376\\
1395.7	75.6555380426995\\
1396.00000000001	75.6607692858678\\
1396.09999999999	75.6553502363558\\
1396.29999999999	75.6588370094529\\
1396.39999999999	75.6534192624418\\
1397.79999999998	75.6778024179126\\
1397.90000000001	75.6723891273247\\
1399.30000000002	75.6967271687866\\
1399.39999999999	75.6913183279743\\
1400.30000000001	75.706940874036\\
1400.40000000002	75.7015351660121\\
1400.49999999999	75.7032700271312\\
1400.6	75.6978653530378\\
1400.69999999999	75.6996002284409\\
1400.80000000001	75.6941965879078\\
1400.89999999999	75.6959314775161\\
1401.10000000002	75.6851270339709\\
1401.4	75.6903317873707\\
1401.5	75.6849315068493\\
1402.00000000001	75.6936024534627\\
1402.20000000002	75.6828068173715\\
1402.5	75.6880079851704\\
1402.70000000001	75.6772169945823\\
1403.6	75.6928118543848\\
1403.80000000002	75.6820286345181\\
1404.39999999998	75.6924172303311\\
1404.49999999999	75.6870283354692\\
1404.79999999998	75.6922200868389\\
1404.90000000001	75.6868327402135\\
1405.00000000002	75.6885630915949\\
1405.1	75.6831767719898\\
1405.2	75.6849071372661\\
1405.30000000001	75.6795218443148\\
1405.5	75.6829823562891\\
1405.7	75.6722151088348\\
1406.50000000001	75.6860514716337\\
1406.80000000001	75.6699125737437\\
1407.2	75.6768279684502\\
1407.29999999998	75.6714509023732\\
1407.59999999999	75.6766356467997\\
1407.7	75.6712601221764\\
1408.00000000001	75.6764434344152\\
1408.09999999999	75.6710694503622\\
1408.19999999999	75.6727969892779\\
1408.39999999999	75.662051828186\\
1408.80000000001	75.6689616012492\\
1408.99999999999	75.6582215598609\\
1409.20000000001	75.6616760093663\\
1409.30000000001	75.6563076486448\\
1409.40000000002	75.6580347641007\\
1409.5	75.6526674233825\\
1409.69999999998	75.6561214356646\\
1409.89999999999	75.645390070922\\
1409.99999999998	75.6471172257287\\
1410.19999999998	75.6363894206906\\
1410.7	75.6450240998015\\
1410.79999999999	75.6396626266922\\
1411.00000000001	75.6431153001205\\
1411.10000000001	75.6377551020408\\
1411.2	75.6394813292709\\
1411.3	75.6341221482216\\
1411.40000000001	75.6358483882395\\
1411.50000000001	75.6304902238595\\
1411.59999999999	75.6322164765885\\
1411.8	75.6215029393016\\
1412.10000000001	75.62668177312\\
1412.2	75.6213269135453\\
1412.59999999998	75.6282296312027\\
1412.69999999998	75.6228765571914\\
1412.90000000001	75.6263269639066\\
1413.00000000002	75.6209751609936\\
1413.70000000001	75.63304569246\\
1413.8	75.6276964424641\\
1414.70000000001	75.6432004523608\\
1414.80000000002	75.6378542653191\\
1415.29999999999	75.6464603645612\\
1415.4	75.6411162133522\\
1415.70000000002	75.6462777228422\\
1415.80000000001	75.6409350942863\\
1416.00000000001	75.6443753972177\\
1416.10000000001	75.6390340347409\\
1417.89999999998	75.6699576868829\\
1417.99999999999	75.6646216768916\\
1418.79999999998	75.6783423778984\\
1418.9	75.6730091613813\\
1419.09999999998	75.6764374295378\\
1419.2	75.6711054745297\\
1420.1	75.6865230249261\\
1420.19999999999	75.6811941139196\\
1420.3	75.682906223599\\
1420.40000000001	75.6775783174938\\
1420.90000000001	75.6861365235749\\
1421.19999999998	75.6701611201013\\
1421.7	75.6787171191447\\
1421.89999999998	75.6680731364276\\
1422.1	75.6714948671073\\
1422.20000000002	75.6661745060817\\
1422.40000000001	75.6695957820738\\
1422.50000000001	75.6642766765078\\
1423.00000000002	75.6728269271309\\
1423.10000000002	75.6675098369871\\
1423.30000000001	75.6709287621189\\
1423.50000000001	75.6602978364709\\
1424.00000000002	75.6688434800927\\
1424.19999999998	75.6582180720354\\
1424.49999999999	75.6633440965885\\
1424.69999999998	75.6527231892195\\
1425.30000000002	75.6629717973902\\
1425.39999999998	75.6576639775517\\
1425.60000000002	75.6610787683243\\
1425.69999999998	75.6557721980642\\
1426.20000000002	75.6643062469326\\
1426.39999999999	75.6536978618997\\
1426.70000000001	75.6588169329969\\
1426.8	75.6535146120961\\
1426.89999999998	75.6552207428171\\
1427	75.6499194169995\\
1427.49999999998	75.6584477444662\\
1427.59999999999	75.6531484205365\\
1428.50000000002	75.6684866302674\\
1428.59999999998	75.6631903128718\\
1428.80000000001	75.666596682763\\
1429.00000000001	75.6560072773074\\
1429.40000000001	75.6628191675411\\
1429.60000000001	75.6522347345597\\
1430.40000000002	75.6658511010136\\
1430.50000000001	75.6605620019572\\
1430.69999999998	75.6639642158233\\
1430.80000000002	75.6586763575372\\
1431	75.6620781217245\\
1431.39999999998	75.6409360810339\\
1431.90000000001	75.6494413407821\\
1432	75.6441589274492\\
1433.70000000002	75.6730366857302\\
1433.80000000002	75.6677592579678\\
1434.19999999999	75.6745450742523\\
1434.30000000002	75.6692693809258\\
1434.59999999998	75.6743570084338\\
1434.69999999998	75.6690827989964\\
1435.00000000001	75.6741690474531\\
1435.10000000001	75.6688963210702\\
1436.10000000002	75.6858376270714\\
1436.19999999999	75.680568126436\\
1436.50000000001	75.6856466657385\\
1436.60000000001	75.6803786455071\\
1436.79999999999	75.6837636578746\\
1436.9	75.678496868476\\
1437.29999999999	75.6852650619173\\
1437.4	75.68\\
1437.90000000001	75.6884561891516\\
1438	75.6831931020096\\
1438.29999999998	75.6882647385984\\
1438.50000000001	75.6777422494091\\
1438.69999999999	75.6741729218793\\
1439.30000000002	75.6843129081562\\
1439.70000000002	75.6632865675788\\
1439.89999999999	75.6666666666667\\
1440	75.6614124019165\\
1440.39999999999	75.6681707740368\\
1440.6	75.6576664121608\\
1440.89999999999	75.6627342123525\\
1440.99999999998	75.6574838664909\\
1441.19999999999	75.6539235412475\\
1441.39999999998	75.6503642039542\\
1442.5	75.6689310966311\\
1442.59999999998	75.6636861440355\\
1443.1	75.6721175166297\\
1443.2	75.6668745236611\\
1444.00000000002	75.6803545460841\\
1444.1	75.6751142501039\\
1444.30000000001	75.6715591248961\\
1444.50000000001	75.6680049840786\\
1444.69999999998	75.6644518272425\\
1446.00000000002	75.6863287462831\\
1446.10000000001	75.6810952841931\\
1447	75.6962200262594\\
1447.10000000002	75.6909894969596\\
1447.49999999999	75.6977065487704\\
1447.60000000002	75.6924777232852\\
1448.00000000002	75.6991920447483\\
1448.10000000001	75.6939649219721\\
1448.49999999998	75.7006765152561\\
1448.90000000002	75.67977915804\\
1449.09999999999	75.6762351642285\\
1449.3	75.6726921484752\\
1449.8	75.6810814538934\\
1449.9	75.6758620689655\\
1450.1	75.6792166597711\\
1450.30000000002	75.6687810259239\\
1451	75.6805182275515\\
1451.09999999998	75.6753031973539\\
1451.6	75.6836812013501\\
1451.7	75.6784681085549\\
1452.3	75.6885155604517\\
1452.4	75.6833046471601\\
1452.59999999999	75.6797687065464\\
1452.79999999998	75.6831165255695\\
1452.9	75.6779077770131\\
1453.09999999999	75.6743737957611\\
1453.39999999999	75.6793945648435\\
1453.50000000002	75.6741882223445\\
1453.70000000002	75.6706562113083\\
1453.90000000001	75.6740027510316\\
1454	75.6687985695619\\
1454.40000000001	75.6754898590581\\
1454.49999999999	75.6702873642238\\
1454.9	75.6769759450172\\
1455.09999999998	75.6665750412314\\
1455.50000000002	75.6732618851333\\
1455.60000000001	75.668063474617\\
1455.9	75.6593406593407\\
1456.50000000001	75.6693670190855\\
1456.69999999998	75.658978583196\\
1457.29999999999	75.668999588308\\
1457.4	75.663807890223\\
1457.59999999999	75.6602867531042\\
1458.69999999998	75.6786399780642\\
1458.80000000001	75.6734526012749\\
1459.90000000001	75.6917808219178\\
1460.00000000002	75.6865968084378\\
1460.20000000002	75.6830788194207\\
1460.7	75.6914019715225\\
1460.8	75.6862208227805\\
1461.09999999998	75.6912127018889\\
1461.30000000002	75.6808539756398\\
1461.59999999999	75.685845248683\\
1461.69999999999	75.6806676699959\\
1462.49999999999	75.6939696431013\\
1462.70000000002	75.683620453924\\
1463.09999999998	75.6902679059595\\
1463.19999999999	75.6850953324677\\
1463.5	75.6900792566275\\
1463.59999999998	75.6849081095853\\
1464.39999999999	75.698190508706\\
1464.50000000002	75.693021985525\\
1464.70000000002	75.6895139268159\\
1464.90000000001	75.6928327645051\\
1465	75.6876663708962\\
1465.3	75.6926436467859\\
1465.50000000002	75.6823144104803\\
1465.79999999999	75.6736475885122\\
1466.00000000001	75.6769661005388\\
1466.10000000002	75.6718046651207\\
1466.59999999999	75.6800981795868\\
1466.79999999998	75.6697798077578\\
1467.29999999999	75.6780700558812\\
1467.39999999998	75.6729131175468\\
1467.60000000001	75.676228112012\\
1467.7	75.6710723531816\\
1467.90000000002	75.6743869209809\\
1468.00000000002	75.6692323411212\\
1469.1	75.6874489518105\\
1469.19999999999	75.6822976927789\\
1469.39999999999	75.6856073494386\\
1469.50000000001	75.6804572672836\\
1469.69999999999	75.6769628520887\\
1470.29999999999	75.686887921654\\
1470.49999999998	75.6765945872433\\
1470.7	75.6799020940985\\
1470.90000000001	75.6696125084976\\
1471.19999999999	75.6745735064229\\
1471.30000000002	75.6694304743781\\
1471.50000000002	75.6727371568361\\
1471.60000000001	75.6675952979547\\
1472.09999999998	75.675859258253\\
1472.20000000001	75.6707192827549\\
1472.4	75.6740237691002\\
1472.80000000002	75.6534727408514\\
1473.99999999998	75.6732921782783\\
1474.20000000002	75.6630265210608\\
1474.5	75.6679777566798\\
1474.59999999998	75.6628466806808\\
1476.7	75.6974539544962\\
1476.80000000002	75.6923285259666\\
1477.90000000001	75.7104194857916\\
1477.99999999999	75.7052973411813\\
1478.70000000002	75.7167974033\\
1478.8	75.7116775982149\\
1479	75.7149618010953\\
1479.09999999998	75.709843158464\\
1479.30000000002	75.706367446262\\
1479.59999999999	75.697776576333\\
1479.79999999999	75.7010608824921\\
1479.90000000001	75.6959459459459\\
1480.10000000001	75.6924739900013\\
1480.79999999998	75.7039638057938\\
1480.90000000001	75.6988521269412\\
1481.1	75.702133405347\\
1481.19999999998	75.6970228853034\\
1481.4	75.6935538305771\\
1481.79999999999	75.7001147175923\\
1481.90000000001	75.6950067476383\\
1483.10000000002	75.7146709816613\\
1483.29999999999	75.7044627207766\\
1483.89999999999	75.7142857142857\\
1484.00000000002	75.7091840172495\\
1484.39999999999	75.7157292017514\\
1484.60000000002	75.7055297366471\\
1485.19999999999	75.7153437016091\\
1485.3	75.7102463982766\\
1485.5	75.7067851373182\\
1486.89999999999	75.7296570275723\\
1487.10000000002	75.7194728348574\\
1487.3	75.7160145219847\\
1488.30000000001	75.7323300188121\\
1488.49999999999	75.7221550450087\\
1488.80000000002	75.7270468130835\\
1488.99999999998	75.7168759653482\\
1489.50000000001	75.7250268528464\\
1489.70000000001	75.7148610551752\\
1490.49999999998	75.7278948074601\\
1490.59999999999	75.7228147850003\\
1491.30000000001	75.7342094676143\\
1491.40000000002	75.7291317465639\\
1491.6	75.7323858684722\\
1491.70000000002	75.7273092907896\\
1492.29999999999	75.7370678102385\\
1492.4	75.7319932998325\\
1492.60000000001	75.7285455885308\\
1492.9	75.7334226389819\\
1493.00000000001	75.7283504118947\\
1493.20000000001	75.7316011518114\\
1493.3	75.7265300656221\\
1494.29999999998	75.742773019272\\
1494.50000000001	75.7326374949819\\
1494.69999999998	75.7358843992507\\
1494.99999999999	75.7206875794261\\
1495.29999999999	75.725558379029\\
1495.40000000002	75.7204948177867\\
1495.90000000002	75.7286096256685\\
1496	75.7235478911837\\
1496.39999999998	75.7300367524223\\
1496.5	75.7249766136576\\
1496.69999999999	75.7282202031\\
1496.79999999999	75.7231611998129\\
1498.09999999998	75.7442264050194\\
1498.2	75.7391710605353\\
1498.5	75.744027759242\\
1498.6	75.7389737772736\\
1499.2	75.748682718602\\
1499.49999999998	75.7335289410509\\
1499.69999999998	75.7300973463128\\
1499.90000000002	75.7266666666667\\
1500.10000000001	75.7299026796427\\
1500.19999999999	75.7248550289942\\
1500.4	75.7214261912696\\
1500.59999999999	75.7246618244819\\
1500.70000000002	75.7196162046908\\
1501.2	75.7277026576967\\
1501.29999999998	75.7226588517384\\
1501.59999999998	75.7141905840048\\
1501.80000000001	75.7107663626074\\
1502.89999999999	75.7285429141717\\
1502.99999999998	75.7235047568359\\
1503.30000000002	75.728349075429\\
1503.39999999999	75.7233122713668\\
1503.60000000002	75.719890935692\\
1504.5	75.7344144623156\\
1504.60000000001	75.7293812720144\\
1504.89999999998	75.7209302325581\\
1506.19999999999	75.7418840868353\\
1506.3	75.7368560807222\\
1507	75.7481255391149\\
1507.20000000002	75.7380747031115\\
1507.7	75.7461201750895\\
1507.80000000002	75.7410968897142\\
1508.00000000001	75.7443140375307\\
1508.09999999999	75.7392918711046\\
1508.39999999998	75.7441166721909\\
1508.50000000002	75.7390958504574\\
1508.79999999998	75.7439194114918\\
1509.00000000001	75.7338811211981\\
1509.19999999999	75.7304710793083\\
1509.50000000001	75.7352941176471\\
1509.59999999998	75.7302775385838\\
1509.79999999998	75.7334922842572\\
1509.90000000003	75.728476821192\\
1510.40000000002	75.7365110890434\\
1510.49999999998	75.7314974182444\\
1510.79999999999	75.736316102985\\
1511.00000000001	75.726292105089\\
1511.29999999999	75.7311102289268\\
1511.40000000002	75.7260999007608\\
1512.09999999999	75.7373363311731\\
1512.30000000002	75.7273208145993\\
1512.79999999998	75.7353427192809\\
1513.19999999998	75.7153241260821\\
1513.60000000002	75.7217414282883\\
1513.69999999998	75.7167393314837\\
1513.90000000002	75.7199471598415\\
1514.09999999998	75.7099458459913\\
1514.30000000001	75.7131537242472\\
1514.4	75.7081545064378\\
1514.7	75.7129654079747\\
1514.80000000002	75.7079675226088\\
1515.00000000001	75.7045739555145\\
1516.40000000002	75.727002967359\\
1516.59999999999	75.717017208413\\
1517.19999999998	75.7266196533316\\
1517.40000000002	75.7166392092257\\
1518.00000000002	75.7262367432975\\
1518.1	75.7212488473192\\
1518.40000000001	75.7260454395785\\
1518.60000000002	75.7160729571344\\
1518.79999999999	75.7126868128251\\
1519.3	75.7206792154798\\
1519.39999999999	75.715695952616\\
1519.89999999999	75.7236842105263\\
1519.99999999999	75.7187027169265\\
1520.50000000002	75.7266868341444\\
1520.59999999999	75.7217071085684\\
1521.20000000001	75.6984158285677\\
1521.70000000001	75.706400315416\\
1521.79999999998	75.7014258492674\\
1522.00000000002	75.6980487484396\\
1522.79999999998	75.710814892639\\
1522.90000000001	75.7058437294813\\
1523.10000000001	75.702468487395\\
1524.29999999998	75.7215953817896\\
1524.39999999998	75.716628402755\\
1524.7	75.7214060860441\\
1524.90000000002	75.7114754098361\\
1525.09999999999	75.7146603724102\\
1525.2	75.7096964531568\\
1526.3	75.7272012578616\\
1526.4	75.7222404192597\\
1526.60000000002	75.7188707670138\\
1526.79999999999	75.7220512148798\\
1527.3	75.6972633232945\\
1527.8	75.7052163099679\\
1528.00000000001	75.6953078986977\\
1528.30000000002	75.7000785134781\\
1528.40000000001	75.6951259404645\\
1528.69999999998	75.6998953427525\\
1528.90000000002	75.6899934597776\\
1529.1	75.6866335338739\\
1529.3	75.6832744867268\\
1529.70000000002	75.6896326317166\\
1529.79999999998	75.6846852735473\\
1529.99999999998	75.6813280177766\\
1530.50000000001	75.6892721808441\\
1530.60000000001	75.6843274318939\\
1530.79999999998	75.6875040825658\\
1530.89999999998	75.6825604180274\\
1531.3	75.6889121065691\\
1531.4	75.6839699640875\\
1531.6	75.6806163086766\\
1531.89999999998	75.6853785900783\\
1532.09999999998	75.6754992820781\\
1532.40000000001	75.6802610114192\\
1532.49999999999	75.6753229805559\\
1533.80000000002	75.6959384575266\\
1534	75.686070008474\\
1534.30000000002	75.6908237747654\\
1534.40000000001	75.6858911697621\\
1534.6	75.6890597510914\\
1534.7	75.6841282251759\\
1534.9	75.6807817589577\\
1535.10000000002	75.6774361646691\\
1535.3	75.6740914419695\\
1535.49999999998	75.6772597030477\\
1535.70000000001	75.6674046099753\\
1536	75.6721567606276\\
1536.09999999998	75.6672308293191\\
1536.30000000001	75.6703983337673\\
1536.5	75.6605492646102\\
1536.70000000001	75.6637168141593\\
1536.9	75.6538711776187\\
1537.19999999999	75.6456124373902\\
1537.40000000001	75.6422764227642\\
1537.7	75.6340226297308\\
1538.40000000001	75.6451088722782\\
1538.5	75.6401923826856\\
1538.80000000001	75.6449411917604\\
1539.10000000001	75.6301975051975\\
1540.10000000001	75.6460199974029\\
1540.20000000002	75.6411088748945\\
1540.40000000002	75.6442713404739\\
1540.5	75.6393612878099\\
1540.69999999998	75.642523364486\\
1540.80000000001	75.6376143812058\\
1541.10000000002	75.6293797041266\\
1541.30000000002	75.6260542364085\\
1542.49999999998	75.6450149098924\\
1542.59999999999	75.6401114928372\\
1542.79999999998	75.643269168449\\
1543.1	75.6285640228097\\
1543.99999999999	75.6427692506962\\
1544.19999999999	75.6329728679661\\
1544.99999999998	75.6455892822471\\
1545.29999999999	75.6309046201631\\
1546.40000000002	75.6482379566764\\
1546.5	75.6433466959783\\
1546.7	75.6400310318076\\
1546.90000000002	75.6431803490627\\
1547	75.6382909960571\\
1547.30000000001	75.6300891818534\\
1547.49999999998	75.6332385629362\\
1547.60000000002	75.6283517477547\\
1547.8	75.6315007429421\\
1547.9	75.6266149870801\\
1549	75.6439222774514\\
1549.10000000002	75.6390395042603\\
1550.29999999998	75.6578947368421\\
1550.40000000002	75.6530151564012\\
1550.80000000001	75.6592946031337\\
1550.89999999999	75.6544165054803\\
1551.09999999999	75.6511088189789\\
1551.40000000001	75.6429262004512\\
1553.1	75.6695853721349\\
1553.20000000002	75.6647138350608\\
1553.99999999998	75.6772408467924\\
1554.10000000001	75.6723716381418\\
1554.50000000003	75.6593335906342\\
1554.69999999999	75.6624646256753\\
1554.90000000001	75.6527331189711\\
1555.09999999998	75.6494341563786\\
1555.4	75.6541305046609\\
1555.49999999999	75.6492671637953\\
1555.8	75.6539623369111\\
1555.89999999998	75.6491002570694\\
1556.1	75.6522297905154\\
1556.20000000002	75.6473687592367\\
1556.39999999999	75.6440732412464\\
1557.09999999998	75.6550218340611\\
1557.30000000002	75.6453062796969\\
1557.5	75.6484334874165\\
1557.7	75.638721273591\\
1558.00000000002	75.6434118477633\\
1558.09999999999	75.6385573097163\\
1558.50000000001	75.6448094443732\\
1558.90000000002	75.6254008980115\\
1561.00000000001	75.6581897380052\\
1561.19999999999	75.6484980464997\\
1561.4	75.6452129362792\\
1561.60000000001	75.6483319459563\\
1561.80000000002	75.6386452397721\\
1562.4	75.648\\
1562.5	75.6431588378344\\
1563.69999999999	75.661849341348\\
1563.80000000001	75.6570113178592\\
1564.09999999998	75.6616800920598\\
1564.19999999998	75.6568433164994\\
1564.40000000001	75.6599552572707\\
1564.5	75.655119519366\\
1564.89999999998	75.6613418530351\\
1565	75.6565075714012\\
1565.20000000002	75.6596179646074\\
1565.29999999998	75.6547847195605\\
1566.39999999998	75.6718799872327\\
1566.5	75.6670496616877\\
1567.39999999998	75.6810207336523\\
1567.5	75.6761929063537\\
1567.8	75.6808469927929\\
1567.90000000003	75.6760204081633\\
1568.90000000001	75.691523263225\\
1569.10000000001	75.6818761152179\\
1569.70000000001	75.6911708497898\\
1569.90000000002	75.6815286624204\\
1570.19999999998	75.6861746163154\\
1570.30000000002	75.6813550687723\\
1570.50000000001	75.6844518018592\\
1570.59999999999	75.6796332845228\\
1570.80000000002	75.6763638678464\\
1571.79999999998	75.6918379031745\\
1572.09999999999	75.6773947334945\\
1572.59999999998	75.6851274877599\\
1572.8	75.6755038463984\\
1573.39999999999	75.6847791547506\\
1573.50000000002	75.6799694966955\\
1573.70000000002	75.6767060617614\\
1574	75.6813417190776\\
1574.10000000001	75.6765341125651\\
1574.30000000001	75.6796239837398\\
1574.39999999998	75.6748174023499\\
1575.70000000001	75.6948851377078\\
1575.79999999999	75.6900818579859\\
1575.99999999999	75.6868219021636\\
1576.69999999999	75.6976154236428\\
1576.80000000002	75.6928150168051\\
1577.20000000002	75.69897926837\\
1577.29999999998	75.6941802966908\\
1577.6	75.6988020536224\\
1577.7	75.6940043097985\\
1578.20000000002	75.7017043654565\\
1578.50000000002	75.6873178765995\\
1579.2	75.6980940923194\\
1579.30000000002	75.6933012536406\\
1579.6	75.6979173260746\\
1579.69999999998	75.6931257121155\\
1579.89999999999	75.6898734177215\\
1580.19999999999	75.6818325634373\\
1580.79999999998	75.6910620532608\\
1580.90000000002	75.6862745098039\\
1581.10000000002	75.683025550215\\
1581.30000000002	75.6861009232326\\
1581.40000000002	75.6813152070819\\
1581.70000000002	75.6859274244532\\
1581.80000000001	75.6811429293887\\
1582.00000000001	75.6842171796979\\
1582.10000000001	75.6794336999115\\
1582.5	75.6666245418931\\
1582.70000000002	75.6633813495072\\
1583.30000000002	75.6726032588102\\
1583.39999999998	75.6678244395327\\
1583.60000000001	75.6708972658963\\
1583.69999999998	75.6661194595277\\
1584.00000000002	75.6581023925257\\
1584.59999999998	75.6673187354073\\
1584.90000000001	75.6529968454259\\
1585.09999999999	75.6497602826142\\
1585.40000000001	75.6543677073478\\
1585.50000000001	75.6495963673057\\
1586.79999999998	75.6695443947319\\
1587.20000000002	75.6504756504756\\
1588.70000000003	75.6734642497482\\
1588.80000000001	75.6687016174712\\
1589.00000000001	75.6654710213328\\
1589.70000000002	75.6761856837338\\
1589.89999999999	75.6666666666667\\
1590.40000000001	75.6743162527507\\
1590.69999999998	75.6600452602464\\
1591.00000000001	75.6646345295707\\
1591.20000000002	75.655124740778\\
1591.6	75.661242696488\\
1591.70000000001	75.6564895087322\\
1591.90000000001	75.6595477386935\\
1592.00000000003	75.6547955530432\\
1592.60000000003	75.6639668487474\\
1592.79999999998	75.6544666959633\\
1593.50000000001	75.6651606425703\\
1593.69999999999	75.6556657046053\\
1593.99999999998	75.6602471614077\\
1594.09999999998	75.6555011918203\\
1594.49999999999	75.6616079267528\\
1594.60000000002	75.6568633598796\\
1594.99999999998	75.662967839007\\
1595.09999999999	75.6582246740221\\
1595.29999999998	75.6612761689858\\
1595.40000000001	75.6565340018803\\
1595.69999999999	75.6611104148389\\
1595.80000000002	75.6563694467072\\
1596.00000000001	75.6531545642504\\
1596.20000000002	75.6562049740024\\
1596.39999999999	75.6467272157845\\
1596.8	75.6528273529964\\
1596.99999999998	75.6433535783608\\
1597.30000000001	75.6479278828096\\
1597.39999999998	75.6431924882629\\
1598.09999999999	75.6538605931673\\
1598.30000000002	75.6443943943944\\
1598.59999999999	75.6489647838869\\
1598.69999999999	75.6442331748812\\
1599.5	75.6564141035259\\
1599.59999999998	75.6516846908795\\
1599.90000000002	75.65625\\
1600	75.6515217798888\\
1600.20000000001	75.6483159407611\\
1600.99999999999	75.6604834176504\\
1601.09999999999	75.655758181364\\
1601.29999999999	75.6525540152367\\
1601.49999999998	75.6493506493506\\
1602.39999999997	75.6630265210608\\
1602.59999999999	75.653584576028\\
1602.79999999999	75.6566223719508\\
1602.90000000002	75.6519026824704\\
1603.10000000001	75.6487025948104\\
1603.3	75.6517400523887\\
1603.59999999998	75.6375880775706\\
1603.80000000002	75.6406259741879\\
1603.90000000003	75.6359102244389\\
1604.10000000002	75.6389477621244\\
1604.39999999998	75.6248052352758\\
1605.09999999998	75.6354348367805\\
1605.19999999998	75.6307232293029\\
1607.30000000002	75.6625606569615\\
1607.40000000002	75.6578538102644\\
1607.60000000001	75.6546619394166\\
1607.79999999999	75.6576901548604\\
1608.10000000003	75.6435766695685\\
1608.30000000002	75.6466053220592\\
1608.39999999998	75.6419023935343\\
1608.60000000001	75.6387144899608\\
1608.79999999998	75.6355273789546\\
1609.20000000003	75.6415832970857\\
1609.29999999999	75.6368833105505\\
1609.60000000001	75.6289991923961\\
1610.1	75.6365668861011\\
1610.20000000002	75.6318698379184\\
1610.39999999998	75.6348959950326\\
1610.59999999999	75.6255044390638\\
1610.89999999999	75.6300434512725\\
1611	75.6253491403389\\
1611.70000000002	75.635934979526\\
1611.79999999998	75.6312426329177\\
1614.40000000001	75.6704862186435\\
1614.49999999999	75.6657995788431\\
1614.69999999998	75.668813475353\\
1614.79999999998	75.6641278097715\\
1615.2	75.6701541509317\\
1615.29999999998	75.6654698526681\\
1615.49999999999	75.6684822975984\\
1615.60000000003	75.6637989725815\\
1616.00000000001	75.6698224119795\\
1616.30000000001	75.6557782727048\\
1616.50000000001	75.6587900531981\\
1616.6	75.6541102245315\\
1617.7	75.6706638645074\\
1617.80000000001	75.6659867729773\\
1618.39999999999	75.6750077232005\\
1618.50000000002	75.6703323860126\\
1619.30000000002	75.6823514882055\\
1619.49999999999	75.6730056804149\\
1619.79999999998	75.6651645163282\\
1620	75.6619961730757\\
1620.40000000002	75.6680037025609\\
1620.60000000002	75.6586660085148\\
1621.20000000002	75.6676740886943\\
1621.29999999998	75.6630072776613\\
1621.49999999998	75.6598421312284\\
1621.79999999998	75.664344287564\\
1621.90000000001	75.6596794081381\\
1622.09999999998	75.6565158426828\\
1624.59999999998	75.6939742721733\\
1624.69999999998	75.6893156080748\\
1624.9	75.6861538461538\\
1625.09999999997	75.6829928624169\\
1625.30000000001	75.6798326565768\\
1625.59999999998	75.6720182075414\\
1626.39999999999	75.6839840147556\\
1626.7	75.6700270469634\\
1626.99999999999	75.6745129371274\\
1627.1	75.6698623402163\\
1627.59999999999	75.6773361184493\\
1627.8	75.6680385773082\\
1628.10000000002	75.6725218032183\\
1628.20000000001	75.6678744703065\\
1628.50000000002	75.6723566253224\\
1628.59999999998	75.6677104439123\\
1628.79999999998	75.6645589047824\\
1629.40000000001	75.6735194845045\\
1629.5	75.6688757977418\\
1629.70000000001	75.6657258559332\\
1629.90000000003	75.6687116564417\\
1629.99999999997	75.6640696889761\\
1630.29999999999	75.668547595682\\
1630.39999999998	75.6639067770623\\
1630.60000000001	75.6607591831729\\
1632.1	75.6831270677613\\
1632.19999999998	75.6784904735649\\
1633.40000000001	75.6963575145393\\
1633.49999999998	75.6917238001959\\
1633.9	75.6976744186046\\
1634	75.6930420414907\\
1634.20000000003	75.6960166432112\\
1634.39999999998	75.6867543591312\\
1634.60000000002	75.6836116718664\\
1634.90000000001	75.6758409785933\\
1635.70000000002	75.68773688715\\
1635.79999999998	75.6831102145608\\
1636	75.67997066194\\
1636.20000000001	75.6829432255699\\
1636.39999999998	75.6736938588451\\
1637.50000000002	75.690034196385\\
1637.60000000002	75.6854124687061\\
1637.79999999998	75.6883814640698\\
1637.89999999999	75.6837606837607\\
1638.20000000001	75.6882133919307\\
1638.29999999999	75.68359375\\
1639.1	75.6954612005857\\
1639.19999999999	75.6908436527786\\
1639.40000000002	75.6938090881366\\
1639.50000000002	75.6891924859722\\
1639.70000000002	75.6921575801927\\
1640.20000000001	75.6690849234896\\
1640.59999999999	75.656731882733\\
1640.80000000002	75.6536047291121\\
1640.99999999998	75.6504783377003\\
1641.50000000002	75.6578947368421\\
1641.59999999999	75.6532862276908\\
1641.89999999998	75.6455542021924\\
1642.29999999998	75.6514856307842\\
1642.50000000002	75.6422744429563\\
1642.70000000002	75.645239834429\\
1642.8	75.6406354616836\\
1643.09999999998	75.6450827653359\\
1643.2	75.6404795229112\\
1643.99999999998	75.6523325831762\\
1644.19999999998	75.6431308155446\\
1644.5	75.6475738781467\\
1644.69999999999	75.6383754863813\\
1645.19999999998	75.6457788853097\\
1645.40000000001	75.6365846247341\\
1645.70000000003	75.6288734961721\\
1645.90000000001	75.6257594167679\\
1646.09999999998	75.6287206900741\\
1646.20000000003	75.6241268298609\\
1646.50000000002	75.6285679582169\\
1646.59999999998	75.6239752231736\\
1646.8	75.6208634404032\\
1647.00000000002	75.623823690122\\
1647.09999999998	75.6192326372025\\
1647.3	75.616122374651\\
1647.49999999999	75.6190823015295\\
1647.70000000003	75.609904114577\\
1648.10000000003	75.6158233224123\\
1648.2	75.6112358187223\\
1648.59999999999	75.598956753806\\
1649.10000000003	75.6063545961678\\
1649.19999999998	75.6017704480689\\
1649.89999999998	75.6121212121212\\
1650.10000000003	75.602957217307\\
1650.40000000003	75.607391699485\\
1650.49999999999	75.6028110989943\\
1651.70000000001	75.6205351737498\\
1651.90000000001	75.6113801452784\\
1652.10000000002	75.6143324052778\\
1652.29999999999	75.6051803437424\\
1652.50000000002	75.6081326394772\\
1652.60000000001	75.6035578144854\\
1652.8	75.6004597979309\\
1653.00000000001	75.5973625310024\\
1653.2	75.5942660134277\\
1653.7	75.5774579755714\\
1653.90000000002	75.5804111245466\\
1653.99999999998	75.5758418475304\\
1654.29999999998	75.5802707930367\\
1654.39999999999	75.5757026291931\\
1654.60000000003	75.5726113494893\\
1655.8	75.5903134247237\\
1655.9	75.5857487922705\\
1656.60000000001	75.5960644655037\\
1656.7	75.5915016900048\\
1657.00000000001	75.595920584153\\
1657.1	75.591358918658\\
1657.29999999998	75.5882707855678\\
1657.99999999998	75.5985766841566\\
1658.10000000002	75.594017609456\\
1658.3	75.5909310178485\\
1658.59999999999	75.5832881172002\\
1659.70000000002	75.5994698156404\\
1659.8	75.5949153563468\\
1660.2	75.6007950370415\\
1660.40000000001	75.5916892502258\\
1660.60000000003	75.5946287710002\\
1660.70000000002	75.5900770712909\\
1660.9	75.5869957856713\\
1661.19999999998	75.5914043219166\\
1661.40000000003	75.5823051459525\\
1661.69999999997	75.5867132025515\\
1662	75.5730702123819\\
1662.30000000002	75.565447545717\\
1662.90000000002	75.5742633794348\\
1662.99999999999	75.5697191990861\\
1663.19999999999	75.5666446221367\\
1663.49999999999	75.5710507333494\\
1663.59999999998	75.5665083849252\\
1663.79999999999	75.5634353026023\\
1664.10000000002	75.5678404037976\\
1664.20000000002	75.5632998858379\\
1664.40000000001	75.5602282967858\\
1664.90000000002	75.5675675675676\\
1665	75.5630292474926\\
1665.29999999997	75.5674312477483\\
1665.40000000001	75.5628940258181\\
1666.29999999998	75.576092174748\\
1666.39999999998	75.5715571557156\\
1666.6	75.5744885102298\\
1666.70000000001	75.5699544036477\\
1667.29999999999	75.5787453520451\\
1667.40000000001	75.5742128935532\\
1667.60000000002	75.5771421718535\\
1667.7	75.5726106247752\\
1667.90000000003	75.5755395683453\\
1668.29999999999	75.5574202829058\\
1668.8	75.564743244053\\
1668.89999999998	75.5602156980228\\
1669.10000000001	75.5571531272466\\
1669.50000000003	75.563009103977\\
1669.60000000002	75.5584835599209\\
1669.79999999998	75.5614108629259\\
1669.89999999998	75.5568862275449\\
1670.09999999998	75.5538258891151\\
1671.4	75.572838767574\\
1671.49999999999	75.568317779373\\
1671.80000000001	75.5727017166098\\
1672.1	75.5591436431049\\
1672.30000000002	75.56206649127\\
1672.39999999998	75.5575485799701\\
1672.60000000002	75.5604710946374\\
1672.69999999998	75.5559540889527\\
1673.2	75.5632582322357\\
1673.30000000002	75.5587426795745\\
1673.50000000003	75.5616634799235\\
1673.79999999999	75.5481211541908\\
1674.00000000002	75.551042351114\\
1674.10000000001	75.5465296858201\\
1674.3	75.5434782608696\\
1674.69999999999	75.54931932171\\
1674.80000000001	75.5448086452923\\
1675.00000000002	75.5417587009731\\
1675.79999999999	75.5534339757742\\
1675.90000000002	75.5489260143198\\
1676.4	75.5562183119594\\
1676.50000000002	75.5517117976858\\
1678.09999999999	75.5750208556787\\
1678.20000000001	75.5705177858547\\
1678.40000000003	75.5674709562109\\
1679.20000000002	75.5791103435955\\
1679.3	75.5746099797547\\
1679.79999999998	75.581879873802\\
1679.99999999997	75.5728825665139\\
1680.19999999999	75.5757900374933\\
1680.30000000001	75.571292549393\\
1681.29999999999	75.5858213393601\\
1681.40000000002	75.5813261968481\\
1681.90000000002	75.5885850178359\\
1681.99999999998	75.5840913144284\\
1682.20000000002	75.5810497533139\\
1682.79999999998	75.5897557787153\\
1682.89999999997	75.5852644087938\\
1683.1	75.5881653992395\\
1683.19999999999	75.5836749242559\\
1683.39999999999	75.5865755865756\\
1683.50000000001	75.5820860061772\\
1684	75.5893355501455\\
1684.09999999999	75.5848474052963\\
1684.40000000002	75.589195607005\\
1684.50000000002	75.584708536151\\
1684.79999999999	75.5890557303104\\
1684.99999999998	75.580084267996\\
1685.39999999999	75.5858795609611\\
1685.49999999998	75.5813953488372\\
1686.20000000001	75.5915317559153\\
1686.5	75.5780860903593\\
1686.89999999999	75.5838767042086\\
1687.19999999999	75.5704379778344\\
1687.5	75.5747807537331\\
1687.59999999998	75.5703027789299\\
1687.80000000001	75.5731974643048\\
1687.90000000003	75.5687203791469\\
1688.39999999999	75.5759549896358\\
1688.59999999998	75.5670042044176\\
1689.09999999999	75.5742363248875\\
1689.29999999997	75.5652894518764\\
1689.49999999999	75.5622632575758\\
1689.70000000001	75.5592377796189\\
1690.4	75.5693581780538\\
1690.5	75.5648882053709\\
1690.79999999999	75.5692234904489\\
1690.89999999998	75.5647545830869\\
1691.1	75.5617313150426\\
1691.4	75.5660656222288\\
1691.50000000001	75.5615984866399\\
1691.80000000001	75.5659317926591\\
1691.89999999998	75.5614657210402\\
1693.00000000003	75.5773433347115\\
1693.09999999999	75.5728797543114\\
1693.30000000002	75.5698594543522\\
1693.59999999999	75.5741866918581\\
1693.69999999998	75.5697248789704\\
1693.89999999999	75.5667060212515\\
1694.19999999999	75.5710322847194\\
1694.30000000002	75.5665722379603\\
1695.09999999998	75.5781028787164\\
1695.20000000002	75.5736447826343\\
1695.60000000002	75.5794067346818\\
1695.7	75.5749498761646\\
1695.90000000002	75.5719339622642\\
1696.09999999998	75.5748142907676\\
1696.19999999998	75.5703590166834\\
1697.29999999999	75.5861906445151\\
1697.39999999998	75.5817378497791\\
1697.90000000003	75.5889281507656\\
1697.99999999997	75.5844767681526\\
1698.50000000002	75.591663723066\\
1698.60000000001	75.5872137516925\\
1699	75.592960979342\\
1699.20000000002	75.5840640263638\\
1699.49999999999	75.5883737349965\\
1699.69999999999	75.5794799388163\\
1700.30000000003	75.5880969183721\\
1700.40000000001	75.5836518670979\\
1700.59999999998	75.5806432645381\\
1700.79999999999	75.5835146099124\\
1700.90000000002	75.5790711346267\\
1701.3	75.5848125073469\\
1701.50000000001	75.5759285378467\\
1702.09999999997	75.5845376571496\\
1702.19999999999	75.5800975151266\\
1702.40000000002	75.5770925110132\\
1702.90000000002	75.5842630651791\\
1703.00000000001	75.5798250249545\\
1703.40000000002	75.585559142941\\
1703.49999999999	75.5811223291853\\
1703.8	75.585421679676\\
1703.99999999998	75.5765506719089\\
1704.20000000001	75.5794167693481\\
1704.39999999997	75.5705485479613\\
1704.99999999998	75.5791449181866\\
1705.09999999998	75.5747126436782\\
1705.29999999999	75.5775771080098\\
1705.40000000003	75.5731457050718\\
1705.89999999999	75.5803048065651\\
1706.00000000002	75.5758748021804\\
1706.40000000002	75.5815997656021\\
1706.50000000001	75.5771709832415\\
1706.7	75.5741738926646\\
1707.2	75.5813272418438\\
1707.29999999999	75.5769005505447\\
1708.39999999998	75.5926251097454\\
1708.5	75.5882008662063\\
1708.9	75.5939145699239\\
1708.99999999998	75.5894915452577\\
1709.19999999999	75.5864973965951\\
1709.39999999999	75.5893536121673\\
1709.59999999999	75.5805112007955\\
1709.80000000001	75.5833674483888\\
1709.99999999999	75.5745278053915\\
1711.1	75.5902290790089\\
1711.20000000002	75.5858119558231\\
1711.5	75.5900911427904\\
1711.8	75.5768444418482\\
1712.00000000001	75.579697447579\\
1712.10000000002	75.5752832613013\\
1712.29999999999	75.5781359495445\\
1712.4	75.5737226277372\\
1713.69999999998	75.5922511378224\\
1713.79999999998	75.5878405974678\\
1713.99999999999	75.5906889913074\\
1714.09999999998	75.5862793139657\\
1714.39999999998	75.5905511811024\\
1714.7	75.5773268019594\\
1715.09999999997	75.5830223880597\\
1715.20000000002	75.5786159855419\\
1715.90000000001	75.5885780885781\\
1716	75.5841734164676\\
1716.2	75.5870185864942\\
1716.29999999999	75.5826147751107\\
1716.50000000001	75.5854596295002\\
1716.60000000002	75.5810566785111\\
1716.9	75.5736750145603\\
1717.10000000003	75.5765199161426\\
1717.2	75.5721190240494\\
1718.4	75.5891766075065\\
1718.5	75.5847783079251\\
1718.70000000001	75.5876192692576\\
1718.80000000002	75.5832218279132\\
1719.00000000001	75.5802454772846\\
1719.19999999997	75.583086139708\\
1719.5	75.5698999767388\\
1720.19999999999	75.5798407254549\\
1720.3	75.5754475703325\\
1720.50000000002	75.5724747181216\\
1721.19999999999	75.5824086446291\\
1721.50000000001	75.5692379182156\\
1721.7	75.5720757346963\\
1721.8	75.5676868575411\\
1722.00000000001	75.5705243597933\\
1722.10000000003	75.5661363372431\\
1722.29999999999	75.5689735253135\\
1722.39999999999	75.5645863570392\\
1722.59999999998	75.5674232309746\\
1722.79999999998	75.5586511114981\\
1723.00000000002	75.5556845220823\\
1723.3	75.559939654172\\
1723.39999999998	75.5555555555556\\
1724.29999999999	75.5683136163303\\
1724.4	75.5639315743694\\
1724.89999999999	75.5710144927536\\
1725.20000000002	75.5578739929288\\
1726.29999999999	75.5734476367006\\
1726.50000000001	75.5646936175142\\
1727.1	75.573182028717\\
1727.29999999999	75.5644320944773\\
1727.50000000001	75.5614725630933\\
1727.80000000003	75.5657156085422\\
1727.90000000001	75.5613425925926\\
1728.2	75.5655846785859\\
1728.39999999999	75.5568411917848\\
1729.19999999998	75.5681489620078\\
1729.39999999998	75.5594102341717\\
1730.50000000001	75.5749451057437\\
1730.60000000001	75.5705783786907\\
1731.70000000002	75.5860953920776\\
1731.9	75.5773672055427\\
1732.10000000002	75.5744140399492\\
1732.30000000002	75.5772338951743\\
1732.49999999999	75.5685097541268\\
1732.99999999997	75.5755582482257\\
1733.10000000002	75.5711977844449\\
1733.3	75.5740163839852\\
1733.40000000001	75.5696567637727\\
1734.1	75.579517933341\\
1734.20000000001	75.5751600069192\\
1734.40000000002	75.577976362064\\
1734.5	75.5736192782198\\
1734.69999999998	75.5706709707171\\
1734.90000000002	75.5734870317003\\
1735.00000000003	75.5691314621636\\
1735.59999999998	75.5775767701792\\
1735.80000000001	75.5688691744916\\
1736	75.5659236219112\\
1736.29999999998	75.5701451278507\\
1736.4	75.5657932623092\\
1736.70000000001	75.5584983878397\\
1736.89999999997	75.5613126079447\\
1737.09999999998	75.552613400875\\
1737.80000000002	75.5624604407618\\
1737.89999999998	75.5581127733026\\
1738.90000000002	75.5721679125934\\
1738.99999999997	75.5678224368927\\
1739.20000000001	75.5648824239637\\
1739.90000000002	75.5747126436782\\
1740.00000000002	75.5703695189932\\
1740.20000000002	75.5731770384417\\
1740.39999999999	75.5644929617926\\
1741.79999999999	75.584132269361\\
1742	75.5754549107399\\
1742.70000000002	75.5852650906587\\
1742.79999999998	75.5809283378278\\
1743.10000000001	75.5736576411198\\
1743.30000000001	75.5707238728921\\
1743.50000000001	75.5677907777013\\
1743.7	75.570592957908\\
1743.79999999997	75.5662595332301\\
1744.00000000002	75.5690614070294\\
1744.20000000002	75.5603967207476\\
1744.40000000002	75.5574663227286\\
1745.29999999999	75.5700698980176\\
1745.49999999998	75.5614115490376\\
1746.49999999997	75.5754036413604\\
1746.60000000002	75.5710768878457\\
1747	75.5766699101368\\
1747.09999999999	75.5723443223443\\
1747.5	75.5779354543374\\
1747.6	75.5736110316416\\
1747.80000000001	75.5706848217861\\
1747.99999999999	75.5677592815056\\
1748.69999999998	75.577538883806\\
1749.10000000002	75.5602561170821\\
1749.4	75.5644469848528\\
1749.69999999998	75.5514915990399\\
1749.89999999999	75.5542857142857\\
1750.20000000001	75.5413357710107\\
1750.39999999998	75.5441302485004\\
1750.50000000002	75.5398149205986\\
1750.79999999998	75.5440059398024\\
1750.90000000001	75.5396916047973\\
1751.6	75.5494662328024\\
1751.7	75.545153556342\\
1752.09999999999	75.5507362173268\\
1752.29999999998	75.5421136726775\\
1752.59999999998	75.5462999942945\\
1752.69999999998	75.5419899589229\\
1753.50000000001	75.553147810219\\
1753.59999999998	75.5488395962821\\
1754.09999999999	75.5558089157451\\
1754.3	75.547195622435\\
1754.69999999998	75.5527695463871\\
1754.79999999999	75.5484642999601\\
1755.19999999997	75.5540363470632\\
1755.49999999998	75.5411255411255\\
1755.70000000001	75.5439116072446\\
1755.8	75.5396093171593\\
1755.99999999997	75.5367006434713\\
1757.09999999998	75.5520145686319\\
1757.20000000001	75.5477152449781\\
1757.80000000003	75.5560612093976\\
1757.89999999997	75.551763367463\\
1758.50000000001	75.5601046286819\\
1758.70000000001	75.5515123948147\\
1758.89999999998	75.5542922114838\\
1758.99999999999	75.5499971576374\\
1759.20000000001	75.5527766725402\\
1759.49999999998	75.5398954307797\\
1759.80000000001	75.5440650036934\\
1759.90000000002	75.5397727272727\\
1760.39999999997	75.5467196819085\\
1760.49999999999	75.5424287174827\\
1760.80000000001	75.5465954909421\\
1760.89999999999	75.542305508234\\
1761.40000000002	75.5492478001703\\
1761.50000000001	75.5449591280654\\
1761.90000000003	75.5505107832009\\
1762	75.5462232563419\\
1762.19999999998	75.5433240651421\\
1763.29999999999	75.5585800158784\\
1763.40000000001	75.5542954352141\\
1763.89999999998	75.5612244897959\\
1764	75.5569412164843\\
1764.30000000001	75.5610972568579\\
1764.4	75.5568149617455\\
1764.80000000002	75.5623548076378\\
1765	75.5537929862331\\
1765.2	75.5565626239166\\
1765.30000000003	75.5522827687776\\
1765.60000000003	75.5564365407487\\
1765.70000000001	75.5521576622494\\
1765.9	75.5492638731597\\
1766.10000000001	75.5463707394406\\
1766.29999999999	75.5491394927536\\
1766.4	75.5448627228984\\
1766.70000000003	75.5490151686665\\
1767.00000000001	75.5361892366024\\
1767.2	75.53329938324\\
1767.39999999999	75.5360678925035\\
1767.49999999998	75.5317945236479\\
1767.70000000002	75.5289059848399\\
1768.00000000002	75.5330580849499\\
1768.10000000001	75.5287863363873\\
1768.30000000002	75.5315539470708\\
1768.39999999998	75.527283008199\\
1769.09999999999	75.5369658602758\\
1769.20000000001	75.5326965466569\\
1769.70000000001	75.5396089953667\\
1769.80000000002	75.5353409797164\\
1770.49999999999	75.5450129899469\\
1770.60000000002	75.5407465973909\\
1771.19999999998	75.549031784565\\
1771.30000000001	75.5447668510782\\
1771.89999999998	75.5530474040632\\
1772	75.5487839286722\\
1773.30000000002	75.5667080184955\\
1773.40000000002	75.5624471384268\\
1773.6	75.5595647516491\\
1773.80000000002	75.5566830148261\\
1774.59999999999	75.5677015833662\\
1774.69999999997	75.5634437683119\\
1774.99999999998	75.5563066869472\\
1775.4	75.5449169248099\\
1775.59999999999	75.5420397589683\\
1775.9	75.5461711711712\\
1776.1	75.5376646774012\\
1776.40000000003	75.5417956656347\\
1776.49999999998	75.53754362265\\
1777.40000000003	75.549929676512\\
1777.50000000002	75.5456795679568\\
1777.80000000002	75.5385567242252\\
1778.3	75.5454340980657\\
1778.49999999997	75.5369391656359\\
1778.90000000001	75.5424395727937\\
1778.99999999998	75.5381934686077\\
1779.69999999999	75.5478143611642\\
1779.8	75.5435698634755\\
1780.09999999999	75.5476912706438\\
1780.29999999999	75.5392046731072\\
1780.50000000001	75.5363360664944\\
1780.70000000002	75.5390835579515\\
1780.80000000003	75.5348419338537\\
1781.19999999998	75.5403357098748\\
1781.29999999998	75.5360952060177\\
1781.50000000001	75.533228558599\\
1781.69999999997	75.5359748568863\\
1781.8	75.5317357876424\\
1782.30000000002	75.5385996409336\\
1782.50000000002	75.5301245371929\\
1782.79999999998	75.5230242862752\\
1783.1	75.5271422162405\\
1783.2	75.5229069702237\\
1784.40000000001	75.5393667694032\\
1784.49999999999	75.5351339235683\\
1784.79999999999	75.5392458961286\\
1784.9	75.5350140056022\\
1785.20000000003	75.5391250770179\\
1785.30000000001	75.5348941413689\\
1785.49999999999	75.5320340501792\\
1785.80000000001	75.5361442409989\\
1785.9	75.531914893617\\
1786.69999999997	75.5428699350795\\
1786.79999999998	75.5386423414853\\
1787.00000000002	75.5357842314364\\
1787.30000000001	75.5398903435157\\
1787.6	75.527213738323\\
1788.00000000003	75.5326883283933\\
1788.09999999998	75.5284643775864\\
1788.40000000003	75.5325691920604\\
1788.50000000003	75.5283461925528\\
1789.1	75.5365526492287\\
1789.40000000002	75.5238893545683\\
1789.60000000001	75.5266245739509\\
1789.89999999998	75.5139664804469\\
1790.10000000003	75.5167020444643\\
1790.20000000003	75.5124839412389\\
1790.40000000002	75.5152192125105\\
1790.5	75.5110018988049\\
1791.00000000003	75.5178381999888\\
1791.09999999999	75.5136221527468\\
1791.3	75.516355922742\\
1791.60000000002	75.5037115588547\\
1791.79999999998	75.5008650036274\\
1792.09999999999	75.5049659636201\\
1792.19999999999	75.5007532221168\\
1792.79999999998	75.5089519772436\\
1792.89999999999	75.5047406581149\\
1793.1	75.5018960517511\\
1793.49999999997	75.507359500446\\
1793.6	75.5031499135864\\
1793.80000000003	75.5003065945705\\
1794.30000000001	75.5071333036112\\
1794.49999999999	75.4987183773543\\
1795.2	75.5082715980616\\
1795.30000000002	75.5040659463072\\
1795.50000000002	75.5012252171976\\
1795.90000000001	75.5066815144766\\
1795.99999999999	75.5024775903346\\
1796.4	75.5079320901753\\
1796.49999999999	75.5037292663921\\
1796.80000000003	75.5078190216484\\
1796.89999999999	75.5036171396772\\
1797.10000000002	75.5007789895393\\
1797.40000000001	75.5048678720445\\
1797.5	75.5006675567423\\
1797.69999999998	75.5033930359328\\
1797.79999999997	75.4991935035319\\
1798.79999999999	75.5128133859581\\
1798.9	75.508615897721\\
1800	75.5235820232209\\
1800.10000000001	75.5193867348072\\
1800.4	75.5234657039711\\
1800.69999999997	75.5108840515327\\
1801.00000000001	75.5038587529843\\
1801.39999999998	75.5092978073827\\
1801.49999999999	75.5051065719361\\
1801.90000000002	75.4938956714761\\
1802.3	75.4993342210386\\
1802.39999999999	75.495145631068\\
1804.40000000001	75.5223053477418\\
1804.49999999999	75.5181203590823\\
1804.69999999999	75.5208333333333\\
1804.79999999997	75.516649121835\\
1805.50000000002	75.5261408949934\\
1805.60000000002	75.5219582433405\\
1805.99999999998	75.5273794363546\\
1806.09999999997	75.5231978739896\\
1806.29999999998	75.5259078830824\\
1806.40000000002	75.5217270965956\\
1807.29999999998	75.5339161226071\\
1807.40000000001	75.5297372060858\\
1807.59999999998	75.5324445427892\\
1807.69999999998	75.5282664011506\\
1808.4	75.537738457285\\
1808.70000000001	75.5252100840336\\
1808.90000000002	75.5223880597015\\
1809.70000000001	75.5332080892916\\
1809.8	75.5290347533013\\
1810.1	75.5330902662689\\
1810.20000000002	75.5289178589184\\
1810.39999999999	75.5316210991439\\
1810.50000000002	75.527449464266\\
1810.80000000002	75.5315036722072\\
1810.89999999999	75.5273329652126\\
1811.30000000002	75.5327371094181\\
1811.49999999999	75.5243983219254\\
1811.8	75.5174126607429\\
1812.10000000003	75.521465621896\\
1812.2	75.5172984605198\\
1812.59999999998	75.5227009433442\\
1812.79999999999	75.5143692426499\\
1813	75.5170702112404\\
1813.10000000003	75.5129053606883\\
1813.29999999999	75.5100915407522\\
1813.50000000001	75.5072783414204\\
1813.70000000003	75.5044657624876\\
1814.1	75.4933303935619\\
1814.30000000002	75.4960317460317\\
1814.39999999997	75.4918710388537\\
1814.6	75.4945721055822\\
1814.69999999999	75.4904121666299\\
1816.29999999998	75.5120017617265\\
1816.40000000001	75.5078447563997\\
1817.59999999998	75.5240138636739\\
1817.7	75.5198591704258\\
1818.40000000001	75.5292823755843\\
1818.50000000001	75.5251292202793\\
1818.8	75.5291659794381\\
1818.89999999998	75.5250137438153\\
1819.1	75.5222075637643\\
1819.89999999997	75.532967032967\\
1820.09999999998	75.524667618943\\
1821.60000000002	75.5448207718066\\
1821.7	75.5406740586233\\
1822.1	75.5460432444298\\
1822.20000000002	75.5418976019316\\
1822.4	75.5445816186557\\
1822.6	75.5362923136007\\
1822.99999999998	75.5252043223082\\
1823.30000000003	75.5292311067237\\
1823.50000000001	75.5209475762229\\
1823.70000000001	75.5236319771905\\
1823.79999999999	75.5194912001755\\
1824.19999999997	75.5248588499699\\
1824.39999999998	75.5165798849\\
1824.79999999998	75.5219464080224\\
1824.9	75.5178082191781\\
1825.09999999998	75.5204909051063\\
1825.19999999997	75.5163534761409\\
1825.39999999997	75.5135579293344\\
1825.6	75.5107629950156\\
1825.89999999997	75.5147864184009\\
1826.09999999998	75.5065162632789\\
1826.80000000002	75.5159012534895\\
1826.89999999997	75.511767925561\\
1828.10000000003	75.5278415928235\\
1828.5	75.5113201356229\\
1828.69999999997	75.508530183727\\
1829.09999999997	75.5138858517385\\
1829.20000000002	75.5097578308643\\
1829.49999999997	75.5137735024049\\
1829.60000000001	75.5096463901186\\
1830.80000000003	75.5256977442788\\
1830.90000000003	75.5215729109776\\
1831.2	75.5255829192377\\
1831.29999999999	75.52145899312\\
1831.50000000002	75.5241319065298\\
1831.6	75.5200087350549\\
1831.79999999999	75.5226813690704\\
1831.9	75.5185589519651\\
1832.10000000001	75.5157733871848\\
1832.30000000002	75.5129884304737\\
1834.40000000001	75.5410193513219\\
1834.6	75.5327846514417\\
1835.3	75.5421161599651\\
1835.40000000002	75.5380005448107\\
1835.8	75.5433302467455\\
1835.9	75.5392156862745\\
1836.1	75.536433939658\\
1836.99999999999	75.5484187033912\\
1837.09999999998	75.5443065534509\\
1837.5	75.5333043099695\\
1838.20000000003	75.5426208997443\\
1838.30000000001	75.5385117493473\\
1839.29999999998	75.5518103729477\\
1839.50000000002	75.5435964340074\\
1839.69999999999	75.5462550277204\\
1839.8	75.5421490298386\\
1840.10000000002	75.5461362895337\\
1840.19999999998	75.5420311905668\\
1841.00000000002	75.5526587366248\\
1841.30000000001	75.5403497338981\\
1841.5	75.5430060816681\\
1841.60000000001	75.5389042732258\\
1842	75.5442158406167\\
1842.09999999999	75.5401150797959\\
1842.30000000003	75.5427702996092\\
1842.49999999997	75.5345707152936\\
1842.7	75.5372259604949\\
1842.8	75.5331271365782\\
1843.89999999999	75.5477223427332\\
1844.09999999999	75.5395293352131\\
1844.29999999999	75.542181739319\\
1844.40000000002	75.5380862022228\\
1844.59999999999	75.5353173957825\\
1844.80000000001	75.532549189658\\
1844.99999999998	75.529781583654\\
1845.20000000002	75.5324337506097\\
1845.29999999999	75.5283407391351\\
1845.59999999999	75.5214823644146\\
1846.3	75.5307625649913\\
1846.4	75.5266720823179\\
1847.49999999999	75.5412426932236\\
1847.59999999999	75.5371542999405\\
1848.30000000003	75.546418524129\\
1848.5	75.5382451584983\\
1848.70000000001	75.5408913890091\\
1848.90000000003	75.5327203893997\\
1849.20000000002	75.5366895582112\\
1849.3	75.5326051692441\\
1849.60000000002	75.5365734984051\\
1849.70000000001	75.5324899989188\\
1849.90000000001	75.5351351351351\\
1850	75.5310523755473\\
1850.29999999999	75.5350194552529\\
1850.69999999999	75.5186946185433\\
1851.20000000001	75.5253065413493\\
1851.30000000001	75.5212271794318\\
1851.49999999997	75.5238712464895\\
1851.59999999997	75.5197926229951\\
1852.7	75.5343264248705\\
1852.79999999999	75.5302498785687\\
1853.1	75.5342110943233\\
1853.20000000002	75.5301354340905\\
1853.39999999999	75.5273806312382\\
1854.4	75.5405769749259\\
1854.49999999999	75.5365038283188\\
1854.80000000002	75.540460402178\\
1855.00000000003	75.5323163171797\\
1855.60000000001	75.5402274074473\\
1855.70000000001	75.5361569134605\\
1856.4	75.5453810934554\\
1856.49999999999	75.5413120758376\\
1856.70000000003	75.5385609651013\\
1858.40000000001	75.5609362389023\\
1858.49999999999	75.5568707629398\\
1858.69999999999	75.5595007531741\\
1858.99999999999	75.5473078371255\\
1859.30000000002	75.5512530923954\\
1859.49999999999	75.5431275543127\\
1859.7	75.5403806860953\\
1859.90000000003	75.5430107526882\\
1860.00000000003	75.5389495188431\\
1860.2	75.5415793151642\\
1860.29999999998	75.5375188131585\\
1860.69999999999	75.5266552020636\\
1861.40000000001	75.535858178888\\
1861.59999999998	75.5277434602782\\
1861.80000000003	75.5250013427144\\
1862.10000000002	75.528944259478\\
1862.19999999999	75.5248885786393\\
1863	75.535397992593\\
1863.20000000002	75.5272902914185\\
1864.39999999997	75.5430410297667\\
1864.60000000003	75.5349385960208\\
1865.49999999999	75.5467409948542\\
1865.80000000002	75.5345945656252\\
1866.00000000002	75.5318578854295\\
1866.2	75.5291217917805\\
1866.50000000001	75.5330547519554\\
1866.6	75.5290084105641\\
1866.79999999997	75.5262735015266\\
1867.00000000001	75.5288950779283\\
1867.10000000003	75.5248500428449\\
1868.09999999998	75.537950968847\\
1868.19999999999	75.5339078306482\\
1868.80000000003	75.541762534111\\
1868.90000000002	75.53772070626\\
1869.60000000002	75.5468791784778\\
1869.70000000002	75.5428388062895\\
1869.90000000002	75.5454545454545\\
1869.99999999999	75.5414148975991\\
1870.20000000001	75.5440303694594\\
1870.40000000001	75.5359529537557\\
1871.09999999999	75.5451047456178\\
1871.20000000001	75.5410677069417\\
1871.80000000001	75.5489075271115\\
1872.20000000002	75.532767184746\\
1873.80000000003	75.5536581461124\\
1873.89999999997	75.5496264674493\\
1874.59999999999	75.5587560676375\\
1874.69999999997	75.5547258374227\\
1875.2	75.5612435343678\\
1875.29999999999	75.557214460915\\
1875.5	75.554489230113\\
1875.70000000001	75.5570956391939\\
1875.89999999997	75.5490405117271\\
1876.20000000002	75.5529499546981\\
1876.29999999999	75.5489234704754\\
1876.49999999999	75.5462005755089\\
1877.30000000002	75.5566208586343\\
1877.4	75.5525965379494\\
1877.70000000003	75.5565022899137\\
1877.79999999998	75.5524788327387\\
1878.09999999997	75.5563837716963\\
1878.19999999999	75.5523611776607\\
1878.40000000001	75.5496406707479\\
1878.59999999997	75.5522435726832\\
1878.69999999998	75.5482222695338\\
1879.59999999998	75.5599297760281\\
1879.80000000001	75.551891058035\\
1880.09999999999	75.555791937028\\
1880.20000000003	75.5517736531405\\
1881.4	75.5673664629285\\
1881.49999999998	75.5633503401361\\
1881.70000000001	75.5659474970773\\
1881.79999999997	75.5619320899091\\
1882.20000000003	75.5671253253998\\
1882.49999999999	75.5550833953044\\
1882.69999999998	75.5523688124071\\
1882.89999999997	75.5496548061604\\
1883.29999999997	75.5389189763194\\
1883.59999999998	75.5428146732495\\
1883.70000000002	75.5388045440068\\
1884.40000000003	75.5478906871849\\
1884.70000000002	75.5358658743633\\
1885.4	75.5449482895784\\
1885.49999999998	75.5409418752652\\
1885.89999999998	75.5461293743372\\
1886.09999999999	75.5381189693564\\
1886.40000000002	75.5314073681421\\
1887.10000000002	75.5404832556168\\
1887.3	75.5324785419095\\
1887.90000000002	75.5402542372881\\
1888.00000000002	75.5362533764101\\
1888.8	75.5466144316798\\
1889.10000000002	75.5346178276519\\
1889.49999999999	75.5397967823878\\
1889.69999999999	75.5318023071224\\
1890.59999999998	75.5434495160522\\
1890.7	75.5394541992807\\
1891.19999999998	75.5459207952202\\
1891.39999999999	75.5379328575205\\
1891.59999999998	75.5352328593329\\
1892.60000000003	75.5481587150631\\
1892.70000000001	75.5441673710904\\
1894.09999999998	75.5622426354134\\
1894.4	75.5502771179731\\
1895.00000000001	75.5580180465411\\
1895.09999999998	75.5540312368088\\
1895.29999999997	75.5566107417959\\
1895.39999999999	75.5526246372989\\
1895.90000000001	75.5590717299578\\
1896.00000000002	75.5550867570276\\
1896.59999999998	75.5628196341013\\
1896.70000000003	75.558835934205\\
1897.09999999997	75.5639890364748\\
1897.50000000002	75.5480607082631\\
1897.8	75.5519258127404\\
1897.9	75.5479452054795\\
1898.19999999998	75.541273771269\\
1898.7	75.5477143459027\\
1898.79999999998	75.5437358470694\\
1899	75.5410457585172\\
1899.20000000001	75.5436213341758\\
1899.30000000001	75.5396440981363\\
1899.59999999997	75.5435068695057\\
1899.79999999999	75.5355545028686\\
1900.30000000001	75.5419911597558\\
1900.40000000002	75.538016311497\\
1900.89999999997	75.5444502893214\\
1901.00000000003	75.5404765661985\\
1901.19999999999	75.5377899332036\\
1901.40000000001	75.5351038653695\\
1902	75.5428210924767\\
1902.09999999998	75.5388497529177\\
1902.4	75.5321944809461\\
1902.60000000001	75.5347663846113\\
1902.8	75.5268274738557\\
1903.10000000003	75.5306851618327\\
1903.3	75.5227487653672\\
1903.70000000003	75.5121336274819\\
1904.29999999997	75.5198487712665\\
1904.4	75.5158834339722\\
1904.59999999997	75.5184543497664\\
1904.79999999998	75.5105254869022\\
1904.99999999998	75.5130964253845\\
1905.10000000001	75.5091328994331\\
1905.49999999997	75.5142737195634\\
1905.59999999997	75.5103111717479\\
1906.20000000001	75.5180191994964\\
1906.29999999999	75.5140579101972\\
1906.6	75.5179105260398\\
1906.69999999997	75.5139500734214\\
1907.20000000003	75.5203691081634\\
1907.30000000003	75.5164097724651\\
1907.49999999999	75.5189767246802\\
1907.59999999998	75.5150180846045\\
1908.20000000002	75.522716554001\\
1908.29999999998	75.5187591699853\\
1908.49999999998	75.5160850885466\\
1909.40000000003	75.5276250327311\\
1909.5	75.5236698785086\\
1909.70000000001	75.526233113415\\
1909.90000000003	75.5183246073298\\
1911.10000000001	75.5336961071578\\
1911.19999999999	75.5297441531942\\
1911.4	75.532304472927\\
1911.5	75.528353211969\\
1911.70000000001	75.5309132754472\\
1911.79999999999	75.5269627072546\\
1912.29999999998	75.5333612215018\\
1912.40000000002	75.5294117647059\\
1912.59999999999	75.5319705128875\\
1912.7	75.5280217482225\\
1913.30000000003	75.5356956203617\\
1913.49999999997	75.5278010033445\\
1914.29999999999	75.538027580443\\
1914.40000000003	75.5340820057456\\
1914.69999999999	75.527470231878\\
1914.90000000002	75.5300261096606\\
1914.99999999998	75.5260821889196\\
1915.39999999997	75.5311929000261\\
1915.50000000001	75.527249947797\\
1916.50000000002	75.540018783262\\
1916.60000000003	75.5360776334325\\
1916.80000000002	75.5386300798164\\
1916.89999999998	75.5346896191967\\
1917.09999999999	75.5372418109743\\
1917.30000000003	75.5293626786273\\
1917.5	75.531914893617\\
1917.60000000001	75.5279762215154\\
1918.19999999999	75.5356305061773\\
1918.30000000001	75.5316930775646\\
1919.19999999997	75.5431667795551\\
1919.50000000003	75.5313607001459\\
1919.80000000003	75.5351841241731\\
1919.90000000002	75.53125\\
1921.49999999999	75.5516236469609\\
1921.59999999999	75.5476921475777\\
1921.79999999997	75.5502367448879\\
1921.9	75.5463059313215\\
1922.09999999998	75.5488502757257\\
1922.19999999998	75.5449201477397\\
1922.49999999999	75.5487360865495\\
1922.60000000001	75.5448067821293\\
1922.8	75.5473503562328\\
1922.90000000003	75.5434217368695\\
1923.39999999999	75.5289836236028\\
1923.79999999999	75.5340714174333\\
1923.90000000003	75.5301455301455\\
1924.1	75.5274919447043\\
1924.30000000002	75.530035335689\\
1924.40000000001	75.5261106780982\\
1924.60000000002	75.5234582012781\\
1925.50000000003	75.5348982135438\\
1925.6	75.5309757490783\\
1925.8	75.5335167973415\\
1925.90000000003	75.5295950155763\\
1926.5	75.5372158206166\\
1926.60000000002	75.5332952717081\\
1927.19999999998	75.5409121569034\\
1927.29999999999	75.5369928400955\\
1927.59999999997	75.5304248586398\\
1927.80000000001	75.5329633279734\\
1927.89999999997	75.5290456431535\\
1928.09999999999	75.5263976765896\\
1929.50000000002	75.5441542288557\\
1929.60000000002	75.5402394154532\\
1929.89999999998	75.5336787564767\\
1930.10000000003	75.5310330535696\\
1930.30000000002	75.5335681723995\\
1930.5	75.5257432922407\\
1930.89999999999	75.530813050233\\
1930.99999999998	75.5269017658329\\
1931.79999999999	75.537036078472\\
1931.9	75.5331262939958\\
1932.09999999998	75.5356588344892\\
1932.20000000001	75.5317497283031\\
1933.09999999999	75.5431409062694\\
1933.2	75.5392334350592\\
1934.09999999997	75.5506152414435\\
1934.20000000002	75.5467094039187\\
1934.40000000001	75.5492375290773\\
1934.59999999999	75.541427611516\\
1935.99999999997	75.5591136821445\\
1936.10000000002	75.5552112385084\\
1937.2	75.5690910029422\\
1937.30000000001	75.5651904614432\\
1937.50000000002	75.5677126341866\\
1937.60000000003	75.5638127677143\\
1938.20000000001	75.5713769798277\\
1938.40000000001	75.5635800876967\\
1939.40000000003	75.5761794276875\\
1939.69999999998	75.5644911846582\\
1940.60000000003	75.575823156593\\
1940.70000000001	75.5719291014015\\
1941.50000000002	75.5819942315616\\
1941.70000000002	75.5742094963436\\
1943.49999999997	75.5968306235851\\
1943.59999999999	75.5929412975253\\
1943.90000000002	75.59670781893\\
1944	75.5928192994187\\
1944.20000000002	75.5953299387955\\
1944.40000000001	75.587554641296\\
1944.60000000001	75.5849231243894\\
1945.00000000003	75.58994396175\\
1945.10000000002	75.5860579888957\\
1945.40000000002	75.5795425340529\\
1945.69999999999	75.5730290882927\\
1945.89999999999	75.5755395683453\\
1946.09999999999	75.5677730962902\\
1946.39999999997	75.5612638068328\\
1946.60000000002	75.5637745929008\\
1946.79999999998	75.5560121218347\\
1947.10000000001	75.5597781429745\\
1947.19999999997	75.5558979099266\\
1947.50000000002	75.55966317519\\
1947.6	75.5557837449299\\
1947.8	75.5531598131321\\
1948.30000000002	75.5594333812359\\
1948.49999999997	75.5516781278867\\
1948.69999999997	75.5490558292282\\
1948.99999999999	75.5425581037402\\
1949.39999999997	75.5475763016158\\
1949.49999999998	75.5437012720558\\
1949.7	75.5410811365268\\
1949.9	75.5435897435897\\
1949.99999999999	75.5397159120045\\
1950.3	75.5434782608696\\
1950.4	75.5396052294283\\
1950.79999999997	75.5446204315957\\
1950.99999999999	75.5368766336938\\
1952.19999999999	75.5519131281053\\
1952.59999999998	75.5364367286322\\
1952.79999999997	75.5338214962364\\
1953	75.531206799447\\
1953.20000000001	75.5337121793887\\
1953.30000000002	75.529845397768\\
1953.90000000003	75.5373592630501\\
1954.00000000002	75.533493679955\\
1954.30000000001	75.5372492836676\\
1954.39999999998	75.5333844973139\\
1954.7	75.5371393492941\\
1954.89999999998	75.5294117647059\\
1955.10000000002	75.531914893617\\
1955.20000000002	75.5280519613358\\
1956.10000000003	75.539310908905\\
1956.20000000001	75.5354495731738\\
1957.50000000002	75.5516959542297\\
1957.59999999999	75.5478367472034\\
1957.79999999998	75.5503345421114\\
1957.89999999998	75.5464759959142\\
1958.10000000003	75.5438668164641\\
1958.39999999998	75.547612969109\\
1958.5	75.5437557438987\\
1959	75.5499974478077\\
1959.09999999998	75.546141282156\\
1959.59999999998	75.531969178956\\
1959.90000000003	75.5357142857143\\
1960.19999999997	75.5241544661531\\
1960.39999999999	75.5266513644478\\
1960.6	75.5189473147345\\
1961.7	75.5326740748292\\
1961.90000000003	75.5249745158002\\
1962.9	75.5374426897606\\
1963.00000000003	75.5335948245122\\
1963.2	75.5360872001223\\
1963.39999999998	75.528393175452\\
1963.6	75.5257931455925\\
1963.99999999997	75.530777455323\\
1964.1	75.5269320843091\\
1964.29999999999	75.5243331297088\\
1964.49999999999	75.5268247989413\\
1964.59999999997	75.5229806077264\\
1964.89999999998	75.5267175572519\\
1965.09999999998	75.5190311418685\\
1965.80000000002	75.5277481051936\\
1965.90000000003	75.5239064089522\\
1966.2	75.5276407465799\\
1966.30000000003	75.5237998372661\\
1967.09999999999	75.533753558357\\
1967.29999999998	75.5260750228728\\
1967.5	75.5234803821915\\
1967.99999999998	75.529698694172\\
1968.19999999997	75.5220240816949\\
1968.60000000003	75.5269975110479\\
1968.90000000003	75.5154900964957\\
1970.49999999999	75.5353699380899\\
1970.59999999999	75.5315370173035\\
1970.79999999999	75.5340199908671\\
1971.00000000003	75.5263558419157\\
1972	75.5387657826682\\
1972.10000000002	75.5349356049082\\
1973.4	75.5510514314669\\
1973.69999999999	75.5395683453237\\
1973.9	75.5420466058764\\
1974	75.5382199483309\\
1974.19999999997	75.5356328825407\\
1975.80000000003	75.5554430892252\\
1975.90000000003	75.5516194331984\\
1976.99999999999	75.5652217894897\\
1977.10000000002	75.5613999595387\\
1977.8	75.5700490419132\\
1977.90000000001	75.5662285136501\\
1978.50000000002	75.5736379258061\\
1978.60000000001	75.5698185677465\\
1978.90000000001	75.5735219807984\\
1978.99999999998	75.5697034005356\\
1979.80000000003	75.5795747259963\\
1979.89999999999	75.5757575757576\\
1980.49999999997	75.5831566192063\\
1980.59999999999	75.5793406371485\\
1981.20000000003	75.5867359814263\\
1981.30000000001	75.5829211668517\\
1981.49999999997	75.5853855470327\\
1981.60000000003	75.5815713781097\\
1981.80000000002	75.5789898582169\\
1981.99999999999	75.5764088592907\\
1982.19999999999	75.5788730262826\\
1982.30000000001	75.5750605326877\\
1982.59999999999	75.5787562414889\\
1982.69999999997	75.5749445228969\\
1983.7	75.5872567799173\\
1983.79999999998	75.5834467463078\\
1984.00000000003	75.5808678998034\\
1984.6	75.5882501133673\\
1984.70000000001	75.5844417573559\\
1985.10000000002	75.5893612734233\\
1985.20000000002	75.5855538205813\\
1985.49999999998	75.5791700241741\\
1985.70000000003	75.5816295699466\\
1985.80000000001	75.5778236567803\\
1986.00000000001	75.580282966618\\
1986.09999999999	75.5764776961031\\
1986.5	75.5813953488372\\
1986.7	75.5737869941615\\
1986.9	75.5762455963765\\
1987.09999999998	75.5686392914654\\
1987.4	75.5723270440252\\
1987.5	75.5685248540954\\
1987.80000000002	75.5722118818854\\
1987.90000000002	75.5684104627767\\
1988.20000000001	75.5720967660816\\
1988.29999999999	75.5682961174814\\
1989.20000000001	75.5793495199316\\
1989.3	75.5755504172112\\
1989.89999999997	75.5829145728643\\
1990.00000000003	75.5791166273052\\
1990.30000000001	75.5827974276527\\
1990.5	75.5752034562443\\
1990.79999999999	75.5788839218444\\
1990.89999999998	75.5750878955299\\
1991.30000000002	75.5649291955408\\
1991.60000000003	75.5686097303811\\
1991.7	75.5648157445527\\
1991.90000000002	75.5622489959839\\
1992.2	75.56592882598\\
1992.30000000003	75.5621361172455\\
1992.59999999998	75.5658152255733\\
1992.70000000002	75.5620232838218\\
1993.29999999999	75.5693789505368\\
1993.40000000001	75.565588161525\\
1994.50000000002	75.5790634713727\\
1994.60000000003	75.575274477365\\
1995.39999999997	75.5850663993986\\
1995.70000000001	75.5737047800381\\
1995.90000000003	75.5761523046092\\
1996	75.5723661139222\\
1997.30000000001	75.5882647441674\\
1997.39999999998	75.5844806007509\\
1997.59999999999	75.5819192070881\\
1997.9	75.5755755755756\\
1998.1	75.5780202181964\\
1998.20000000002	75.574238102387\\
1998.69999999997	75.5803482089254\\
1998.79999999997	75.5765671119116\\
1999.39999999998	75.5838959739935\\
1999.59999999998	75.5763364504676\\
2000.29999999999	75.5848830233953\\
2000.49999999999	75.5773268019594\\
2000.69999999997	75.5747700919632\\
2001.00000000002	75.5784318624756\\
2001.10000000002	75.5746552068759\\
2001.30000000001	75.5720995303288\\
2001.59999999999	75.575760603487\\
2001.7	75.571985213308\\
2002.09999999997	75.5768654480072\\
2002.20000000002	75.5730909454128\\
2003.00000000003	75.5828465877889\\
2003.1	75.5790734824281\\
2003.49999999998	75.5839488919944\\
2003.60000000001	75.5801766731547\\
2003.80000000001	75.5826139028894\\
2003.99999999998	75.5750711042363\\
2004.2	75.5775083570324\\
2004.3	75.5737377768908\\
2004.99999999998	75.5822652236796\\
2005.09999999998	75.5784959106324\\
2005.49999999998	75.5833665735939\\
2005.60000000003	75.5795981452859\\
2005.79999999999	75.5820330026422\\
2005.90000000002	75.5782652043868\\
2006.10000000001	75.5757152826239\\
2006.29999999998	75.5781499202552\\
2006.4	75.5743832544231\\
2006.6	75.571834354911\\
2007.1	75.5779194898366\\
2007.19999999997	75.5741543366712\\
2007.40000000003	75.5765877957659\\
2007.49999999998	75.572823271568\\
2009.10000000002	75.5922755325503\\
2009.20000000002	75.5885134126313\\
2009.50000000002	75.5921576433121\\
2009.60000000002	75.5883962780514\\
2011.59999999998	75.6126659044589\\
2011.70000000001	75.6089074460682\\
2012.70000000002	75.6210254372019\\
2012.79999999998	75.6172686174177\\
2013.69999999999	75.6281656569669\\
2013.79999999997	75.6244103480808\\
2014.30000000001	75.6304606830818\\
2014.39999999998	75.626706378754\\
2014.6	75.6241624063136\\
2015.00000000001	75.6290010421319\\
2015.10000000001	75.6252481143311\\
2015.30000000002	75.6227051701895\\
2015.50000000001	75.6251240325461\\
2015.60000000002	75.6213722280101\\
2015.89999999999	75.6150793650794\\
2016.20000000002	75.6087883747458\\
2016.79999999998	75.616044424612\\
2016.89999999998	75.612295488349\\
2017.30000000001	75.6022603350848\\
2018.60000000002	75.6179719621539\\
2018.70000000003	75.6142262730335\\
2018.90000000001	75.6166419019316\\
2019	75.6128968352236\\
2019.20000000001	75.6153122369138\\
2019.39999999997	75.6078237187423\\
2019.69999999997	75.6015447073968\\
2019.89999999997	75.6039603960396\\
2020.00000000001	75.6002178109995\\
2020.3	75.5939417937042\\
2020.49999999998	75.591408492527\\
2020.70000000001	75.5938242280285\\
2020.8	75.5900836261072\\
2020.99999999998	75.5924991341349\\
2021.09999999999	75.5887591529784\\
2021.40000000003	75.5923818946327\\
2021.5	75.5886426592798\\
2021.7	75.5910574735384\\
2021.80000000003	75.5873188584994\\
2022.40000000002	75.5945611866502\\
2022.49999999998	75.5908236922773\\
2022.69999999997	75.5882934546174\\
2023.20000000003	75.5943261009242\\
2023.29999999997	75.5905900958782\\
2023.50000000003	75.5930025696778\\
2023.60000000003	75.5892671838711\\
2023.8	75.5867384752211\\
2024.09999999998	75.5903566841221\\
2024.19999999997	75.5866225361853\\
2024.49999999997	75.5902400474168\\
2024.60000000001	75.5865066429594\\
2025.09999999997	75.5925340707091\\
2025.19999999999	75.5888016590135\\
2025.50000000003	75.5825434439178\\
2026.3	75.5921831819976\\
2026.40000000001	75.5884529977794\\
2026.69999999999	75.5920663114269\\
2026.89999999998	75.5846077947706\\
2027.39999999999	75.5906288532676\\
2027.59999999998	75.583173053213\\
2027.79999999999	75.5806499334287\\
2027.99999999997	75.5781273112766\\
2028.20000000002	75.5805354237539\\
2028.69999999999	75.5619085173502\\
2028.99999999997	75.5655216598492\\
2029.10000000002	75.561797752809\\
2029.3	75.5592786045137\\
2029.49999999999	75.5616870319275\\
2029.60000000003	75.5579642311672\\
2029.80000000001	75.5603724321395\\
2029.89999999998	75.5566502463054\\
2031.50000000001	75.5759007678677\\
2031.69999999997	75.5684614627424\\
2032.60000000002	75.5792787917548\\
2032.8	75.571843179694\\
2033.49999999997	75.5802517702597\\
2033.6	75.5765353788661\\
2033.79999999999	75.5789370175525\\
2033.89999999997	75.5752212389381\\
2034.19999999997	75.5788231824215\\
2034.29999999999	75.5751081399921\\
2034.89999999999	75.5823095823096\\
2035.1	75.5748820754717\\
2035.40000000002	75.5784819454679\\
2035.50000000002	75.5747691098448\\
2035.70000000003	75.577168680617\\
2035.90000000002	75.5697445972495\\
2036.09999999999	75.5721441901581\\
2036.19999999999	75.5684329421009\\
2036.4	75.5708323103364\\
2036.49999999997	75.5671216733772\\
2038.10000000003	75.5863016387008\\
2038.20000000003	75.5825933375853\\
2038.39999999999	75.5849889624724\\
2038.49999999998	75.5812812714608\\
2038.70000000001	75.5787718265646\\
2038.89999999998	75.5811672388426\\
2039.09999999998	75.5737544134955\\
2039.30000000003	75.5761498479945\\
2039.39999999999	75.5724442265261\\
2039.7	75.5760368663594\\
2039.80000000003	75.5723319770577\\
2040.49999999998	75.5807115554249\\
2040.59999999998	75.5770078894497\\
2040.80000000002	75.579401244549\\
2040.90000000003	75.5756981871632\\
2042.20000000003	75.5912451647652\\
2042.49999999999	75.5801429550573\\
2042.69999999999	75.5776385353437\\
2042.90000000001	75.5800293685756\\
2042.99999999998	75.5763300866331\\
2043.29999999999	75.5799158265636\\
2043.39999999998	75.5762172742843\\
2043.70000000002	75.579802328995\\
2043.8	75.5761045060913\\
2044	75.5784942028277\\
2044.1	75.5747969865962\\
2044.40000000003	75.5685986793837\\
2045	75.5757664661875\\
2045.10000000003	75.5720711910816\\
2045.3	75.5744597633715\\
2045.39999999999	75.570765094109\\
2045.79999999998	75.5755413265556\\
2045.99999999998	75.5681540491667\\
2046.60000000001	75.5753163629257\\
2046.69999999998	75.5716239984366\\
2046.9	75.5691255495848\\
2047.40000000003	75.5750915750916\\
2047.60000000002	75.5677101137862\\
2047.80000000001	75.5652131451731\\
2048.70000000001	75.5759468957439\\
2048.80000000003	75.5722582849334\\
2048.99999999999	75.5697623346835\\
2049.39999999999	75.5745303732618\\
2049.50000000002	75.5708430913349\\
2049.7	75.568348131525\\
2049.99999999997	75.5719233208136\\
2050.10000000001	75.5682372451468\\
2050.69999999999	75.5753852155256\\
2050.79999999998	75.5717002291677\\
2051.49999999997	75.5800350945603\\
2051.59999999998	75.5763513184189\\
2052.59999999997	75.5882496224485\\
2052.7	75.5845674201091\\
2053.90000000002	75.5988315481986\\
2054.19999999999	75.5877914618118\\
2054.99999999998	75.5972945355457\\
2055.09999999999	75.5936161930712\\
2055.29999999999	75.5959910479712\\
2055.39999999997	75.592313305765\\
2055.90000000001	75.5787937743191\\
2056.69999999998	75.5882924931933\\
2056.79999999999	75.58461762847\\
2057.19999999999	75.5893647013075\\
2057.29999999998	75.5856906775542\\
2057.90000000002	75.5928085519922\\
2057.99999999997	75.5891356105146\\
2058.6	75.5962500607179\\
2058.70000000001	75.5925782008937\\
2059.2	75.598504346137\\
2059.40000000003	75.5911629036174\\
2059.99999999997	75.5982719285471\\
2060.1	75.59460246578\\
2060.39999999999	75.5981557874302\\
2060.49999999998	75.5944870426089\\
2061.60000000002	75.6075083668817\\
2061.70000000002	75.6038413037152\\
2061.89999999998	75.6062075654704\\
2062	75.6025410988798\\
2062.2	75.6000581874606\\
2062.59999999999	75.6047898385611\\
2062.70000000001	75.6011246848943\\
2063.00000000003	75.6046725800979\\
2063.2	75.5973440604856\\
2063.40000000001	75.5948630966804\\
2063.6	75.5972282793042\\
2063.70000000001	75.5935652679523\\
2064.90000000003	75.6077481840194\\
2064.99999999999	75.604086969154\\
2065.39999999998	75.6088114258049\\
2065.59999999998	75.6014910199932\\
2065.9	75.6050338818974\\
2066	75.6013745704467\\
2066.20000000003	75.5988965784252\\
2066.40000000001	75.6012581659811\\
2066.5	75.5975999225782\\
2066.79999999998	75.5914654797039\\
2066.99999999998	75.5889894054473\\
2067.99999999999	75.6007929984043\\
2068.09999999998	75.5971376075815\\
2069.20000000003	75.610109698932\\
2069.29999999998	75.6064559775781\\
2070	75.6147046036423\\
2070.39999999997	75.6000965950254\\
2070.60000000003	75.6024532766697\\
2070.69999999999	75.5988023952096\\
2070.90000000001	75.6011588604539\\
2071.10000000002	75.5938586326767\\
2071.30000000003	75.5913874674133\\
2071.49999999998	75.5937439660166\\
2071.60000000002	75.5900950909881\\
2072.49999999997	75.600694779504\\
2072.60000000003	75.5970473295701\\
2073.00000000001	75.6017558246105\\
2073.3	75.59081701553\\
2073.60000000001	75.5943482663838\\
2073.80000000002	75.5870581995275\\
2074.09999999999	75.590589142802\\
2074.2	75.5869449934918\\
2074.59999999999	75.5916518050802\\
2074.69999999999	75.5880084827453\\
2076	75.6032946389866\\
2076.10000000003	75.5996532125999\\
2076.60000000002	75.6055280011557\\
2076.70000000001	75.6018875192604\\
2077.2	75.6077600731719\\
2077.29999999998	75.6041205352845\\
2077.6	75.6076430668528\\
2077.70000000001	75.6040042352488\\
2078.00000000001	75.6075261055772\\
2078.09999999998	75.6038879799827\\
2078.50000000003	75.6085826998942\\
2078.60000000002	75.6049453985664\\
2078.89999999998	75.6084656084656\\
2078.99999999999	75.6048290125535\\
2079.19999999998	75.6023661809263\\
2079.40000000003	75.6047126713152\\
2079.49999999997	75.6010771302173\\
2079.90000000003	75.6057692307692\\
2079.99999999998	75.6021345127638\\
2080.29999999997	75.6056527590848\\
2080.39999999997	75.6020187454939\\
2080.80000000002	75.6067086356865\\
2080.89999999999	75.6030754444978\\
2081.50000000002	75.6101076095311\\
2081.70000000003	75.602843692958\\
2082.30000000002	75.609873223204\\
2082.40000000003	75.6062424969988\\
2082.70000000002	75.609756097561\\
2082.80000000002	75.6061260742234\\
2083.10000000003	75.6096390168971\\
2083.40000000002	75.598752099832\\
2083.69999999997	75.6022650926193\\
2083.79999999998	75.5986371706896\\
2084.1	75.6021495058056\\
2084.20000000003	75.5985222856595\\
2084.89999999999	75.6067146282974\\
2084.99999999999	75.6030885808834\\
2085.20000000003	75.6006330024457\\
2086.2	75.6123280448641\\
2086.39999999997	75.6050802779775\\
2086.70000000001	75.608587310715\\
2086.79999999997	75.6049643011165\\
2087.20000000003	75.609639246874\\
2087.29999999998	75.6060170547092\\
2087.80000000003	75.6118588054983\\
2087.89999999998	75.6082375478927\\
2088.09999999997	75.6057848865051\\
2088.59999999997	75.6116244554029\\
2088.70000000003	75.6080045959402\\
2089.4	75.6161761186887\\
2089.9	75.5980861244019\\
2090.10000000003	75.6004210123433\\
2090.19999999999	75.5968042864661\\
2090.4	75.5943554173643\\
2090.70000000001	75.5978572795102\\
2090.79999999999	75.5942417140944\\
2091.69999999999	75.6047423271823\\
2091.80000000003	75.601128161002\\
2092.00000000002	75.6034606376368\\
2092.10000000001	75.5998470509512\\
2092.80000000003	75.6080080271394\\
2093.00000000002	75.6007835268262\\
2093.20000000003	75.6031146992786\\
2093.29999999997	75.599503200535\\
2093.50000000002	75.6018341612534\\
2093.79999999999	75.5910024356464\\
2094.5	75.5991597441039\\
2094.59999999998	75.5955506755144\\
2095.39999999998	75.6048675733715\\
2095.5	75.6012597824012\\
2095.7	75.6035881286382\\
2095.79999999997	75.59998091512\\
2096.49999999999	75.6081274444338\\
2096.69999999998	75.6009156810378\\
2096.89999999999	75.5984740104912\\
2097.40000000001	75.6042908224076\\
2097.59999999999	75.5970825189493\\
2097.89999999997	75.6005719733079\\
2098.30000000003	75.5861608844834\\
2099.00000000002	75.5943023200419\\
2099.09999999999	75.5907012195122\\
2099.70000000003	75.5976759691399\\
2099.79999999998	75.5940759083766\\
2100.09999999999	75.5880392343586\\
2100.29999999997	75.59036374024\\
2100.39999999998	75.5867650559391\\
2101.30000000002	75.5972209003521\\
2101.50000000001	75.5900266463647\\
2101.8	75.5839954327037\\
2102.19999999997	75.5886410122247\\
2102.29999999999	75.5850456621005\\
2102.49999999998	75.5826120041853\\
2103.19999999998	75.590738363524\\
2103.30000000003	75.5871446229913\\
2103.89999999998	75.5941064638783\\
2104.09999999997	75.5869213953046\\
2104.30000000001	75.5892415890515\\
2104.40000000001	75.5856497980518\\
2104.60000000001	75.5832185109517\\
2104.8	75.5807876858758\\
2105	75.5783573226925\\
2105.79999999999	75.587634740491\\
2105.89999999999	75.5840455840456\\
2106.19999999998	75.5875231448512\\
2106.30000000003	75.5839346752754\\
2106.59999999997	75.5874115915887\\
2106.69999999997	75.5838238086197\\
2107.00000000001	75.5873000806796\\
2107.09999999997	75.5837129840547\\
2107.50000000003	75.5883469349023\\
2107.60000000001	75.5847606395597\\
2109.99999999999	75.6125302118383\\
2110.09999999997	75.6089470192399\\
2110.30000000002	75.606520090978\\
2110.60000000002	75.6099872080352\\
2110.79999999999	75.6028234402388\\
2111.1	75.6062902614627\\
2111.19999999997	75.6027092312793\\
2112.20000000002	75.6142593381622\\
2112.29999999998	75.6106797954933\\
2112.79999999997	75.6164513228264\\
2112.90000000002	75.6128726928538\\
2113.29999999998	75.6174884073058\\
2113.40000000002	75.6139105748758\\
2113.70000000001	75.617371558331\\
2113.80000000001	75.6137944084394\\
2114.29999999998	75.6195611048051\\
2114.4	75.6159848663987\\
2116.09999999999	75.6355731972403\\
2116.19999999999	75.6319992439635\\
2116.99999999998	75.6412073118889\\
2117.10000000002	75.6376346117514\\
2117.5	75.6422364941443\\
2117.60000000002	75.6386645889408\\
2117.80000000001	75.6362434486992\\
2119.70000000003	75.6580809510331\\
2119.80000000001	75.6545120052833\\
2120	75.6520918824584\\
2120.20000000003	75.6496722161958\\
2120.90000000002	75.6577086280057\\
2121.10000000003	75.6505751461437\\
2122.09999999999	75.6620488172651\\
2122.20000000003	75.6584837204919\\
2122.69999999999	75.664217071792\\
2122.79999999997	75.6606528804937\\
2123.00000000001	75.6582355988884\\
2123.2	75.6558187726652\\
2123.70000000002	75.661550051794\\
2123.80000000001	75.6579876642026\\
2124.00000000002	75.6555717715738\\
2124.19999999998	75.6531563338511\\
2124.59999999999	75.6577399162235\\
2124.69999999999	75.6541792168675\\
2124.99999999997	75.657616112183\\
2125.19999999999	75.6504964005082\\
2125.40000000002	75.6480828040461\\
2125.89999999998	75.653809971778\\
2125.99999999999	75.6502516344481\\
2126.20000000001	75.6525419743216\\
2126.50000000002	75.6418696510863\\
2126.69999999997	75.6441602407372\\
2126.79999999998	75.6406036955193\\
2127.09999999999	75.6440391124483\\
2127.20000000001	75.6404832416678\\
2127.5	75.6345177664975\\
2127.89999999997	75.6390977443609\\
2128.00000000002	75.6355434425074\\
2128.29999999997	75.6389776357828\\
2128.39999999997	75.635424007517\\
2129	75.6422901695552\\
2129.10000000002	75.6387375540109\\
2129.59999999999	75.6444569657698\\
2129.70000000004	75.6409052493192\\
2130.10000000001	75.6454792977185\\
2130.19999999997	75.6419283668967\\
2130.60000000002	75.6465011498569\\
2130.70000000003	75.6429510043176\\
2132.00000000003	75.6578021668777\\
2132.09999999997	75.6542538223431\\
2132.30000000001	75.6518476833615\\
2132.70000000001	75.6423480870218\\
2133.30000000002	75.6491984625481\\
2133.40000000003	75.6456526833841\\
2133.89999999999	75.651358950328\\
2134	75.6478140668197\\
2134.50000000003	75.6535182235548\\
2134.60000000001	75.6499742352555\\
2135.40000000001	75.659096230391\\
2135.60000000001	75.6520110502411\\
2137.10000000003	75.669099756691\\
2137.29999999999	75.6620192757556\\
2137.69999999999	75.6665731125456\\
2137.90000000001	75.6594948550047\\
2138.20000000003	75.6535565636253\\
2138.80000000002	75.6603861798121\\
2138.90000000002	75.6568489948574\\
2139.40000000001	75.6625379761627\\
2139.49999999998	75.6590016825575\\
2139.70000000002	75.6566034208805\\
2140.10000000003	75.6611531632558\\
2140.2	75.6576180909218\\
2140.59999999998	75.6621665810249\\
2140.70000000002	75.6586322869955\\
2140.99999999999	75.6620428751576\\
2141.1	75.6585092471511\\
2141.29999999998	75.6607826655459\\
2141.39999999999	75.6572495914079\\
2141.6	75.6548536209553\\
2141.80000000001	75.6524580979504\\
2142.00000000001	75.6547313384062\\
2142.20000000003	75.6476683937824\\
2142.40000000002	75.6499416569428\\
2142.49999999998	75.6464109026416\\
2142.89999999998	75.6509566028931\\
2142.99999999997	75.6474266249825\\
2143.9	75.6576492537313\\
2144.00000000002	75.6541206100462\\
2144.20000000003	75.6563913631488\\
2144.29999999999	75.6528632717776\\
2144.60000000003	75.6562689420432\\
2144.80000000002	75.6492144155905\\
2145.1	75.6526198023494\\
2145.20000000003	75.6490933668951\\
2146.20000000003	75.6604388948423\\
2146.40000000001	75.6533892382949\\
2146.90000000001	75.6590591523055\\
2147.00000000003	75.6555353732942\\
2147.19999999997	75.6578028221487\\
2147.40000000001	75.65075669383\\
2147.79999999998	75.6552912146748\\
2148.20000000001	75.6412046734627\\
2148.4	75.6434721898999\\
2148.50000000002	75.6399515963883\\
2148.8	75.6433524128624\\
2148.89999999999	75.6398324802234\\
2149.09999999999	75.6374464917179\\
2149.80000000001	75.6453788548305\\
2150.00000000001	75.6383424026789\\
2150.39999999997	75.6428737502906\\
2150.59999999997	75.6358394941182\\
2151.29999999999	75.6437668494934\\
2151.39999999999	75.640250987683\\
2152.19999999997	75.6493053942294\\
2152.3	75.6457907452146\\
2152.50000000001	75.6434079717551\\
2153.6	75.6558480754051\\
2153.70000000003	75.6523354071873\\
2154.70000000003	75.663634676072\\
2154.80000000003	75.6601234396028\\
2154.99999999999	75.6577421001346\\
2155.49999999999	75.6633883837447\\
2155.70000000002	75.6563688653864\\
2156.00000000002	75.659756041\\
2156.09999999999	75.6562471013821\\
2156.50000000002	75.6607623110452\\
2156.60000000003	75.6572541382668\\
2158.10000000001	75.6741729218793\\
2158.19999999998	75.6706667284437\\
2158.49999999997	75.6740479940702\\
2158.79999999999	75.6635323544398\\
2159.00000000002	75.6657866703719\\
2159.09999999997	75.6622823267877\\
2159.29999999998	75.6599055293137\\
2159.50000000001	75.6621596591961\\
2159.60000000002	75.6586562948558\\
2159.9	75.6527777777778\\
2160.2	75.6561588668241\\
2160.29999999999	75.652656915386\\
2160.49999999998	75.6549106729612\\
2160.69999999998	75.6479081821548\\
2160.9	75.6501619620546\\
2161.00000000003	75.6466614224238\\
2161.40000000002	75.6372889197317\\
2161.60000000002	75.634916963501\\
2161.79999999999	75.6371710069846\\
2161.89999999997	75.6336725254394\\
2162.09999999998	75.6313014522246\\
2162.39999999997	75.6346820809248\\
2162.5	75.6311846851013\\
2162.90000000003	75.6218215441516\\
2163.70000000001	75.6308346427581\\
2163.79999999997	75.6273395258561\\
2163.99999999999	75.6295919781896\\
2164.09999999998	75.6260974031975\\
2164.89999999998	75.635103926097\\
2165.00000000004	75.6316105491663\\
2165.80000000003	75.6406112932268\\
2165.90000000004	75.6371191135734\\
2166.20000000002	75.6404930065088\\
2166.5	75.6300193852119\\
2166.90000000002	75.6345177664975\\
2167.19999999998	75.6240483550962\\
2168.89999999998	75.6431535269709\\
2169.00000000002	75.6396662210133\\
2169.69999999997	75.6475251175223\\
2169.90000000001	75.6405529953917\\
2170.09999999998	75.6381900285688\\
2170.89999999998	75.6471672040534\\
2171.00000000003	75.6436829257059\\
2172.49999999997	75.6604989413606\\
2172.60000000003	75.6570166152713\\
2172.90000000004	75.6603773584906\\
2173.09999999997	75.6534143198969\\
2174.00000000001	75.6634929396072\\
2174.09999999999	75.6600128783001\\
2174.89999999998	75.6689655172414\\
2175.00000000002	75.6654866442922\\
2175.39999999999	75.6699609285222\\
2175.6	75.6630050098819\\
2175.80000000003	75.6606461694012\\
2176.10000000001	75.6640014704531\\
2176.20000000001	75.6605247438313\\
2176.99999999998	75.6694685590924\\
2177.20000000003	75.6625177972719\\
2178.39999999999	75.6759238007804\\
2178.49999999998	75.6724501973745\\
2178.69999999999	75.6700936295208\\
2179.10000000002	75.6745594713656\\
2179.19999999997	75.6710870462993\\
2179.59999999997	75.6755516814241\\
2179.69999999998	75.6720800073401\\
2180.09999999998	75.676543436382\\
2180.29999999997	75.6696019079068\\
2180.6	75.6729490530564\\
2180.70000000001	75.6694790902421\\
2180.99999999999	75.6728256384393\\
2181.10000000002	75.6693563176233\\
2181.60000000001	75.6749323921712\\
2181.69999999999	75.6714639288661\\
2182.39999999998	75.6792668957617\\
2182.49999999998	75.6757995051773\\
2184.39999999999	75.6969558251316\\
2184.5	75.693490799231\\
2184.99999999998	75.6990526749348\\
2185.19999999998	75.6921246510776\\
2185.39999999998	75.6943491191947\\
2185.5	75.6908857979502\\
2185.90000000002	75.6953339432754\\
2186.00000000003	75.6918713691048\\
2186.3	75.6952067325283\\
2186.39999999998	75.6917447976218\\
2187.99999999997	75.7095196746035\\
2188.20000000001	75.7026001919298\\
2188.49999999998	75.7059307319748\\
2188.60000000003	75.7024717869055\\
2188.89999999998	75.7058017359525\\
2188.99999999998	75.7023434288064\\
2189.19999999998	75.69999543233\\
2189.80000000004	75.7066532718389\\
2189.9	75.703196347032\\
2190.59999999997	75.7109599671338\\
2190.69999999999	75.7075041080884\\
2191.10000000001	75.7119386637459\\
2191.19999999999	75.7084835485785\\
2191.50000000002	75.7118087242198\\
2191.60000000001	75.7083542455628\\
2192.20000000003	75.7150025087807\\
2192.30000000003	75.7115489874111\\
2192.70000000004	75.7022984312295\\
2192.89999999999	75.7045143638851\\
2192.99999999999	75.7010624230541\\
2193.49999999998	75.7066010211524\\
2193.70000000004	75.6996991521561\\
2194.2	75.7052362940345\\
2194.29999999997	75.7017863652935\\
2194.49999999997	75.7040007290622\\
2194.60000000002	75.7005513281998\\
2194.99999999997	75.6913124686802\\
2195.19999999997	75.6935270805812\\
2195.30000000001	75.6900792566275\\
2195.49999999999	75.6877391145928\\
2195.89999999999	75.6921675774135\\
2196.00000000002	75.6887209143482\\
2196.80000000002	75.6975738540671\\
2196.89999999997	75.6941283568503\\
2197.7	75.7029757029757\\
2197.80000000002	75.699531370854\\
2198.20000000003	75.690306145658\\
2198.59999999999	75.6947287033247\\
2198.69999999998	75.6912861560851\\
2198.90000000001	75.6889495225102\\
2199.09999999997	75.6911604219716\\
2199.3	75.6842775302355\\
2199.70000000001	75.6886989726339\\
2199.80000000002	75.6852584208373\\
2200.69999999997	75.6952017448201\\
2200.79999999999	75.6917624608115\\
2201.50000000001	75.6994912790698\\
2201.60000000001	75.696053049916\\
2202.80000000002	75.7092922965182\\
2202.9	75.7058556513845\\
2203.39999999999	75.71136827774\\
2203.50000000003	75.7079324741332\\
2203.89999999997	75.7123411978221\\
2204.09999999998	75.7054713728337\\
2204.39999999999	75.708777500567\\
2204.60000000001	75.7019095568558\\
2204.90000000004	75.6961451247166\\
2205.09999999997	75.6983493560675\\
2205.80000000003	75.6743279387098\\
2206.09999999998	75.6685703925301\\
2206.40000000002	75.6718785406753\\
2206.49999999999	75.668449197861\\
2206.70000000003	75.6706543411274\\
2206.79999999998	75.6672255199601\\
2208.30000000001	75.6837529433074\\
2208.49999999997	75.6768993932808\\
2208.80000000001	75.6802028158812\\
2209.00000000001	75.6733511384727\\
2209.59999999998	75.6799565551885\\
2209.70000000003	75.6765318128337\\
2209.89999999999	75.6742081447964\\
2210.29999999999	75.6650380021715\\
2210.49999999997	75.6672396634398\\
2210.80000000003	75.6569722737347\\
2210.99999999997	75.6546515309122\\
2211.20000000001	75.6523312078868\\
2211.39999999999	75.6545331223152\\
2211.50000000001	75.6511123168747\\
2211.79999999999	75.6544147565442\\
2211.89999999997	75.6509945750452\\
2212.09999999999	75.6531959135702\\
2212.2	75.6497762509605\\
2212.40000000003	75.6519774011299\\
2212.50000000002	75.6485582572539\\
2213.09999999999	75.6551599493945\\
2213.19999999997	75.6517417430985\\
2213.39999999997	75.6494239891574\\
2214.29999999999	75.6593208092486\\
2214.4	75.655904267329\\
2214.90000000001	75.6613995485327\\
2215	75.6579838382014\\
2215.80000000001	75.666771966244\\
2215.90000000004	75.663357400722\\
2216.40000000002	75.6688472817505\\
2216.70000000003	75.6586070010826\\
2217.70000000003	75.6695824691135\\
2217.80000000001	75.6661707020154\\
2217.99999999999	75.668364816735\\
2218.09999999997	75.6649535659544\\
2218.69999999999	75.6715341626104\\
2218.80000000003	75.6681238451485\\
2219.60000000003	75.6768932738658\\
2219.69999999998	75.6734840976665\\
2220	75.6767713166074\\
2220.09999999999	75.6733627601117\\
2220.39999999997	75.6766494032875\\
2220.50000000004	75.6732414662704\\
2220.80000000004	75.6765275338827\\
2220.90000000002	75.6731202161189\\
2221.29999999999	75.6775006752498\\
2221.59999999999	75.6672818112256\\
2221.79999999997	75.664971420856\\
2222.00000000003	75.6671616938932\\
2222.1	75.6637566375664\\
2222.59999999998	75.6692311153102\\
2222.69999999999	75.6658268850099\\
2223.30000000002	75.6723936313754\\
2223.40000000001	75.6689903305599\\
2223.6	75.671178666187\\
2223.70000000001	75.6677758791258\\
2224.19999999999	75.6732455154431\\
2224.30000000003	75.6698435533177\\
2224.49999999999	75.6675357367617\\
2225.49999999998	75.6784687275341\\
2225.69999999999	75.6716686135322\\
2225.90000000002	75.6738544474394\\
2226.00000000003	75.6704550559274\\
2226.20000000003	75.6726407043076\\
2226.3	75.6692418253683\\
2226.70000000002	75.6736123585414\\
2226.79999999998	75.6702141991109\\
2227	75.6723990840106\\
2227.09999999999	75.6690014367816\\
2227.39999999998	75.672278338945\\
2227.50000000002	75.6688813072365\\
2227.79999999999	75.6631805736344\\
2228.19999999997	75.6675492527936\\
2228.39999999997	75.6607583576397\\
2228.70000000003	75.6640344580043\\
2228.90000000003	75.6572454015254\\
2229.10000000003	75.65942939171\\
2229.20000000001	75.656035526847\\
2229.69999999998	75.6614943044219\\
2229.8	75.6581012601462\\
2229.99999999998	75.6558001883324\\
2230.3	75.6590746054519\\
2230.40000000002	75.6556825823806\\
2230.79999999999	75.6600475144561\\
2230.89999999998	75.6566562079785\\
2231.4	75.6621106878781\\
2231.70000000001	75.6519401380052\\
2232.70000000002	75.6628448584737\\
2232.80000000001	75.6594563124188\\
2233.00000000001	75.6616362903587\\
2233.10000000001	75.6582482536271\\
2233.30000000001	75.6559505686397\\
2233.80000000001	75.6613993464345\\
2234.09999999999	75.6512398173843\\
2234.30000000003	75.6534192624418\\
2234.39999999998	75.6500335645558\\
2234.79999999998	75.6543916953779\\
2235.00000000003	75.6476220303342\\
2235.30000000003	75.6508902209895\\
2235.40000000003	75.6475061507493\\
2236.10000000003	75.6551292370986\\
2236.19999999997	75.6517461878997\\
2236.40000000004	75.6494522691706\\
2236.59999999999	75.6471587606742\\
2236.89999999997	75.6504246759052\\
2237.19999999998	75.6402806954812\\
2238.4	75.6533392897029\\
2238.59999999999	75.6465806048153\\
2239.00000000001	75.6509311777053\\
2239.09999999998	75.6475526973919\\
2239.29999999997	75.6497276056086\\
2239.40000000001	75.6463496316142\\
2239.59999999999	75.6485243559405\\
2239.70000000002	75.645146888115\\
2239.90000000001	75.6428571428571\\
2240.10000000001	75.6405678064459\\
2240.49999999999	75.6449165402124\\
2240.60000000003	75.6415405899942\\
2241.29999999998	75.6491478540198\\
2241.40000000002	75.6457729199197\\
2241.90000000004	75.6512042818912\\
2242	75.6478301592257\\
2242.3	75.6421691045309\\
2242.70000000001	75.6465132869627\\
2242.8	75.6431405769317\\
2243.30000000003	75.6485691361327\\
2243.39999999998	75.6451972364609\\
2243.99999999999	75.6517089256272\\
2244.10000000001	75.6483379377952\\
2244.30000000003	75.6460523970772\\
2244.70000000002	75.6503920171062\\
2244.90000000002	75.6436525612472\\
2245.19999999998	75.6469068721329\\
2245.30000000003	75.6435378997061\\
2245.89999999998	75.6500445235975\\
2246.00000000003	75.646676461422\\
2246.90000000001	75.6564307966177\\
2246.99999999997	75.6530639490899\\
2247.19999999999	75.6507809371246\\
2247.4	75.6484983314794\\
2247.59999999998	75.6462161320461\\
2247.80000000002	75.6439343387161\\
2248.00000000003	75.646101152084\\
2248.10000000001	75.6427364113513\\
2249.39999999997	75.6568126250278\\
2249.50000000002	75.6534495021337\\
2249.79999999997	75.656695853149\\
2250	75.649971112395\\
2250.20000000001	75.6521352708528\\
2250.5	75.6420510086199\\
2250.69999999998	75.6397725253243\\
2250.89999999999	75.6374944469125\\
2251.40000000002	75.6429047301799\\
2251.5	75.6395452122935\\
2251.69999999998	75.6372679634071\\
2251.89999999997	75.6349911190053\\
2252.09999999999	75.6371547819909\\
2252.29999999999	75.630438643225\\
2252.59999999997	75.6336840236161\\
2252.80000000002	75.626969683519\\
2253.09999999998	75.6302148056098\\
2253.20000000004	75.6268583854791\\
2253.40000000003	75.6245839804748\\
2254.09999999998	75.6321533138142\\
2254.2	75.6287982965887\\
2254.39999999999	75.630960301619\\
2254.49999999997	75.627605783731\\
2254.90000000004	75.6319290465632\\
2254.99999999999	75.6285752294798\\
2255.2	75.626302487474\\
2255.40000000004	75.6240301485258\\
2255.59999999999	75.6217582125283\\
2255.80000000003	75.6239194999778\\
2255.89999999998	75.6205673758865\\
2256.19999999998	75.6149448211674\\
2256.40000000003	75.6171061378241\\
2256.50000000002	75.6137552069485\\
2256.79999999998	75.6081350525057\\
2257.10000000002	75.6113769271664\\
2257.39999999997	75.6013289036545\\
2257.7	75.6045708211533\\
2257.79999999997	75.601222374773\\
2258.09999999997	75.6044637321761\\
2258.19999999997	75.6011158836293\\
2258.40000000001	75.598848793447\\
2258.59999999997	75.6010094302032\\
2258.7	75.5976624756508\\
2259.20000000001	75.6030628955871\\
2259.29999999998	75.5997167389572\\
2259.49999999997	75.6018764383077\\
2259.60000000002	75.5985307784219\\
2259.90000000004	75.6017699115044\\
2259.99999999997	75.598424848458\\
2260.20000000003	75.5961598017962\\
2260.99999999999	75.6047941267525\\
2261.09999999998	75.6014505572262\\
2261.30000000002	75.6036083841868\\
2261.40000000002	75.6002653106345\\
2261.90000000003	75.605658709107\\
2262.00000000001	75.6023164316343\\
2262.39999999998	75.6066298342542\\
2262.50000000002	75.6032882524529\\
2262.80000000003	75.5976843872906\\
2262.99999999999	75.5954222084751\\
2263.39999999998	75.5997349237906\\
2263.59999999999	75.5930556169104\\
2264.39999999998	75.6016780746302\\
2264.50000000001	75.5983396626336\\
2264.69999999997	75.6004945249029\\
2264.90000000002	75.5938189845475\\
2265.70000000003	75.6024362256157\\
2265.79999999998	75.5990996954852\\
2265.99999999999	75.5968403865672\\
2266.19999999998	75.5945814764153\\
2266.40000000004	75.5923229649239\\
2266.6	75.5944765518154\\
2266.69999999998	75.5911416975472\\
2267.00000000002	75.5943716642407\\
2267.10000000003	75.591037402964\\
2267.39999999999	75.5942668136714\\
2267.50000000002	75.5909331451755\\
2267.70000000001	75.5930858100362\\
2267.89999999999	75.5864197530864\\
2268.79999999999	75.5961038388647\\
2268.89999999997	75.59277214632\\
2269.5	75.5992245329574\\
2269.59999999999	75.595893730449\\
2269.9	75.5991189427313\\
2269.99999999997	75.5957887317739\\
2270.20000000001	75.593533894199\\
2270.40000000003	75.5956837700947\\
2270.59999999998	75.5890254106663\\
2270.99999999999	75.5933248205715\\
2271.09999999999	75.589996477633\\
2271.70000000004	75.5964433488863\\
2271.8	75.5931158941855\\
2271.99999999998	75.5952642929448\\
2272.10000000003	75.5919373294604\\
2272.29999999997	75.5896849146277\\
2272.69999999997	75.5939809926082\\
2272.8	75.5906551102116\\
2273.4	75.5970969870244\\
2273.5	75.5937719915552\\
2273.90000000001	75.5980650835532\\
2274.09999999997	75.5914167619383\\
2274.49999999998	75.5957091356722\\
2274.70000000001	75.5890627747494\\
};
\addplot [color=mycolor2, forget plot]
  table[row sep=crcr]{%
2274.70000000001	75.5890627747494\\
2275.00000000002	75.5922816579491\\
2275.20000000002	75.5856370588494\\
2275.69999999998	75.591000966693\\
2275.79999999998	75.5876795992794\\
2276.00000000001	75.589824700145\\
2276.10000000003	75.5865038221597\\
2276.3	75.5842558425584\\
2276.60000000001	75.5874730970264\\
2276.70000000001	75.5841531974701\\
2276.89999999999	75.5819060166886\\
2277.49999999998	75.5883386020372\\
2277.60000000002	75.5850199762919\\
2277.90000000004	75.5882352941177\\
2277.99999999999	75.5849172556077\\
2278.40000000003	75.5892034233048\\
2278.50000000003	75.5858860703941\\
2278.69999999999	75.5836405125505\\
2278.89999999998	75.5857832382624\\
2279.00000000001	75.582466763196\\
2279.29999999998	75.5856804422216\\
2279.39999999998	75.5823645536302\\
2280.10000000003	75.5898605385492\\
2280.30000000003	75.5832310121031\\
2280.49999999997	75.5809874594405\\
2280.79999999997	75.5841992196063\\
2281.10000000004	75.5742591618447\\
2281.30000000003	75.5764004558604\\
2281.40000000001	75.5730878807802\\
2281.60000000003	75.5708462988123\\
2282.59999999999	75.5815481666448\\
2282.89999999997	75.5716162943495\\
2283.09999999997	75.5693763139453\\
2283.30000000004	75.5715161601121\\
2283.50000000002	75.5648975302154\\
2283.99999999997	75.5702464865812\\
2284.19999999997	75.5636299960601\\
2284.7	75.5689775910364\\
2284.80000000001	75.5656702700337\\
2284.99999999997	75.5634326725307\\
2286.69999999997	75.5815987405982\\
2286.79999999997	75.5782937601119\\
2287.00000000001	75.5804293646977\\
2287.09999999998	75.5771248688353\\
2287.60000000003	75.5824627354985\\
2287.70000000003	75.5791590173966\\
2287.90000000002	75.5769230769231\\
2288.30000000001	75.5811920992833\\
2288.39999999999	75.5778894472362\\
2289.3	75.5874901720975\\
2289.5	75.5808874912648\\
2289.69999999998	75.5830203511224\\
2289.80000000002	75.5797196384122\\
2290.59999999998	75.5882481337582\\
2290.79999999997	75.5816491335283\\
2291.7	75.5912383279518\\
2291.79999999998	75.5879401370042\\
2292.69999999998	75.597522679693\\
2292.89999999999	75.5909289140863\\
2293.10000000003	75.5930577359149\\
2293.40000000002	75.5831698277741\\
2293.59999999999	75.5809390940402\\
2293.8	75.5787087492916\\
2294.00000000003	75.5764787934266\\
2294.20000000003	75.5742492263436\\
2294.40000000004	75.5720200479407\\
2294.80000000001	75.5762778334568\\
2294.90000000003	75.5729847494553\\
2295.49999999998	75.5793692280885\\
2295.6	75.5760770135471\\
2295.79999999998	75.578204625637\\
2295.90000000002	75.5749128919861\\
2296.50000000004	75.5812940869111\\
2296.60000000002	75.5780032220142\\
2297.2	75.5843816654333\\
2297.29999999998	75.581091668843\\
2297.50000000001	75.583217270195\\
2297.59999999999	75.5799277538408\\
2298.09999999998	75.5852406230963\\
2298.20000000002	75.5819518774747\\
2299.60000000002	75.5968169761273\\
2299.80000000003	75.5902430540458\\
2300	75.5880179122647\\
2300.19999999999	75.5857931574142\\
2300.50000000002	75.5889767886638\\
2300.8	75.5791212134382\\
2301.19999999996	75.5833659236084\\
2301.40000000002	75.576797740604\\
2301.9	75.5647263249348\\
2302.2	75.567910350519\\
2302.29999999997	75.5646282140375\\
2302.5	75.5667506297229\\
2302.59999999999	75.5634689712077\\
2304.59999999997	75.5846747949842\\
2304.69999999999	75.5813953488372\\
2305	75.5845733373823\\
2305.09999999999	75.5812944646885\\
2305.30000000001	75.5834128567711\\
2305.40000000004	75.5801344610714\\
2305.59999999997	75.5779156004684\\
2305.89999999998	75.5810928013877\\
2306.00000000001	75.5778153592645\\
2306.39999999999	75.5820507262085\\
2306.50000000002	75.5787739530044\\
2307.60000000002	75.5904146986177\\
2307.7	75.5871392668342\\
2307.90000000003	75.5892547660312\\
2308	75.5859798102335\\
2308.3	75.589152659851\\
2308.39999999999	75.5858782759368\\
2308.59999999999	75.5879932429506\\
2308.70000000002	75.5847193347193\\
2308.99999999998	75.5878913862544\\
2309.10000000001	75.584618049541\\
2309.39999999998	75.5877895648409\\
2309.49999999999	75.5845167994458\\
2309.69999999997	75.5866308771322\\
2309.80000000003	75.5833585869518\\
2310.80000000003	75.5939244450214\\
2310.90000000003	75.5906533967979\\
2311.20000000001	75.5851685198806\\
2311.70000000002	75.5904490007786\\
2311.80000000001	75.5871793762706\\
2312.09999999999	75.5903468558083\\
2312.40000000001	75.5805405405405\\
2312.59999999998	75.5783283607904\\
2312.90000000002	75.5814958927799\\
2313.00000000003	75.5782283515628\\
2313.20000000003	75.5803397743483\\
2313.30000000003	75.5770727068384\\
2313.70000000003	75.5812948396577\\
2313.80000000002	75.5780284368382\\
2314.19999999997	75.5822494922871\\
2314.29999999999	75.5789837538887\\
2314.60000000002	75.5821488745842\\
2314.70000000001	75.5788837048557\\
2314.90000000002	75.5809935205184\\
2314.99999999997	75.5777288238089\\
2315.30000000002	75.5808931502116\\
2315.40000000002	75.5776290218095\\
2315.60000000001	75.579738308071\\
2315.80000000001	75.5732112785526\\
2316.50000000002	75.5805922472589\\
2316.69999999998	75.574067679558\\
2317.19999999997	75.5793380226988\\
2317.30000000001	75.5760766376111\\
2317.70000000003	75.5802916558806\\
2317.8	75.5770309331723\\
2318	75.5791380872266\\
2318.09999999997	75.5758778362523\\
2318.59999999997	75.5811446068918\\
2318.69999999997	75.5778851129895\\
2318.89999999999	75.5799913755929\\
2319.00000000003	75.576732353068\\
2319.70000000001	75.5841020777653\\
2319.9	75.5775862068966\\
2321.2	75.5912635161332\\
2321.40000000003	75.5847512384234\\
2321.60000000001	75.5868544600939\\
2321.70000000001	75.5835989318632\\
2322.09999999997	75.5878046679873\\
2322.29999999999	75.5812952118498\\
2322.60000000002	75.5844491324751\\
2322.7	75.5811951093508\\
2322.89999999998	75.5832974601808\\
2323.00000000001	75.5800439068486\\
2323.19999999997	75.5778418628675\\
2323.69999999999	75.5830966520355\\
2323.90000000001	75.5765920826162\\
2325.29999999997	75.5912961210975\\
2325.40000000004	75.5880455815954\\
2326.20000000001	75.5964406998238\\
2326.49999999999	75.5866930284535\\
2326.90000000002	75.59088955737\\
2326.99999999998	75.5876412702505\\
2327.49999999998	75.5928853754941\\
2327.60000000002	75.5896378399278\\
2327.80000000001	75.5917350401649\\
2327.89999999999	75.5884879725086\\
2328.10000000002	75.5862898376428\\
2328.40000000001	75.5894352587503\\
2328.50000000001	75.5861891265138\\
2328.80000000003	75.589334020353\\
2329.00000000002	75.5828431583015\\
2329.40000000004	75.5870358446018\\
2329.50000000003	75.5837912087912\\
2330.09999999998	75.5900781048837\\
2330.20000000003	75.5868343131786\\
2330.40000000001	75.5846384895945\\
2330.70000000002	75.5877810193925\\
2330.79999999999	75.5845381612253\\
2331.39999999996	75.5908213596397\\
2331.50000000003	75.587579344656\\
2332.50000000002	75.5980450998885\\
2332.59999999996	75.5948043040254\\
2332.80000000001	75.5968965665052\\
2332.89999999998	75.5936562366052\\
2333.7	75.6020224526523\\
2334.00000000003	75.5923053853734\\
2334.20000000003	75.5901126676091\\
2334.7	75.5953400719548\\
2334.90000000004	75.5888650963598\\
2335.5	75.5951361534509\\
2335.60000000001	75.5918996446461\\
2336.5	75.6013010356929\\
2336.60000000002	75.5980656481363\\
2336.90000000001	75.6011981172443\\
2337.00000000003	75.5979632878354\\
2337.3	75.6010952340207\\
2337.40000000004	75.5978609625668\\
2337.60000000003	75.5956709586346\\
2337.90000000002	75.590248075278\\
2338.69999999997	75.5985975714041\\
2338.89999999999	75.5921333903378\\
2339.19999999999	75.5952635403753\\
2339.39999999998	75.5888010258602\\
2339.59999999997	75.5908877206479\\
2339.70000000001	75.5876570647064\\
2340.10000000002	75.5918297581403\\
2340.19999999997	75.5885997521685\\
2340.39999999999	75.5906857509079\\
2340.50000000001	75.58745620781\\
2340.79999999996	75.5820410953052\\
2341.59999999999	75.5903830550455\\
2341.7	75.5871551797762\\
2341.90000000002	75.5892399658412\\
2341.99999999999	75.5860125528372\\
2342.60000000003	75.59226533487\\
2342.79999999997	75.5858124546502\\
2343.09999999998	75.5889382041652\\
2343.20000000004	75.5857124567917\\
2343.89999999999	75.5930034129693\\
2343.99999999999	75.5897785930634\\
2344.20000000001	75.5875954442691\\
2344.80000000003	75.5938419548808\\
2344.90000000002	75.590618336887\\
2345.19999999997	75.593740672835\\
2345.29999999997	75.5905176089367\\
2346.40000000001	75.6019603665033\\
2346.50000000002	75.5987386005284\\
2346.79999999997	75.601857769824\\
2346.89999999998	75.5986365573072\\
2347.09999999999	75.6007157464213\\
2347.3	75.5942745164863\\
2347.80000000003	75.5994718684782\\
2347.9	75.5962521294719\\
2348.60000000001	75.6035253544514\\
2348.69999999998	75.6003065395095\\
2348.90000000001	75.6023839931886\\
2349.00000000003	75.5991656379039\\
2349.39999999998	75.6033198552884\\
2349.49999999997	75.600102145046\\
2350.89999999999	75.614632071459\\
2351.09999999997	75.6082000680504\\
2351.29999999999	75.6060219443736\\
2351.59999999998	75.609133818089\\
2351.70000000001	75.6059188706523\\
2352	75.6090302283066\\
2352.09999999998	75.6058158319871\\
2352.3	75.6036388369325\\
2352.89999999998	75.6098597535062\\
2353.00000000001	75.6066465513578\\
2353.2	75.6044703182765\\
2353.50000000002	75.6075798776343\\
2353.59999999999	75.6043675914518\\
2353.9	75.607476635514\\
2354.00000000003	75.604264899537\\
2354.30000000002	75.6073734284743\\
2354.49999999997	75.6009513293128\\
2355.39999999999	75.6102738272129\\
2355.50000000003	75.60706401766\\
2355.70000000003	75.6048900585788\\
2356.00000000001	75.6079962650142\\
2356.09999999997	75.6047873694933\\
2356.7	75.6109979633401\\
2356.79999999996	75.6077898935042\\
2357.09999999999	75.6024096385542\\
2357.29999999999	75.6002375498431\\
2357.69999999998	75.6043769615743\\
2357.79999999998	75.6011705331015\\
2358.09999999997	75.6042744466118\\
2358.20000000001	75.6010685663402\\
2358.50000000003	75.5956923598745\\
2359.00000000001	75.6008647365521\\
2359.09999999998	75.5976602238047\\
2359.30000000001	75.5997287445961\\
2359.50000000002	75.5933209018478\\
2359.90000000003	75.5974576271186\\
2360.09999999999	75.5910516057961\\
2360.39999999999	75.5941537809786\\
2360.59999999997	75.5877493963655\\
2361.40000000002	75.5960194791446\\
2361.80000000003	75.583216901647\\
2362.1	75.5863178393023\\
2362.20000000001	75.5831181475681\\
2362.4	75.5851851851852\\
2362.59999999998	75.5787869809963\\
2362.90000000003	75.5818874312315\\
2363.10000000001	75.575490859851\\
2363.69999999999	75.5816904983501\\
2363.79999999997	75.5784931680697\\
2364.20000000002	75.5826248783995\\
2364.29999999998	75.5794281847403\\
2365.00000000003	75.5866559553507\\
2365.10000000003	75.5834601725013\\
2365.30000000002	75.5855246469942\\
2365.40000000001	75.5823293172691\\
2365.60000000003	75.5843936255654\\
2365.69999999997	75.5811987488376\\
2366.4	75.5884217198394\\
2366.50000000003	75.5852277528944\\
2366.69999999996	75.5872908568531\\
2366.8	75.5840973425155\\
2367.2	75.5882228699362\\
2367.30000000004	75.5850299907071\\
2367.6	75.5881234953753\\
2368	75.5753557704489\\
2368.3	75.5700050667117\\
2368.60000000004	75.5646557183265\\
2368.90000000002	75.5677501055298\\
2369.30000000002	75.5549928251878\\
2369.5	75.557056043214\\
2369.59999999997	75.5538675781744\\
2369.79999999999	75.55171104266\\
2370.29999999998	75.5568680391495\\
2370.50000000001	75.5504935459377\\
2370.8	75.5451516301826\\
2371.79999999997	75.5554618660146\\
2371.89999999999	75.5522765598651\\
2372.10000000003	75.5543377455526\\
2372.20000000003	75.5511528896008\\
2372.79999999999	75.5573349066543\\
2372.89999999997	75.5541508638854\\
2373.49999999998	75.5603302999663\\
2373.70000000003	75.553964108181\\
2374.10000000003	75.5580827226013\\
2374.20000000003	75.5549003916944\\
2375.09999999997	75.5641630178511\\
2375.19999999998	75.5609817707237\\
2376.09999999998	75.5702381954381\\
2376.29999999999	75.5638781349941\\
2376.59999999997	75.566962595195\\
2376.70000000002	75.563783237967\\
2376.89999999997	75.5658392932268\\
2377.00000000003	75.5626603845021\\
2377.2	75.560509822067\\
2377.6	75.5646212726585\\
2377.70000000003	75.5614433509967\\
2378.5	75.569662826873\\
2378.59999999997	75.5664858956573\\
2379	75.5705939220714\\
2379.20000000001	75.5642415836591\\
2379.60000000003	75.5683489515485\\
2379.69999999998	75.5651735439953\\
2380.1	75.569279892446\\
2380.19999999997	75.5661051128009\\
2380.80000000003	75.572262589777\\
2380.89999999998	75.5690886182276\\
2381.30000000002	75.5731922398589\\
2381.40000000001	75.570018895654\\
2381.59999999999	75.5720703699039\\
2381.80000000001	75.5657248415131\\
2381.99999999997	75.5677763318081\\
2382.09999999998	75.5646041474267\\
2382.30000000002	75.5624580255205\\
2382.49999999997	75.5603122639134\\
2382.69999999997	75.5581668625147\\
2382.99999999998	75.5612437581302\\
2383.09999999999	75.5580731789191\\
2383.3	75.5601241923303\\
2383.39999999998	75.5569540591567\\
2383.60000000004	75.5590049083358\\
2383.70000000002	75.5558352210756\\
2384.90000000003	75.5681341719078\\
2385	75.564965829525\\
2385.19999999998	75.5628222865048\\
2385.39999999998	75.5606791029134\\
2385.60000000002	75.5627279205265\\
2385.69999999998	75.5595607343449\\
2386.80000000003	75.5708240814446\\
2386.89999999999	75.5676581483033\\
2387.30000000002	75.5591857250565\\
2387.89999999998	75.5653266331658\\
2388.00000000004	75.5621623885097\\
2388.19999999998	75.5600217728091\\
2388.60000000001	75.5641143718341\\
2388.69999999997	75.5609511051574\\
2388.99999999998	75.5640199238207\\
2389.09999999997	75.5608571906915\\
2389.50000000003	75.56494810847\\
2389.70000000001	75.5586241526488\\
2390.00000000004	75.5533241286975\\
2390.20000000003	75.5511860435928\\
2390.60000000004	75.5552766972017\\
2390.69999999996	75.5521164463778\\
2390.99999999997	75.5551838066162\\
2391.09999999997	75.5520240883239\\
2391.3	75.554068746341\\
2391.40000000003	75.5509094710433\\
2391.60000000002	75.5529539657984\\
2391.69999999998	75.5497951333724\\
2391.99999999998	75.5528615024456\\
2392.10000000004	75.5497032020734\\
2392.30000000001	75.551747199465\\
2392.40000000001	75.5485893416928\\
2393.10000000001	75.555741266923\\
2393.30000000001	75.5494275925462\\
2393.80000000004	75.554534441706\\
2393.89999999997	75.5513784461153\\
2394.10000000001	75.5534207668532\\
2394.19999999998	75.5502652132147\\
2394.7	75.5553699682646\\
2394.80000000002	75.5522151238048\\
2395.49999999997	75.5593588245116\\
2395.59999999997	75.5562048670535\\
2395.79999999999	75.5540715388789\\
2396.00000000003	75.5519385668378\\
2396.20000000004	75.5498059508409\\
2396.39999999998	75.551846442729\\
2396.50000000002	75.5486939831428\\
2397.50000000001	75.5588922255589\\
2397.6	75.5557409183801\\
2397.79999999999	75.5577797239251\\
2397.99999999997	75.5514782536174\\
2398.9	75.5606502709462\\
2399.00000000004	75.5575007294402\\
2399.30000000001	75.5605568058681\\
2399.50000000004	75.5542590431739\\
2399.69999999998	75.552129344112\\
2399.90000000004	75.55\\
2400.10000000003	75.5520373302225\\
2400.20000000002	75.5488897221181\\
2400.39999999998	75.5467610914393\\
2400.59999999999	75.5446328154288\\
2401	75.5487068426971\\
2401.10000000001	75.5455605530568\\
2401.29999999997	75.5475972349463\\
2401.49999999998	75.5413057961359\\
2401.90000000002	75.5453788509575\\
2401.99999999999	75.5422338786895\\
2402.39999999997	75.5463059313215\\
2402.49999999998	75.5431615749605\\
2402.89999999998	75.5347482313774\\
2403.70000000001	75.5428904234961\\
2403.80000000002	75.5397479096468\\
2404.00000000001	75.5376232269872\\
2404.59999999998	75.5437268682164\\
2404.69999999999	75.5405854956753\\
2404.99999999998	75.5436364392333\\
2405.10000000002	75.5404955928821\\
2405.70000000003	75.5465957269931\\
2405.89999999996	75.5403158769742\\
2406.39999999997	75.5287762310409\\
2407.10000000002	75.5358923230309\\
2407.2	75.5327545382794\\
2407.9	75.5398671096346\\
2407.99999999997	75.5367302022341\\
2408.20000000003	75.53876178217\\
2408.29999999997	75.5356253114101\\
2408.59999999998	75.5386723128659\\
2408.69999999999	75.5355363666556\\
2408.89999999998	75.5375674553757\\
2408.99999999998	75.5344319455398\\
2409.19999999998	75.5364628730337\\
2409.30000000001	75.5333277994521\\
2409.59999999999	75.5363738224675\\
2409.70000000002	75.5332392729687\\
2410.20000000003	75.5383147326059\\
2410.40000000003	75.5320472930927\\
2410.70000000001	75.5350920856147\\
2410.79999999997	75.5319590194533\\
2411.79999999999	75.5421037356441\\
2411.89999999998	75.5389718076285\\
2412.10000000001	75.5368543238537\\
2412.49999999998	75.540910221338\\
2412.59999999997	75.537779251461\\
2412.80000000001	75.5356624808322\\
2413.19999999997	75.5397173994116\\
2413.29999999999	75.5365873870887\\
2413.50000000001	75.5386145177329\\
2413.60000000003	75.5354849401334\\
2413.80000000004	75.5375119101868\\
2413.89999999997	75.5343827671914\\
2414.40000000002	75.539449161317\\
2414.59999999999	75.5331925290926\\
2414.80000000001	75.5352188496418\\
2414.99999999996	75.5289636039916\\
2415.59999999997	75.535041602848\\
2415.7	75.531914893617\\
2415.89999999999	75.533940397351\\
2415.99999999998	75.530814121932\\
2416.20000000003	75.5328394652982\\
2416.29999999997	75.5297136235723\\
2417.09999999997	75.5378123448618\\
2417.2	75.5346874612171\\
2417.4	75.5325749741469\\
2417.60000000001	75.5304628365802\\
2417.9	75.5334987593052\\
2418.00000000001	75.5303750878789\\
2418.29999999997	75.5334105193516\\
2418.40000000003	75.5302873682034\\
2420.1	75.5474754152549\\
2420.30000000003	75.5412328540737\\
2420.50000000001	75.5391225316037\\
2420.89999999996	75.5431639818257\\
2420.99999999998	75.5400437817521\\
2421.29999999997	75.5430742545635\\
2421.39999999998	75.5399545736114\\
2421.60000000001	75.5378453152744\\
2422.09999999996	75.5428948889439\\
2422.30000000004	75.5366578599736\\
2422.90000000002	75.5427156417664\\
2423.00000000001	75.5395980355743\\
2423.19999999998	75.537490199315\\
2423.6	75.5415274167595\\
2423.70000000003	75.5384107599637\\
2424.40000000003	75.5454732934626\\
2424.59999999998	75.5392419680785\\
2424.9	75.5340206185567\\
2425.89999999997	75.5441055234955\\
2426.00000000001	75.5409917150983\\
2427.29999999998	75.5540907967373\\
2427.59999999998	75.5447542941879\\
2427.79999999998	75.5426500267721\\
2428.10000000001	75.5456716909645\\
2428.20000000004	75.5425606391303\\
2428.39999999999	75.5445748404365\\
2428.59999999999	75.5383538518549\\
2428.89999999998	75.5331412103746\\
2429.09999999998	75.5351556067841\\
2429.19999999998	75.5320462684724\\
2429.69999999998	75.5370812412544\\
2429.79999999998	75.5339725914647\\
2430.29999999997	75.5390059249506\\
2430.40000000003	75.535897963382\\
2430.70000000001	75.5306894849432\\
2431.30000000003	75.5367278111376\\
2431.40000000003	75.5336212214682\\
2431.60000000002	75.5315211580376\\
2431.80000000003	75.533533451211\\
2432.09999999999	75.5242167584903\\
2432.90000000002	75.5322646937937\\
2432.99999999997	75.5291603304427\\
2433.4	75.533182658722\\
2433.49999999998	75.5300788954635\\
2433.70000000002	75.532089736215\\
2433.79999999998	75.5289864004273\\
2434.5	75.536022344533\\
2434.6	75.532919866924\\
2434.89999999996	75.5359342915811\\
2435.00000000003	75.5328323272145\\
2435.30000000002	75.5358462675536\\
2435.50000000002	75.529643619642\\
2435.8	75.5326573340449\\
2435.89999999998	75.5295566502463\\
2436.10000000003	75.5315655529103\\
2436.19999999999	75.5284652957353\\
2436.4	75.5304740406321\\
2436.50000000003	75.5273742099647\\
2436.70000000001	75.529382797111\\
2436.89999999997	75.5231842429216\\
2437.19999999997	75.5261970212941\\
2437.29999999998	75.5230983835234\\
2437.60000000002	75.5261106780982\\
2437.70000000004	75.5230125523013\\
2438.09999999998	75.5270281355098\\
2438.49999999996	75.5146395472812\\
2438.9	75.5186551865519\\
2439.10000000003	75.5124631026566\\
2439.59999999999	75.5174816575809\\
2440.00000000002	75.5051022499078\\
2440.19999999997	75.5030119247634\\
2440.60000000004	75.5070266726759\\
2440.69999999996	75.5039331366765\\
2441.09999999999	75.5079469113551\\
2441.30000000001	75.5017612845089\\
2441.50000000001	75.5037680209699\\
2441.59999999999	75.5006757586927\\
2441.80000000002	75.4985871657316\\
2442.09999999997	75.5015969208091\\
2442.30000000001	75.4954143465444\\
2442.60000000003	75.49842387522\\
2442.69999999997	75.495333224169\\
2442.89999999996	75.4973393368809\\
2443.19999999997	75.4880694143167\\
2443.40000000002	75.4900757110702\\
2443.50000000003	75.4869864134883\\
2443.70000000002	75.488992552582\\
2443.79999999997	75.4859036785466\\
2444.30000000004	75.4909180166912\\
2444.40000000003	75.4878298220495\\
2444.59999999997	75.4857446721479\\
2444.80000000002	75.4877500102254\\
2444.89999999999	75.4846625766871\\
2445.39999999998	75.4896749131057\\
2445.49999999998	75.4865881583251\\
2445.70000000003	75.4845040477553\\
2446.00000000001	75.4793344507583\\
2446.19999999998	75.48133916527\\
2446.30000000002	75.4782537606279\\
2446.50000000004	75.4802583176653\\
2446.59999999999	75.4771733355131\\
2446.79999999997	75.4750909313826\\
2447.60000000002	75.4831065898599\\
2447.70000000001	75.4800228776861\\
2447.9	75.4779411764706\\
2449.20000000001	75.4909565998448\\
2449.29999999999	75.4878745815302\\
2449.60000000004	75.482712168837\\
2450.59999999997	75.4927163667524\\
2450.80000000004	75.4865559590355\\
2450.99999999999	75.4844763575538\\
2451.19999999998	75.4823970954188\\
2451.39999999998	75.4803181725474\\
2451.59999999997	75.4782395888567\\
2452.09999999996	75.4832395400049\\
2452.19999999996	75.4801614810586\\
2453.79999999998	75.4961489873263\\
2453.89999999996	75.4930725346373\\
2454.1	75.495069676473\\
2454.20000000004	75.4919936438088\\
2456.79999999997	75.5179290976434\\
2456.89999999996	75.5148555148555\\
2457.4	75.5198372329603\\
2457.5	75.5167643229167\\
2457.70000000001	75.5146879322972\\
2458.30000000003	75.5206638464042\\
2458.4	75.5175920276591\\
2458.89999999997	75.5225701504677\\
2458.99999999999	75.5194990037005\\
2459.69999999998	75.5264655663062\\
2459.79999999999	75.5233952599699\\
2460.30000000001	75.528369370834\\
2460.39999999998	75.5252997358261\\
2460.60000000004	75.5272889828098\\
2460.70000000003	75.5242197659298\\
2460.89999999999	75.5221454693214\\
2461.79999999997	75.5310938705878\\
2461.89999999998	75.5280259951259\\
2462.19999999998	75.5310075945254\\
2462.39999999998	75.5248730964467\\
2462.69999999999	75.5197336365113\\
2463.09999999998	75.5237089964274\\
2463.30000000001	75.5175773321426\\
2463.49999999999	75.5155057639227\\
2463.90000000003	75.5194805194805\\
2464.09999999998	75.5133511890269\\
2464.40000000003	75.516331913167\\
2464.50000000003	75.5132678730829\\
2464.70000000002	75.5152547874067\\
2464.80000000001	75.5121911639418\\
2465.09999999999	75.5151711828655\\
2465.20000000003	75.512108059871\\
2465.80000000004	75.5180664260513\\
2465.89999999997	75.51500405515\\
2466.29999999997	75.5189750243269\\
2466.50000000002	75.5128516986946\\
2466.69999999998	75.5107832009081\\
2467.79999999997	75.5216986101544\\
2467.90000000003	75.5186385737439\\
2468.20000000002	75.5216140663615\\
2468.40000000002	75.5154952400243\\
2468.69999999997	75.5184705119896\\
2468.79999999996	75.5154117218194\\
2469	75.5133449435017\\
2469.4	75.5173111965985\\
2469.59999999996	75.5111956917844\\
2469.79999999999	75.5091299242884\\
2470.7	75.5180508337381\\
2470.80000000001	75.5149945364037\\
2471.09999999999	75.517966979605\\
2471.19999999998	75.5149111803504\\
2471.69999999999	75.5198640666721\\
2471.79999999996	75.5168089324002\\
2472.09999999999	75.511689992719\\
2472.59999999998	75.5166417276661\\
2472.70000000002	75.5135878356519\\
2473.20000000001	75.5185379856871\\
2473.60000000003	75.5063265553624\\
2474.20000000002	75.5122660954613\\
2474.3	75.5092143549952\\
2474.59999999996	75.5041015072534\\
2474.90000000004	75.5070707070707\\
2475.10000000001	75.5009696186167\\
2475.40000000003	75.4958594223389\\
2475.69999999998	75.4988286614428\\
2476.1	75.4866327437202\\
2476.60000000001	75.4915815399524\\
2476.80000000002	75.4854858896201\\
2477.19999999997	75.4894441529084\\
2477.3	75.4863970291435\\
2477.89999999999	75.4923325262308\\
2477.99999999998	75.4892861466446\\
2479.00000000002	75.4991730870074\\
2479.09999999997	75.4961277831559\\
2480.00000000003	75.5050199588726\\
2480.09999999997	75.5019756471252\\
2480.90000000001	75.5098750503829\\
2480.99999999997	75.5068316472532\\
2481.39999999998	75.5107797703002\\
2481.50000000002	75.5077369439072\\
2482.09999999997	75.5136572395456\\
2482.20000000002	75.5106151552995\\
2482.59999999998	75.5145607604624\\
2482.69999999996	75.5115192524569\\
2482.90000000001	75.5134917438582\\
2482.99999999999	75.5104506463694\\
2483.19999999998	75.5083960858535\\
2483.4	75.5103684316489\\
2483.50000000002	75.5073280721533\\
2484.90000000002	75.5211267605634\\
2485.00000000002	75.5180878033077\\
2485.20000000001	75.5160342815757\\
2485.69999999998	75.5209590473892\\
2485.9	75.5148833467418\\
2486.29999999996	75.5188223938224\\
2486.70000000001	75.5066752452952\\
2486.99999999996	75.5096296891963\\
2487.10000000001	75.5065937600515\\
2487.69999999998	75.5125010049039\\
2487.89999999998	75.5064308681672\\
2488.30000000001	75.5103681080212\\
2488.39999999999	75.5073337351818\\
2488.70000000002	75.50225008036\\
2489.09999999997	75.4941346617387\\
2489.50000000004	75.4860218508997\\
2489.99999999997	75.4909441387896\\
2490.1	75.4879126174604\\
2490.60000000002	75.4928333400249\\
2490.69999999999	75.489802473101\\
2490.90000000004	75.4877559213167\\
2491.09999999997	75.4857096981375\\
2491.29999999997	75.4876776109818\\
2491.39999999997	75.4846478025286\\
2491.7	75.4875993257886\\
2491.79999999998	75.4845700068221\\
2492.29999999999	75.4894880436527\\
2492.50000000004	75.4834309556287\\
2492.80000000003	75.4863813229572\\
2492.90000000001	75.4833533894906\\
2493.3	75.4872864361916\\
2493.4	75.4842590735913\\
2493.60000000002	75.4862252877251\\
2493.79999999996	75.4801716187498\\
2494.00000000004	75.482137845315\\
2494.09999999997	75.47911153877\\
2494.49999999999	75.4830433736872\\
2494.59999999997	75.4800176373913\\
2494.90000000002	75.4829659318637\\
2495.10000000001	75.4769156781019\\
2495.6	75.4818287454422\\
2495.69999999999	75.4788043913775\\
2495.90000000001	75.4767628205128\\
2496.60000000002	75.4836384026916\\
2496.69999999998	75.4806151874399\\
2496.90000000004	75.4825790949139\\
2496.99999999999	75.4795562852909\\
2497.3	75.482501801874\\
2497.4	75.4794794794795\\
2497.60000000002	75.4774392441046\\
2498.49999999996	75.486272312495\\
2498.69999999997	75.4802305106451\\
2499.29999999999	75.4861166680003\\
2499.40000000003	75.4830966193239\\
2500	75.4889804407824\\
2500.10000000001	75.4859611231101\\
2500.30000000004	75.4839225723884\\
2500.49999999997	75.4818843477565\\
2501.19999999996	75.4887458521569\\
2501.30000000002	75.4857279923243\\
2501.5	75.4836904381196\\
2501.79999999998	75.4786362364603\\
2501.99999999999	75.4765996562887\\
2502.20000000002	75.4745634016705\\
2502.39999999997	75.4725274725275\\
2502.6	75.4704918687817\\
2503.2	75.4763711900292\\
2503.30000000001	75.4733562355197\\
2503.60000000004	75.4762950832767\\
2503.69999999998	75.4732806134675\\
2503.9	75.4752396166134\\
2504	75.4722255500978\\
2504.19999999996	75.4741844028271\\
2504.40000000003	75.4681573168297\\
2504.89999999996	75.4730538922156\\
2504.99999999997	75.4700411161231\\
2505.70000000002	75.4768936068321\\
2505.90000000002	75.4708699122107\\
2506.1	75.4728273880776\\
2506.40000000001	75.4637941352483\\
2506.60000000003	75.4657517852156\\
2506.70000000004	75.4627413435456\\
2507.49999999999	75.4705694688148\\
2507.6	75.4675599154604\\
2508.09999999997	75.47245036281\\
2508.20000000001	75.4694414543715\\
2508.50000000003	75.4644024555529\\
2508.8	75.4673362828331\\
2508.90000000003	75.4643284176963\\
2509.40000000002	75.4692169754931\\
2509.5	75.4662097545426\\
2509.69999999998	75.4641804127819\\
2509.99999999999	75.4591450539819\\
2511.00000000004	75.468918004062\\
2511.09999999997	75.4659127110545\\
2511.29999999997	75.4638846858326\\
2511.69999999999	75.4677920216578\\
2511.80000000001	75.4647876109718\\
2513.00000000003	75.4765031236322\\
2513.10000000001	75.4734999204202\\
2513.49999999999	75.4774029280713\\
2513.80000000002	75.4683957197979\\
2514.89999999998	75.479125248509\\
2515.00000000002	75.476124209773\\
2515.70000000004	75.482947770093\\
2515.79999999999	75.4799475336858\\
2516.60000000001	75.4877418842134\\
2516.80000000003	75.4817434145179\\
2517.20000000002	75.4856393755214\\
2517.30000000002	75.4826408198935\\
2517.49999999997	75.4845884969812\\
2517.60000000004	75.48159034039\\
2517.8	75.4795663052544\\
2518.60000000003	75.4873545876841\\
2518.69999999999	75.4843576306178\\
2519.00000000003	75.4793378587591\\
2519.40000000001	75.4713236753324\\
2519.80000000001	75.4752172705266\\
2519.90000000003	75.4722222222222\\
2520.60000000002	75.4790336017773\\
2520.70000000002	75.4760393525865\\
2521.29999999998	75.4818751487269\\
2521.39999999997	75.4788816180845\\
2521.89999999996	75.4837430610627\\
2521.99999999996	75.4807501685104\\
2522.30000000003	75.4757373929591\\
2522.6	75.4786538232846\\
2522.80000000001	75.4726703396884\\
2523.09999999996	75.4755865567533\\
2523.30000000003	75.4696045018626\\
2523.49999999996	75.4675859882707\\
2523.80000000004	75.4705020008717\\
2523.90000000004	75.4675118858954\\
2524.3	75.4713991443511\\
2524.39999999998	75.4684095860566\\
2524.69999999998	75.4713244613435\\
2524.79999999999	75.468335379619\\
2525.30000000002	75.4731923655659\\
2525.40000000002	75.4702039200158\\
2525.90000000003	75.4750593824228\\
2526	75.4720715727802\\
2526.30000000002	75.4749841671944\\
2526.40000000002	75.4719968335642\\
2526.7	75.4669938261833\\
2527.09999999999	75.4708768597658\\
2527.19999999999	75.4678906342737\\
2527.4	75.4658753709199\\
2527.60000000001	75.4678165921589\\
2527.69999999999	75.4648310784081\\
2528.40000000002	75.471623492189\\
2528.50000000001	75.4686387724433\\
2528.89999999996	75.4725187821273\\
2529.00000000003	75.4695346170574\\
2529.70000000002	75.4763222389122\\
2529.90000000001	75.4703557312253\\
2530.20000000003	75.4732640398372\\
2530.29999999997	75.4702813784382\\
2530.49999999999	75.4682683948471\\
2530.90000000003	75.4721453970762\\
2531.00000000004	75.4691636047568\\
2531.20000000001	75.4711018053964\\
2531.30000000002	75.4681204076795\\
2531.70000000001	75.4719962082313\\
2532.1	75.4600742437406\\
2532.89999999997	75.4678247137781\\
2533.00000000001	75.4648454462911\\
2533.20000000003	75.4628350373031\\
2533.40000000003	75.4608249457273\\
2533.6	75.4588151714883\\
2533.90000000002	75.4617205998421\\
2533.99999999999	75.4587427489049\\
2534.29999999998	75.4616477272727\\
2534.40000000003	75.4586703491813\\
2534.60000000001	75.4606067779224\\
2534.69999999999	75.4576297932776\\
2534.90000000003	75.4595660749507\\
2534.99999999996	75.4565894836496\\
2535.90000000004	75.4652996845426\\
2536.10000000003	75.4593486318114\\
2536.50000000001	75.4632184814318\\
2536.60000000004	75.4602436236055\\
2536.90000000002	75.4631454473788\\
2537.1	75.4571969099795\\
2537.30000000001	75.4551903523292\\
2537.90000000001	75.4609929078014\\
2537.99999999996	75.4580197785745\\
2538.19999999996	75.4599535121932\\
2538.40000000003	75.4540082726019\\
2539.30000000003	75.4627077262345\\
2539.40000000001	75.4597361685371\\
2539.70000000003	75.462634853138\\
2539.80000000002	75.45966376629\\
2539.99999999997	75.4576591472777\\
2540.49999999996	75.4467448634181\\
2541.00000000001	75.4515760890953\\
2541.09999999997	75.448606957343\\
2541.40000000001	75.4515050167224\\
2541.50000000003	75.4485363550519\\
2541.70000000003	75.4504681721615\\
2541.79999999997	75.4474999016484\\
2542	75.4494315723221\\
2542.09999999998	75.4464636928645\\
2542.40000000001	75.4414945919371\\
2542.60000000003	75.4434262791521\\
2542.90000000001	75.4345261502163\\
2543.19999999996	75.4295600204459\\
2543.6	75.433423752801\\
2543.70000000001	75.4304583693687\\
2544.1	75.4343212011634\\
2544.50000000003	75.4224632555215\\
2544.99999999997	75.427291658481\\
2545.09999999999	75.4243281471004\\
2545.40000000004	75.4193675112944\\
2545.69999999996	75.4222641212978\\
2545.80000000002	75.4193016222161\\
2546.19999999999	75.4231630208538\\
2546.29999999999	75.4202010681747\\
2546.49999999999	75.4221314694102\\
2546.60000000002	75.4191699061531\\
2546.79999999997	75.4171738191527\\
2547.10000000002	75.4200690954774\\
2547.20000000004	75.4171083107604\\
2547.69999999997	75.4219326477746\\
2547.79999999997	75.4189724871463\\
2548.5	75.4257239268618\\
2548.69999999999	75.419805398619\\
2548.90000000004	75.4217340133386\\
2548.99999999998	75.4187752540112\\
2549.20000000004	75.4207037225905\\
2549.40000000002	75.4147872131791\\
2549.60000000002	75.4127936619995\\
2551.49999999996	75.4311020536134\\
2551.59999999999	75.4281459419211\\
2551.80000000003	75.4261530624241\\
2551.99999999997	75.4280788370362\\
2552.09999999999	75.4251234229292\\
2552.30000000001	75.4270490518728\\
2552.39999999998	75.4240940254652\\
2553.10000000004	75.430831897227\\
2553.29999999999	75.4249236312368\\
2553.90000000002	75.4306969459671\\
2554.00000000001	75.4277436278924\\
2554.20000000004	75.4296676193086\\
2554.4	75.4237619886475\\
2554.60000000002	75.4256859905273\\
2554.69999999998	75.422733677783\\
2555.30000000004	75.4285043437427\\
2555.50000000004	75.4226013460635\\
2555.79999999998	75.417661097852\\
2556.90000000001	75.4282362143136\\
2557.00000000001	75.4252864573149\\
2557.29999999997	75.4281692343787\\
2557.39999999997	75.425219941349\\
2558	75.4309839333881\\
2558.10000000004	75.4280353373466\\
2558.29999999999	75.4260475297061\\
2558.50000000003	75.4240600328305\\
2558.69999999996	75.4259809285603\\
2558.79999999997	75.423033334636\\
2559.60000000003	75.4307145368598\\
2559.7	75.4277677943589\\
2559.90000000004	75.4296875\\
2560	75.4267411429241\\
2560.19999999996	75.4286607038238\\
2560.30000000002	75.4257147320731\\
2560.70000000003	75.4295532646048\\
2560.80000000001	75.4266078331837\\
2561.00000000004	75.4285268048885\\
2561.09999999999	75.4255817585507\\
2561.3	75.4275005856172\\
2561.40000000004	75.4245559242631\\
2561.70000000004	75.4274338355844\\
2561.80000000002	75.4244896365978\\
2561.99999999998	75.4264080246673\\
2562.10000000002	75.4234642104442\\
2562.30000000002	75.4253824539494\\
2562.39999999996	75.4224390243902\\
2562.59999999998	75.4243571233465\\
2562.79999999997	75.4184712630224\\
2563.30000000003	75.4232659748771\\
2563.40000000003	75.4203237760874\\
2563.59999999998	75.4222412918828\\
2563.7	75.4192994773383\\
2563.90000000003	75.421216848674\\
2564.00000000002	75.4182754182754\\
2564.79999999997	75.4259425318726\\
2564.89999999997	75.4230019493177\\
2565.10000000002	75.4249181350382\\
2565.20000000004	75.4219779363037\\
2565.39999999998	75.4199961021243\\
2565.59999999998	75.4180145769186\\
2566.90000000001	75.4304635761589\\
2567.00000000003	75.4275252230143\\
2567.20000000003	75.4294394889573\\
2567.3	75.4265015190465\\
2567.9	75.4322429906542\\
2568.00000000001	75.4293057123944\\
2568.19999999996	75.4273254682085\\
2568.50000000004	75.4301954372031\\
2568.6	75.4272589247479\\
2568.80000000002	75.4252793024252\\
2569.29999999997	75.4300614929556\\
2569.40000000003	75.4271258999805\\
2569.59999999998	75.4251469043079\\
2569.8	75.4231682166621\\
2570.2	75.4153211687352\\
2571.59999999997	75.4287047478322\\
2571.69999999997	75.4257718329575\\
2571.90000000002	75.4237947122862\\
2573.09999999997	75.4352557127312\\
2573.39999999997	75.4264620167088\\
2573.80000000004	75.4302808966937\\
2573.89999999999	75.4273504273504\\
2574.10000000003	75.4253748737472\\
2574.40000000001	75.4282384929112\\
2574.50000000001	75.4253087858308\\
2574.8	75.4281719678434\\
2574.90000000003	75.4252427184466\\
2575.39999999998	75.4300135895943\\
2575.49999999998	75.4270849510794\\
2575.90000000002	75.430900621118\\
2575.99999999996	75.4279725165948\\
2576.40000000003	75.4317873083641\\
2576.59999999996	75.4259323941476\\
2576.90000000004	75.4287931703531\\
2577.00000000003	75.4258662838074\\
2577.19999999998	75.4277732510767\\
2577.30000000002	75.4248467447816\\
2577.90000000001	75.4305663304887\\
2578.00000000002	75.4276405104534\\
2578.70000000004	75.4343105320304\\
2578.90000000002	75.4284606436603\\
2579.10000000001	75.4303660049628\\
2579.29999999996	75.4245173296115\\
2579.6	75.4273752761949\\
2579.90000000002	75.4186046511628\\
2580.29999999998	75.4224151294373\\
2580.39999999998	75.4194923464445\\
2580.59999999999	75.4175223776495\\
2580.90000000004	75.4203796977915\\
2580.99999999998	75.4174576730851\\
2581.60000000004	75.4231707789441\\
2581.7	75.4202494383763\\
2582.39999999998	75.4269119070668\\
2582.5	75.4239913265701\\
2582.69999999999	75.4258943781942\\
2582.80000000004	75.4229741763134\\
2583.69999999998	75.4315349485254\\
2583.80000000002	75.4286156585007\\
2584.00000000003	75.4266475755582\\
2584.19999999998	75.4246797972372\\
2584.90000000003	75.431334622824\\
2585.09999999997	75.4254989942751\\
2585.40000000002	75.4283504157803\\
2585.50000000003	75.4254331683168\\
2585.89999999996	75.4292343387471\\
2585.99999999999	75.4263176211283\\
2586.20000000001	75.4282179174883\\
2586.3	75.4253015774822\\
2586.50000000003	75.4272017320034\\
2586.70000000003	75.4213700324726\\
2588.4	75.4375120726289\\
2588.5	75.4345978521208\\
2588.69999999999	75.4364956736712\\
2588.99999999998	75.4277548182766\\
2589.39999999997	75.431550492373\\
2589.69999999997	75.4228125723994\\
2590.09999999998	75.4150258667284\\
2590.40000000004	75.4178729974908\\
2590.49999999997	75.4149617849147\\
2590.69999999996	75.4168596572487\\
2590.80000000003	75.4139488208731\\
2591.29999999999	75.4186925985954\\
2591.40000000001	75.4157823654254\\
2591.60000000004	75.417679515376\\
2591.70000000004	75.4147696581526\\
2592.59999999998	75.4233038916959\\
2592.70000000003	75.4203949398334\\
2593	75.4232385947322\\
2593.40000000003	75.4116059379217\\
2593.59999999998	75.4135019470255\\
2593.99999999999	75.4018734821325\\
2594.90000000003	75.4104046242775\\
2595	75.4074987476398\\
2595.19999999996	75.4093939043656\\
2595.39999999997	75.4035831246388\\
2595.60000000001	75.4054782910198\\
2595.70000000004	75.4025733877803\\
2596.09999999996	75.4063631461367\\
2596.19999999997	75.4034587682471\\
2597.10000000003	75.4119821346065\\
2597.19999999999	75.409078658607\\
2597.39999999998	75.4109720885467\\
2597.70000000002	75.4022634536916\\
2597.99999999996	75.3974057965436\\
2598.39999999999	75.4011929959592\\
2598.50000000002	75.3982913876703\\
2598.69999999997	75.3963367708173\\
2598.9	75.3943824547903\\
2599.10000000003	75.3962757771622\\
2599.19999999996	75.3933751394606\\
2599.49999999996	75.3885213109709\\
2599.69999999998	75.3904146472806\\
2599.80000000001	75.3875149044194\\
2600.10000000002	75.3903545881086\\
2600.20000000003	75.38745529362\\
2600.4	75.3855027879254\\
2600.80000000002	75.3892883232727\\
2601.00000000003	75.3834915997078\\
2601.20000000001	75.3853842309614\\
2601.30000000001	75.382486353502\\
2601.49999999997	75.3843788437884\\
2601.60000000003	75.3814813391244\\
2601.90000000003	75.3843197540354\\
2602.09999999999	75.3785258627315\\
2602.60000000003	75.3832558496945\\
2603.19999999998	75.3658817654515\\
2603.70000000001	75.3706121821952\\
2603.80000000004	75.3677176542878\\
2605.29999999999	75.3818991325708\\
2605.39999999999	75.3790059489541\\
2605.7	75.3818405096324\\
2605.80000000003	75.3789477723627\\
2606.09999999997	75.3817819046888\\
2606.2	75.3788896136285\\
2606.40000000003	75.3807788221753\\
2606.59999999997	75.3749952046649\\
2606.80000000002	75.3730484483486\\
2607.59999999996	75.3806035970395\\
2607.70000000003	75.3777130148018\\
2607.99999999996	75.3728768068709\\
2608.30000000004	75.375709247048\\
2608.40000000002	75.3728196281388\\
2609.10000000004	75.3794266441821\\
2609.2	75.3765377687502\\
2609.39999999996	75.3745928338762\\
2610.30000000004	75.3830830524058\\
2610.39999999997	75.3801953648726\\
2611.09999999999	75.3867953431373\\
2611.19999999997	75.3839083981159\\
2611.40000000004	75.3819643882826\\
2611.70000000003	75.3847920974041\\
2611.80000000001	75.3819058922623\\
2612.10000000003	75.3847331751014\\
2612.5	75.3731914567863\\
2612.69999999999	75.3712492345377\\
2613.70000000003	75.3806718188079\\
2613.79999999997	75.3777879796473\\
2615.80000000002	75.3966130203754\\
2615.89999999997	75.3937308868502\\
2616.10000000001	75.3917896185307\\
2616.50000000004	75.3955514790186\\
2616.60000000004	75.3926701570681\\
2616.80000000002	75.3945508043869\\
2617.09999999996	75.3859086046156\\
2617.30000000002	75.3877894093375\\
2617.70000000001	75.3762701505081\\
2618.00000000004	75.3714525801153\\
2618.20000000004	75.3733338425696\\
2618.50000000001	75.3646986939586\\
2619.20000000003	75.3712824036956\\
2619.29999999996	75.3684049782393\\
2619.90000000003	75.3740458015267\\
2620.1	75.368292496756\\
2620.29999999997	75.3663562814837\\
2621.09999999999	75.3738745612696\\
2621.20000000004	75.3709991225728\\
2621.4	75.3728781232119\\
2621.6	75.36712819926\\
2621.80000000003	75.3690072085129\\
2621.9	75.3661327231121\\
2622.30000000002	75.3698901769372\\
2622.39999999998	75.3670162059104\\
2623.90000000003	75.3810975609756\\
2623.99999999997	75.378224915209\\
2624.80000000004	75.3857289801516\\
2624.90000000002	75.3828571428571\\
2625.30000000002	75.3866077550088\\
2625.40000000001	75.383736431156\\
2625.59999999997	75.3856114559927\\
2625.70000000004	75.3827404981339\\
2625.90000000004	75.3846153846154\\
2626.09999999998	75.3788744193131\\
2626.70000000003	75.3844982488199\\
2626.79999999998	75.3816285355362\\
2626.99999999997	75.3796962430056\\
2627.20000000004	75.3815704335249\\
2627.30000000004	75.3787013777879\\
2627.8	75.3833859735911\\
2627.90000000001	75.3805175038052\\
2628.09999999999	75.3785861045583\\
2628.39999999996	75.3813962335933\\
2628.50000000003	75.3785284942555\\
2628.80000000003	75.3813382022899\\
2628.89999999997	75.3784709014835\\
2629.39999999998	75.3831526906256\\
2629.5	75.3802859750532\\
2629.99999999998	75.3849663510893\\
2630.20000000002	75.3792343078736\\
2630.80000000004	75.3848492911171\\
2630.90000000004	75.381984036488\\
2631.20000000001	75.3847907878235\\
2631.49999999998	75.3761969904241\\
2631.79999999998	75.3790037615411\\
2632	75.3732760913339\\
2632.29999999997	75.3760826622094\\
2632.40000000001	75.3732193732194\\
2633.79999999997	75.3863092752192\\
2634.10000000001	75.377723787108\\
2634.29999999999	75.3795930762223\\
2634.50000000002	75.3738707963258\\
2635	75.3785435087852\\
2635.20000000003	75.3728228285205\\
2635.50000000004	75.3680376384884\\
2635.7	75.3661127551408\\
2636.40000000003	75.3726531386308\\
2636.59999999998	75.3669359426556\\
2637.59999999999	75.3762747848504\\
2637.70000000003	75.3734172416408\\
2637.90000000003	75.371493555724\\
2638.80000000002	75.3798931372921\\
2639.10000000003	75.3713246438315\\
2640.10000000001	75.3806529808348\\
2640.20000000002	75.3777979775026\\
2641.29999999998	75.388051790717\\
2641.39999999999	75.3851978042779\\
2641.70000000002	75.3879930350519\\
2641.80000000001	75.3851394829479\\
2642.20000000001	75.3888657608901\\
2642.29999999996	75.386012715713\\
2642.50000000002	75.3878755770832\\
2642.69999999998	75.3821704253065\\
2643.10000000004	75.3858958837772\\
2643.19999999998	75.3830439223698\\
2643.79999999996	75.3886304323159\\
2643.90000000002	75.3857791225416\\
2644.29999999998	75.3895023445772\\
2644.40000000003	75.386651540934\\
2644.59999999998	75.3847317276062\\
2644.80000000001	75.3865930659004\\
2644.89999999999	75.3837429111531\\
2645.10000000001	75.3856041131105\\
2645.2	75.3827543189808\\
2645.39999999998	75.3846153846154\\
2645.5	75.381765951013\\
2645.69999999998	75.3836268803387\\
2645.79999999997	75.3807778071734\\
2646.30000000004	75.3854292623942\\
2646.49999999996	75.3797324869644\\
2646.69999999998	75.3778147196615\\
2647.00000000002	75.3806051905859\\
2647.09999999999	75.3777576307041\\
2648.00000000001	75.3861259015898\\
2648.1	75.383279208519\\
2648.49999999999	75.3869969040248\\
2648.70000000004	75.3813047417699\\
2649	75.3840927107319\\
2649.1	75.3812471689567\\
2649.30000000004	75.3831056088171\\
2649.40000000003	75.3802604264956\\
2650.10000000002	75.3867632631499\\
2650.29999999998	75.3810745547842\\
2650.50000000001	75.3791594355995\\
2651.39999999996	75.3875165000943\\
2651.59999999996	75.3818305238149\\
2651.79999999999	75.3799162864361\\
2653.69999999999	75.3975431456779\\
2653.80000000002	75.3947021364784\\
2654.79999999997	75.4039700177031\\
2654.9	75.4011299435028\\
2655.20000000003	75.4039091628065\\
2655.29999999996	75.4010695187166\\
2655.50000000002	75.3991564994728\\
2655.69999999998	75.397243768356\\
2656.3	75.4028007830146\\
2656.49999999998	75.3971241436422\\
2656.79999999997	75.3999021415936\\
2656.9	75.3970643582988\\
2658.70000000003	75.4137204754024\\
2658.9	75.4080481383979\\
2659.80000000004	75.4163690364299\\
2660.10000000004	75.4078640703706\\
2660.70000000003	75.413409500902\\
2660.79999999999	75.410575369236\\
2661	75.4124234339183\\
2661.10000000002	75.4095896588005\\
2661.4	75.41236145031\\
2661.49999999999	75.4095281033965\\
2661.80000000001	75.41229948533\\
2661.89999999999	75.4094665664914\\
2662.09999999997	75.407557659079\\
2662.29999999999	75.4094050480769\\
2662.39999999998	75.406572769953\\
2662.80000000004	75.4102670021405\\
2662.99999999997	75.4046036573918\\
2663.19999999997	75.406450643938\\
2663.30000000003	75.4036194338064\\
2663.49999999999	75.4054662862292\\
2663.60000000002	75.402635431918\\
2664.19999999998	75.4081747550951\\
2664.29999999998	75.4053445428614\\
2664.90000000001	75.4108818011257\\
2665.00000000003	75.4080522306855\\
2665.39999999997	75.4117426374039\\
2665.59999999997	75.4060847057058\\
2665.79999999996	75.4079297798117\\
2665.90000000002	75.4051012753188\\
2666.10000000004	75.4069462155877\\
2666.19999999998	75.4041180662341\\
2666.40000000004	75.4022126382899\\
2666.70000000003	75.4049797510124\\
2666.79999999999	75.4021523116727\\
2667.00000000001	75.4039968505118\\
2667.10000000003	75.4011697660468\\
2667.29999999999	75.3992652020694\\
2667.5	75.4011096116359\\
2667.59999999996	75.3982831652735\\
2668.09999999997	75.4028933363316\\
2668.29999999998	75.3972417928347\\
2668.49999999998	75.3953383796747\\
2668.69999999999	75.3934352517986\\
2669	75.3887078041287\\
2669.39999999999	75.3923955796966\\
2669.50000000001	75.3895714713815\\
2670.59999999998	75.3997079417381\\
2670.69999999997	75.3968848285158\\
2671.09999999999	75.4005690326445\\
2671.30000000004	75.3949240098825\\
2671.90000000002	75.4004491017964\\
2671.99999999997	75.3976273343063\\
2672.49999999998	75.4022300381651\\
2672.80000000001	75.3937670694751\\
2673.1	75.3965285051624\\
2673.20000000001	75.3937081509744\\
2673.4	75.3918084907425\\
2673.69999999997	75.3870895354926\\
2673.99999999997	75.3898507909203\\
2674.09999999999	75.3870316356293\\
2674.69999999997	75.3925527142216\\
2674.80000000001	75.3897341956709\\
2676.09999999996	75.401688961961\\
2676.29999999999	75.3960544014348\\
2676.60000000001	75.398811969963\\
2677	75.3875462253932\\
2677.19999999998	75.3856497217346\\
2677.69999999997	75.3902457241019\\
2677.79999999996	75.3874304492326\\
2678.09999999999	75.3827197371369\\
2679.00000000003	75.3909895114031\\
2679.20000000003	75.3853618482439\\
2679.8	75.390872793761\\
2680.00000000003	75.3852468191485\\
2680.19999999997	75.3870835354251\\
2680.39999999996	75.3814586830815\\
2680.79999999998	75.3739415867806\\
2681.10000000002	75.3692376547814\\
2681.4	75.3645347753123\\
2682.2	75.371882339783\\
2682.29999999999	75.3690724724128\\
2682.79999999998	75.3736628275374\\
2682.89999999997	75.3708535221767\\
2683.09999999999	75.3689624329159\\
2683.60000000003	75.3735514401759\\
2683.70000000001	75.3707429763768\\
2683.99999999996	75.3734957713945\\
2684.10000000004	75.3706877281872\\
2685	75.3789430561245\\
2685.09999999998	75.3761358558022\\
2685.30000000004	75.3742459223952\\
2685.49999999999	75.3760798331844\\
2685.60000000003	75.3732732620918\\
2686.19999999996	75.378773778059\\
2686.30000000004	75.3759678379988\\
2686.7	75.3796337650737\\
2686.80000000002	75.3768283151587\\
2687.3	75.3814095408201\\
2687.50000000001	75.3757999702337\\
2687.90000000004	75.3794642857143\\
2687.99999999996	75.3766600944905\\
2688.59999999999	75.3821549447688\\
2688.70000000001	75.3793513835168\\
2689.09999999999	75.383013535624\\
2689.19999999997	75.3802104636894\\
2689.39999999998	75.3820412716118\\
2689.60000000002	75.3764360337584\\
2689.89999999999	75.3791821561338\\
2689.99999999998	75.3763800602208\\
2690.20000000004	75.3744935509051\\
2690.4	75.3763241033265\\
2690.60000000003	75.3707213736203\\
2690.89999999998	75.3734671125976\\
2690.99999999998	75.3706662702984\\
2691.80000000001	75.3779858092797\\
2692	75.3723858697671\\
2692.19999999998	75.3705010585745\\
2692.49999999998	75.3732451905222\\
2692.70000000001	75.3676470588235\\
2693.19999999997	75.3722199532172\\
2693.29999999999	75.3694215489716\\
2693.60000000002	75.3647399487694\\
2693.99999999998	75.3683976095913\\
2694.1	75.3656001781605\\
2694.30000000002	75.3674287410926\\
2694.4	75.3646316570792\\
2695.40000000003	75.3737710999814\\
2695.50000000003	75.3709749220953\\
2695.7	75.372802136657\\
2695.89999999999	75.3672106824926\\
2696.10000000002	75.3690379051999\\
2696.19999999997	75.3662426287876\\
2696.39999999997	75.3643612089746\\
2696.59999999997	75.3624800682315\\
2696.79999999997	75.3605992064963\\
2696.99999999998	75.3587186237069\\
2697.2	75.3605457309161\\
2697.30000000001	75.3577519092459\\
2698.19999999999	75.3659711670311\\
2698.4	75.3603853992959\\
2698.69999999999	75.3631243515637\\
2698.79999999998	75.3603319871059\\
2700.49999999998	75.3758424053914\\
2700.6	75.3730514311105\\
2701.70000000003	75.3830779480347\\
2701.8	75.3802879455198\\
2702.20000000003	75.3728305517522\\
2702.89999999997	75.3792082870884\\
2702.99999999996	75.3764196663091\\
2703.19999999998	75.374542226168\\
2703.50000000005	75.3698772007693\\
2704.60000000004	75.3798942581432\\
2704.69999999998	75.377107364685\\
2705.10000000003	75.3696584356055\\
2705.69999999997	75.3751201123512\\
2705.80000000001	75.3723345282531\\
2705.99999999999	75.3741546875577\\
2706.1	75.3713694479344\\
2706.40000000001	75.3667097727693\\
2706.70000000002	75.3694399290675\\
2706.89999999998	75.3638714444034\\
2707.80000000003	75.3720595295247\\
2707.90000000003	75.3692762186115\\
2708.20000000004	75.3720045785179\\
2708.29999999999	75.3692216806971\\
2708.90000000003	75.374677002584\\
2709.10000000004	75.3691126531817\\
2709.29999999997	75.3709308333949\\
2709.40000000001	75.3681491050009\\
2709.60000000002	75.3699671550356\\
2709.69999999998	75.3671857701675\\
2710.1	75.3708213415984\\
2710.29999999999	75.3652597402597\\
2710.60000000005	75.3679861290442\\
2710.69999999997	75.3652058432935\\
2711.29999999997	75.3706572250498\\
2711.39999999997	75.3678775585469\\
2711.70000000003	75.3632273766502\\
2711.90000000004	75.3613569321534\\
2712.10000000001	75.359486763513\\
2712.39999999998	75.3622119815668\\
2712.49999999999	75.3594337535943\\
2712.79999999999	75.3621585756939\\
2712.89999999996	75.359380759307\\
2713.10000000001	75.3575114256229\\
2713.30000000002	75.3593277806442\\
2713.4	75.3565505804312\\
2713.89999999997	75.3610906411201\\
2714.00000000003	75.3583139899046\\
2714.30000000004	75.3610374300029\\
2714.70000000001	75.3499336967732\\
2715.50000000004	75.3571954632494\\
2715.59999999999	75.3544205913761\\
2716.00000000003	75.3580501454291\\
2716.10000000003	75.3552757528901\\
2716.69999999999	75.3607184923439\\
2716.79999999996	75.3579447164047\\
2717.40000000001	75.3633854645814\\
2717.50000000001	75.360612304975\\
2717.7	75.3624254912061\\
2717.79999999998	75.3596526730196\\
2718.89999999998	75.3696211842589\\
2719.30000000004	75.3585349709495\\
2720.10000000003	75.3657819277994\\
2720.20000000001	75.3630114325626\\
2720.40000000003	75.3611468480059\\
2720.6	75.3592825375822\\
2721.10000000004	75.3638100837866\\
2721.20000000005	75.3610406790872\\
2721.79999999999	75.366471949741\\
2721.99999999998	75.3609345725726\\
2722.2	75.359071373471\\
2722.40000000002	75.3572084481175\\
2722.60000000003	75.3553457964521\\
2722.79999999999	75.3571559734107\\
2722.99999999999	75.3516213139437\\
2723.50000000003	75.3561462769864\\
2723.59999999998	75.3533795939347\\
2724.20000000003	75.3588077671328\\
2724.30000000001	75.3560416972544\\
2724.49999999996	75.3578506936798\\
2724.60000000002	75.3550849634822\\
2724.80000000001	75.3568938309663\\
2724.90000000003	75.354128440367\\
2726.49999999999	75.3685909190934\\
2726.59999999996	75.365826823633\\
2727.19999999999	75.3712462875371\\
2727.29999999998	75.3684828041358\\
2727.5	75.366622671946\\
2727.79999999998	75.369331720371\\
2727.99999999997	75.3638063120853\\
2728.2	75.3656122860389\\
2728.39999999996	75.3600879604178\\
2728.70000000001	75.355467604808\\
2729	75.3581766882855\\
2729.09999999999	75.3554155063755\\
2729.30000000002	75.3572213673335\\
2729.50000000004	75.3516998827667\\
2729.69999999996	75.3535057513371\\
2729.8	75.3507454485512\\
2730.10000000002	75.3534539594169\\
2730.29999999997	75.3479343685907\\
2730.90000000002	75.3533504210912\\
2730.99999999997	75.350591336824\\
2731.40000000003	75.3542009884679\\
2731.49999999997	75.3514423780934\\
2731.99999999999	75.3559532959994\\
2732.30000000004	75.3476796955058\\
2732.89999999997	75.3530918404684\\
2732.99999999998	75.3503347846767\\
2733.19999999996	75.3521384407127\\
2733.50000000004	75.3438688908399\\
2735.59999999999	75.3627956281756\\
2735.69999999999	75.3600409386651\\
2736.09999999998	75.3526788977414\\
2736.29999999996	75.3508259026458\\
2736.79999999996	75.3553290218861\\
2736.90000000005	75.3525758129339\\
2737.10000000005	75.35437673535\\
2737.30000000004	75.3488711916417\\
2737.50000000004	75.3506721215663\\
2737.59999999997	75.3479197866823\\
2738	75.3515211277893\\
2738.09999999996	75.3487692644803\\
2739.29999999996	75.3595677885668\\
2739.50000000002	75.3540662870492\\
2740.30000000003	75.3612611297621\\
2740.39999999998	75.3585112205802\\
2740.79999999997	75.3621073370061\\
2740.90000000002	75.3593578985772\\
2741.10000000004	75.3611556982344\\
2741.29999999998	75.3556576931495\\
2742.19999999999	75.3637457608577\\
2742.30000000002	75.3609976662777\\
2742.49999999996	75.3627944286443\\
2742.60000000005	75.3600466693404\\
2743.30000000004	75.366333746446\\
2743.39999999998	75.3635866593767\\
2743.90000000001	75.3680758017493\\
2744.09999999999	75.3625829021208\\
2744.30000000001	75.3643783704999\\
2744.50000000003	75.3588865408438\\
2744.69999999997	75.3570387642087\\
2745.19999999997	75.3615269733727\\
2745.30000000003	75.3587819625555\\
2745.5	75.3605769230769\\
2745.60000000001	75.3578322467859\\
2745.89999999997	75.360524399126\\
2746.10000000004	75.3550360498143\\
2746.60000000004	75.359522335894\\
2746.79999999999	75.3540354581528\\
2747.10000000002	75.3494467093768\\
2747.80000000003	75.3557261909094\\
2748.00000000001	75.3502419853717\\
2748.89999999999	75.3583121134958\\
2748.99999999998	75.3555709141173\\
2749.29999999997	75.3582599839965\\
2749.40000000002	75.3555191853064\\
2749.79999999996	75.359103967417\\
2749.90000000004	75.3563636363636\\
2750.60000000004	75.3626349656451\\
2750.69999999999	75.3598953031845\\
2751.29999999997	75.3652685905357\\
2751.4	75.3625295293476\\
2752.00000000002	75.3679008756949\\
2752.09999999996	75.3651624155221\\
2752.50000000003	75.3578434934244\\
2753.20000000004	75.3641085243163\\
2753.29999999996	75.3613713953657\\
2753.90000000002	75.3667392883079\\
2753.99999999997	75.3640027595222\\
2755.30000000004	75.3756260434057\\
2755.40000000003	75.3728905824714\\
2755.60000000002	75.3746779402693\\
2755.70000000004	75.3719428115248\\
2755.89999999996	75.3701015965167\\
2756.39999999997	75.3745692000726\\
2756.50000000004	75.3718348690416\\
2757.10000000005	75.3771942550413\\
2757.19999999999	75.3744605229754\\
2757.40000000002	75.3726201269266\\
2758.19999999996	75.3797628974368\\
2758.39999999999	75.3742976255211\\
2758.80000000003	75.3669940918482\\
2758.99999999998	75.3687796745316\\
2759.09999999996	75.3660481298927\\
2759.4	75.3687262185178\\
2759.50000000004	75.3659950717495\\
2759.7	75.3677802739329\\
2759.80000000002	75.3650494583137\\
2760.39999999999	75.3704039123347\\
2760.5	75.3676736941245\\
2760.89999999996	75.3712423035132\\
2761.29999999997	75.3603244730934\\
2761.50000000003	75.3621089223638\\
2761.59999999997	75.3593800919723\\
2761.8	75.357543719903\\
2762.50000000004	75.3637877361905\\
2762.59999999997	75.3610598327723\\
2762.89999999997	75.3637350705755\\
2763.00000000001	75.361007563968\\
2763.3	75.3564449591083\\
2763.50000000003	75.354609929078\\
2763.79999999996	75.3572849958392\\
2763.89999999997	75.3545586107091\\
2764.89999999997	75.3634719710669\\
2764.99999999998	75.3607464467831\\
2765.60000000001	75.3660917670029\\
2765.70000000001	75.3633668378046\\
2766.10000000003	75.3669293615791\\
2766.19999999996	75.3642048946246\\
2766.4	75.3659859027652\\
2766.50000000005	75.3632617653438\\
2766.89999999998	75.3668232743043\\
2767	75.364099598858\\
2767.30000000004	75.3667702536677\\
2767.39999999999	75.3640469738031\\
2767.80000000003	75.3676072112432\\
2767.89999999997	75.3648843930636\\
2768.29999999997	75.3684438664933\\
2768.39999999999	75.3657215098429\\
2768.80000000002	75.3692802195818\\
2768.90000000003	75.3665583243048\\
2769.3	75.3701162706724\\
2769.39999999999	75.3673948366131\\
2769.6	75.3655630573708\\
2770.09999999996	75.3700093856039\\
2770.29999999998	75.3645682933872\\
2770.50000000003	75.366346639717\\
2770.60000000002	75.3636265203739\\
2770.89999999997	75.3662937567665\\
2771.00000000001	75.3635740319729\\
2771.29999999997	75.3662408890813\\
2771.39999999999	75.3635215587227\\
2771.60000000001	75.3616913807411\\
2772.20000000002	75.3670237708762\\
2772.39999999998	75.3615870153291\\
2772.79999999996	75.3543221897652\\
2772.99999999998	75.3524935992211\\
2773.60000000001	75.3578252875221\\
2773.70000000001	75.355108515394\\
2773.9	75.3568853640952\\
2774.00000000002	75.3541689196496\\
2774.19999999998	75.3523411310961\\
2774.40000000003	75.3541178590737\\
2774.59999999998	75.3486863444697\\
2774.80000000003	75.3468593462827\\
2775	75.3486360851861\\
2775.10000000004	75.3459210147016\\
2775.3	75.3440945449305\\
2775.49999999996	75.3422683383773\\
2776.49999999996	75.3511488871281\\
2776.60000000001	75.3484351928548\\
2776.90000000001	75.3510983075261\\
2776.99999999999	75.3483850059415\\
2777.20000000003	75.3501602275591\\
2777.30000000002	75.3474472528264\\
2777.59999999999	75.3501098030745\\
2777.70000000002	75.3473972208222\\
2778.20000000002	75.3518338552352\\
2778.29999999996	75.3491217967175\\
2778.49999999999	75.3472972000288\\
2778.69999999999	75.3490715416727\\
2778.79999999999	75.3463600705315\\
2779.19999999999	75.3499082502789\\
2779.5	75.3417757950784\\
2780.70000000002	75.352416570771\\
2780.79999999996	75.3497069294113\\
2780.99999999996	75.3514796303621\\
2781.09999999996	75.348770314972\\
2781.30000000001	75.3505428920687\\
2781.4	75.3478339025706\\
2781.89999999997	75.352264557872\\
2782.09999999998	75.3468478182733\\
2782.80000000003	75.3530489776851\\
2782.89999999999	75.3503413582465\\
2783.19999999998	75.3458125247009\\
2783.80000000001	75.3511261180359\\
2784.00000000003	75.345713156855\\
2784.59999999999	75.3510252450892\\
2784.79999999996	75.3456138460986\\
2785.00000000002	75.3473842949984\\
2785.09999999998	75.3446790176648\\
2785.49999999996	75.3374497415279\\
2785.70000000004	75.3356306985426\\
2786.20000000003	75.3400567060259\\
2786.30000000003	75.3373528567327\\
2786.69999999996	75.3408927802497\\
2786.80000000005	75.3381893860562\\
2786.99999999997	75.3399590972696\\
2787.30000000003	75.331850469972\\
2787.49999999997	75.3336203185536\\
2787.69999999998	75.328215797403\\
2788.10000000002	75.3209956244172\\
2788.59999999996	75.3254204468032\\
2788.70000000003	75.3227194492255\\
2789.70000000002	75.3315649867374\\
2789.80000000001	75.328864833865\\
2790.00000000004	75.330633310634\\
2790.10000000002	75.3279334814709\\
2790.3	75.3297018348624\\
2790.40000000004	75.3270023293317\\
2790.89999999996	75.3314224292368\\
2791.19999999997	75.3233260487945\\
2791.90000000005	75.3295128939828\\
2792.00000000002	75.3268149421582\\
2793.39999999998	75.3391802398425\\
2793.49999999997	75.3364833906071\\
2793.90000000001	75.3400143163923\\
2793.99999999999	75.3373179199026\\
2794.20000000001	75.3390831335218\\
2794.3	75.336387059834\\
2796.09999999999	75.3522637865675\\
2796.20000000003	75.3495690734184\\
2796.79999999998	75.3548571632879\\
2796.90000000005	75.3521630318198\\
2797.40000000004	75.3565683646113\\
2797.60000000005	75.351181327519\\
2798.09999999996	75.3555857336859\\
2798.30000000004	75.3502001143511\\
2798.50000000001	75.3519616951333\\
2798.60000000004	75.3492693036053\\
2799.39999999999	75.3563136274335\\
2799.49999999999	75.3536219459923\\
2799.79999999997	75.3562627236687\\
2799.89999999999	75.3535714285714\\
2800.30000000004	75.3463790887016\\
2800.60000000003	75.3490198878852\\
2800.70000000003	75.3463296201085\\
2801.29999999998	75.351609909331\\
2801.39999999999	75.34892022131\\
2801.60000000002	75.3471106827997\\
2801.79999999997	75.3488704093651\\
2801.90000000001	75.3461812990721\\
2802.20000000003	75.3488206116404\\
2802.40000000001	75.3434433541481\\
2802.80000000001	75.3469620749938\\
2802.89999999996	75.3442739921513\\
2803.3	75.3477919668973\\
2803.59999999997	75.3397296429718\\
2803.8	75.3414886408217\\
2803.9	75.3388017118402\\
2804.5	75.3440775868216\\
2804.60000000001	75.3413912361393\\
2804.8	75.3431494883953\\
2804.99999999999	75.3377776193362\\
2805.19999999997	75.3359711973764\\
2805.60000000001	75.3394874719321\\
2805.69999999997	75.3368023380141\\
2805.9	75.3385602280827\\
2806.00000000003	75.335875414276\\
2806.59999999997	75.3411479673638\\
2806.7	75.3384637309391\\
2806.90000000001	75.3366583541147\\
2807.4	75.3410507569012\\
2807.60000000004	75.3356840118246\\
2807.8	75.33387941166\\
2807.99999999998	75.3320750685517\\
2808.60000000001	75.337344679033\\
2808.70000000001	75.3346624893193\\
2808.99999999999	75.3372966430529\\
2809.10000000001	75.3346148369643\\
2809.30000000002	75.3363707553214\\
2809.40000000004	75.3336892685531\\
2809.59999999996	75.3318859664733\\
2809.80000000002	75.3336417666109\\
2809.89999999996	75.3309608540925\\
2810.50000000002	75.3362271401124\\
2810.70000000005	75.3308666571795\\
2811.10000000001	75.3343767785999\\
2811.30000000003	75.3290175713168\\
2811.49999999999	75.3272158201736\\
2812.19999999996	75.3333570387228\\
2812.5	75.3253217663372\\
2812.90000000001	75.3288304301458\\
2813.00000000001	75.3261526429917\\
2813.19999999997	75.3243521842676\\
2813.39999999997	75.326106273325\\
2813.49999999998	75.323429058857\\
2813.99999999998	75.3278135105362\\
2814.1	75.3251368061972\\
2814.70000000001	75.3303964757709\\
2814.89999999997	75.3250444049734\\
2815.10000000003	75.3267973856209\\
2815.19999999997	75.3241217632224\\
2815.4	75.3258746226248\\
2815.49999999996	75.323199318085\\
2815.90000000005	75.3267045454545\\
2816.00000000003	75.3240296864458\\
2816.79999999998	75.3310376655188\\
2817	75.3256895388875\\
2817.20000000002	75.3238916693288\\
2817.5	75.3194207836457\\
2817.80000000002	75.3149508499237\\
2818.2	75.3184543873967\\
2818.29999999996	75.3157820039739\\
2818.89999999999	75.3210358283079\\
2819.09999999999	75.3156923950057\\
2819.49999999998	75.319194211945\\
2819.60000000002	75.3165230343654\\
2820.19999999998	75.321774279332\\
2820.30000000004	75.3191036732378\\
2820.50000000003	75.3173083741048\\
2821.30000000004	75.3243070815907\\
2821.4	75.3216374269006\\
2822.6	75.3321288128388\\
2822.69999999996	75.3294601105286\\
2822.90000000003	75.3312079348211\\
2823.00000000001	75.328539548723\\
2823.30000000004	75.3311610115464\\
2823.39999999998	75.3284930051355\\
2823.60000000001	75.3302404646386\\
2823.69999999996	75.3275727742758\\
2823.89999999998	75.3293201133145\\
2824.00000000001	75.3266527389257\\
2824.20000000003	75.3248592571611\\
2824.69999999997	75.3292268479184\\
2824.80000000004	75.3265602322206\\
2824.99999999997	75.3283069625854\\
2825.09999999997	75.325640662608\\
2825.6	75.3300067239976\\
2825.70000000002	75.3273409300021\\
2825.90000000001	75.3290870488323\\
2826.00000000003	75.326421570362\\
2826.29999999998	75.3290404755166\\
2826.40000000002	75.3263753759066\\
2827.6	75.3368462000919\\
2827.70000000002	75.3341820496499\\
2829.19999999998	75.3472590393384\\
2829.30000000003	75.3445960274263\\
2829.49999999998	75.3463387051173\\
2829.60000000003	75.3436760080574\\
2830.39999999996	75.3506447624095\\
2830.49999999997	75.3479827598389\\
2830.9	75.3514659131049\\
2831.00000000002	75.3488043516654\\
2831.69999999997	75.3548979447701\\
2831.89999999999	75.3495762711864\\
2832.30000000004	75.3424657534247\\
2832.49999999998	75.344206735861\\
2832.59999999996	75.3415469340205\\
2832.79999999998	75.3397578453175\\
2833.2	75.3326509723644\\
2833.7	75.3370033171007\\
2833.8	75.3343448957267\\
2834.40000000002	75.3395660610337\\
2834.50000000004	75.3369082057433\\
2835.10000000001	75.3421275395034\\
2835.29999999997	75.3368131480567\\
2835.89999999997	75.3420310296192\\
2835.99999999998	75.339374493142\\
2836.29999999996	75.334931603441\\
2836.59999999998	75.3375400994113\\
2836.7	75.3348843767625\\
2836.90000000002	75.3330983433204\\
2837.30000000004	75.336575738352\\
2837.39999999997	75.3339207048458\\
2837.9	75.338266384778\\
2838.00000000004	75.3356118530002\\
2838.19999999996	75.3338265863369\\
2838.70000000003	75.3381710581936\\
2838.80000000002	75.3355172778189\\
2839.00000000002	75.3372547638336\\
2839.1	75.3346012961398\\
2839.80000000004	75.3406810098947\\
2839.89999999996	75.3380281690141\\
2840.50000000005	75.343237344223\\
2840.60000000001	75.3405850670609\\
2840.80000000002	75.3388010841635\\
2841	75.3370173524339\\
2841.69999999997	75.3430924062214\\
2841.80000000002	75.3404412540906\\
2842.39999999997	75.3456464379947\\
2842.50000000001	75.3429958488708\\
2843.19999999998	75.3490662258643\\
2843.29999999998	75.3464162622213\\
2843.70000000005	75.3498839580842\\
2843.89999999998	75.3445850914205\\
2844.09999999996	75.346318824274\\
2844.2	75.3436697957318\\
2844.40000000003	75.3454034100896\\
2844.5	75.3427546931027\\
2845.00000000002	75.3470879758181\\
2845.09999999998	75.3444397581892\\
2845.80000000001	75.3505042341614\\
2845.89999999996	75.3478566408995\\
2846.30000000002	75.3513209668353\\
2846.39999999998	75.348673809942\\
2846.60000000003	75.3468928935258\\
2846.79999999999	75.345112227335\\
2847.00000000002	75.3468441572126\\
2847.10000000004	75.3441978083731\\
2847.29999999998	75.3459296200042\\
2847.49999999999	75.3406377300183\\
2847.69999999997	75.3423695484233\\
2847.79999999999	75.3397240071632\\
2848.00000000004	75.3414557073136\\
2848.19999999997	75.3361654320121\\
2848.59999999998	75.3396286025204\\
2848.80000000002	75.3343395696585\\
2849.79999999999	75.3429944910348\\
2849.99999999998	75.3377074488615\\
2850.79999999996	75.3446280122067\\
2850.89999999996	75.3419852683269\\
2851.29999999998	75.3454443431297\\
2851.39999999997	75.3428020340172\\
2852.00000000004	75.3479892009397\\
2852.10000000001	75.3453474510904\\
2852.30000000002	75.347076146403\\
2852.40000000004	75.3444347063979\\
2853.40000000001	75.3530751708428\\
2853.6	75.3477940918807\\
2853.79999999996	75.3460177301237\\
2854	75.3442416173225\\
2854.50000000002	75.3485602185946\\
2854.60000000001	75.3459207622517\\
2854.8	75.3441451539458\\
2854.99999999997	75.342369794403\\
2855.19999999998	75.3440969425279\\
2855.3	75.3414582895566\\
2855.59999999998	75.344048744616\\
2855.70000000004	75.3414104629176\\
2855.90000000002	75.343137254902\\
2856.00000000002	75.340499282238\\
2856.20000000002	75.3387249238525\\
2857.1	75.3464930701386\\
2857.19999999997	75.3438560879152\\
2857.69999999999	75.3481699209182\\
2857.79999999998	75.3455334336401\\
2858.00000000001	75.3437598404534\\
2858.20000000004	75.3454850785432\\
2858.29999999997	75.3428491463756\\
2858.99999999996	75.3279003882341\\
2859.89999999999	75.3356643356643\\
2860	75.3330303136254\\
2860.4	75.3364796364272\\
2860.49999999998	75.333846046284\\
2861.09999999996	75.3390185935971\\
2861.20000000003	75.3363855590116\\
2862	75.3432794102233\\
2862.10000000003	75.3406470547132\\
2862.50000000001	75.3440927827849\\
2862.60000000001	75.34146085863\\
2862.79999999996	75.3431834852772\\
2862.89999999998	75.3405518686692\\
2863.20000000001	75.343135542905\\
2863.49999999997	75.3352423522838\\
2863.80000000002	75.3308425573519\\
2864.29999999997	75.3351487222455\\
2864.40000000002	75.3325187641822\\
2865.59999999995	75.3428481697317\\
2865.70000000004	75.3402191360179\\
2866.09999999997	75.343660595911\\
2866.2	75.3410319924642\\
2866.60000000003	75.3444727386891\\
2866.70000000002	75.3418445653691\\
2866.89999999996	75.3400767352633\\
2867.19999999996	75.3356816517281\\
2867.40000000003	75.3374019180471\\
2867.50000000004	75.3347747245083\\
2868.39999999999	75.3425135088025\\
2868.50000000001	75.3398870529178\\
2869.19999999996	75.3459031819608\\
2869.49999999999	75.338026205743\\
2869.69999999998	75.3397449299603\\
2869.89999999997	75.3344947735192\\
2870.7	75.3413682597185\\
2870.80000000001	75.3387439478909\\
2870.99999999997	75.3369788582773\\
2872.20000000004	75.3472826654597\\
2872.60000000003	75.3367911720681\\
2873.29999999996	75.34279947101\\
2873.4	75.3401774839047\\
2873.90000000001	75.3444676409186\\
2874.09999999998	75.3392248277782\\
2874.30000000003	75.3374617311439\\
2875.09999999998	75.3443238731219\\
2875.20000000003	75.3417034744201\\
2875.80000000003	75.3468479432526\\
2875.9	75.3442280945758\\
2876.59999999996	75.350227691452\\
2876.69999999999	75.3476084538376\\
2877.39999999997	75.3536055603823\\
2877.50000000001	75.3509869335557\\
2877.69999999998	75.3492251025089\\
2878.29999999999	75.3543635352974\\
2878.39999999999	75.3517457008859\\
2878.60000000004	75.3499843679439\\
2878.79999999996	75.3516968286498\\
2878.99999999998	75.346462436178\\
2879.49999999998	75.350743158772\\
2879.6	75.3481265409591\\
2879.79999999997	75.3463661932706\\
2880.10000000001	75.3489341017985\\
2880.3	75.3437022635745\\
2880.7	75.3471257983893\\
2880.79999999998	75.3445103960568\\
2881.00000000004	75.3462219291243\\
2881.10000000003	75.3436068304873\\
2882.30000000003	75.3538717735221\\
2882.50000000001	75.3486435856518\\
2882.79999999997	75.3512088521974\\
2882.90000000002	75.3485952133195\\
2883.19999999996	75.3442236326431\\
2883.49999999997	75.3467887363018\\
2883.60000000004	75.3441758851476\\
2883.90000000002	75.3398058252427\\
2884.40000000003	75.3440804298839\\
2884.60000000004	75.3388567268693\\
2884.80000000001	75.3371000727928\\
2885	75.3388097466292\\
2885.09999999998	75.3361985304312\\
2885.90000000004	75.3430353430354\\
2885.99999999998	75.3404247947057\\
2886.29999999998	75.342987804878\\
2886.39999999999	75.340377619955\\
2886.90000000004	75.3446484239695\\
2887.00000000004	75.3420387239791\\
2887.40000000002	75.3454545454545\\
2887.49999999998	75.3428452694279\\
2888.4	75.3505279556863\\
2888.50000000003	75.3479194073253\\
2889.50000000001	75.3564507198228\\
2889.59999999997	75.3538429594768\\
2889.80000000003	75.3555486349009\\
2889.90000000003	75.3529411764706\\
2890.09999999998	75.3511867690817\\
2890.60000000002	75.3554502369668\\
2890.69999999996	75.3528435035284\\
2891.80000000001	75.362218610602\\
2892.09999999996	75.3544014936726\\
2892.29999999997	75.3526483197345\\
2892.50000000002	75.354352485653\\
2892.60000000001	75.3517475023335\\
2893.00000000001	75.3551553696727\\
2893.09999999996	75.352550808793\\
2893.90000000001	75.3593642017968\\
2893.99999999999	75.356760305449\\
2894.20000000003	75.3584631862627\\
2894.29999999999	75.3558595909342\\
2894.99999999998	75.3618182446202\\
2895.09999999997	75.3592152528323\\
2895.30000000001	75.3609173171237\\
2895.50000000002	75.355712114933\\
2895.79999999997	75.3582651334646\\
2895.89999999998	75.3556629834254\\
2896.30000000003	75.3590664272891\\
2896.4	75.3564646987744\\
2896.69999999999	75.3590168461751\\
2896.80000000003	75.3564154786151\\
2896.99999999997	75.3581167374271\\
2897.10000000004	75.3555156703024\\
2897.39999999998	75.351164797239\\
2897.60000000005	75.3494150533182\\
2897.80000000004	75.3511163256151\\
2897.89999999999	75.3485162180814\\
2898.30000000002	75.3519182997516\\
2898.39999999999	75.3493186130757\\
2899.10000000005	75.355270419426\\
2899.19999999998	75.35267133446\\
2899.8	75.3577709576192\\
2900.00000000003	75.3525740491707\\
2900.60000000004	75.3576722859999\\
2900.70000000002	75.3550744622173\\
2900.90000000004	75.3567735263702\\
2900.99999999998	75.3541760022061\\
2901.19999999998	75.3558749526075\\
2901.39999999999	75.3506806824057\\
2901.69999999999	75.346336756496\\
2902.60000000001	75.3539807765184\\
2902.70000000005	75.3513848697809\\
2902.99999999999	75.3470428162998\\
2903.40000000001	75.3504391251937\\
2903.49999999997	75.347844055655\\
2903.79999999999	75.3435035641723\\
2904.1	75.3391639694236\\
2904.30000000005	75.3374190882799\\
2904.50000000003	75.3356744474282\\
2905.00000000001	75.3399194519982\\
2905.09999999999	75.3373261737574\\
2905.50000000002	75.3303964757709\\
2907.49999999996	75.3473655248315\\
2907.60000000001	75.344774220174\\
2909.70000000002	75.3625678740807\\
2909.8	75.359978006117\\
2911.20000000001	75.3718270188575\\
2911.30000000001	75.3692381672048\\
2911.79999999999	75.3734674954497\\
2912.10000000003	75.3657029050203\\
2912.60000000004	75.3699316785113\\
2912.69999999997	75.3673441362263\\
2912.90000000001	75.3656024716787\\
2913.19999999998	75.3681392235609\\
2913.39999999996	75.3629655054059\\
2914.19999999995	75.3697285797619\\
2914.40000000004	75.3645565277063\\
2914.59999999998	75.3628160702645\\
2914.90000000001	75.3653516295026\\
2915.00000000002	75.3627662858907\\
2916.4	75.3711640665181\\
2916.49999999998	75.3685798532538\\
2916.69999999996	75.3668403730115\\
2917.60000000002	75.3744387702642\\
2917.79999999998	75.3692724219473\\
2918.20000000003	75.3726484597197\\
2918.29999999999	75.3700657894737\\
2918.90000000004	75.3682768071257\\
2919.90000000003	75.3767123287671\\
2920.00000000004	75.3741310229102\\
2920.79999999996	75.380875757472\\
2920.89999999999	75.3782951044163\\
2921.59999999999	75.3807714686655\\
2921.79999999999	75.3756117594716\\
2922.79999999996	75.3840364022033\\
2922.89999999997	75.3814574067739\\
2923.09999999998	75.3797208538588\\
2923.3	75.377984538551\\
2924.79999999996	75.3906116448426\\
2924.90000000004	75.3880341880342\\
2926.10000000004	75.3981272640284\\
2926.19999999996	75.3955506954174\\
2926.80000000004	75.4005944856333\\
2926.89999999998	75.3980184489238\\
2927.39999999998	75.3953885567891\\
2927.9	75.3893442622951\\
2928.20000000001	75.3918655875423\\
2928.29999999998	75.3892910804535\\
2928.59999999998	75.384983098303\\
2929.69999999997	75.3942248617653\\
2930.19999999998	75.3813602702795\\
2931.1	75.3889192139738\\
2931.20000000001	75.3863473544161\\
2932.10000000001	75.3904917809154\\
2932.30000000005	75.385349884054\\
2933.49999999997	75.3954185983092\\
2933.59999999996	75.3928486211951\\
2934.1	75.3970417831095\\
2934.20000000003	75.3944722761817\\
2934.39999999999	75.3927415232578\\
2934.7	75.3884421425651\\
2937.10000000001	75.4085523627945\\
2937.2	75.4059850883464\\
2937.49999999995	75.4084967320261\\
2937.90000000004	75.3982300884956\\
2938.39999999998	75.4024161987409\\
2938.5	75.3998502688355\\
2938.79999999997	75.4023614277451\\
2938.89999999999	75.3997958489282\\
2939.39999999998	75.4039802687532\\
2939.60000000004	75.3988502228119\\
2939.79999999996	75.3971223511004\\
2940.19999999996	75.4004693398633\\
2940.29999999996	75.3979050469324\\
2940.59999999999	75.39361376543\\
2940.80000000002	75.3918868373627\\
2941.09999999997	75.3943968448252\\
2941.19999999999	75.3918335429912\\
2941.60000000004	75.3951796580209\\
2941.99999999999	75.3849291322525\\
2942.19999999995	75.3832036162186\\
2942.80000000004	75.3882225016141\\
2942.90000000005	75.385660890248\\
2943.20000000003	75.3813746475045\\
2943.39999999999	75.3796500764396\\
2944.29999999998	75.3769868224426\\
2944.8	75.3811674420184\\
2944.90000000001	75.3786078098472\\
2945.09999999997	75.3768844221106\\
2945.39999999999	75.3793922933288\\
2945.70000000001	75.3717156629778\\
2946.4	75.377566604446\\
2946.60000000003	75.3724505378898\\
2946.89999999996	75.3749575839837\\
2947.10000000001	75.3698425624321\\
2947.69999999999	75.3748558246828\\
2947.79999999999	75.3722989246582\\
2947.99999999996	75.3705776601879\\
2948.30000000002	75.3663003663004\\
2948.79999999996	75.3704771270643\\
2948.89999999996	75.3679213292641\\
2949.40000000003	75.3619257501271\\
2950.10000000001	75.3643820757915\\
2950.29999999998	75.359273318872\\
2950.60000000001	75.3617785610194\\
2950.89999999997	75.3541172483904\\
2951.30000000003	75.3506810327302\\
2951.50000000004	75.3489632741564\\
2952.19999999999	75.3548081157064\\
2952.29999999996	75.3522557918981\\
2952.80000000003	75.3564292729181\\
2952.90000000005	75.3538774128005\\
2953.10000000003	75.3521603684139\\
2953.70000000004	75.3571670390683\\
2953.79999999996	75.3546159314804\\
2954.00000000004	75.3528993602112\\
2954.3	75.355402112104\\
2954.40000000002	75.352851582332\\
2954.99999999997	75.3578559101215\\
2955.1	75.3553059014618\\
2955.39999999999	75.3578074775842\\
2955.50000000003	75.3552578156719\\
2955.80000000001	75.3509929293954\\
2955.99999999999	75.3492777646223\\
2956.19999999999	75.3475628319183\\
2956.40000000002	75.3458481312363\\
2956.59999999998	75.3441336625292\\
2956.79999999998	75.3424194257499\\
2957.19999999995	75.3389916477868\\
2957.80000000005	75.3439940498326\\
2957.90000000003	75.341446923597\\
2958.10000000002	75.339733621797\\
2958.29999999997	75.3380205516495\\
2959	75.3438545503701\\
2959.09999999996	75.3413084617464\\
2959.90000000003	75.347972972973\\
2960.29999999998	75.3377921902446\\
2960.6	75.3402911473638\\
2960.69999999996	75.3377465549851\\
2961.39999999997	75.343575890596\\
2961.50000000001	75.3410318746623\\
2961.89999999996	75.3443619176232\\
2961.99999999996	75.3418183045812\\
2962.39999999996	75.3451476793249\\
2962.50000000003	75.3426044690475\\
2962.8	75.3383509399575\\
2963.19999999996	75.3349306516384\\
2963.70000000004	75.3289695660976\\
2964.19999999998	75.3331309246702\\
2964.29999999999	75.330589664013\\
2964.70000000003	75.3339179708581\\
2964.90000000003	75.3288364249578\\
2965.30000000003	75.3321642948675\\
2965.39999999999	75.3296240094419\\
2966.59999999998	75.3362321771666\\
2966.7	75.3336928677363\\
2967.50000000002	75.3302331850654\\
2967.90000000001	75.3335579514825\\
2968.00000000005	75.3310198443449\\
2968.60000000003	75.3225317479031\\
2968.99999999997	75.325856320097\\
2969.09999999996	75.3233194126364\\
2969.29999999997	75.3216137940325\\
2970.00000000003	75.3274300528602\\
2970.19999999997	75.3223580109753\\
2970.89999999999	75.3180747223157\\
2971.39999999996	75.3121319199058\\
2971.60000000004	75.3104283743312\\
2972.40000000005	75.3170731707317\\
2972.5	75.314539460405\\
2972.79999999997	75.3170305089307\\
2973.1	75.3094309161846\\
2974.00000000001	75.313540230658\\
2974.20000000003	75.3084759439196\\
2974.39999999998	75.3067742477727\\
2974.80000000003	75.3100944569565\\
2974.90000000003	75.3075630252101\\
2975.69999999997	75.3007594596411\\
2976.1	75.297359048451\\
2976.40000000004	75.2998488157232\\
2976.60000000002	75.2947895320321\\
2977.09999999998	75.2989386000269\\
2977.19999999996	75.2964094985389\\
2977.49999999996	75.298898441698\\
2977.60000000004	75.2963696812976\\
2978.10000000003	75.2938016251427\\
2978.70000000002	75.2987780314221\\
2978.89999999999	75.2937227257469\\
2979.50000000004	75.2852731910324\\
2979.79999999995	75.2810496996543\\
2980.50000000004	75.2868549956385\\
2980.60000000005	75.2843291844198\\
2981.59999999996	75.2892645135325\\
2981.70000000003	75.28673955329\\
2983.40000000005	75.3008211831741\\
2983.50000000004	75.2982973588953\\
2983.80000000001	75.3007808572673\\
2983.90000000003	75.2982573726541\\
2984.30000000003	75.3015681544029\\
2984.39999999996	75.2990450661752\\
2984.89999999998	75.3031825795645\\
2985.00000000005	75.3006599443905\\
2986.00000000003	75.3089313820703\\
2986.10000000004	75.3064094836247\\
2987.40000000005	75.3138075313807\\
2987.50000000004	75.3112866514928\\
2987.80000000001	75.3137655209344\\
2988.10000000001	75.306204403989\\
2988.39999999997	75.3086832859294\\
2988.5	75.3061634209998\\
2988.80000000002	75.3086419753086\\
2988.89999999997	75.3061224489796\\
2989.39999999996	75.3002174276635\\
2989.99999999998	75.3051737400087\\
2990.10000000002	75.3026553407799\\
2990.39999999999	75.2984450760742\\
2991.00000000005	75.3034000869245\\
2991.10000000002	75.3008825889275\\
2991.6	75.2983253668483\\
2992.29999999997	75.3041037294479\\
2992.40000000004	75.3015873015873\\
2992.9	75.3057133311059\\
2993.00000000001	75.303197353914\\
2993.39999999999	75.2998162685819\\
2994.20000000001	75.3030758441038\\
2994.29999999999	75.3005610472883\\
2994.89999999997	75.2988313856427\\
2995.60000000004	75.3012651467103\\
2995.70000000003	75.2987515855531\\
2996.70000000005	75.3069941270689\\
2996.79999999998	75.3044812973406\\
2997.70000000002	75.3118953899526\\
2997.79999999999	75.3093832349311\\
2998.40000000001	75.3076538269135\\
2999.09999999996	75.310082688717\\
2999.19999999996	75.3075717667456\\
3000	75.3141561947935\\
3000.1	75.311645890274\\
3000.69999999996	75.3165822447347\\
3000.89999999997	75.3115628123959\\
3002.79999999999	75.3171933797329\\
3002.89999999998	75.3146853146853\\
3003.60000000003	75.3171088990245\\
3003.69999999998	75.3146015047606\\
3004.1	75.3178882897277\\
3004.19999999998	75.3153812868222\\
3004.8	75.3136543645379\\
3005.40000000001	75.3119281317584\\
3005.79999999999	75.3152134136199\\
3005.90000000002	75.3127079174983\\
3006.3	75.3159925492283\\
3006.39999999999	75.3134874438716\\
3006.79999999998	75.3101200572018\\
3007.80000000003	75.3183284018751\\
3007.89999999998	75.3158244680851\\
3008.3	75.319106501795\\
3008.40000000004	75.3166029582849\\
3008.99999999999	75.3082316971852\\
3009.89999999999	75.3156146179402\\
3009.99999999997	75.3131125211787\\
3010.29999999998	75.308929045974\\
3010.9	75.2972434407174\\
3011.30000000001	75.293883243674\\
3011.89999999998	75.2988047808765\\
3012.10000000004	75.2938051922183\\
3012.80000000002	75.2962262272229\\
3012.90000000002	75.2937271822104\\
3014.20000000002	75.3043824436851\\
3014.30000000003	75.3018842887473\\
3014.70000000001	75.3051612047234\\
3014.79999999997	75.3026634382567\\
3015.20000000003	75.3059397074918\\
3015.29999999998	75.3034423293759\\
3015.60000000003	75.2992671684849\\
3015.89999999997	75.301724137931\\
3016.00000000001	75.299227479195\\
3016.40000000001	75.3025029007127\\
3016.49999999995	75.3000066299808\\
3017.00000000004	75.2974710815021\\
3017.59999999997	75.302382609272\\
3017.70000000004	75.2998873351448\\
3018.00000000003	75.2957158477188\\
3018.39999999998	75.2989895643532\\
3018.50000000002	75.2964950639369\\
3019.80000000005	75.3038180072188\\
3019.89999999999	75.3013245033113\\
3020.69999999996	75.3078654661017\\
3020.90000000004	75.3028798411122\\
3021.70000000002	75.3094182275465\\
3021.90000000001	75.3044341495698\\
3022.59999999998	75.3068448737883\\
3022.69999999999	75.3043535794627\\
3023.00000000003	75.3068042737587\\
3023.10000000004	75.3043133103996\\
3023.90000000001	75.3108465608466\\
3023.99999999996	75.308356205152\\
3024.30000000004	75.3108054490147\\
3024.39999999998	75.308315424037\\
3024.80000000002	75.3115805481173\\
3024.90000000001	75.3090909090909\\
3025.19999999997	75.3115393514693\\
3025.29999999999	75.3090500429695\\
3025.90000000004	75.3139458030403\\
3026.10000000001	75.3089683431366\\
3026.39999999996	75.3048075334545\\
3026.70000000001	75.3072551869962\\
3026.89999999998	75.3022794846383\\
3027.39999999996	75.2997522708505\\
3028.09999999997	75.3054619906215\\
3028.30000000004	75.3004887069079\\
3028.99999999997	75.302895249414\\
3029.09999999996	75.300409349003\\
3029.4	75.302855256643\\
3029.5	75.3003696857671\\
3030.9	75.3084790498185\\
3031	75.3059945234403\\
3031.6	75.2976877659399\\
3032.70000000004	75.3066473226062\\
3032.80000000003	75.3041643311682\\
3033.09999999996	75.3066068838191\\
3033.30000000002	75.30164172216\\
3033.60000000003	75.297491512015\\
3033.99999999996	75.3007481625523\\
3034.30000000004	75.2933034537306\\
3034.99999999997	75.2957068959837\\
3035.1	75.2932261465472\\
3035.7	75.298109229857\\
3035.80000000002	75.2956289732863\\
3036.19999999999	75.2988835095346\\
3036.29999999996	75.2964036358846\\
3037.40000000005	75.2987654320988\\
3037.59999999998	75.2938078151233\\
3038.19999999999	75.2986867656255\\
3038.29999999996	75.2962085308057\\
3038.60000000001	75.2986474479218\\
3038.70000000005	75.2961695406081\\
3039.19999999997	75.293653143816\\
3039.79999999999	75.2985295568933\\
3039.90000000004	75.2960526315789\\
3041.39999999997	75.3082360677297\\
3041.50000000001	75.3057601262493\\
3042.40000000001	75.3130649137223\\
3042.60000000003	75.3081145035659\\
3042.90000000001	75.303976339139\\
3044.3	75.3153330705558\\
3044.40000000005	75.3128592543932\\
3044.99999999999	75.3078716626712\\
3045.80000000001	75.3143570044979\\
3045.89999999997	75.311884438608\\
3046.59999999997	75.3142744608921\\
3046.70000000004	75.3118025469345\\
3047.19999999997	75.3158533784006\\
3047.30000000004	75.313381899324\\
3047.60000000005	75.3158119237458\\
3047.90000000002	75.3083989501312\\
3049.40000000001	75.3172651254304\\
3049.70000000002	75.3098563840252\\
3050.19999999999	75.3139035504704\\
3050.30000000001	75.3114345659586\\
3051.10000000005	75.3146303093865\\
3051.19999999998	75.312162029299\\
3052.30000000003	75.3177827283449\\
3052.4	75.3153153153153\\
3052.8	75.3119984277245\\
3053.20000000004	75.3152326990469\\
3053.29999999998	75.3127660968101\\
3053.89999999997	75.3176162409954\\
3053.99999999996	75.31515012606\\
3054.50000000001	75.3093694755451\\
3054.79999999995	75.3117941667485\\
3054.9	75.3093289689034\\
3055.2	75.3117533466435\\
3055.30000000002	75.3092884728677\\
3055.79999999999	75.3067835989398\\
3056.19999999998	75.3001995877368\\
3056.50000000002	75.3026238303998\\
3056.80000000001	75.2952337335209\\
3057.30000000005	75.2992738928501\\
3057.50000000001	75.2943485086342\\
3058.00000000001	75.2918478794022\\
3058.89999999998	75.2991173586139\\
3059.00000000003	75.2966558791802\\
3059.99999999997	75.3014607365772\\
3060.20000000002	75.2965395549456\\
3060.7	75.3005750130685\\
3060.8	75.2981149335163\\
3061.29999999998	75.2956163846606\\
3062.39999999996	75.3044897959184\\
3062.5	75.3020309540913\\
3063.30000000004	75.3084807729973\\
3063.40000000002	75.3060225232577\\
3064.29999999997	75.3034851847017\\
3065.59999999999	75.3106957627948\\
3065.69999999999	75.3082392850153\\
3066	75.3106552297707\\
3066.20000000002	75.3057430779767\\
3066.60000000005	75.3089640329996\\
3066.79999999995	75.3040529524928\\
3067.19999999998	75.3007531053369\\
3067.70000000002	75.2949996740335\\
3068.20000000004	75.2925072515725\\
3069.29999999998	75.2948458982212\\
3069.6	75.2874873766166\\
3070.30000000005	75.2931214174049\\
3070.39999999999	75.2906692721055\\
3070.70000000003	75.2865702748469\\
3071.29999999999	75.2913980595168\\
3071.40000000003	75.2889467686798\\
3072.70000000001	75.2928924759177\\
3072.79999999997	75.2904422532461\\
3073.09999999997	75.2928543537681\\
3073.20000000005	75.2904044512413\\
3073.60000000003	75.2871132511305\\
3074.09999999998	75.2911326523974\\
3074.19999999995	75.2886836027714\\
3074.60000000003	75.2918983965915\\
3074.7	75.289449720307\\
3075.4	75.2918224678914\\
3075.49999999998	75.2893744310053\\
3076.19999999997	75.2949972369405\\
3076.30000000001	75.2925497334547\\
3076.70000000005	75.2892615704628\\
3077	75.2916707289331\\
3077.09999999996	75.2892239698427\\
3077.50000000001	75.2859370938394\\
3077.99999999998	75.2899515935155\\
3078.20000000002	75.2850599356788\\
3079.49999999996	75.2954929211586\\
3079.59999999996	75.2930480241582\\
3080.50000000004	75.2970200610271\\
3080.80000000005	75.289688078159\\
3081.50000000004	75.2855659397715\\
3082.09999999996	75.2903770034391\\
3082.29999999996	75.2854918245523\\
3082.60000000004	75.2814091543128\\
3083.39999999997	75.2878222798768\\
3083.50000000005	75.2853807238293\\
3083.90000000002	75.2821011673152\\
3084.59999999997	75.2844685058515\\
3084.69999999999	75.2820280082988\\
3085.19999999998	75.2860337730529\\
3085.40000000004	75.2811537838276\\
3085.99999999995	75.2859596254172\\
3086.2	75.2810809059391\\
3086.49999999996	75.2834834445668\\
3086.59999999995	75.2810444811611\\
3087.29999999995	75.2866489602902\\
3087.4	75.2842105263158\\
3088.20000000004	75.2873749311919\\
3088.29999999999	75.2849371843025\\
3089.70000000001	75.2831898504758\\
3089.99999999996	75.2855894631242\\
3090.10000000001	75.283153193968\\
3090.69999999999	75.287951339459\\
3090.79999999995	75.285515545634\\
3092.00000000003	75.2951068852883\\
3092.10000000003	75.2926718840955\\
3092.40000000003	75.2950687146322\\
3092.49999999997	75.2926340296191\\
3092.90000000001	75.2958292919496\\
3093.1	75.2909608172766\\
3093.4	75.2933570389526\\
3093.69999999995	75.2860559829336\\
3094.89999999998	75.2956381260097\\
3094.99999999995	75.2932053891635\\
3095.60000000001	75.2979939916659\\
3095.70000000003	75.2955617287938\\
3096.00000000002	75.2979554923937\\
3096.20000000003	75.2930917546749\\
3096.49999999996	75.2890266744171\\
3097.30000000003	75.2954090527539\\
3097.49999999998	75.2905475206611\\
3098.09999999998	75.2953327738687\\
3098.39999999997	75.2880426012587\\
3099.30000000005	75.2919919984513\\
3099.5	75.2871338237192\\
3100.80000000004	75.2974942758554\\
3100.90000000002	75.2950661077072\\
3102.19999999998	75.2989717306515\\
3102.29999999997	75.296544610624\\
3102.70000000005	75.2932834858837\\
3103.50000000003	75.2964299523134\\
3103.59999999997	75.2940039307923\\
3103.89999999995	75.2963917525773\\
3103.99999999998	75.2939660449083\\
3105.09999999999	75.2962772124179\\
3105.30000000004	75.2914278353835\\
3106.29999999996	75.2993819211949\\
3106.40000000003	75.2969579913085\\
3107.70000000002	75.3040736212111\\
3107.79999999997	75.3016506322597\\
3108.30000000003	75.2959722043495\\
3108.59999999996	75.2983562260752\\
3108.79999999996	75.2935121747242\\
3110.3	75.3054269547325\\
3110.40000000001	75.3030059475968\\
3111.49999999997	75.3117367270857\\
3111.6	75.3093164508147\\
3112.40000000003	75.3156626506024\\
3112.6	75.310823400906\\
3113.49999999997	75.3179599177801\\
3113.69999999999	75.3131222300726\\
3114.30000000004	75.3178782430003\\
3114.39999999997	75.3154599454166\\
3114.70000000004	75.3178374213433\\
3114.80000000001	75.3154194356159\\
3115.19999999996	75.3185888999454\\
3115.29999999996	75.3161712781665\\
3115.59999999998	75.3185479988446\\
3115.70000000003	75.3161306887477\\
3116.09999999999	75.3128810731019\\
3116.80000000004	75.3184253585293\\
3116.90000000002	75.3160089829965\\
3117.80000000005	75.3199268738574\\
3118.00000000001	75.3150957313749\\
3118.39999999998	75.3182619849287\\
3118.49999999998	75.3158468543577\\
3118.79999999997	75.3182211677194\\
3118.89999999998	75.3158063481885\\
3119.39999999995	75.3197627824972\\
3119.5	75.3173483779972\\
3119.80000000002	75.3197217859547\\
3119.99999999998	75.3148937534053\\
3120.59999999996	75.3196398243984\\
3120.70000000001	75.3172263522174\\
3121.10000000003	75.3139818018711\\
3121.50000000001	75.3107380830344\\
3122.1	75.3154826724745\\
3122.29999999998	75.3106584678452\\
3123.70000000004	75.3025161662078\\
3124.39999999996	75.3080492878861\\
3124.59999999998	75.3032291099946\\
3124.89999999999	75.3056\\
3125	75.3031902979105\\
3125.39999999996	75.3063509838426\\
3125.49999999998	75.3039416432045\\
3125.99999999997	75.307891622149\\
3126.10000000001	75.3054826946452\\
3126.59999999999	75.309431669172\\
3126.80000000002	75.3046147942051\\
3127.20000000001	75.3077734787197\\
3127.39999999997	75.3029576338929\\
3127.90000000005	75.3005115089514\\
3128.60000000004	75.3060376514207\\
3128.70000000002	75.3036307849655\\
3129.89999999997	75.3130990415335\\
3130.09999999996	75.3082870104147\\
3130.40000000005	75.3106532502795\\
3130.50000000004	75.3082476202645\\
3130.89999999998	75.305014372405\\
3131.79999999996	75.3089179092564\\
3131.90000000003	75.3065134099617\\
3132.89999999996	75.3048196616661\\
3133.40000000002	75.3087601723313\\
3133.49999999998	75.3063569057952\\
3134.70000000002	75.2966696439964\\
3135.50000000004	75.3029723178977\\
3135.60000000001	75.3005708454253\\
3136	75.2941551608686\\
3137.00000000004	75.2988428803672\\
3137.09999999999	75.296442687747\\
3140.30000000003	75.3152464654184\\
3140.49999999999	75.3104502324397\\
3140.80000000005	75.3128084307046\\
3140.90000000003	75.3104106972302\\
3141.60000000005	75.3127287774135\\
3141.70000000005	75.3103316570119\\
3142	75.3126889659782\\
3142.09999999998	75.3102921519954\\
3142.39999999995	75.3062848050915\\
3142.79999999998	75.3030640491266\\
3143.69999999995	75.3005916406896\\
3144.00000000004	75.3029483795045\\
3144.10000000005	75.300553399911\\
3144.99999999996	75.3076213792884\\
3145.09999999998	75.3052270125906\\
3146.39999999998	75.3154298426823\\
3146.50000000003	75.3130362931418\\
3146.89999999996	75.316174134096\\
3147.00000000005	75.3137809411839\\
3147.30000000005	75.3161339518333\\
3147.40000000005	75.3137410643368\\
3148.10000000003	75.3192300362112\\
3148.20000000003	75.3168376584188\\
3148.50000000003	75.3128374515658\\
3149.60000000001	75.3214591865892\\
3149.69999999996	75.3190678773255\\
3151.09999999998	75.3300330033003\\
3151.19999999999	75.3276425602132\\
3151.50000000005	75.323645132631\\
3152.30000000005	75.3299073721609\\
3152.39999999997	75.3275178429818\\
3152.90000000005	75.325087218522\\
3153.29999999996	75.3282171624279\\
3153.39999999998	75.3258284445854\\
3153.9	75.3233988585923\\
3154.50000000002	75.3280923096431\\
3154.59999999996	75.3257045043903\\
3154.90000000005	75.3280507131537\\
3155.00000000001	75.3256632119426\\
3155.30000000005	75.3280091272105\\
3155.4	75.3256219299636\\
3155.69999999996	75.3279675518094\\
3155.8	75.3255806584493\\
3156.20000000005	75.328707664037\\
3156.30000000002	75.3263211253327\\
3156.79999999999	75.330229022142\\
3156.90000000002	75.3278428888185\\
3157.60000000002	75.3333122209203\\
3157.69999999997	75.3309265944645\\
3158.80000000004	75.339516920447\\
3158.9	75.3371320037987\\
3159.50000000001	75.3418154196734\\
3159.60000000001	75.3394309586353\\
3160.20000000001	75.344112900674\\
3160.50000000002	75.3369613364551\\
3160.99999999997	75.3408623580399\\
3161.1	75.3384790585853\\
3161.7	75.343158960086\\
3161.80000000003	75.3407761156267\\
3162.70000000001	75.3351460730998\\
3163.8	75.3247574196403\\
3164.10000000005	75.3270968965299\\
3164.20000000003	75.324716366969\\
3164.49999999999	75.3270555520445\\
3164.70000000005	75.322295247725\\
3165.40000000004	75.3277523298057\\
3165.50000000001	75.3253727571392\\
3165.79999999998	75.3213936005559\\
3167.39999999996	75.3338595106551\\
3167.49999999998	75.3314812476323\\
3167.80000000001	75.3338173553458\\
3167.9	75.3314393939394\\
3168.60000000004	75.333733076656\\
3168.70000000005	75.331355718253\\
3169.09999999997	75.3281585258109\\
3169.80000000003	75.3304520647339\\
3169.89999999996	75.3280757097792\\
3170.70000000001	75.3311467137631\\
3170.80000000001	75.3287710113848\\
3171.09999999995	75.3311049445005\\
3171.19999999998	75.3287295430896\\
3173.39999999998	75.3426815818497\\
3173.50000000002	75.3403075371817\\
3174.2	75.345745518697\\
3174.29999999999	75.3433719758064\\
3174.99999999996	75.3488079115618\\
3175.09999999996	75.3464348702444\\
3175.39999999995	75.3487639741773\\
3175.49999999998	75.3463912331528\\
3175.79999999996	75.3487200478605\\
3176.09999999999	75.3416031736037\\
3177.29999999995	75.3509158431422\\
3177.40000000005	75.3485444531865\\
3178.50000000003	75.350783363745\\
3178.60000000002	75.3484128731872\\
3178.90000000003	75.3507392261718\\
3178.99999999995	75.3483690352616\\
3180.39999999998	75.3466436094954\\
3180.99999999998	75.3512935776932\\
3181.09999999996	75.3489249339872\\
3182.19999999997	75.3574458724822\\
3182.30000000004	75.3550779286073\\
3182.60000000002	75.3574009488799\\
3182.70000000001	75.355033304009\\
3183.59999999996	75.3525771900619\\
3184.00000000005	75.3493922929556\\
3184.50000000001	75.3469823525718\\
3184.80000000001	75.3430248987409\\
3186.00000000003	75.3491729700888\\
3186.09999999996	75.3468081099743\\
3186.79999999998	75.3490853180206\\
3186.89999999998	75.346721054283\\
3187.39999999998	75.3505882352941\\
3187.59999999995	75.3458606518807\\
3188.4	75.342637603889\\
3188.70000000002	75.3449573507276\\
3188.80000000002	75.3425946251058\\
3189.60000000001	75.3456437909521\\
3189.69999999998	75.3432817104521\\
3190.49999999998	75.3463298439165\\
3190.59999999996	75.3439684081863\\
3191.39999999998	75.3501488328372\\
3191.50000000003	75.3477879433513\\
3192.00000000003	75.345383916544\\
3193.79999999999	75.3592786248787\\
3193.9	75.3569192235442\\
3194.20000000003	75.3592336349122\\
3194.39999999998	75.3545155736422\\
3194.79999999995	75.3576011768756\\
3194.90000000002	75.3552425665102\\
3195.29999999996	75.3520685986105\\
3196.00000000002	75.3543380995588\\
3196.10000000004	75.3519804768162\\
3197.19999999998	75.3604603884528\\
3197.29999999999	75.3581034590605\\
3197.60000000002	75.3541608030772\\
3198.69999999996	75.3595098161811\\
3198.80000000002	75.357154021695\\
3199.09999999997	75.3594648662166\\
3199.19999999996	75.3571093676742\\
3199.50000000003	75.3531691461433\\
3200.10000000005	75.3577901381164\\
3200.19999999996	75.3554354279286\\
3201.00000000001	75.3584705257568\\
3201.09999999999	75.3561164563289\\
3201.79999999997	75.3615041069365\\
3201.90000000002	75.3591505309182\\
3202.39999999997	75.3567525370804\\
3202.89999999995	75.3605994380269\\
3202.99999999995	75.3582466985108\\
3204.09999999999	75.3635852943012\\
3204.20000000005	75.3612333426958\\
3204.7	75.3557164253619\\
3205.09999999997	75.35879196306\\
3205.19999999998	75.356440894768\\
3205.59999999997	75.3595158623702\\
3205.69999999999	75.357165138187\\
3206.80000000002	75.3624996102155\\
3206.90000000004	75.3601496725912\\
3207.19999999996	75.3562186262588\\
3207.90000000003	75.3615960099751\\
3207.99999999997	75.3592469062685\\
3208.29999999997	75.3615509288119\\
3208.50000000005	75.3568534563361\\
3209.50000000004	75.3520687936191\\
3209.89999999997	75.3489096573209\\
3210.19999999996	75.3512132822478\\
3210.49999999998	75.3441724288295\\
3210.80000000003	75.3464760659005\\
3210.89999999996	75.3441295546559\\
3211.29999999999	75.3409727844554\\
3212.50000000004	75.3501836518708\\
3212.70000000001	75.3454930278884\\
3213.49999999998	75.3516305700772\\
3213.79999999997	75.3445969071844\\
3214.30000000004	75.3484320557491\\
3214.39999999996	75.3460880385752\\
3215.80000000002	75.3537112472403\\
3216.00000000003	75.3490252168776\\
3216.49999999996	75.3466393085867\\
3216.90000000005	75.3497046938141\\
3217.00000000001	75.3473625314724\\
3217.80000000004	75.3503837906709\\
3217.90000000001	75.3480422622747\\
3218.2	75.3441257806917\\
3218.59999999998	75.3409761705036\\
3219.60000000001	75.3486349659906\\
3219.99999999995	75.3392751777895\\
3221.10000000001	75.3414876443561\\
3221.19999999999	75.3391487908608\\
3221.89999999998	75.3414028553693\\
3222	75.3390645852084\\
3222.79999999996	75.3420832169785\\
3223.00000000002	75.3374080853836\\
3223.89999999995	75.3442928039702\\
3224.00000000002	75.3419558946683\\
3224.60000000002	75.3465438645455\\
3224.90000000005	75.3395348837209\\
3225.20000000004	75.3356276935479\\
3225.49999999997	75.3379216269841\\
3225.59999999995	75.3355860743405\\
3226.30000000004	75.331638978428\\
3226.69999999996	75.3284988223627\\
3227.39999999998	75.3307513555383\\
3227.49999999997	75.3284173999256\\
3228.09999999998	75.3330029118394\\
3228.2	75.3306693925595\\
3228.60000000002	75.3275312045096\\
3228.99999999998	75.3243937939364\\
3229.70000000005	75.3297417796768\\
3229.80000000004	75.3274095173225\\
3230.30000000004	75.3312283308569\\
3230.59999999995	75.3242331383291\\
3231.39999999998	75.3272474083243\\
3231.50000000004	75.3249164500557\\
3232.70000000001	75.3278891363524\\
3232.79999999998	75.3255590955489\\
3233.2	75.3224260043918\\
3233.69999999999	75.3262415733812\\
3233.79999999997	75.3239123040292\\
3234.30000000001	75.3184516448182\\
3234.90000000005	75.323029366306\\
3235.09999999999	75.3183728981207\\
3235.8	75.3206217744677\\
3235.89999999999	75.3182941903585\\
3236.50000000004	75.3228696780572\\
3236.60000000001	75.3205425278833\\
3237.39999999996	75.3173745173745\\
3238.60000000004	75.3203445827029\\
3238.69999999998	75.3180190193899\\
3238.99999999999	75.3203050230002\\
3239.09999999995	75.3179797480859\\
3240.1	75.3225109561138\\
3240.20000000003	75.3201864024936\\
3241.59999999999	75.3277601258599\\
3241.80000000002	75.3231129892964\\
3242.59999999996	75.3292009744966\\
3242.69999999999	75.3268780066609\\
3243.20000000005	75.33068171307\\
3243.29999999999	75.3283591293088\\
3243.80000000002	75.3321619038811\\
3243.90000000003	75.329839704069\\
3244.19999999995	75.3259562925747\\
3245.80000000002	75.3381188576358\\
3245.89999999998	75.335797905114\\
3246.20000000002	75.3380771955765\\
3246.29999999997	75.3357565303105\\
3246.69999999997	75.3295552544043\\
3247.60000000001	75.3271546017181\\
3248.30000000002	75.3232360546731\\
3249.00000000005	75.325474746853\\
3249.19999999995	75.3208383344105\\
3249.70000000005	75.3154040248631\\
3251.60000000002	75.3113755881539\\
3251.89999999998	75.3136531365314\\
3252.00000000004	75.3113372897512\\
3252.79999999999	75.3143349011651\\
3252.89999999996	75.312019674147\\
3253.50000000004	75.3165724120974\\
3253.59999999996	75.3142576144082\\
3254.09999999997	75.3119046155737\\
3255.19999999997	75.3141031548552\\
3255.30000000004	75.3117896418259\\
3255.69999999995	75.3148227778119\\
3255.80000000001	75.3125095979606\\
3256.20000000004	75.3094002395357\\
3256.90000000005	75.30242554498\\
3257.39999999997	75.3062164236378\\
3257.50000000001	75.3039047151277\\
3258.10000000003	75.3084525197962\\
3258.19999999998	75.3061412392966\\
3258.99999999999	75.3091344236139\\
3259.10000000002	75.306823760432\\
3259.7	75.3113687956316\\
3259.80000000005	75.3090585600785\\
3260.79999999996	75.3166303781165\\
3260.90000000003	75.3143207605029\\
3262.00000000005	75.3226449219828\\
3262.09999999997	75.3203359695911\\
3262.99999999996	75.3271429009224\\
3263.09999999999	75.3248345182643\\
3264.30000000002	75.3277784585222\\
3264.39999999997	75.3254709756471\\
3266.19999999998	75.3360071028381\\
3266.29999999997	75.3337007102621\\
3266.80000000003	75.337475894579\\
3266.90000000001	75.3351698806244\\
3267.19999999996	75.3374345790102\\
3267.29999999997	75.3351288486258\\
3267.70000000002	75.3320276638717\\
3267.99999999999	75.3342920963251\\
3268.30000000004	75.3273773099988\\
3269.09999999999	75.3334149027285\\
3269.20000000003	75.3311106353042\\
3269.59999999995	75.3341285133193\\
3269.70000000006	75.3318245764267\\
3270.20000000004	75.3355961226799\\
3270.30000000001	75.3332925636008\\
3270.60000000003	75.3355550799523\\
3270.70000000003	75.33325180384\\
3271.00000000003	75.3355140472624\\
3271.10000000005	75.3332110540474\\
3271.69999999999	75.3377345803533\\
3271.79999999999	75.3354320119808\\
3272.09999999997	75.3315811991932\\
3273.49999999999	75.3421309872923\\
3273.59999999997	75.3398295506613\\
3273.89999999996	75.3420891875382\\
3274.00000000003	75.3397880333527\\
3274.30000000001	75.3420473979966\\
3274.39999999999	75.3397465261872\\
3275.70000000002	75.3342694914219\\
3276.80000000001	75.3364460313101\\
3276.90000000005	75.3341470857492\\
3277.30000000004	75.3371575028986\\
3277.40000000004	75.3348588863463\\
3279.09999999999	75.3476457672603\\
3279.20000000004	75.3453480925807\\
3279.90000000005	75.3506097560976\\
3280.10000000002	75.3460154868606\\
3280.40000000002	75.3482700807804\\
3280.50000000001	75.3459732975675\\
3281.19999999998	75.3512327431201\\
3281.39999999998	75.3466402559805\\
3281.9	75.3503960999391\\
3281.99999999996	75.3481003016362\\
3282.30000000002	75.3503533999513\\
3282.39999999996	75.3480578827113\\
3282.70000000003	75.3503107103692\\
3282.79999999999	75.3480154741235\\
3283.39999999996	75.3525201766408\\
3283.49999999997	75.3502253624071\\
3284.20000000002	75.3554790975246\\
3284.30000000002	75.3531847521617\\
3284.70000000005	75.3500974184121\\
3285.20000000004	75.353848963565\\
3285.30000000005	75.3515553661655\\
3286.09999999999	75.3545128111497\\
3286.19999999995	75.352219821684\\
3286.89999999997	75.3452996653483\\
3287.40000000002	75.3429657794677\\
3287.89999999995	75.3467153284672\\
3288.10000000002	75.3421324736938\\
3288.50000000004	75.3451316669707\\
3288.59999999999	75.3428406361176\\
3288.89999999997	75.3450896929158\\
3289.09999999995	75.3405083302931\\
3289.90000000004	75.3373860182371\\
3291.09999999998	75.3403014098201\\
3291.20000000002	75.3380123355513\\
3292.10000000004	75.3356418200595\\
3292.59999999997	75.3333130865247\\
3292.89999999998	75.3294867901609\\
3294.00000000003	75.3346892929784\\
3294.10000000001	75.3324024042256\\
3294.79999999998	75.3376430240675\\
3294.89999999998	75.3353566009105\\
3295.40000000005	75.3330298892429\\
3295.79999999998	75.3299553991323\\
3296.09999999997	75.332200715976\\
3296.30000000001	75.3276301419731\\
3296.60000000002	75.3298753298753\\
3296.70000000003	75.3275903906819\\
3297.50000000003	75.3244784085395\\
3298.89999999998	75.3349499848439\\
3298.99999999996	75.3326664847989\\
3299.30000000001	75.3349093774626\\
3299.40000000004	75.3326261554781\\
3300.10000000002	75.3348281922308\\
3300.29999999996	75.3302629984244\\
3301.19999999999	75.3339593493472\\
3301.29999999998	75.3316774701642\\
3301.89999999998	75.336159903089\\
3302.00000000005	75.3338784409921\\
3302.59999999995	75.3383595240258\\
3302.69999999995	75.3360784788664\\
3303.40000000001	75.3382775843802\\
3303.49999999998	75.3359970940792\\
3303.80000000003	75.3382366294379\\
3303.89999999995	75.3359564164649\\
3304.79999999997	75.3426730006959\\
3304.9	75.3403933434191\\
3305.39999999995	75.3380729087884\\
3306.59999999999	75.3470227114646\\
3306.79999999996	75.3424657534247\\
3307.29999999996	75.3461933845317\\
3307.4	75.3439153439153\\
3307.90000000004	75.3355501813785\\
3308.40000000002	75.339277618256\\
3308.50000000004	75.3370005440367\\
3309.20000000005	75.3422173873629\\
3309.30000000005	75.3399407747628\\
3310.09999999999	75.336837653314\\
3310.70000000005	75.3413072369216\\
3310.89999999995	75.3367562669888\\
3312.00000000001	75.3389088493705\\
3312.09999999996	75.3366342612161\\
3312.79999999998	75.3388270095687\\
3313.09999999999	75.332005312085\\
3314.59999999995	75.3401514465864\\
3314.80000000003	75.3356059006305\\
3315.60000000001	75.3415568356607\\
3315.69999999999	75.3392846371916\\
3316.70000000005	75.3467197298601\\
3316.80000000002	75.3444481292773\\
3317.99999999998	75.3473373316054\\
3318.19999999996	75.3427960100051\\
3318.49999999996	75.3450250105466\\
3318.60000000004	75.3427546931027\\
3319.00000000004	75.3457262510922\\
3319.2	75.3411863947218\\
3319.89999999997	75.3463855421687\\
3319.99999999996	75.3441161410801\\
3321.20000000006	75.3530244181495\\
3321.49999999997	75.3462186897881\\
3322	75.3439089732398\\
3322.50000000001	75.3476193342563\\
3322.59999999997	75.3453516718331\\
3323.30000000001	75.3475356562556\\
3323.39999999995	75.3452685421995\\
3324.20000000005	75.3481936046687\\
3324.60000000004	75.3391283424068\\
3325.39999999997	75.3450608930988\\
3325.59999999995	75.3405298132724\\
3326.29999999997	75.3427128427128\\
3326.50000000004	75.3381831299224\\
3326.80000000002	75.3343953830894\\
3327.19999999995	75.3373606227271\\
3327.4	75.3328324567994\\
3327.70000000004	75.3350561932808\\
3327.79999999999	75.3327924516963\\
3328.09999999999	75.3350159245238\\
3328.30000000004	75.3304891239034\\
3329.30000000003	75.3378987204902\\
3329.40000000004	75.3356359813786\\
3330.69999999997	75.3452623994236\\
3330.80000000001	75.3430003902849\\
3331.20000000005	75.3459610362321\\
3331.3	75.3436993456205\\
3331.79999999996	75.3413968006243\\
3332.30000000003	75.339094946585\\
3332.69999999995	75.3420547287566\\
3333.1	75.3330133205328\\
3333.69999999995	75.3374527566141\\
3333.89999999998	75.3329334133173\\
3334.69999999999	75.3388509056015\\
3334.89999999998	75.3343328335832\\
3335.40000000001	75.3320341777844\\
3335.70000000005	75.3342526530367\\
3335.80000000005	75.3319943643395\\
3336.20000000002	75.3349518928154\\
3336.29999999998	75.3326939215921\\
3337.19999999996	75.3393461780481\\
3337.30000000001	75.3370887517229\\
3338.20000000003	75.3437378306324\\
3338.30000000002	75.3414809489576\\
3339.00000000005	75.3466502949897\\
3339.09999999995	75.3443938667944\\
3339.50000000001	75.3383638759133\\
3339.99999999996	75.342055627077\\
3340.19999999996	75.3375445319283\\
3340.60000000002	75.3404975005238\\
3340.69999999995	75.3382423371648\\
3342.59999999995	75.3522601489814\\
3342.69999999999	75.3500059830083\\
3343.10000000003	75.3469729600383\\
3343.60000000003	75.3506594491133\\
3343.70000000005	75.3484060051439\\
3345.29999999998	75.3601960901536\\
3345.39999999995	75.3579435062024\\
3345.7	75.3601530276765\\
3345.80000000005	75.3579007143071\\
3346.49999999998	75.3630550409371\\
3346.60000000003	75.3608031792512\\
3347.50000000003	75.3554785517983\\
3348.20000000001	75.3606307678523\\
3348.29999999997	75.3583801218493\\
3349.90000000002	75.3701492537313\\
3350.00000000003	75.3678994656876\\
3350.30000000002	75.3641356255969\\
3350.60000000004	75.3663413615066\\
3350.89999999999	75.3595941509997\\
3351.4	75.3632701775324\\
3351.49999999997	75.3610216016231\\
3352.50000000005	75.3564397780827\\
3353.00000000005	75.3541498911455\\
3354.09999999995	75.3592510881879\\
3354.20000000004	75.3570044420594\\
3354.69999999997	75.3547156313342\\
3355.2	75.3524275027568\\
3355.69999999997	75.3501400560224\\
3356.29999999997	75.3545465379573\\
3356.40000000003	75.3523015045434\\
3356.90000000005	75.347036044087\\
3357.20000000002	75.3492389717928\\
3357.29999999999	75.3469946982784\\
3358.00000000005	75.3521336470028\\
3358.10000000001	75.3498898219284\\
3358.89999999995	75.3527835665377\\
3359.00000000004	75.3505403233009\\
3359.99999999995	75.3578762536829\\
3360.10000000001	75.3556335932385\\
3360.39999999997	75.357833655706\\
3360.60000000002	75.3533490046716\\
3361.49999999996	75.3569728700619\\
3361.79999999996	75.3502483714566\\
3362.49999999997	75.3524058764052\\
3362.59999999998	75.3501650459452\\
3362.89999999996	75.3523639607493\\
3363	75.3501233980554\\
3364.19999999999	75.355943286865\\
3364.40000000004	75.3514638133452\\
3364.80000000002	75.3484501768255\\
3365.99999999999	75.3542675499837\\
3366.30000000001	75.3475522813688\\
3366.79999999998	75.3452730998842\\
3367.20000000003	75.3482018234194\\
3367.40000000001	75.3437268002969\\
3368.09999999999	75.3488510183481\\
3368.19999999997	75.3466140189413\\
3368.49999999998	75.3488095944903\\
3368.59999999997	75.3465728619349\\
3369.10000000004	75.3502315089635\\
3369.2	75.3479951325201\\
3369.59999999998	75.3420185773214\\
3370.49999999996	75.3456357918472\\
3370.59999999997	75.3434004806123\\
3371.1	75.3470574276222\\
3371.39999999998	75.3403529586238\\
3372.19999999996	75.3462028882365\\
3372.30000000001	75.3439686869885\\
3373.79999999999	75.3549304958653\\
3373.89999999997	75.3526970954357\\
3374.80000000004	75.3474177012652\\
3375.30000000002	75.3451442791965\\
3376.19999999998	75.3517163759145\\
3376.29999999995	75.3494846582158\\
3377.80000000004	75.3574706178395\\
3377.9	75.3552397868561\\
3378.59999999995	75.3603456950898\\
3378.70000000005	75.3581153072097\\
3379.40000000003	75.3632194111555\\
3379.49999999999	75.360989466209\\
3379.9	75.3639053254438\\
3379.99999999995	75.3616756900683\\
3380.4	75.3645910368289\\
3380.49999999998	75.3623617109389\\
3381	75.3571322942238\\
3381.30000000001	75.3593186254214\\
3381.39999999995	75.3570900487949\\
3381.79999999995	75.3540908956504\\
3382.30000000001	75.357734153264\\
3382.40000000003	75.3555062823356\\
3382.70000000004	75.3517795908715\\
3383.29999999999	75.3561506177218\\
3383.50000000005	75.3516964180163\\
3384.20000000005	75.3567946104069\\
3384.29999999997	75.3545680179648\\
3384.99999999995	75.3596644116865\\
3385.09999999996	75.3574382606641\\
3385.40000000001	75.359621916999\\
3385.69999999998	75.3529446511903\\
3386.40000000003	75.3550863723608\\
3386.60000000005	75.3506363126347\\
3387.29999999995	75.3527779417843\\
3387.50000000004	75.348329200614\\
3388.69999999995	75.3511567516525\\
3388.80000000001	75.348933282186\\
3389.10000000004	75.3511153074472\\
3389.20000000001	75.3488921016139\\
3389.80000000001	75.3532552582672\\
3389.89999999996	75.3510324483776\\
3390.19999999995	75.3532135799192\\
3390.30000000004	75.3509910335064\\
3391.09999999998	75.3568058504364\\
3391.20000000004	75.3545837879279\\
3391.49999999999	75.3567637693124\\
3391.60000000001	75.3545419701035\\
3392.39999999996	75.3603537214444\\
3392.50000000003	75.3581324058244\\
3393.19999999996	75.3632157486812\\
3393.59999999997	75.3543330288476\\
3394.20000000005	75.3586895678048\\
3394.59999999998	75.349809997938\\
3394.90000000005	75.3519882179676\\
3395.00000000003	75.3497687844246\\
3395.30000000001	75.3519467514873\\
3395.39999999998	75.3497275806214\\
3396.29999999999	75.3533152749971\\
3396.39999999996	75.3510967172089\\
3396.69999999999	75.3532736693358\\
3396.80000000001	75.3510553740175\\
3397.30000000002	75.3429092835698\\
3398.40000000004	75.3508900985729\\
3398.49999999996	75.348672982993\\
3398.90000000002	75.3456899087967\\
3399.20000000002	75.3478657370635\\
3399.29999999998	75.3456492322175\\
3400.69999999999	75.3410962126558\\
3401.80000000004	75.3461301037655\\
3401.90000000005	75.3439153439153\\
3402.40000000004	75.3416605437179\\
3403.49999999999	75.343753672582\\
3403.59999999999	75.341540088727\\
3404.30000000003	75.343672893902\\
3404.49999999998	75.3392469012512\\
3405.30000000002	75.342103717625\\
3405.39999999996	75.3398913522243\\
3406.20000000001	75.3368757889792\\
3406.49999999998	75.3390477308754\\
3406.59999999995	75.3368362344791\\
3407.40000000001	75.3426265590609\\
3407.6	75.3382046541656\\
3407.90000000005	75.3345070422535\\
3408.59999999995	75.3395722709537\\
3408.70000000001	75.3373621215677\\
3409.39999999999	75.339492594222\\
3409.50000000002	75.3372829657438\\
3409.80000000001	75.3394527698759\\
3409.90000000002	75.3372434017595\\
3410.40000000004	75.3408591115672\\
3410.49999999995	75.3386500908931\\
3410.80000000005	75.3408191386438\\
3410.89999999998	75.3386103781882\\
3411.59999999995	75.3407392209163\\
3411.70000000004	75.338530980714\\
3412.20000000001	75.3362834451836\\
3412.60000000004	75.3391742608492\\
3412.80000000004	75.3347592956137\\
3413.4	75.3303061373956\\
3413.69999999996	75.3266155017869\\
3414.30000000001	75.3309512652296\\
3414.40000000001	75.3287450578416\\
3414.99999999998	75.3242950426049\\
3416.30000000004	75.3278304648168\\
3416.40000000003	75.3256256402751\\
3416.90000000005	75.3292361720808\\
3417.20000000001	75.3226231235186\\
3417.60000000005	75.3255113087749\\
3417.70000000006	75.3233073907192\\
3418.50000000001	75.3290820803838\\
3418.70000000002	75.3246753246753\\
3419.69999999999	75.3289666062343\\
3419.79999999996	75.3267639404661\\
3420.20000000003	75.329649445955\\
3420.29999999997	75.3274470822126\\
3421.19999999996	75.3310145266419\\
3421.3	75.3288127667037\\
3422.40000000005	75.336742147553\\
3422.5	75.3345409922281\\
3423.6	75.3395449367643\\
3423.70000000003	75.3373444710556\\
3424.09999999998	75.3343846737924\\
3424.39999999996	75.3307052124398\\
3424.8	75.3335863820841\\
3424.90000000006	75.3313868613139\\
3425.80000000004	75.3261916576666\\
3426.79999999995	75.333391695118\\
3427.00000000005	75.3289953605089\\
3427.49999999995	75.3325942350333\\
3427.59999999996	75.3303964757709\\
3427.90000000005	75.3267211201867\\
3428.99999999996	75.3346359102972\\
3429.09999999995	75.3324390528403\\
3429.60000000002	75.3272881010001\\
3430.29999999995	75.332322761194\\
3430.40000000004	75.3301268036729\\
3430.70000000004	75.33228401539\\
3430.79999999996	75.3300883150194\\
3431.19999999996	75.3271354880075\\
3432.59999999998	75.3371981239258\\
3432.70000000004	75.3350034956887\\
3433.99999999998	75.3414286130282\\
3434.1	75.3392347562751\\
3436.10000000005	75.3535882661079\\
3436.19999999996	75.351395396211\\
3436.6	75.3542642651381\\
3436.7	75.3520716945996\\
3437	75.3484041779407\\
3439.10000000005	75.3518260060479\\
3439.40000000001	75.3452536705917\\
3439.89999999995	75.3488372093023\\
3440.00000000002	75.3466468997994\\
3441.1	75.3370917121934\\
3441.50000000004	75.3341469084147\\
3442.20000000001	75.3391627690788\\
3442.29999999995	75.3369742040437\\
3442.89999999999	75.3325588149869\\
3444.39999999996	75.3403977355204\\
3444.50000000003	75.3382105324276\\
3445.39999999996	75.3417501088376\\
3445.59999999997	75.3373770206344\\
3446.69999999998	75.3452477660439\\
3446.79999999996	75.3430618816908\\
3447.60000000002	75.3400817936595\\
3448.19999999999	75.3443725893919\\
3448.30000000004	75.3421876812435\\
3449.40000000002	75.3384548485288\\
3449.90000000003	75.3420289855072\\
3450.09999999997	75.3376615848357\\
3451.2	75.3426245182975\\
3451.29999999996	75.3404415599467\\
3451.59999999999	75.3425848132804\\
3451.79999999995	75.3382195312726\\
3453.49999999996	75.3271948112115\\
3453.80000000002	75.3293378499667\\
3453.89999999999	75.3271569195136\\
3454.19999999999	75.3235098283299\\
3455.39999999998	75.3320792938793\\
3455.60000000005	75.3277194200885\\
3456.10000000006	75.3312886985707\\
3456.20000000004	75.3291091629777\\
3456.69999999999	75.3326776209211\\
3456.79999999998	75.330498423443\\
3457.80000000003	75.3347407385986\\
3457.89999999998	75.3325621746674\\
3458.39999999996	75.3274540985977\\
3459.49999999999	75.3352988784831\\
3459.59999999998	75.3331213689048\\
3460	75.3359729487587\\
3460.10000000003	75.3337957343506\\
3460.59999999995	75.337359493744\\
3460.70000000002	75.335182616736\\
3461.59999999996	75.3415951700032\\
3461.69999999997	75.3394187994685\\
3462.09999999998	75.3422679221304\\
3462.20000000005	75.3400918464604\\
3462.99999999995	75.3429008691635\\
3463.19999999998	75.3385499379205\\
3463.99999999995	75.3442452585087\\
3464.40000000005	75.3355462548708\\
3464.8	75.338393604433\\
3464.89999999996	75.3362193362193\\
3465.39999999998	75.3397778098398\\
3465.49999999998	75.3376038781163\\
3466.40000000001	75.3440069234098\\
3466.50000000005	75.341833496798\\
3467.40000000001	75.3453496755588\\
3467.59999999996	75.3410041237708\\
3468.60000000002	75.3452302015164\\
3468.70000000004	75.3430581180812\\
3469.09999999996	75.3459010722933\\
3469.20000000004	75.3437292825642\\
3469.69999999997	75.3472822641074\\
3469.80000000005	75.3451108101098\\
3470.59999999998	75.3507937879967\\
3470.70000000003	75.3486227958972\\
3471.00000000002	75.3507533634871\\
3471.10000000001	75.3485826227241\\
3471.60000000001	75.346372094363\\
3471.99999999999	75.3434520895135\\
3472.99999999997	75.347672108491\\
3473.10000000003	75.3455027064379\\
3473.40000000001	75.3476320713977\\
3473.49999999999	75.3454629203132\\
3474.3	75.3511397651393\\
3474.40000000004	75.3489710749748\\
3476.30000000001	75.3566908295938\\
3476.39999999996	75.3545232273839\\
3477.59999999995	75.3601518244817\\
3477.70000000003	75.3579849330036\\
3478.30000000004	75.3622355105796\\
3478.40000000002	75.3600689952566\\
3478.99999999997	75.3643183581961\\
3479.09999999995	75.3621522189009\\
3480.00000000004	75.3685238929916\\
3480.19999999999	75.364192742005\\
3480.70000000001	75.3677315559641\\
3480.79999999998	75.3655663765118\\
3481.09999999996	75.3619441571872\\
3481.80000000001	75.3640253884374\\
3481.90000000001	75.3618609994256\\
3482.40000000001	75.3653984206748\\
3482.5	75.3632343651295\\
3483.00000000002	75.3610289684477\\
3483.50000000006	75.3645653921231\\
3483.59999999999	75.362402043804\\
3484.00000000001	75.3652306190982\\
3484.29999999998	75.3587418206865\\
3485.80000000005	75.3664763762586\\
3485.90000000005	75.364314400459\\
3486.20000000002	75.3664343286579\\
3486.29999999996	75.3642726021111\\
3486.60000000003	75.3663922907047\\
3486.70000000001	75.3642308133532\\
3487.00000000003	75.3606148375441\\
3488.70000000001	75.3726209584958\\
3489.09999999996	75.3639802820131\\
3489.70000000004	75.3682159436071\\
3489.9	75.3638968481375\\
3490.20000000006	75.3660143827178\\
3490.29999999996	75.3638551455421\\
3490.89999999998	75.3680893726726\\
3490.99999999997	75.3659305090086\\
3491.49999999995	75.3694581280788\\
3491.6	75.3672995961852\\
3491.99999999998	75.3701211305518\\
3492.09999999995	75.3679628887234\\
3492.60000000002	75.3714891058493\\
3492.69999999997	75.3693311956024\\
3493.50000000004	75.372108999313\\
3493.70000000005	75.3677943786135\\
3494.40000000004	75.3727285734726\\
3494.49999999996	75.3705717392549\\
3495.39999999995	75.368330710914\\
3495.99999999997	75.3639770029461\\
3496.40000000002	75.3667953667954\\
3496.49999999995	75.3646399359378\\
3496.8	75.3667534101633\\
3496.89999999998	75.3645982270518\\
3497.49999999999	75.368824336688\\
3497.60000000005	75.3666695256883\\
3498.7	75.3715559620441\\
3498.79999999999	75.3694018119981\\
3499.20000000001	75.372217300603\\
3499.30000000001	75.3700634394468\\
3500.70000000002	75.3770566727605\\
3500.79999999995	75.3749035962181\\
3501.1	75.3713012681366\\
3501.40000000004	75.3734113951164\\
3501.5	75.3712588530957\\
3502.00000000002	75.374775134919\\
3502.09999999995	75.3726229227343\\
3502.70000000005	75.3682768071257\\
3503.09999999998	75.3653802237954\\
3504.49999999996	75.375221137933\\
3504.59999999995	75.3730704482552\\
3505.50000000004	75.3765403925148\\
3505.60000000006	75.374390278689\\
3506.10000000006	75.3721978210028\\
3506.49999999996	75.3693035989277\\
3506.79999999998	75.3714106475805\\
3506.90000000003	75.3692614770459\\
3508.09999999996	75.374836098284\\
3508.30000000001	75.3705392771634\\
3508.89999999997	75.3747506412083\\
3508.99999999999	75.3726026616511\\
3509.70000000005	75.3775143882842\\
3509.79999999999	75.3753668195675\\
3510.20000000001	75.3781728057431\\
3510.29999999996	75.3760255241568\\
3511.79999999999	75.3723055895669\\
3512.30000000003	75.3701172987131\\
3512.99999999999	75.3750249067775\\
3513.10000000001	75.3728794261642\\
3513.39999999999	75.37498221147\\
3513.5	75.3728369763206\\
3514.29999999999	75.3784429774641\\
3514.40000000002	75.3762981931996\\
3514.69999999995	75.3783999089564\\
3514.79999999995	75.3762553699963\\
3515.20000000003	75.3733678491167\\
3515.79999999995	75.3775704655991\\
3515.99999999999	75.3732828986661\\
3516.39999999997	75.3675529645955\\
3517.79999999996	75.3745132039001\\
3517.89999999996	75.3723706651507\\
3518.40000000002	75.3673440386528\\
3518.7	75.3694441286802\\
3518.80000000004	75.3673022819631\\
3520.49999999997	75.3763563029029\\
3520.6	75.374215354901\\
3520.9	75.3763135472877\\
3520.99999999997	75.3741728437136\\
3521.50000000003	75.377669241254\\
3521.70000000005	75.3733886080981\\
3522.80000000004	75.3810780890743\\
3522.90000000001	75.3789384047687\\
3523.4	75.3824322406698\\
3523.50000000005	75.3802928822795\\
3524.50000000002	75.3872779889917\\
3524.69999999995	75.3830004539265\\
3525.69999999999	75.3871461795904\\
3525.99999999998	75.3807322537648\\
3526.59999999999	75.384920747441\\
3526.80000000002	75.3806458929938\\
3527.29999999998	75.3841356239723\\
3527.39999999998	75.3819985825656\\
3527.79999999995	75.3791207233765\\
3528.50000000003	75.3840049878139\\
3528.60000000002	75.3818686768498\\
3529.50000000001	75.3881459655485\\
3529.60000000003	75.386010142505\\
3530.39999999999	75.3887551338337\\
3530.50000000001	75.3866198379879\\
3530.90000000004	75.3894080996885\\
3531.00000000005	75.3872730877064\\
3531.49999999995	75.3907577302073\\
3531.70000000005	75.3864884761311\\
3532.20000000002	75.3899725391388\\
3532.29999999996	75.3878382969086\\
3532.59999999998	75.3899283833895\\
3532.70000000006	75.387794384058\\
3533.2	75.3856168454419\\
3534.09999999995	75.3918850093373\\
3534.40000000001	75.3854859244589\\
3534.79999999997	75.3882712382246\\
3535.3	75.3776093228489\\
3536.00000000005	75.3796555527276\\
3536.20000000002	75.3753923592455\\
3537.30000000001	75.383049697518\\
3537.39999999999	75.3809187279152\\
3538.10000000002	75.3857893844328\\
3538.19999999998	75.383658819207\\
3539.10000000005	75.3814421338155\\
3539.50000000003	75.384224206125\\
3539.60000000004	75.3820945277848\\
3540.00000000003	75.3848761334426\\
3540.09999999999	75.3827467374725\\
3540.89999999996	75.3854843264615\\
3540.99999999998	75.3833554545198\\
3541.3	75.3854407861298\\
3541.39999999996	75.3833121558662\\
3542.39999999996	75.3874382498236\\
3542.50000000002	75.3853102241292\\
3542.89999999997	75.388089189952\\
3543.00000000002	75.3859614461912\\
3543.79999999996	75.3886960693022\\
3543.89999999997	75.3865688487585\\
3544.50000000006	75.3907352028438\\
3544.59999999999	75.3886083448529\\
3546.09999999995	75.3849190682985\\
3546.49999999995	75.3876952574297\\
3546.60000000004	75.3855696844955\\
3547.1	75.3890392422192\\
3547.19999999999	75.3869139909227\\
3548	75.3896451621995\\
3548.2	75.385395823352\\
3548.89999999995	75.3789799943646\\
3550.09999999996	75.387302123824\\
3550.20000000001	75.3851787172915\\
3550.89999999999	75.387214869051\\
3551.09999999996	75.3829691371931\\
3551.59999999996	75.38643466509\\
3551.7	75.3843121797399\\
3552.99999999996	75.3933185105964\\
3553.09999999996	75.3911966677924\\
3553.49999999995	75.3939666816749\\
3553.60000000003	75.3918451191716\\
3554.10000000001	75.3953069607788\\
3554.20000000003	75.3931857187069\\
3554.69999999999	75.3910205918758\\
3555.00000000003	75.393097240584\\
3555.09999999997	75.3909765976598\\
3556.30000000006	75.3964683387695\\
3556.49999999998	75.3922285328685\\
3557.69999999995	75.3892855135196\\
3558.09999999999	75.3920521612051\\
3558.19999999995	75.3899333951606\\
3558.69999999994	75.3933910306845\\
3558.79999999995	75.3912725842255\\
3560.79999999997	75.402285938948\\
3560.89999999997	75.4001684919966\\
3561.40000000005	75.4036220693528\\
3561.50000000003	75.4015049415993\\
3562.49999999997	75.4084095885028\\
3562.59999999996	75.4062929800432\\
3563.2	75.3992085987708\\
3563.49999999995	75.4012796048939\\
3563.70000000005	75.3970480947304\\
3564.89999999995	75.4025245441795\\
3565.20000000001	75.3961798446134\\
3565.99999999997	75.3932867838815\\
3567	75.4001850242494\\
3567.10000000003	75.3980713164387\\
3567.40000000006	75.4001401541696\\
3567.59999999998	75.3959133335202\\
3568.5	75.4021184778344\\
3568.7	75.397892849137\\
3569.09999999996	75.4006500056035\\
3569.29999999998	75.3964251694963\\
3570.29999999996	75.4005153484204\\
3570.40000000006	75.3984035849321\\
3570.79999999999	75.401159371587\\
3570.89999999995	75.3990478857463\\
3571.29999999999	75.4018032144257\\
3571.39999999998	75.3996920061599\\
3571.79999999999	75.4024468770122\\
3571.90000000005	75.4003359462486\\
3572.39999999999	75.4037788663401\\
3572.50000000002	75.4016682528131\\
3572.80000000002	75.4037336617314\\
3572.99999999995	75.3995130279029\\
3573.59999999995	75.4036432828721\\
3573.79999999997	75.3994235988696\\
3574.19999999995	75.4021766499734\\
3574.29999999995	75.4000671441361\\
3575.09999999998	75.3971805773104\\
3575.59999999996	75.4006208574545\\
3575.89999999998	75.3942953020134\\
3576.80000000006	75.4004864547513\\
3576.89999999997	75.398378529494\\
3577.40000000004	75.3962264150943\\
3578.30000000003	75.3940308517773\\
3578.80000000005	75.3918801866495\\
3579.30000000005	75.3897301223669\\
3580.20000000005	75.3875373572047\\
3581.59999999995	75.3971577742413\\
3581.7	75.3950527667653\\
3582.19999999998	75.3984870055551\\
3582.29999999998	75.3963823135328\\
3582.90000000001	75.3921295004186\\
3583.99999999995	75.3996819285176\\
3584.09999999999	75.3975782601417\\
3584.60000000001	75.401009847407\\
3584.69999999996	75.3989064940861\\
3585.20000000005	75.3967589880903\\
3585.79999999998	75.4008756518587\\
3585.90000000006	75.398773006135\\
3586.80000000003	75.4049457749031\\
3586.99999999995	75.4007415460957\\
3587.40000000003	75.4034843205575\\
3587.50000000001	75.4013825398595\\
3588.30000000001	75.4068665700591\\
3588.40000000004	75.4047652222377\\
3589.1	75.4095620193915\\
3589.20000000006	75.4074610648316\\
3589.59999999997	75.4046299133632\\
3589.89999999996	75.4011142061281\\
3590.19999999999	75.4031696515612\\
3590.30000000002	75.4010695187166\\
3591.19999999995	75.3988806281848\\
3591.89999999998	75.4036748329621\\
3591.99999999997	75.4015756799644\\
3592.80000000001	75.407052798575\\
3592.99999999998	75.4028554729899\\
3593.90000000001	75.4090150250417\\
3594.00000000004	75.4069168915723\\
3595.50000000002	75.4116141951274\\
3595.59999999995	75.4095169229913\\
3596.09999999999	75.4073744508092\\
3596.60000000004	75.4107932271249\\
3596.69999999998	75.4086966192171\\
3597.70000000005	75.4155317138251\\
3597.8	75.41343561522\\
3598.29999999998	75.4112939084037\\
3598.99999999995	75.4160762412826\\
3599.09999999995	75.413980884641\\
3600.30000000004	75.4166203755138\\
3600.49999999998	75.4124312614564\\
3601.20000000005	75.4088801266209\\
3601.70000000003	75.4122938530735\\
3601.80000000005	75.4102001721314\\
3602.49999999995	75.4149780713929\\
3602.99999999999	75.4045127806611\\
3604.10000000002	75.4092447699906\\
3604.19999999999	75.4071525677663\\
3604.89999999995	75.4119278779473\\
3605.00000000001	75.4098360655738\\
3605.39999999995	75.4125641381223\\
3605.50000000003	75.4104725981806\\
3606.19999999999	75.4152455425228\\
3606.29999999997	75.4131543921917\\
3606.6	75.4151994898384\\
3606.7	75.413108572696\\
3607.10000000002	75.410290530051\\
3607.70000000001	75.4143799545429\\
3607.79999999998	75.412289697608\\
3608.59999999995	75.4177404605537\\
3608.80000000003	75.4135609188395\\
3609.09999999997	75.4156045661088\\
3609.19999999997	75.4135150860278\\
3610.19999999998	75.4203251807329\\
3610.29999999998	75.4182362065145\\
3610.60000000006	75.4202786163348\\
3610.69999999995	75.41818987482\\
3611.00000000004	75.4202320622525\\
3611.20000000003	75.4160551601916\\
3611.50000000004	75.4180972422195\\
3611.60000000002	75.4160090815959\\
3611.90000000003	75.4125138427464\\
3612.89999999997	75.4165513423748\\
3612.99999999999	75.4144640336553\\
3613.30000000001	75.4165052305308\\
3613.49999999997	75.4123311932699\\
3614.39999999999	75.4184534513764\\
3614.49999999997	75.416366956233\\
3614.80000000002	75.4184071481922\\
3614.90000000004	75.4163208852005\\
3615.19999999995	75.4183608552541\\
3615.30000000004	75.4162748243624\\
3615.90000000005	75.4203539823009\\
3615.99999999996	75.4182683001023\\
3616.7	75.4202610042026\\
3616.8	75.418175785894\\
3617.69999999998	75.4242910055835\\
3617.90000000004	75.4201216141515\\
3618.50000000003	75.4241972033383\\
3618.59999999995	75.4221129134772\\
3619.00000000004	75.4193031416651\\
3619.50000000005	75.4171731683059\\
3620.2	75.4219263596939\\
3620.29999999996	75.4198431112584\\
3620.70000000006	75.4225585505965\\
3620.79999999998	75.4204755723715\\
3621.20000000003	75.4231905669235\\
3621.40000000004	75.4190252657739\\
3622.80000000002	75.4285241105192\\
3622.89999999996	75.4264421749931\\
3623.39999999995	75.429833034359\\
3623.80000000002	75.4215072159828\\
3624.79999999997	75.4282876769014\\
3624.89999999995	75.4262068965517\\
3625.19999999997	75.4282404214824\\
3625.29999999997	75.4261598720141\\
3626.30000000004	75.4329362453121\\
3626.40000000002	75.4308561974355\\
3627.40000000005	75.4376292212267\\
3627.50000000001	75.4355496747161\\
3628.09999999996	75.4396119287801\\
3628.19999999999	75.4375327288262\\
3628.49999999999	75.4395634680042\\
3628.60000000004	75.4374844985808\\
3629.10000000005	75.4408685109666\\
3629.20000000001	75.4387898492822\\
3629.50000000005	75.4353096759974\\
3630.29999999998	75.4324592331424\\
3631.30000000004	75.4392245414991\\
3631.39999999996	75.4371471843591\\
3631.99999999996	75.4412048126428\\
3632.09999999997	75.4391278013325\\
3632.40000000001	75.4411562284928\\
3632.5	75.4390794472279\\
3632.89999999994	75.4417836498761\\
3633.00000000006	75.4397071371556\\
3633.49999999996	75.4348304711581\\
3634.00000000004	75.438210285903\\
3634.10000000005	75.4361345000275\\
3634.40000000001	75.4326592378594\\
3634.90000000002	75.4360385144429\\
3634.99999999999	75.4339633022475\\
3635.59999999996	75.4380174381825\\
3635.70000000001	75.4359425710985\\
3636.00000000002	75.4324688539919\\
3636.80000000002	75.4268745360059\\
3637.3	75.4247539451256\\
3638.09999999995	75.427409158375\\
3638.2	75.4253360085754\\
3638.60000000002	75.4225410173963\\
3639.39999999996	75.4197005083116\\
3640.10000000005	75.4161859238503\\
3640.49999999998	75.4188869966489\\
3640.6	75.4168154475788\\
3641.30000000006	75.4215411654858\\
3641.69999999997	75.4132571805151\\
3642.19999999995	75.4111413118085\\
3643.5	75.4199143704029\\
3643.70000000002	75.4157747406554\\
3644.79999999999	75.4204504924689\\
3644.89999999997	75.4183813443073\\
3645.40000000005	75.4162666300919\\
3645.79999999999	75.4134781535423\\
3646.19999999998	75.4079477826838\\
3646.70000000004	75.4113195129977\\
3646.79999999998	75.4092516932189\\
3647.40000000003	75.4132967786155\\
3647.49999999996	75.4112293014585\\
3647.8	75.4077688533129\\
3648.5	75.4124869813079\\
3648.79999999996	75.4062868261668\\
3649.49999999995	75.4027838667251\\
3649.9	75.4\\
3650.50000000005	75.4040431709856\\
3650.60000000003	75.4019777029063\\
3651.40000000001	75.407366835547\\
3651.49999999996	75.4053017855187\\
3651.99999999999	75.4031926836615\\
3653.2	75.4003229956478\\
3653.49999999995	75.3968688416904\\
3653.89999999995	75.3995621237001\\
3653.99999999999	75.3974987000903\\
3654.39999999996	75.3947188397866\\
3654.99999999998	75.3987578999207\\
3655.10000000004	75.3966951192821\\
3656.50000000003	75.3924410654707\\
3656.80000000006	75.3944597883453\\
3656.89999999999	75.3923981405524\\
3657.59999999996	75.3971074719086\\
3657.69999999998	75.3950462026355\\
3658.50000000005	75.3976931066528\\
3658.60000000002	75.3956323284227\\
3659.20000000002	75.3996666029022\\
3659.30000000002	75.3976061649451\\
3659.59999999996	75.3996229199115\\
3659.70000000002	75.3975627083447\\
3661.19999999996	75.4049108240243\\
3661.39999999999	75.4007920251263\\
3662.59999999997	75.4088513937805\\
3662.70000000003	75.4067926176695\\
3663.50000000006	75.4094333442516\\
3663.59999999999	75.4073750580015\\
3665.50000000001	75.4201222173723\\
3665.59999999996	75.4180647625283\\
3666.00000000004	75.4207468426939\\
3666.10000000004	75.4186896514102\\
3666.8	75.4097466524858\\
3667.49999999994	75.3980804886029\\
3667.80000000002	75.3946399847324\\
3668.2	75.3973230106589\\
3668.30000000004	75.3952676916367\\
3669.00000000001	75.3999618435039\\
3669.10000000001	75.3979069006868\\
3669.49999999995	75.3951384347068\\
3669.79999999995	75.3971497860977\\
3670.00000000004	75.3930410615515\\
3671.6	75.3983168559523\\
3671.70000000001	75.3962634130399\\
3671.99999999996	75.3982734674982\\
3672.09999999998	75.3962202494418\\
3672.39999999999	75.3982300884956\\
3672.50000000005	75.3961770952459\\
3672.89999999997	75.3988565205554\\
3672.99999999998	75.3968037897144\\
3673.29999999999	75.3988130886917\\
3673.39999999998	75.3967605825507\\
3673.70000000001	75.3987696662856\\
3673.80000000001	75.3967173847955\\
3676.29999999999	75.4080078337504\\
3676.40000000005	75.405956752346\\
3676.89999999996	75.4093010606473\\
3677.09999999996	75.4051996083977\\
3677.89999999997	75.4078303425775\\
3677.99999999997	75.4057801582339\\
3678.29999999995	75.4077859939104\\
3678.40000000004	75.4057360337094\\
3679.69999999995	75.4035545410077\\
3680	75.4055596315317\\
3680.09999999999	75.4035106787675\\
3681.69999999997	75.4087674507035\\
3681.80000000005	75.4067193568538\\
3682.10000000003	75.4087230460051\\
3682.20000000002	75.4066751758412\\
3683.40000000005	75.4038278810914\\
3683.69999999999	75.4058309354471\\
3683.80000000002	75.4037840332257\\
3684.29999999999	75.4071219194441\\
3684.4	75.4050753155109\\
3684.70000000005	75.4070777247069\\
3684.79999999995	75.4050313441342\\
3685.09999999999	75.4070335395637\\
3685.19999999996	75.4049873823027\\
3685.50000000004	75.4069893640113\\
3685.6	75.4049434300133\\
3687.19999999999	75.4020557047162\\
3687.60000000001	75.3993003769287\\
3687.90000000004	75.4013015184382\\
3688	75.3992570700361\\
3688.39999999997	75.396502643351\\
3688.90000000005	75.3998373542966\\
3688.99999999995	75.3977934997696\\
3689.40000000003	75.3950399783168\\
3689.89999999998	75.3929539295393\\
3690.7	75.3982876341173\\
3690.90000000006	75.3942021132484\\
3691.80000000005	75.4002004387984\\
3692.00000000006	75.3961160315268\\
3693.5	75.4061078622482\\
3693.70000000005	75.4020250148898\\
3694.19999999999	75.3999404487995\\
3694.80000000005	75.4039351538607\\
3694.89999999995	75.4018944519621\\
3695.39999999995	75.4052225679881\\
3695.50000000002	75.4031821625717\\
3695.79999999997	75.4051787115452\\
3695.90000000004	75.4031385281385\\
3696.40000000004	75.4064655755444\\
3696.49999999994	75.4044256884705\\
3696.90000000004	75.4016770354341\\
3697.80000000005	75.4049595716488\\
3698.00000000004	75.4008815337606\\
3698.70000000006	75.3974261922786\\
3699.10000000004	75.4000865051903\\
3699.40000000002	75.3939721583998\\
3699.90000000003	75.3918918918919\\
3701	75.3938018426954\\
3701.1	75.3917648330271\\
3701.69999999995	75.3957534172565\\
3701.80000000004	75.3937167400524\\
3703.19999999995	75.4003186347312\\
3703.30000000006	75.3982826591781\\
3703.60000000003	75.3948753948754\\
3703.99999999995	75.3921330417645\\
3705.3	75.3980676850003\\
3705.39999999999	75.3960329240318\\
3705.70000000005	75.3926277726807\\
3706.09999999995	75.3898872160164\\
3706.70000000006	75.3938707240747\\
3706.79999999997	75.3918368448029\\
3707.69999999994	75.3951130050165\\
3707.79999999996	75.393079640767\\
3708.99999999999	75.3956485400771\\
3709.10000000003	75.3936158740429\\
3709.49999999999	75.3962691395299\\
3709.60000000004	75.3942367307329\\
3710.49999999999	75.3975098366841\\
3710.59999999997	75.3954779421672\\
3710.99999999997	75.3981299345208\\
3711.09999999999	75.3960982970468\\
3711.69999999996	75.4000754350989\\
3711.80000000003	75.398044128344\\
3712.79999999999	75.4019768913787\\
3713.10000000005	75.3958849509857\\
3714.10000000003	75.4025092886759\\
3714.20000000002	75.400479228926\\
3714.50000000005	75.4024659451893\\
3714.69999999995	75.398406374502\\
3715.50000000006	75.3956292388847\\
3716.90000000001	75.3995157384988\\
3716.99999999998	75.3974872884776\\
3717.39999999997	75.4001344989912\\
3717.50000000001	75.3981063051431\\
3717.79999999995	75.4000914494742\\
3717.89999999997	75.3980634749866\\
3718.50000000004	75.4020330231808\\
3718.60000000003	75.4000053782236\\
3718.99999999999	75.3972735339195\\
3720.3	75.4004945704763\\
3720.4	75.3984679478565\\
3720.70000000001	75.4004515158031\\
3720.80000000001	75.398425112204\\
3721.30000000005	75.3963562100285\\
3721.59999999996	75.3983394685225\\
3721.69999999997	75.3963136116933\\
3722.29999999996	75.4002793896411\\
3722.39999999998	75.3982538616521\\
3723.79999999996	75.4075028867585\\
3723.89999999998	75.4054779806659\\
3725.1	75.4026629442714\\
3726.00000000001	75.4086041705805\\
3726.10000000005	75.4065804304654\\
3726.49999999996	75.4038533784146\\
3726.89999999998	75.406493158036\\
3726.99999999995	75.4044699632422\\
};
\addplot [color=mycolor2, forget plot]
  table[row sep=crcr]{%
3726.99999999995	75.4044699632422\\
3727.30000000002	75.4064495358695\\
3727.40000000006	75.4044265593561\\
3728.29999999997	75.4076815792297\\
3728.40000000003	75.4056591122435\\
3728.69999999996	75.4076378459558\\
3728.79999999995	75.405615597093\\
3729.1	75.4075941220637\\
3729.30000000002	75.403550168928\\
3729.80000000002	75.3988042574868\\
3731.20000000003	75.4053547021146\\
3731.30000000006	75.4033338693252\\
3731.80000000001	75.4012701304965\\
3732.30000000005	75.4045654270711\\
3732.40000000001	75.4025452109846\\
3732.80000000004	75.4051809585041\\
3732.89999999999	75.4031609965175\\
3733.50000000001	75.4071137775873\\
3733.59999999995	75.4050941425396\\
3734.00000000002	75.4023727270293\\
3734.59999999994	75.3982916967896\\
3734.99999999995	75.4009263473535\\
3735.10000000004	75.3989076890126\\
3735.49999999996	75.4015419209765\\
3735.59999999998	75.3995235163423\\
3736.00000000002	75.4021573298359\\
3736.10000000005	75.4001391788448\\
3737.60000000006	75.4073360622843\\
3737.69999999997	75.4053186366312\\
3738.09999999995	75.407950350436\\
3738.19999999999	75.4059331781826\\
3738.50000000004	75.4079067030439\\
3738.59999999996	75.4058897477733\\
3739.09999999995	75.4038296961917\\
3739.40000000003	75.4004546062308\\
3739.90000000004	75.403743315508\\
3740.1	75.3997112453879\\
3740.40000000003	75.4016842668093\\
3740.50000000005	75.3996685023793\\
3741.2	75.3962526394569\\
3742.09999999997	75.4021698466143\\
3742.19999999998	75.4001549849023\\
3742.60000000003	75.3974403505491\\
3743.00000000006	75.4000694611418\\
3743.09999999995	75.3980551399872\\
3743.69999999995	75.401997969977\\
3743.9	75.3979700854701\\
3744.19999999995	75.3999412440242\\
3744.30000000004	75.3979275718406\\
3744.69999999994	75.3952146977142\\
3745.49999999999	75.4004698846647\\
3745.59999999998	75.3984568972422\\
3746.19999999995	75.3943891306089\\
3746.70000000004	75.3923347923561\\
3747.49999999998	75.3975877895186\\
3747.89999999995	75.3895410885806\\
3748.69999999996	75.3947930004268\\
3748.8	75.3927818826856\\
3749.19999999998	75.3900728135919\\
3750.29999999995	75.3946245733788\\
3750.39999999999	75.3926143180909\\
3751.40000000003	75.3965080634413\\
3751.49999999999	75.3944983473718\\
3752.10000000005	75.3904376099355\\
3752.99999999999	75.3856811702326\\
3754.09999999999	75.3928932928453\\
3754.19999999999	75.3908851183976\\
3754.8	75.394817438547\\
3755.09999999997	75.3887942053686\\
3755.49999999998	75.3914154862073\\
3755.59999999997	75.3894080996885\\
3756.60000000005	75.3959592195278\\
3756.80000000003	75.3919454869706\\
3758.30000000001	75.401766709238\\
3758.40000000005	75.3997605427697\\
3758.9	75.4030327214685\\
3759.00000000006	75.4010268415312\\
3759.40000000003	75.3983242452454\\
3760.50000000005	75.4028612455459\\
3760.60000000004	75.4008562235754\\
3761.20000000002	75.4047802621434\\
3761.30000000003	75.4027755622906\\
3761.80000000001	75.4007283553524\\
3762.09999999995	75.402689915475\\
3762.19999999996	75.4006857507376\\
3763.30000000002	75.4078758569379\\
3763.49999999995	75.4038686364119\\
3764.00000000002	75.4071358359236\\
3764.10000000003	75.4051325646884\\
3764.5	75.4077458428518\\
3764.59999999994	75.4057428214732\\
3765.10000000004	75.4036970147668\\
3766.69999999996	75.4114898587661\\
3766.90000000003	75.4074860631803\\
3767.80000000004	75.4107062289339\\
3767.89999999997	75.4087048832272\\
3768.59999999999	75.4106190463555\\
3768.70000000003	75.408618127786\\
3769.69999999997	75.4124887261924\\
3769.79999999998	75.4104883418658\\
3770.19999999999	75.4130971010264\\
3770.29999999995	75.4110969658392\\
3770.80000000001	75.4090535415948\\
3771.10000000001	75.4110097581672\\
3771.20000000001	75.4090101556492\\
3771.50000000004	75.4056633789373\\
3771.99999999996	75.4036213249914\\
3773.10000000006	75.4081416304463\\
3773.2	75.4061431638089\\
3773.89999999995	75.4107048224695\\
3774.10000000003	75.4067087064808\\
3774.6	75.4046679206294\\
3774.99999999996	75.4072739794972\\
3775.10000000006	75.4052765416402\\
3775.89999999998	75.4104872881356\\
3776.00000000004	75.4084902412542\\
3776.40000000006	75.4031510658017\\
3777.50000000002	75.4103134265142\\
3777.60000000001	75.4083172300606\\
3777.90000000005	75.4102699841186\\
3778.00000000003	75.4082740001588\\
3778.30000000004	75.410226550921\\
3778.4	75.4082307794098\\
3779.89999999996	75.4126984126984\\
3779.99999999994	75.4107034205444\\
3780.39999999996	75.4080148128554\\
3780.70000000004	75.4099661447313\\
3780.90000000003	75.4059772546945\\
3781.19999999996	75.4079284902018\\
3781.29999999999	75.4059343100439\\
3781.59999999998	75.4078853425708\\
3781.69999999994	75.4058913744778\\
3782.09999999996	75.4032044841627\\
3783.19999999996	75.4050696481907\\
3783.30000000003	75.4030765977692\\
3783.69999999996	75.4056768328136\\
3783.9	75.4016913319239\\
3784.30000000001	75.404291301131\\
3784.4	75.4022988505747\\
3784.99999999998	75.4061979868432\\
3785.09999999996	75.4042058543802\\
3785.60000000001	75.4021713289484\\
3788.60000000002	75.4137303032703\\
3788.70000000005	75.4117398648649\\
3789.29999999996	75.4156330817544\\
3789.40000000001	75.4136429608128\\
3790.09999999995	75.4076302042109\\
3790.70000000003	75.4115226337449\\
3790.79999999998	75.4095333561951\\
3791.80000000003	75.4133811545663\\
3791.9	75.4113924050633\\
3792.19999999998	75.4133375524088\\
3792.29999999997	75.4113490138171\\
3793.8	75.407891615488\\
3794.19999999994	75.4104841472735\\
3794.40000000001	75.4065094215312\\
3795.00000000004	75.4103976179811\\
3795.09999999996	75.408410623946\\
3795.40000000004	75.4103543670136\\
3795.49999999995	75.4083675835178\\
3795.8	75.4103111251614\\
3795.90000000003	75.4083245521602\\
3797.09999999996	75.4134625513536\\
3797.2	75.4114765754615\\
3798.09999999996	75.4173029329682\\
3798.19999999994	75.4153173788274\\
3799.50000000001	75.4105695336351\\
3800.30000000002	75.4131144090096\\
3800.50000000005	75.4091459243277\\
3802.80000000002	75.4187593678509\\
3802.89999999997	75.4167762292927\\
3803.40000000001	75.4147495727619\\
3803.70000000005	75.4114306745886\\
3804.19999999995	75.4094051468076\\
3805.80000000001	75.4197430305578\\
3806.00000000003	75.4157799322141\\
3806.50000000002	75.4190090894762\\
3806.6	75.4170278719101\\
3807.10000000006	75.4150031519227\\
3807.40000000005	75.4169402495075\\
3807.50000000001	75.4149595545751\\
3808.50000000006	75.4187890563462\\
3808.59999999994	75.4168088849214\\
3809.19999999994	75.4206809650067\\
3809.39999999998	75.4167213545085\\
3809.80000000003	75.4140528622798\\
3810.30000000001	75.4172790258241\\
3810.39999999997	75.4152998294187\\
3810.69999999999	75.4172352261992\\
3810.80000000004	75.4152562386838\\
3811.19999999995	75.4178364337628\\
3811.29999999998	75.4158576900876\\
3811.59999999999	75.4125455833355\\
3811.90000000001	75.414480587618\\
3811.99999999995	75.4125022953228\\
3812.30000000006	75.4144370999895\\
3812.40000000001	75.4124590163934\\
3812.69999999996	75.4143936214855\\
3812.79999999997	75.4124157465446\\
3813.19999999999	75.4149948863189\\
3813.30000000006	75.4130172549431\\
3813.60000000002	75.4097071085822\\
3814.10000000002	75.4050652823659\\
3814.40000000006	75.4069996067637\\
3814.50000000004	75.4050228071095\\
3816.09999999994	75.4153346260678\\
3816.19999999998	75.4133584885884\\
3816.50000000005	75.4152910967877\\
3816.59999999997	75.4133151675531\\
3816.89999999999	75.4152475766309\\
3817.10000000005	75.4112962380803\\
3818.20000000001	75.4183799072886\\
3818.30000000004	75.4164047768699\\
3819.29999999999	75.4228412839713\\
3819.40000000002	75.4208666055766\\
3820.1	75.4227527354589\\
3820.3	75.4188043136844\\
3821.40000000002	75.4258798900955\\
3821.6	75.4219326477746\\
3821.89999999995	75.4238618524333\\
3821.99999999997	75.4218884906203\\
3822.6	75.4178983441023\\
3824.29999999996	75.4288254366698\\
3824.80000000005	75.4189652017046\\
3825.69999999999	75.4169062679701\\
3826.20000000003	75.4201186524841\\
3826.29999999995	75.418147606105\\
3827.09999999994	75.4154473244147\\
3827.99999999995	75.4081659308795\\
3828.30000000003	75.4100929892383\\
3828.49999999999	75.4061536854203\\
3829.20000000001	75.4080380226151\\
3829.29999999997	75.406068835849\\
3829.89999999995	75.402088772846\\
3830.19999999999	75.404015351278\\
3830.29999999999	75.4020467836257\\
3830.70000000004	75.4046152239741\\
3830.89999999999	75.4006786739755\\
3831.39999999994	75.3960589847318\\
3833.00000000005	75.4063290809006\\
3833.09999999999	75.4043618908484\\
3833.40000000005	75.4062866831877\\
3833.69999999997	75.4003860399604\\
3834.09999999996	75.4029523759846\\
3834.19999999998	75.4009858383538\\
3834.70000000002	75.3963700844894\\
3835.40000000005	75.3930387172468\\
3835.69999999994	75.3949632410449\\
3835.90000000003	75.3910323253389\\
3836.39999999999	75.3942395412485\\
3836.60000000001	75.3903093804572\\
3836.99999999999	75.3928748273436\\
3837.09999999997	75.3909100385698\\
3837.89999999999	75.3856175091193\\
3839.00000000006	75.3926701570681\\
3839.10000000006	75.3907063971661\\
3839.69999999995	75.3945517995729\\
3839.80000000003	75.3925883486549\\
3840.70000000004	75.3983545094772\\
3840.80000000005	75.3963914707491\\
3841.29999999999	75.399593898058\\
3841.40000000001	75.3976311336717\\
3841.89999999998	75.4008328995315\\
3842.00000000004	75.3988704094115\\
3842.70000000004	75.4033517227022\\
3842.79999999996	75.401389575581\\
3844.59999999994	75.4077041121544\\
3844.69999999996	75.4057428214732\\
3844.99999999998	75.4076616993056\\
3845.30000000004	75.4017787486347\\
3846.20000000001	75.3997348100772\\
3846.59999999999	75.4022928744118\\
3846.69999999997	75.400332744099\\
3847.10000000006	75.3976918278228\\
3848.20000000001	75.4021256139074\\
3848.30000000005	75.4001663028791\\
3849.10000000004	75.4052790190169\\
3849.20000000005	75.4033200841711\\
3849.80000000005	75.4071534325567\\
3849.89999999995	75.4051948051948\\
3850.39999999998	75.4005973250227\\
3850.80000000005	75.4031525098029\\
3850.90000000004	75.4011944949364\\
3851.80000000006	75.4069420286093\\
3851.90000000001	75.404984423676\\
3852.79999999996	75.4029432375613\\
3853.30000000004	75.4061348419578\\
3853.39999999998	75.4041780199818\\
3853.79999999998	75.4015412958302\\
3854.19999999995	75.3989051189581\\
3854.50000000006	75.4008197997198\\
3854.59999999999	75.3988637248035\\
3855.00000000005	75.3962283728049\\
3855.39999999996	75.3987809622617\\
3855.60000000002	75.3948699328267\\
3856.80000000001	75.4025253441883\\
3856.90000000004	75.400570391496\\
3857.80000000003	75.3933487130304\\
3858.50000000005	75.3952210646348\\
3858.7	75.3913133616668\\
3859.09999999999	75.3886815920398\\
3859.99999999998	75.3944198336831\\
3860.09999999995	75.3924667115694\\
3860.89999999999	75.3949753949754\\
3861.00000000001	75.3930227137344\\
3861.30000000002	75.3949344797224\\
3861.49999999999	75.3910296250259\\
3862.49999999994	75.3974007145446\\
3862.60000000004	75.3954487793512\\
3862.89999999994	75.3973595651048\\
3863	75.3954078330874\\
3863.40000000003	75.397955221949\\
3863.49999999994	75.3960037270939\\
3864.30000000003	75.393334023393\\
3865.59999999997	75.3990221693354\\
3865.69999999999	75.3970717574629\\
3866.30000000003	75.4008897165322\\
3866.60000000003	75.3950396979336\\
3867.00000000003	75.3924129192418\\
3867.40000000003	75.3897866839043\\
3867.79999999994	75.3845756095039\\
3869.09999999996	75.3825080119921\\
3869.40000000001	75.3844165912909\\
3869.50000000001	75.3824684721935\\
3871.90000000005	75.3925619834711\\
3871.99999999994	75.390614911805\\
3872.29999999999	75.3925214337362\\
3872.39999999994	75.390574564235\\
3873.49999999999	75.3975629904998\\
3873.7	75.3936702978987\\
3874.00000000003	75.395575746625\\
3874.10000000004	75.3936296525734\\
3874.79999999998	75.3980747890268\\
3874.90000000003	75.3961290322581\\
3875.20000000006	75.392872809847\\
3876.40000000003	75.4004901328518\\
3876.49999999996	75.398545116855\\
3878.00000000006	75.4003249013692\\
3878.10000000004	75.3983806920736\\
3878.40000000003	75.3951269820807\\
3878.69999999998	75.3970300092812\\
3878.80000000005	75.3950862357885\\
3880.30000000005	75.396866302443\\
3880.50000000003	75.3929804669381\\
3880.80000000004	75.3948826303177\\
3880.90000000006	75.3929399639268\\
3881.50000000005	75.3967436108821\\
3881.60000000002	75.3948012468764\\
3882.69999999996	75.3991964561657\\
3882.80000000004	75.3972546292719\\
3884.09999999996	75.3951907728747\\
3884.90000000001	75.3925353925354\\
3885.39999999997	75.3957019688586\\
3885.49999999997	75.393761581223\\
3885.99999999995	75.3866344149662\\
3886.60000000004	75.3827154141045\\
3887.50000000003	75.3884144459306\\
3887.60000000001	75.3864752938755\\
3888.00000000001	75.3890074843754\\
3888.10000000001	75.3870685664318\\
3889.20000000001	75.3940297739953\\
3889.30000000002	75.3920913251401\\
3889.59999999996	75.3939892536699\\
3889.70000000007	75.3920510051931\\
3890.20000000004	75.3952137367298\\
3890.30000000001	75.3932757557064\\
3890.80000000005	75.3912976432188\\
3891.09999999994	75.3880550986842\\
3891.39999999997	75.3899524604908\\
3891.50000000006	75.388015212252\\
3893.20000000004	75.3987619757018\\
3893.50000000001	75.3929525374974\\
3893.99999999995	75.3909760920367\\
3894.49999999997	75.3864324962769\\
3894.80000000005	75.3831934067627\\
3895.10000000001	75.385089340727\\
3895.19999999999	75.3831540574539\\
3896.09999999998	75.3785739951748\\
3896.7	75.3823650174502\\
3896.80000000004	75.380430598681\\
3897.39999999995	75.3842206542656\\
3897.59999999999	75.3803525155861\\
3899.09999999995	75.3898235535494\\
3899.20000000004	75.3878901341266\\
3899.99999999997	75.3903746057793\\
3900.10000000006	75.3884416183786\\
3901.00000000002	75.3941196072902\\
3901.40000000002	75.3863898500577\\
3901.89999999994	75.3895438236802\\
3901.99999999998	75.387611798775\\
3902.50000000002	75.3907651309383\\
3902.59999999997	75.3888333717683\\
3903.40000000004	75.3913154861022\\
3903.49999999999	75.3893841582129\\
3903.90000000004	75.3919057377049\\
3904.00000000004	75.389974642043\\
3904.30000000005	75.3918655875423\\
3904.39999999997	75.3899346907415\\
3904.90000000004	75.3828425096031\\
3906.30000000001	75.3916649600655\\
3906.40000000001	75.3897350569564\\
3907.20000000004	75.3947738847798\\
3907.30000000003	75.3928443466244\\
3908.00000000004	75.3972518615184\\
3908.09999999998	75.3953226549307\\
3908.50000000002	75.3927237373996\\
3909.00000000001	75.3958711723926\\
3909.09999999999	75.393942494628\\
3910.79999999999	75.3892965813496\\
3911.09999999998	75.3911842912661\\
3911.20000000001	75.3892567688492\\
3911.49999999997	75.3860312915431\\
3912.40000000001	75.3840255591054\\
3913.10000000002	75.3858734539507\\
3913.20000000002	75.3839470523599\\
3913.90000000001	75.3883495145631\\
3914.09999999994	75.3844974707475\\
3914.40000000006	75.3863839570826\\
3914.49999999997	75.3844581821897\\
3915.59999999996	75.3862655463902\\
3915.70000000004	75.3843403646764\\
3916.59999999994	75.3899966808793\\
3916.69999999998	75.3880718954248\\
3918.30000000001	75.3955696202532\\
3918.39999999996	75.3936455276254\\
3919.50000000001	75.3979997958975\\
3919.60000000001	75.3960762303237\\
3920.10000000005	75.3992143257997\\
3920.30000000003	75.3953678196102\\
3920.60000000004	75.3972504909838\\
3920.69999999995	75.3953274841869\\
3921.29999999996	75.3990921609629\\
3921.40000000003	75.3971694504654\\
3921.79999999998	75.3945791580611\\
3922.69999999994	75.3976751300092\\
3922.79999999996	75.3957531418084\\
3923.19999999997	75.3931639181301\\
3923.49999999994	75.3899480069324\\
3924.10000000002	75.3937108200397\\
3924.20000000006	75.3917896185307\\
3924.60000000004	75.3942976533238\\
3924.69999999999	75.3923766816144\\
3926.00000000006	75.4005246937164\\
3926.20000000003	75.3966839008736\\
3926.49999999998	75.3985636428462\\
3926.59999999999	75.3966434919907\\
3927.5	75.3997352072513\\
3927.60000000001	75.3978155154416\\
3928.59999999999	75.4040776847303\\
3928.70000000001	75.4021584198738\\
3928.99999999998	75.4040365478099\\
3929.10000000003	75.4021174793851\\
3929.40000000004	75.4039954192645\\
3929.49999999996	75.4020765472313\\
3929.90000000001	75.4045801526718\\
3930.00000000004	75.4026615098852\\
3931.20000000004	75.4076259761402\\
3931.30000000007	75.405707890319\\
3931.7	75.403123251437\\
3932.19999999998	75.4062507947003\\
3932.29999999994	75.4043332316143\\
3932.70000000004	75.406834825061\\
3932.90000000006	75.4030002542588\\
3933.70000000005	75.4054603691088\\
3933.9	75.401626842908\\
3934.39999999998	75.3996695895285\\
3935.89999999995	75.4065040650407\\
3935.99999999996	75.4045882980615\\
3936.80000000003	75.4070461530646\\
3936.90000000004	75.4051308102616\\
3937.50000000007	75.4012596505486\\
3937.9	75.3986795327577\\
3938.50000000001	75.4024272584167\\
3938.60000000002	75.400512859573\\
3939.29999999997	75.4048839924862\\
3939.4	75.4029699200406\\
3940.80000000006	75.4117079854855\\
3940.90000000004	75.409794468409\\
3941.19999999999	75.4116662015071\\
3941.39999999995	75.4078396549537\\
3941.7	75.4097112994064\\
3941.79999999998	75.4077982698699\\
3942.19999999997	75.4052203028689\\
3942.90000000004	75.4070504691859\\
3943.1	75.4032258064516\\
3943.89999999999	75.4006085192698\\
3945.30000000003	75.4068028590257\\
3945.69999999998	75.3991585990167\\
3947.50000000003	75.4103759246124\\
3947.59999999994	75.4084656888821\\
3948.19999999999	75.412202720158\\
3948.29999999997	75.410292776821\\
3948.69999999998	75.4077188006483\\
3949.00000000001	75.4095869945051\\
3949.09999999998	75.4076775043047\\
3950.00000000004	75.4132806764386\\
3950.09999999994	75.4113715761227\\
3950.4	75.4132388305278\\
3950.49999999997	75.4113299245684\\
3950.90000000001	75.4087572766388\\
3951.70000000004	75.4061440356293\\
3952.19999999995	75.4092553702907\\
3952.30000000002	75.4073474344702\\
3952.80000000004	75.4053985681398\\
3953.09999999997	75.4072650005059\\
3953.30000000003	75.4034501947691\\
3953.59999999994	75.4002579861902\\
3954.39999999997	75.4052345429258\\
3954.50000000003	75.4033277701917\\
3955.6	75.4101676062391\\
3955.69999999997	75.4082612872238\\
3956.09999999997	75.4107476871746\\
3956.2	75.408841594419\\
3956.50000000006	75.4107061618561\\
3956.59999999997	75.4088002628453\\
3957.00000000004	75.4062318364459\\
3957.30000000005	75.4080962247941\\
3957.50000000006	75.4042854255104\\
3958.20000000002	75.4061086830205\\
3958.29999999999	75.4042037186742\\
3958.89999999998	75.4079312957818\\
3959.09999999995	75.4041220448575\\
3959.79999999998	75.4008939619687\\
3960.89999999995	75.4077253218884\\
3960.99999999994	75.4058216152079\\
3961.40000000003	75.4083049349994\\
3961.59999999994	75.4044980690108\\
3962.40000000006	75.4094637223975\\
3962.5	75.4075606924746\\
3962.99999999995	75.4056168151195\\
3963.69999999995	75.4074373076341\\
3963.89999999997	75.4036326942482\\
3964.5	75.4073550925692\\
3964.70000000006	75.4035512510089\\
3965.09999999998	75.4060324825986\\
3965.19999999999	75.4041308349936\\
3965.60000000004	75.4066116952871\\
3965.69999999995	75.4047102728327\\
3966.40000000004	75.4014874574562\\
3967.29999999997	75.3944648888441\\
3967.80000000001	75.3900047884271\\
3968.89999999995	75.3943058704964\\
3969.00000000001	75.3924063389685\\
3969.30000000006	75.3942661359399\\
3969.40000000001	75.3923667968258\\
3969.99999999994	75.3960857409133\\
3970.10000000004	75.3941866908468\\
3971.50000000001	75.400342431262\\
3971.60000000003	75.398443991238\\
3972.00000000002	75.3958863069913\\
3972.60000000007	75.3996022855992\\
3972.69999999997	75.397704389851\\
3973.4	75.3995218321379\\
3973.50000000001	75.3976243205154\\
3973.8	75.3944487782783\\
3974.99999999999	75.3993610223642\\
3975.09999999995	75.3974642785269\\
3975.39999999999	75.3993208401459\\
3975.60000000004	75.3955278315768\\
3976.00000000003	75.3929730137572\\
3976.29999999997	75.3948294940147\\
3976.40000000003	75.3929334842198\\
3977.19999999997	75.3978829859452\\
3977.30000000001	75.3959873284055\\
3977.69999999999	75.3934335562371\\
3978.00000000006	75.3952892084161\\
3978.09999999994	75.3933939972852\\
3978.49999999998	75.3958678932288\\
3978.6	75.393972905723\\
3978.89999999995	75.3958280975119\\
3979.09999999999	75.3920386007238\\
3979.49999999995	75.3945120112574\\
3979.60000000002	75.392617533985\\
3980.00000000006	75.3950905756137\\
3980.10000000004	75.3931963217929\\
3981.3	75.4006128497514\\
3981.50000000007	75.3968253968254\\
3981.99999999995	75.394892142337\\
3982.89999999995	75.4004519206628\\
3982.99999999996	75.3985589114007\\
3983.80000000004	75.4034990838123\\
3983.99999999995	75.3997138626038\\
3984.99999999994	75.4058869288098\\
3985.09999999998	75.4039947806885\\
3985.60000000006	75.4020623729834\\
3986.29999999995	75.4038731687738\\
3986.50000000006	75.4000903025134\\
3987.00000000001	75.4031752401495\\
3987.10000000005	75.4012841091493\\
3988.00000000006	75.3993129560442\\
3989.3	75.4073294229709\\
3989.39999999996	75.405439278105\\
3989.70000000006	75.4022758032984\\
3990.00000000003	75.4041252098945\\
3990.30000000001	75.3984562951083\\
3990.79999999995	75.3965270991506\\
3991.69999999995	75.402074252217\\
3991.90000000002	75.3982965931864\\
3992.20000000001	75.4001452796633\\
3992.30000000005	75.3982566877066\\
3993.1	75.4031854152059\\
3993.19999999996	75.4012971727644\\
3993.69999999998	75.3968651409685\\
3994.60000000002	75.4024081908529\\
3994.69999999998	75.4005206768799\\
3995.60000000006	75.4060615161298\\
3995.79999999999	75.4022873445281\\
3996.20000000001	75.3997447639066\\
3996.60000000003	75.3972026922211\\
3996.89999999994	75.3990492869652\\
3997.00000000002	75.3971629431338\\
3997.89999999998	75.4027013506753\\
3998.09999999995	75.3989295182832\\
3999.8	75.4018850471262\\
3999.89999999997	75.4\\
4001.39999999994	75.4092215419218\\
4001.59999999999	75.4054526826099\\
4002.70000000002	75.4072149495353\\
4002.79999999999	75.4053311349272\\
4004.80000000004	75.4126195410622\\
4005.19999999997	75.4050882580581\\
4006.09999999994	75.4081174180021\\
4006.49999999997	75.4005890281036\\
4007.99999999995	75.4097951647913\\
4008.10000000002	75.4079137767577\\
4008.4	75.4097542721716\\
4008.6	75.4059919674708\\
4009.99999999997	75.4120844866712\\
4010.29999999996	75.4064432475564\\
4010.80000000005	75.4020294696951\\
4011.20000000005	75.4044823373968\\
4011.29999999996	75.4026025826395\\
4011.70000000006	75.4050550874919\\
4011.90000000006	75.401296111665\\
4012.29999999998	75.4037483800219\\
4012.40000000003	75.4018691588785\\
4013.99999999996	75.4116738496799\\
4014.09999999995	75.4097952269443\\
4014.80000000007	75.4140825425291\\
4014.99999999996	75.4103260192772\\
4015.89999999995	75.4058764940239\\
4016.80000000001	75.4088974084493\\
4016.89999999995	75.4070201643017\\
4017.40000000001	75.4100808960797\\
4017.50000000001	75.4082039028276\\
4018.60000000003	75.4124468111578\\
4018.70000000003	75.4105703194984\\
4019.40000000002	75.4123647219803\\
4019.49999999998	75.4104886058314\\
4020.99999999994	75.4196612867126\\
4021.19999999997	75.4159102777709\\
4022.80000000001	75.4182306296453\\
4022.90000000006	75.4163559532687\\
4023.3	75.4138290997664\\
4023.90000000006	75.4174950298211\\
4024.00000000003	75.4156208841729\\
4025.70000000003	75.4110984152218\\
4026.40000000007	75.4128896063579\\
4026.49999999999	75.4110167386877\\
4026.89999999999	75.4134591507325\\
4027.00000000005	75.4115865014527\\
4027.30000000003	75.4134180861101\\
4027.39999999998	75.4115456238361\\
4027.90000000004	75.409632571996\\
4028.69999999997	75.4120333598094\\
4028.79999999997	75.410161582566\\
4029.39999999997	75.4138230549696\\
4029.49999999996	75.4119515584673\\
4032.20000000003	75.4259355702701\\
4032.30000000006	75.4240650729094\\
4032.7	75.4215433445745\\
4033.00000000005	75.4233716000099\\
4033.30000000004	75.4177616898894\\
4033.79999999996	75.4208086467191\\
4033.90000000003	75.4189390183441\\
4034.49999999995	75.4225945570812\\
4034.60000000002	75.4207252088136\\
4034.89999999996	75.4225526641883\\
4034.99999999996	75.4206835022676\\
4035.40000000002	75.4181637963078\\
4035.90000000005	75.4212091179385\\
4036.00000000001	75.4193404524169\\
4036.49999999997	75.4174305108259\\
4036.99999999995	75.4204750935077\\
4037.1	75.4186069553156\\
4038.20000000005	75.422826436867\\
4038.29999999997	75.4209587955626\\
4038.79999999994	75.4165738196044\\
4040.09999999995	75.4244839364388\\
4040.20000000003	75.4226171323912\\
4040.6	75.4250501150791\\
4040.69999999996	75.4231835280143\\
4041.2	75.4262242347759\\
4041.40000000002	75.4224916491402\\
4042.30000000006	75.4205422521274\\
4043.2	75.4235401775777\\
4043.30000000003	75.4216748281149\\
4044.80000000004	75.4283171401024\\
4044.89999999995	75.4264524103832\\
4045.39999999995	75.4245457916203\\
4045.89999999995	75.427582797825\\
4045.99999999997	75.4257185932132\\
4046.69999999996	75.4299693585055\\
4046.8	75.4281054634412\\
4047.20000000007	75.4255923702221\\
4047.70000000004	75.4236869410544\\
4048.09999999999	75.4211748431402\\
4048.49999999996	75.4236032208665\\
4048.60000000005	75.421740311705\\
4048.99999999994	75.4241683337038\\
4049.10000000004	75.4223056406204\\
4049.49999999996	75.4247333069933\\
4049.60000000006	75.422870829938\\
4050.09999999999	75.4209668658338\\
4051.70000000007	75.4232686707143\\
4051.90000000002	75.4195459032576\\
4053.39999999998	75.4212409029234\\
4053.49999999998	75.4193803039274\\
4053.89999999995	75.421805624075\\
4054.00000000006	75.4199452406206\\
4054.79999999997	75.4223285407778\\
4054.89999999997	75.4204685573366\\
4055.30000000003	75.4228929328796\\
4055.39999999999	75.4210331648379\\
4055.70000000004	75.4228512254056\\
4055.80000000002	75.4209916418058\\
4056.59999999998	75.4258387359184\\
4056.70000000006	75.4239794912246\\
4058.49999999996	75.434879022323\\
4058.59999999999	75.4330204252593\\
4060.69999999997	75.4358747044917\\
4060.79999999995	75.4340170898077\\
4061.40000000004	75.4376461898313\\
4061.49999999994	75.4357888516841\\
4061.80000000001	75.437603092149\\
4061.90000000006	75.4357459379616\\
4062.20000000006	75.437560002954\\
4062.30000000006	75.43570303269\\
4062.60000000004	75.4375169222438\\
4062.79999999994	75.433803440892\\
4063.39999999995	75.427587055494\\
4063.99999999996	75.4312147831008\\
4064.20000000005	75.4275028910267\\
4064.50000000002	75.4293165379127\\
4064.70000000003	75.4256051958276\\
4065.1	75.4231034143462\\
4065.50000000007	75.4255214482487\\
4065.60000000001	75.4236662813292\\
4065.89999999994	75.4254795868175\\
4066.09999999995	75.4217697112783\\
4066.89999999999	75.4266043766904\\
4067.09999999995	75.4228953579858\\
4068.80000000004	75.4282484209491\\
4068.9	75.4263946915704\\
4069.60000000003	75.4306214217264\\
4069.70000000002	75.4287679984274\\
4070.50000000002	75.4335970127254\\
4070.59999999998	75.4317439261061\\
4070.90000000005	75.4335544092361\\
4071.10000000004	75.42984869326\\
4071.99999999995	75.4328233589548\\
4072.10000000003	75.4309709739207\\
4072.89999999998	75.4357967100417\\
4072.99999999994	75.4339446613145\\
4073.39999999998	75.4363569412053\\
4073.50000000003	75.4345051060487\\
4074.00000000004	75.4326108833853\\
4074.59999999999	75.426411760375\\
4075.39999999998	75.4287817445712\\
4075.49999999997	75.4269310040239\\
4075.80000000001	75.4238327731299\\
4076.29999999999	75.42194092827\\
4076.69999999997	75.424352433281\\
4076.79999999997	75.422502391523\\
4077.70000000001	75.4205699151503\\
4078.39999999996	75.4247885251931\\
4078.50000000006	75.4229392438582\\
4078.90000000005	75.4204461877911\\
4079.30000000002	75.4228563023974\\
4079.4	75.4210074764064\\
4080.49999999996	75.4276331911974\\
4080.70000000005	75.4239364830425\\
4080.99999999996	75.4257430594693\\
4081.30000000002	75.4201989513402\\
4081.80000000007	75.4158602611529\\
4082.19999999998	75.418269113\\
4082.29999999996	75.416421712718\\
4082.80000000004	75.414533787259\\
4083.19999999995	75.4095951803688\\
4084.19999999999	75.413167495042\\
4084.30000000001	75.4113211242777\\
4084.89999999995	75.4149326805386\\
4084.99999999999	75.4130865829478\\
4085.70000000001	75.4148514366832\\
4085.80000000003	75.413005702538\\
4086.19999999996	75.4154124758339\\
4086.30000000004	75.413566953798\\
4086.79999999997	75.4116812253786\\
4087.29999999994	75.4097959583109\\
4087.59999999999	75.4116006556254\\
4087.70000000001	75.4097558588972\\
4088.2	75.4127632512291\\
4088.29999999997	75.4109186968007\\
4090.29999999994	75.4082730295326\\
4090.70000000006	75.4057885988071\\
4091.10000000003	75.4033046538913\\
4091.40000000003	75.4002199682268\\
4091.79999999995	75.4026246975732\\
4091.89999999995	75.4007820136852\\
4092.59999999994	75.3952158721626\\
4093.20000000002	75.398822465981\\
4093.39999999996	75.3951386344204\\
4094.30000000002	75.3981047284095\\
4094.40000000006	75.3962632800098\\
4095.30000000005	75.4016701665283\\
4095.40000000004	75.3998290806983\\
4096.40000000001	75.4058342487489\\
4096.49999999996	75.4039935556315\\
4097.19999999995	75.4081956410319\\
4097.29999999999	75.4063552496705\\
4097.69999999996	75.4087559178096\\
4097.80000000006	75.4069157373289\\
4098.89999999996	75.4135154915833\\
4099.1	75.4098360655738\\
4099.90000000001	75.4073170731707\\
4101.10000000005	75.4120745147762\\
4101.19999999996	75.4102357788994\\
4101.49999999998	75.4120343280671\\
4101.70000000006	75.4083573065483\\
4102.20000000002	75.4113546059528\\
4102.29999999998	75.4095163806552\\
4102.59999999997	75.4113145002072\\
4102.70000000002	75.4094764551038\\
4103.10000000004	75.4118736595828\\
4103.19999999996	75.4100358248239\\
4103.6	75.412432682701\\
4103.70000000004	75.4105950582387\\
4104.49999999997	75.4153876138966\\
4104.60000000005	75.4135503203645\\
4105.00000000006	75.4159460183674\\
4105.1	75.4141089350093\\
4105.79999999995	75.4158649747924\\
4105.90000000002	75.4140282513395\\
4106.59999999995	75.4182190079626\\
4106.90000000002	75.4127100073046\\
4107.40000000005	75.4157029823494\\
4107.50000000005	75.4138669782842\\
4108.20000000004	75.4156220334445\\
4108.39999999999	75.4119508336376\\
4108.89999999996	75.4076417619859\\
4109.20000000007	75.4094371304115\\
4109.30000000007	75.4076020830292\\
4109.79999999994	75.4105939317258\\
4109.89999999998	75.4087591240876\\
4110.19999999997	75.4105539741625\\
4110.39999999998	75.4068848072011\\
4110.70000000006	75.403814342707\\
4111.19999999996	75.4068056332547\\
4111.29999999994	75.4049715425403\\
4112.00000000005	75.4091583375891\\
4112.20000000002	75.4054908445395\\
4112.69999999994	75.4084808403034\\
4112.99999999995	75.4029807201381\\
4113.29999999999	75.4047746389848\\
4113.39999999995	75.4029415339735\\
4113.89999999994	75.4010695187166\\
4114.20000000005	75.4028631845028\\
4114.30000000002	75.4010305269298\\
4115.80000000004	75.4099953837557\\
4115.99999999994	75.4063312358786\\
4116.39999999994	75.4038625045548\\
4116.99999999999	75.4074469893857\\
4117.09999999997	75.4056154668221\\
4117.50000000004	75.4080046629104\\
4117.70000000006	75.4043421244354\\
4119.80000000004	75.412024563703\\
4119.90000000006	75.4101941747573\\
4120.30000000006	75.4125813027861\\
4120.40000000006	75.4107511224366\\
4120.79999999998	75.4082845980247\\
4121.70000000004	75.413654228735\\
4121.80000000003	75.4118246439749\\
4122.50000000005	75.4159996118954\\
4122.80000000005	75.4105120182396\\
4123.10000000007	75.4123011253395\\
4123.20000000004	75.4104721946014\\
4123.99999999999	75.4152421134308\\
4124.19999999997	75.4115849962418\\
4125.09999999998	75.4096771065645\\
4126.49999999998	75.4131730722629\\
4126.69999999997	75.4095182708152\\
4127.00000000001	75.4113057594921\\
4127.1	75.4094785811204\\
4127.60000000004	75.4124573006759\\
4127.90000000005	75.4069767441861\\
4129.20000000006	75.412297483835\\
4129.29999999995	75.4104712549039\\
4129.89999999996	75.4043583535109\\
4130.19999999999	75.4061448320945\\
4130.30000000004	75.4043191942669\\
4132.09999999998	75.4126131358598\\
4132.50000000001	75.4053138460049\\
4133.00000000001	75.4010307033462\\
4133.49999999995	75.3991677956261\\
4133.89999999998	75.4015481373972\\
4133.99999999996	75.3997242446966\\
4134.59999999996	75.3960384066559\\
4134.89999999996	75.397823458283\\
4134.99999999994	75.3960000967328\\
4136.29999999994	75.4037327144377\\
4136.50000000007	75.400087027994\\
4137.00000000003	75.3982258103503\\
4137.89999999999	75.3963267278879\\
4138.30000000004	75.3938720278368\\
4138.59999999995	75.3908232053543\\
4139.1	75.3841321994588\\
4139.89999999997	75.3888888888889\\
4140.00000000001	75.3870679452187\\
4141.20000000007	75.3917851882259\\
4141.30000000001	75.3899647462211\\
4142.00000000002	75.3941237536515\\
4142.1	75.392303606779\\
4142.39999999996	75.3940856970428\\
4142.50000000002	75.3922657268382\\
4142.89999999999	75.394641564084\\
4142.99999999995	75.3928218001014\\
4144.30000000003	75.4005404883699\\
4144.50000000001	75.3969019929547\\
4145.29999999994	75.4016500217108\\
4145.39999999994	75.3998311422024\\
4145.79999999997	75.3973805446345\\
4146.09999999995	75.3991606772466\\
4146.20000000003	75.3973422087162\\
4146.60000000001	75.3948923240167\\
4146.90000000002	75.396672293224\\
4147.00000000005	75.3948542354899\\
4147.40000000005	75.3924050632911\\
4148.70000000006	75.4001156961049\\
4148.8	75.3982983441394\\
4149.70000000002	75.403633910068\\
4149.79999999994	75.4018169112509\\
4151.50000000007	75.4046632623567\\
4151.59999999999	75.4028470265193\\
4152.40000000005	75.4075857916918\\
4152.5	75.4057698791119\\
4153.20000000003	75.4099150073436\\
4153.30000000006	75.4080993884528\\
4154.09999999998	75.4128352029272\\
4154.19999999996	75.4110199070842\\
4155.40000000001	75.4181205631091\\
4155.5	75.4163057079603\\
4156.29999999994	75.4186315080358\\
4156.40000000005	75.4168170335619\\
4156.69999999995	75.4137798306389\\
4157.00000000007	75.4155541122417\\
4157.09999999998	75.4137400173194\\
4157.79999999996	75.4178792178744\\
4157.90000000002	75.4160654160654\\
4158.69999999998	75.4207944599404\\
4158.89999999994	75.4171675883626\\
4159.39999999994	75.4201226108907\\
4159.49999999998	75.418309452832\\
4159.99999999997	75.4212639119252\\
4160.19999999999	75.4176381510949\\
4160.70000000005	75.4205921938089\\
4160.90000000006	75.4169670752223\\
4161.60000000004	75.4186990893145\\
4161.70000000001	75.4168869239272\\
4163.59999999995	75.4208996805726\\
4163.79999999994	75.4172770719758\\
4164.29999999998	75.4130246854289\\
4164.99999999995	75.4147559482365\\
4165.10000000004	75.4129453567656\\
4165.40000000005	75.4147161205137\\
4165.60000000004	75.4110953741268\\
4166.20000000005	75.4074358543552\\
4167.90000000003	75.4150671785029\\
4167.99999999996	75.4132578393033\\
4168.89999999994	75.41616694651\\
4169.00000000005	75.4143580149193\\
4171.09999999999	75.4123513617184\\
4172.29999999999	75.4170261719873\\
4172.49999999997	75.4134113023055\\
4173.19999999996	75.4175352838281\\
4173.30000000005	75.4157281832559\\
4173.80000000003	75.413881501713\\
4174.10000000003	75.4156485074984\\
4174.20000000005	75.4138418417459\\
4175.30000000004	75.4179240312305\\
4175.40000000003	75.4161178302\\
4176.00000000005	75.4196499125979\\
4176.09999999996	75.4178439729898\\
4177.10000000003	75.4141530211625\\
4177.80000000003	75.4158787907801\\
4177.90000000003	75.414073719483\\
4178.69999999999	75.4116014166746\\
4179.10000000004	75.4139548238897\\
4179.30000000004	75.4103459826769\\
4179.8	75.4085025957559\\
4180.09999999997	75.4102674513181\\
4180.19999999993	75.4084635074038\\
4181.20000000003	75.4143448209887\\
4181.30000000003	75.4125412541254\\
4181.89999999998	75.4088952654232\\
4182.19999999997	75.4106592066566\\
4182.30000000002	75.4088561591431\\
4182.79999999999	75.404623586507\\
4183.49999999998	75.4063485992925\\
4183.60000000006	75.4045462150728\\
4185.90000000001	75.3989488772097\\
4186.40000000006	75.3947211274334\\
4186.89999999997	75.3976594220205\\
4187.00000000003	75.3958587088916\\
4187.59999999998	75.3993839100222\\
4187.69999999998	75.3975834567076\\
4188.39999999994	75.4016951175839\\
4188.50000000002	75.3998949529676\\
4188.80000000006	75.4016567595311\\
4188.89999999999	75.399856767725\\
4189.29999999996	75.3974316131188\\
4190.30000000003	75.4033027873234\\
4190.39999999998	75.4015034005489\\
4191.30000000002	75.3996278093239\\
4191.79999999998	75.4025620840192\\
4191.89999999997	75.4007633587786\\
4192.49999999999	75.3947431188284\\
4192.80000000001	75.396503613251\\
4192.89999999997	75.3947054614834\\
4193.39999999995	75.3976392035293\\
4193.50000000005	75.3958412819535\\
4196.60000000006	75.4021016512974\\
4196.70000000006	75.4003049942814\\
4197.29999999995	75.4038214132558\\
4197.4	75.4020250148898\\
4198.7	75.4048775840716\\
4198.80000000005	75.4030817595084\\
4199.09999999998	75.4048390169556\\
4199.19999999996	75.4030433643702\\
4199.69999999996	75.4059717129387\\
4199.79999999999	75.4041762899117\\
4200.19999999999	75.406518582006\\
4200.40000000006	75.4029282228306\\
4200.79999999996	75.4052702992216\\
4200.90000000004	75.4034753630088\\
4202.00000000006	75.4099140905738\\
4202.19999999996	75.4063251076791\\
4202.69999999996	75.4092509755401\\
4202.79999999994	75.4074567560494\\
4203.10000000005	75.4092120289303\\
4203.19999999996	75.4074179811101\\
4203.79999999996	75.4109279478579\\
4203.9	75.4091341579448\\
4204.29999999996	75.4114736942251\\
4204.39999999994	75.4096801046498\\
4204.90000000006	75.4054696789536\\
4205.60000000005	75.4095632118316\\
4205.7	75.4077702220743\\
4206.79999999994	75.4142004801635\\
4206.99999999995	75.4106153882722\\
4207.70000000007	75.4147060221493\\
4207.90000000006	75.4111216730038\\
4208.49999999997	75.4074989307608\\
4209.99999999999	75.4138856559227\\
4210.09999999996	75.4120944373189\\
4211.00000000005	75.4173493861461\\
4211.20000000001	75.4137677201814\\
4211.59999999998	75.4161027613553\\
4211.70000000001	75.4143121705684\\
4214.39999999994	75.4253173567446\\
4214.49999999996	75.4235277369145\\
4215.00000000005	75.4216981803516\\
4215.90000000004	75.4174573055029\\
4216.29999999995	75.4150460108149\\
4217.10000000001	75.4125960352841\\
4217.59999999998	75.4107689024824\\
4218.29999999994	75.4148492319363\\
4218.50000000004	75.4112738823306\\
4219.19999999999	75.4153532576494\\
4219.30000000001	75.413565909845\\
4220.09999999996	75.4182266243306\\
4220.19999999994	75.4164395896026\\
4220.80000000005	75.4199341372693\\
4220.89999999997	75.4181473584459\\
4221.80000000002	75.4233875743149\\
4222.00000000005	75.4198147841122\\
4223.00000000002	75.4232672681206\\
4223.19999999995	75.4196954987806\\
4223.50000000004	75.4167061274742\\
4223.90000000007	75.4142992424242\\
4224.90000000005	75.4201183431953\\
4225.09999999999	75.4165483290732\\
4226.19999999994	75.4229467856044\\
4226.3	75.4211622184365\\
4226.90000000002	75.4246510527561\\
4227.09999999994	75.4210825132475\\
4227.70000000005	75.4174748095936\\
4228.59999999997	75.4203419490624\\
4228.69999999996	75.4185584562996\\
4229.29999999999	75.4220456802383\\
4229.79999999998	75.4131303340505\\
4230.09999999994	75.4148740012293\\
4230.19999999994	75.4130912701227\\
4230.49999999994	75.4148347752092\\
4230.59999999994	75.4130522135817\\
4230.90000000002	75.414795556606\\
4231.00000000003	75.4130131644253\\
4231.89999999997	75.4158790170132\\
4232.00000000001	75.4140970203918\\
4232.39999999998	75.4164205552274\\
4232.59999999997	75.4128570416047\\
4233.30000000003	75.4169225681485\\
4233.39999999996	75.4151411361757\\
4233.9	75.4085970713273\\
4234.70000000002	75.4132426560876\\
4234.79999999999	75.4114618999268\\
4235.40000000001	75.4078621178137\\
4236.00000000001	75.4113453412337\\
4236.09999999996	75.4095651763373\\
4236.60000000001	75.4124672504544\\
4236.80000000002	75.408907455923\\
4237.30000000001	75.411809128239\\
4237.39999999997	75.4100294985251\\
4238.70000000003	75.4152118524111\\
4238.89999999994	75.4116536919085\\
4239.60000000002	75.4157133759464\\
4239.69999999997	75.4139346195575\\
4240.60000000003	75.4191524984083\\
4240.70000000004	75.4173740803622\\
4241.10000000003	75.4149768933321\\
4241.59999999994	75.4131598179975\\
4242.30000000003	75.4077880445031\\
4243.20000000006	75.4106473735065\\
4243.49999999995	75.4053162409275\\
4244.69999999996	75.4122691292876\\
4244.89999999993	75.4087161366313\\
4245.19999999997	75.4057428214732\\
4245.49999999995	75.4027699265122\\
4245.80000000007	75.4045078781884\\
4245.89999999996	75.4027319830429\\
4246.69999999996	75.3955919751342\\
4247.00000000001	75.3926208471663\\
4247.29999999997	75.3943589019165\\
4247.5	75.3908089273943\\
4248.00000000004	75.388997434147\\
4248.50000000004	75.3918938003107\\
4248.60000000003	75.3901193306188\\
4249.30000000004	75.3918200216501\\
4249.39999999997	75.3900458877515\\
4250.2	75.3946780227278\\
4250.29999999998	75.392904197252\\
4250.89999999994	75.3963773229828\\
4251.00000000002	75.3946037496177\\
4251.29999999996	75.3963400291669\\
4251.40000000007	75.3945666235446\\
4252.20000000001	75.39919572937\\
4252.3	75.3974226319255\\
4253.39999999994	75.403785118138\\
4253.49999999998	75.4020124130149\\
4253.99999999997	75.4049035048541\\
4254.19999999998	75.4013586253908\\
4254.60000000005	75.4036712341646\\
4254.69999999996	75.4018990316819\\
4255.60000000006	75.3977019056794\\
4256.29999999995	75.3993985527676\\
4256.39999999994	75.3976271584635\\
4257.10000000001	75.4016724607723\\
4257.19999999996	75.3999013459235\\
4257.60000000001	75.4022124621274\\
4257.70000000002	75.4004415425807\\
4258.20000000004	75.3962849024258\\
4259.10000000004	75.4014838467318\\
4259.19999999998	75.3997135679572\\
4259.5	75.401446145178\\
4259.59999999994	75.3996760335235\\
4259.99999999996	75.4019858688763\\
4260.09999999996	75.4002159523027\\
4260.59999999998	75.4031027765391\\
4260.69999999999	75.4013330829891\\
4260.99999999993	75.4030649362841\\
4261.29999999999	75.3977566058103\\
4261.90000000001	75.3918348193337\\
4262.40000000001	75.3900293255132\\
4263.19999999998	75.3923017380902\\
4263.30000000004	75.3905333771169\\
4263.69999999994	75.3928420657629\\
4263.80000000004	75.3910738994817\\
4264.10000000002	75.3928052155152\\
4264.39999999994	75.387501465588\\
4264.79999999997	75.3851204014162\\
4265.10000000001	75.3868517302823\\
4265.19999999995	75.38508428481\\
4265.59999999994	75.3873924561033\\
4265.69999999995	75.3856252051198\\
4267.09999999997	75.3913573303337\\
4267.20000000003	75.3895906076442\\
4267.60000000002	75.3918972748787\\
4267.79999999998	75.3883643009443\\
4268.30000000002	75.3865617093056\\
4268.80000000005	75.3894445875987\\
4268.90000000006	75.3876786132584\\
4270.59999999998	75.3904512140867\\
4270.70000000001	75.3886859604758\\
4271.59999999994	75.3915303040944\\
4271.69999999998	75.3897654384569\\
4272.09999999999	75.3920696596601\\
4272.20000000004	75.3903049879456\\
4272.70000000006	75.3885040254634\\
4273.69999999998	75.3919228789368\\
4273.90000000003	75.3883949461862\\
4274.20000000005	75.3901223592167\\
4274.29999999995	75.3883586000374\\
4274.69999999999	75.3859829699635\\
4275.29999999998	75.3894372456378\\
4275.39999999998	75.3876739562624\\
4276.40000000007	75.3910908453174\\
4276.49999999995	75.3893279708179\\
4277.39999999993	75.3921683226183\\
4277.49999999995	75.3904058350477\\
4278.20000000004	75.3850828600145\\
4278.60000000005	75.3827097015449\\
4279.30000000003	75.3727158012805\\
4279.59999999997	75.3697689090357\\
4280.00000000002	75.3720707460106\\
4280.10000000002	75.3703097986075\\
4280.90000000002	75.3725765008176\\
4280.99999999999	75.3708159117984\\
4281.70000000004	75.3748423560185\\
4281.90000000001	75.3713218122373\\
4282.3	75.3736222678872\\
4282.40000000002	75.3718622300058\\
4282.90000000002	75.3747373336446\\
4283	75.3729775162849\\
4283.29999999997	75.3747023392632\\
4283.49999999994	75.3711831170044\\
4283.90000000001	75.3734827264239\\
4284.00000000002	75.3717233491282\\
4284.30000000003	75.3734478573429\\
4284.49999999995	75.3699295150072\\
4285.79999999995	75.3680673837467\\
4286.29999999996	75.3709406494961\\
4286.40000000004	75.3691823165753\\
4286.90000000005	75.3673897830651\\
4288.40000000006	75.3760055963624\\
4288.49999999995	75.3742480063424\\
4289.10000000005	75.377692809848\\
4289.20000000006	75.3759354673257\\
4290.40000000005	75.3828225148584\\
4290.49999999999	75.3810655852328\\
4290.89999999997	75.3833605220228\\
4291.10000000006	75.3798471290082\\
4291.60000000003	75.3827154740546\\
4291.70000000002	75.3809590381658\\
4292.00000000005	75.3826798070874\\
4292.20000000002	75.3791673461781\\
4293.40000000002	75.3837195761034\\
4293.50000000005	75.3819638531768\\
4294.00000000005	75.3801727952307\\
4294.3	75.3772354694486\\
4294.89999999999	75.3806752037253\\
4295.00000000004	75.378920164839\\
4295.59999999995	75.3823591032893\\
4295.70000000003	75.3806043111877\\
4296.30000000003	75.3840424541477\\
4296.40000000003	75.3822879087629\\
4296.69999999997	75.3840067026624\\
4296.90000000001	75.3804980218757\\
4297.50000000003	75.3839352196575\\
4297.59999999999	75.3821811666705\\
4298.10000000003	75.3850449025173\\
4298.19999999997	75.383291068562\\
4298.69999999999	75.3861542756118\\
4298.79999999995	75.3844006606341\\
4301.10000000005	75.3975634706593\\
4301.19999999994	75.3958105688978\\
4301.50000000007	75.3975265017668\\
4301.60000000004	75.3957737638608\\
4302.10000000006	75.3939844730603\\
4302.40000000001	75.3957001743173\\
4302.59999999998	75.3921955981128\\
4302.99999999999	75.3898352350631\\
4303.50000000006	75.3926944883353\\
4303.59999999997	75.3909426772312\\
4304.20000000002	75.3943730687917\\
4304.39999999994	75.3908700197468\\
4304.7	75.3925850213715\\
4304.89999999998	75.3890824622532\\
4305.19999999995	75.3907973892644\\
4305.49999999995	75.3855444072835\\
4306.30000000004	75.3831506594836\\
4306.99999999996	75.375542708551\\
4307.80000000007	75.3731516516168\\
4308.70000000004	75.3782955811363\\
4308.80000000006	75.3765462182924\\
4309.09999999993	75.3736192332684\\
4309.99999999998	75.3764413818705\\
4310.1	75.374692589671\\
4310.90000000003	75.3792623521225\\
4310.99999999999	75.3775138595718\\
4311.89999999996	75.3803339517625\\
4312.00000000004	75.378585839846\\
4312.30000000006	75.3802986735924\\
4312.40000000001	75.3785507246377\\
4313.39999999994	75.3749855106062\\
4313.80000000006	75.3772688286701\\
4313.90000000004	75.3755215577191\\
4314.49999999999	75.378945904603\\
4314.60000000003	75.3771988782534\\
4316.39999999996	75.3735665469709\\
4316.80000000007	75.3758484097385\\
4316.9	75.3741023859161\\
4317.49999999994	75.3775245506763\\
4317.59999999998	75.375778771105\\
4318.69999999999	75.3820505696027\\
4318.79999999994	75.380305170298\\
4319.29999999996	75.3785247951104\\
4320.10000000005	75.3830841164761\\
4320.29999999994	75.3795944819924\\
4320.59999999998	75.3766750757979\\
4320.90000000006	75.3783846331868\\
4321.10000000002	75.3748958622605\\
4321.60000000006	75.373117060416\\
4323.30000000002	75.3781745848175\\
4323.40000000007	75.3764311321846\\
4324.30000000004	75.3792433632411\\
4324.39999999999	75.3775002890508\\
4324.70000000006	75.3792082870884\\
4324.90000000001	75.3757225433526\\
4325.20000000004	75.3728065105311\\
4326.20000000007	75.378498948293\\
4326.50000000003	75.373272315444\\
4327.20000000002	75.3772560256973\\
4327.5	75.3720306867548\\
4328.39999999996	75.3679103615571\\
4328.69999999994	75.3649972278692\\
4329.50000000003	75.3695491500369\\
4329.59999999999	75.3678083931912\\
4330.09999999998	75.3706526257448\\
4330.20000000007	75.3689120846131\\
4331.10000000002	75.374030291836\\
4331.30000000001	75.3705499376645\\
4331.69999999998	75.3682072117826\\
4332.00000000007	75.3699129752314\\
4332.20000000006	75.3664335341505\\
4332.79999999994	75.3606129843754\\
4334.29999999996	75.3668327796235\\
4334.49999999995	75.3633553269044\\
4334.89999999998	75.361014994233\\
4335.30000000003	75.358675093417\\
4335.6	75.3603801000992\\
4335.69999999997	75.3586420037825\\
4336.90000000005	75.3562370302052\\
4337.50000000006	75.3596458871265\\
4337.59999999997	75.3579085690573\\
4338.19999999998	75.3613166447687\\
4338.30000000003	75.3595795685045\\
4338.70000000003	75.3572416336314\\
4339.09999999998	75.3549041297935\\
4339.69999999995	75.3583114429236\\
4339.90000000002	75.3548387096774\\
4341.09999999994	75.3593476458122\\
4341.30000000005	75.3558759847054\\
4341.79999999997	75.3587139270826\\
4341.89999999996	75.3569783509903\\
4342.20000000003	75.3540750293623\\
4342.80000000005	75.3505721983007\\
4344.2	75.3470064222084\\
4345.1	75.3521126760563\\
4345.20000000005	75.3503785699491\\
4345.99999999994	75.3549159016129\\
4346.19999999994	75.3514483583738\\
4347.69999999994	75.3553521321128\\
4347.89999999999	75.351885924563\\
4348.39999999999	75.3501207312867\\
4349.7	75.3574877005839\\
4349.79999999996	75.3557553047197\\
4352.20000000002	75.3647496725869\\
4352.30000000006	75.3630181049536\\
4353.20000000007	75.3658144396205\\
4353.29999999995	75.3640832452796\\
4353.79999999999	75.3623188405797\\
4354.40000000006	75.3657136295786\\
4354.49999999998	75.363982914619\\
4355.00000000001	75.3668113246539\\
4355.10000000001	75.3650808229243\\
4355.9	75.3696051423324\\
4356.10000000002	75.3661448051054\\
4356.5	75.3684065555709\\
4356.59999999995	75.3666766130328\\
4358.00000000001	75.3700006883734\\
4358.09999999999	75.3682713046671\\
4358.7	75.3647792970542\\
4359.19999999994	75.3630169981419\\
4359.60000000007	75.3606899557309\\
4360.80000000004	75.3582975991195\\
4363.09999999998	75.3689952328566\\
4363.20000000003	75.3672678935668\\
4363.60000000005	75.3626509613402\\
4364.40000000002	75.3671669148814\\
4364.5	75.3654401319709\\
4366.00000000004	75.3716131100983\\
4366.20000000006	75.3681606852484\\
4367.50000000001	75.3754922611961\\
4367.60000000003	75.3737665132679\\
4368.00000000004	75.3714429614707\\
4368.49999999997	75.3673945886554\\
4369	75.3702135451237\\
4369.09999999999	75.3684885104825\\
4370.50000000003	75.3740905138883\\
4370.70000000006	75.3706415301547\\
4370.99999999999	75.3677563999909\\
4371.50000000007	75.370573702992\\
4371.69999999994	75.3671256690608\\
4372.6	75.3721956685801\\
4372.69999999997	75.3704720087816\\
4373.09999999998	75.3681514680326\\
4373.60000000007	75.3663945858198\\
4373.90000000004	75.363511659808\\
4374.20000000003	75.3606291292321\\
4375.10000000004	75.3656975681112\\
4375.19999999997	75.3639750417114\\
4375.70000000004	75.3667900726724\\
4375.79999999996	75.3650677574899\\
4376.60000000004	75.3627162017045\\
4377.59999999997	75.3683441076364\\
4377.69999999999	75.3666225044543\\
4378.10000000007	75.3688730528528\\
4378.29999999995	75.3654302941714\\
4378.90000000001	75.3688056633935\\
4379.20000000003	75.3636425912817\\
4380.40000000007	75.3658258189704\\
4380.50000000004	75.3641053736931\\
4381.69999999997	75.3662878269204\\
4381.79999999995	75.3645678815126\\
4382.59999999998	75.3622196362973\\
4383.29999999995	75.3661541269334\\
4383.40000000001	75.3644348123645\\
4383.90000000005	75.3626824817518\\
4384.79999999999	75.3654587333805\\
4384.89999999996	75.363740022805\\
4385.70000000005	75.3682338455926\\
4385.99999999997	75.3630788171724\\
4387.99999999994	75.3720288963333\\
4388.30000000006	75.3668763102725\\
4389.20000000005	75.369648918962\\
4389.29999999994	75.3679318357862\\
4389.79999999996	75.3661814619923\\
4390.60000000003	75.3706698248571\\
4390.69999999995	75.3689532659196\\
4391.79999999999	75.3728454655161\\
4391.90000000003	75.3711293260474\\
4392.50000000006	75.3744934662842\\
4392.60000000001	75.3727775627746\\
4393.00000000003	75.3704673237577\\
4393.50000000007	75.3687181354698\\
4394.60000000002	75.3748833822559\\
4394.79999999996	75.3714532753874\\
4395.19999999999	75.373694628353\\
4395.29999999997	75.3719797970606\\
4395.60000000004	75.3691107218418\\
4395.90000000001	75.3662420382166\\
4396.70000000006	75.3639010189228\\
4397.89999999998	75.3683492496589\\
4398.10000000004	75.364922013551\\
4400.50000000005	75.3715402445121\\
4400.6	75.3698275274388\\
4400.99999999994	75.3720660743905\\
4401.10000000001	75.3703535399437\\
4401.59999999998	75.3640638844083\\
4402.19999999995	75.3674215750857\\
4402.30000000006	75.3657096129384\\
4402.70000000006	75.3679476696648\\
4402.80000000002	75.3662358899816\\
4403.20000000003	75.363931596757\\
4404.00000000005	75.3684067119275\\
4404.1	75.3666954270923\\
4404.99999999999	75.3717282241039\\
4405.19999999994	75.3683063582503\\
4405.60000000004	75.3705427060399\\
4405.70000000004	75.3688319941895\\
4406.20000000006	75.3716269886299\\
4406.29999999998	75.3699164851126\\
4407.59999999997	75.3771808426163\\
4407.7	75.3754707563864\\
4408.50000000002	75.3731343283582\\
4408.99999999997	75.3713909868227\\
4409.40000000005	75.3690894659258\\
4409.79999999993	75.3713236127803\\
4409.89999999999	75.3696145124717\\
4410.70000000002	75.3718146368006\\
4410.80000000005	75.3701058740847\\
4411.20000000003	75.3678054088364\\
4412.09999999999	75.3614976655637\\
4412.50000000001	75.3591986583874\\
4412.99999999999	75.3619904375609\\
4413.10000000001	75.3602827879996\\
4414.29999999999	75.3624501631026\\
4414.39999999998	75.3607430060029\\
4414.80000000004	75.3584452648984\\
4415.30000000001	75.3567060741949\\
4415.80000000007	75.3549672773387\\
4416.2	75.3526707877635\\
4416.99999999994	75.3571347716828\\
4417.10000000001	75.3554287784117\\
4417.50000000002	75.3576602680188\\
4417.70000000003	75.3542487210829\\
4418.59999999996	75.3502161269152\\
4419.49999999999	75.3529731197393\\
4419.6	75.3512681856235\\
4420.29999999996	75.3551714776943\\
4420.39999999993	75.3534668023979\\
4420.89999999993	75.3517303777426\\
4421.80000000007	75.3499626857233\\
4422.60000000005	75.3544215072241\\
4422.69999999995	75.3527177353713\\
4423.19999999993	75.3509822982841\\
4423.9	75.3548824593128\\
4423.99999999997	75.3531791776859\\
4424.4	75.3554073906656\\
4424.69999999996	75.3502983185681\\
4425.6	75.3553110242448\\
4425.69999999998	75.3536083871842\\
4426.30000000007	75.3479125248509\\
4427.2	75.3461477650035\\
4428.10000000003	75.35115848426\\
4428.19999999993	75.3494569021972\\
4430.30000000006	75.3521126760563\\
4430.40000000007	75.3504119173908\\
4431.80000000007	75.3559421467091\\
4431.90000000003	75.3542418772563\\
4432.60000000005	75.3581338687482\\
4432.70000000001	75.3564338567046\\
4433.60000000003	75.361436272188\\
4433.79999999993	75.3580369426464\\
4434.19999999996	75.3602597929775\\
4434.29999999998	75.3585603463828\\
4434.79999999994	75.3613384743737\\
4435.09999999994	75.356240981241\\
4435.60000000003	75.3500011272178\\
4436.10000000006	75.3527794057977\\
4436.19999999993	75.351080855668\\
4437	75.3555250050709\\
4437.10000000007	75.3538267375823\\
4437.39999999998	75.350985915493\\
4438.19999999999	75.3464164207016\\
4438.59999999995	75.3441322909861\\
4439.10000000004	75.3424040367634\\
4440.39999999995	75.3406147956311\\
4441.00000000007	75.3439463196055\\
4441.10000000006	75.3422498423849\\
4441.60000000007	75.3450255532792\\
4442.39999999995	75.3472144063028\\
4442.59999999998	75.3438224503117\\
4443.09999999998	75.3420957868203\\
4444.00000000006	75.3403388762629\\
4444.89999999997	75.334083239595\\
4446.29999999997	75.3396005757467\\
4446.99999999998	75.3344876436329\\
4447.60000000007	75.3378150504755\\
4449.19999999995	75.3444362034477\\
4449.69999999996	75.3427120320014\\
4450.09999999997	75.3449283178284\\
4450.60000000001	75.3409576021749\\
4451.39999999999	75.3431427608671\\
4451.90000000001	75.3414195867026\\
4452.30000000006	75.3391429341479\\
4452.79999999994	75.3419120123964\\
4453.10000000001	75.3368364322285\\
4453.89999999997	75.3390211046251\\
4454.5	75.3356081354106\\
4455.29999999996	75.3377923418773\\
4455.59999999998	75.3349642031555\\
4456.00000000003	75.3304459056125\\
4457.6	75.3393005361509\\
4459.00000000004	75.3425579152744\\
4459.59999999999	75.3346637666211\\
4460.49999999997	75.3329148545039\\
4461.40000000006	75.3356494452538\\
4461.59999999997	75.3322724522043\\
4461.99999999997	75.3344837632505\\
4462.50000000002	75.3305248061668\\
4464.60000000005	75.3421282504983\\
4464.80000000001	75.3387533875339\\
4465.40000000005	75.3420669577875\\
4465.69999999995	75.3370056876707\\
4466.00000000002	75.3341841875462\\
4466.40000000006	75.3363931489981\\
4466.60000000002	75.3330199028366\\
4467.60000000002	75.3385410837791\\
4468.50000000002	75.3412701964821\\
4468.79999999997	75.3384501778961\\
4470.10000000004	75.3366739743188\\
4470.49999999999	75.3388806871561\\
4471.00000000007	75.3349287647335\\
4471.39999999997	75.3371351895337\\
4471.69999999997	75.3343172771591\\
4471.99999999999	75.3314997428501\\
4472.49999999999	75.3342574788714\\
4474.59999999996	75.341363666838\\
4476.59999999994	75.3456787365693\\
4476.79999999999	75.3423127610623\\
4477.30000000003	75.3405994550409\\
4477.90000000002	75.3439035283609\\
4478.19999999996	75.3410892526182\\
4478.7	75.3438421005627\\
4479.60000000003	75.342098801259\\
4479.99999999993	75.3443003504386\\
4480.40000000005	75.3398058252427\\
4481.10000000007	75.3436579487637\\
4481.50000000005	75.341395930025\\
4482.20000000007	75.3363228699552\\
4482.59999999996	75.3385236576171\\
4482.89999999994	75.3357126923935\\
4483.80000000007	75.333972657731\\
4484.09999999994	75.3311627492083\\
4484.7	75.3344630752765\\
4485.89999999997	75.3276861346411\\
4486.30000000004	75.3254279600571\\
4486.79999999997	75.3237201631416\\
4487.10000000001	75.3209128186843\\
4488.30000000005	75.3275109170306\\
4488.60000000006	75.3247042573574\\
4488.99999999994	75.3269029426834\\
4489.40000000004	75.3224189776144\\
4490.29999999993	75.3273650454303\\
4490.50000000003	75.324010154545\\
4491.00000000001	75.3267573645655\\
4491.30000000002	75.3239524424456\\
4492.89999999998	75.3216113955041\\
4493.40000000007	75.3176810949149\\
4493.80000000003	75.3154275796079\\
4494.59999999999	75.3175962800632\\
4495.69999999997	75.3214110947996\\
4496.10000000005	75.319158400427\\
4497.09999999997	75.3224228408788\\
4497.60000000001	75.3207194788448\\
4499.59999999994	75.3183545569705\\
4500.40000000003	75.3160759915565\\
4503.29999999995	75.3253097659546\\
4503.59999999999	75.3202922041877\\
4504.00000000006	75.3224839590595\\
4505.30000000005	75.3273849158787\\
4505.60000000005	75.3245888541181\\
4505.99999999998	75.3223408268791\\
4507.50000000002	75.3283343686219\\
4509.69999999996	75.3381524679587\\
4511.9	75.3302304964539\\
4512.50000000007	75.3268625626025\\
4514.00000000004	75.3239848474779\\
4514.50000000003	75.3267177601559\\
4514.69999999994	75.3233808806592\\
4515.70000000004	75.3199875990965\\
4517.49999999994	75.3298211439702\\
4517.99999999999	75.3259113344105\\
4518.99999999994	75.3313712907437\\
4519.4	75.3291293284655\\
4519.69999999996	75.3263418735342\\
4520.90000000005	75.3284671532847\\
4521.70000000001	75.3306205493388\\
4522.09999999993	75.3261686789616\\
4522.50000000005	75.3283509485694\\
4522.80000000005	75.3255654557916\\
4524.50000000001	75.332626088494\\
4526.19999999996	75.3308441773634\\
4526.79999999996	75.3341138527469\\
4527.40000000001	75.3307564881281\\
4527.79999999996	75.3329357980521\\
4527.99999999993	75.3296084450432\\
4529.29999999995	75.325650196494\\
4531.89999999998	75.3309796999117\\
4533.50000000004	75.3352743956238\\
4534.29999999995	75.3330098800282\\
4535.00000000004	75.325792154528\\
4535.40000000002	75.3235585933194\\
4535.99999999994	75.3268226009127\\
4537.69999999996	75.3250473797876\\
4538.59999999996	75.323330469077\\
4539.50000000006	75.3260199136488\\
4540.20000000001	75.3232165275422\\
4540.50000000001	75.3204422323041\\
4541.00000000003	75.3165532580212\\
4541.70000000005	75.3137522568145\\
4541.99999999998	75.3109795028731\\
4542.79999999994	75.3153272138942\\
4545.69999999997	75.3266751726869\\
4546.10000000004	75.3244467907263\\
4546.59999999994	75.3271603580619\\
4547.90000000005	75.3320140721196\\
4549.60000000001	75.334637448623\\
4549.79999999998	75.331325963208\\
4550.70000000008	75.329612375846\\
4550.99999999995	75.3268440596779\\
4551.89999999994	75.3317223198594\\
4552.09999999995	75.3284126356487\\
4553.29999999999	75.3239337637809\\
4553.79999999998	75.3266430971255\\
4554.50000000002	75.323848416985\\
4554.89999999998	75.3260153677278\\
4556.10000000007	75.3303191255871\\
4558.10000000004	75.3279803431179\\
4558.79999999995	75.3251880936191\\
4559.30000000007	75.3278940211431\\
4560.99999999995	75.33489728355\\
4561.19999999998	75.331594063096\\
4562.50000000006	75.3386227151186\\
4563.30000000001	75.3363720033309\\
4564.20000000003	75.3324715728589\\
4565.49999999994	75.3373050639565\\
4566.29999999999	75.3394358794674\\
4568.09999999998	75.3447747471652\\
4568.50000000006	75.3425557063433\\
4569.39999999999	75.3474121895174\\
4569.70000000006	75.3446540329993\\
4570.30000000004	75.3478907754245\\
4571.09999999993	75.3456422821141\\
4571.60000000007	75.3417765820154\\
4572.09999999999	75.344473120161\\
4572.49999999995	75.342256046888\\
4573.09999999999	75.3454911221902\\
4573.29999999995	75.3421961778983\\
4573.80000000001	75.3383327138766\\
4574.19999999996	75.3361170015084\\
4575.00000000004	75.3382439728093\\
4575.80000000005	75.335999475513\\
4577.10000000007	75.3408197151097\\
4577.60000000004	75.336959608537\\
4578.1	75.3396531387882\\
4579.00000000001	75.3379485051648\\
4579.39999999997	75.3357353422863\\
4580.50000000003	75.3416582980396\\
4580.70000000003	75.3383688438701\\
4581.10000000002	75.3405221339387\\
4581.89999999999	75.3426451331296\\
4582.80000000001	75.336577276397\\
4583.49999999999	75.3403438345405\\
4583.90000000003	75.3359511343805\\
4585.20000000002	75.3407628726583\\
4586.39999999997	75.3384934045569\\
4588.40000000006	75.3492426718971\\
4589.20000000003	75.3513607739742\\
4590.19999999995	75.3480164695118\\
4590.59999999999	75.3458078288714\\
4590.89999999998	75.3430625136136\\
4592.20000000004	75.3478649042963\\
4592.89999999999	75.3450903548879\\
4593.30000000002	75.3407062306788\\
4593.70000000004	75.3428534111193\\
4594.90000000004	75.3449401523395\\
4595.59999999997	75.3421676784821\\
4596.40000000004	75.3377569890134\\
4597.10000000006	75.3415122248325\\
4598.00000000004	75.3441638937822\\
4598.19999999994	75.3408868494879\\
4598.9	75.3381169819526\\
4599.30000000004	75.3359133800061\\
4599.90000000007	75.3391304347826\\
4601.00000000004	75.3363326161135\\
4601.40000000007	75.3341301749429\\
4602.20000000008	75.3318992677574\\
4602.80000000004	75.3351148189185\\
4604.19999999998	75.3382707469105\\
4606.20000000007	75.3424657534247\\
4606.79999999993	75.3391651652955\\
4607.50000000008	75.342911711086\\
4608.29999999993	75.3406822324451\\
4609.19999999995	75.343327620246\\
4609.50000000001	75.3405935439084\\
4610.80000000007	75.3388709362597\\
4611.49999999996	75.3339404978749\\
4613.20000000007	75.3321917065875\\
4613.79999999999	75.328897462017\\
4614.90000000003	75.3347778981582\\
4615.09999999996	75.3315132605304\\
4615.49999999999	75.3336510962822\\
4616.3	75.3314270860411\\
4617.09999999999	75.3335354760461\\
4617.40000000004	75.3308067135896\\
4618.30000000004	75.3334488134419\\
4619.60000000005	75.3382254258935\\
4620.80000000008	75.3359735116536\\
4621.20000000007	75.3337805379439\\
4621.89999999997	75.3310255300736\\
4622.70000000005	75.3288050532145\\
4623.00000000005	75.3260799030953\\
4623.69999999996	75.3233271335265\\
4624.70000000005	75.3265006054316\\
4625.09999999995	75.3243103000951\\
4626.30000000001	75.330710703787\\
4627.09999999994	75.3284923928077\\
4627.60000000004	75.3311580266655\\
4627.79999999997	75.3279025043756\\
4628.20000000006	75.3235529243999\\
4628.59999999997	75.3256853976278\\
4628.89999999998	75.3229639230935\\
4629.99999999994	75.3201874689532\\
4631.20000000002	75.315786064388\\
4631.80000000005	75.3125067466914\\
4632.19999999996	75.3146385165037\\
4633.20000000001	75.3091748861503\\
4633.60000000005	75.3113062995015\\
4635.19999999995	75.3155135589929\\
4635.60000000001	75.3133291628017\\
4636	75.3111451435474\\
4638.19999999997	75.3206993941746\\
4638.50000000006	75.3158280515673\\
4640.70000000007	75.3124461299776\\
4641.30000000003	75.3156375231611\\
4642.49999999993	75.3198638693835\\
4643.30000000005	75.3176551664728\\
4643.60000000001	75.3149428257639\\
4644.00000000003	75.3127624297496\\
4646.50000000003	75.3174364051134\\
4646.90000000003	75.3152571551539\\
4647.30000000007	75.3130782803288\\
4647.7	75.3108997805413\\
4648.89999999994	75.3086685308669\\
4651.00000000001	75.3155167594763\\
4651.30000000002	75.3128090467386\\
4651.79999999995	75.3154624991939\\
4652.40000000003	75.3121977431488\\
4653.2	75.3099950572712\\
4654.30000000002	75.3158301821932\\
4654.69999999995	75.3136547220074\\
4655.70000000006	75.3103655655312\\
4656.60000000004	75.3129898855413\\
4656.89999999995	75.3102855915826\\
4659.00000000007	75.3128286578953\\
4660.19999999996	75.3106023217389\\
4661.69999999996	75.3185464841906\\
4662.19999999999	75.3147588100294\\
4662.59999999996	75.3168764878718\\
4662.89999999995	75.3141754235471\\
4663.8	75.3082184437917\\
4665.10000000003	75.3150990311241\\
4665.90000000004	75.3129018431204\\
4667.09999999995	75.3106787795681\\
4668.00000000007	75.3154388295024\\
4668.30000000001	75.312740981921\\
4668.99999999998	75.3164421408837\\
4669.29999999998	75.3137448066133\\
4669.89999999993	75.3104925053533\\
4670.60000000006	75.3077697133192\\
4671.40000000004	75.311998287488\\
4672.79999999994	75.3086948147831\\
4673.20000000005	75.3065285772366\\
4673.49999999996	75.3038343033208\\
4674.40000000001	75.3085891539202\\
4674.6	75.3053671893384\\
4675.69999999998	75.3026220112066\\
4676.39999999993	75.3063188281835\\
4677.20000000002	75.3084044213542\\
4677.59999999997	75.3041024435086\\
4678.19999999995	75.3072697347327\\
4678.39999999997	75.3040504435182\\
4678.90000000008	75.3066894635606\\
4679.19999999995	75.3018613895241\\
4679.99999999997	75.3060832033504\\
4680.39999999995	75.3039205213118\\
4680.89999999993	75.3065584276864\\
4681.99999999997	75.303816663463\\
4682.39999999998	75.301655098772\\
4683.00000000003	75.3048194571972\\
4683.30000000003	75.3021309305206\\
4683.70000000007	75.29997010974\\
4684.19999999999	75.2962022073736\\
4685.00000000001	75.3004204819534\\
4685.20000000005	75.2972061554223\\
4685.9	75.3008962868118\\
4687.59999999998	75.297053992363\\
4688.59999999998	75.2937914560539\\
4689.89999999998	75.2985074626866\\
4690.70000000001	75.2963247207299\\
4690.99999999993	75.2936411502633\\
4691.69999999993	75.2909331173537\\
4692.7	75.29619843164\\
4693.59999999995	75.2924132347615\\
4694.59999999999	75.2976761028394\\
4695.40000000004	75.2997550846555\\
4695.69999999996	75.2970739810043\\
4697.60000000006	75.3028077569875\\
4698.00000000006	75.3006534556523\\
4698.59999999997	75.3038074360993\\
4699.40000000004	75.3016278327482\\
4699.99999999993	75.3047807493458\\
4701.79999999998	75.3078542716774\\
4702	75.3046511133323\\
4702.40000000003	75.3024986709197\\
4702.8	75.3045992897999\\
4703.70000000008	75.3071984353076\\
4704.09999999994	75.3050465541431\\
4704.49999999993	75.3007694596778\\
4705.29999999994	75.3049687592978\\
4705.50000000006	75.3017681060864\\
4706.00000000001	75.3043921718621\\
4706.30000000003	75.3017168111508\\
4706.70000000004	75.3038157559276\\
4706.89999999999	75.3006161036754\\
4707.69999999996	75.3026891541697\\
4708.00000000006	75.297890868928\\
4708.49999999994	75.300513953192\\
4708.80000000007	75.2978402599333\\
4709.10000000003	75.2951669073303\\
4710.49999999997	75.3025092344924\\
4710.79999999999	75.2977138126473\\
4712.60000000001	75.307148768222\\
4713.79999999997	75.3113133498801\\
4714.19999999998	75.3091657298008\\
4714.69999999994	75.3117841689997\\
4715.59999999996	75.3080136565091\\
4716.00000000005	75.305867135981\\
4716.79999999994	75.3079352964871\\
4718.09999999993	75.3020219575262\\
4718.90000000005	75.299851663488\\
4719.29999999996	75.3019451625207\\
4720.29999999995	75.3050588933141\\
4720.79999999998	75.3013196636235\\
4721.70000000003	75.3039095260282\\
4723.90000000002	75.3069432684166\\
4724.09999999995	75.3037551331442\\
4724.59999999994	75.3063686583275\\
4725.80000000005	75.308406864301\\
4726.99999999998	75.3104440354551\\
4727.29999999996	75.3077801751491\\
4727.89999999998	75.3109137055838\\
4730.69999999998	75.3170711084806\\
4731.50000000007	75.3149040493702\\
4732.29999999999	75.3190770010988\\
4733.39999999993	75.3226999049329\\
4734.19999999999	75.3184208858754\\
4735.19999999999	75.3151859438684\\
4735.79999999999	75.3183133089804\\
4736.19999999994	75.3140637206258\\
4737.1	75.3166427425483\\
4738.60000000004	75.3139046573955\\
4739.40000000001	75.3180715265323\\
4742.59999999999	75.3199654205411\\
4742.80000000005	75.3167893061207\\
4743.40000000004	75.3114788658164\\
4745.59999999998	75.3144952272584\\
4745.90000000004	75.3097345132743\\
4746.80000000007	75.3059891718806\\
4747.59999999995	75.303831328854\\
4748.20000000006	75.3069519617547\\
4749.00000000008	75.3090059169106\\
4749.29999999999	75.3063544868826\\
4749.79999999993	75.3089538727131\\
4750.70000000005	75.3115264797508\\
4752.39999999997	75.3098369279327\\
4752.80000000005	75.3119148309453\\
4754.00000000006	75.3160429944679\\
4754.29999999995	75.3133939088003\\
4755.89999999997	75.3195962994113\\
4756.09999999999	75.3164290820403\\
4757	75.3189968678396\\
4757.19999999993	75.3158304080045\\
4757.59999999996	75.3179057107426\\
4758.29999999999	75.3110289172831\\
4759.39999999995	75.308330707007\\
4759.89999999993	75.3109243697479\\
4760.30000000002	75.3087975800353\\
4760.80000000003	75.3050893738579\\
4761.79999999997	75.3102753102753\\
4762.70000000004	75.3128411858571\\
4762.90000000002	75.309678773882\\
4763.69999999994	75.3138250976112\\
4764.10000000004	75.3116997607153\\
4765.80000000001	75.318407855809\\
4766.30000000005	75.3147029204431\\
4768.20000000001	75.3203447769645\\
4769.10000000006	75.3229053090665\\
4769.40000000004	75.3202641786351\\
4770.30000000006	75.3249203421097\\
4770.59999999997	75.322279749303\\
4771.49999999996	75.3269343616397\\
4773.40000000005	75.3304703048078\\
4774.8	75.3230434145218\\
4775.20000000003	75.3209222457228\\
4776.39999999997	75.3166544540982\\
4777.60000000005	75.3207610356448\\
4777.90000000001	75.3181247383843\\
4779.10000000006	75.3159524606629\\
4779.49999999997	75.3117415683321\\
4779.90000000003	75.3096234309623\\
4780.29999999995	75.3116893983767\\
4780.59999999998	75.3069634153994\\
4781.29999999999	75.3105784916552\\
4781.49999999996	75.307428475824\\
4782.29999999995	75.3094680495149\\
4782.60000000001	75.3068350513308\\
4783.80000000007	75.3109387738038\\
4784.30000000001	75.3072485578129\\
4785.09999999999	75.3092869681518\\
4787.19999999999	75.3054957909469\\
4787.89999999996	75.3028404344194\\
4788.39999999995	75.3054192335805\\
4789.49999999999	75.3027392684149\\
4790.20000000006	75.300085589629\\
4791.09999999998	75.3026381699783\\
4791.80000000004	75.2999853920157\\
4795.00000000003	75.3039561218744\\
4797.09999999995	75.3064287501042\\
4798.20000000003	75.303753412667\\
4799.00000000007	75.3078702256673\\
4799.20000000005	75.3047319400746\\
4799.70000000007	75.3073044710196\\
4800	75.3025978625445\\
4802	75.3108015243331\\
4802.20000000001	75.3076650771505\\
4802.59999999993	75.3097216149249\\
4802.99999999997	75.307613832733\\
4803.59999999995	75.3106980036222\\
4803.89999999996	75.308076602831\\
4804.60000000004	75.3116739858888\\
4804.79999999993	75.3085391995671\\
4805.19999999993	75.3064324808024\\
4805.79999999996	75.3032730601968\\
4806.50000000002	75.3068697208006\\
4807.60000000005	75.3041995132808\\
4809.09999999994	75.3015054478915\\
4809.99999999996	75.3061266917528\\
4811.10000000002	75.3096940472231\\
4811.40000000006	75.3070767951782\\
4811.79999999993	75.3091294499054\\
4812.09999999998	75.3044345621545\\
4813.10000000003	75.3012548824067\\
4815.00000000002	75.3068472098191\\
4815.29999999994	75.304232254849\\
4816.40000000006	75.3015675282882\\
4816.7	75.2989536621824\\
4817.50000000002	75.3009797409498\\
4817.70000000002	75.2978537921873\\
4819.69999999996	75.3081040707083\\
4820.9	75.3038788633064\\
4821.69999999995	75.3059023601145\\
4822.90000000007	75.3099730458221\\
4823.30000000005	75.3078741136957\\
4825.29999999996	75.3160359762921\\
4825.7	75.3139375854781\\
4826.39999999993	75.3175178700922\\
4826.89999999993	75.313859540087\\
4827.30000000001	75.3159050420516\\
4827.70000000005	75.31173619454\\
4828.09999999995	75.3137815334907\\
4828.50000000006	75.3116845462453\\
4829.60000000003	75.3152369712405\\
4829.79999999996	75.3121182633181\\
4830.60000000006	75.3099964808413\\
4830.99999999999	75.3120407360642\\
4832.30000000007	75.3166128631736\\
4833.30000000008	75.3196507634377\\
4833.50000000002	75.3165342601787\\
4834.79999999999	75.3190345198453\\
4836.39999999997	75.3230641993177\\
4837.20000000004	75.3209435015401\\
4837.90000000001	75.3245142620918\\
4838.30000000005	75.3203538359788\\
4839.09999999994	75.3244337907092\\
4841.29999999994	75.3273846408064\\
4841.70000000007	75.3252922466851\\
4842.59999999995	75.3216181056023\\
4843.20000000002	75.3246753246753\\
4844.39999999994	75.3266590979461\\
4844.60000000008	75.3235494457861\\
4845.00000000006	75.3255866751976\\
4847.00000000005	75.331641600132\\
4847.40000000007	75.3295513151109\\
4847.79999999996	75.3274613750284\\
4848.30000000007	75.3300057751011\\
4849.10000000007	75.3278891363524\\
4849.50000000003	75.3258000659848\\
4849.99999999996	75.3283437454898\\
4851.39999999997	75.3313408224261\\
4851.70000000005	75.3287439713096\\
4852.59999999998	75.3312588868053\\
4853.09999999999	75.3276188906289\\
4854.39999999994	75.3239262539911\\
4855.00000000002	75.3269757574509\\
4856.29999999993	75.3294621530352\\
4857.49999999994	75.3334980237154\\
4857.69999999993	75.3303964757709\\
4858.29999999998	75.3272682364564\\
4858.59999999997	75.3246753246753\\
4861.30000000005	75.3301518081211\\
4862.10000000003	75.3321541688947\\
4862.70000000003	75.3290285432261\\
4864.29999999996	75.3309760710468\\
4865.10000000005	75.3288662336595\\
4865.80000000004	75.326250025689\\
4866.20000000004	75.3282781579434\\
4866.40000000001	75.3251823692592\\
4866.89999999998	75.3277172796384\\
4867.30000000007	75.3256358630891\\
4868.20000000008	75.3281432943738\\
4869.20000000001	75.3229416959317\\
4870.00000000005	75.3249419929776\\
4871.6	75.3227826015559\\
4872.29999999997	75.3263278876939\\
4874.49999999995	75.333360686005\\
4875.99999999999	75.3368470704046\\
4876.19999999996	75.333757151939\\
4876.59999999997	75.3316792093014\\
4877.80000000001	75.3377478013079\\
4878.50000000005	75.3330873611282\\
4878.89999999995	75.3310104529617\\
4879.19999999998	75.3284282581518\\
4880.49999999996	75.3329508666967\\
4881.99999999993	75.3282398967657\\
4882.40000000007	75.3302611367127\\
4883.29999999995	75.3327599623213\\
4884.19999999995	75.3352578670434\\
4885.50000000006	75.3315867037825\\
4886.09999999995	75.3346158569031\\
4886.39999999994	75.3320372454722\\
4887.59999999993	75.3340016776807\\
4888.60000000007	75.326773988995\\
4891.09999999997	75.3312070657507\\
4891.80000000004	75.3286044277275\\
4892.09999999997	75.3260291893218\\
4892.60000000003	75.3285506979786\\
4893.00000000007	75.326480145511\\
4895.20000000004	75.3375686883337\\
4895.8	75.332421005331\\
4896.2	75.3344362069318\\
4896.99999999998	75.3364235976394\\
4899.29999999998	75.3337143323672\\
4899.89999999995	75.3367346938775\\
4902.30000000006	75.3386096605744\\
4903.09999999996	75.3405938978626\\
4903.30000000005	75.3375209038626\\
4903.69999999997	75.3395326073657\\
4904.49999999994	75.3415161277168\\
4904.79999999994	75.3389467675182\\
4905.19999999997	75.3409577395878\\
4905.6	75.3388914935687\\
4905.99999999999	75.3409021422311\\
4906.30000000003	75.3383336050872\\
4906.79999999999	75.3408465630031\\
4907.10000000006	75.3382784479948\\
4908.30000000002	75.3443077173824\\
4908.60000000004	75.3417401756066\\
4908.99999999999	75.3437493634271\\
4909.50000000001	75.3401499103797\\
4910.00000000008	75.3426610455999\\
4911.29999999997	75.3471515250234\\
4912.50000000005	75.34910230835\\
4914.30000000001	75.3520266970536\\
4914.50000000005	75.3489602409148\\
4915.90000000005	75.3458096013019\\
4916.80000000001	75.3482885557974\\
4917.70000000005	75.3507666029525\\
4918.30000000008	75.3436076772934\\
4918.70000000008	75.3415467187119\\
4919.49999999992	75.3394584925604\\
4919.89999999997	75.3414634146341\\
4920.09999999993	75.3384008780131\\
4921.70000000003	75.3443862001707\\
4922.00000000002	75.3418256435261\\
4922.8	75.3458327408641\\
4923.10000000003	75.3432726681833\\
4923.59999999994	75.339683571298\\
4923.90000000006	75.3371242891958\\
4924.59999999997	75.3406298860844\\
4924.79999999997	75.3375703059961\\
4925.59999999998	75.3395456483343\\
4925.80000000007	75.3364867333888\\
4927.60000000007	75.3434665259655\\
4927.80000000006	75.3404086933582\\
4928.19999999996	75.3383519672098\\
4928.70000000004	75.3408537575069\\
4929.70000000002	75.3336849365086\\
4930.09999999996	75.331629548497\\
4930.90000000007	75.3295477590752\\
4931.30000000003	75.3274932067973\\
4933.29999999999	75.3354684396157\\
4933.50000000001	75.332414464083\\
4933.99999999995	75.334914168744\\
4934.29999999995	75.3303339818418\\
4935.00000000001	75.3338331543434\\
4935.29999999995	75.3292539611784\\
4935.90000000001	75.3322528363047\\
4936.20000000004	75.3297003828779\\
4937.10000000008	75.334197520862\\
4938.39999999994	75.3366406803685\\
4940.30000000001	75.3441016921707\\
4940.79999999998	75.3405250055658\\
4941.19999999995	75.3425211988748\\
4941.39999999999	75.3394718202975\\
4941.79999999999	75.3374208300451\\
4942.89999999997	75.334816912806\\
4945.70000000003	75.3427150309353\\
4945.99999999998	75.3401670002628\\
4946.60000000005	75.3431580649726\\
4946.89999999995	75.338589043865\\
4947.59999999995	75.3360147139075\\
4948.6	75.3389779133914\\
4948.89999999993	75.3344109921196\\
4950.39999999999	75.3398646601353\\
4950.70000000001	75.3352993455603\\
4951.79999999998	75.3327005795755\\
4952.19999999999	75.3346929709428\\
4952.40000000001	75.331650681474\\
4952.80000000006	75.3336429162713\\
4953.10000000005	75.3310990874586\\
4956.2	75.3384581240038\\
4956.40000000007	75.3354181377989\\
4958.09999999993	75.3378242103989\\
4958.89999999997	75.3357531760436\\
4959.29999999994	75.3337097229504\\
4961.99999999994	75.3390701517503\\
4962.49999999993	75.3334945391529\\
4963.40000000006	75.33796716027\\
4963.80000000008	75.3359253812527\\
4965.00000000002	75.331816076212\\
4965.29999999997	75.3292786079671\\
4965.90000000002	75.3262182843335\\
4966.59999999998	75.3296957738539\\
4968.29999999996	75.3361243056115\\
4970.40000000003	75.3384971330852\\
4970.80000000007	75.3364581866463\\
4971.20000000004	75.3344195683222\\
4971.59999999995	75.3364040469055\\
4973.49999999998	75.341804728969\\
4973.69999999995	75.3387751819534\\
4975.00000000008	75.345219191574\\
4975.20000000003	75.3421904206782\\
4977.09999999998	75.3395483404324\\
4978.60000000002	75.3449695703698\\
4978.90000000003	75.3424382406106\\
4979.99999999994	75.3478845806309\\
4980.40000000001	75.3458488103604\\
4982.00000000008	75.3517592982879\\
4983.99999999999	75.3536245259927\\
4985.29999999995	75.3480162073254\\
4986.50000000006	75.3499378333935\\
4988.09999999995	75.3558397818853\\
4989.29999999997	75.3597626969175\\
4989.49999999998	75.3567420234087\\
4990.20000000003	75.3601987856441\\
4990.39999999996	75.3571786394149\\
4991.59999999993	75.3550894484845\\
4992.10000000002	75.3515484155282\\
4992.70000000006	75.354510495113\\
4993.20000000003	75.3469649330102\\
4994.8	75.3548619591984\\
4996.39999999998	75.352746922846\\
4996.99999999993	75.3557063096596\\
4997.30000000005	75.3511826149598\\
4997.69999999994	75.3531553883709\\
4998.80000000007	75.3565784472584\\
5000.60000000008	75.3594496770452\\
5001.1	75.3559145805007\\
5002.30000000005	75.3618263233648\\
5003.20000000004	75.3642595886715\\
5005.40000000001	75.3670961941864\\
5006.4	75.3700189753321\\
5007.89999999993	75.3674121405751\\
5008.29999999999	75.3653861512659\\
5009.09999999997	75.3673241236126\\
5011.30000000004	75.3761423953386\\
5012.00000000008	75.373595897927\\
5013.10000000001	75.3710205058645\\
5013.50000000006	75.3729854794958\\
5014.99999999999	75.3683874698411\\
5015.40000000004	75.3703519090818\\
5016.49999999999	75.367778973807\\
5017.19999999994	75.3652362824627\\
5018.80000000004	75.3611349100401\\
5019.90000000005	75.3565737051793\\
5020.30000000004	75.3585371683531\\
5021.3	75.3554785517983\\
5022.00000000007	75.3589136018797\\
5024.09999999997	75.3612515425341\\
5025.29999999999	75.3571854976718\\
5025.69999999998	75.3591468024991\\
5026.30000000002	75.3561196880471\\
5027.1	75.3600413749204\\
5027.39999999997	75.3575335653904\\
5028.19999999999	75.3594654256906\\
5030.00000000004	75.3623188405797\\
5031.30000000006	75.3666971419486\\
5031.59999999997	75.3641910288769\\
5032.69999999995	75.3675886186616\\
5034.29999999996	75.3694581280788\\
5034.69999999996	75.3674425995074\\
5035.69999999997	75.3703483061281\\
5038.3	75.367180057161\\
5039.49999999994	75.3651083419319\\
5040.79999999994	75.3694776726378\\
5041.20000000003	75.3674647412374\\
5042.79999999994	75.3693311388288\\
5044.20000000006	75.3722022877307\\
5044.50000000001	75.3677199381517\\
5045.00000000001	75.3701611464589\\
5045.2	75.3671734089152\\
5045.90000000003	75.3705905667856\\
5046.60000000006	75.3680622981354\\
5046.99999999993	75.3660517921183\\
5047.49999999999	75.362548537919\\
5048.1	75.3654768036132\\
5048.29999999996	75.3624910862848\\
5050.1	75.367312185656\\
5050.30000000002	75.3643275780136\\
5051.39999999997	75.3677125606255\\
5052.89999999996	75.3710666930536\\
5053.8	75.3734739508103\\
5054.19999999998	75.3714658805374\\
5055.09999999995	75.3758506092736\\
5057.09999999997	75.377679348256\\
5057.39999999995	75.375185368265\\
5058.09999999998	75.372662211854\\
5058.90000000004	75.3706266060486\\
5060.40000000005	75.3779270823041\\
5060.69999999998	75.3754347138792\\
5062.00000000008	75.3778076292448\\
5062.79999999995	75.3797230836082\\
5063.09999999994	75.3772317901722\\
5063.50000000002	75.3791768702109\\
5064.40000000007	75.3815776483365\\
5064.69999999996	75.3771126204391\\
5065.89999999995	75.3789972364785\\
5066.40000000007	75.3755057732162\\
5066.80000000007	75.3774497227101\\
5067.10000000006	75.3729870539943\\
5067.70000000004	75.3759027585935\\
5070.20000000001	75.380155020413\\
5070.60000000001	75.3781529177431\\
5071.39999999995	75.376121463078\\
5072.90000000005	75.3735462251133\\
5073.60000000004	75.3710310030156\\
5074.00000000008	75.3690309611557\\
5075.99999999993	75.366915545399\\
5076.39999999992	75.3649167733675\\
5077.29999999993	75.361405443731\\
5078.10000000001	75.3652869126856\\
5078.60000000007	75.3618051863666\\
5079.29999999996	75.3652006142458\\
5079.8	75.3617197188921\\
5082.90000000006	75.3669093055282\\
5083.20000000001	75.3644286192041\\
5084.40000000008	75.362375848166\\
5084.80000000002	75.3643139491436\\
5084.99999999994	75.3613498259621\\
5085.70000000006	75.3568760077077\\
5086.29999999996	75.3538848694558\\
5086.60000000004	75.3514066093931\\
5087.20000000001	75.3484166453718\\
5087.90000000006	75.3518081761006\\
5088.09999999995	75.3488463503793\\
5088.89999999993	75.3468264885046\\
5089.29999999996	75.348764097929\\
5089.70000000002	75.344807261582\\
5089.99999999993	75.3423311919216\\
5091.10000000003	75.345694531741\\
5091.89999999997	75.3436763550668\\
5092.99999999997	75.3470381496535\\
5093.20000000005	75.3440794769599\\
5093.99999999993	75.3420623858974\\
5094.40000000009	75.3439984296791\\
5094.59999999994	75.3410406893438\\
5095.20000000002	75.3439444193669\\
5095.59999999993	75.3419549816512\\
5096.09999999999	75.3443742396295\\
5096.49999999998	75.3404230271161\\
5097.69999999997	75.3383812625054\\
5098.40000000008	75.3358831028734\\
5099.09999999999	75.3333856291183\\
5099.50000000007	75.3353204172876\\
5100.10000000003	75.3323399082389\\
5101.49999999995	75.3371491296848\\
5103.19999999997	75.3394862148022\\
5105.19999999998	75.3471882161675\\
5106.40000000005	75.3451483403505\\
5107.09999999999	75.3485275689223\\
5107.80000000003	75.3460326161436\\
5108.30000000004	75.3484456972829\\
5109.09999999996	75.3503483911376\\
5109.69999999994	75.3473717170926\\
5110.39999999995	75.3429214362587\\
5110.79999999999	75.3448512003757\\
5111.00000000008	75.3419029171803\\
5112.5	75.3452255212612\\
5112.89999999992	75.3432427146489\\
5113.80000000001	75.3397602612488\\
5114.4	75.342653240786\\
5115.29999999993	75.3391719122649\\
5116.3	75.3420373700258\\
5116.50000000002	75.3390923660243\\
5117.69999999997	75.3429207862754\\
5119.00000000005	75.3472290050986\\
5119.29999999993	75.3447669648787\\
5120.5	75.3427332734445\\
5122.1	75.3445785014252\\
5122.49999999998	75.3425994612111\\
5123.40000000008	75.3449790182492\\
5123.70000000006	75.3405675475233\\
5124.09999999997	75.3385894383513\\
5126.40000000008	75.3418511655125\\
5128.10000000007	75.3383253383253\\
5128.99999999995	75.3426527071026\\
5129.29999999995	75.3401957343939\\
5129.69999999995	75.338219813638\\
5129.99999999996	75.3357634354106\\
5130.49999999992	75.3381670759755\\
5132.99999999994	75.3345931308566\\
5134.59999999999	75.3325413363975\\
5135.60000000008	75.3353973168215\\
5135.90000000002	75.3329439252336\\
5137.2	75.3372394059136\\
5137.99999999998	75.3352406531597\\
5138.90000000006	75.3376143218525\\
5139.89999999995	75.34046692607\\
5140.19999999993	75.3380152909363\\
5141.39999999994	75.3437712729748\\
5141.80000000002	75.339854917443\\
5142.69999999993	75.3441704907832\\
5143.10000000008	75.3421994089283\\
5144.99999999992	75.3396435443432\\
5146.5	75.3429448567987\\
5146.79999999993	75.3385533039305\\
5147.40000000006	75.3355998057309\\
5147.79999999999	75.333631189417\\
5148.20000000006	75.3316628790086\\
5148.59999999999	75.3335793501272\\
5149.50000000002	75.3359484231785\\
5149.80000000005	75.3335016213907\\
5151.10000000003	75.3358440751669\\
5151.40000000008	75.3314568572261\\
5152.60000000007	75.3352611252353\\
5154.00000000005	75.3400205661512\\
5154.90000000001	75.3423860329777\\
5155.69999999997	75.3403933434191\\
5156.90000000005	75.3422532480124\\
5157.29999999999	75.336409818901\\
5158	75.3339407921521\\
5159.69999999997	75.3381914027675\\
5159.90000000009	75.3352713178295\\
5160.79999999996	75.3395725551745\\
5161.80000000009	75.3424126775025\\
5162.40000000009	75.3394673123487\\
5163.09999999994	75.3428106600558\\
5163.70000000001	75.3398659901623\\
5164.60000000001	75.3422270412609\\
5166.20000000002	75.3479279174651\\
5166.49999999995	75.3435528200364\\
5166.99999999995	75.3459387277196\\
5169.00000000006	75.3477394517421\\
5169.90000000001	75.3500967117988\\
5170.10000000003	75.3471819271982\\
5170.49999999995	75.3490890805709\\
5170.80000000005	75.3466514533253\\
5171.30000000004	75.3490350775419\\
5172.20000000004	75.3513910639367\\
5172.69999999996	75.3460408289514\\
5173.50000000002	75.3498531003557\\
5174.69999999998	75.3517044136971\\
5175	75.3492686131669\\
5177.29999999995	75.3544250009657\\
5177.69999999992	75.3505349762447\\
5178.19999999995	75.3529150493405\\
5178.49999999996	75.3504808249334\\
5179.30000000004	75.3542881414836\\
5179.60000000007	75.3499237407572\\
5181.60000000005	75.3555782851188\\
5181.89999999998	75.3531455036665\\
5182.19999999996	75.3507130038786\\
5182.89999999996	75.3540420605827\\
5183.70000000003	75.3559165091246\\
5184.00000000001	75.3534846935823\\
5184.60000000002	75.3486219067641\\
5185.09999999996	75.3509989971457\\
5185.30000000008	75.3480927218729\\
5186	75.345635448603\\
5186.40000000007	75.3436807095344\\
5188.39999999992	75.3531849282066\\
5188.69999999994	75.3507554733272\\
5190.89999999993	75.3534964361395\\
5192.80000000006	75.3605884958309\\
5193.19999999996	75.3567096066085\\
5194.20000000005	75.3614539013919\\
5194.39999999998	75.3585523149485\\
5195.89999999994	75.3560431100847\\
5197.00000000004	75.3612591637644\\
5200.09999999994	75.3740240759971\\
5200.99999999992	75.3763626925074\\
5201.39999999994	75.3744112275305\\
5201.80000000006	75.3724600626694\\
5203.80000000006	75.3761601875516\\
5205.60000000006	75.3827535201798\\
5207.69999999993	75.3773186374285\\
5208.40000000002	75.3806278199098\\
5208.69999999998	75.3782061127323\\
5209.29999999994	75.3752831420125\\
5209.80000000006	75.3718881360487\\
5211.40000000001	75.3698551280821\\
5212.19999999993	75.367879822727\\
5213.79999999997	75.3696848807994\\
5214.20000000002	75.3658209155591\\
5215.10000000007	75.3700720969474\\
5215.30000000001	75.3671818077233\\
5215.69999999998	75.3690708999578\\
5216.49999999994	75.3709312579074\\
5218.29999999996	75.3621799785375\\
5218.69999999999	75.3640683682072\\
5220.70000000004	75.3658443150475\\
5221.10000000005	75.3619857504022\\
5222.90000000009	75.3704767375072\\
5223.19999999996	75.3680623360711\\
5223.60000000002	75.369948503934\\
5225.19999999997	75.3736627562054\\
5225.90000000006	75.3673938002296\\
5226.29999999996	75.3692790448492\\
5226.59999999996	75.3668662827405\\
5226.89999999993	75.3644537975894\\
5227.70000000001	75.3663108764681\\
5229.50000000007	75.3728774667279\\
5229.80000000007	75.3704659744928\\
5230.20000000002	75.3685257059824\\
5231.00000000001	75.3703809906138\\
5231.70000000004	75.3679421996254\\
5232.29999999994	75.3650332543384\\
5233.20000000008	75.3673590277645\\
5234.09999999998	75.3696840013756\\
5234.89999999996	75.3677172874881\\
5235.89999999996	75.3724216959511\\
5236.80000000001	75.3747446008135\\
5238.59999999992	75.3717525340256\\
5239.20000000005	75.3745729391331\\
5239.60000000007	75.3726358379297\\
5241.29999999997	75.3768077231274\\
5241.49999999998	75.3739316239316\\
5242.00000000009	75.3686499685241\\
5242.49999999999	75.3709991225728\\
5242.90000000003	75.3652489032996\\
5243.99999999996	75.3704162773402\\
5244.29999999998	75.3661047974983\\
5245.2	75.3703315348979\\
5245.39999999999	75.3674578209894\\
5245.79999999992	75.3655235517261\\
5247.00000000004	75.3673457719502\\
5247.19999999993	75.3644731576239\\
5248.39999999998	75.3662951319425\\
5248.80000000002	75.3624568957305\\
5250.00000000009	75.3680882268909\\
5250.20000000001	75.3652172256823\\
5250.99999999995	75.3670659480871\\
5251.70000000007	75.3627327773335\\
5252.39999999996	75.3603046168491\\
5253.19999999994	75.3640568785335\\
5253.89999999992	75.3616292348687\\
5255.10000000002	75.3672552899985\\
5255.29999999999	75.3643871065951\\
5259.39999999994	75.3664797033939\\
5259.60000000006	75.3636138943286\\
5262.8	75.3709931786658\\
5263.19999999997	75.369065035244\\
5264.00000000001	75.3728082673202\\
5266.49999999994	75.3807010215319\\
5268.20000000003	75.3772564204772\\
5268.90000000002	75.3805276143481\\
5271.79999999999	75.3826893529847\\
5273	75.3863950996567\\
5274.20000000009	75.3844112014865\\
5274.89999999994	75.381990521327\\
5275.40000000006	75.384323760781\\
5276.10000000003	75.3819036427732\\
5276.80000000007	75.3794841668404\\
5277.60000000003	75.3756371146522\\
5278.30000000006	75.3789026977872\\
5279.50000000008	75.3769224941283\\
5280.80000000003	75.3829839610672\\
5281.00000000006	75.3801291397626\\
5281.69999999993	75.3758188496346\\
5283.99999999993	75.3732896803619\\
5284.50000000002	75.3756197252394\\
5284.80000000007	75.3732331737592\\
5285.39999999995	75.3760287579226\\
5285.70000000008	75.3736425895796\\
5286.20000000008	75.3759718517678\\
5288.49999999994	75.3828990659154\\
5289.30000000001	75.3790600068061\\
5290.09999999999	75.3827832596121\\
5290.49999999998	75.3808641741957\\
5291.19999999997	75.3784514202559\\
5291.80000000004	75.3755739904382\\
5292.60000000009	75.3792960114875\\
5292.90000000003	75.3769129038353\\
5293.49999999998	75.3797037932598\\
5293.89999999993	75.3777861730261\\
5294.50000000002	75.3805764363691\\
5296.29999999992	75.3776149837626\\
5296.70000000003	75.3794743996375\\
5297.2	75.376135012176\\
5298.00000000001	75.3741907476265\\
5298.70000000008	75.3717822903299\\
5299.09999999992	75.3698671497584\\
5299.9	75.3735849056604\\
5301.60000000002	75.3758228492748\\
5301.89999999998	75.3734439834025\\
5302.70000000005	75.3771592366297\\
5304.39999999997	75.3812800452446\\
5305.00000000002	75.3784094550527\\
5306.19999999996	75.3820929838117\\
5306.49999999996	75.3778313797912\\
5307.29999999997	75.3796585898934\\
5307.69999999996	75.3777459587777\\
5308.70000000005	75.3805003013864\\
5308.89999999999	75.3776605763797\\
5309.40000000006	75.3799792824183\\
5310.20000000003	75.3818051710826\\
5310.59999999993	75.3780104317698\\
5311.39999999997	75.3741880824626\\
5311.99999999994	75.376969560061\\
5312.29999999994	75.3745952864995\\
5313.59999999995	75.376856051339\\
5313.80000000001	75.3740190820302\\
5315.2	75.3767426109533\\
5316.10000000008	75.3790301343065\\
5318.00000000003	75.3840657377635\\
5319.19999999994	75.3820991483842\\
5320.00000000004	75.3839213548618\\
5320.69999999998	75.3815215757029\\
5321.90000000009	75.3870725291244\\
5322.40000000007	75.3837482386097\\
5323.60000000002	75.381783346169\\
5323.99999999997	75.3798764110366\\
5324.4	75.3817259836604\\
5325.20000000005	75.3835464668657\\
5325.60000000009	75.3816399722102\\
5327.39999999994	75.3786954481464\\
5327.79999999996	75.3767901049194\\
5328.29999999999	75.379100668118\\
5329.19999999992	75.3813821702663\\
5330.40000000004	75.3831723102898\\
5331.19999999994	75.3812390974059\\
5331.9	75.3788447111778\\
5334.10000000005	75.3815005061677\\
5335.80000000003	75.3837215839877\\
5336.39999999998	75.3808676098567\\
5337.39999999997	75.383606557377\\
5338.70000000001	75.3858544991384\\
5339.00000000002	75.3816186248619\\
5339.89999999997	75.3857677902622\\
5341.10000000003	75.3875533587958\\
5341.29999999995	75.3847305949751\\
5342.10000000007	75.3884167571413\\
5342.40000000009	75.3860552175948\\
5343.50000000007	75.3892506924171\\
5344.90000000001	75.3919550982226\\
5345.20000000002	75.3895945971227\\
5346.00000000006	75.3932773423617\\
5347.20000000004	75.3913189834122\\
5347.60000000006	75.3875497877592\\
5348.99999999996	75.3902525658522\\
5349.39999999993	75.386484718198\\
5350.5	75.3896759241954\\
5350.7	75.3868580399193\\
5351.39999999997	75.3900775483509\\
5351.99999999995	75.3853627548065\\
5352.59999999997	75.3881218824145\\
5352.90000000004	75.3857649915935\\
5354.19999999992	75.3917412173393\\
5354.59999999993	75.3879769174744\\
5355.19999999992	75.390734412638\\
5355.50000000009	75.3883785196803\\
5355.90000000004	75.390216579537\\
5357.19999999998	75.3924551546488\\
5357.60000000006	75.3868264367172\\
5359.09999999992	75.3899835796388\\
5359.49999999993	75.3880886633331\\
5360.79999999992	75.3921916096178\\
5361.10000000009	75.3879728419011\\
5362.30000000006	75.3916156944652\\
5364.00000000008	75.3956861356052\\
5364.19999999996	75.3928751188412\\
5364.80000000008	75.395627131913\\
5365.00000000006	75.3928165365045\\
5365.50000000008	75.3895184135977\\
5367.30000000001	75.3940455341506\\
5368.09999999997	75.3958496330241\\
5368.89999999996	75.3939281057925\\
5370.30000000008	75.3984805601072\\
5370.60000000008	75.3961308581749\\
5371.80000000008	75.4016269848657\\
5373.20000000007	75.3987307613571\\
5374.50000000003	75.4046812786068\\
5375.39999999995	75.4069388894056\\
5377.09999999994	75.4109945696645\\
5377.90000000007	75.4127928597992\\
5378.79999999996	75.4076112216252\\
5379.99999999995	75.4130964108474\\
5380.80000000001	75.4093181438049\\
5382.49999999995	75.4115111656077\\
5382.79999999999	75.4091660629029\\
5383.20000000008	75.4109932569242\\
5383.40000000002	75.4081916968515\\
5383.89999999994	75.4049034175334\\
5384.79999999993	75.4071570502702\\
5385.29999999998	75.4038697218405\\
5385.69999999997	75.4056964610643\\
5386.10000000005	75.403809735992\\
5388.39999999992	75.4068850329405\\
5389.20000000001	75.4049691054497\\
5389.69999999994	75.4016846636239\\
5390.40000000008	75.4048789537149\\
5390.80000000006	75.4029939342225\\
5391.60000000006	75.4010794369123\\
5391.99999999992	75.3973405537731\\
5392.79999999994	75.3991359009067\\
5393.19999999994	75.3972521461814\\
5393.69999999996	75.3939708554266\\
5394.50000000003	75.3957661365069\\
5394.70000000006	75.3929710091199\\
5396.09999999998	75.3975019458137\\
5396.89999999995	75.3955901426718\\
5397.89999999996	75.3982956650611\\
5398.19999999992	75.3941055517478\\
5398.49999999992	75.3917682362094\\
5398.90000000008	75.3935914058159\\
5400.89999999995	75.3915941492316\\
5401.49999999995	75.3943276066351\\
5401.80000000007	75.3919917066217\\
5403.29999999992	75.3969722767147\\
5403.50000000005	75.3941816566733\\
5403.80000000007	75.3918466292862\\
5404.70000000005	75.3940941385435\\
5408.49999999994	75.3965906149466\\
5409.49999999994	75.3992901508429\\
5410.79999999994	75.4033524921917\\
5411.60000000003	75.4014450172774\\
5412.09999999994	75.3963268171908\\
5412.80000000002	75.3939662657725\\
5413.49999999996	75.3916063248116\\
5414.00000000002	75.3938789457158\\
5415.10000000007	75.397030580588\\
5416.30000000009	75.3932501292371\\
5416.90000000006	75.3959756322688\\
5417.20000000009	75.3918003433445\\
5419.20000000005	75.3990367759674\\
5419.39999999997	75.3962542669988\\
5420.19999999999	75.3943508661882\\
5421.59999999999	75.4007045760555\\
5421.80000000006	75.3979232372416\\
5422.70000000004	75.4020063435863\\
5424.09999999994	75.4046679694702\\
5424.50000000007	75.4027946761052\\
5425.4	75.4050317943047\\
5427.4	75.4067250115154\\
5427.79999999994	75.4030103723355\\
5428.60000000009	75.4047930443753\\
5430.60000000007	75.4028025853021\\
5431.09999999996	75.4050670201797\\
5432.20000000006	75.4082064687149\\
5433.90000000006	75.403018034597\\
5434.60000000007	75.4061861740298\\
5435.00000000002	75.4043163879229\\
5436.39999999999	75.4106502345259\\
5437.19999999996	75.4124289629044\\
5437.50000000006	75.4082683536854\\
5438.29999999995	75.4100470726684\\
5438.60000000007	75.4077261110192\\
5439.10000000007	75.4099867627592\\
5439.60000000003	75.4048936522235\\
5440.69999999994	75.402514336127\\
5441.40000000006	75.4001653955711\\
5443.00000000008	75.3963733901637\\
5443.49999999998	75.3986332574032\\
5445.49999999995	75.4003231967093\\
5445.79999999996	75.3980058392552\\
5446.29999999995	75.4002643948296\\
5446.49999999998	75.3974956853817\\
5447.70000000002	75.3955725246889\\
5449.20000000004	75.387664470666\\
5450.70000000004	75.390768327585\\
5451.10000000003	75.3870707367185\\
5451.70000000002	75.3824424960563\\
5452.40000000003	75.38010087116\\
5453.49999999994	75.3850667449024\\
5454.50000000006	75.3877461225388\\
5455.09999999996	75.3831206921836\\
5455.9	75.3812316715542\\
5457.10000000008	75.3774829582936\\
5457.80000000008	75.3806409058429\\
5460.10000000005	75.3781912750449\\
5461.10000000003	75.3808686735516\\
5461.29999999999	75.3781081773904\\
5461.90000000006	75.3808128890516\\
5462.20000000006	75.3785035607711\\
5462.60000000004	75.3803064418694\\
5463.39999999998	75.3784204264665\\
5464.60000000008	75.381997181913\\
5464.80000000006	75.3792384124138\\
5465.09999999999	75.3769303959599\\
5467.10000000007	75.3841088674276\\
5467.70000000001	75.3813233841764\\
5468.69999999999	75.3858250438853\\
5469.10000000007	75.3839684048855\\
5469.60000000002	75.3862186225936\\
5470.59999999995	75.3834061454658\\
5471.10000000004	75.3856557976312\\
5472.3	75.3892259337768\\
5473.79999999991	75.392316264455\\
5474.60000000008	75.3940855206678\\
5474.80000000001	75.3913313485178\\
5475.80000000006	75.3939991599554\\
5477.3	75.3970862087852\\
5477.99999999998	75.3929282050346\\
5479.69999999992	75.3950874119493\\
5480.40000000005	75.3927561353891\\
5480.79999999997	75.3909029538944\\
5482.40000000004	75.3889648882809\\
5483.20000000008	75.3907318585523\\
5483.79999999992	75.3879538284797\\
5485.69999999993	75.3910095154763\\
5486.79999999996	75.3941205416537\\
5486.99999999995	75.3913724918445\\
5493.10000000001	75.3950338600451\\
5494.29999999994	75.3931275480489\\
5494.99999999997	75.3962621244381\\
5495.3	75.3939658623576\\
5496.4	75.3970708632766\\
5496.80000000006	75.3952227619204\\
5497.79999999999	75.3978791902363\\
5499.00000000006	75.3996108454111\\
5500.09999999999	75.4027126286317\\
5501.30000000004	75.400807067292\\
5502.50000000002	75.398902337077\\
5504.19999999992	75.3955998037898\\
5505.00000000006	75.3973588127373\\
5505.29999999993	75.3950666618229\\
5506.00000000002	75.392746226912\\
5506.40000000007	75.3909016616726\\
5507.19999999999	75.3944764222033\\
5507.60000000007	75.3926321331953\\
5507.99999999994	75.3889726039832\\
5508.59999999999	75.3916532031151\\
5508.80000000005	75.3889161175552\\
5510.60000000008	75.3915110603009\\
5510.99999999998	75.3896681243309\\
5511.50000000003	75.3864576529501\\
5512.60000000008	75.3841130480527\\
5513.00000000008	75.3858990404673\\
5516.09999999994	75.3834161197926\\
5516.89999999998	75.3779227841218\\
5517.30000000007	75.3797078333998\\
5518.09999999998	75.3814649704614\\
5518.50000000005	75.3778132134962\\
5519.60000000004	75.3754733047086\\
5520.49999999992	75.3722421475926\\
5520.90000000005	75.3740264444847\\
5522.10000000004	75.3757560392597\\
5522.40000000001	75.3734721593481\\
5523.10000000006	75.3765932792584\\
5523.30000000003	75.3738639243944\\
5524.80000000003	75.3678799616283\\
5525.50000000009	75.371000434342\\
5525.79999999994	75.366908557882\\
5526.39999999993	75.3695829186646\\
5528.30000000006	75.3744302148904\\
5528.69999999996	75.3725944147012\\
5529.1	75.3707588801273\\
5530.5	75.3751853325137\\
5530.79999999998	75.3710969281672\\
5531.29999999995	75.3733232093141\\
5532.89999999997	75.3714079161395\\
5534.89999999998	75.3803071364047\\
5535.49999999996	75.3721367150806\\
5536	75.374361012265\\
5536.20000000006	75.3716380976464\\
5536.59999999994	75.3679989885672\\
5536.99999999994	75.3661664047967\\
5537.60000000005	75.3688354370948\\
5538.00000000009	75.367003123815\\
5538.49999999999	75.3692268804391\\
5541.40000000005	75.3748984931878\\
5541.90000000002	75.3717069649946\\
5542.29999999999	75.3734844110854\\
5543.69999999996	75.3760958187525\\
5544.99999999997	75.3800652828623\\
5546.89999999993	75.3776816297098\\
5548.49999999995	75.3739682081967\\
5549.19999999996	75.3770745859838\\
5550.30000000009	75.3747477659268\\
5550.99999999991	75.3724487038605\\
5551.69999999997	75.3701502215498\\
5552.39999999998	75.3678523187753\\
5553.39999999998	75.3704870802197\\
5556.30000000008	75.3743430998488\\
5559.90000000001	75.3848920863309\\
5561.20000000006	75.3870497905166\\
5561.59999999992	75.3834259309204\\
5561.90000000001	75.381157856886\\
5563.20000000001	75.3869106465587\\
5563.99999999998	75.3850577811326\\
5564.40000000007	75.3832329948782\\
5565.00000000008	75.3858870460549\\
5566.20000000008	75.3893969063831\\
5566.99999999994	75.3875446821505\\
5567.60000000003	75.390197029294\\
5568.00000000007	75.3883730536449\\
5568.79999999999	75.3865215751764\\
5570.10000000009	75.3922659868586\\
5570.69999999995	75.3895311265886\\
5571.19999999994	75.3917398093802\\
5571.70000000008	75.3885638393338\\
5572.50000000001	75.3867135627894\\
5573.09999999998	75.3893633818991\\
5573.99999999994	75.391543029368\\
5574.20000000005	75.3888380603843\\
5575.39999999997	75.3923414940364\\
5577.30000000003	75.3899666511278\\
5577.60000000003	75.3877046094268\\
5578.09999999991	75.3899107238894\\
5578.49999999999	75.3880902018428\\
5579.20000000005	75.3911781047802\\
5580.39999999995	75.3893020338679\\
5580.70000000007	75.3870412844037\\
5582.00000000003	75.3838161265474\\
5583.59999999997	75.3819152175081\\
5584.29999999999	75.3796289664064\\
5585.09999999994	75.3759936976295\\
5586.7	75.3830457506981\\
5586.89999999996	75.3803472346519\\
5588.50000000007	75.3784489854346\\
5591.10000000009	75.3827443124911\\
5591.49999999997	75.3791401387796\\
5592.30000000002	75.3826621843931\\
5592.70000000006	75.3790587898727\\
5596.80000000005	75.3845878968715\\
5597.00000000003	75.3818941952082\\
5597.70000000002	75.3849726678338\\
5599.10000000009	75.3875553650522\\
5599.30000000003	75.3848626638569\\
5600.10000000004	75.3883789864648\\
5601.00000000002	75.3851921943904\\
5601.60000000008	75.3878286948605\\
5601.79999999997	75.3851371855977\\
5602.30000000002	75.3873339997144\\
5602.79999999998	75.3806064716486\\
5603.69999999996	75.3845604768193\\
5604.10000000005	75.3827486527961\\
5606.2000000001	75.3866186254749\\
5606.40000000009	75.3839293676982\\
5606.80000000008	75.3821184611818\\
5607.29999999999	75.3789635125013\\
5607.6	75.3767141608859\\
5608.60000000003	75.3811043557331\\
5608.80000000007	75.3784164452923\\
5611.49999999991	75.3866989806829\\
5612.7	75.384834663626\\
5613.49999999999	75.3829984323785\\
5614.39999999998	75.386944518657\\
5614.59999999998	75.3842591768038\\
5616.09999999997	75.3783697161782\\
5617.99999999998	75.3813566864242\\
5618.20000000003	75.3786732641546\\
5620.10000000005	75.3852176079143\\
5621.30000000004	75.383356459245\\
5621.90000000008	75.3859836357168\\
5622.09999999995	75.3833019102842\\
5623.30000000002	75.3849983995448\\
5623.60000000004	75.382755125629\\
5624.70000000009	75.3875693357986\\
5624.90000000006	75.3848888888889\\
5626.59999999996	75.3905486341906\\
5626.90000000003	75.3865292340501\\
5628.79999999998	75.384178080975\\
5629.49999999998	75.3872388802046\\
5630.09999999995	75.3827572732763\\
5630.89999999992	75.3844787781922\\
5631.29999999998	75.3826757111908\\
5632.09999999999	75.3808458506445\\
5632.89999999997	75.3772412568791\\
5633.29999999999	75.3754393439131\\
5633.70000000006	75.3736376868188\\
5634.69999999999	75.3780080925676\\
5635.10000000001	75.3762067007382\\
5635.50000000001	75.377954432536\\
5635.79999999992	75.375716389574\\
5636.49999999994	75.3734520810418\\
5637.29999999999	75.3716252172988\\
5637.99999999999	75.3746829605718\\
5640.10000000007	75.3767596893727\\
5640.49999999995	75.3749601106265\\
5640.90000000005	75.3767062577557\\
5641.40000000004	75.3717982805991\\
5641.99999999994	75.3673277680296\\
5642.4	75.3690739920248\\
5643.49999999992	75.3667871571337\\
5644.70000000008	75.3720238095238\\
5645.10000000002	75.3702260327358\\
5645.80000000007	75.3679661347172\\
5646.60000000008	75.3714558945933\\
5646.79999999996	75.3687864137845\\
5647.20000000002	75.3652187771147\\
5648.00000000001	75.3687080611179\\
5648.40000000005	75.3669115694432\\
5649.39999999993	75.3641915213736\\
5650.59999999997	75.3694232572956\\
5651.40000000005	75.3658320799788\\
5652.10000000002	75.3688829128481\\
5652.39999999998	75.3666519239275\\
5653.29999999993	75.3688046131531\\
5654.29999999998	75.3713921901528\\
5654.69999999997	75.3678291009408\\
5655.20000000001	75.3700068961859\\
5655.40000000005	75.3673415259482\\
5656.20000000001	75.365521630748\\
5657.30000000003	75.3703114504896\\
5657.80000000008	75.3671857049435\\
5658.20000000006	75.365392432356\\
5658.89999999997	75.3684396536491\\
5659.99999999991	75.3714598682002\\
5660.40000000007	75.3696669905485\\
5662.80000000006	75.3783397199315\\
5663.59999999996	75.3800519095291\\
5664.00000000006	75.3782595646263\\
5664.80000000004	75.3799714028491\\
5665.19999999992	75.3781794432775\\
5666.19999999996	75.380759931525\\
5666.69999999996	75.3776381732195\\
5667.39999999999	75.3753859726511\\
5668.00000000003	75.3726998465094\\
5668.80000000003	75.3761752720986\\
5669.09999999996	75.373950469202\\
5671.10000000003	75.3720552969389\\
5671.70000000009	75.3746606015727\\
5672.39999999998	75.368884971353\\
5672.80000000007	75.370621727864\\
5674.99999999993	75.3731211784814\\
5675.29999999993	75.3691369771294\\
5675.6	75.3669150941734\\
5675.89999999993	75.3646934460888\\
5677.40000000006	75.369440774989\\
5677.60000000005	75.3667858463815\\
5679.99999999995	75.3648703367898\\
5680.69999999995	75.3626249823968\\
5681.10000000006	75.3608392593114\\
5681.70000000008	75.3634411630117\\
5682.00000000001	75.3612220833847\\
5682.40000000003	75.3629564452266\\
5683.49999999999	75.3606868885917\\
5685.89999999993	75.3640520576856\\
5686.69999999996	75.3657593022438\\
5687.00000000001	75.3635420513091\\
5687.79999999996	75.3652490374303\\
5688.19999999993	75.3617073642389\\
5689.10000000008	75.3656050059762\\
5690.99999999998	75.3685579237757\\
5691.50000000003	75.3654508398341\\
5692.50000000006	75.3680216421319\\
5693.3	75.3697263498086\\
5694.70000000006	75.3722694387863\\
5695.10000000007	75.3687315634218\\
5695.39999999995	75.3665174260381\\
5696.29999999999	75.3686538866653\\
5696.49999999993	75.3660077941228\\
5697.40000000008	75.3681439227731\\
5697.60000000004	75.365498358987\\
5698.30000000006	75.3632598624175\\
5699.89999999994	75.3614035087719\\
5700.39999999997	75.3635645995965\\
5700.69999999993	75.3613527925905\\
5702.49999999992	75.3691298705854\\
5702.69999999992	75.3664866381427\\
5703.90000000008	75.3699158485273\\
5704.10000000008	75.3672732372638\\
5705.59999999996	75.3702437912964\\
5705.80000000004	75.3676019558702\\
5708.4	75.3700621879653\\
5708.70000000002	75.3678531390134\\
5709.39999999992	75.3708731062265\\
5710.20000000008	75.3725723692275\\
5710.39999999996	75.3699325803345\\
5710.89999999999	75.3720889511469\\
5711.10000000008	75.3694495027315\\
5712.79999999997	75.3732780199198\\
5713.20000000001	75.3697512820962\\
5714.00000000009	75.3714495721111\\
5714.20000000001	75.368811577971\\
5714.70000000004	75.3709666130048\\
5715.50000000004	75.3656658968437\\
5716.10000000002	75.3630033938631\\
5716.89999999995	75.3664509358055\\
5717.09999999994	75.3638144546281\\
5717.79999999999	75.3615837982476\\
5718.40000000006	75.3641689254175\\
5718.80000000006	75.3606462781304\\
5719.29999999996	75.3558065531349\\
5720.09999999992	75.359253172966\\
5720.99999999997	75.3613815524986\\
5721.30000000006	75.3574299996504\\
5722.40000000006	75.3604193971166\\
5722.79999999994	75.3551521081969\\
5724.70000000002	75.3580911123533\\
5725.69999999995	75.3606482936882\\
5726.69999999995	75.357966054341\\
5727.30000000001	75.351817578657\\
5728.29999999993	75.3561203826549\\
5728.50000000001	75.3534895087805\\
5729.30000000009	75.3499493838796\\
5731.99999999994	75.3545820903334\\
5732.19999999995	75.3519529682675\\
5733.40000000004	75.3553675765239\\
5733.59999999996	75.352739069013\\
5734.2	75.3500863226549\\
5735.00000000007	75.3482938396889\\
5737.59999999997	75.3507503006431\\
5738.20000000005	75.3480996113832\\
5738.79999999991	75.3506769589991\\
5738.99999999999	75.3480510881497\\
5740.00000000003	75.3453772582359\\
5741.29999999996	75.3422510189152\\
5742.29999999995	75.3465449986069\\
5742.60000000001	75.3443502185383\\
5743.20000000008	75.3469259833197\\
5743.50000000002	75.3447315272651\\
5743.90000000009	75.3429665738162\\
5745.50000000005	75.3463519910888\\
5745.69999999997	75.34372933273\\
5746.10000000008	75.3454456858446\\
5746.40000000005	75.3432524145132\\
5747.80000000002	75.347518224047\\
5748.09999999993	75.3435858181692\\
5748.60000000006	75.3457303390332\\
5749.09999999997	75.3409169971474\\
5750.19999999996	75.3386779820183\\
5750.89999999997	75.3416797078769\\
5751.90000000008	75.3442280945758\\
5753.39999999998	75.3402276874946\\
5754.40000000007	75.3427752193935\\
5755.00000000003	75.340133099338\\
5755.49999999999	75.3353255959414\\
5756.1	75.3378965289601\\
5757.20000000008	75.340871589113\\
5757.80000000001	75.3382309522569\\
5760.00000000001	75.3407058905227\\
5761.30000000003	75.3375915576075\\
5762.5	75.3357859299622\\
5764.49999999996	75.3391388821427\\
5764.7000000001	75.3365251179573\\
5765.90000000006	75.3347207769684\\
5766.79999999993	75.331633980128\\
5767.80000000003	75.3341770835139\\
5768.90000000007	75.3319466111978\\
5769.20000000009	75.3297627095141\\
5769.60000000002	75.3314730401927\\
5769.99999999999	75.3297169893069\\
5770.39999999992	75.3314270860411\\
5770.59999999994	75.3288162614588\\
5770.99999999997	75.3305262428307\\
5771.29999999994	75.3266105277749\\
5772.49999999994	75.3248103107785\\
5773.20000000004	75.322605788717\\
5773.70000000002	75.3247428036995\\
5774.00000000008	75.3208292201382\\
5774.80000000003	75.3242480389271\\
5775.09999999999	75.3203352264857\\
5776.00000000009	75.3172555876803\\
5776.59999999994	75.3198192739799\\
5777.20000000004	75.317189690686\\
5777.50000000008	75.3150096926059\\
5779.00000000001	75.3196864563686\\
5779.30000000008	75.3157767242274\\
5780.60000000003	75.3213278668673\\
5781.80000000003	75.3195316418478\\
5782.69999999993	75.3216434945009\\
5784.90000000005	75.3258426966292\\
5785.20000000002	75.3236651513318\\
5785.69999999994	75.3257976425041\\
5785.90000000004	75.3231939163498\\
5787.20000000008	75.3287370621879\\
5787.39999999997	75.3261339092873\\
5788.80000000006	75.3321010900171\\
5789.19999999994	75.3303508196155\\
5790.30000000009	75.328129317491\\
5791.00000000003	75.325931170244\\
5791.79999999995	75.3241596022031\\
5792.40000000003	75.326715580492\\
5793.70000000007	75.3305257344057\\
5793.90000000006	75.3279254401104\\
5795.90000000009	75.3260869565217\\
5797.19999999999	75.3298949510979\\
5797.90000000003	75.327699206623\\
5798.39999999995	75.3298266793136\\
5799.80000000002	75.3323333160917\\
5800.89999999996	75.330115497328\\
5801.60000000001	75.3279211265664\\
5802.19999999996	75.3304723988763\\
5802.50000000003	75.3265777410126\\
5803.79999999997	75.32865831596\\
5804.19999999997	75.3269128060231\\
5805.50000000009	75.3324376464104\\
5806.6	75.335388430606\\
5806.89999999998	75.3314964697779\\
5807.29999999993	75.3331955780556\\
5807.50000000003	75.33060128108\\
5807.90000000003	75.3323002754821\\
5809.49999999994	75.3356513357202\\
5811.70000000008	75.3398258715028\\
5812.00000000001	75.3376576452573\\
5812.30000000004	75.3354896428326\\
5812.89999999994	75.332874591433\\
5813.70000000002	75.3362688774984\\
5814.10000000009	75.3345258161054\\
5814.9	75.3379191745486\\
5815.40000000009	75.3348809216748\\
5815.79999999997	75.3331384652418\\
5817.09999999991	75.3386508973389\\
5817.99999999996	75.3407469792544\\
5818.29999999996	75.3368623676612\\
5818.90000000007	75.3342498711119\\
5819.50000000004	75.336792906729\\
5819.70000000007	75.3342039245335\\
5820.89999999994	75.3375708641127\\
5822.29999999997	75.3349134377576\\
5824.90000000006	75.3390557939914\\
5825.20000000007	75.3368925205569\\
5827.60000000008	75.3401856650136\\
5828.29999999997	75.3380001372589\\
5828.80000000007	75.3401156307365\\
5829.19999999996	75.3383768205445\\
5830.00000000004	75.3417608617348\\
5830.19999999999	75.3391763717133\\
5830.60000000004	75.3374380434596\\
5833.39999999993	75.3441330247707\\
5833.59999999995	75.3415499597168\\
5834.60000000004	75.3389205957461\\
5835.40000000009	75.3423014308971\\
5835.70000000006	75.340141882861\\
5836.09999999995	75.3366916829444\\
5836.99999999991	75.3387812441109\\
5838.10000000003	75.3417149121305\\
5839.40000000009	75.3437794331706\\
5839.60000000003	75.341199034197\\
5840.7	75.3458430352007\\
5841.09999999993	75.3441073751969\\
5842.10000000009	75.3466160008216\\
5842.79999999999	75.3444351263927\\
5843.59999999998	75.3478104625494\\
5844.39999999993	75.3460518436137\\
5844.69999999998	75.343895428415\\
5845.10000000004	75.3421610894409\\
5846.50000000002	75.3395135634386\\
5847.09999999994	75.3420440552743\\
5848.70000000001	75.3402407331418\\
5849.39999999995	75.3431917257885\\
5849.99999999996	75.3405924685048\\
5850.90000000001	75.3358400273458\\
5851.60000000003	75.338790436967\\
5851.89999999996	75.3366370471634\\
5852.99999999997	75.341272146384\\
5853.30000000007	75.3391191444289\\
5853.89999999996	75.3416467372737\\
5854.29999999999	75.3399152773982\\
5854.99999999999	75.3428634865331\\
5855.30000000003	75.3407111384363\\
5856.20000000009	75.3427932312211\\
5857.1000000001	75.3448746841494\\
5859.99999999998	75.3502499957339\\
5860.20000000008	75.3476784464959\\
5860.69999999995	75.3497815997816\\
5863.19999999996	75.3517643647775\\
5863.59999999997	75.3500349608609\\
5865.19999999997	75.353349359794\\
5865.50000000006	75.3512002182215\\
5866.39999999997	75.3532770817353\\
5867.09999999992	75.3511044450505\\
5868.2	75.3540207555851\\
5868.40000000007	75.3514526710403\\
5868.90000000005	75.353552564321\\
5869.19999999995	75.3514047671784\\
5869.80000000003	75.3539242576534\\
5870.89999999998	75.3568386986885\\
5871.09999999999	75.3542716991416\\
5873.39999999994	75.3588150166\\
5874.39999999999	75.361307345306\\
5874.59999999995	75.3587417229816\\
5875.59999999992	75.3629354800279\\
5875.90000000004	75.3607896528251\\
5877.19999999997	75.3628366767053\\
5877.59999999994	75.3611106385151\\
5878.09999999993	75.3632064237351\\
5879.89999999999	75.3656462585034\\
5880.59999999992	75.3634771370755\\
5881.69999999992	75.3680845999524\\
5881.99999999993	75.3659407354516\\
5882.50000000001	75.3680345425492\\
5882.69999999998	75.365472224111\\
5884.39999999991	75.3691902455604\\
5884.89999999998	75.3661852166525\\
5885.80000000009	75.3682529434751\\
5887.50000000006	75.3702697194103\\
5887.80000000006	75.3681278554323\\
5889.20000000006	75.3739833256924\\
5890.00000000003	75.3705370027674\\
5892.09999999998	75.3725263908218\\
5892.50000000002	75.3708040593286\\
5892.8	75.368663985474\\
5894.10000000006	75.3656136541006\\
5894.40000000001	75.3634744253117\\
5894.69999999996	75.3613354142634\\
5897.50000000001	75.3679462832338\\
5898.80000000006	75.3699842343488\\
5899.00000000002	75.3674289298367\\
5899.70000000002	75.3703515373403\\
5901.20000000001	75.3732228492027\\
5902.89999999991	75.3786210401491\\
5904.79999999996	75.3814628528849\\
5905.00000000003	75.3789097559737\\
5906.09999999991	75.376722765907\\
5906.8	75.3796407591122\\
5907.29999999993	75.3766462403088\\
5908.49999999999	75.3816470906814\\
5909.60000000001	75.3845372861567\\
5909.89999999998	75.3824027072758\\
5911.39999999997	75.3801911528377\\
5912.69999999997	75.3822216208903\\
5913.00000000009	75.3783971182628\\
5915.10000000005	75.3820665404382\\
5916.70000000003	75.3785830178475\\
5917.49999999997	75.376841962958\\
5918.49999999993	75.3793126752948\\
5919.40000000006	75.3746093420052\\
5920.09999999993	75.3775210296949\\
5920.40000000002	75.3753905920108\\
5921.09999999997	75.3783016956022\\
5921.60000000001	75.3753145211679\\
5922.49999999991	75.3773680478168\\
5922.90000000003	75.3722775620463\\
5923.69999999996	75.3688510753233\\
5925.09999999997	75.3729831904408\\
5925.69999999994	75.3704141213001\\
5926.29999999998	75.3678455723542\\
5926.80000000001	75.3699235688134\\
5927.10000000006	75.3661087866109\\
5928.39999999996	75.3698237328161\\
5929.49999999992	75.3676470588235\\
5929.9000000001	75.3659359190556\\
5931.30000000005	75.3683784603972\\
5931.7	75.3666677905526\\
5932.40000000007	75.3695743784239\\
5932.79999999992	75.3678639451196\\
5934.30000000001	75.3656645996225\\
5935.50000000006	75.3622211739335\\
5938.50000000003	75.3595123429765\\
5939.50000000008	75.3636608525827\\
5939.69999999997	75.3611232701438\\
5940.19999999995	75.3631971449253\\
5940.49999999996	75.3593913072754\\
5941.10000000001	75.3551471083283\\
5941.39999999992	75.3530253303038\\
5941.79999999999	75.351318601794\\
5942.60000000006	75.3495885708516\\
5945.49999999996	75.351520452099\\
5945.79999999992	75.3494004271851\\
5949.40000000008	75.3575930750483\\
5950.20000000009	75.3541838226644\\
5950.69999999991	75.3562546212274\\
5950.89999999996	75.3537220635187\\
5951.80000000003	75.3557687461147\\
5952.99999999999	75.3590566259596\\
5953.20000000004	75.3565249525473\\
5954.29999999996	75.3543598011554\\
5955.79999999996	75.3521717960342\\
5956.20000000006	75.3487903564293\\
5957.70000000007	75.3466044513075\\
5958.20000000004	75.3486732792911\\
5959.50000000001	75.3506946774951\\
5961.00000000007	75.3552196742212\\
5962.39999999994	75.3576519916143\\
5963.79999999999	75.3617599221986\\
5965.40000000001	75.3650155058252\\
5965.60000000007	75.3624888948489\\
5968.19999999998	75.3598177035337\\
5968.60000000006	75.358118183189\\
5969.19999999996	75.3605950446451\\
5969.79999999996	75.3580461984288\\
5970.70000000005	75.3600857506532\\
5971.20000000006	75.3571249141728\\
5971.6	75.3554264279853\\
5974.40000000003	75.3653025357771\\
5974.69999999993	75.3631920733748\\
5975.19999999992	75.3652536274329\\
5975.50000000007	75.3631434500301\\
5976.10000000006	75.365616947224\\
5976.29999999995	75.3630948397028\\
5977.10000000003	75.3597001940708\\
5979.30000000004	75.3670936883299\\
5981.00000000002	75.3724231328685\\
5982.69999999996	75.3693922578057\\
5983.30000000006	75.3718621519537\\
5984.29999999994	75.3743065303121\\
5985.49999999994	75.3775728414862\\
5985.70000000008	75.3750542951652\\
5987.6000000001	75.3811981228184\\
5987.89999999999	75.3790915163661\\
5988.70000000002	75.3773710927064\\
5989.40000000005	75.3802487686785\\
5989.60000000006	75.3777317728768\\
5990.39999999997	75.3810199482514\\
5991.90000000003	75.3788384512684\\
5992.50000000003	75.3813036077829\\
5994.10000000006	75.3862066664442\\
5994.49999999996	75.3828445601041\\
5995.40000000004	75.3865399049287\\
5995.70000000005	75.3844357717069\\
5996.20000000009	75.3798175541584\\
5996.99999999999	75.3831018325524\\
5998.30000000003	75.3851026940517\\
5999.40000000009	75.3829485790483\\
6001.30000000008	75.3874096044256\\
6001.49999999997	75.3848973607038\\
6002.09999999991	75.3873579687448\\
6003.40000000009	75.3910219038894\\
6003.80000000006	75.3876646846217\\
6005.19999999994	75.3900721029757\\
6006.49999999991	75.3920687244032\\
6006.79999999996	75.389968203233\\
6007.79999999996	75.3924000066579\\
6010.80000000003	75.3996905621454\\
6011.89999999993	75.3975382568197\\
6012.69999999996	75.3958222458755\\
6013.59999999997	75.3978415950247\\
6014.30000000004	75.3940542697526\\
6016.89999999998	75.3980388898122\\
6017.29999999995	75.3946887359989\\
6017.8	75.3967330796457\\
6017.99999999999	75.3942274139679\\
6019.09999999991	75.397062732589\\
6019.99999999994	75.3990797495058\\
6020.79999999993	75.3957049610523\\
6022.60000000001	75.4013980440666\\
6022.90000000001	75.3993026730865\\
6024.50000000004	75.4025163496332\\
6025.20000000002	75.4003950010788\\
6025.70000000006	75.397457598991\\
6026.40000000009	75.3953372604331\\
6028.40000000005	75.3935473169113\\
6029.2000000001	75.391836531604\\
6029.80000000005	75.3942851456906\\
6030.10000000005	75.3905343106365\\
6030.69999999998	75.3929826888638\\
6033.59999999993	75.3965228632514\\
6034.40000000006	75.3948131576767\\
6035.3000000001	75.3984822878351\\
6036.49999999997	75.3967465129378\\
6036.79999999996	75.3946561977174\\
6039.99999999994	75.3927915100743\\
6041.39999999992	75.3968385334768\\
6041.60000000006	75.3943426519026\\
6043.09999999994	75.392176330421\\
6043.80000000002	75.3950263902447\\
6045.00000000003	75.393293742039\\
6045.69999999991	75.3961427768037\\
6045.90000000005	75.393648693351\\
6047.69999999998	75.3976652667086\\
6047.99999999991	75.3955787768059\\
6049.99999999991	75.3937951438819\\
6050.90000000006	75.3974549661213\\
6052.20000000002	75.3994349255655\\
6053.40000000001	75.4026596184026\\
6054.70000000003	75.3980313139988\\
6056.30000000002	75.4045307443366\\
6056.79999999999	75.401608083343\\
6057.09999999995	75.3995245327874\\
6058.60000000008	75.4056150659382\\
6058.79999999999	75.4031259799634\\
6059.50000000001	75.4010165687504\\
6060.60000000006	75.4054812150412\\
6061.09999999995	75.4025605490662\\
6062	75.4062123686511\\
6062.30000000009	75.4024808656638\\
6063.40000000009	75.4069431846293\\
6065.29999999998	75.4113496224486\\
6065.5	75.4088630968082\\
6066.39999999997	75.4125113327289\\
6067.20000000006	75.409160582137\\
6067.89999999991	75.4119973632169\\
6068.20000000004	75.4082692022478\\
6069.30000000001	75.4110785250601\\
6069.60000000008	75.4073512694202\\
6071.20000000003	75.4055968244033\\
6072.10000000004	75.407595270248\\
6072.29999999995	75.4051116527238\\
6073.60000000004	75.4087294400448\\
6073.80000000002	75.4062463985248\\
6074.89999999995	75.4041152263375\\
6075.89999999991	75.406517445688\\
6077.80000000001	75.4043337336909\\
6078.3000000001	75.4014214266912\\
6078.79999999998	75.3985095987761\\
6080.09999999997	75.3955461991382\\
6080.40000000008	75.3934709316668\\
6081.20000000007	75.3917747849966\\
6081.69999999994	75.3937978887829\\
6082.90000000005	75.3970080552359\\
6084.19999999996	75.3989776966948\\
6085.0000000001	75.3972818852607\\
6085.29999999997	75.3952082032405\\
6085.99999999992	75.3931088874649\\
6087.40000000005	75.3971252566735\\
6090.2	75.393658768862\\
6091.00000000001	75.3968905452217\\
6091.70000000004	75.3915098985522\\
6092.29999999993	75.3939334252511\\
6092.70000000001	75.390625\\
6093.40000000008	75.393452039058\\
6094.00000000005	75.3909519043009\\
6094.89999999997	75.3945857260049\\
6096.80000000001	75.3973330709049\\
6098.09999999998	75.3943786691155\\
6099.80000000003	75.397957343563\\
6100.7	75.3999475478626\\
6101.09999999991	75.3966432832885\\
6102.49999999991	75.3990102579229\\
6102.80000000009	75.396942437202\\
6103.39999999999	75.3993610223642\\
6105.8	75.3975662883441\\
6106.30000000007	75.3995807677191\\
6108.59999999997	75.4022950873345\\
6108.90000000006	75.4002291700769\\
6110.09999999997	75.3985139602632\\
6110.60000000004	75.4005269445399\\
6111.29999999998	75.3984357103119\\
6112.4999999999	75.4016294211956\\
6113.60000000002	75.3962412287158\\
6114.90000000006	75.3932951757972\\
6115.60000000007	75.3961116470723\\
6116.19999999998	75.3936203260141\\
6116.69999999993	75.3907271776092\\
6117.29999999993	75.3931408768431\\
6118.6	75.396734600487\\
6119.99999999993	75.3990947860329\\
6121.49999999995	75.4018557239937\\
6121.70000000008	75.3993923355876\\
6122.8000000001	75.4021787061686\\
6123.00000000002	75.3997158302167\\
6123.50000000003	75.4017244757985\\
6123.90000000003	75.3984323971261\\
6125.19999999998	75.3954908331021\\
6125.49999999994	75.3934308475904\\
6126.29999999999	75.3966440323844\\
6127.80000000005	75.3928752101046\\
6128.89999999999	75.3956599771578\\
6129.09999999997	75.3931997650591\\
6130.40000000001	75.3951553706875\\
6130.69999999997	75.3930971488223\\
6132.90000000009	75.3970324474156\\
6133.59999999991	75.3933188776758\\
6134.3	75.3912363067293\\
6134.8	75.393241943634\\
6135.09999999991	75.3895553527187\\
6136.19999999997	75.3923374020175\\
6137.69999999993	75.3950927042263\\
6137.99999999994	75.393036933253\\
6138.59999999996	75.3905550035024\\
6139.30000000002	75.3933609147474\\
6139.89999999995	75.3908794788274\\
6140.90000000003	75.3932584269663\\
6142.20000000008	75.3903260993439\\
6143.20000000001	75.3943320365276\\
6143.90000000002	75.3922526041667\\
6145.59999999993	75.3892965813496\\
6146.30000000004	75.3872185344267\\
6146.90000000009	75.3896209533106\\
6148.09999999991	75.392797892066\\
6148.50000000008	75.3895195654295\\
6149.69999999991	75.3878174899997\\
6152.20000000005	75.3945678851811\\
6152.39999999993	75.3921170255993\\
6154.09999999993	75.3956647492769\\
6154.29999999992	75.393214610685\\
6155.00000000003	75.3895143864438\\
6155.60000000004	75.3870396543041\\
6157.30000000009	75.3905869360444\\
6157.89999999997	75.388113023709\\
6161.6000000001	75.3947774153237\\
6161.8000000001	75.3923302877359\\
6163.00000000008	75.3906313381253\\
6164.19999999993	75.3856885615561\\
6164.89999999997	75.3836171938362\\
6165.40000000004	75.3856134944449\\
6165.69999999993	75.3835674202861\\
6167.89999999994	75.3891050583658\\
6170.70000000007	75.3954106436767\\
6171.39999999991	75.393340354857\\
6172.60000000006	75.391643851151\\
6173.40000000004	75.3948327528954\\
6173.70000000006	75.39278888205\\
6174.79999999999	75.3955529644205\\
6177.60000000002	75.4018485844246\\
6177.89999999994	75.3998057623826\\
6178.39999999999	75.4017965525613\\
6178.60000000004	75.3993558515545\\
6179.10000000002	75.4013464526152\\
6179.40000000003	75.3993041508213\\
6180.09999999991	75.4020905472315\\
6180.99999999991	75.404054294543\\
6182.19999999998	75.3958882616502\\
6183.19999999992	75.3933983471609\\
6184.09999999995	75.3953623750849\\
6184.49999999993	75.3921029654303\\
6184.99999999995	75.3892418877625\\
6186.60000000001	75.387524851698\\
6187.60000000009	75.3898863875107\\
6189.70000000001	75.3869268797053\\
6190.20000000002	75.3889149152707\\
6191.20000000002	75.3912748534234\\
6192.09999999993	75.3883918478085\\
6193.90000000008	75.384242815628\\
6194.50000000007	75.3866270622801\\
6197.20000000007	75.3941232472206\\
6198.50000000006	75.3960571742006\\
6199.50000000001	75.3935737789535\\
6200.00000000008	75.3955581361591\\
6200.29999999993	75.391910199342\\
6200.79999999992	75.3938944346788\\
6201.60000000008	75.3906187013238\\
6202.90000000002	75.3925519909721\\
6203.69999999998	75.3892775395725\\
6204.29999999995	75.3916575333634\\
6206.39999999998	75.3935390316604\\
6208.39999999991	75.3918015623742\\
6209.10000000002	75.3865232236037\\
6210.09999999995	75.3840456023961\\
6210.69999999995	75.3864236491273\\
6211.89999999991	75.3895685769478\\
6212.40000000001	75.3851106639839\\
6213.00000000003	75.3874877275434\\
6213.90000000011	75.3894431927905\\
6215.30000000003	75.3853332046208\\
6217.49999999991	75.3876093669583\\
6217.70000000001	75.3851844703915\\
6218.60000000007	75.3871387910657\\
6219.00000000003	75.3822900419675\\
6219.70000000005	75.3786295379273\\
6220.80000000002	75.3765532318475\\
6221.89999999997	75.3744776599164\\
6222.69999999994	75.3776435045317\\
6222.99999999993	75.3756166540792\\
6223.9	75.3775706940874\\
6225.2	75.379499783143\\
6226.59999999994	75.3770054764161\\
6227.09999999999	75.3789825282631\\
6229.29999999991	75.3764407487077\\
6229.60000000002	75.3744161035042\\
6230.49999999995	75.3763682470388\\
6230.79999999999	75.3743439952495\\
6234.00000000004	75.3773600038498\\
6234.20000000004	75.3749418539371\\
6235.29999999996	75.3712672803669\\
6236.00000000002	75.3740318468273\\
6236.19999999994	75.3716145791575\\
6237.5	75.3751442862639\\
6238.20000000008	75.3730984402802\\
6239	75.3762561907968\\
6239.20000000001	75.3738400141041\\
6240.2	75.3761838373155\\
6242.10000000007	75.3740668354106\\
6243.30000000006	75.3788000128135\\
6244.10000000009	75.3755485090164\\
6245.10000000008	75.3778902196887\\
6245.69999999991	75.3754523039483\\
6246.40000000006	75.3782117986072\\
6247.10000000004	75.3745678063773\\
6247.80000000004	75.3725251684566\\
6248.79999999991	75.3700651314631\\
6250.59999999993	75.3723582958709\\
6250.79999999992	75.3699467276712\\
6252.9999999999	75.3722153811709\\
6253.50000000008	75.3693872329538\\
6254.69999999998	75.3661188207457\\
6255.20000000005	75.3680878614934\\
6255.50000000005	75.3644734318051\\
6256.20000000001	75.3624346658568\\
6257.99999999999	75.3647273134018\\
6258.19999999994	75.3623188405797\\
6258.60000000001	75.3591001326154\\
6259.49999999998	75.3610454342131\\
6260.90000000005	75.3585689187031\\
6262.40000000006	75.3644710578842\\
6262.79999999994	75.3612543709783\\
6263.69999999996	75.3647945336697\\
6264.2	75.3603754609454\\
6266.00000000002	75.3626657729688\\
6266.60000000004	75.3602374455455\\
6268.29999999996	75.3621338778636\\
6270.4	75.3671955984371\\
6271.30000000004	75.3691360780687\\
6271.49999999993	75.3667325722304\\
6272.19999999998	75.3694816893325\\
6273.1000000001	75.3714212841931\\
6273.80000000003	75.369387462344\\
6275.29999999998	75.3657137393632\\
6275.90000000007	75.368068833652\\
6278.09999999996	75.3719218884394\\
6278.80000000005	75.3698896303493\\
6281.4999999999	75.3677407030056\\
6282.59999999991	75.3656867270441\\
6283.79999999994	75.3687996308025\\
6286.30000000001	75.3706413845762\\
6288.5000000001	75.3681264510384\\
6288.99999999997	75.3700847498052\\
6289.59999999992	75.3676645944958\\
6290.89999999994	75.3648068669528\\
6291.2	75.3628026004164\\
6293.29999999997	75.3646677471637\\
6294.39999999998	75.36261815871\\
6294.70000000001	75.3606151108852\\
6296.09999999999	75.3645055747911\\
6297.50000000007	75.366806402439\\
6298.79999999993	75.370302751274\\
6301.40000000011	75.3757041974133\\
6302.49999999992	75.3784152571954\\
6302.7	75.3760233546995\\
6304.10000000004	75.3814917039434\\
6304.39999999995	75.3779046712666\\
6305.50000000005	75.375856381629\\
6307.00000000002	75.3785416435446\\
6307.2000000001	75.3761514435654\\
6307.59999999997	75.3729568622477\\
6308.20000000003	75.3752992089786\\
6308.70000000008	75.372495561755\\
6310.49999999992	75.3763509016575\\
6310.70000000008	75.3739620967231\\
6311.79999999994	75.3782537746162\\
6313.80000000004	75.3765501512536\\
6314.30000000006	75.3784999366527\\
6317.20000000006	75.3818878318269\\
6317.39999999997	75.3795013850416\\
6319.20000000001	75.377019606602\\
6320.00000000006	75.380136390247\\
6320.89999999999	75.3820598006645\\
6321.19999999995	75.380064227295\\
6322.10000000006	75.3819872829078\\
6322.39999999995	75.3799920917359\\
6323.29999999993	75.377170509536\\
6326.50000000005	75.380140992002\\
6326.80000000008	75.3781472759171\\
6327.70000000008	75.3800689023041\\
6328.40000000006	75.3780516710121\\
6331.90000000003	75.3806064434618\\
6333.89999999994	75.3789074834228\\
6334.69999999993	75.3820167961104\\
6334.90000000002	75.379636937648\\
6335.60000000002	75.3776220464984\\
6336.20000000002	75.3799536006818\\
6338.5	75.3825765942006\\
6338.90000000002	75.3793973812904\\
6340.59999999992	75.3828441654707\\
6340.90000000007	75.3808547547705\\
6341.79999999995	75.382771724562\\
6342.19999999991	75.3795941535405\\
6343.50000000006	75.3846396367993\\
6343.90000000006	75.3798865069357\\
6344.39999999999	75.3818267791\\
6344.60000000003	75.3794505650385\\
6345.89999999991	75.3829183737788\\
6346.09999999996	75.3805426869623\\
6346.89999999996	75.3836458169214\\
6348.00000000001	75.38633606906\\
6348.29999999996	75.3843488122992\\
6349.50000000006	75.381126370165\\
6350.6	75.3790920685909\\
6351.80000000011	75.3821691147531\\
6353.10000000002	75.379336397406\\
6354.09999999997	75.3769160555223\\
6355.40000000001	75.3788057587916\\
6355.90000000006	75.3728760226558\\
6358.3000000001	75.371162556618\\
6359.59999999997	75.3761969904241\\
6359.90000000009	75.374213836478\\
6360.79999999996	75.3761260199028\\
6360.99999999994	75.3737561113644\\
6361.7	75.3701782514383\\
6362.30000000004	75.3725009430404\\
6362.59999999998	75.3705188049099\\
6363.50000000009	75.372430699604\\
6365.40000000002	75.3703558243657\\
6367.10000000008	75.3737906772208\\
6369.30000000002	75.3760165792696\\
6369.99999999995	75.3724431327609\\
6370.50000000002	75.3743760399334\\
6371.10000000007	75.3719864389754\\
6372.70000000002	75.3750313833794\\
6373.0000000001	75.3730523607036\\
6374.09999999996	75.3710269524019\\
6374.50000000006	75.3678662190569\\
6378.89999999991	75.3707477661075\\
6379.10000000009	75.3683847504389\\
6379.89999999993	75.371473354232\\
6380.60000000011	75.3694735687307\\
6382.00000000002	75.3654753137682\\
6382.7000000001	75.368176975622\\
6385.69999999999	75.3703529706536\\
6386.40000000001	75.3667893212245\\
6387.09999999998	75.3694889779559\\
6387.30000000006	75.3671290352882\\
6387.89999999992	75.3631809643081\\
6389.39999999992	75.3595743015885\\
6391.19999999992	75.3618199740272\\
6391.70000000002	75.357489283144\\
6392.39999999998	75.3539303871725\\
6393.0999999999	75.3519364324595\\
6394.50000000008	75.349513652144\\
6395.59999999993	75.3474991009585\\
6396.00000000008	75.3443504635637\\
6397.89999999997	75.3422944670209\\
6399.80000000001	75.3402396912452\\
6401.59999999997	75.3424871518503\\
6405.29999999997	75.3442407968277\\
6405.6	75.342273287853\\
6406.59999999996	75.3398785646277\\
6407.29999999999	75.3378905640353\\
6407.89999999997	75.3401997503121\\
6408.19999999998	75.3366727525241\\
6408.89999999991	75.3346855983773\\
6413.0000000001	75.3364207637492\\
6413.30000000009	75.334455982786\\
6413.99999999995	75.3324706505979\\
6415.80000000006	75.3347153166352\\
6415.99999999995	75.3323670142298\\
6416.30000000001	75.3304033414376\\
6417.00000000009	75.3284193794705\\
6417.59999999994	75.3307259610141\\
6418.20000000003	75.3283579764112\\
6418.90000000009	75.326374824739\\
6420.4999999999	75.3294084664985\\
6422.59999999998	75.332803960951\\
6424.19999999996	75.3358342543156\\
6424.39999999999	75.3334889874698\\
6425.40000000002	75.331102637927\\
6426.2999999999	75.3330013693514\\
6428.00000000001	75.3301908806646\\
6429.89999999992	75.3281493001555\\
6430.70000000005	75.3312185109162\\
6431.79999999994	75.3292184269034\\
6432.40000000011	75.3268558103381\\
6434.9000000001	75.3302253302253\\
6435.09999999996	75.3278841372451\\
6436.50000000006	75.3316968585899\\
6439.40000000004	75.3350415404923\\
6441.90000000002	75.3399565352375\\
6442.20000000002	75.3380004035826\\
6444	75.3433373163048\\
6445.09999999991	75.3459939179544\\
6445.59999999993	75.343252090541\\
6447.7000000001	75.3404261918794\\
6448.69999999994	75.3442500930406\\
6448.90000000002	75.3419134749574\\
6450.29999999996	75.3441647029642\\
6452.60000000008	75.3421048553319\\
6453.19999999997	75.3381990609456\\
6454.19999999997	75.3420200486497\\
6454.59999999997	75.3373510775094\\
6456.4	75.3411290947108\\
6458.09999999991	75.3445232417702\\
6459.10000000002	75.342147634382\\
6460.60000000009	75.344776881762\\
6462.69999999994	75.3496936312434\\
6462.89999999996	75.3473619062355\\
6463.50000000006	75.3496503496504\\
6463.90000000002	75.3465346534653\\
6464.40000000009	75.3438007579859\\
6465.50000000002	75.3464488987874\\
6467.19999999994	75.3513831119633\\
6467.39999999998	75.3490529570931\\
6468.10000000009	75.3470826505055\\
6468.69999999996	75.3493692802374\\
6471.79999999998	75.3519059317975\\
6472.49999999994	75.3483916818589\\
6474.00000000008	75.3525586568017\\
6474.30000000004	75.3506116396886\\
6475.39999999995	75.3486217280519\\
6477.69999999999	75.3465682793541\\
6478.19999999991	75.3484710495037\\
6479.10000000001	75.3503518952957\\
6480.50000000004	75.347961608493\\
6482.29999999991	75.3532642231272\\
6483.19999999997	75.350515940956\\
6484.80000000008	75.353513546855\\
6489.00000000002	75.3555963076544\\
6491.30000000001	75.3627876883261\\
6491.49999999995	75.3604658327685\\
6492.59999999995	75.3584795231568\\
6494.20000000008	75.3614708282648\\
6494.5	75.3595294552397\\
6495.09999999993	75.3618056410888\\
6495.3000000001	75.3594851741232\\
6495.89999999992	75.3571428571429\\
6496.99999999997	75.3551584553108\\
6497.49999999992	75.3570549125831\\
6499.00000000001	75.355049160653\\
6501.89999999995	75.356813288219\\
6502.19999999992	75.3548744290482\\
6503.29999999991	75.3528923332411\\
6504.7999999999	75.3508893295823\\
6505.30000000008	75.3466351031451\\
6507.60000000006	75.3445917912627\\
6508.49999999994	75.3403189626033\\
6509.99999999998	75.342928680051\\
6510.39999999995	75.3382996697642\\
6513.09999999997	75.3408462813978\\
6514.79999999993	75.3426760195859\\
6516.9	75.3444836581249\\
6517.29999999998	75.3413938073465\\
6517.60000000006	75.3394602390414\\
6518.20000000004	75.3417302057285\\
6518.59999999992	75.3386411401046\\
6519.49999999993	75.3420455242653\\
6520.39999999993	75.3439153439153\\
6521.10000000005	75.3404281420597\\
6522.10000000002	75.3380761092883\\
6525.29999999998	75.3363778465688\\
6528.09999999994	75.3392972029043\\
6528.70000000001	75.3369685087612\\
6530.29999999995	75.3414798480951\\
6531.49999999996	75.3444791475289\\
6532.19999999999	75.3425286652481\\
6533.19999999993	75.3463027872591\\
6533.39999999997	75.3439963266243\\
6533.89999999991	75.3412917049281\\
6534.49999999992	75.3435558412145\\
6535.89999999998	75.3411872705018\\
6536.60000000008	75.339238453654\\
6538.19999999998	75.3360965388557\\
6539.40000000011	75.3390932028443\\
6539.79999999993	75.3360143121455\\
6540.10000000001	75.3340876425797\\
6541.20000000007	75.3367067708254\\
6542.2000000001	75.3389480763646\\
6542.59999999995	75.3358705121739\\
6543.30000000008	75.3385090320017\\
6544.2999999999	75.3361652710715\\
6546.00000000001	75.33798750401\\
6547.0999999999	75.3360215053764\\
6547.9	75.3390348197923\\
6548.19999999991	75.3371103950644\\
6549.00000000006	75.3401230703455\\
6549.3000000001	75.3366720615629\\
6550.20000000003	75.3400607605759\\
6552.89999999996	75.3425911796124\\
6553.09999999997	75.3402917658548\\
6553.60000000001	75.342173123579\\
6553.80000000009	75.3398739681716\\
6554.69999999996	75.3417343015805\\
6555.29999999993	75.3394148335723\\
6556.70000000002	75.3370546608102\\
6557.40000000009	75.3335874952345\\
6557.9000000001	75.3354681305276\\
6558.10000000006	75.3331706870788\\
6558.9	75.3300808050008\\
6559.40000000006	75.3319612775364\\
6560.69999999998	75.3338007560054\\
6561.00000000002	75.3318803249455\\
6563.99999999991	75.3340138023491\\
6564.19999999999	75.3317185381534\\
6564.70000000004	75.3335973677797\\
6565.40000000006	75.3316579087655\\
6565.9000000001	75.3335363996345\\
6567.19999999996	75.3353737456793\\
6567.40000000001	75.3330795584317\\
6570.10000000005	75.3310401509847\\
6570.69999999994	75.3287270956352\\
6572.00000000008	75.3336072183929\\
6575.10000000002	75.3361114490814\\
6575.3000000001	75.3338199957417\\
6575.9	75.3360705596107\\
6577.90000000008	75.3314077227121\\
6579.60000000008	75.333221879417\\
6580.60000000005	75.3354506359506\\
6583.4999999999	75.3326447536302\\
6583.80000000008	75.3307310256839\\
6584.29999999998	75.3326043375251\\
6586.30000000001	75.3355399004008\\
6587.89999999997	75.3384942319369\\
6588.10000000002	75.3362071582526\\
6589.20000000001	75.3403244654212\\
6590.10000000001	75.3421747443173\\
6591.49999999996	75.3398264457795\\
6592.09999999997	75.3375200995116\\
6593.29999999995	75.3404920071587\\
6594.09999999996	75.3374177307331\\
6596.20000000005	75.3392053120689\\
6596.40000000011	75.3369210945198\\
6597.90000000002	75.3349499848439\\
6598.49999999991	75.3371927378535\\
6598.80000000007	75.3337677491703\\
6599.59999999992	75.336757731412\\
6599.7999999999	75.3344747647692\\
6600.3999999999	75.3367169153852\\
6600.59999999991	75.3344342266729\\
6602.29999999996	75.3392705682782\\
6603.60000000003	75.3410966579342\\
6603.89999999998	75.3391883706844\\
6606.89999999997	75.3458453155744\\
6607.99999999996	75.3423828331896\\
6609.1000000001	75.3464867154875\\
6609.49999999991	75.3434398450738\\
6611.09999999991	75.3494070667957\\
6611.39999999996	75.3475005671935\\
6611.90000000007	75.3448275862069\\
};
\addplot [color=mycolor2, forget plot]
  table[row sep=crcr]{%
6611.90000000007	75.3448275862069\\
6612.1999999999	75.3429215250367\\
6612.7000000001	75.3402492136463\\
6614.89999999998	75.3439153439153\\
6615.20000000002	75.3420101885024\\
6616.4	75.3389254137384\\
6618.10000000008	75.3422380707745\\
6620.80000000006	75.3477623887991\\
6621.00000000011	75.3454863995409\\
6622.4	75.3506983767459\\
6623.19999999999	75.3476363746169\\
6625.09999999992	75.3456499426432\\
6627.00000000002	75.3481915166513\\
6628.19999999993	75.3511458443341\\
6628.4999999999	75.3477355701053\\
6629.10000000008	75.3454413805587\\
6629.69999999998	75.3431476062626\\
6632.79999999993	75.3411026851\\
6634.89999999994	75.3458929917106\\
6636.70000000006	75.3480593056895\\
6636.99999999991	75.3461602205783\\
6638.00000000007	75.3498742109941\\
6638.19999999992	75.3476040552551\\
6639.2	75.3452924254063\\
6639.70000000001	75.3411247326727\\
6640.19999999991	75.3429814918001\\
6640.49999999993	75.3410836370208\\
6641.00000000009	75.338422851636\\
6641.8000000001	75.3413932760204\\
6642.09999999998	75.3379904248592\\
6642.69999999993	75.3357018124887\\
6644.20000000007	75.3412699607182\\
6645.29999999994	75.3438468715201\\
6647.19999999994	75.3478856076903\\
6647.89999999994	75.3459687123947\\
6649.80000000001	75.350005263237\\
6650.40000000004	75.3477182166754\\
6651.39999999998	75.3514244907164\\
6651.70000000004	75.348026098199\\
6652.59999999995	75.3498579524103\\
6652.79999999996	75.3475927790888\\
6654.89999999992	75.3448534936138\\
6655.7999999999	75.3466848961072\\
6656.19999999999	75.3421570542193\\
6657.49999999991	75.3439677962028\\
6658.40000000011	75.3412930840279\\
6659.09999999994	75.3393801057184\\
6659.49999999998	75.3363565379302\\
6662.50000000008	75.3444601206736\\
6664.99999999991	75.3462063584942\\
6665.30000000003	75.3428151348756\\
6666.30000000001	75.3405136205448\\
6667.60000000004	75.3438217076353\\
6668.70000000007	75.3418905950096\\
6670.40000000009	75.3466756614946\\
6671.4000000001	75.3443753278873\\
6671.9	75.3462230215827\\
6672.09999999999	75.3439645094572\\
6673.10000000009	75.3416651681352\\
6673.59999999999	75.3435125942131\\
6673.79999999992	75.3412547386086\\
6675.99999999996	75.3344018214227\\
6678.10000000004	75.3361684286185\\
6679.79999999995	75.3379541609904\\
6680.2000000001	75.3334431088424\\
6682.60000000001	75.3408053631017\\
6683.69999999995	75.3433675454083\\
6687.60000000003	75.3502698984703\\
6689.49999999992	75.3527864147333\\
6691.99999999999	75.3500395989301\\
6692.50000000001	75.3473986193707\\
6692.99999999996	75.3492402623597\\
6693.30000000007	75.3473570980369\\
6693.79999999991	75.3491985240293\\
6696.49999999998	75.3456978168026\\
6697.3000000001	75.3486427568907\\
6697.50000000006	75.3463927376971\\
6698.79999999997	75.3511770589201\\
6700.00000000006	75.3540991925494\\
6700.30000000007	75.3522177780431\\
6701.79999999998	75.3502738029514\\
6703.7999999999	75.3456942973493\\
6705.90000000009	75.347450044736\\
6707.80000000002	75.3514512738711\\
6707.99999999999	75.349204692834\\
6709.79999999997	75.3528368530082\\
6713.8999999999	75.3500148942508\\
6714.40000000004	75.3518504728573\\
6714.59999999998	75.349606088135\\
6715.09999999995	75.3514415058375\\
6716.50000000009	75.3536015245809\\
6717.7999999999	75.3553937986573\\
6718.0999999999	75.3535173111845\\
6719.3	75.3579188618031\\
6719.50000000002	75.3556759330913\\
6722.50000000001	75.3532859310386\\
6723.2000000001	75.351389942439\\
6725.29999999999	75.3486781455378\\
6726.39999999998	75.3527094328403\\
6730.40000000011	75.3569571354283\\
6731.90000000007	75.3535353535353\\
6732.50000000008	75.3557318123756\\
6732.79999999997	75.3538594067935\\
6734.10000000006	75.3511924207775\\
6734.99999999995	75.3544861991656\\
6735.30000000008	75.3511298512338\\
6735.90000000008	75.353325415677\\
6736.99999999996	75.3558652832821\\
6737.40000000011	75.3528756957328\\
6741.19999999995	75.3504516932936\\
6741.70000000008	75.3522798065798\\
6741.99999999992	75.3504101096098\\
6743.09999999995	75.3484992288528\\
6744.09999999998	75.3462234216067\\
6745.20000000005	75.3487613597616\\
6746.60000000001	75.3509122978641\\
6748.10000000002	75.3489819507424\\
6750.00000000001	75.3470318958238\\
6754.30000000001	75.3494018713727\\
6754.70000000001	75.3464203233256\\
6755.60000000007	75.3437837677813\\
6757.1999999999	75.3407426042946\\
6759.10000000004	75.3461948159545\\
6759.8	75.3443098270684\\
6760.40000000006	75.3420604984838\\
6763.49999999993	75.3444910994145\\
6764.70000000011	75.339995269631\\
6765.89999999999	75.3369790127106\\
6766.89999999992	75.3406236146003\\
6767.1	75.3383969736375\\
6767.99999999991	75.3357663155095\\
6769.69999999997	75.3390055836214\\
6769.9999999999	75.3371442076188\\
6770.6	75.333421950463\\
6771.60000000002	75.3355878139906\\
6771.80000000003	75.3333628671422\\
6772.39999999995	75.3311184939092\\
6774.50000000004	75.3343370826322\\
6774.89999999997	75.3313653136531\\
6775.70000000008	75.328374509283\\
6776.50000000011	75.3312870761149\\
6778.69999999997	75.3333923408273\\
6779.40000000011	75.3315141234604\\
6780.20000000011	75.3270504254974\\
6783.60000000007	75.3305718118431\\
6784.7000000001	75.3286758636953\\
6785.29999999998	75.3308574291862\\
6785.79999999996	75.3282541740963\\
6788.59999999991	75.3310648577784\\
6789.80000000009	75.3280607961826\\
6790.50000000006	75.3306040703325\\
6792.79999999992	75.3345404760853\\
6794.70000000007	75.339965856243\\
6795.8	75.3380714842773\\
6796.80000000007	75.3416998925981\\
6799.19999999989	75.34452075949\\
6799.69999999998	75.34192182123\\
6800.19999999998	75.3437348352279\\
6801.30000000009	75.3462522421854\\
6802.60000000007	75.3480235788731\\
6804.69999999995	75.3497531154479\\
6804.89999999995	75.3475385745775\\
6806.00000000005	75.3441765475089\\
6806.99999999998	75.3419224045482\\
6807.30000000008	75.340071099098\\
6808.00000000001	75.3382000851926\\
6808.69999999998	75.3363294559981\\
6809.1999999999	75.333734745128\\
6810.99999999997	75.3358488349899\\
6811.50000000005	75.3332550355276\\
6812.40000000002	75.3306422018349\\
6814.99999999997	75.3341843846752\\
6816.3000000001	75.3359544627663\\
6817.00000000008	75.3340863416995\\
6817.50000000004	75.3358953297348\\
6817.80000000007	75.3340471406151\\
6819.2	75.3361782001085\\
6819.39999999996	75.3339687660386\\
6820.70000000002	75.3357377433732\\
6821.60000000011	75.3316621956404\\
6822.90000000001	75.3275685182471\\
6823.40000000009	75.3293764197259\\
6823.70000000009	75.3260646560567\\
6824.69999999997	75.3282147462197\\
6825.30000000004	75.3259882204706\\
6826.69999999991	75.3281185914338\\
6826.89999999997	75.3259118207119\\
6827.39999999993	75.3218601244965\\
6828.19999999991	75.3247514022524\\
6830.89999999991	75.3227931488801\\
6832.10000000008	75.3198091390767\\
6832.70000000006	75.3219763493736\\
6832.99999999989	75.3201328825862\\
6834.89999999996	75.3182150694952\\
6836.29999999993	75.3218067989\\
6837.89999999996	75.3188066686166\\
6838.69999999996	75.3158448850675\\
6840.40000000006	75.3175937431474\\
6841.69999999998	75.3193604022333\\
6842.29999999992	75.3171401847305\\
6843.19999999996	75.3189250800053\\
6844.59999999991	75.3166683711485\\
6846.39999999993	75.3143942160228\\
6847.20000000001	75.3172783432886\\
6849.90000000011	75.3153284671533\\
6850.50000000006	75.3131112603276\\
6851.99999999992	75.3185154916011\\
6853.90000000008	75.3166034432448\\
6854.29999999989	75.3136671335201\\
6854.60000000007	75.3118298393803\\
6855.30000000011	75.3099746185489\\
6859.20000000004	75.3079760325398\\
6859.70000000004	75.3054024898685\\
6860.19999999993	75.3072023089369\\
6862.1000000001	75.305295677771\\
6864.00000000001	75.3077606678224\\
6864.59999999991	75.3055486765627\\
6865.19999999997	75.3077068736982\\
6867.30000000008	75.3094329731776\\
6867.49999999998	75.3072397926495\\
6869.00000000002	75.3053529574471\\
6869.80000000004	75.3082286496164\\
6870.40000000009	75.303107488538\\
6870.8999999999	75.304904671809\\
6871.90000000005	75.3070430733411\\
6872.2000000001	75.3052107736857\\
6872.90000000002	75.303360977739\\
6874.2999999999	75.3011171884092\\
6874.59999999992	75.2992857870162\\
6876.39999999999	75.3042972442376\\
6876.7999999999	75.2999171138158\\
6877.39999999991	75.2977099236641\\
6879.90000000002	75.2994186046512\\
6880.09999999991	75.2972297316938\\
6881.10000000003	75.2950066848805\\
6881.40000000001	75.2931773595873\\
6882.10000000011	75.29569033158\\
6883.60000000009	75.2938100149629\\
6884.20000000002	75.2959632787647\\
6885.60000000011	75.2995338164602\\
6887.59999999992	75.3023505669527\\
6888.59999999991	75.3044841552107\\
6892.00000000007	75.3065103524325\\
6892.29999999997	75.3046834194185\\
6892.99999999998	75.3013883448666\\
6894.40000000007	75.2991514975705\\
6897.00000000007	75.3012135535225\\
6897.29999999991	75.2993881752544\\
6898.3	75.3029688043604\\
6899.29999999994	75.3007507899238\\
6900.19999999993	75.3025230787067\\
6903.19999999999	75.3074616487767\\
6903.5999999999	75.3030983385721\\
6904.5	75.304869217623\\
6904.80000000005	75.3030456632247\\
6908.90000000005	75.3061224489796\\
6909.19999999997	75.3043000014473\\
6909.80000000009	75.3021027800692\\
6911.30000000002	75.3045692623781\\
6912.20000000012	75.306337977229\\
6912.49999999995	75.3030697566762\\
6913.19999999989	75.3012309606122\\
6913.90000000006	75.3037315591553\\
6915.2000000001	75.3069281159169\\
6915.70000000003	75.304375488013\\
6916.10000000004	75.3014661230155\\
6917.49999999999	75.3050190817625\\
6922.20000000011	75.3030062262543\\
6924.9	75.301083032491\\
6926.90000000009	75.3038833549877\\
6927.20000000006	75.3020657398987\\
6927.99999999989	75.3049176541909\\
6929.00000000003	75.3070384321196\\
6929.20000000006	75.3048648492633\\
6931.39999999989	75.3083748106471\\
6932.3	75.3101379031793\\
6932.60000000005	75.3083214332078\\
6933.4999999999	75.3057574708665\\
6935.40000000002	75.3038713863456\\
6938.5000000001	75.3062577465195\\
6938.79999999992	75.304443067345\\
6941.19999999995	75.312981718122\\
6941.50000000001	75.3111674541892\\
6942.49999999997	75.3147235905857\\
6942.80000000006	75.3129095910931\\
6943.70000000004	75.3146692013019\\
6944.20000000004	75.3092464323258\\
6945.3000000001	75.3131569096092\\
6947.10000000001	75.3109166282819\\
6951.59999999992	75.3081979947351\\
6952.79999999992	75.3110213004646\\
6953.29999999997	75.3084821813789\\
6954.2999999999	75.3105947313931\\
6954.79999999992	75.3080561905994\\
6956.30000000005	75.306192858375\\
6958.19999999989	75.3043128350316\\
6959.30000000001	75.3024686036153\\
6960.8000000001	75.3006076800414\\
6963.49999999997	75.2986960767419\\
6964.20000000009	75.3011788693767\\
6964.69999999993	75.2972088215024\\
6965.19999999995	75.2946750319441\\
6966.69999999993	75.2928173623471\\
6968.00000000001	75.2945566223217\\
6968.3	75.2927501291545\\
6968.59999999996	75.290943791525\\
6970.59999999999	75.2951640437833\\
6971.09999999997	75.2926325453294\\
6972.29999999998	75.2897137284149\\
6973.10000000002	75.2925486146963\\
6973.30000000006	75.29038919322\\
6975.0000000001	75.2921105073762\\
6975.1999999999	75.2899516866658\\
6976.20000000011	75.2877599873859\\
6977.09999999994	75.2895144183913\\
6977.40000000002	75.286277319957\\
6978.90000000002	75.2887233127955\\
6979.49999999996	75.2865493724569\\
6980.9000000001	75.2843432173041\\
6985.99999999989	75.2823463735131\\
6987.10000000003	75.2848065033204\\
6988.29999999995	75.2876194837159\\
6988.5999999999	75.2858185356361\\
6990.19999999993	75.2914753301003\\
6990.4	75.289321221658\\
6990.70000000006	75.287520741546\\
6991.70000000003	75.2896249892732\\
6991.99999999995	75.2878248308806\\
6992.59999999996	75.2899452285955\\
6993.59999999999	75.2920485579879\\
6994.70000000004	75.294504489049\\
6996.5999999999	75.2926379578944\\
6997.50000000007	75.2943866468504\\
6998.89999999991	75.2921845977997\\
6999.59999999995	75.2946554852351\\
6999.90000000001	75.2914285714286\\
7000.4000000001	75.2931933433326\\
7000.69999999991	75.2913952691121\\
7001.69999999999	75.289211345654\\
7002.70000000008	75.2870280459245\\
7004.10000000002	75.2905399617373\\
7005.50000000008	75.2940504739066\\
7007.90000000008	75.296803652968\\
7008.10000000006	75.29465483291\\
7008.69999999997	75.2967697751398\\
7010.39999999993	75.2984808501533\\
7010.70000000003	75.2952587436527\\
7011.70000000008	75.2973558857925\\
7011.9	75.2952082144894\\
7012.49999999998	75.2973219633232\\
7012.80000000008	75.2955268148697\\
7013.89999999991	75.2994011976048\\
7014.20000000003	75.2976063185207\\
7014.4999999999	75.2958115929632\\
7015.59999999992	75.2996849922317\\
7017.00000000003	75.2974875660885\\
7017.59999999996	75.2995995839093\\
7018.7	75.3020459337778\\
7019.00000000009	75.3002521690815\\
7019.79999999996	75.3030669952563\\
7020.59999999992	75.2959106641788\\
7022.3999999999	75.3008187967248\\
7022.80000000002	75.2979538367341\\
7024.10000000011	75.299678255175\\
7025.9	75.3031596925705\\
7026.20000000002	75.3013677184293\\
7026.79999999999	75.30347663977\\
7027	75.30133340923\\
7027.69999999993	75.3037935057913\\
7027.89999999992	75.3016505406944\\
7028.80000000004	75.3033902886654\\
7029.89999999991	75.3001422475107\\
7031.29999999998	75.3036379668345\\
7031.99999999999	75.3018301787517\\
7032.7000000001	75.3042884768513\\
7033.30000000009	75.3021298376319\\
7034.59999999998	75.3038509104866\\
7038.3999999999	75.3015557292037\\
7039.7000000001	75.2990141765391\\
7040.39999999993	75.3014700660464\\
7040.69999999994	75.2982615611862\\
7040.99999999994	75.2964735623695\\
7043.2999999999	75.2946020387881\\
7045.1000000001	75.2980752853006\\
7046.90000000003	75.2930324960976\\
7047.19999999994	75.2912462929065\\
7047.49999999999	75.2894602417844\\
7049.9	75.2936170212766\\
7050.09999999995	75.291481092735\\
7051.69999999997	75.294251113191\\
7054.2	75.2987539514906\\
7055.10000000002	75.3004875836263\\
7055.40000000001	75.2972858054\\
7056.49999999999	75.2954680724428\\
7058.59999999997	75.2929009591001\\
7060.79999999999	75.29493407356\\
7061.10000000002	75.2917351158443\\
7062.0000000001	75.2948839580295\\
7064.59999999992	75.2968986651946\\
7066.00000000004	75.3003778604888\\
7066.4	75.2975306021368\\
7069.09999999992	75.3069654274883\\
7070.10000000002	75.3048004299737\\
7070.3999999999	75.3030195884308\\
7071.1	75.299807670551\\
7071.70000000004	75.3019033343703\\
7072.59999999995	75.303632276217\\
7074.00000000008	75.3014517747841\\
7074.49999999992	75.2989568314816\\
7075.29999999994	75.2960963337762\\
7077.90000000003	75.2995196383159\\
7079.70000000009	75.297324783186\\
7081.00000000005	75.2947988306901\\
7081.6	75.2968919892116\\
7082.29999999998	75.2922738054897\\
7082.90000000011	75.2901313002965\\
7084.10000000005	75.2929053386409\\
7084.29999999999	75.2907797414036\\
7086.00000000004	75.2938852118937\\
7088.20000000009	75.295910161816\\
7089.09999999992	75.2976358404333\\
7089.4	75.2958600747585\\
7091.99999999994	75.2978666403463\\
7092.29999999995	75.2946816310417\\
7093.20000000009	75.2907673438315\\
7093.90000000007	75.2932055257965\\
7096.70000000006	75.2973171006651\\
7098.09999999992	75.2951452480911\\
7098.5999999999	75.2968853452041\\
7099.10000000006	75.294399368943\\
7100.0999999999	75.2978789329878\\
7100.49999999999	75.2936371574233\\
7100.99999999997	75.2953767726127\\
7105.69999999992	75.3046806833854\\
7106.59999999991	75.3021796333038\\
7108.7000000001	75.3066621652037\\
7109.10000000002	75.3038316547572\\
7110.40000000011	75.3069404401941\\
7112.79999999998	75.3096486665073\\
7112.99999999997	75.3075311748745\\
7113.29999999992	75.3057609581916\\
7115.19999999997	75.3025171110143\\
7117.50000000005	75.3048780487805\\
7117.70000000009	75.3027620894096\\
7119.00000000004	75.3058673146887\\
7119.20000000001	75.3037517733485\\
7120.29999999992	75.3075669906185\\
7120.60000000004	75.3057985872176\\
7121.90000000009	75.3018814939624\\
7122.40000000009	75.3036153036153\\
7123.80000000003	75.3056612248909\\
7124.09999999993	75.3038937705286\\
7125.10000000007	75.3073597934093\\
7126.60000000005	75.3055411340452\\
7128.49999999992	75.3037061975703\\
7129.39999999993	75.306823760432\\
7129.69999999998	75.3050576453758\\
7130.29999999993	75.3029283069674\\
7131.19999999993	75.306045181103\\
7131.40000000001	75.3039332538737\\
7131.89999999999	75.3056646102075\\
7132.90000000007	75.3021169213515\\
7133.89999999999	75.2999719652369\\
7134.99999999999	75.298173816765\\
7135.70000000001	75.2963928361221\\
7137.49999999991	75.2984196368527\\
7137.70000000008	75.2963097873294\\
7139.69999999992	75.3004285834337\\
7143.69999999999	75.3030599960805\\
7144.00000000007	75.3012975742221\\
7147.79999999995	75.3032359154437\\
7148.09999999991	75.3014744970762\\
7149.09999999993	75.3035304649471\\
7149.4999999999	75.2993174443325\\
7150.9	75.3041532652776\\
7151.59999999997	75.3023756589342\\
7152.50000000006	75.2998909487459\\
7153.39999999998	75.297406863773\\
7154.10000000006	75.2998238796791\\
7155.49999999999	75.3018614791212\\
7156.89999999993	75.3038982814028\\
7160.89999999999	75.3065214355537\\
7165.09999999998	75.309830849104\\
7165.60000000011	75.3073670402054\\
7170.20000000007	75.3092618161025\\
7170.5	75.306111064625\\
7171.70000000006	75.3102428957863\\
7176.30000000012	75.3079538487264\\
7178.70000000008	75.305064913356\\
7182.90000000012	75.3069748016149\\
7183.20000000002	75.3038297161472\\
7184.99999999998	75.3072330238967\\
7186.10000000012	75.3054465503326\\
7186.99999999998	75.3085389099915\\
7187.5	75.304691412989\\
7188.60000000012	75.3015148775161\\
7188.89999999994	75.2997635276116\\
7190.39999999992	75.2965718656561\\
7191.4999999999	75.3003504088103\\
7191.79999999999	75.2985998136793\\
7192.59999999997	75.3013471992437\\
7193.69999999996	75.303733770747\\
7194.60000000004	75.3054331660806\\
7195.8999999999	75.3085047248471\\
7196.59999999995	75.3067378103853\\
7197.59999999989	75.3087791933534\\
7197.99999999989	75.3059835234298\\
7199.20000000002	75.310099593016\\
7200.99999999997	75.3079390648651\\
7202.09999999992	75.3061564521952\\
7202.5999999999	75.3078706596138\\
7206.3	75.3108348134991\\
7206.49999999989	75.3087447617462\\
7206.99999999992	75.3104577430589\\
7207.1999999999	75.3083679047632\\
7208.10000000009	75.3100635387475\\
7211.69999999995	75.3140686097784\\
7215.19999999991	75.3177276066137\\
7216.30000000007	75.3159470095893\\
7217.80000000004	75.3141495448815\\
7219.29999999992	75.3123528271047\\
7220.29999999993	75.3143870145698\\
7222.10000000011	75.3191548281687\\
7227.99999999991	75.3213154217568\\
7228.20000000011	75.319231354537\\
7229.10000000007	75.3223039893764\\
7229.70000000008	75.3202024952281\\
7231.20000000009	75.322556110243\\
7231.8000000001	75.3204552054094\\
7233.19999999992	75.3224669238107\\
7233.40000000001	75.3203843229419\\
7237.19999999989	75.3153800450444\\
7239.20000000002	75.3125302170099\\
7239.79999999995	75.3145761681791\\
7240.10000000001	75.3128366619707\\
7244.99999999989	75.3171108749362\\
7245.69999999998	75.3153551022661\\
7246.40000000002	75.3135996688056\\
7247.39999999998	75.3156260779579\\
7249.59999999994	75.3189787163607\\
7250.29999999998	75.3158446430542\\
7251.19999999996	75.3189083336781\\
7256.8999999999	75.3231362822103\\
7258.29999999995	75.3210073845476\\
7258.90000000004	75.3230472516876\\
7259.49999999997	75.3195768361893\\
7260.99999999991	75.316412113867\\
7261.50000000002	75.3181117109177\\
7262.20000000004	75.3163598309076\\
7262.99999999994	75.3135713400614\\
7264.49999999999	75.3172920738926\\
7265.69999999996	75.3144870489141\\
7268.09999999994	75.3185107729562\\
7268.30000000006	75.3164382807771\\
7269.09999999998	75.3136521212788\\
7269.70000000003	75.3156895650499\\
7269.89999999997	75.3136176066025\\
7270.50000000008	75.311528622122\\
7272.39999999993	75.3097284290134\\
7273.09999999997	75.3079799813012\\
7276.09999999998	75.3099145158187\\
7278.20000000001	75.3074206888971\\
7279.00000000006	75.3046393098048\\
7279.30000000007	75.3029095804599\\
7280.00000000007	75.3011634455571\\
7280.79999999992	75.3038772679202\\
7281.20000000001	75.301113812094\\
7281.89999999995	75.2993683054106\\
7282.60000000012	75.2976231342771\\
7282.9000000001	75.2958945489496\\
7283.99999999993	75.2996252110762\\
7285.09999999995	75.2978641629605\\
7286.00000000009	75.3009154417315\\
7286.30000000002	75.2991875274484\\
7287.40000000003	75.2960548885077\\
7290.09999999991	75.2997174288771\\
7290.8000000001	75.2979741870002\\
7291.99999999994	75.3020391930994\\
7293.10000000006	75.3043931333297\\
7294.99999999993	75.3026003755946\\
7296.20000000008	75.2984389348026\\
7298.09999999991	75.2966484886684\\
7299.60000000002	75.2948751318547\\
7300.29999999992	75.2931346227604\\
7301.79999999991	75.2913625220833\\
7302.60000000003	75.294069316828\\
7302.80000000002	75.2920072847773\\
7303.70000000007	75.2950518907966\\
7305.70000000005	75.297708669824\\
7306.29999999989	75.2956312274171\\
7308.40000000006	75.2986248888281\\
7309.70000000007	75.2961777340009\\
7310.40000000003	75.2985431913002\\
7310.59999999997	75.2964832368993\\
7312.2999999999	75.3008588151633\\
7315.39999999997	75.2990226231973\\
7316.20000000006	75.3017235487883\\
7318.89999999998	75.2999043585189\\
7319.50000000003	75.29783048254\\
7321.49999999995	75.2936516608392\\
7322.1999999999	75.2960135476558\\
7322.59999999998	75.2932661450012\\
7322.90000000006	75.2915471801174\\
7324.09999999994	75.288768739248\\
7324.79999999998	75.2870346352851\\
7325.29999999998	75.2846260955033\\
7325.60000000005	75.2829081179955\\
7327.60000000001	75.2855602712993\\
7327.80000000008	75.2835055063524\\
7329.30000000012	75.2858351297514\\
7331.00000000001	75.2833817571715\\
7334.29999999993	75.2863219895288\\
7334.6000000001	75.2846060506906\\
7336.39999999999	75.2824916513324\\
7337.10000000008	75.2848498064657\\
7337.30000000012	75.2827977212637\\
7338.00000000003	75.2810673062509\\
7338.59999999994	75.2830882853911\\
7339.30000000002	75.2813581491675\\
7340.10000000008	75.284052205662\\
7340.59999999998	75.2802866211669\\
7342.39999999996	75.2822608103507\\
7342.80000000006	75.2795217148538\\
7345.0000000001	75.277395814897\\
7346.10000000005	75.2756527184122\\
7347.60000000003	75.2738952325217\\
7348.30000000002	75.270807250558\\
7350.20000000009	75.2690366379604\\
7351.10000000003	75.2720644248558\\
7351.60000000007	75.2696655195397\\
7352.29999999994	75.2720200206735\\
7352.4999999999	75.2699725267252\\
7353.30000000012	75.2726629858297\\
7354.79999999992	75.276346381324\\
7355.4000000001	75.274284548977\\
7356.80000000007	75.278989791896\\
7357.0000000001	75.2769433608351\\
7360.59999999999	75.2795250451723\\
7361.39999999993	75.2767778306052\\
7361.69999999989	75.2750685973539\\
7364.00000000008	75.273285262286\\
7364.59999999995	75.2752997406547\\
7365.50000000001	75.272890192245\\
7367.69999999999	75.2748445940444\\
7367.90000000001	75.2728013029316\\
7368.9000000001	75.2761568733885\\
7370.49999999993	75.2733834423249\\
7371.09999999995	75.2713262426742\\
7374.69999999991	75.2766176709877\\
7374.90000000011	75.2745762711864\\
7376	75.2769078510324\\
7376.30000000009	75.2752019955534\\
7377.09999999989	75.2778832077211\\
7377.30000000011	75.2758424377152\\
7378.69999999995	75.2791781861549\\
7378.89999999991	75.2771378235533\\
7380.2	75.2801376637806\\
7382.39999999998	75.2820860142228\\
7382.69999999996	75.2803814271008\\
7383.6999999999	75.2837292451041\\
7383.90000000006	75.2816901408451\\
7384.59999999999	75.2840332037862\\
7388.09999999996	75.2808532524837\\
7389.20000000003	75.283179732857\\
7389.50000000003	75.2814766699145\\
7390.60000000003	75.2797434613771\\
7392.39999999997	75.2776462631045\\
7393.09999999996	75.2759292322675\\
7395	75.2795770172141\\
7395.69999999995	75.2778604072582\\
7396.00000000008	75.2761590568002\\
7397.5	75.2811722720882\\
7398.09999999992	75.2791219485821\\
7398.70000000008	75.2811266691896\\
7400.59999999996	75.279365465429\\
7402.00000000001	75.2813390794504\\
7404.30000000007	75.2863162443952\\
7407.7000000001	75.2909095817922\\
7411.89999999997	75.2927684835402\\
7415.39999999989	75.2949902231812\\
7415.70000000011	75.2932926993716\\
7417.19999999989	75.2915481374624\\
7418.49999999994	75.2958779284501\\
7422.29999999997	75.2977473593447\\
7423.2	75.2953538183827\\
7424.70000000012	75.2936106023058\\
7427.39999999999	75.2904745876809\\
7429.10000000006	75.2867065094492\\
7429.80000000003	75.2849971062868\\
7432.69999999989	75.2825314820794\\
7435.50000000007	75.2797353273441\\
7436.5	75.2817147621225\\
7436.70000000003	75.2796901893287\\
7438.10000000005	75.282998574924\\
7439.29999999991	75.2802645374627\\
7440.80000000001	75.2825599053878\\
7442.3999999999	75.2865300638226\\
7444.29999999992	75.2888076943743\\
7444.90000000008	75.2867696440564\\
7446.09999999993	75.2894093631651\\
7446.29999999992	75.2873871938118\\
7447.99999999994	75.2930277520442\\
7448.60000000012	75.2909903741593\\
7450.40000000004	75.2969599355748\\
7451.0000000001	75.2949228972903\\
7452.09999999992	75.2932020074609\\
7453.30000000002	75.2958381409826\\
7453.69999999993	75.2917974724302\\
7454.30000000002	75.2897617514488\\
7454.89999999997	75.2877263581489\\
7456.69999999997	75.2896684905053\\
7457.70000000002	75.2916409665049\\
7457.9000000001	75.2896218825422\\
7460.20000000004	75.2972400573703\\
7460.90000000011	75.2955367913148\\
7461.59999999989	75.2938338448343\\
7462.19999999992	75.291800115246\\
7463.60000000003	75.2897356538982\\
7464.29999999996	75.2920529446439\\
7465.00000000007	75.28901153367\\
7466.80000000005	75.2869329976295\\
7470.70000000001	75.2851100283771\\
7473.49999999989	75.290355384286\\
7475.70000000001	75.292276411889\\
7475.90000000012	75.2902621722846\\
7478.6	75.2938344899515\\
7479.10000000008	75.2914750240667\\
7480.50000000009	75.2960992433762\\
7480.70000000011	75.29408619399\\
7482.59999999989	75.3003594959039\\
7484.50000000012	75.3039574593165\\
7484.90000000002	75.3012692050768\\
7486.1000000001	75.3038924955251\\
7489.99999999991	75.3020653929854\\
7491.39999999999	75.3053460588667\\
7492.9000000001	75.3036167089283\\
7494.99999999995	75.3092020119812\\
7496.39999999994	75.3111452010938\\
7500	75.314995800056\\
7500.50000000003	75.3126416553342\\
7503.59999999989	75.3108466489865\\
7504.2000000001	75.3088229415135\\
7505	75.3114548773501\\
7508.80000000006	75.307967878118\\
7509.70000000002	75.3109270553144\\
7510.80000000011	75.3092172708996\\
7511.60000000008	75.3118468522438\\
7512	75.309167875827\\
7513.10000000006	75.3074588723846\\
7514.9000000001	75.3093812375249\\
7516.29999999995	75.3113192485765\\
7516.89999999995	75.3079686044965\\
7520.2000000001	75.3108253660093\\
7521.0000000001	75.3081331188257\\
7522.30000000004	75.305753482931\\
7523.70000000005	75.3037029160796\\
7526.90000000005	75.3115451042912\\
7528.80000000011	75.3098062133911\\
7530.99999999993	75.3077239712658\\
7535.70000000005	75.3098542954962\\
7537.60000000012	75.3134245194157\\
7537.79999999989	75.311426259303\\
7540.29999999998	75.3182855020954\\
7540.49999999996	75.3162878285548\\
7542.0999999999	75.3215242237013\\
7542.50000000009	75.3188555670458\\
7543.40000000004	75.3218002253596\\
7545.99999999995	75.3236771312334\\
7547.79999999991	75.3216126339777\\
7548.69999999994	75.3245548961424\\
7550.60000000003	75.3267908935595\\
7552.8999999999	75.3250364093738\\
7553.99999999994	75.3233343482347\\
7554.59999999999	75.3213231498273\\
7555.70000000012	75.319622012229\\
7556.40000000009	75.3219082908754\\
7557.00000000009	75.3185745854891\\
7557.90000000002	75.3162212225456\\
7558.79999999994	75.3138684200082\\
7560.5999999999	75.3118097530652\\
7561.30000000008	75.3140952733621\\
7562.29999999991	75.3120702422511\\
7569.40000000001	75.3101261642116\\
7570.40000000011	75.3133874909187\\
7570.99999999998	75.3113814373077\\
7572.00000000002	75.3146419091137\\
7572.9	75.3096527135878\\
7575.60000000011	75.3131723801101\\
7577.09999999996	75.3114606978831\\
7578.2	75.313724713986\\
7579.8000000001	75.3176163273922\\
7580.39999999994	75.314293252424\\
7580.89999999999	75.3119641208284\\
7581.29999999997	75.3093096261904\\
7583.20000000012	75.307583769599\\
7584.79999999994	75.30488206832\\
7585.40000000011	75.3028804956826\\
7586.29999999998	75.3005378044923\\
7589.20000000013	75.30470530879\\
7592.19999999994	75.3065605942863\\
7593.50000000012	75.3094711335862\\
7594.60000000011	75.3051470104152\\
7595.60000000009	75.3070816382953\\
7596.19999999992	75.3050827376486\\
7598.10000000008	75.3086257271459\\
7598.3000000001	75.3066435038956\\
7598.89999999993	75.3046453480721\\
7600.5000000001	75.3098439596874\\
7603.59999999989	75.3067585517577\\
7606.09999999995	75.3122452735926\\
7607.00000000005	75.3099078492461\\
7608.50000000004	75.308203874563\\
7609.50000000004	75.3101345668629\\
7609.80000000011	75.3071656657775\\
7610.30000000003	75.3048460002102\\
7610.90000000005	75.3067927998949\\
7612.59999999995	75.3096798770476\\
7612.80000000001	75.3077014015684\\
7614.39999999988	75.3128898811478\\
7614.59999999988	75.3109117890396\\
7617.1000000001	75.31376358767\\
7617.70000000012	75.3117698023051\\
7618.70000000007	75.3150102378327\\
7622.30000000008	75.3174853064652\\
7624.2999999999	75.3213367609254\\
7625.49999999997	75.3186634494335\\
7627.10000000011	75.3159744073841\\
7628.60000000003	75.314273729469\\
7630.50000000006	75.3125573349409\\
7631.60000000013	75.3161156754065\\
7631.80000000007	75.3141419567867\\
7637.50000000007	75.3116162145177\\
7638.29999999992	75.3142019271051\\
7640.70000000003	75.3114857083028\\
7642.39999999989	75.3143604841348\\
7643.20000000001	75.3104025748041\\
7644.29999999991	75.3126471665533\\
7644.50000000009	75.3106768176229\\
7645.50000000002	75.3139060374595\\
7649.50000000012	75.3163564107927\\
7650.99999999998	75.3146606370326\\
7651.99999999999	75.3165797624182\\
7653.40000000008	75.3132553733586\\
7654.39999999995	75.3164805016657\\
7654.90000000012	75.3141737426519\\
7656.69999999992	75.3121408421273\\
7657.69999999997	75.3101412938442\\
7658.59999999995	75.3130426834841\\
7661.5999999999	75.3214038659827\\
7662.00000000004	75.3187768366375\\
7663.09999999998	75.3210147196993\\
7663.30000000003	75.3190489860897\\
7664.99999999999	75.3232182228542\\
7665.59999999992	75.3212361558631\\
7666.30000000007	75.3234895126787\\
7666.50000000012	75.321524534996\\
7667.30000000007	75.3188825416699\\
7668.29999999992	75.3168848782014\\
7669.70000000011	75.3148713134632\\
7670.90000000011	75.3187328901056\\
7671.50000000009	75.3167526982637\\
7674.10000000005	75.3133877146804\\
7674.89999999997	75.3159609120521\\
7676.2999999999	75.3139492470429\\
7678.40000000012	75.316793644592\\
7681.2	75.3127725775585\\
7681.89999999988	75.3098151523041\\
7682.50000000001	75.3078384921771\\
7684.49999999998	75.3116622856102\\
7686.2000000001	75.309316576246\\
7687.99999999995	75.3072930893199\\
7689.00000000005	75.3105044803683\\
7690.40000000004	75.3084974969118\\
7691.50000000004	75.312028706641\\
7692.49999999999	75.3100382185477\\
7693.39999999994	75.3129264963931\\
7696.79999999994	75.310839428861\\
7697.4000000001	75.312763884378\\
7697.69999999991	75.3098287822495\\
7698.50000000009	75.3123944613306\\
7698.7999999999	75.3094597929574\\
7699.60000000009	75.3068301362391\\
7703.40000000012	75.3112221717401\\
7705.30000000009	75.3056298180497\\
7706.19999999996	75.3085138133735\\
7707.59999999996	75.306511670148\\
7708.90000000008	75.3041899078999\\
7714.19999999999	75.3017124042363\\
7715.90000000012	75.2993779160187\\
7717.60000000005	75.3048187931638\\
7717.80000000008	75.3028673602923\\
7719.10000000008	75.3070266348844\\
7720.99999999995	75.302741837303\\
7722.60000000004	75.3065637665583\\
7723.20000000003	75.3020082089262\\
7723.89999999992	75.3042465044019\\
7725.6000000001	75.3019143896346\\
7728.49999999993	75.2995367854463\\
7731.29999999998	75.3058954393771\\
7731.80000000006	75.3036123074535\\
7734.09999999999	75.3019058208994\\
7735.99999999996	75.2989232300513\\
7739.70000000006	75.3029794051526\\
7741.6000000001	75.3051655321183\\
7741.79999999999	75.303220139759\\
7742.80000000009	75.3012437200532\\
7743.99999999989	75.2986144290492\\
7747.5	75.302029015437\\
7751.80000000004	75.3041189901831\\
7752.0000000001	75.3021761845177\\
7754.20000000005	75.2988664353971\\
7755.40000000006	75.3026884146735\\
7755.6000000001	75.3007465477004\\
7758.60000000003	75.3038524495083\\
7759.09999999996	75.3015774822147\\
7760.90000000002	75.3034402783147\\
7761.10000000004	75.3014997680771\\
7762.4000000001	75.2991948470209\\
7763.00000000009	75.297239504837\\
7766.09999999999	75.299374211326\\
7766.59999999993	75.2971017291771\\
7767.6000000001	75.2989945543726\\
7768.50000000007	75.2967072574209\\
7769.80000000002	75.2995534048057\\
7770.40000000008	75.2963129785728\\
7771.19999999988	75.2988560472508\\
7771.60000000006	75.2962672259609\\
7772.99999999999	75.2994300858088\\
7775.00000000007	75.3019253771656\\
7780.10000000012	75.3078327035295\\
7781.5	75.3109900277578\\
7782.49999999995	75.3128774445558\\
7783.8	75.3105769601357\\
7784.29999999992	75.3083089255434\\
7784.90000000002	75.3063583815029\\
7786.30000000001	75.3043768622213\\
7787.00000000002	75.3065968075407\\
7788.99999999995	75.3090857737094\\
7789.5	75.3068193488754\\
7790.90000000013	75.3048389166988\\
7792.9000000001	75.3073270884127\\
7793.99999999988	75.3043969156157\\
7794.90000000012	75.302116741501\\
7796.7	75.3039708598399\\
7796.89999999999	75.3020392458638\\
7798.1000000001	75.3045574619784\\
7798.80000000002	75.3016451037967\\
7800.20000000012	75.2996679614887\\
7802.09999999992	75.3018379431442\\
7802.29999999997	75.2999077207013\\
7802.89999999988	75.3018069973087\\
7805.09999999992	75.2998001332445\\
7805.89999999991	75.3023315398411\\
7806.19999999998	75.2994376337061\\
7810.09999999995	75.3028091470129\\
7810.29999999993	75.3008808767797\\
7810.9	75.2976571501728\\
7812.60000000003	75.3004723079089\\
7814.79999999992	75.298468310535\\
7815.80000000005	75.3003492879899\\
7819.20000000009	75.3034159067947\\
7821.59999999999	75.2994873237276\\
7824.90000000011	75.3035143769968\\
7826.6000000001	75.2986571607446\\
7827.4	75.3011817310763\\
7827.90000000013	75.2989269289729\\
7828.70000000003	75.301451052524\\
7829.6000000001	75.2991813224006\\
7830.60000000009	75.3023356788027\\
7832.60000000007	75.3060885773743\\
7832.79999999993	75.3041657623613\\
7834.30000000009	75.3063412641683\\
7835.19999999998	75.3040725945401\\
7839.89999999995	75.3073979591837\\
7841.29999999995	75.3105312826791\\
7841.49999999993	75.3086104876556\\
7842.90000000008	75.3053678439373\\
7845.30000000004	75.3090983251332\\
7845.50000000002	75.3071785459366\\
7846.20000000002	75.3093814919134\\
7847.1999999999	75.3112535521772\\
7848.30000000002	75.3134396819734\\
7850.00000000008	75.3162380097069\\
7850.1999999999	75.3143191979924\\
7851.50000000008	75.317132813694\\
7853.69999999994	75.3151340752247\\
7854.6999999999	75.3131842949534\\
7855.89999999995	75.3105906313646\\
7858.70000000011	75.3168422659948\\
7862.10000000006	75.3147973849559\\
7862.69999999997	75.3166810805311\\
7862.90000000013	75.3147653567341\\
7863.39999999999	75.3125198702868\\
7865.00000000008	75.3099134149598\\
7866.19999999996	75.3124086292158\\
7867.1000000001	75.3101484645109\\
7868.79999999997	75.3078575150275\\
7871.60000000006	75.3102887559231\\
7872.10000000006	75.3080460354158\\
7872.90000000001	75.3105550616029\\
7873.69999999988	75.3079834387462\\
7874.30000000002	75.3098648785939\\
7874.70000000005	75.3073093919846\\
7875.50000000007	75.3098176646859\\
7882.60000000011	75.3130272622325\\
7883.70000000002	75.3152033283442\\
7884.10000000003	75.3126506176911\\
7885.50000000009	75.3106929085929\\
7887.50000000001	75.3042750646585\\
7888.89999999996	75.3023196856382\\
7889.60000000011	75.2994410433857\\
7890.70000000011	75.2965478785421\\
7893.9	75.3027615910818\\
7894.2	75.299899927796\\
7894.79999999996	75.3017770966067\\
7895.00000000007	75.2998695393345\\
7899.49999999994	75.3025469644033\\
7900.00000000007	75.3003126542702\\
7903.40000000009	75.3033466185867\\
7903.60000000012	75.3014410972076\\
7905.79999999994	75.2994598970389\\
7906.40000000008	75.2975399987352\\
7908.90000000003	75.3002908079403\\
7909.20000000008	75.2974346655203\\
7910.59999999993	75.2992781928274\\
7911.10000000012	75.2970472241885\\
7911.80000000009	75.2992328012235\\
7916.5000000001	75.2974761892732\\
7920.70000000011	75.3004746995253\\
7922.29999999996	75.2966272846612\\
7923.40000000002	75.2987947245535\\
7924.40000000011	75.3006498832734\\
7924.80000000011	75.2968491715984\\
7927.50000000006	75.3002169635198\\
7929.49999999998	75.3026634382567\\
7929.7	75.3007642059068\\
7931.3999999999	75.3035365315514\\
7933.70000000012	75.3069147193022\\
7934.20000000002	75.3046897646925\\
7934.90000000004	75.306868304978\\
7935.4	75.3046436897486\\
7937.60000000005	75.3026695390352\\
7938.39999999999	75.3051584052403\\
7941.00000000007	75.3069474002342\\
7941.90000000004	75.3047091412742\\
7942.80000000013	75.3024713895429\\
7944.49999999994	75.3002039120912\\
7945.49999999999	75.3020539669754\\
7945.89999999996	75.2995217719607\\
7946.70000000007	75.2957165148236\\
7947.20000000006	75.2934959042694\\
7950.79999999998	75.2971361732634\\
7952.20000000007	75.2951976157841\\
7954.39999999995	75.2970016971526\\
7960.00000000013	75.2993052851095\\
7962.19999999994	75.2973387086646\\
7963.19999999995	75.2991850112391\\
7964.59999999988	75.2947380315643\\
7967.4000000001	75.2971446501412\\
7967.70000000011	75.2943095961244\\
7972.10000000004	75.2966558791802\\
7973.49999999996	75.294722584529\\
7975.29999999999	75.2927752839983\\
7975.90000000012	75.2946339017051\\
7976.89999999989	75.2927165601103\\
7980.50000000009	75.2963436333108\\
7980.70000000005	75.2944566960706\\
7982.10000000009	75.2925258700609\\
7983.09999999994	75.2906102815913\\
7984.89999999992	75.2874139010645\\
7985.90000000013	75.2854996243426\\
7987.59999999991	75.290759542797\\
7989.50000000011	75.2928807449685\\
7990.00000000005	75.290672206856\\
7991.59999999999	75.2931166084813\\
7992.59999999994	75.2912032229409\\
7993.20000000006	75.2880537450115\\
7993.79999999988	75.2861556937165\\
7995.29999999987	75.2882907671911\\
7997.00000000004	75.2860411899314\\
7997.70000000005	75.2882042561704\\
7999.10000000012	75.2900290029003\\
8001.00000000005	75.2946469860394\\
8002.79999999988	75.2927063939322\\
8004.99999999994	75.295748960038\\
8005.20000000004	75.2938678125742\\
8005.8000000001	75.2957194069374\\
8006.29999999989	75.2935151878497\\
8010.59999999995	75.2955422122911\\
8013.50000000005	75.2982429869222\\
8013.69999999993	75.2963637724924\\
8014.69999999989	75.2981983330838\\
8015.29999999999	75.2950570152456\\
8016.49999999998	75.2987550832024\\
8018.80000000007	75.3020987916048\\
8019.00000000009	75.3002207230238\\
8019.99999999993	75.3020535903542\\
8020.2	75.3001758038976\\
8021.20000000001	75.302008402628\\
8027.09999999995	75.3002292206498\\
8027.80000000008	75.3023829394985\\
8029.29999999993	75.304505940668\\
8029.90000000003	75.3026151930262\\
8031.29999999989	75.3056752247429\\
8031.50000000008	75.3037999900393\\
8032.19999999995	75.3059522179202\\
8032.7000000001	75.3037546061149\\
8034.40000000005	75.3015122285145\\
8035.1	75.2986857825567\\
8038.20000000004	75.3007476705273\\
8039.59999999991	75.3025610408349\\
8040.5999999999	75.3043889213626\\
8040.79999999994	75.302515887525\\
8042.80000000009	75.3061706598366\\
8045.00000000003	75.3042224459609\\
8047.50000000009	75.30692380337\\
8050.39999999995	75.310850257748\\
8052.99999999997	75.3088872608064\\
8053.70000000012	75.3060667014329\\
8056.99999999991	75.3161807598267\\
8058.59999999989	75.3185997741571\\
8060.9000000001	75.3206798163999\\
8061.40000000013	75.318489114929\\
8062.4000000001	75.3203100775194\\
8063.30000000005	75.3181040255971\\
8065.50000000007	75.3223566752629\\
8068.09999999994	75.3253513794898\\
8071.70000000008	75.3227285116083\\
8073.30000000004	75.3251418237669\\
8075.9	75.3231797919762\\
8077.49999999995	75.3280677428939\\
8078.00000000013	75.3246431710427\\
8079.0000000001	75.3227463455088\\
8080.19999999989	75.3264111481009\\
8082.59999999988	75.3238398060054\\
8083.40000000007	75.3262819323313\\
8083.59999999999	75.3244182738103\\
8085.49999999997	75.321559315326\\
8086.10000000005	75.3233904677104\\
8089.60000000008	75.3291222171403\\
8091.90000000002	75.3324270884825\\
8092.09999999991	75.330565235659\\
8093.39999999992	75.3283499104219\\
8095.10000000011	75.3261191817373\\
8096.09999999997	75.3242261801833\\
8097.39999999994	75.3220129669651\\
8098.29999999994	75.3247555072607\\
8099.29999999993	75.3265674000543\\
8099.89999999999	75.3246913580247\\
8100.9999999999	75.3268074705904\\
8104.10000000008	75.3300757631845\\
8104.40000000013	75.3272873095194\\
8105.29999999989	75.3250919140326\\
8106.09999999988	75.327527078039\\
8109.50000000004	75.3292394199467\\
8109.79999999995	75.3264528539193\\
8111.10000000005	75.3304073380018\\
8111.90000000011	75.3266765285996\\
8113.10000000006	75.3290933293891\\
8113.40000000012	75.3263080051766\\
8116.40000000007	75.3243393088154\\
8118.79999999991	75.3291702077868\\
8119.00000000002	75.3273146038354\\
8119.90000000013	75.3300492610837\\
8121.99999999993	75.3327341451103\\
8122.80000000009	75.3302392002856\\
8126.29999999988	75.3273282142154\\
8129.00000000004	75.3244516613155\\
8131.60000000013	75.3261925550623\\
8133.40000000007	75.3279645908895\\
8133.59999999988	75.3261123473942\\
8136.10000000013	75.3238612620142\\
8137.09999999988	75.3268937718134\\
8137.3999999999	75.3241167434716\\
8139.70000000012	75.3261750902971\\
8141.40000000006	75.3300988761285\\
8142.99999999988	75.3324900836291\\
8143.30000000013	75.3297148611145\\
8143.9999999999	75.331835316364\\
8145.29999999997	75.3296339037985\\
8147.09999999987	75.3326296150825\\
8149.99999999987	75.3291370658029\\
8150.79999999996	75.3315584782049\\
8151.90000000011	75.3275269872424\\
8152.80000000012	75.3302505856812\\
8154.39999999998	75.3338647372616\\
8157.19999999987	75.3374278253834\\
8158.09999999989	75.3340197592606\\
8159.49999999991	75.3321241237316\\
8160.99999999992	75.3354327235299\\
8161.20000000009	75.3335865609645\\
8161.79999999999	75.3317242308776\\
8163.30000000013	75.3338069921846\\
8163.79999999999	75.3316429647595\\
8169.40000000004	75.3338637615521\\
8170.5000000001	75.3359606393655\\
8171.40000000002	75.3313345163067\\
8172.39999999991	75.3331293973692\\
8172.90000000003	75.3309678208736\\
8175.80000000013	75.3360486307318\\
8178.80000000005	75.3377593563927\\
8179.79999999988	75.3358843017641\\
8181.00000000006	75.3333903753774\\
8182.70000000005	75.3360707826172\\
8185.70000000005	75.3341151750593\\
8186.40000000004	75.3313381787088\\
8187.39999999989	75.3282442748092\\
8188.09999999999	75.3303534354315\\
8192.70000000005	75.3283370764574\\
8194.39999999991	75.3261333821466\\
8194.99999999992	75.3242791424144\\
8197.59999999996	75.3260060748747\\
8198.20000000001	75.3229328031421\\
8199.2	75.3210639932677\\
8200.1	75.3188946611058\\
8201.49999999994	75.3157920405774\\
8203.90000000012	75.3132618235007\\
8204.99999999989	75.3153526465247\\
8205.99999999989	75.3110490976225\\
8208.40000000009	75.3085216543827\\
8210.30000000013	75.3057098314333\\
8211.00000000005	75.3078150308728\\
8221.40000000012	75.309858298364\\
8222.00000000007	75.3080113353036\\
8223.00000000013	75.3110140944413\\
8223.89999999993	75.3088521400778\\
8224.50000000009	75.3106534056368\\
8224.6999999999	75.3088220990176\\
8225.99999999989	75.3066459196946\\
8227.00000000007	75.3047854043345\\
8231.20000000005	75.30280757596\\
8232.10000000001	75.3055076407279\\
8232.29999999996	75.3036781497498\\
8233.50000000008	75.3072774970851\\
8233.69999999994	75.3054482741869\\
8234.69999999987	75.3035896439501\\
8235.40000000012	75.3056887863518\\
8236.29999999999	75.3035306687388\\
8237.39999999997	75.3007587253414\\
8239.09999999987	75.2985726769589\\
8240.29999999999	75.300956264259\\
8240.50000000013	75.2991287042206\\
8241.09999999989	75.3009270494588\\
8243.90000000011	75.2983988355167\\
8244.50000000002	75.3001964922495\\
8245.00000000011	75.2980558149689\\
8245.3999999999	75.2956157904311\\
8246.99999999996	75.2979835336057\\
8247.89999999992	75.2958292919496\\
8248.89999999998	75.2976118317372\\
8249.10000000012	75.2957862580614\\
8249.70000000013	75.2975829717084\\
8250.70000000007	75.2957288990159\\
8251.30000000006	75.2975252684393\\
8252.70000000002	75.2956572314851\\
8253.49999999994	75.2932053891635\\
8254.10000000006	75.2950013326549\\
8255.40000000001	75.2928350796439\\
8256.2000000001	75.2952290977799\\
8257.5000000001	75.293063359814\\
8259.99999999995	75.2956985992906\\
8262.20000000014	75.2974353388282\\
8264.00000000001	75.2991856342493\\
8264.89999999995	75.29703569268\\
8265.90000000013	75.2988144205178\\
8266.49999999998	75.2969782014371\\
8268.90000000012	75.3041480227355\\
8273.00000000013	75.3067169501154\\
8275.90000000002	75.3093281778637\\
8276.50000000003	75.3074934151705\\
8280.10000000004	75.3097751262047\\
8280.40000000005	75.3070466759254\\
8280.99999999987	75.3088357826859\\
8281.59999999989	75.3070021855416\\
8282.60000000001	75.3051541164113\\
8283.60000000002	75.3081352535703\\
8284.69999999991	75.3053785245268\\
8285.29999999995	75.3071668235692\\
8286.20000000001	75.3050215415807\\
8286.69999999997	75.302891345272\\
8287.09999999993	75.3004633651897\\
8288.60000000007	75.3049332223388\\
8289.49999999991	75.3027890368655\\
8290.7	75.3063636802239\\
8293.20000000002	75.3113959461252\\
8293.40000000003	75.3095797914029\\
8295.70000000014	75.3116034619928\\
8295.89999999987	75.309787849566\\
8296.59999999996	75.3118709848494\\
8296.8999999999	75.3091478847776\\
8297.40000000013	75.3070201868033\\
8299	75.304551095902\\
8300.40000000009	75.306306848985\\
8302.69999999996	75.3095341330636\\
8303.19999999998	75.3074078980646\\
8304.59999999997	75.3091622815996\\
8305.49999999991	75.3070217684454\\
8305.89999999999	75.3045990849988\\
8307.29999999998	75.3027421335195\\
8308.39999999993	75.3048083288199\\
8310.50000000011	75.3026255625346\\
8312.10000000008	75.3049734125743\\
8313.80000000003	75.3076173636921\\
8317.70000000009	75.3131837745558\\
8318.29999999992	75.3101557991922\\
8323.3	75.3141744959992\\
8323.79999999994	75.3120532442725\\
8326.00000000009	75.3101692268889\\
8327.30000000007	75.3140235847924\\
8327.80000000009	75.3119033609914\\
8328.99999999989	75.3094572042598\\
8330.40000000006	75.3136066262529\\
8330.89999999999	75.3114872164206\\
8331.70000000014	75.313857749826\\
8331.90000000007	75.3120499279885\\
8333.69999999999	75.313782428184\\
8335.30000000014	75.3113227919476\\
8335.90000000012	75.3130998080614\\
8336.29999999989	75.3106856676743\\
8336.80000000011	75.3085679329247\\
8340.90000000014	75.3111137753267\\
8341.60000000002	75.3083903760624\\
8342.30000000003	75.3104622171078\\
8344.4999999999	75.3121779354313\\
8344.70000000003	75.3103729268527\\
8345.50000000007	75.307946702454\\
8347.09999999993	75.3114816944604\\
8347.29999999987	75.3096772647771\\
8349.19999999993	75.3117027774783\\
8349.59999999993	75.3080949016132\\
8350.69999999987	75.3101499257556\\
8352.60000000005	75.3145689417793\\
8352.89999999996	75.3118640009577\\
8354.20000000007	75.3157056844978\\
8355.60000000006	75.3186447574709\\
8356.70000000003	75.3206969174804\\
8358.49999999987	75.3236187878353\\
8360.19999999993	75.3262442735308\\
8360.39999999987	75.3244423180432\\
8360.79999999995	75.322034709182\\
8361.69999999998	75.3246908560358\\
8362.40000000007	75.3219730941704\\
8364.39999999993	75.3195050511088\\
8366.10000000009	75.3221295211685\\
8368.30000000008	75.3238372926724\\
8370.50000000003	75.3279334814709\\
8370.79999999994	75.325233845823\\
8371.50000000001	75.3272970519375\\
8372.40000000005	75.3251716930427\\
8376.50000000009	75.3276985889263\\
8377.60000000012	75.3297444405983\\
8378.10000000014	75.3276360077344\\
8380.49999999992	75.3335083406916\\
8380.69999999998	75.3317105765559\\
8381.19999999993	75.3284096739169\\
8381.79999999994	75.3301757358117\\
8383.10000000009	75.3280370264338\\
8384.09999999994	75.3297869802724\\
8387.10000000005	75.3326497520031\\
8387.30000000007	75.3308534229916\\
8389.79999999989	75.3286689948629\\
8392.00000000013	75.331561826003\\
8393.60000000012	75.3350727331213\\
8394.59999999991	75.3368196600236\\
8397.50000000009	75.3393826807659\\
8398.99999999989	75.3414056267933\\
8401.69999999994	75.3386179152086\\
8402.79999999997	75.3406562020255\\
8406.70000000009	75.3437693295903\\
8407.50000000001	75.3401684190494\\
8408.19999999987	75.342221376497\\
8409.90000000002	75.3448275862069\\
8411.7	75.3477258137378\\
8411.90000000013	75.3459343794579\\
8412.49999999998	75.3476927465944\\
8412.70000000009	75.3459014834538\\
8413.29999999995	75.3440939453729\\
8414.10000000002	75.346438164056\\
8415.90000000006	75.3445817490494\\
8417.70000000001	75.3427261279669\\
8420.60000000012	75.3452800836035\\
8424.99999999988	75.3486605500232\\
8426.29999999989	75.3465299534795\\
8428.59999999998	75.3485116328734\\
8429.59999999999	75.3502497123267\\
8431.70000000007	75.3540169358856\\
8431.89999999999	75.352229601518\\
8434.20000000002	75.3542084108936\\
8438.60000000002	75.3563937573323\\
8438.89999999993	75.3537148951298\\
8439.89999999994	75.3566350710901\\
8441.99999999995	75.3592115705808\\
8442.19999999988	75.3574262937825\\
8444.00000000011	75.3591264906858\\
8444.69999999992	75.3552482000758\\
8445.30000000008	75.3569990764203\\
8446.49999999987	75.3593161745554\\
8448.09999999998	75.3627991761559\\
8448.3	75.3610151034515\\
8453.29999999989	75.35902713701\\
8455.80000000006	75.3568514291796\\
8458.70000000012	75.3605712394193\\
8459.70000000014	75.3587555261354\\
8461.50000000011	75.3628155431597\\
8464.90000000005	75.3668044890726\\
8468.40000000007	75.3699002184566\\
8469.79999999995	75.3680680999776\\
8470.19999999994	75.3656895269353\\
8471.69999999995	75.3688708420879\\
8472.19999999993	75.3656032010198\\
8473.10000000005	75.3682197988953\\
8473.29999999989	75.3664408619916\\
8474.60000000002	75.3702195947939\\
8477.89999999999	75.3727294173154\\
8478.09999999989	75.3709513811894\\
8478.80000000012	75.3729847032044\\
8479.00000000003	75.3712068497836\\
8481.49999999989	75.369034144501\\
8482.80000000001	75.3669146164637\\
8484.20000000008	75.3615501573495\\
8486.00000000007	75.3585274743404\\
8486.70000000013	75.3558467266814\\
8487.50000000007	75.3581695650125\\
8488.49999999995	75.3563602949839\\
8490.50000000011	75.3586318988057\\
8493.00000000005	75.3564658369735\\
8496.60000000008	75.3539609495451\\
8498.00000000005	75.3568444711171\\
8500.10000000012	75.3546975365285\\
8501.09999999999	75.3575965746012\\
8501.70000000009	75.3558070055753\\
8502.20000000007	75.3537278148266\\
8504.30000000008	75.3515827101265\\
8505.30000000004	75.3486020645707\\
8505.89999999991	75.35034093581\\
8506.89999999991	75.3485364993535\\
8510.70000000007	75.3513183249518\\
8510.89999999998	75.3495476442251\\
8513.19999999988	75.3538580808852\\
8513.4	75.3520878604569\\
8515.00000000002	75.3543704712804\\
8515.60000000014	75.3525840506359\\
8516.70000000013	75.3498966748074\\
8517.60000000002	75.3525012620778\\
8520.10000000007	75.355038614117\\
8522.40000000007	75.358169551188\\
8525.30000000014	75.3559950266263\\
8526.39999999997	75.3580015246584\\
8527.00000000013	75.3562172368097\\
8528.00000000011	75.3591069523106\\
8529.10000000009	75.3564226422173\\
8531.10000000014	75.3610277569392\\
8533.99999999987	75.358854477918\\
8536.79999999988	75.3610795487824\\
8539.80000000009	75.3591962435157\\
8540.39999999996	75.3574146712722\\
8541.10000000011	75.3594342715309\\
8543.00000000003	75.3614027694865\\
8544.50000000001	75.3633874025701\\
8544.70000000006	75.3616234434978\\
8545.29999999998	75.3633533831067\\
8545.49999999999	75.3615895899644\\
8546.90000000013	75.3597753597754\\
8548.40000000011	75.3617593729894\\
8551.30000000009	75.3595902425334\\
8552.39999999989	75.3627594270681\\
8552.90000000006	75.3606921547995\\
8553.49999999994	75.3589132061354\\
8557.60000000004	75.3613704616895\\
8557.89999999992	75.3587286749241\\
8558.60000000007	75.3560704312571\\
8560.00000000004	75.3542598801416\\
8561.0999999999	75.3574265289913\\
8565.09999999995	75.3630971839537\\
8565.59999999994	75.3610329570263\\
8566.29999999999	75.363046320508\\
8568.39999999987	75.3609149792846\\
8569.50000000002	75.3629107542942\\
8569.70000000001	75.3611519522043\\
8571.19999999989	75.3654638152906\\
8574.09999999996	75.3632991999254\\
8578.90000000009	75.3665928429887\\
8580.30000000009	75.369446645844\\
8581.29999999997	75.3676556272869\\
8583.20000000006	75.3707781389442\\
8585.20000000005	75.3683622005055\\
8587.80000000004	75.3665040347466\\
8588.20000000004	75.3641582152463\\
8589.20000000003	75.3623694596766\\
8592.09999999995	75.364865808524\\
8596.39999999995	75.366719013552\\
8596.89999999995	75.3634988949634\\
8603.80000000012	75.3669847394786\\
8604.20000000013	75.3634810501726\\
8606.30000000006	75.3613589886596\\
8606.89999999993	75.3630765655862\\
8607.19999999995	75.3604498507081\\
8610.50000000013	75.358279330128\\
8613.99999999992	75.3682915220394\\
8614.30000000008	75.3656667904903\\
8616.30000000003	75.3690636460703\\
8616.7	75.3667254665305\\
8618.80000000011	75.3692466556057\\
8619.00000000005	75.3674977665882\\
8620.40000000002	75.365698045357\\
8621.29999999989	75.3682696545805\\
8621.60000000009	75.3656471461545\\
8627.40000000008	75.3636627064619\\
8628.1000000001	75.3610254746065\\
8629.09999999999	75.3592453529875\\
8632.79999999991	75.3570642541904\\
8633.79999999999	75.3552855604072\\
8634.79999999996	75.3581396426131\\
8634.99999999989	75.3563942513694\\
8635.60000000001	75.3546325138669\\
8638.20000000014	75.3585774978873\\
8639.50000000005	75.3564979860179\\
8640.59999999989	75.3596352147396\\
8642.30000000003	75.3633250023142\\
8642.70000000003	75.359837089832\\
8644.20000000003	75.3617991046123\\
8644.40000000013	75.3600555266354\\
8646.10000000006	75.3579607226296\\
8646.7	75.3596706295971\\
8646.90000000011	75.3579276049497\\
8648.70000000008	75.3607436869855\\
8649.89999999993	75.3630057803468\\
8650.09999999989	75.3612633233914\\
8653.89999999991	75.3593713889531\\
8656.49999999995	75.3621514220364\\
8658.50000000008	75.3597579285335\\
8659.0999999999	75.3568458980044\\
8662.60000000003	75.3541043785425\\
8664.1000000001	75.3514461808361\\
8665.80000000006	75.3493578278078\\
8667.80000000009	75.34696985429\\
8668.70000000009	75.3495293466224\\
8668.89999999996	75.3477909793517\\
8669.49999999999	75.3460367260312\\
8670.60000000007	75.3480111179028\\
8672.09999999989	75.3453564262817\\
8673.19999999997	75.3484832762616\\
8675.10000000006	75.350424197713\\
8676.2	75.3477864988532\\
8678.29999999996	75.3537518436578\\
8678.50000000008	75.3520153020072\\
8679.10000000006	75.353719236796\\
8679.60000000005	75.3493784347385\\
8680.99999999991	75.3464422711408\\
8683.00000000009	75.348665799081\\
8684.30000000002	75.3512044585694\\
8684.49999999987	75.3494691753218\\
8685.19999999991	75.3514559082588\\
8686.90000000003	75.3493726257626\\
8689.10000000009	75.3475578879529\\
8691.79999999987	75.3506137898503\\
8699.00000000013	75.3526226851054\\
8701.39999999996	75.3548238809401\\
8701.59999999997	75.3530919245665\\
8702.70000000012	75.3562071976835\\
8703.69999999995	75.3544428870149\\
8704.70000000006	75.3572741475967\\
8704.90000000011	75.3555427914991\\
8707.99999999994	75.357425844903\\
8708.80000000013	75.3550965104663\\
8710.9000000001	75.3518539777293\\
8714.29999999991	75.3499954098963\\
8714.89999999987	75.3482501434309\\
8716.19999999998	75.3507795739018\\
8717.89999999988	75.3532920394586\\
8722.70000000008	75.3553904709497\\
8723.79999999999	75.3573516431871\\
8729.49999999991	75.3596957478006\\
8729.90000000005	75.3573883161512\\
8730.59999999991	75.3593640830632\\
8732.90000000002	75.3612733310432\\
8734.89999999999	75.3634802518603\\
8735.09999999987	75.361754739445\\
8736.99999999995	75.3579563012899\\
8738.40000000012	75.356182411169\\
8740.79999999992	75.361804848471\\
8741.00000000006	75.3600805390626\\
8741.90000000001	75.3626172500572\\
8744.4	75.360512322031\\
8745.10000000002	75.3624845629602\\
8746.50000000005	75.3607115907896\\
8747.59999999997	75.3638099157493\\
8748.80000000014	75.361474013876\\
8749.80000000004	75.3642898775986\\
8750.20000000008	75.361987588997\\
8752.40000000001	75.3647529277349\\
8753.29999999992	75.3627162017045\\
8755.00000000005	75.3663578942559\\
8756.20000000014	75.3640236172813\\
8757.09999999989	75.3619878499977\\
8759.7000000001	75.3601680403662\\
8760.10000000008	75.3578685418141\\
8761.49999999988	75.3606647187728\\
8761.70000000014	75.3589445091191\\
8762.69999999997	75.3571917651892\\
8763.90000000001	75.3548607941579\\
8764.79999999998	75.3528277561638\\
8766.3	75.3547636430005\\
8766.50000000003	75.3530445098442\\
8767.69999999992	75.3552772645361\\
8768.50000000006	75.3529639851288\\
8770.69999999994	75.3500250832307\\
8772.99999999998	75.3542077486863\\
8775.1999999999	75.3569678529509\\
8775.40000000011	75.3552504130819\\
8776.60000000013	75.3529230804289\\
8777.80000000011	75.3562925073195\\
8777.99999999987	75.3545755915289\\
8779.19999999992	75.3568052122607\\
8780.40000000012	75.3544786743352\\
8782.10000000002	75.3569720571155\\
8782.29999999997	75.3552559664784\\
8784.39999999999	75.3577323695145\\
8785.00000000011	75.356000500848\\
8787.29999999999	75.3590368026948\\
8787.89999999997	75.3561675011379\\
8790.20000000002	75.353514669579\\
8791.40000000001	75.3557413410681\\
8794.10000000001	75.357622069091\\
8797.20000000002	75.3628954338263\\
8797.7	75.3608856759645\\
8798.89999999999	75.363109444255\\
8799.39999999992	75.3611000625036\\
8800.20000000004	75.3633398861403\\
8801.8000000001	75.3610016019269\\
8803.19999999996	75.3637840355321\\
8804.10000000001	75.3617591604007\\
8808.39999999993	75.363569279673\\
8809.29999999991	75.3615456217222\\
8809.8999999999	75.3586833144154\\
8811.7000000001	75.3569077827459\\
8812.40000000012	75.3588652482269\\
8812.59999999996	75.3571550149216\\
8814.20000000004	75.3536866228742\\
8815.29999999987	75.3556276516097\\
8820.20000000005	75.3591147693389\\
8820.80000000013	75.3573898355043\\
8821.60000000011	75.3550902887199\\
8823.09999999989	75.3592800797896\\
8826.59999999987	75.3611202374613\\
8827.10000000009	75.357984411818\\
8828.29999999991	75.3602011689547\\
8828.59999999998	75.3576404227123\\
8829.69999999986	75.3595777933815\\
8830.60000000002	75.3575594233753\\
8833.10000000004	75.3554770638047\\
8834.40000000001	75.3534438847699\\
8836.29999999988	75.3553483318999\\
8836.60000000009	75.3527900686908\\
8838.99999999998	75.3560882895317\\
8839.20000000001	75.3543832656432\\
8839.80000000009	75.3526623604339\\
8841.60000000002	75.3576800841467\\
8843.20000000011	75.3553537706512\\
8846.29999999988	75.3526858383071\\
8847.69999999995	75.3509346956306\\
8850.39999999994	75.3539348059432\\
8850.69999999996	75.3513806661545\\
8851.39999999997	75.3533299440773\\
8851.99999999999	75.3516114820212\\
8852.39999999997	75.3493363456651\\
8853.30000000013	75.3473241918359\\
8853.99999999991	75.3492732180572\\
8854.4	75.3469987012254\\
8855.69999999995	75.3449716569932\\
8857.50000000005	75.3477239884393\\
8858.99999999998	75.3439965685002\\
8859.49999999989	75.3420018962481\\
8862.59999999998	75.3449851625351\\
8867.49999999987	75.3484595606478\\
8867.99999999997	75.3464665486406\\
8871.00000000011	75.3446584978188\\
8871.59999999987	75.3429444187698\\
8874.50000000011	75.3453676785433\\
8874.89999999993	75.3430985915493\\
8876.39999999999	75.3450121106292\\
8879.49999999999	75.3412315870084\\
8883.40000000003	75.3430517251083\\
8884.30000000011	75.3410472288506\\
8888.60000000009	75.3428510355845\\
8890.99999999997	75.3405090483742\\
8892.39999999999	75.3443913410177\\
8893.39999999996	75.3426659920166\\
8894.40000000002	75.3454381921412\\
8900.10000000001	75.3410035729534\\
8901.00000000006	75.3434968711732\\
8902.50000000006	75.3454047132298\\
8903.09999999987	75.3436966483961\\
8903.50000000012	75.3414349251988\\
8904.09999999995	75.3386042541722\\
8906.20000000003	75.3365595140518\\
8906.89999999997	75.3384978107107\\
8908.10000000014	75.3362070900968\\
8909	75.338698633981\\
8911.30000000008	75.3405749938281\\
8912.00000000007	75.3380235859113\\
8916.50000000002	75.3403763766458\\
8916.89999999998	75.3369967477851\\
8918.50000000014	75.3414212993071\\
8922.49999999993	75.3390267410844\\
8924.10000000013	75.3412070549741\\
8926.7000000001	75.3439082313931\\
8927.29999999986	75.3422048972825\\
8928.59999999988	75.3401951011905\\
8929.39999999987	75.3424043899434\\
8930.00000000008	75.3407016718738\\
8932.60000000004	75.3389232818745\\
8933.79999999992	75.3411164217195\\
8934.2	75.338862585765\\
8937.00000000005	75.3421132134585\\
8937.40000000004	75.3398601398601\\
8939.90000000003	75.3422818791946\\
8940.40000000014	75.3403053520497\\
8941.90000000013	75.3377320509953\\
8942.90000000003	75.3360169965336\\
8943.90000000013	75.3387745974955\\
8944.9	75.3370598099497\\
8947.19999999999	75.3389290623987\\
8947.70000000008	75.3369543351438\\
8948.89999999995	75.3391440384401\\
8949.49999999988	75.3363278805757\\
8950.19999999986	75.3382568182072\\
8953.3999999999	75.3403696878316\\
8953.89999999992	75.3372794281885\\
8954.49999999996	75.3355817121926\\
8956.89999999992	75.3399575750809\\
8957.19999999988	75.3374342714881\\
8958.39999999997	75.3351565552269\\
8960.40000000003	75.3328497293678\\
8961.80000000004	75.3311239804059\\
8962.79999999988	75.3338763123543\\
8963.79999999991	75.3321656868104\\
8964.99999999994	75.3354675352199\\
8967.60000000015	75.3336976036219\\
8969.40000000001	75.3319583031384\\
8970.80000000006	75.3358080014268\\
8971.29999999988	75.3338386427982\\
8974.20000000002	75.3306664586653\\
8975.80000000009	75.333949798906\\
8976.29999999991	75.3308676083954\\
8977.99999999987	75.3288557712656\\
8980.50000000009	75.3312696256375\\
8980.90000000014	75.32902794789\\
8982.49999999996	75.3311958675662\\
8982.99999999997	75.3281161291759\\
8986.10000000005	75.3254990986179\\
8987.10000000011	75.3237938401282\\
8988.79999999997	75.3217857579904\\
8989.80000000011	75.3245308624123\\
8993.30000000013	75.3219027286677\\
8994.49999999999	75.3196362261802\\
8995.99999999993	75.3215282177833\\
8998.00000000008	75.3236794434381\\
8999.50000000004	75.3211253833504\\
9000.60000000004	75.3230304309665\\
9003.00000000012	75.3251657762326\\
9003.80000000009	75.3229156254512\\
9004.60000000004	75.3206658744878\\
9007.49999999996	75.3241707002975\\
9009.59999999987	75.3265924503591\\
9012.59999999995	75.324819421483\\
9016.90000000009	75.3277143174005\\
9018.30000000004	75.3260001774151\\
9019.10000000008	75.3281887528827\\
9021.49999999987	75.3303183470781\\
9022.19999999993	75.326690533456\\
9023.69999999994	75.3241428223143\\
9027.40000000011	75.3220714483523\\
9028.19999999999	75.3198276530465\\
9031.09999999991	75.3233235893347\\
9032.10000000013	75.3216270676026\\
9033.29999999986	75.3237983483517\\
9033.7	75.3215701033895\\
9036.09999999998	75.3270179942896\\
9036.60000000001	75.3250633527726\\
9037.39999999999	75.3272475795297\\
9039.60000000014	75.3299335154928\\
9039.89999999999	75.3274336283186\\
9041.80000000011	75.3304062199317\\
9042.69999999988	75.3284380943956\\
9044.30000000003	75.330591305117\\
9047.50000000009	75.3348954418851\\
9048.29999999988	75.3326554971045\\
9049.00000000002	75.3345636582644\\
9050.3999999999	75.3317496270924\\
9052.5000000001	75.3341581424121\\
9053.5000000001	75.3313599010338\\
9054.90000000004	75.3296521258973\\
9056.19999999994	75.3276724490134\\
9058.30000000015	75.3322882628279\\
9059.90000000007	75.3355408388521\\
9064.20000000005	75.3373123131406\\
9064.50000000001	75.3348189660879\\
9065.59999999986	75.3378117519883\\
9066.49999999996	75.3358480577063\\
9069.00000000009	75.3404417196855\\
9071.50000000009	75.3384188015345\\
9072.29999999987	75.3361844715841\\
9073.29999999988	75.33339211321\\
9075.39999999992	75.3357941711201\\
9076.29999999993	75.3338327971442\\
9077.40000000008	75.3357201872762\\
9079.19999999987	75.334001519941\\
9085.00000000009	75.3321372356936\\
9086.40000000003	75.335937929896\\
9087.49999999997	75.3378229675602\\
9088.6999999999	75.3355778540621\\
9089.89999999998	75.3377337733773\\
9091.2999999999	75.3404316166927\\
9091.60000000009	75.3379455987329\\
9092.29999999986	75.3398442655404\\
9095.49999999999	75.3375258366683\\
9097.70000000013	75.3412913011937\\
9098.10000000014	75.3390780593964\\
9101.3000000001	75.3433537697497\\
9101.6999999999	75.3411413127074\\
9105.70000000004	75.338795053702\\
9107.59999999997	75.3406458271572\\
9109.99999999992	75.3438491344771\\
9110.29999999988	75.3413681067791\\
9113.3	75.3396098053416\\
9114.09999999999	75.3417743740537\\
9115.80000000002	75.3397909147753\\
9121.60000000014	75.3456044377693\\
9122.09999999996	75.3436670978492\\
9123.89999999988	75.3419552827707\\
9127.30000000006	75.3401844994193\\
9128.29999999998	75.3374085272337\\
9129.29999999989	75.3335378009508\\
9129.79999999989	75.3305074535318\\
9130.69999999988	75.32855828624\\
9132.79999999999	75.3331362436904\\
9133.90000000013	75.3306328005255\\
9137.19999999995	75.3285981635713\\
9138.00000000003	75.3307580350401\\
9138.79999999999	75.3285406339931\\
9141.10000000001	75.3303723799939\\
9141.49999999998	75.328170123392\\
9142.19999999986	75.330059175481\\
9143.70000000008	75.3319188958639\\
9146.19999999997	75.3299148289472\\
9146.99999999995	75.3276994894557\\
9148.00000000006	75.3238377367978\\
9149.30000000006	75.3262509016985\\
9151.0000000001	75.3242779556556\\
9153.49999999998	75.3222775738507\\
9157.29999999991	75.3205058204294\\
9159.50000000001	75.323158216516\\
9161.19999999986	75.3211880409985\\
9162.79999999993	75.3189492409608\\
9164.19999999991	75.3216284931746\\
9166.90000000002	75.319079306207\\
9168.19999999996	75.3171253122171\\
9170.49999999999	75.3189540488082\\
9171.4000000001	75.3170146649948\\
9174.80000000004	75.319622012229\\
9176.00000000012	75.3174006386155\\
9176.89999999997	75.3154625694671\\
9178.09999999988	75.3132422479353\\
9181.49999999995	75.3104034155267\\
9182.60000000002	75.3079159724264\\
9183.7000000001	75.3108734946318\\
9184.00000000005	75.3084134536863\\
9184.50000000003	75.3064913006554\\
9185.59999999992	75.3040051384217\\
9186.7	75.3015195715592\\
9187.69999999997	75.3042077537604\\
9190.50000000015	75.3073792788284\\
9190.80000000002	75.3049211720289\\
9197.09999999999	75.308789631627\\
9197.50000000009	75.3066017221884\\
9198.80000000011	75.3046559914772\\
9200.50000000009	75.3026976501532\\
9202.70000000002	75.3009953492415\\
9205.0000000001	75.3028212621264\\
9205.50000000004	75.3009037976884\\
9209.00000000001	75.3026897308097\\
9211.19999999998	75.2999033795447\\
9213.09999999995	75.3028263795424\\
9215.29999999999	75.3011263754151\\
9217.59999999986	75.3029497597014\\
9221.10000000011	75.3047325727671\\
9222.29999999996	75.3014399722415\\
9223.19999999994	75.2995131894224\\
9224.00000000005	75.3016554460598\\
9225.59999999995	75.3037709875673\\
9227.19999999992	75.3015508328547\\
9228.70000000008	75.3033980582524\\
9232.09999999986	75.3059942375598\\
9233.2	75.3078530969426\\
9234.60000000008	75.3105136062893\\
9238.89999999997	75.3122632319515\\
9239.3999999999	75.3092699821419\\
9241.80000000011	75.3113537259654\\
9242.29999999989	75.3094434346057\\
9243.39999999989	75.3069724671391\\
9244.60000000003	75.3101777234524\\
9247.09999999987	75.3082014015053\\
9250.10000000015	75.3108040907224\\
9253.59999999997	75.312577671634\\
9255.69999999999	75.3160180643488\\
9256.99999999986	75.3184042518715\\
9259.10000000001	75.3142820114049\\
9264.60000000011	75.3170636933738\\
9265.10000000011	75.3140784872426\\
9266.69999999998	75.3118660163163\\
9267.79999999992	75.3147962321562\\
9268.49999999989	75.3123449064584\\
9270.80000000014	75.3098404685629\\
9271.49999999986	75.3117045601622\\
9274.19999999992	75.3091877554101\\
9276.6999999999	75.307218006209\\
9279.3	75.3055154428088\\
9280.90000000002	75.3033078332076\\
9281.40000000013	75.3014060227334\\
9282.60000000007	75.2992125136006\\
9283.89999999996	75.3015941404567\\
9284.4	75.2996930367817\\
9285.80000000004	75.2969555993495\\
9286.60000000009	75.2990836357371\\
9286.99999999997	75.296917229275\\
9288.3999999999	75.2941809764763\\
9293.90000000004	75.2958898213901\\
9294.2	75.2934594321251\\
9294.99999999994	75.2912825036847\\
9297.09999999996	75.2893344232672\\
9299.69999999992	75.2865653024796\\
9300.40000000006	75.2884253534756\\
9305.50000000007	75.2901478679505\\
9308.29999999993	75.2922091874006\\
9310.69999999987	75.288911801349\\
9313.69999999996	75.28720822865\\
9316.89999999989	75.291402812064\\
9317.89999999987	75.2876153681047\\
9319.2000000001	75.2899895914929\\
9321.29999999994	75.2880468599137\\
9322.20000000008	75.2861418319513\\
9323.99999999986	75.2834053688828\\
9325.49999999986	75.2852363386806\\
9327.0999999999	75.2873316751008\\
9327.59999999999	75.2843680650107\\
9331.70000000009	75.2877258406738\\
9332.60000000008	75.2858229665584\\
9334.80000000001	75.2884337271958\\
9337.8	75.285663800212\\
9341.09999999998	75.2836894617394\\
9342.09999999995	75.2809830660872\\
9345.30000000001	75.2851670340488\\
9345.7000000001	75.2830148301911\\
9347.29999999999	75.2872456511971\\
9351.2000000001	75.288997251719\\
9351.79999999997	75.2863054566452\\
9353.60000000009	75.2889230999498\\
9354.00000000003	75.2857035952149\\
9354.89999999988	75.2838054516301\\
9355.59999999997	75.2856547345469\\
9356.59999999988	75.2829523229344\\
9358.29999999995	75.2853051803727\\
9359.00000000005	75.2828797640799\\
9362.40000000005	75.2811748998665\\
9362.80000000007	75.279026797253\\
9366.10000000003	75.2823984113087\\
9368.19999999987	75.2847368252511\\
9371.29999999991	75.2865100198476\\
9372.30000000001	75.2838120438735\\
9373.40000000007	75.2856457033125\\
9380.49999999993	75.2883610856448\\
9381.29999999987	75.2862046176477\\
9382.10000000007	75.2883119097866\\
9384.30000000002	75.2909083159286\\
9385.19999999988	75.2890158013063\\
9387.99999999992	75.2921251371417\\
9389.50000000009	75.293942233961\\
9390.29999999988	75.2917873573011\\
9392.00000000007	75.2898712747948\\
9394.79999999993	75.2865916614333\\
9395.30000000014	75.284713796113\\
9395.80000000008	75.2817718366522\\
9397.50000000014	75.2798586873244\\
9399.59999999996	75.2853814483441\\
9401.39999999991	75.2879859596873\\
9402.3	75.2860971666808\\
9403.09999999997	75.2881997617832\\
9405.99999999996	75.2915661113533\\
9410.10000000011	75.289579392574\\
9411.29999999998	75.2927300932911\\
9412.10000000007	75.2905803106606\\
9412.60000000006	75.287643290448\\
9414.6999999999	75.2846581977312\\
9417.09999999996	75.2824618782653\\
9417.79999999995	75.2842990475584\\
9419.00000000008	75.2863861727766\\
9420.70000000008	75.288722826087\\
9421.8000000001	75.2905464927456\\
9424.30000000004	75.2886125376682\\
9425.89999999997	75.2928071292171\\
9429.70000000001	75.2910984326285\\
9430.49999999996	75.2931944945178\\
9432.10000000008	75.2952651555311\\
9433	75.2933818150979\\
9434.30000000012	75.296786229119\\
9434.79999999989	75.2949156853809\\
9436.00000000002	75.2969977003211\\
9438.10000000005	75.2950774512089\\
9438.9000000001	75.2971713105202\\
9442.09999999987	75.3002478236004\\
9442.49999999993	75.2981170440345\\
9443.20000000012	75.2999481113594\\
9445.20000000007	75.3020020539316\\
9447.30000000013	75.3000825623981\\
9450.49999999988	75.3052716229657\\
9450.90000000015	75.3031425246006\\
9453.09999999996	75.3088901112851\\
9454.29999999988	75.3067354882383\\
9455.90000000011	75.3045685279188\\
9456.70000000012	75.3066576431774\\
9459.39999999998	75.308420106771\\
9459.89999999988	75.3065539112051\\
9461.89999999989	75.3096596913972\\
9462.20000000006	75.3072720163174\\
9465.60000000001	75.3055769779308\\
9466.9000000001	75.3026301890778\\
9469.50000000013	75.2999070710484\\
9470.70000000006	75.3019808252735\\
9471.70000000009	75.2993095293397\\
9473.59999999987	75.3010967203944\\
9475.49999999992	75.3039385368736\\
9477.10000000007	75.3017768961297\\
9479.00000000015	75.303562574506\\
9480.99999999993	75.3066627290082\\
9481.89999999996	75.3047880194052\\
9485.4000000001	75.3023035158927\\
9487.29999999986	75.3040875266143\\
9488.80000000011	75.3069375796984\\
9492.10000000002	75.3049872526917\\
9494.19999999991	75.3030765827918\\
9496.20000000013	75.3009066689132\\
9499.70000000008	75.2984273353123\\
9501.29999999991	75.3025869871808\\
9502.6999999999	75.3051732121059\\
9503.89999999992	75.3072390572391\\
9505.00000000015	75.3090446181524\\
9505.80000000011	75.306914653005\\
9506.50000000013	75.3087328803147\\
9507.00000000004	75.3068759137907\\
9507.80000000004	75.3047465791605\\
9508.50000000003	75.3065645836401\\
9508.89999999993	75.3044484172889\\
9510.19999999987	75.302566690851\\
9512.19999999998	75.3004005340454\\
9513.70000000014	75.3021926044272\\
9514.20000000016	75.298235287935\\
9516.70000000011	75.2963180901143\\
9521.69999999993	75.2945871578903\\
9525.40000000009	75.2926355571886\\
9527.70000000016	75.2944016457105\\
9531.19999999986	75.2961295940743\\
9531.70000000006	75.2942781006735\\
9534.59999999999	75.2913043934261\\
9537.29999999996	75.2930568079351\\
9539.19999999989	75.2969295440965\\
9543.59999999988	75.300983895135\\
9546.40000000003	75.2988006075525\\
9547.50000000001	75.3016464870753\\
9551	75.3044151982494\\
9553.50000000003	75.3014570423715\\
9554.59999999987	75.3032538959884\\
9556.10000000006	75.3050375672338\\
9556.5	75.3029320051064\\
9559.10000000016	75.3054648924596\\
9560.39999999993	75.3035929083207\\
9561.19999999995	75.3056592722747\\
9562.79999999985	75.3035167156407\\
9565.29999999999	75.307880485918\\
9565.79999999987	75.3060349784129\\
9566.69999999992	75.304176945269\\
9569.50000000015	75.3072228724294\\
9571.60000000011	75.3042824158718\\
9572.40000000006	75.3021676677984\\
9574.39999999998	75.3041934304663\\
9575.59999999987	75.3020666896415\\
9576.29999999988	75.3038720187127\\
9577.79999999989	75.3056515520104\\
9579.49999999989	75.3079460520272\\
9583.10000000009	75.3109608481509\\
9583.50000000012	75.3088609708252\\
9584.4999999999	75.3114370970098\\
9584.99999999992	75.3095951007293\\
9585.49999999994	75.3077532966116\\
9586.30000000014	75.3098139030293\\
9594.89999999988	75.3121417404898\\
9596.60000000002	75.3154730271864\\
9598.69999999988	75.3135808642747\\
9603.20000000013	75.3105703247842\\
9604.09999999993	75.3087191020595\\
9606.09999999997	75.3107368158065\\
9610.30000000012	75.3132023641055\\
9611.09999999988	75.3110953887132\\
9613.19999999994	75.313367938169\\
9614.70000000008	75.315139160461\\
9616.99999999994	75.3168834679893\\
9620.19999999991	75.3188570003014\\
9620.89999999999	75.3144163808336\\
9623.30000000012	75.3164162354261\\
9623.70000000009	75.314324902845\\
9627.7000000001	75.3162716300713\\
9629.8000000001	75.3143854037944\\
9631.29999999985	75.3161534148722\\
9635.00000000005	75.314215732063\\
9636.19999999997	75.3172898311593\\
9638.69999999997	75.3153919575051\\
9640.80000000011	75.3197315603315\\
9642.1	75.3178735143432\\
9643.3999999999	75.3201638409291\\
9645.89999999986	75.322413435621\\
9647.59999999993	75.3246887859282\\
9652.00000000013	75.322468685571\\
9654.40000000013	75.3244600963281\\
9656.29999999992	75.322066194441\\
9657.29999999988	75.3246215337461\\
9657.70000000005	75.3225372237984\\
9658.80000000003	75.32431229229\\
9659.20000000006	75.3222283188223\\
9659.99999999985	75.3242720054658\\
9661.40000000011	75.3216374269006\\
9662.30000000006	75.3197963238947\\
9663.19999999997	75.3220949365124\\
9664.90000000016	75.3202276254527\\
9668.50000000007	75.3180398403078\\
9671.8999999999	75.3143093465674\\
9672.60000000002	75.3160958160596\\
9672.89999999989	75.3137599503773\\
9675.19999999998	75.3113598544748\\
9681.39999999999	75.3096111139803\\
9682.30000000007	75.3077749318351\\
9683.50000000014	75.3056714445041\\
9683.99999999992	75.3038485765327\\
9684.80000000004	75.3058885481523\\
9685.99999999999	75.3037858374371\\
9688.39999999985	75.3057748877535\\
9689.80000000012	75.3031507033096\\
9691.00000000007	75.3010494164749\\
9694.70000000004	75.29912943021\\
9695.50000000007	75.301167539915\\
9695.90000000003	75.2990924092409\\
9696.80000000002	75.2972599490559\\
9699.69999999997	75.2953669147818\\
9702.59999999997	75.29347501211\\
9704.20000000009	75.2965180383953\\
9705.10000000013	75.2946873840828\\
9706.09999999999	75.2972326966269\\
9706.60000000005	75.2954145075051\\
9707.29999999993	75.2971959536024\\
9708.9999999999	75.2953414837627\\
9711.10000000009	75.2996540077436\\
9711.60000000015	75.2968069442013\\
9713.20000000016	75.2988170858514\\
9716.50000000009	75.2969145585905\\
9719.70000000009	75.2988744624375\\
9721.89999999992	75.2962353425221\\
9723.89999999998	75.2982311805841\\
9724.69999999994	75.2961500493583\\
9726.70000000009	75.2981453304273\\
9727.99999999993	75.2952786258365\\
9729.20000000014	75.2931865601842\\
9735.70000000007	75.2953018755521\\
9737.30000000004	75.2973072894202\\
9738.09999999985	75.2952290977799\\
9742.59999999991	75.2974021575128\\
9744.10000000016	75.2991523162497\\
9746.19999999997	75.2962662754071\\
9751.20000000004	75.2986781249679\\
9752.39999999991	75.300692130223\\
9753.00000000009	75.2970850293753\\
9754.60000000003	75.3001117410069\\
9760.79999999989	75.2973598746017\\
9763.09999999986	75.3001065224517\\
9770.79999999989	75.302172778352\\
9774.49999999988	75.3043602807276\\
9776.70000000012	75.3068488667049\\
9777.6000000001	75.3029853646563\\
9780.10000000005	75.3052084824441\\
9782.19999999995	75.303354016949\\
9783.19999999987	75.3058783845942\\
9784.79999999992	75.3037844024977\\
9785.60000000005	75.3017157689281\\
9786.69999999994	75.303469979973\\
9787.19999999987	75.3016664452913\\
9788.80000000009	75.299574007294\\
9791.9999999999	75.2964124140889\\
9793.70000000005	75.2935530641835\\
9795.69999999998	75.2955348210457\\
9797.60000000005	75.293181052696\\
9799.30000000006	75.2903238973815\\
9801.70000000002	75.2882123691567\\
9802.20000000015	75.2864123726064\\
9804.30000000011	75.2886459140794\\
9805.19999999985	75.2868346710453\\
9807.80000000016	75.2893075989763\\
9808.69999999987	75.2874969415219\\
9809.09999999985	75.2854463157036\\
9809.90000000005	75.2833843017329\\
9810.79999999991	75.2856516731391\\
9813.00000000009	75.2830400179352\\
9813.70000000011	75.2797081660519\\
9816.2999999999	75.2821808402266\\
9818.89999999989	75.2856706385579\\
9821.49999999997	75.2881404251853\\
9824.00000000002	75.2862857666351\\
9825.90000000016	75.288011398331\\
9827.49999999986	75.289999592983\\
9828.49999999986	75.2874264900393\\
9829.79999999995	75.2856081954038\\
9832.29999999999	75.2837557463081\\
9833.90000000007	75.2857433394346\\
9834.90000000001	75.2831723436706\\
9836.29999999993	75.2856736204302\\
9837.5	75.2836057575018\\
9838.30000000005	75.2856155472435\\
9838.79999999992	75.2838223785179\\
9842.10000000016	75.2860132897116\\
9845.7999999999	75.2881910236748\\
9848.70000000015	75.2863292990009\\
9849.99999999991	75.2845148780216\\
9851.10000000002	75.2862595419847\\
9852.09999999994	75.2836929822781\\
9854.10000000017	75.2856649956364\\
9854.89999999998	75.2836123795028\\
9857.69999999986	75.2896183732679\\
9858.09999999996	75.2865634700047\\
9859.90000000009	75.290060851927\\
9861.39999999997	75.2877351315723\\
9863.10000000005	75.2859112661205\\
9865.29999999995	75.2883816165589\\
9865.69999999996	75.2863427192929\\
9867.00000000002	75.2845314226064\\
9868.90000000002	75.2862498733408\\
9870.99999999987	75.2884683571233\\
9877.89999999994	75.2864952419518\\
9879.10000000016	75.2884848975626\\
9879.50000000015	75.2864488440828\\
9883.10000000009	75.2883681398737\\
9883.59999999998	75.2865829598227\\
9884.30000000013	75.2883331309943\\
9885.99999999991	75.2915709936173\\
9887.50000000004	75.2892511833003\\
9889.00000000001	75.2909769342003\\
9891.40000000006	75.2939392407623\\
9891.70000000013	75.2916557148345\\
9893.40000000003	75.2938798200839\\
9895.4000000001	75.2958415441362\\
9899.10000000008	75.2939631485373\\
9899.90000000016	75.2919191919192\\
9901.09999999985	75.2939037692401\\
9906.00000000014	75.2919918030305\\
9907.49999999986	75.2896766118939\\
9910.40000000009	75.2918621663892\\
9911.20000000012	75.289820709695\\
9911.99999999994	75.2918150543275\\
9912.69999999993	75.2895246549915\\
9915.90000000008	75.2874142799516\\
9917.40000000002	75.2901436854046\\
9919.60000000007	75.2925995745839\\
9923.70000000013	75.2907152502066\\
9926.39999999995	75.2883695159422\\
9927.10000000002	75.2860826819244\\
9927.60000000017	75.2843055289745\\
9928.30000000005	75.2860481044277\\
9929.80000000014	75.2877672484114\\
9930.59999999993	75.2857301096599\\
9931.90000000015	75.2889649617398\\
9932.39999999995	75.2871885225271\\
9933.40000000001	75.2846428751195\\
9934.49999999999	75.2863728786262\\
9935.60000000007	75.2830701410067\\
9939.60000000006	75.2809440928801\\
9942.79999999992	75.282865160064\\
9943.10000000005	75.2805937726285\\
9944.40000000009	75.2838252300267\\
9945.19999999986	75.2817913989523\\
9945.80000000012	75.279260800933\\
9948.10000000015	75.2829657626505\\
9950.50000000002	75.2849074427673\\
9950.99999999985	75.2831345278411\\
9952.19999999986	75.2810908031309\\
9953.30000000003	75.2828179315611\\
9953.8000000001	75.2810456203096\\
9956.7999999999	75.2864847492694\\
9958.40000000014	75.2844303861023\\
9961.79999999986	75.2868428713398\\
9962.70000000015	75.2840566908901\\
9963.40000000015	75.281778491494\\
9964.29999999984	75.2799967885673\\
9965.09999999989	75.2819812949063\\
9965.59999999989	75.2802111241559\\
9967.60000000001	75.2831646217282\\
9969.0999999999	75.2848774224612\\
9971.30000000011	75.2873217401769\\
9972.29999999996	75.2847860093859\\
9972.70000000016	75.2827691320391\\
9974.30000000013	75.2867340391402\\
9974.79999999987	75.2849652628096\\
9975.19999999986	75.2829488837428\\
9975.70000000012	75.2811804567052\\
9976.99999999986	75.2844012789288\\
9977.40000000011	75.2823853670759\\
9980.30000000011	75.2805498777604\\
9983.99999999994	75.2847026772568\\
9985.2000000001	75.2866714069682\\
9986.40000000007	75.2846342562459\\
9988.09999999998	75.2828337438177\\
9989.70000000016	75.284790486296\\
9991.50000000007	75.2872412826775\\
9993.90000000001	75.2901741044627\\
9994.39999999986	75.2884086247436\\
9999.60000000002	75.2862585877576\\
10000.3000000001	75.2879884804608\\
10000.7999999999	75.2862242398184\\
10004.1	75.2843805601647\\
10005.6000000001	75.2860869304496\\
10006.9	75.2843009893075\\
10008.1999999999	75.2825155121249\\
10012.2	75.284400187769\\
10014.6000000001	75.2863290962285\\
10017.0000000002	75.2842639087161\\
10020.4	75.2766828002595\\
10022.4999999999	75.2738810288747\\
10024.0000000002	75.2765834339242\\
10028.8999999999	75.2747033602553\\
10034.2	75.2768005740311\\
10034.6	75.2747964562967\\
10036.6	75.2767343848078\\
10037.4999999999	75.2739698732765\\
10038	75.2722128689692\\
10039.1	75.2749223045661\\
10042.9	75.2773075774171\\
10044.1999999999	75.2795117628904\\
10044.7000000001	75.2777556546671\\
10049.0999999998	75.2796242486964\\
10049.4000000001	75.2773769839295\\
10050.3	75.2795908620552\\
10051.9	75.2775567051333\\
10054.4	75.2757471778805\\
10056.0999999999	75.2779379885046\\
10062.2999999999	75.2802512323104\\
10066.6999999999	75.2781420113641\\
10067.5000000001	75.2761333386309\\
10070.4000000002	75.2812670671764\\
10072.1000000001	75.2794821389567\\
10072.9000000002	75.2814454482279\\
10074.6000000001	75.2846238597675\\
10077.9	75.2867632466759\\
10078.3999999999	75.2850126506921\\
10079.9999999999	75.2800071427863\\
10080.5000000001	75.2782572465925\\
10082.5000000001	75.2801856663956\\
10083.8000000001	75.2784141056536\\
10087.4000000001	75.2802973977695\\
10088.6999999999	75.2785266830545\\
10090.9	75.2809434149242\\
10091.8000000001	75.2791842963168\\
10092.6999999999	75.2813887127457\\
10093.9	75.2793738854765\\
10098.4999999999	75.2827124551918\\
10102.6000000002	75.2877943520049\\
10105.4999999999	75.285980050665\\
10107.4000000001	75.283700222607\\
10108.8999999999	75.2814323869819\\
10110.1999999999	75.2836216531656\\
10112.0000000001	75.2801099672669\\
10112.6999999999	75.2818210584606\\
10114.4000000001	75.2790548222848\\
10115.8000000002	75.2824761019781\\
10116.2999999999	75.2807322763038\\
10119.6	75.2858286312835\\
10120.1999999999	75.2833414029228\\
10121.9999999999	75.2857608599006\\
10124.8000000001	75.2827188416676\\
10126.8000000001	75.2846379444845\\
10135.6000000001	75.289323875016\\
10136.5999999999	75.2858425325796\\
10137.4000000002	75.2877928483354\\
10139.5	75.2899522663616\\
10141.4	75.2926095745205\\
10144.1999999999	75.294500359808\\
10147.8000000002	75.2963667359749\\
10149.9	75.2985221674877\\
10150.4	75.2967834096843\\
10151.3000000001	75.2950331973915\\
10151.9999999999	75.2967366357699\\
10152.4999999999	75.2940133561846\\
10154.0000000001	75.2917540697846\\
10154.5	75.2900163472712\\
10156.9999999999	75.2882220318792\\
10157.9	75.2864737152983\\
10159.0999999999	75.2884085361052\\
10161.1	75.29032004094\\
10168.9000000001	75.2886222834104\\
10170.2000000001	75.2868646942568\\
10172.4	75.2833620054067\\
10175.7	75.2854812398042\\
10176.8	75.2832394933624\\
10177.7000000002	75.2854251409931\\
10178.1999999999	75.2836917756403\\
10178.9	75.285391492288\\
10180.5	75.2873111604424\\
10180.9999999999	75.2855781791751\\
10182.4999999999	75.2892188635515\\
10182.7999999998	75.2870007561696\\
10185.2999999999	75.2911029512832\\
10186.4000000001	75.2888627104501\\
10189.5	75.2924550522101\\
10190.4	75.2907119375889\\
10196.5000000002	75.2888217641175\\
10196.9	75.2868490732568\\
10201.1000000001	75.2842802807513\\
10203.1000000001	75.2861847263604\\
10205.5999999998	75.2843998941768\\
10207.2000000001	75.2872943873502\\
10208.0999999999	75.2845751454713\\
10209.3000000001	75.2874801653378\\
10214.6000000001	75.2856177861318\\
10217.1	75.2838351015934\\
10218	75.2860120766092\\
10219.2999999999	75.2842632639881\\
10221.7000000002	75.2822399186053\\
10224.1000000001	75.2860859529352\\
10224.5999999999	75.2843604213327\\
10226.7000000001	75.2865021316541\\
10228	75.2847547442829\\
10228.8	75.2827772292231\\
10229.9999999999	75.2807890440954\\
10232.5	75.2790102222309\\
10234.4	75.2767599785041\\
10236.8	75.274741376784\\
10238.5000000001	75.2768933252593\\
10242.5999999999	75.2799554804886\\
10245	75.2818420513221\\
10245.4999999999	75.2801202467401\\
10249.4999999998	75.2829378707462\\
10255.0000000001	75.285467718501\\
10261.1	75.2884652867111\\
10264.7999999999	75.2866564701069\\
10267.6	75.2895000827839\\
10271.2000000002	75.2923193753468\\
10273.1999999999	75.2893422756076\\
10274.9999999999	75.2917246547479\\
10275.4999999999	75.2900073961618\\
10276	75.2882903046876\\
10278.5999999999	75.290649595766\\
10281.2	75.2949529728731\\
10281.5000000001	75.2927559912854\\
10284.4000000001	75.2948611988915\\
10285.2000000002	75.2919214801707\\
10290.2999999999	75.2944491953666\\
10294.8000000001	75.2926206179759\\
10295.6	75.2945404392125\\
10297.5000000001	75.2923011187073\\
10298	75.2905875841175\\
10299.8	75.2929640093593\\
10301.3	75.290737181354\\
10304.8000000001	75.2884550068414\\
10305.8999999999	75.2910925674365\\
10306.7	75.2891295067334\\
10310.7000000002	75.2909570547387\\
10316.7	75.293695719603\\
10318.4	75.2919513495179\\
10322.4	75.2937757326229\\
10323.6999999999	75.2959181696662\\
10326.1999999999	75.2941518259202\\
10329.9	75.296224588577\\
10333.5000000001	75.29805682434\\
10335.1	75.2999458162396\\
10336.1000000001	75.2974981134266\\
10336.9000000002	75.2994098868144\\
10338.7999999999	75.3029819419861\\
10341.9000000001	75.3007155289112\\
10343.3000000002	75.2982578262467\\
10344.2000000001	75.300406987423\\
10345.5000000001	75.2986776987318\\
10352.4999999999	75.3008905975311\\
10353.5	75.2984469170144\\
10354.5	75.3008324802503\\
10355.2999999999	75.2988778801398\\
10355.8000000001	75.2971735918655\\
10356.9999999999	75.2990702030491\\
10357.7999999999	75.2971162108149\\
10360.0000000002	75.2994662213685\\
10362.4	75.297466827503\\
10367.7000000002	75.2956268446536\\
10371.4	75.298654967941\\
10378.6	75.3013383179011\\
10381.0000000001	75.3031952297926\\
10385.2999999999	75.3057176420745\\
10386.1999999999	75.3040062389879\\
10386.8000000001	75.3015818001521\\
10390.2999999999	75.2993147520788\\
10391.9	75.3011932255581\\
10395.6	75.3042123185548\\
10399.5999999999	75.3060184428397\\
10405.9999999999	75.3039082845639\\
10408.9999999999	75.3062224399804\\
10411.8999999999	75.3044563964656\\
10413.7000000001	75.3020031112562\\
10415.1000000001	75.2986020431677\\
10416	75.2968961511506\\
10417.4999999999	75.3004530794041\\
10421.0000000001	75.3039506386082\\
10421.9000000001	75.3022452504318\\
10426.0000000002	75.3052435714217\\
10428.4	75.30325550175\\
10430.8000000001	75.3060617971604\\
10432.1	75.3043461590077\\
10436.0000000001	75.3068675079771\\
10436.4000000002	75.3049393953912\\
10437.4	75.3073053892216\\
10439.8999999999	75.3045977011494\\
10442.1	75.3098006167283\\
10443.1000000001	75.3073770491803\\
10446.3999999999	75.3056047480017\\
10447.2	75.3074957165966\\
10448.2999999999	75.3043528195705\\
10449.7999999999	75.306940736275\\
10454.7000000001	75.3051230057007\\
10456.7000000001	75.3069772779435\\
10457	75.3048168230198\\
10458.8999999999	75.3073907639354\\
10459.3	75.3054668527831\\
10461.4	75.3037327343115\\
10463.8000000001	75.3055744034251\\
10464.5000000002	75.3034038568125\\
10465.5000000001	75.3009860877542\\
10468.2	75.303535435553\\
10469.8999999999	75.305635148042\\
10471.5999999999	75.3039143596551\\
10474.5999999999	75.3062140204493\\
10478.9999999998	75.3089482875438\\
10479.7999999998	75.3070162883234\\
10484.7000000001	75.3052037234854\\
10488.6999999999	75.3031805354283\\
10489.5999999999	75.305299484256\\
10490.1000000001	75.3026634382567\\
10493.0999999999	75.3001944116189\\
10496.2999999999	75.3029610152052\\
10496.6	75.3008088256309\\
10498.1999999999	75.3026680510178\\
10500.3	75.3009409165365\\
10501.3	75.3032928942808\\
10502.9000000001	75.3061030181853\\
10506.2	75.3043412048009\\
10508.9	75.3021219906747\\
10511.7999999999	75.300373862004\\
10513	75.2984371878894\\
10513.3999999999	75.2965235173824\\
10519.2000000001	75.2987366079492\\
10523.7	75.300746878504\\
10524.9	75.2988123515439\\
10527.8000000001	75.3046666476695\\
10528.1000000001	75.3025208487681\\
10529.3000000002	75.3043858149562\\
10530.5000000001	75.3024519020758\\
10531.5000000001	75.2990998518744\\
10532.9999999999	75.2969211343289\\
10535.8000000001	75.2987404967777\\
10538.0999999999	75.301284849405\\
10538.4	75.2991412440101\\
10541.2	75.3009590847429\\
10542.7000000002	75.2987821072201\\
10546.2	75.3012904999858\\
10548.0999999998	75.2981551354734\\
10553.3999999999	75.2963471833989\\
10557.1000000001	75.2993217898685\\
10557.5000000001	75.297416079412\\
10560.3000000001	75.3001780235597\\
10561.1000000001	75.2982615611862\\
10562.1999999999	75.2961002811888\\
10564.6000000001	75.3007657576647\\
10566.6	75.3026015690802\\
10568.5000000001	75.3051492156009\\
10571.7	75.3031650239316\\
10573.4	75.3061900033102\\
10574.3999999999	75.3037968698284\\
10575.6000000001	75.3056535264805\\
10581.7000000001	75.3076036213121\\
10583.3	75.3056673658749\\
10586.2000000001	75.3029859346514\\
10589.2000000001	75.3052609709801\\
10592.3000000002	75.30871190665\\
10595.4	75.311217026096\\
10600.5999999999	75.3091776958126\\
10607.1000000002	75.3111094351007\\
10607.4000000001	75.3089794956399\\
10609	75.307047723181\\
10610.6999999999	75.3053492667848\\
10614.6999999999	75.3033500395674\\
10620.9999999999	75.3010516801461\\
10623.8	75.2990897881192\\
10624.5	75.2969523558534\\
10626.8000000001	75.3004168666309\\
10628	75.3022647509903\\
10629.0999999998	75.3001166597674\\
10633.0999999998	75.3018846631306\\
10639.8000000001	75.3042791755562\\
10640.1000000002	75.3021559745117\\
10642.9999999999	75.3060668414278\\
10649.7999999999	75.3086883444915\\
10651.8999999999	75.3107397671799\\
10654.7000000001	75.3144122836656\\
10655.5	75.3125117309208\\
10656.9000000001	75.3148165525007\\
10658.1	75.3129046180406\\
10659.8000000001	75.3149654311954\\
10664.7000000002	75.316930462831\\
10666.6999999999	75.3140585742678\\
10668.4000000001	75.3161175422974\\
10670.8	75.3188578282994\\
10671.2000000001	75.316971690422\\
10673.2999999999	75.3152697359792\\
10677.6	75.3177182352005\\
10680.9000000001	75.3159816496583\\
10686.7000000002	75.3181494928323\\
10690.1000000002	75.3203868964098\\
10693.5000000002	75.317947183362\\
10694.6999999999	75.3197815760931\\
10695.7000000001	75.3174143121599\\
10697	75.319479111161\\
10701.0999999999	75.3167869024035\\
10704.2999999998	75.3148238107694\\
10706.4	75.3187316116378\\
10706.9999999999	75.3163788514164\\
10708.7000000002	75.3137606454505\\
10710.2999999999	75.3118464296385\\
10711.4000000002	75.3097138589367\\
10713.7999999999	75.3124445813383\\
10716.6000000002	75.3142291936883\\
10719.5	75.3171760140304\\
10722.6999999998	75.3198791360466\\
10723.1000000001	75.3180020889287\\
10726.3000000002	75.3160426610979\\
10736.7	75.319462037106\\
10737.9000000002	75.3166325200224\\
10739.8000000002	75.3191370496932\\
10740.3999999999	75.3167915832596\\
10741.7000000001	75.319778808021\\
10745.4999999999	75.317339189994\\
10749.0999999999	75.3200238157258\\
10753.2999999998	75.3175739766027\\
10754.3	75.319869076843\\
10755.0000000002	75.3177562272782\\
10759.9999999999	75.3199319708925\\
10765.6	75.3216233036403\\
10767.4	75.3183190155561\\
10768.2	75.3201526703379\\
10770.9999999999	75.3219262656553\\
10772.6999999999	75.324892321402\\
10777.9	75.3265912043051\\
10779.5000000001	75.3237596942373\\
10780.6999999999	75.3265063817156\\
10781	75.3244103106362\\
10783.6	75.3275777330601\\
10785.2	75.3256747610173\\
10786	75.3237963675471\\
10789.2999999998	75.3276363838582\\
10791.6999999999	75.3257102614948\\
10792.7999999999	75.3282250368298\\
10795.7000000002	75.3265158672817\\
10797.2000000002	75.3290174395451\\
10801.2000000001	75.3307472248711\\
10806.3000000001	75.3349866745632\\
10807.0000000001	75.3319576944786\\
10807.8000000001	75.33378362124\\
10809.0999999999	75.3358250379306\\
10810.1999999999	75.3337095177747\\
10811.4000000001	75.3355223604495\\
10812.3000000001	75.3320261921498\\
10816.5000000001	75.3342085313315\\
10817.3	75.3295616321852\\
10820.9999999999	75.3278317361451\\
10825.7999999999	75.3258389602712\\
10829.0000000001	75.3285129881523\\
10833.0000000001	75.3311609788519\\
10835.9999999999	75.3287621930399\\
10838.5	75.3316849039544\\
10840.0000000001	75.3286408796967\\
10841.3999999999	75.3309043951483\\
10841.8000000002	75.3281251441168\\
10843.9000000001	75.331980818886\\
10844.3000000002	75.3292021688614\\
10846.3000000001	75.3263755716182\\
10848.5999999999	75.3242323965083\\
10852.6000000002	75.3268771826366\\
10854.1000000001	75.3293655911997\\
10857.0000000001	75.331349992171\\
10859.7999999999	75.329422922863\\
10860.7	75.3314672952269\\
10862.9	75.3337015557397\\
10864.5000000002	75.3318115715259\\
10866.8999999999	75.333578724579\\
10867.2999999999	75.3317260798351\\
10868.9000000002	75.3335173429018\\
10873	75.3317821044596\\
10874.2	75.3335846905088\\
10881.7999999999	75.3361085839789\\
10882.5	75.3340194438829\\
10883.6	75.336512399276\\
10890.8000000002	75.3390445234095\\
10892.5999999999	75.3412836119603\\
10897.9999999999	75.3452436663271\\
10900.8	75.3469896980983\\
10901.4999999998	75.3449035004036\\
10902.4	75.3469387755102\\
10905.6999999999	75.3452291441251\\
10909.0999999999	75.3474131925347\\
10911.1000000002	75.3455165334702\\
10912.7	75.3472985851477\\
10913.5000000001	75.3454405512388\\
10914.5000000002	75.347699411797\\
10917.2000000001	75.3446364943713\\
10918.1999999999	75.3423151955891\\
10924.6999999999	75.3441710603398\\
10925.8000000001	75.3420770828948\\
10930.7000000001	75.3439821422037\\
10934.9000000002	75.3415637860082\\
10938.2000000001	75.3444319501202\\
10938.5	75.3423655678057\\
10939.7000000001	75.3441562002962\\
10941.7	75.3459211464293\\
10943.6999999999	75.347685447468\\
10945.9000000001	75.3508130824045\\
10950.9	75.353848963565\\
10952.6999999999	75.3560733328464\\
10954.1	75.352832703438\\
10956.8000000001	75.3506922578466\\
10958.0000000001	75.3488287203074\\
10960.8000000001	75.3505642784808\\
10962	75.3477892009743\\
10963.1999999999	75.3459268650862\\
10964.3000000001	75.3438400642078\\
10965.4999999998	75.3456263223171\\
10967.5999999998	75.349435159605\\
10969.3000000002	75.3468740313964\\
10978.3000000001	75.3488668658457\\
10983.2999999998	75.3518946774223\\
10983.6999999999	75.3500609989257\\
10991.7000000001	75.3479866809804\\
10992.7	75.3502292409577\\
10996.7999999999	75.3475979594249\\
11012.9999999998	75.3493566752322\\
11013.2999999998	75.3473041930739\\
11014.0000000001	75.3452392841903\\
11014.9999999998	75.3474775535401\\
11021.1	75.3493267520778\\
11022.3000000001	75.3474742342866\\
11024.6999999999	75.3492126841303\\
11031.7	75.3467249224968\\
11035.3000000001	75.3493303369157\\
11038.9	75.3510281728417\\
11039.7	75.3491911085346\\
11041.4000000001	75.3466467418376\\
11042.6999999999	75.348643459992\\
11050.7	75.3456763311253\\
11051.9	75.3483532392327\\
11059	75.3506162345941\\
11059.7000000001	75.3485596484566\\
11061.3000000002	75.3467011409044\\
11062.5000000001	75.348471426247\\
11065.3000000001	75.3501906844759\\
11069.8	75.3484674658308\\
11070.7999999999	75.3461778175216\\
11072.2	75.348391933022\\
11074.6	75.350122350944\\
11076.5999999998	75.3518647250535\\
11082.1000000001	75.3496598148382\\
11090.1999999999	75.3550399899011\\
11092.2999999999	75.3515920810645\\
11093.3000000001	75.3493067950313\\
11094.2000000001	75.3467997079581\\
11098.2000000001	75.3502788715389\\
11104.4000000001	75.3532351749291\\
11104.8	75.3514214445875\\
11107.6000000001	75.3495323064181\\
11108.5999999999	75.3517513300386\\
11109.4	75.3499257392322\\
11117	75.3532845796116\\
11123.1999999998	75.3562342110705\\
11124.2	75.3539548555864\\
11129.0999999999	75.3558207238616\\
11131.1	75.3575535431939\\
11132.7	75.3592986490371\\
11135.4999999999	75.3574122633715\\
11136.5000000001	75.3596250202037\\
11138.1	75.3613689824209\\
11139.4000000001	75.3588581175098\\
11140.5999999999	75.3615122927644\\
11148.1000000001	75.3637358497336\\
11149.9999999999	75.3661402139891\\
11150.7999999999	75.3643203687595\\
11152.5	75.3671789537866\\
11153.9999999999	75.3695950368026\\
11156.8	75.3677096684563\\
11158.8999999999	75.370552916928\\
11171.9	75.3687790905836\\
11176.1	75.3708774001897\\
11176.5000000002	75.3681799473901\\
11177.6999999999	75.3663511603357\\
11179.0000000001	75.3638486103532\\
11180	75.366052182002\\
11181.6000000002	75.3677884400404\\
11182.9999999999	75.369977913101\\
11183.6999999999	75.3679429174341\\
11184.5000000002	75.3652343400747\\
11186.2999999999	75.3629407137238\\
11188.4000000001	75.3657773606828\\
11199.8	75.3676372110465\\
11202.8000000001	75.369770327326\\
11204.7	75.3659146080251\\
11207.5	75.3676076947785\\
11209	75.3655512039325\\
11210.5999999999	75.3672830420937\\
11211.0000000001	75.3654859915619\\
11212.1999999999	75.3627712422964\\
11218	75.3648122231037\\
11219.7000000001	75.3667623308793\\
11222.3000000001	75.3644496720844\\
11227.9000000001	75.3669397933737\\
11228.5999999998	75.3649131244044\\
11231	75.3666159147368\\
11233.6000000002	75.3643056161372\\
11243.8999999999	75.3673070081821\\
11246.3000000001	75.3690069711196\\
11246.7000000001	75.3672155635381\\
11253.9999999999	75.3654223793995\\
11255.6000000001	75.3627051182956\\
11256.8	75.3608897653883\\
11261.2999999999	75.3583035856998\\
11262.9000000002	75.3609162745272\\
11264.7999999999	75.3632966115989\\
11266.4000000001	75.358807082945\\
11269.7000000001	75.3606985039663\\
11275.3	75.363180020221\\
11275.8000000001	75.3607250862459\\
11278.4000000001	75.3584253225163\\
11279.4	75.3561771355113\\
11280.2000000001	75.3543788729023\\
11282.2000000002	75.3569750848675\\
11283	75.3551772119364\\
11284.6	75.3568991643553\\
11285.7000000001	75.3548707225008\\
11286.7000000001	75.3526243044973\\
11288.3000000001	75.3543460543567\\
11291.6000000001	75.3580063232286\\
11296.6	75.3556348314109\\
11297.7	75.3580343075643\\
11298.7000000002	75.3540198959181\\
11303.3000000001	75.3560875488791\\
11305.8	75.3579989209174\\
11307.7000000002	75.3603707175578\\
11310.8000000001	75.3627032331645\\
11314.9000000002	75.3601414052143\\
11317.2000000001	75.3580801074461\\
11321.4000000001	75.3601554564325\\
11324.5999999999	75.3574046111597\\
11329.5	75.3557054088405\\
11330.9999999999	75.3536726354899\\
11332.0000000002	75.3558475481155\\
11333.4999999999	75.3538152043481\\
11336.3000000001	75.3510814720723\\
11338.6000000001	75.3534355790346\\
11342	75.3511254529584\\
11342.8	75.3493374710171\\
11343.9999999999	75.3475374864467\\
11345	75.3497104476823\\
11348.1000000001	75.3476322236126\\
11350.1000000001	75.3493330514\\
11351.8999999999	75.3523608174771\\
11355.2000000001	75.3542398703689\\
11357.4	75.3581333920317\\
11360.5999999999	75.3615534254051\\
11363.4999999999	75.3634411630117\\
11365.8000000001	75.3613880115081\\
11368.8000000001	75.3634916306767\\
11369.9000000001	75.3614775725594\\
11374.8000000002	75.3641790257497\\
11381.7999999999	75.367908697142\\
11385.0000000001	75.3660486073904\\
11390	75.368082808755\\
11396.2000000001	75.3700762528189\\
11398.6	75.368243747094\\
11400.9999999999	75.3707975546219\\
11401.5000000002	75.3683693516699\\
11408.0000000002	75.3666254678693\\
11409.1000000001	75.3646180275567\\
11410.7000000001	75.3628141760437\\
11415.4	75.3606937935264\\
11416.8	75.363715194142\\
11417.1000000001	75.3617349262516\\
11417.8999999999	75.3634612016115\\
11419.8	75.3658088074326\\
11422.3	75.3633211934444\\
11423.8000000001	75.3656807219951\\
11425.7999999998	75.3638663037485\\
11431.8999999999	75.3700139958013\\
11435.0000000001	75.3740675639041\\
11437.1	75.3715944461931\\
11441.2	75.3690577119733\\
11442.0999999999	75.3709950883571\\
11443.3	75.3727039166681\\
11445.1000000001	75.3748296228987\\
11446.2	75.3719542559604\\
11447.0999999999	75.3738905583898\\
11449.6	75.3757740377477\\
11454.9000000001	75.3784373635967\\
11457	75.3812046678479\\
11460.0000000001	75.3789233950838\\
11465.6999999998	75.3772087425212\\
11467.3000000001	75.3797722238694\\
11475.5999999999	75.3818939149681\\
11476.9000000001	75.3838111004618\\
11478	75.3809428389716\\
11480.1000000001	75.382833051689\\
11480.6000000001	75.3804210544653\\
11482.4	75.3834095362508\\
11483.9000000001	75.3814002089864\\
11485.7999999999	75.3793781941337\\
11487.1999999998	75.3815082743552\\
11489.5999999999	75.3857803075798\\
11491.2000000001	75.3831159224805\\
11494.6000000001	75.3869174489112\\
11495.0000000001	75.3851641134048\\
11498.0000000001	75.3828893469356\\
11499.4	75.3850167398583\\
11502.8000000001	75.3870763024976\\
11504.1000000001	75.3889883694651\\
11504.5000000001	75.3863671922535\\
11509.1000000002	75.3901226844611\\
11511.6000000001	75.3919924945925\\
11513.7	75.3938751758759\\
11518.7000000001	75.3958745702677\\
11520.3	75.394083538766\\
11522.7000000001	75.3922657687368\\
11526.8000000001	75.3966808075024\\
11529.2000000001	75.3948635216362\\
11532.0999999999	75.397582421394\\
11538.5	75.3999618671243\\
11541.1999999999	75.4022510462426\\
11543.5000000001	75.400221767906\\
11544.6	75.4025656794893\\
11546.1000000001	75.4005646879493\\
11547.4999999999	75.4026810765873\\
11549.6000000001	75.4045559624925\\
11550.4000000001	75.4019306523527\\
11557.6999999999	75.4036235269688\\
11560.4999999999	75.4017957545456\\
11564.0000000001	75.4040521960204\\
11564.7999999999	75.4014301896255\\
11566.5000000002	75.4033164456279\\
11569.0000000001	75.4069028705777\\
11569.7000000001	75.4049335338554\\
11570.5000000002	75.406634055278\\
11572.2999999998	75.408731118869\\
11574.7000000002	75.4069184780731\\
11578.9	75.4037481647811\\
11582.5000000001	75.4018959473693\\
11584.8000000001	75.3998739738798\\
11587.2000000001	75.4032432059237\\
11590	75.4014201775653\\
11591.8000000002	75.4035145230721\\
11592.2999999999	75.401124874918\\
11593.4999999999	75.4036709908915\\
11593.8000000002	75.4017198699316\\
11599.6	75.4053984154763\\
11607.8000000001	75.4098501882339\\
11609.0999999999	75.4117424111911\\
11612.6000000001	75.4096807805248\\
11619.1000000002	75.4079454695676\\
11622.6000000002	75.410188682492\\
11625.1999999998	75.4079464615967\\
11626.4	75.4061841482819\\
11632.3000000002	75.4083422165675\\
11639.2999999999	75.410244514322\\
11642.5	75.407555013485\\
11643.3	75.4058092996891\\
11644.3	75.4036274947614\\
11646.4000000001	75.4003348645516\\
11648.2999999999	75.4026304041757\\
11649.5999999999	75.4002248984952\\
11651.0000000002	75.4023225274867\\
11655.0999999998	75.4041114695587\\
11655.4	75.4021706490498\\
11658.5000000001	75.4052802223252\\
11658.8000000002	75.40333993773\\
11661.5999999999	75.4015280790965\\
11666.1	75.405016200648\\
11668.1	75.4075178690801\\
11668.7999999999	75.4055652203721\\
11673.5	75.4034745065789\\
11674.7999999998	75.4053567910646\\
11680.6	75.4072957956287\\
11681.8	75.4055419067104\\
11683.9999999999	75.4076052070763\\
11688.7	75.4132160700842\\
11693.3000000001	75.416046658799\\
11695.4999999999	75.4138308423681\\
11696.4999999999	75.4116580886753\\
11698.2	75.4143764478599\\
11704.9999999999	75.4166987039837\\
11709.1000000002	75.4184743620401\\
11711.4999999999	75.4158270432733\\
11712.4000000002	75.4177161152615\\
11715.1999999999	75.4159091103087\\
11717.7999999999	75.4188037105625\\
11718.1999999999	75.4170826826417\\
11721.7	75.4150386459417\\
11723.0000000001	75.4177649256596\\
11725.2000000002	75.4198186826776\\
11730.1	75.4215614397026\\
11730.8000000001	75.4196182731078\\
11732.2	75.4216990700886\\
11740.9999999999	75.4239381318616\\
11746.1999999999	75.4220477937734\\
11752.2999999998	75.4203396753004\\
11757.5999999999	75.424615358446\\
11760.0999999999	75.4221866975051\\
11764.3999999999	75.4243699264737\\
11769.6999999999	75.4260905028123\\
11770.0000000002	75.4241680189633\\
11773	75.4261834181312\\
11775.9000000001	75.4279891304348\\
11783.5	75.4302590040395\\
11785.4	75.4282805141912\\
11788.6000000002	75.4264677190869\\
11791.7000000001	75.4286877321529\\
11793.9999999999	75.4309358068865\\
11794.8000000001	75.4292109301478\\
11796.5	75.4319041079633\\
11796.7999999999	75.4299858437386\\
11798	75.4282469211144\\
11800.1999999999	75.4260484902926\\
11801.5000000001	75.4236713665944\\
11809.1000000001	75.4259390983301\\
11810.5999999999	75.423979950384\\
11813.1000000002	75.4257948735313\\
11813.4999999999	75.4240874923817\\
11818.2999999998	75.4222229743451\\
11822.7999999999	75.4239653553697\\
11827.1999999999	75.4221166284782\\
11830.2000000001	75.4198963678013\\
11834.6000000001	75.4180503096825\\
11836.7	75.4207218167072\\
11838.6999999999	75.4189613812211\\
11839.6999999999	75.4168144732175\\
11842.9	75.4209237524276\\
11843.4	75.4185840334361\\
11845.9000000001	75.4203950700659\\
11848.3999999999	75.4179853989957\\
11849.9000000002	75.4210970464135\\
11853.3000000001	75.4239289992745\\
11859.4000000002	75.422235338758\\
11861.8999999999	75.4265722475131\\
11862.6000000001	75.4246503747039\\
11864.6000000001	75.4228931199272\\
11868.8	75.4248498175905\\
11870.9	75.4266700362227\\
11873.7999999998	75.4293029249025\\
11879.3999999999	75.4274169788291\\
11881.3000000001	75.4254549127207\\
11882.4999999999	75.4237288135593\\
11884.6999999998	75.4257539041465\\
11885.3	75.4236289901896\\
11887.3	75.4218752628834\\
11890.1999999999	75.4236646678385\\
11895.6	75.4255739468884\\
11896.1999999999	75.423450988963\\
11900.2999999999	75.4268764075157\\
11902.0000000001	75.42870585863\\
11906.6999999999	75.4308462391239\\
11909.7	75.4328368234563\\
11912.1999999998	75.4354742577\\
11913.4000000001	75.4337516263063\\
11915.1	75.435578085135\\
11915.4000000001	75.4336788217028\\
11917.8999999999	75.4354757509649\\
11922.9000000001	75.4373899186446\\
11926.1000000002	75.4355955794805\\
11928.1000000001	75.4330074948441\\
11929.6999999999	75.4304347097185\\
11930.8999999999	75.4287151118934\\
11931.9	75.4307743881998\\
11935.3999999998	75.4329521176323\\
11938.4999999999	75.4359807682643\\
11940.2000000002	75.4378030702746\\
11940.8999999999	75.4350556904782\\
11941.8999999998	75.4329258080724\\
11944.8999999999	75.4349100041859\\
11946.7000000001	75.4327518666086\\
11948.1	75.4347935253846\\
11948.4000000001	75.4328995271373\\
11950.0000000002	75.4353520054225\\
11950.5999999999	75.4332382203553\\
11952.8999999999	75.4354555341755\\
11953.5999999999	75.4335477718196\\
11955.2000000001	75.4318168511037\\
11959.9999999999	75.4282990944892\\
11962.4999999998	75.4325982645913\\
11964.7000000001	75.4304292591602\\
11966.1999999998	75.4326734245339\\
11967.0000000001	75.430973251665\\
11970.2	75.4358704460206\\
11972.2	75.4341271100791\\
11975.7000000001	75.4321214449139\\
11980.9000000001	75.4344378599449\\
11983.2000000001	75.4324768636352\\
11984.3	75.4305597276459\\
11985.0999999999	75.4288622634583\\
11986.6	75.4311028056096\\
11988.8	75.4331089591205\\
11990.4	75.4313831783495\\
11991.9999999999	75.434661151925\\
11992.3000000002	75.4327740902572\\
11993.5	75.4352321237994\\
11997.5000000002	75.4325865173035\\
12003.8999999998	75.4356881039653\\
12007.2999999999	75.4376467844829\\
12012.3	75.4412107488928\\
12021.6000000001	75.439413726844\\
12023.8999999999	75.4424484364604\\
12025.0999999999	75.440741110335\\
12026.5	75.442768529759\\
12033.4999999999	75.4404334529983\\
12035.4000000001	75.442648830543\\
12039.7000000001	75.4447748301467\\
12040.9	75.443069512499\\
12042.6	75.4448753186578\\
12044.4000000002	75.4419029432521\\
12046.4	75.4401693437928\\
12049.3999999999	75.4421345284037\\
12050.6999999999	75.4398048262356\\
12052.5	75.4376649021788\\
12055.0000000001	75.4419291420229\\
12057.2000000001	75.4397750740215\\
12062.4000000001	75.4379274611399\\
12063.9000000001	75.4401525198939\\
12066.4999999999	75.4421295145277\\
12066.9999999999	75.4398322712168\\
12068.5999999999	75.4430883193716\\
12069.0999999999	75.4399628807212\\
12071.0999999999	75.4382331499768\\
12072.0999999999	75.4361259753815\\
12074.9000000001	75.4343685300207\\
12077.5	75.4363449691992\\
12077.9999999998	75.4332221127495\\
12083.2000000001	75.4355184428095\\
12083.9	75.4328037073817\\
12086.7000000001	75.4310487473938\\
12088.5999999999	75.4340830693127\\
12090.3999999999	75.4360861833671\\
12091.1000000001	75.4342000793966\\
12095.2999999999	75.4369429700547\\
12098.8000000001	75.4399160254238\\
12099.1	75.4380454906109\\
12103.6000000001	75.4430463412015\\
12106.3000000002	75.446045067072\\
12108	75.447840701679\\
12109.4999999999	75.4459271982559\\
12112.6999999999	75.4482861105607\\
12118.7000000001	75.4464138363534\\
12122.9	75.4441969809453\\
12124.5	75.4424888243736\\
12135.6999999998	75.440432439559\\
12138.1000000001	75.4386976652222\\
12140.0999999999	75.4361542643449\\
12144.3	75.4388854122064\\
12145.3000000001	75.4367908837914\\
12150.3	75.4411377403213\\
12150.5999999998	75.439275103492\\
12152.9	75.4414547848268\\
12153.9999999999	75.4387408364256\\
12155.3999999999	75.4407469869606\\
12158.0000000001	75.4427089759091\\
12161.3000000001	75.4444389626194\\
12166.9000000001	75.4425906139558\\
12175.9	75.4402102496715\\
12177.7000000002	75.4430192645634\\
12183.3000000001	75.4411740565031\\
12184.5999999999	75.4429735652088\\
12188.7000000002	75.4405683906537\\
12191.6999999999	75.4384094227267\\
12192.9000000002	75.4408267038465\\
12197.7000000002	75.4431126924527\\
12204.4000000002	75.4459420705477\\
12208.5	75.4476352734957\\
12211.7000000001	75.445880214219\\
12213.1000000001	75.4429633511283\\
12217.1999999999	75.4446563479656\\
12221.9000000001	75.448371788578\\
12225.0000000002	75.4464176162158\\
12228.1	75.4444644346674\\
12236.6000000001	75.4427255714367\\
12238.9000000001	75.4408039872539\\
12242.9999999999	75.4441277127525\\
12248.6000000001	75.4471903140741\\
12251.0999999999	75.4497518610422\\
12252.8	75.4466289613071\\
12255.5000000002	75.4446946701916\\
12256.9999999999	75.4428045785708\\
12258.5	75.4409149495048\\
12259.8999999999	75.4388254486134\\
12263.5999999998	75.4405277363275\\
12268.7999999999	75.4387108868766\\
12270.7000000002	75.4359943931936\\
12274.0000000001	75.4393397479245\\
12276.4999999999	75.4410830360197\\
12277.7999999998	75.4387965368671\\
12281.7000000002	75.4408962855607\\
12287.2999999999	75.4431368719176\\
12288	75.4412805885369\\
12289	75.4432790033444\\
12291.4	75.4415653093601\\
12295.2	75.4442754548486\\
12295.5000000002	75.442434692085\\
12299.9	75.439837398374\\
12301.9000000001	75.4373272638595\\
12303.1999999999	75.4350458819991\\
12306.6000000001	75.4369571046666\\
12307.4000000001	75.4344911639244\\
12308	75.4316263273779\\
12313.4000000002	75.4334673326024\\
12317.0999999999	75.435975708765\\
12320.8999999998	75.4378703027352\\
12325.3000000002	75.4360913236082\\
12327.6000000002	75.4382407099459\\
12330.4000000002	75.4365192003568\\
12332.5000000001	75.4382693024991\\
12334.7999999999	75.4404170281072\\
12335.6999999999	75.4381556121208\\
12340.6999999998	75.4408142097757\\
12345.2999999999	75.4426750044551\\
12346.3000000002	75.4406142681267\\
12349.1999999998	75.4423327638004\\
12351.5000000002	75.4404287703617\\
12352.4000000001	75.4422181744586\\
12354.5	75.4439641914752\\
12357.2000000001	75.4420464017221\\
12359.8999999999	75.4401294498382\\
12364.1000000001	75.4428106953948\\
12367.9	75.4406532988357\\
12372.5000000001	75.4376606372145\\
12373.8999999999	75.4396314853726\\
12378.5999999999	75.4416861221291\\
12380.5999999999	75.4440378976956\\
12381.1000000002	75.4417988563306\\
12384.0999999998	75.4396731318939\\
12386.4000000001	75.4377749969725\\
12390.3999999998	75.4360195310924\\
12392.3000000001	75.4381717827055\\
12394.6999999998	75.436473359796\\
12396.4999999998	75.438426665376\\
12399.3000000001	75.4367146797426\\
12401.9999999999	75.4348053958604\\
12404.8999999999	75.4373236598146\\
12407.9	75.4392327530625\\
12408.2000000002	75.4374088311856\\
12409.8	75.4405756694252\\
12412.1	75.4386812974332\\
12413.9	75.4406315450298\\
12419.5000000001	75.4388225063609\\
12425.6999999998	75.4357868306266\\
12428.2000000002	75.4375095548064\\
12433.4000000001	75.4357180198657\\
12434.3000000001	75.4374959788972\\
12436.4999999999	75.4394287827863\\
12439.6000000001	75.4415299404326\\
12441.8999999999	75.4396399292718\\
12444.8	75.4413454507469\\
12445.4999999999	75.4395127595295\\
12448.2	75.4376099547729\\
12455.0000000002	75.4357652688457\\
12456	75.4377373335153\\
12459.1000000002	75.4358225247207\\
12460.3	75.4333729254278\\
12464.0000000002	75.4358517662727\\
12465.3999999999	75.4337972804942\\
12469.5000000001	75.4370629370629\\
12471.7000000001	75.4349813178531\\
12480.2000000001	75.4332828537776\\
12482.5000000001	75.436207200423\\
12491.9000000001	75.4378802433557\\
12492.4999999998	75.4358580279525\\
12495.3000000001	75.4341597707956\\
12497.5999999998	75.4322795394353\\
12499.0000000001	75.4350313222552\\
12503.4999999998	75.4326753894878\\
12516	75.4348399261751\\
12518.6	75.4327526021072\\
12522.6000000002	75.4310172726329\\
12523.5999999999	75.428986641328\\
12532.5000000001	75.4264877200262\\
12533.3000000002	75.4240668932612\\
12539.4	75.4264524103832\\
12542.5000000001	75.424553122957\\
12543.6000000001	75.422722163317\\
12545.4999999999	75.424850146665\\
12548.3	75.4295368333811\\
12552.4000000002	75.431985660227\\
12553.1000000002	75.4301691998853\\
12556.9999999999	75.4322255934889\\
12558.2000000001	75.4297954340954\\
12559.6	75.4317380192202\\
12560.2999999999	75.4299226139295\\
12565.5000000001	75.4273572292608\\
12567	75.4294944736654\\
12569.7	75.431590001432\\
12577.2999999998	75.4337144401864\\
12583.1	75.4362960137326\\
12597.0000000001	75.4379976343762\\
12602.1000000001	75.4407960514831\\
12602.6000000001	75.437803010466\\
12607.1	75.4402246335427\\
12609.9000000002	75.4377478191911\\
12611.7	75.4396676128705\\
12612.7000000001	75.4376506406191\\
12614.3	75.4352168949772\\
12617.1	75.4374980185778\\
12620.5000000002	75.4393610446413\\
12624.7000000001	75.4411951080413\\
12625.0000000001	75.4394024601785\\
12632.0999999999	75.4413324678203\\
12637.4	75.4381800197824\\
12639.8999999999	75.440664556962\\
12640.8000000001	75.438457704752\\
12642.3000000002	75.4405809023603\\
12643.4000000001	75.4387630007514\\
12645.5000000002	75.4420509900677\\
12646.2999999999	75.4396508097166\\
12648.3999999998	75.4413566826106\\
12651.4999999998	75.4465838312941\\
12657.1999999999	75.4489504080649\\
12659.5	75.4470915352776\\
12663.2999999998	75.4489315665619\\
12666.0000000002	75.4510070187351\\
12667.1000000001	75.4491916129847\\
12668.8000000001	75.4509073400216\\
12674.1000000001	75.4485490208455\\
12676.1	75.4516337703728\\
12682.3000000002	75.4494417460418\\
12690.5999999998	75.4473748492991\\
12695.1999999999	75.4452435153167\\
12699.6	75.4435144137263\\
12704.5000000002	75.4459014845017\\
12705.4000000001	75.4437054818779\\
12707.6000000002	75.4455959772421\\
12708.6	75.4435937586063\\
12714.4000000001	75.4453576625113\\
12715.0000000001	75.443370480767\\
12717.7	75.4415071789146\\
12721.4999999999	75.4456986542573\\
12730	75.4479540616335\\
12730.6999999999	75.4461620636566\\
12732.1000000001	75.4480765303718\\
12734.0999999999	75.4456502960531\\
12737	75.4433897826036\\
12738.6999999999	75.4411718529218\\
12746.2000000001	75.4446388363682\\
12753.3999999998	75.442819618144\\
12755.1000000002	75.4453085800301\\
12756.8000000002	75.4430935415344\\
12758.3	75.4412779031854\\
12762.1000000002	75.4431054206955\\
12769.7999999998	75.4453832841291\\
12771.3	75.4435692249871\\
12775.3000000001	75.4418648339778\\
12779.3	75.44016150993\\
12785.7	75.4383769494283\\
12788.2000000001	75.4353588827287\\
12798.6999999999	75.4375410194706\\
12799.6	75.4353617662914\\
12806.6	75.4370759055807\\
12810.9	75.4398563734291\\
12812.7999999998	75.4380351052455\\
12819.1000000001	75.4352845731403\\
12822.2999999998	75.4382954829049\\
12823.5000000001	75.4359150316604\\
12834.6000000001	75.4384598003849\\
12835.1000000002	75.4363001745201\\
12835.7000000001	75.4343321024011\\
12836.6000000002	75.436054437667\\
12842.1999999998	75.4382003223722\\
12848.8	75.4430340340418\\
12851.8000000002	75.444875854932\\
12854.3	75.4480955937267\\
12865.9000000001	75.4515778019586\\
12871.0000000001	75.4535354398614\\
12880.8	75.4559075840974\\
12881.8999999999	75.4541220307406\\
12884.4	75.4588847064302\\
12884.8999999999	75.4567326348467\\
12889.2000000001	75.4587138168869\\
12893.9999999998	75.4554408605486\\
12895.1000000001	75.4575345865128\\
12896.9999999999	75.4557226043064\\
12898.3000000001	75.4581963654407\\
12902.8999999999	75.4599705494846\\
12909.8000000002	75.4575945592142\\
12912.1	75.4557705116092\\
12924.1000000002	75.4537998483465\\
12925.5999999999	75.4520064677348\\
12929.1000000002	75.4547845187637\\
12930.8000000001	75.452598040353\\
12933.4	75.4544400201028\\
12935.3000000001	75.4526338574764\\
12937.3999999999	75.4504347826087\\
12939.6000000002	75.4476533459045\\
12942.6000000001	75.4456179931544\\
12946.9000000001	75.4437321387194\\
12948.4999999998	75.4467664457934\\
12949.0000000002	75.4446254952082\\
12951.4000000002	75.4468594371308\\
12952.8	75.4441090412186\\
12957.6999999998	75.446449242927\\
12961.9000000002	75.4443758679216\\
12967.1000000002	75.4465111974829\\
12970.6000000001	75.4508237797497\\
12974.1000000002	75.4489679517812\\
12980.3999999999	75.4539501560032\\
12981.0999999999	75.4514220565125\\
12987.3	75.4531314966814\\
12990.0000000002	75.4513052247481\\
12993.1000000002	75.4548533078841\\
12995.1999999999	75.4511246373689\\
12998.7000000001	75.453118749423\\
12999.1000000001	75.4507969721214\\
13000.1999999998	75.4490280993516\\
13005.3999999998	75.4511552804583\\
13005.7000000002	75.4494148764397\\
13009.6000000002	75.4475506737281\\
13012.3000000001	75.4495711782607\\
13015.7999999998	75.4515630882229\\
13016.8999999998	75.4497964200661\\
13020.7	75.4477451462276\\
13022.1000000001	75.4496168082198\\
13027.6000000002	75.4476999009802\\
13031.2000000001	75.4452740708909\\
13036.6000000002	75.4431719683662\\
13039.7000000002	75.4451755395021\\
13041.7	75.4420402091736\\
13042.6999999998	75.4439230840004\\
13044.4000000002	75.4417570623635\\
13048.2	75.4435443697647\\
13050.4999999999	75.4455733835992\\
13051.9999999999	75.4437983159798\\
13054.9999999999	75.4417813727968\\
13059.7000000002	75.4398995390435\\
13063.0999999999	75.4424643272705\\
13065.7999999999	75.4406508545144\\
13068.7000000001	75.4384488246817\\
13070.0999999999	75.4410797080381\\
13074.7	75.4428366017071\\
13075.7	75.4408908059163\\
13077.8999999999	75.4427282459092\\
13081.3999999999	75.4454764361885\\
13089.1999999998	75.449412879222\\
13092.3999999999	75.4515944242887\\
13096.6000000001	75.4495407239992\\
13101.2000000002	75.4512910932503\\
13115.8999999999	75.4536444037816\\
13116.4	75.4515305150002\\
13119.2000000002	75.453720854009\\
13126.0000000002	75.4588186894813\\
13137.0999999999	75.4605243126389\\
13137.9999999998	75.4583996163829\\
13141.2999999999	75.4607576057346\\
13145.2000000002	75.4649950933033\\
13145.8000000002	75.4630721365597\\
13148.8999999999	75.4665754049738\\
13152.4000000001	75.4647405436229\\
13155.4000000001	75.4665349093535\\
13161.8	75.464788518375\\
13163.7999999998	75.4685161692204\\
13164.1000000002	75.4667963112077\\
13172.3999999998	75.4685898652496\\
13175.3000000001	75.4724714240175\\
13176.1	75.4701659051927\\
13177.3999999999	75.4725858470878\\
13178.0999999998	75.4708533790654\\
13179.4000000002	75.4732728859213\\
13182.4	75.4705101460269\\
13184.1999999999	75.4685497144331\\
13187.9999999999	75.4665190588485\\
13188.6999999999	75.4647883052287\\
13190.9	75.4666060192556\\
13194.3000000001	75.4645910386224\\
13199.3999999999	75.4627069207167\\
13204.9999999998	75.460996130283\\
13215.0999999998	75.4646165022096\\
13215.8000000001	75.4628893983762\\
13218.3	75.4652605459057\\
13218.9999999998	75.4635338260547\\
13230.3000000001	75.4618152134478\\
13230.8000000002	75.4597192934721\\
13239.6999999998	75.4573331923443\\
13240.9999999999	75.4552114250327\\
13248.6999999999	75.4528712034298\\
13252.6999999999	75.4549981890619\\
13255.6999999999	75.4530092487817\\
13257.5	75.4548334540188\\
13259.5000000001	75.4525023379287\\
13267.4999999998	75.4507220597546\\
13273.7	75.4486281245762\\
13275.1999999999	75.4506489495529\\
13276.9	75.4485199969873\\
13279.5999999999	75.4459814604246\\
13280.9999999999	75.4485697720821\\
13284.1000000002	75.4467713524337\\
13284.6999999999	75.4448693243406\\
13290.2000000002	75.4467543998254\\
13291.7	75.4450112099189\\
13292.7999999999	75.4432817519127\\
13294.2999999998	75.4415392947407\\
13296.9	75.4433330826502\\
13297.6	75.4408657136197\\
13302.3000000002	75.4435289872504\\
13304.1000000001	75.4415898738744\\
13307	75.4454389010378\\
13308.1000000001	75.4429599795615\\
13314.2000000002	75.4406915872408\\
13321.6000000002	75.438570152458\\
13330.2999999998	75.4425973714217\\
13331.7000000002	75.4406756777029\\
13333.1000000001	75.4432544325443\\
13336.4000000001	75.4463314962696\\
13346.4999999999	75.4484288133307\\
13347.2	75.4459703460625\\
13356.0000000002	75.4441790642478\\
13358.5000000001	75.4465288278712\\
13360.4000000002	75.4447812581864\\
13363.5	75.4429944027058\\
13366.4999999998	75.4447653105502\\
13369.0000000001	75.448609106073\\
13369.5	75.4465354236477\\
13375.2000000001	75.4487749807481\\
13376.7	75.4462950780456\\
13377.9000000002	75.4432650620422\\
13382.9000000002	75.4449674960771\\
13387.3999999999	75.4427637721755\\
13396.7	75.4456288068793\\
13401.6999999998	75.4473279708696\\
13404.2000000002	75.4451929604679\\
13407.5999999998	75.4469446661247\\
13413.0999999999	75.4450839471565\\
13416.6000000002	75.4470175229378\\
13417.7000000001	75.4445587205056\\
13420.3999999999	75.4465183860512\\
13421.5	75.4448053883293\\
13423.4	75.4467910753529\\
13425.3000000002	75.4495210571007\\
13429.0999999998	75.4475322431716\\
13434.0000000001	75.4453219791426\\
13437.2	75.447448520164\\
13442.1000000002	75.4452396185148\\
13450.1	75.4434878291773\\
13453.7000000002	75.4455990129183\\
13454.2	75.4435384969861\\
13456.1	75.4455195374623\\
13462.6999999999	75.4486436699646\\
13463.7000000001	75.4467535168377\\
13467.0999999999	75.4484970892242\\
13470.8999999998	75.4457723999703\\
13473.6999999999	75.4434532203239\\
13479.3999999999	75.4456767684261\\
13479.9999999998	75.4438023456799\\
13484.7000000002	75.4456869957285\\
13491.1000000002	75.4439931214421\\
13497.6999999999	75.4478507608647\\
13502.4999999999	75.4513945462355\\
13506.3000000002	75.4531185215898\\
13509.8000000002	75.4550366768074\\
13510.3000000002	75.452244197063\\
13513.3	75.4495537762517\\
13516.2000000001	75.4518618260915\\
13526.6	75.4544715266843\\
13531.3999999999	75.4565273620811\\
13533.4000000001	75.458676617283\\
13535.3	75.4569499239033\\
13537.1999999999	75.4552237152165\\
13539.8999999999	75.4534711964549\\
13546.0000000002	75.4556662065096\\
13546.6000000002	75.4538005565931\\
13549.3999999999	75.4566589173032\\
13551.1999999999	75.4584431013999\\
13552.2	75.4565645683759\\
13553.1999999999	75.4583754509972\\
13557.4	75.4600774479071\\
13558.9	75.4576296187035\\
13563.9999999999	75.4594849639858\\
13564.6000000001	75.4576216208246\\
13566.4000000001	75.4594036781779\\
13567.3999999999	75.4575271789202\\
13570.2999999999	75.4554029358015\\
13572.8999999999	75.4534738082959\\
13575.4000000002	75.4565209384553\\
13576.7000000002	75.4544517117436\\
13578.3999999999	75.4523695548109\\
13579.6999999999	75.4503011826389\\
13589.2000000001	75.4520100373088\\
13590.1999999998	75.4501372302304\\
13592.6000000001	75.4522648186159\\
13604.7	75.4549864753616\\
13605.6000000002	75.4529351668786\\
13607.6000000002	75.4506639623155\\
13609.8999999999	75.4489346069067\\
13612.6	75.4471926950568\\
13614.4	75.4489698483235\\
13621.0000000001	75.4513218462532\\
13625.7	75.4487809890062\\
13627.4	75.4467070262337\\
13629.4000000002	75.449576286731\\
13631.2000000002	75.4513509349805\\
13633.9000000002	75.4496112659528\\
13637.4000000001	75.4515123739688\\
13640.2000000001	75.4536190552994\\
13641.1000000001	75.4515731753805\\
13645.3000000002	75.4496020636991\\
13647.1999999999	75.4515545199417\\
13660.8000000002	75.4489089298655\\
13662.5000000002	75.4512318299592\\
13666.0999999999	75.4540398940452\\
13669.6999999999	75.4568464790999\\
13674.3999999999	75.4594317890965\\
13681.3	75.4615755697516\\
13683.6999999998	75.4593022406057\\
13685.0999999998	75.457428462865\\
13685.9999999998	75.4553890443589\\
13686.6000000002	75.4535424901547\\
13691.1999999998	75.451564131967\\
13693	75.4533305095267\\
13696.0000000001	75.4506757398091\\
13699.8000000002	75.4523755647851\\
13702.1	75.454306607698\\
13711.6000000002	75.4567267370202\\
13714.3	75.4586420113166\\
13715.1000000001	75.4564279048063\\
13718.5000000002	75.4544924409196\\
13719.1000000002	75.4526503003091\\
13720.9000000001	75.454412943663\\
13722.8000000002	75.4563539776578\\
13726.3000000002	75.4545984380464\\
13727.5	75.452373320901\\
13731.7999999998	75.4542343011528\\
13745.5	75.4525084390641\\
13750.0000000001	75.4503603610156\\
13751.8000000001	75.4484834822824\\
13757.3000000001	75.4503031095992\\
13764.4999999999	75.4486145619924\\
13766.8	75.4512635379061\\
13773.5	75.4494104663995\\
13777.1	75.4521963824289\\
13785.0999999998	75.4541102051476\\
13785.7000000001	75.4522769806613\\
13787.6000000001	75.4549344705788\\
13789.5000000001	75.4568660439752\\
13793	75.458743864686\\
13795.6000000002	75.4568452488819\\
13800	75.458873486424\\
13801.3000000002	75.4561131479415\\
13802.7	75.4542556582722\\
13809.8	75.4560134396339\\
13811.2000000002	75.457777327261\\
13815.5	75.4560062538001\\
13816.6	75.4536177234796\\
13819.1	75.4515456755818\\
13821.9	75.4492837505426\\
13831.2000000001	75.4513314005191\\
13833.1000000001	75.453257380794\\
13837.4999999998	75.4567265999884\\
13838.4999999999	75.4548870550489\\
13845.5	75.4528514473912\\
13849.6	75.4565080832076\\
13850.9999999999	75.453935066529\\
13854.8000000002	75.4520061494489\\
13856.4999999998	75.4550178254406\\
13863.8000000002	75.4571224547205\\
13864.4	75.4545782393884\\
13867.1000000001	75.4593573324103\\
13868.1000000001	75.4575215240622\\
13876.5	75.455082657135\\
13881.1	75.457453246117\\
13887.4000000001	75.4556255625563\\
13893.3	75.453812601667\\
13898.7999999999	75.4520141881732\\
13900.1999999999	75.4501701402128\\
13914.4000000002	75.4529447698444\\
13916.4999999998	75.4509003635946\\
13917.3999999999	75.448895275732\\
13923.7000000002	75.4470762291903\\
13925.0000000001	75.4493684066901\\
13925.6	75.4475538033995\\
13928.2000000002	75.4492651651673\\
13930.0999999998	75.4511780161089\\
13932.7999999999	75.4494757013974\\
13937.0999999999	75.45131016273\\
13938.8999999999	75.4530454121529\\
13942.1999999999	75.4552692167003\\
13948.8999999999	75.4527206251344\\
13952.9000000002	75.4504407654268\\
13954.4000000001	75.4523630370131\\
13966.8000000001	75.4541093585549\\
13971.0000000001	75.4579095418399\\
13977.1999999998	75.4559178095913\\
13984.5000000002	75.4537133704218\\
13989.8000000001	75.4515757796696\\
13990.6	75.449405676628\\
13994.8	75.4517717168397\\
13997.8999999998	75.455065009287\\
14005.8000000001	75.4532018649284\\
14008.0000000001	75.4513460069531\\
14011.8000000002	75.44872572599\\
14018.7	75.4508231803007\\
14023.2999999999	75.4488925652837\\
14031.2999999999	75.4472112547572\\
14035.5999999999	75.4504584737491\\
14044.1000000001	75.4532120021076\\
14051.1999999999	75.4513817226876\\
14052.6	75.4538273783686\\
14057.2999999999	75.4520750636675\\
14067.6000000002	75.4544097471513\\
14069.8999999999	75.456289978678\\
14071.7000000002	75.4544550093094\\
14077.5999999999	75.4562179901546\\
14081.3999999999	75.4585804069169\\
14085.2999999999	75.4604058102716\\
14085.8999999998	75.458611387193\\
14089.8999999998	75.4606103619588\\
14093.8000000001	75.4624341026969\\
14107.7999999999	75.4605575599487\\
14116.9999999999	75.4637992222199\\
14117.6000000001	75.461300353457\\
14120.3999999999	75.4590843100457\\
14122.7000000002	75.461664825672\\
14124.6999999998	75.4637233801541\\
14128.3999999998	75.4659022543087\\
14133.5	75.4690949227373\\
14135.2999999999	75.4708037975579\\
14144.0000000002	75.4752865152254\\
14144.4	75.4731521085934\\
14151.4	75.4768045790199\\
14154.5	75.4786429853193\\
14155.0000000001	75.4766833155541\\
14157.4000000001	75.4787215256931\\
14158.0000000001	75.4769354645044\\
14158.9999999999	75.478667429427\\
14162.4999999999	75.476960445116\\
14167.9999999999	75.4794220820011\\
14169.9000000002	75.4770642201835\\
14172.5000000002	75.4815630159604\\
14175.4000000002	75.4837571866954\\
14176.9	75.4814135571701\\
14183.6000000002	75.4795998223313\\
14186.8000000002	75.4816062705736\\
14188.1999999998	75.4797967339287\\
14189.6999999999	75.482388758122\\
14202.9000000002	75.4840526649299\\
14206.8999999998	75.4867318927289\\
14210.8	75.488533449676\\
14214.3000000002	75.490347816299\\
14229.5999999998	75.4885907643872\\
14239.0999999999	75.4859823585595\\
14240.1000000001	75.4841926377438\\
14248.1999999999	75.4861983534878\\
14250.3999999998	75.4843689695098\\
14255.5	75.4826173573894\\
14259.1000000002	75.4845994165171\\
14262.2000000001	75.4871233952448\\
14263.6000000002	75.485322882562\\
14269.3999999999	75.4882791968885\\
14273.4999999998	75.4862123080372\\
14281.3999999999	75.4843678885271\\
14283.2	75.4825565520573\\
14284.8000000002	75.4797023430336\\
14285.9999999998	75.4768621247226\\
14302.9	75.4750751590575\\
14306.1	75.472871901693\\
14308.7000000002	75.4710388012971\\
14311.2999999999	75.4692063669522\\
14313.6000000001	75.4710522087231\\
14315.3999999998	75.4692466207956\\
14321.8	75.4711316235974\\
14322.5999999999	75.4690107312169\\
14329	75.472290653286\\
14332.1000000002	75.4741072549922\\
14333.1999999999	75.4718034227987\\
14334.5999999998	75.474198971726\\
14335.9999999998	75.4724088141126\\
14340.3000000001	75.4706981674151\\
14341.9	75.4685538976433\\
14346.7000000001	75.4704881924889\\
14347.3000000001	75.4687260409552\\
14350.1000000002	75.4665440202924\\
14353.6000000002	75.4683461407163\\
14354.2000000001	75.4665849257714\\
14356.2000000001	75.4644302501341\\
14358.6000000001	75.4671383899657\\
14364.0000000001	75.4700955855222\\
14364.9000000002	75.4681517577445\\
14367.5	75.4705030763663\\
14370.1	75.4686782369069\\
14371.0999999999	75.4669060342908\\
14372.2999999998	75.4689543847931\\
14378.1999999998	75.4671970956233\\
14380.6000000001	75.4699006307064\\
14398.1999999999	75.4721043456519\\
14398.6999999998	75.4701780703948\\
14399.6999999999	75.4718815539105\\
14400.4000000002	75.4696017499392\\
14402.5000000002	75.4717898157277\\
14403.0999999998	75.4700344367918\\
14407.7999999998	75.4731779093414\\
14412.5000000001	75.4707686330017\\
14415.9000000002	75.468923418424\\
14416.4000000001	75.4669996184927\\
14418.1000000002	75.4650372445936\\
14419.8999999998	75.4632454923717\\
14422.8000000002	75.4612456579467\\
14424.2000000002	75.4636273510673\\
14426.4	75.4618237271687\\
14431.0999999998	75.4601141970176\\
14433.3000000001	75.4624689955243\\
14435.3999999998	75.4604966921825\\
14437.1999999998	75.4628635548198\\
14445.1	75.4610528064686\\
14456.0999999998	75.4645065784923\\
14457.4	75.4625626837282\\
14466.0000000001	75.4647071429065\\
14466.8	75.462607745958\\
14468.4999999999	75.4647996350718\\
14472.2000000002	75.4676174485051\\
14477.5000000001	75.4648560534895\\
14482.2000000001	75.4672945595658\\
14484.4	75.4654976008837\\
14485.8999999999	75.4673477840674\\
14488.7000000001	75.4651869029871\\
14490.5	75.4675444771093\\
14495.9000000001	75.4656456953642\\
14501.9000000001	75.4682112812026\\
14508.4999999999	75.4662751747239\\
14519.6999999999	75.4679816526398\\
14523.4000000001	75.4659689468792\\
14528.3999999999	75.468217641188\\
14534.1999999999	75.4718149480883\\
14534.7000000002	75.4692187026997\\
14536.7000000002	75.4719057839414\\
14542.6000000002	75.4742929442263\\
14545.8999999998	75.4722947889454\\
14546.7999999998	75.4703751314713\\
14548.0999999999	75.4725670529688\\
14556.8999999999	75.4743422408463\\
14563.9000000001	75.4723976929415\\
14565.4000000001	75.474923620885\\
14568.7000000002	75.472928449838\\
14572.6999999998	75.4748572683355\\
14575.3	75.477859955816\\
14596.6999999998	75.4802422448756\\
14602.6	75.4785073993166\\
14603.7999999998	75.4764138346606\\
14608.5	75.4747203701929\\
14614.4000000001	75.4729891546067\\
14618.6999999998	75.4706268640381\\
14620.1	75.4688718348586\\
14629.6999999999	75.4671970908693\\
14630.9999999998	75.4693768752862\\
14631.6000000001	75.4676490086593\\
14647.2000000002	75.4657855031303\\
14648.7000000001	75.4635191961116\\
14654.1	75.4609599978163\\
14656.0000000001	75.4627765913169\\
14657.4999999998	75.4598297129134\\
14658.6000000002	75.4616712259614\\
14659.6999999999	75.4587374998295\\
14664.9000000002	75.4606205250597\\
14668.3	75.458809413433\\
14670.8999999998	75.4570240610729\\
14680.6000000001	75.458935881804\\
14682.7000000001	75.4569973029667\\
14690.1	75.4605110890253\\
14692.2000000001	75.4626573102918\\
14695.3	75.4644310464499\\
14698.6999999999	75.4626227991401\\
14702.8000000001	75.4606234144285\\
14704.5999999999	75.4588668928982\\
14707.7000000001	75.4565604645154\\
14710.2999999999	75.4588590385034\\
14712.9	75.461836471148\\
14716.9999999999	75.4598392346318\\
14718.2999999999	75.462006739863\\
14723.5000000001	75.4652394794751\\
14731.8999999999	75.4629378224274\\
14739	75.4611882679404\\
14741.9000000002	75.4592321258988\\
14747.3999999999	75.4575351754535\\
14755.4	75.4593202534648\\
14761.7000000003	75.4576000216776\\
14768.4	75.4558689101805\\
14769.9000000002	75.4576844955992\\
14771.0999999998	75.4556163344887\\
14772.3999999998	75.4577762734811\\
14779.0000000001	75.4558802633449\\
14782.7	75.4539058906296\\
14788.3	75.4557626247599\\
14794.8000000003	75.4577590926603\\
14798.5000000001	75.4598407957509\\
14802.8000000001	75.461564963622\\
14812.5000000001	75.4594061812241\\
14813.6	75.4612284574414\\
14815.9999999998	75.4591289205661\\
14820.4	75.4637157990621\\
14826.6999999998	75.462001240996\\
14827.5999999998	75.4601185618808\\
14830.2999999999	75.4625633833207\\
14831.9999999998	75.4606562792861\\
14835.2999999999	75.4627445164943\\
14835.9000000002	75.4610407117822\\
14838.1	75.4633311318084\\
14841.6000000002	75.461032092011\\
14843.7000000002	75.4645036985139\\
14844.5	75.4617840831009\\
14846.9000000001	75.463730046474\\
14856.5000000001	75.4654496991236\\
14857.2999999998	75.4634054410597\\
14859.0000000002	75.4615017060253\\
14862.3999999998	75.4597140454163\\
14863.8	75.4573160476053\\
14865.8999999999	75.4594376429436\\
14869.4999999999	75.4573088717921\\
14872.3999999998	75.4594049420071\\
14879.9	75.4576612903226\\
14884.2	75.4600485074878\\
14885.5999999999	75.4583257757445\\
14889.4000000001	75.4565297693005\\
14893.5000000001	75.4545576623516\\
14896.9000000001	75.4568033832315\\
14900.7000000001	75.4590357564694\\
14905.1000000001	75.4568875291844\\
14906.5000000002	75.4591925724176\\
14911.6999999999	75.4617148835151\\
14912.3000000001	75.4593492663823\\
14918.1000000001	75.4574948720355\\
14921.0999999998	75.455727421387\\
14931.2000000002	75.4575957887123\\
14934.8000000002	75.4554767691782\\
14937.9999999998	75.4533709106245\\
14940.6	75.4556346088202\\
14941.5999999999	75.4539309449393\\
14952.3	75.4567828576015\\
14964.7	75.4550678926548\\
14969.3	75.4572661562922\\
14977.2999999999	75.4590249308959\\
14978.3	75.4573252149762\\
14979.6999999998	75.4596189535241\\
14981.9000000002	75.4578827926845\\
14984.4	75.4546364576729\\
14985.5000000002	75.4564381806534\\
14988.1999999998	75.4581907220966\\
14995.8	75.4599590554752\\
14999.7000000001	75.4616728223043\\
15001.9999999999	75.4587691056585\\
15003.2000000001	75.4607319722994\\
15004.8999999999	75.458847050983\\
15012.9999999998	75.4567677561596\\
15017.8999999999	75.4547875882275\\
15023.2000000002	75.4567904521643\\
15024.9	75.4549084858569\\
15030.8999999998	75.4527310225534\\
15042.6	75.4505507654876\\
15045.1	75.4533007204956\\
15048.1999999999	75.4510476266422\\
15055.2000000002	75.4491773661103\\
15058	75.4464374655501\\
15060.7999999999	75.4483463803624\\
15065.6999999999	75.4503577639422\\
15068.4999999998	75.4522649748484\\
15071.5000000002	75.4491892035351\\
15076.3	75.447056326444\\
15077.7999999999	75.4494989355282\\
15095.1000000001	75.447824474005\\
15098.0999999999	75.4460796651256\\
15102.2000000002	75.4481105526973\\
15103.1	75.4456009322528\\
15104.0000000002	75.4437536827749\\
15112.9999999999	75.448452005214\\
15114.3000000002	75.4465939766051\\
15119.3000000002	75.4447927827824\\
15121.5000000002	75.4470426409904\\
15127.5999999998	75.449671794126\\
15129.7999999999	75.4479540512495\\
15132.1999999999	75.4498655194518\\
15134	75.4481601152364\\
15135.7999999999	75.4464551166432\\
};
\addplot [color=mycolor2]
  table[row sep=crcr]{%
15135.7999999999	75.4464551166432\\
15147.1	75.4482676666315\\
15151.3000000002	75.4464933933498\\
15156.5000000002	75.4489793225394\\
15163.3999999999	75.4469614534903\\
15165.3999999999	75.4495400745112\\
15165.7999999998	75.4475500959389\\
15172.0999999999	75.4439039822834\\
15174.7	75.4421804570736\\
15178.7000000001	75.4387698632303\\
15185.2000000001	75.4413808090719\\
15189.7000000003	75.4440479795652\\
15190.8999999998	75.4420380488447\\
15194.8999999998	75.4399473511023\\
15198.9999999999	75.4380193564093\\
15203.1000000001	75.4360924015997\\
15211	75.4343867307427\\
15212.9	75.4321961480313\\
15217.6999999998	75.4340312003049\\
15220.6000000001	75.4321417543214\\
15222.6	75.434055719419\\
15225.9	75.4360961513201\\
15227.6999999998	75.4337461747593\\
15242.3999999999	75.4357880925045\\
15246.9000000001	75.4377910408605\\
15249.6	75.4395168429543\\
15254.9999999998	75.441655577479\\
15261.6	75.4450683737722\\
15263.6000000001	75.4430446091053\\
15266.8000000002	75.4409867098101\\
15269.9	75.4446627373936\\
15272.5999999999	75.446384725687\\
15273.5000000001	75.4445579300231\\
15274.5999999998	75.4463262780938\\
15277.3999999999	75.4442808051055\\
15279.4999999999	75.4470012303987\\
15290.6	75.4452052554821\\
15292.8000000002	75.4474298530691\\
15293.2999999998	75.4456170635699\\
15301.2	75.443916529968\\
15304.3	75.445623480829\\
15305.7999999998	75.4428031020717\\
15307.9	75.4448654298406\\
15309.1	75.4428709534136\\
15311.8000000002	75.4452419360106\\
15312.7000000001	75.4434198840186\\
15324.0999999998	75.4460265462471\\
15325.7999999998	75.4441827233637\\
15327.6999999999	75.4420073330811\\
15329.2000000002	75.4444103775124\\
15330	75.4424302515965\\
15339	75.4405408400754\\
15340.2999999999	75.4426220959036\\
15341.1999999999	75.4394999120022\\
15342.5000000001	75.4376702775279\\
15344.7999999999	75.4393968028465\\
15348.6000000003	75.4415683412928\\
15350.7000000002	75.4436250879433\\
15355.9	75.4415212294868\\
15361.1	75.4394187954066\\
15364.1000000001	75.4377058356439\\
15366.5000000002	75.4408912836932\\
15368.1000000002	75.4388932991502\\
15371.9000000002	75.4371584699454\\
15373.6000000002	75.4398745910223\\
15381.7000000001	75.441755841319\\
15384.9	75.4390640233994\\
15386.9000000002	75.440956651719\\
15389.3999999999	75.4429968485006\\
15398.2000000002	75.4446919465136\\
15399.1000000001	75.4422307652346\\
15401.4	75.4400545401422\\
15408.6000000001	75.4417958685678\\
15414.1	75.4453685562663\\
15415.7000000001	75.443376276288\\
15421	75.4459798587649\\
15427.6000000001	75.4441686058194\\
15435.3000000001	75.446052580432\\
15440.9000000003	75.4484813159769\\
15441.4000000002	75.4460382734838\\
15445.2000000001	75.4443099195224\\
15453.4000000001	75.4463390170512\\
15456.5000000003	75.4434998641357\\
15461.7000000002	75.4452909751775\\
15467.7999999999	75.448509493855\\
15469.3000000002	75.4502437069311\\
15472.1000000002	75.4533938289319\\
15474	75.4499453926238\\
15476.1000000001	75.4519843372404\\
15478.2000000001	75.4501463338997\\
15479.8999999998	75.4521963824289\\
15481.5	75.4502118644068\\
15490.3	75.4525383463306\\
15495.2000000002	75.4506205107355\\
15501.7	75.4531731798888\\
15508.1999999999	75.4512099972273\\
15515.9000000001	75.449213714875\\
15519.3000000001	75.4475044138304\\
15523.4999999998	75.4502821510474\\
15526.1999999999	75.4481106251972\\
15529.4000000001	75.4499500949805\\
15540.7000000002	75.4523576649851\\
15541.9000000002	75.4503924848797\\
15545.4999999998	75.4528612597777\\
15552.7999999999	75.4508805431784\\
15554.8	75.4527512230873\\
15558.9999999998	75.451022231363\\
15561.8999999999	75.4491710577047\\
15566.9999999999	75.4469361666592\\
15575.9999999998	75.4450729001483\\
15583.2999999998	75.4430997086643\\
15584.6	75.4451481260467\\
15593.2999999998	75.4485872227994\\
15599.3999999998	75.4466489310555\\
15600.6000000001	75.444691584352\\
15606.7999999999	75.4467575239157\\
15614.6000000002	75.4487758330291\\
15615.0000000002	75.4468431198007\\
15621.0000000001	75.4485919685554\\
15628.7	75.4504504504504\\
15630.3000000001	75.4523236769373\\
15648.9999999998	75.45417947358\\
15652.4000000001	75.4524836288133\\
15655.8000000003	75.4507885206216\\
15656.9000000002	75.4525132528581\\
15660.5	75.4549634113636\\
15661.4999999998	75.4526995964653\\
15665	75.4549923077414\\
15668.7000000003	75.4524915756152\\
15670.4	75.4506875977154\\
15675.7999999998	75.4489375410662\\
15677.8999999999	75.4509503763235\\
15681.9000000002	75.4489223313353\\
15690.1000000001	75.4509184076685\\
15693.3	75.4482776198912\\
15698.8999999998	75.4462067647621\\
15701.6999999999	75.4429428473169\\
15704.3000000001	75.4450981890426\\
15716.6	75.4471358491286\\
15717.4000000002	75.4452043900111\\
15722	75.4473002970341\\
15723.4000000002	75.4450345024963\\
15724.8000000002	75.4472206500518\\
15735.1000000002	75.4448624739438\\
15737.4999999999	75.4467008946726\\
15751.7000000001	75.4485201691235\\
15752.7999999998	75.444521326232\\
15755.6999999999	75.4465022404448\\
15766.4999999998	75.4493676506032\\
15772.2000000001	75.4518998497366\\
15779.2000000002	75.4494812824397\\
15780.8000000002	75.4513367425179\\
15783.6000000002	75.4493559811705\\
15800.0999999999	75.451576562322\\
15801.0000000001	75.4498104562341\\
15807.0000000001	75.4515376001923\\
15808.2999999999	75.4497608866172\\
15811.2000000002	75.4523663455883\\
15829.8999999999	75.4548325963361\\
15832.2999999998	75.4572901139436\\
15834.3999999998	75.459281947646\\
15834.9000000002	75.4568992737607\\
15838.4000000002	75.4534836000884\\
15844.8999999999	75.4515620069423\\
15850.6	75.4540808923265\\
15851.7000000002	75.4519991420533\\
15857.0000000003	75.4501138291365\\
15871	75.4484566287151\\
15873	75.4502901134624\\
15877.4999999999	75.4522094019247\\
15881.0999999999	75.4502178676674\\
15888.8999999998	75.4484234375983\\
15899.2	75.4504915310737\\
15900	75.4485820843894\\
15902.5999999998	75.4507096279248\\
15905.1000000001	75.448281065312\\
15906.9999999998	75.4505849589177\\
15909	75.4486426007757\\
15912.6999999999	75.4468101151274\\
15917.0000000001	75.4490453663042\\
15919.2999999999	75.4469389549857\\
15924.2	75.4450745087696\\
15927.8000000001	75.4430904262332\\
15932.7	75.4412281582647\\
15937.1000000001	75.4392239540195\\
15938.4000000002	75.4412272171158\\
15942.4000000003	75.442998275051\\
15944.4999999999	75.4412152076565\\
15948.4999999999	75.4392235055114\\
15951.3999999998	75.4374196784001\\
15955.1000000002	75.4399819494585\\
15958.3000000002	75.4380138359735\\
15966.1	75.4399919830642\\
15971.9000000001	75.4382669671926\\
15974.7999999999	75.4408478300334\\
15979.0000000002	75.4429223172769\\
15988.8999999998	75.4412408530865\\
15990.8999999999	75.4436870739791\\
16003.5000000001	75.4473993351496\\
16010.4000000001	75.4492364385872\\
16012.8	75.4472956179081\\
16020.7999999999	75.4495690004931\\
16021.3000000002	75.4478385159849\\
16023.8000000001	75.4460524591392\\
16024.8	75.4438405231858\\
16027.3000000001	75.4420554799905\\
16030.6999999999	75.4460164183946\\
16032.8	75.4442427757923\\
16034.5000000002	75.4462225437491\\
16036.9	75.4480264388601\\
16038.6	75.446264348108\\
16042.6999999998	75.443812800758\\
16046.0000000002	75.4463701460168\\
16058.1999999998	75.4482105826893\\
16061.5000000002	75.4464063356079\\
16065.3	75.4484793406949\\
16065.8999999998	75.4462840781775\\
16068.1999999999	75.4485539851758\\
16077	75.4514184772129\\
16082.6000000002	75.4493959347622\\
16083.5000000001	75.4476609714243\\
16084.7999999999	75.445915112932\\
16090.1000000001	75.4440591167294\\
16093.8999999998	75.446128992171\\
16096.7000000001	75.4479151135629\\
16101.6000000002	75.4460709117671\\
16108.9999999999	75.4443140833442\\
16115.1999999999	75.4463149925847\\
16118.6999999999	75.4485445566667\\
16126.3000000001	75.4508135727751\\
16141.0999999999	75.4528783485738\\
16150.7000000001	75.4550858161825\\
16153.6000000001	75.4570160396689\\
16156.8	75.4587823159146\\
16159.3000000001	75.4570095424335\\
16162.2000000001	75.4589383936692\\
16163.0000000001	75.4570595987156\\
16171.8	75.4605210272139\\
16172.4000000002	75.4583397743082\\
16174.4000000002	75.4601378713407\\
16178.3999999998	75.4581697932441\\
16180.7999999998	75.4599558739007\\
16183.8	75.4638869493756\\
16188.3	75.4657656099429\\
16195.1999999999	75.4675739257686\\
16197.6999999999	75.4658039980738\\
16200.9000000001	75.4638602555398\\
16211.1000000001	75.4663442558231\\
16222.5999999998	75.4646267267471\\
16228.0000000001	75.4672450872252\\
16231.4999999999	75.4700707262377\\
16238.9000000002	75.4683170145945\\
16245.4000000002	75.466436859438\\
16250.2	75.4638375906906\\
16256.4000000001	75.4670439516501\\
16258.1000000002	75.4646885879126\\
16258.9999999999	75.4629715051879\\
16261.3999999998	75.4604433785321\\
16263.4999999998	75.4629971224083\\
16267.2999999998	75.4607374257718\\
16282.8999999998	75.4578394644722\\
16289.0000000001	75.4559797656101\\
16295.2000000001	75.4579541340141\\
16295.7000000003	75.4562525313271\\
16299.0000000003	75.4538594155506\\
16306.9999999999	75.4560896787289\\
16316.6000000001	75.4607242886123\\
16325.2000000001	75.4632380415674\\
16330.5999999998	75.4652280673823\\
16331.7000000002	75.4632067500214\\
16340.7000000002	75.4608097522765\\
16346.5000000001	75.4627873686271\\
16350.8	75.4588432441028\\
16352.8	75.461233175767\\
16356.1000000001	75.4594587984984\\
16360.1	75.4611801811714\\
16370.6999999998	75.4593544603807\\
16377.0000000003	75.4632993631351\\
16379.3999999998	75.4656735553588\\
16383.3000000001	75.4684619798088\\
16388.4000000002	75.4706043872227\\
16389.3000000001	75.4689006308956\\
16393.2999999999	75.4669562140862\\
16395.0000000002	75.4652304652\\
16401.4999999998	75.4633694273729\\
16405.0000000002	75.4655564428135\\
16408.3	75.4674435045465\\
16413.0000000002	75.4695944093438\\
16423.6999999999	75.4715717434455\\
16424.5000000001	75.4691134030661\\
16425.3999999998	75.4674134729536\\
16431.5000000003	75.4655663477689\\
16442.0999999998	75.4637457274574\\
16443.3999999998	75.462036671025\\
16448.0999999998	75.4599287460026\\
16454.9999999998	75.4580646729585\\
16457.5	75.4563241298853\\
16464.9	75.4600668083814\\
16467.1000000001	75.4578799067237\\
16468.6000000001	75.4558647616388\\
16477.1000000001	75.4539606243779\\
16479.9999999998	75.4516052693855\\
16481.2000000003	75.4533926328627\\
16489.4999999999	75.4511934795265\\
16491.4999999997	75.4493196536419\\
16497.4999999999	75.4515808360004\\
16498.6000000002	75.449580876069\\
16501.4999999998	75.451471372473\\
16512.9999999998	75.4491888258413\\
16520.6000000002	75.4471662823004\\
16524.5999999999	75.4494786592192\\
16528.2999999998	75.4513443527504\\
16531.2000000001	75.4496016647209\\
16533.2000000002	75.4513617971004\\
16541.2999999998	75.453105541248\\
16542.6000000003	75.4508030732589\\
16550.7999999999	75.4526944154094\\
16556	75.4549682594331\\
16556.9	75.4526786253548\\
16560.8000000001	75.4572517194114\\
16564.4000000001	75.4553412418123\\
16585.6000000001	75.4571709364091\\
16597.1000000001	75.459716096691\\
16598.0999999998	75.4569772625947\\
16603.3000000002	75.4592432875194\\
16610.6000000002	75.4616000529779\\
16612.4000000002	75.4594431903687\\
16621.4000000001	75.4570887104052\\
16625.3999999998	75.4551742804728\\
16626.4000000001	75.452440381319\\
16628.3000000001	75.4546438623079\\
16629.8999999998	75.4564040889958\\
16631.8000000002	75.4543978739651\\
16635.5999999998	75.4521901693346\\
16644.2000000001	75.4498537036703\\
16647.4000000002	75.4515693047004\\
16654.7000000001	75.4551240483224\\
16667.0000000001	75.4534382106065\\
16672.0999999998	75.4555487578124\\
16685.9999999998	75.4574166521836\\
16700.1999999999	75.4591234887996\\
16704.0000000002	75.4611143371987\\
16704.8999999999	75.4582460341215\\
16714.2000000002	75.4605337944156\\
16723.2999999998	75.4625255629836\\
16727.8000000001	75.4607571781276\\
16744.4999999997	75.4625371761643\\
16745.3	75.4607235419876\\
16753.0000000002	75.4588702986313\\
16755.0999999999	75.4607524828113\\
16757.3999999999	75.4587498135163\\
16760.6999999999	75.4605985394492\\
16770.9000000001	75.4624053425556\\
16774.6999999999	75.4602141307199\\
16778.6999999999	75.4583164469449\\
16783.1	75.4605796272463\\
16785.5000000002	75.4623010199218\\
16787.5999999998	75.4606050858664\\
16789.1999999998	75.4629436605457\\
16795.0999999998	75.4656092216824\\
16795.9000000003	75.4638009049774\\
16797.5000000002	75.4619707577273\\
16802.3999999998	75.4643654218122\\
16804.8000000003	75.4625139096335\\
16807.4	75.4603599583519\\
16809.6000000001	75.4582175767563\\
16811.2999999998	75.4559406117278\\
16822.3000000001	75.4577230359521\\
16825.1000000001	75.4558638233127\\
16828.4000000001	75.4582999078943\\
16834.7000000001	75.4603559293844\\
16837.2000000001	75.4586542973042\\
16840.0999999998	75.4569423165996\\
16842.4999999999	75.4550960065548\\
16843.2999999999	75.4532932780792\\
16848.6000000001	75.4515185147816\\
16860.7	75.4549013095464\\
16867.7	75.4567874885877\\
16868.0999999999	75.4549981622224\\
16874.5999999998	75.457934066976\\
16878.6000000001	75.4560481553674\\
16896.9	75.458365390306\\
16907.9999999999	75.4602823498796\\
16912.8999999999	75.4585230296222\\
16916.9	75.4607790979488\\
16918.0999999999	75.4583820973862\\
16942.0999999999	75.4600937304482\\
16946.2000000001	75.4624903371237\\
16955.4000000001	75.4651882869865\\
16959.3	75.4631649704589\\
16960.9999999998	75.4609076062284\\
16976.9000000001	75.4591506155387\\
16984.3999999999	75.457034354853\\
17004.0999999999	75.4590042460098\\
17004.8	75.4570741374545\\
17012.1000000001	75.4552615182046\\
17013.3000000001	75.4534660914338\\
17018.9999999998	75.4569865621566\\
17025.8999999998	75.4587102079173\\
17026.6999999999	75.456926727277\\
17028.1000000002	75.4589445742944\\
17034.9000000002	75.4570002935134\\
17038.8000000003	75.4549882914977\\
17045.6000000003	75.4530468094593\\
17048.9000000002	75.4501730306763\\
17059.5000000002	75.4484278646627\\
17063.0000000002	75.4505336076094\\
17064.6999999998	75.4482912193521\\
17067.9999999997	75.4501086822786\\
17072.8000000002	75.4523250297255\\
17075.5000000003	75.4497645763546\\
17088.0999999998	75.447969944172\\
17093.8000000003	75.4462118065509\\
17101.9	75.4443924687171\\
17104.2000000002	75.4424326046667\\
17106.1999999998	75.4406271373704\\
17108.8000000002	75.4426058951774\\
17112.9000000001	75.4443989949162\\
17120.1000000001	75.4465485216294\\
17121.7000000002	75.4447546402832\\
17122.7000000003	75.4426846076576\\
17125.5	75.4449479142337\\
17137.7	75.4431724025254\\
17163.1	75.445138435723\\
17165.1000000002	75.4427562743225\\
17169.9999999999	75.4445227459363\\
17173.5000000002	75.4425397121163\\
17181.2	75.444232974222\\
17182.0000000001	75.44246628759\\
17183.2000000001	75.4406895066722\\
17185.2000000001	75.4388925418817\\
17188.2000000001	75.4408522076063\\
17192.4999999999	75.4435047636774\\
17195.5999999999	75.4415348023052\\
17198.0000000001	75.4437990243108\\
17210.8000000001	75.4463741001342\\
17216.3000000001	75.4437629237239\\
17222.0999999998	75.4456457363171\\
17224.8000000001	75.4436890780208\\
17239.6999999999	75.4405503544125\\
17241.2999999997	75.4387694734766\\
17243.0999999998	75.4361139463673\\
17244.7999999998	75.4379555694727\\
17245.2000000001	75.4362058067995\\
17247.1	75.4342733893038\\
17256.5999999998	75.4315715055602\\
17272.7999999997	75.4297193870167\\
17274.8	75.4279330126368\\
17284.1000000002	75.4261117089596\\
17288.9000000001	75.4288854184742\\
17293.7	75.426453411049\\
17298.5999999999	75.4247428997555\\
17301.1000000003	75.4271380019883\\
17303.8000000001	75.4292384953681\\
17306.2	75.4314902665503\\
17312.5	75.4346545290713\\
17314.4000000001	75.4327297929481\\
17322.1	75.4309498793456\\
17328.2000000002	75.4326737187145\\
17333.9	75.4344063689858\\
17351.5000000002	75.4362710067083\\
17356.9000000002	75.4341187993317\\
17359.5000000001	75.4320376045531\\
17368.7000000001	75.4341117405923\\
17375.1000000002	75.4322252405728\\
17377.4999999998	75.4350428137372\\
17383.5	75.4377689316367\\
17394.4999999997	75.4360548676026\\
17398.6999999998	75.4379612387061\\
17401.9000000002	75.4361567635904\\
17404.9999999998	75.4342118114806\\
17406.7999999998	75.4361776077303\\
17422.1999999998	75.4383749562343\\
17423.9999999998	75.4403383818963\\
17432.7000000002	75.4382543251801\\
17435.1000000002	75.4364733412866\\
17442.0000000002	75.43472402979\\
17443.7000000001	75.4325319024525\\
17449.5999999999	75.4345346911408\\
17453.1000000003	75.4365961542869\\
17458.5999999999	75.4345970776747\\
17461.5000000002	75.4363861272736\\
17463.4000000001	75.4344776247602\\
17464.2999999999	75.4323080094363\\
17474.6000000002	75.4341991564948\\
17481.0999999997	75.4370409354049\\
17485.3	75.4389376279639\\
17486.8999999999	75.4371819065592\\
17488.9000000001	75.4394190634113\\
17492.9000000003	75.4376036128737\\
17495.6999999999	75.4398198424765\\
17505.4000000001	75.4431464396904\\
17510.0000000001	75.4410311762925\\
17518.9999999998	75.4428024270653\\
17520.5000000003	75.4409095578919\\
17525.5000000002	75.4427808463048\\
17536.5999999999	75.4446389571584\\
17541.8	75.4467874061533\\
17549.9000000001	75.4444444444444\\
17559.7000000002	75.4461895921366\\
17569.8999999998	75.447922595333\\
17577.5999999999	75.4461618983143\\
17582.2000000002	75.4480358087395\\
17584.2000000001	75.4502596065809\\
17586.0999999999	75.4477942932527\\
17590.9	75.4459666875107\\
17592.5	75.4481998112843\\
17597.2	75.4462332289613\\
17603.9999999997	75.4443567123568\\
17614.0999999999	75.4465147437863\\
17617.6999999999	75.4447206802211\\
17629.5	75.4469755411354\\
17631.3999999997	75.445084082466\\
17634.8000000003	75.4469829712672\\
17644.5999999999	75.4492850544356\\
17645.4	75.4475645348673\\
17654.9000000002	75.4494477485132\\
17656.0999999999	75.4471517087482\\
17660.3999999999	75.4452025707086\\
17669.4999999997	75.4470955765835\\
17672.1	75.4450492864499\\
17673.2000000002	75.4431826540601\\
17679.0999999998	75.4411964342278\\
17681.1000000001	75.4388842386263\\
17682.6999999999	75.4365824416948\\
17692.7000000003	75.439726894556\\
17697.7999999998	75.4417190740144\\
17699.7999999998	75.4399742371426\\
17707.4000000001	75.442044331498\\
17713.5000000002	75.44033962605\\
17714.7999999998	75.4381904498473\\
17717.1000000001	75.4402501523943\\
17719.0999999998	75.4385073818231\\
17726.6000000002	75.4404373064361\\
17730.6	75.4386459643443\\
17733.1000000002	75.4404168452394\\
17740.7999999999	75.4426212875333\\
17742.4000000001	75.4408905171199\\
17744.9	75.4432234432234\\
17749.7000000002	75.4453571307846\\
17752.0000000003	75.4434686600459\\
17756.2	75.4453348952203\\
17769.0000000001	75.447265196324\\
17774.3	75.4489603024575\\
17785.0999999998	75.4509367339136\\
17811.9999999998	75.4492732468378\\
17813.1000000002	75.4468596321829\\
17817.8999999998	75.4450555617914\\
17824.4999999998	75.4468543473626\\
17828.7000000002	75.4487121959975\\
17832.4000000002	75.4515631571569\\
17833.0000000001	75.4495853216771\\
17836.2000000002	75.4478226986539\\
17839.7999999999	75.4499744953727\\
17841.7999999999	75.4482426199004\\
17843.2999999998	75.4463835367699\\
17847.1999999999	75.4433443714175\\
17854	75.4414952307873\\
17859.8	75.4393921578508\\
17871.7	75.4372810796898\\
17876.8999999999	75.4399507747385\\
17884.9	75.4380766005032\\
17894.9000000002	75.4400670578374\\
17901.4000000001	75.437812473815\\
17903.7	75.4398507579397\\
17933.4999999997	75.4416291207566\\
17942.2000000003	75.4396036182652\\
17961.1000000002	75.4415072489589\\
17968.8000000002	75.4392311159837\\
17975.9999999998	75.4412803667091\\
18010.5000000001	75.4427947986186\\
18015.8999999999	75.4451598579041\\
18017.2000000003	75.4430464053992\\
18018.5999999999	75.4449544084756\\
18023.5000000001	75.4477462881999\\
18026.9999999997	75.4453017956299\\
18032.6999999999	75.4430814959407\\
18036.6999999998	75.4407655459949\\
18058.0000000001	75.4381690211041\\
18079.7000000003	75.4399938052412\\
18083.3	75.4382472322683\\
18087.4000000003	75.4404975812025\\
18088.7999999999	75.4385286004124\\
18095.3999999999	75.4408554613025\\
18105.9000000003	75.439080967635\\
18111.4	75.4371531899622\\
18123.6000000003	75.4349277465418\\
18126.6	75.4373382910292\\
18128.1000000003	75.4355093169758\\
18131.3000000003	75.4337778660225\\
18135.7000000003	75.4314670430861\\
18137.5	75.4333539167255\\
18146.6999999998	75.4353384618776\\
18150.5000000002	75.4393794144546\\
18158.0000000002	75.4412631277501\\
18165.0000000003	75.4430198567583\\
18166.4000000001	75.4405086285195\\
18168.2000000001	75.4385385534145\\
18172.5000000001	75.4366463797145\\
18179.4999999998	75.4340029483597\\
18182.5999999998	75.4321415411353\\
18203.7999999998	75.4349342723263\\
18205.6999999998	75.4369486647113\\
18207.1999999997	75.4351276685725\\
18209.5000000002	75.4371320622089\\
18218.3999999998	75.4348601696078\\
18220.1999999999	75.4372869821024\\
18221.7999999997	75.4350534247252\\
18224.8000000002	75.4330613611049\\
18233.0999999998	75.4349209134984\\
18234.6000000001	75.4331028204467\\
18242.9000000002	75.4349613550403\\
18244.1999999999	75.4328749253191\\
18255.1999999998	75.4350791277054\\
18270.9000000003	75.4381259920092\\
18277.2000000001	75.4400266997861\\
18278.2000000002	75.4380877871574\\
18280.6000000003	75.4407653973863\\
18281.6	75.438826804947\\
18290.7999999999	75.4407929626208\\
18300.3999999998	75.4427474659162\\
18303.6999999998	75.4449895650084\\
18305.4999999997	75.4430338257145\\
18313.8000000002	75.4448806644134\\
18319.5999999999	75.4466503272433\\
18320.7	75.4448495698878\\
18323.7000000002	75.4466868226023\\
18327.2999999998	75.4444165566311\\
18339.2	75.4461729727961\\
18342.9000000003	75.4440385978302\\
18345.5	75.4464285714286\\
18361.6000000003	75.4483517321381\\
18364.4	75.4466497862724\\
18368.4000000003	75.4487301630509\\
18368.8999999999	75.4466764657847\\
18381.7999999999	75.4448669615219\\
18385.4999999998	75.448176834044\\
18391.4999999997	75.4463994432241\\
18405.6000000003	75.4445633689564\\
18409.7000000001	75.446772914426\\
18414.7	75.4501813758499\\
18423.3000000003	75.4529565661061\\
18428.1	75.4506679979597\\
18429.8999999999	75.4487249050461\\
18437.0000000003	75.4505860466125\\
18447.4000000003	75.448705786692\\
18454.0999999998	75.4505749368707\\
18461.5	75.4522901590328\\
18462.5000000002	75.4498283015393\\
18465.6000000002	75.4523251217121\\
18479.0999999999	75.4502359409498\\
18497.0999999998	75.4519602966936\\
18497.6999999999	75.450053519878\\
18508.8000000002	75.4523499505643\\
18513.0000000001	75.4552181968444\\
18516.5999999998	75.4535095346363\\
18519.6999999999	75.4516787438309\\
18523.0000000002	75.449573775448\\
18527	75.4478574628517\\
18532.6000000002	75.4498804815272\\
18536.4000000001	75.451676422194\\
18539.5999999999	75.4499803125185\\
18541.9000000002	75.4519469312911\\
18547.4999999999	75.4539670900817\\
18565.6000000001	75.4558136779114\\
18575.0999999998	75.4575993798182\\
18576.9	75.4556709910104\\
18582.1	75.4576960747382\\
18588.2999999999	75.4594263088808\\
18597.8999999999	75.4575760834498\\
18601.3000000001	75.459374025611\\
18614.7	75.457700324473\\
18618.6	75.4596185555382\\
18629.3999999999	75.4577417536702\\
18632.5999999998	75.4598099040933\\
18647.8999999999	75.457957957958\\
18650.9000000002	75.4597608707308\\
18670.7999999998	75.4580657600866\\
18676.2000000003	75.4598073494215\\
18681.3	75.4622244585524\\
18684.1000000001	75.4600143436701\\
18687.3	75.4631462910838\\
18694.8	75.4612220445148\\
18705.0999999999	75.4635074738575\\
18708.9000000003	75.4652840878721\\
18713.1999999998	75.4629060614643\\
18715.9000000001	75.4648429151528\\
18717	75.4630792163316\\
18718.8000000001	75.4649044548558\\
18719.5	75.4631509220282\\
18722.2999999998	75.4652181344272\\
18724.9000000001	75.4632843791722\\
18730.4	75.4651504231067\\
18734.8000000003	75.4634398902583\\
18737.0000000001	75.4615175240566\\
18740.3	75.4637040831572\\
18743.7000000001	75.4660207641994\\
18746.7000000003	75.4640792028506\\
18756.1999999998	75.4621113972372\\
18762.2000000001	75.4603646674448\\
18766.0999999999	75.4585371572295\\
18767.9	75.4566283034953\\
18769.1999999998	75.4545987330375\\
18777.1000000002	75.4526766504058\\
18779.7000000003	75.4496853001629\\
18783.5000000003	75.4514576545497\\
18790.7000000002	75.4496881452626\\
18792.8999999998	75.4514978981536\\
18798.6999999998	75.4537523671724\\
18800.9000000003	75.4518376682091\\
18804.1	75.4538879612002\\
18815.3	75.4557436993101\\
18820.0000000001	75.4576224355875\\
18823.4	75.4556803995006\\
18825.5000000002	75.453637599864\\
18829.3000000002	75.4554048456138\\
18837.6999999998	75.457325165359\\
18842.5999999998	75.459992463925\\
18843.7000000001	75.4577102282979\\
18853.6	75.4594588860542\\
18868.7999999998	75.4612086555125\\
18873.9000000002	75.463600720568\\
18892	75.4654061750679\\
18907.3000000002	75.4672773623026\\
18909.4999999997	75.4653720861361\\
18911.8999999999	75.4674280879865\\
18916.5999999999	75.4698229606644\\
18934.1999999998	75.4678018199775\\
18938.4000000003	75.4700741875017\\
18944.4	75.4683417350682\\
18947.6	75.4703737129045\\
18958.0000000003	75.4685332390904\\
18961.1	75.4709617534755\\
18978.5000000003	75.4692126921901\\
18979.6000000003	75.4669462636396\\
18989.5000000003	75.4697308000168\\
19013.9	75.4717576522562\\
19019.3999999998	75.4699124582665\\
19021.0999999999	75.4721048093706\\
19025.7000000002	75.4738302725772\\
19028.6	75.4717873527882\\
19032.4000000002	75.4698541967687\\
19033.4000000002	75.4679906480679\\
19049.5000000002	75.4661515202419\\
19050.2999999998	75.4640322512913\\
19053.8	75.4622413259228\\
19058.7999999998	75.4644811610324\\
19061.1	75.4663924621745\\
19063.4000000001	75.4646313636006\\
19069.6999999997	75.4664443255829\\
19074.1000000002	75.4689580690147\\
19077	75.4711145824051\\
19078.2000000001	75.4689883270522\\
19080	75.467109711165\\
19087.1999999998	75.4700769621686\\
19098.7000000001	75.472280981004\\
19099.8000000003	75.470552201844\\
19102.7000000001	75.4732290554264\\
19121.6000000002	75.4713231564139\\
19131.0000000001	75.4692620915682\\
19133.7999999999	75.4671028906809\\
19135.6999999998	75.4690161895505\\
19145.6000000002	75.4670761580929\\
19148.4999999999	75.4650470530483\\
19173.2000000003	75.4632744493645\\
19178.2000000003	75.4649786477425\\
19191.4000000002	75.4667430893885\\
19204.5000000001	75.4647324078606\\
19211.2000000001	75.4665223071838\\
19219.6000000001	75.4647575144253\\
19229.1000000001	75.4628377675618\\
19231	75.4605820779883\\
19235.9999999998	75.4622818554697\\
19243.5000000002	75.4640503855827\\
19244.8000000001	75.4620704706182\\
19252.2999999997	75.4643576904698\\
19258.5000000003	75.4665448163418\\
19262.7000000003	75.4646261187366\\
19265.7000000002	75.4668895140612\\
19266.8	75.46517602728\\
19268.9999999999	75.4633065374096\\
19278.6	75.4615197082791\\
19280.3999999998	75.4632919270766\\
19284.8000000003	75.4652603850681\\
19287.9000000002	75.4635006221485\\
19289.6	75.4656630222347\\
19300.7000000001	75.467856254663\\
19308.3999999999	75.469870782298\\
19315.4999999999	75.4680154900702\\
19324.7000000001	75.4662402715681\\
19335.4	75.4679216984303\\
19343.3999999997	75.46617726885\\
19347.5000000001	75.4641402551221\\
19356.2	75.4658689935576\\
19366.1	75.4675672047175\\
19372.9	75.4658545398235\\
19376.5000000002	75.467832333846\\
19380.4000000002	75.4660612471298\\
19385.0000000001	75.4677561632388\\
19394.1999999998	75.4695967371857\\
19403.2999999998	75.4713091520043\\
19410.3999999999	75.4694624043688\\
19412.2	75.4671007557064\\
19430.8999999998	75.4701250578972\\
19436.5999999999	75.4680578493263\\
19440.7999999998	75.4702714380507\\
19455.5999999998	75.4729976305144\\
19471.2999999999	75.4712039195949\\
19477.7999999999	75.4732286334769\\
19480.6000000003	75.4757272582607\\
19489.5	75.4792299482801\\
19492.5999999998	75.477486443643\\
19502.1000000002	75.4755873696301\\
19525.3	75.4786073524742\\
19530.9999999999	75.4806436913435\\
19545.7000000002	75.4827123985716\\
19550.7999999997	75.4809241518293\\
19554.0999999998	75.4840392345378\\
19562.3999999999	75.4821725239617\\
19572.0000000003	75.4804032270426\\
19577.1000000002	75.4786179841857\\
19581.6000000001	75.4806783884954\\
19582.9000000001	75.4787315528775\\
19585.0999999998	75.4804648407982\\
19589.4000000003	75.4787003241533\\
19596.9000000003	75.4819615247232\\
19612.6999999998	75.4797887094143\\
19615.3999999999	75.4816344217583\\
19617.5999999997	75.4797963063968\\
19623.5	75.4820726064535\\
19626.9999999998	75.4838972644965\\
19632.8	75.4865557304321\\
19636.9000000001	75.4845444823547\\
19640.3999999997	75.4828033909524\\
19650.0000000002	75.4846031317907\\
19657.5999999998	75.4828896564705\\
19663.1000000002	75.4811017535294\\
19670.9999999999	75.4828148908805\\
19674.5999999999	75.4806934794431\\
19680.1000000002	75.4789077346775\\
19688.1000000001	75.4771893824728\\
19691.9000000001	75.4753199268739\\
19718.5999999998	75.4735352736235\\
19722.2000000001	75.4754769981189\\
19724.9000000002	75.4737642585551\\
19730.3999999998	75.471984997846\\
19736.8999999998	75.4739828748037\\
19740.6999999998	75.476171178473\\
19747.4	75.4779085960248\\
19760.0000000003	75.4803872450038\\
19764.9	75.4824184163926\\
19778.9000000002	75.4805601901006\\
19785.0000000002	75.4845818317825\\
19791.3999999998	75.4864462016522\\
19810.7	75.4845841662124\\
19825.3999999999	75.486620766185\\
19837	75.4883526321892\\
19839.6	75.486524493818\\
19843.4999999998	75.48428712532\\
19860.7999999999	75.4865086677844\\
19865.2999999997	75.4845107573973\\
19867.9000000001	75.4872156231125\\
19870.9999999998	75.4855040737554\\
19888.8000000002	75.4838125788757\\
19894.1000000002	75.4858199877351\\
19896.4000000003	75.4831251727691\\
19900.5999999999	75.4852844372308\\
19908.8999999999	75.4834496961173\\
19919.3000000002	75.4816912155989\\
19921.0999999999	75.4834046141799\\
19938.6	75.4863657109039\\
19966.6999999998	75.4838031131679\\
19979.2	75.4821239983383\\
19988.3999999999	75.4844035320309\\
19989.7000000002	75.4824960729972\\
19992.3999999999	75.4848068025509\\
20004.8999999998	75.4826293426643\\
20006.9999999997	75.4807043499558\\
20012.8000000002	75.4788161635745\\
20021.1999999999	75.4806131469984\\
20025.0000000003	75.4787741384562\\
20040.9	75.4767726161369\\
20051.4	75.4786424955739\\
20054.9	75.4769384193468\\
20065.4000000003	75.4748199646159\\
20075.0999999999	75.4727225631625\\
20078.6999999998	75.4746299579656\\
20087.5	75.4764133097035\\
20095.2000000001	75.4743646524312\\
20097.2999999999	75.4764297869376\\
20102.2000000003	75.4784278415903\\
20110.9999999998	75.4762295448782\\
20125.9000000003	75.4745105833251\\
20126.8000000001	75.4721293393419\\
20134.6999999997	75.470329975962\\
20140.2	75.4685878561888\\
20143.0000000001	75.4705085115995\\
20146.4	75.4741518377882\\
20152.2000000002	75.4722785984726\\
20161.2	75.4698357744788\\
20173.0999999999	75.4679475740091\\
20176.4000000002	75.4660124402151\\
20187.1999999998	75.4642770454692\\
20198.5999999999	75.4622822260839\\
20202.6000000003	75.4641706306583\\
20206	75.4663195767615\\
20215.9000000001	75.4644835773645\\
20226.4999999997	75.4664649520928\\
20236.8000000001	75.468574732296\\
20246.0000000001	75.4703374970982\\
20250.4999999999	75.4683811837674\\
20264.9000000004	75.46656797434\\
20278.8000000003	75.464645518248\\
20284.5000000003	75.4616802894807\\
20289.4000000001	75.4636634712536\\
20292.0000000001	75.4618792535026\\
20297.3000000002	75.4643451870683\\
20330.8000000001	75.4669001372295\\
20339.3999999998	75.4689151650729\\
20353.8000000002	75.471039948118\\
20369.3000000004	75.4730134417312\\
20372.7000000003	75.4712165239928\\
20377.4	75.4734388418599\\
20382.0999999997	75.471735141447\\
20387.0000000001	75.4697823623762\\
20388.5999999999	75.4677836252434\\
20399.5000000001	75.4696170513147\\
20401.8000000001	75.471402173327\\
20403.9999999999	75.4696360045285\\
20408.4000000003	75.4724747041674\\
20411.4000000003	75.4746099012811\\
20417.7000000003	75.4728717099785\\
20419.1000000002	75.4706354803322\\
20422.7999999998	75.4731208594274\\
20427.9999999997	75.4749585130286\\
20440.6000000002	75.4768672305743\\
20444.6000000003	75.4748174343473\\
20449.6	75.4729898238116\\
20454.7000000003	75.4747052036686\\
20465.4999999999	75.4725001954499\\
20468.4	75.4706011676478\\
20470.9000000001	75.4687118362562\\
20475.0000000002	75.4711820699288\\
20479.2000000003	75.4688880967611\\
20487.3999999997	75.4713849908481\\
20490.8999999998	75.4731345468742\\
20501.6000000002	75.4713023798027\\
20511.7000000001	75.4731422888289\\
20537.6000000003	75.4748584310804\\
20553.9999999997	75.4725334604775\\
20565.7999999998	75.474936667007\\
20570.9000000002	75.4766418744835\\
20572.4000000003	75.4740551707376\\
20578.4000000003	75.476346672498\\
20586.3000000003	75.4745851630203\\
20590.5000000003	75.4727885540004\\
20599.9999999997	75.4705074247212\\
20609.9000000003	75.4730713245997\\
20619.8999999997	75.4747817652764\\
20625.0999999998	75.4722378449663\\
20634.8000000002	75.4740754740755\\
20642.1000000001	75.4720911530748\\
20645.6	75.4699525809248\\
20648.4000000001	75.4718260406325\\
20660.4999999999	75.4736067684385\\
20675.0999999999	75.4754488469277\\
20698.8999999998	75.4736943813711\\
20701.7999999999	75.4718165965443\\
20707.2999999998	75.4696388730599\\
20711.4000000002	75.4715979045458\\
20714.5999999999	75.4734560481204\\
20733.3999999998	75.4754383003352\\
20747.2000000003	75.4729531071513\\
20754.7000000004	75.471216296953\\
20759.6999999998	75.4694168537269\\
20777.0000000001	75.4725154136044\\
20778.4000000002	75.4707991433453\\
20818.1999999998	75.4725409855752\\
20822.4000000002	75.4707647976948\\
20833.7999999997	75.4688272478989\\
20847.7999999999	75.4709107392111\\
20865.4000000001	75.4690757470466\\
20875.1999999997	75.4710111950487\\
20880.4000000004	75.4737673906276\\
20884.6000000002	75.4719962460557\\
20887.1000000003	75.4696656325405\\
20895.3000000001	75.467806311437\\
20900.4999999998	75.4657761021215\\
20904.3	75.4678440902394\\
20906.0000000002	75.466012312196\\
20914.9000000003	75.4678460435094\\
20922.2	75.4706700506159\\
20934.2000000002	75.468489512427\\
20939.4999999999	75.4665800683872\\
20949.8000000001	75.4647993546509\\
20957.7999999998	75.4665305207106\\
20971.6000000001	75.4683692785993\\
20991.6000000001	75.46649389997\\
21010.6999999998	75.4645230072153\\
21014.5	75.4627735003283\\
21018.4000000003	75.464471774865\\
21026	75.4624014914796\\
21030.4999999999	75.4605194335873\\
21047.1000000003	75.4632445170854\\
21056.2000000002	75.4615008334798\\
21061.9000000001	75.4633937897636\\
21064.5000000001	75.461675037741\\
21067.4999999999	75.463745277108\\
21071.6000000002	75.4618754063507\\
21076.5999999998	75.4643753528778\\
21091.3999999999	75.46642012185\\
21094.6000000003	75.4644531564801\\
21106.7000000003	75.4624102185078\\
21109.6000000001	75.46057025917\\
21115.5	75.462691090947\\
21117.3000000003	75.4605207080417\\
21134.9	75.4629761059853\\
21138.2	75.4611297975712\\
21149.2999999998	75.4593510927024\\
21156.9999999999	75.4616653511114\\
21159.4999999998	75.4598385602752\\
21162.5000000003	75.4618997665693\\
21176.5999999997	75.4635991443426\\
21186.2999999999	75.4653928935544\\
21205.2000000004	75.4636812494989\\
21208.1000000001	75.4660933035335\\
21216.9000000004	75.4677852665315\\
21226.7000000001	75.4659204401982\\
21229.7	75.4679742625931\\
21236.1999999999	75.4703032072442\\
21239.2	75.4681180641547\\
21249.3000000001	75.4651896053536\\
21271.4999999999	75.4635288365708\\
21277.3	75.4655174034421\\
21281.3	75.4635503303354\\
21292.9999999997	75.4657612090302\\
21300.8999999999	75.4640627200601\\
21304.3999999998	75.4662160576404\\
21306.9	75.4644013704416\\
21311.4000000001	75.462543697065\\
21313.1000000003	75.4645008726986\\
21316.5000000002	75.4627848718839\\
21320.0000000003	75.4649368436358\\
21321.8000000003	75.4627870874547\\
21323.7000000002	75.4607527738958\\
21329.8000000001	75.4588629107497\\
21332.4000000001	75.4571662955584\\
21341.3000000002	75.4589670780736\\
21343.7999999998	75.4571563772319\\
21359.4000000001	75.4554179639036\\
21370.4000000003	75.457289253878\\
21379.5000000003	75.455574472862\\
21390.4000000002	75.457329188191\\
21394.2999999997	75.4552593201959\\
21409.3999999999	75.4534202106542\\
21413.5999999999	75.4512298201619\\
21418.8	75.4529877818189\\
21425.6000000003	75.4547109312648\\
21433.2000000004	75.4564159508802\\
21436.8000000001	75.4544733613536\\
21444.0000000002	75.4561860838179\\
21447.2	75.4537867237368\\
21449.2000000002	75.4556092739623\\
21450.5999999999	75.4534817045597\\
21462.4999999999	75.4517160083121\\
21464.4000000003	75.4496960096904\\
21482.8000000001	75.4479143877223\\
21515.1999999999	75.4495637987851\\
21526.9999999997	75.4476915144167\\
21528.9000000004	75.4493938408658\\
21535.7000000002	75.4511093156511\\
21549.5000000003	75.4529086386754\\
21552.7000000003	75.4509854868045\\
21555.3000000001	75.4530187331249\\
21565.1000000002	75.4548995604029\\
21566.1999999998	75.4529056908232\\
21571.5000000001	75.4547645978972\\
21576.5000000002	75.4567448068741\\
21579.5	75.4587666129122\\
21585.8000000001	75.4608332290986\\
21590.5999999998	75.4588781280829\\
21608.7999999999	75.4605741152951\\
21614.6000000003	75.4629950913036\\
21618.4000000002	75.4608321576428\\
21627.1000000001	75.4591440408375\\
21633.5	75.4613194290363\\
21636.3000000002	75.4635706494611\\
21643.7000000001	75.4664153244809\\
21648.6999999999	75.4646908835594\\
21662.6999999997	75.4625440847905\\
21679.5000000002	75.4649532279193\\
21683.9000000001	75.4630142040214\\
21692.3999999998	75.4647919787945\\
21698.7999999998	75.4664983017572\\
21701.3000000001	75.4647165620651\\
21716.8999999998	75.466685085417\\
21729.8999999999	75.4647952139899\\
21736.4999999998	75.4667243267116\\
21738.9999999999	75.4649456509239\\
21756.5000000003	75.4630778706232\\
21765.6999999998	75.4610443907414\\
21781.6999999997	75.4639194189645\\
21786.0999999997	75.4619896999018\\
21793.5	75.463897658028\\
21809.3	75.4679175034618\\
21816.1000000002	75.4700635307707\\
21820.7000000002	75.4683604634111\\
21823.6	75.4707038678134\\
21848.5999999998	75.4726825852339\\
21856.6999999997	75.474451886827\\
21864.4000000001	75.4766859521142\\
21870.8999999998	75.4734580037493\\
21877.5000000001	75.4758291585914\\
21886.8	75.4775687740155\\
21890.0999999997	75.475783684023\\
21895.0999999997	75.47407650992\\
21903.9000000001	75.4766252739226\\
21905.9999999997	75.4744112370527\\
21910.0000000004	75.4724989844866\\
21911.2	75.4706475654115\\
21916.0000000001	75.4687193433138\\
21921.3000000001	75.4705447644767\\
21922.4000000002	75.4681263542023\\
21937.2000000004	75.4705456004157\\
21938.6000000003	75.4684644030868\\
21942.2999999999	75.4703222983812\\
21949.3999999999	75.4682339005444\\
21954.2000000002	75.4663095612249\\
21958.5000000001	75.4683814086508\\
21968.8999999999	75.4704356138195\\
21983.1000000003	75.4721787546854\\
21985.9000000002	75.4739379605203\\
21989.2	75.4721614603466\\
22001.3999999998	75.4703088425789\\
22010.4999999997	75.47318110365\\
22022.3000000002	75.4713382737576\\
22039.9	75.4732304900182\\
22042.7000000004	75.4713557261328\\
22056.3	75.4692515551042\\
22070.8999999998	75.4673553531784\\
22079.1000000001	75.4705786441538\\
22088.4000000002	75.4723045928877\\
22100.0000000003	75.474319120728\\
22113.9999999999	75.4762798395594\\
22124.0999999997	75.4743674347547\\
22128.2000000003	75.477555889969\\
22133.4999999997	75.4793616944374\\
22143.4999999997	75.477338824762\\
22152.5000000001	75.4751135306917\\
22155.3	75.4768589147567\\
22164.2999999999	75.4786955658624\\
22195.7000000002	75.4768920246173\\
22200.9999999999	75.4786924972186\\
22203.9	75.4769410916952\\
22212.4999999998	75.4751807532662\\
22229.3000000004	75.4734720685219\\
22235.5	75.4717659968699\\
22238.7000000002	75.4698994550066\\
22246.7000000003	75.4679324666918\\
22251.9999999997	75.4656863846558\\
22258.0999999997	75.4674681690343\\
22270.6999999998	75.4692242757333\\
22297.9	75.4673064848865\\
22310.2000000004	75.4655921256101\\
22345.1000000001	75.4636342480712\\
22348.2000000001	75.465695377277\\
22350.3000000001	75.4639737991266\\
22354.0000000003	75.4622194586228\\
22365.8000000001	75.4648818066789\\
22382.7000000003	75.4628554068302\\
22386.7	75.4645594725463\\
22390.3999999999	75.4668274491414\\
22391.0999999999	75.4649147879524\\
22421.6000000002	75.4670698475138\\
22432.7999999998	75.4690655242969\\
22434.9000000002	75.4673501225763\\
22437.4000000004	75.4651810584958\\
22447.6999999999	75.467083634031\\
22456.2000000004	75.4687994015043\\
22462.5000000002	75.4667758852493\\
22475.1000000001	75.4649569303054\\
22478.8000000003	75.4632121678552\\
22486.7999999997	75.4652708910521\\
22495.0000000002	75.4635453943303\\
22504.3000000001	75.4652423526066\\
22512.5000000001	75.4630740118867\\
22516.1999999999	75.464885438549\\
22525.0999999997	75.4665885319553\\
22532.4000000003	75.4687673360701\\
22544.0999999997	75.4668606559559\\
22546.9	75.4690202687719\\
22552.0999999997	75.4671384609927\\
22554.4000000004	75.4652064998116\\
22562.8999999999	75.4669148606125\\
22567.4	75.4651600753296\\
22585.1000000003	75.4671200609249\\
22587.0999999999	75.4653077849401\\
22592.8000000001	75.4635305781905\\
22601.6999999998	75.4652284331336\\
22610.6	75.467809488428\\
22612.2000000002	75.4660074384295\\
22626.9	75.467803951032\\
22669.9000000004	75.4715483017203\\
22673.1999999999	75.4698257421725\\
22682.2	75.4725049928799\\
22684.2999999998	75.4708081324611\\
22697.2	75.4688883699823\\
22710.3000000003	75.4671868395097\\
22722.0000000003	75.4652959013471\\
22740.9	75.4632601908447\\
22744.4000000001	75.4652773197916\\
22765.7000000001	75.4671480905569\\
22774.6	75.4653189723685\\
22784.9000000001	75.4671933289445\\
22829.3000000003	75.4689128930239\\
22838.7999999999	75.4707976303587\\
22858.5999999998	75.4727959157782\\
22861.7000000001	75.4704353987875\\
22870.2000000001	75.4729933581982\\
22872.9999999998	75.4703122882338\\
22889.7000000002	75.4685493101731\\
22907.4999999999	75.4666573538913\\
22916.5000000004	75.464946807118\\
22924.5000000003	75.4630397040734\\
22933.3999999997	75.4612248457497\\
22941.9000000002	75.463342341557\\
22952.4999999999	75.4655246028772\\
22961.7000000003	75.4675156128875\\
22972.3999999999	75.4693655457612\\
22979.4999999997	75.4717227453916\\
23011.7000000004	75.4738873099888\\
23013.5999999999	75.4720014599999\\
23016.3	75.4700995811682\\
23023.8000000003	75.4681005390051\\
23036.4000000003	75.4697979293729\\
23044.9999999999	75.4681038485405\\
23061.2000000003	75.4662573228743\\
23066.3999999998	75.4644180955065\\
23073.2000000001	75.4625476199763\\
23082.4999999999	75.4603034320224\\
23096.6999999997	75.458505074296\\
23100.8000000002	75.4602634529391\\
23105.2999999998	75.458550814961\\
23110.8000000004	75.4566027285826\\
23114.6999999997	75.4585806496271\\
23119.1000000003	75.4567632097996\\
23126.3999999998	75.4549975136748\\
23135.9000000001	75.457295988935\\
23138.0000000002	75.4552015939079\\
23150.1999999999	75.4577694457523\\
23164.0999999997	75.4560917277523\\
23172.5999999999	75.4543061447307\\
23185.0999999998	75.4524438003554\\
23188.6000000003	75.4505427212392\\
23193.9	75.4522721393464\\
23202.3999999997	75.4548001292964\\
23206.4	75.4529980824338\\
23212.3	75.4510520239183\\
23218.1000000003	75.4490012145644\\
23231.4000000003	75.4471299743882\\
23236.9999999999	75.4448704872811\\
23249.7999999999	75.4467761151661\\
23254.5999999997	75.4449638137667\\
23257.2000000002	75.4429791936295\\
23262.9000000002	75.4446975884452\\
23268.0000000004	75.4427735827163\\
23273.9000000001	75.4447022428461\\
23291.9000000003	75.4465052378499\\
23296.3	75.4442746518775\\
23304.2999999998	75.4424057259573\\
23319.2000000001	75.4443744023191\\
23324.0000000001	75.442567987618\\
23340.8000000002	75.4443916044368\\
23352.2	75.4461016687864\\
23357.5000000004	75.4478199815049\\
23372.2000000001	75.4495706455933\\
23383.6999999999	75.4513808705172\\
23409.6999999997	75.4495980315936\\
23413.7000000001	75.447812828332\\
23424.3000000001	75.4495312580045\\
23434.8000000003	75.4515700941758\\
23446.3000000003	75.4533745052545\\
23450.4000000003	75.4512696957421\\
23452.7999999998	75.4490915835568\\
23458.5000000004	75.4524992966332\\
23468.1999999999	75.450714367892\\
23485	75.4533725638809\\
23495.5000000002	75.4515739117111\\
23504.5000000001	75.4537409698527\\
23516.4000000001	75.4555312227585\\
23529.6	75.4535756936977\\
23537.9000000002	75.4554337666752\\
23539.9000000004	75.4536958368734\\
23552.4999999998	75.4519670864363\\
23561.8999999998	75.449452508276\\
23573.6999999996	75.4477428331453\\
23579.1999999997	75.4458359662925\\
23595.4999999998	75.44415060435\\
23602.3999999999	75.4462451011545\\
23607.8000000003	75.4480491699812\\
23613.4000000001	75.4462489677515\\
23620.7	75.4445234708393\\
23622.6000000004	75.4426886003717\\
23635.3000000004	75.4448835221744\\
23643.1000000004	75.4474859579076\\
23655.1	75.4502181338564\\
23673.9	75.4519726281997\\
23677.8999999998	75.450206943154\\
23682.3000000004	75.4484342803094\\
23700.2999999997	75.450203372095\\
23710.8999999998	75.4518999620429\\
23738.3000000001	75.453695278536\\
23750.3999999998	75.456095661144\\
23752.6999999998	75.4542622343471\\
23761.8000000002	75.4560872657489\\
23787.1000000003	75.4578092419453\\
23791.0000000004	75.4597307396463\\
23797.5000000003	75.4580293811141\\
23806.5999999999	75.4556490399761\\
23811.2000000003	75.4574508741648\\
23824.9999999997	75.455297144608\\
23827.7999999996	75.4531452624864\\
23831.2	75.4511923394863\\
23841.0999999998	75.452997332349\\
23843.7999999997	75.4511636099799\\
23853.8000000003	75.4492975991347\\
23859.3000000003	75.4474127597509\\
23868.8000000003	75.4492247233848\\
23871.8000000001	75.446864304894\\
23875.1999999997	75.4440781895934\\
23897.3999999997	75.4417826132441\\
23915.5999999999	75.439982940077\\
23918.5999999998	75.4418091284225\\
23920.2000000004	75.4401073565131\\
23926.2000000001	75.4383251902718\\
23941.3999999998	75.4363761669068\\
23944.1000000004	75.4387283768094\\
23950.4000000004	75.4368384793637\\
23966.1999999999	75.4346728531313\\
23971.9999999998	75.4364448671581\\
24003.5000000002	75.43451815561\\
24011.8999999998	75.4327003165084\\
24016.2000000001	75.434600667047\\
24027.9999999999	75.43667622492\\
24034.4000000002	75.4348956708065\\
24037.9000000001	75.4330643148348\\
24044.9999999997	75.4311689283887\\
24049.3999999999	75.4294268072101\\
24059.5999999998	75.4311151011858\\
24075.0999999999	75.4328105270153\\
24088.5999999999	75.4357852437035\\
24105.6000000004	75.4377595340521\\
24120.8999999998	75.4396583889557\\
24130.6999999998	75.4417590796824\\
24158.6000000001	75.4436290032162\\
24177.6000000004	75.4418327632488\\
24182.2999999997	75.4399894137885\\
24185.7999999997	75.4377550556316\\
24193.8000000004	75.4400902706881\\
24202.1999999999	75.4382847911149\\
24205.4999999999	75.4403939584228\\
24207.6999999999	75.4384950305273\\
24215.7	75.4404149357031\\
24222.8	75.4430724644861\\
24229.4999999997	75.4449103575792\\
24257.0999999997	75.4472898768201\\
24275.7999999998	75.445606548058\\
24281.5999999999	75.4481770222019\\
24289.8	75.4498783444971\\
24290.7999999997	75.4480072784458\\
24304.5999999997	75.4463128530696\\
24321.3000000002	75.4487817313148\\
24367.0999999997	75.4506057322959\\
24376.2999999997	75.4487947358921\\
24379.3999999999	75.4469944010336\\
24398.7	75.4487925635687\\
24401.7999999998	75.4506821190153\\
24403.2	75.4488122508021\\
24408.8000000004	75.4470705357472\\
24436.8	75.4490135819191\\
24442.6000000002	75.4511571962181\\
24449.4999999998	75.4486780969832\\
24468.5000000004	75.4468992913366\\
24472.0000000002	75.4487763616527\\
24477.4	75.4505157797978\\
24497.0000000002	75.4521963824289\\
24497.7000000004	75.4504486117121\\
24537.5999999998	75.4524670201364\\
24539.7999999997	75.4505927082018\\
24565.1999999997	75.4487020309135\\
24567.5000000003	75.4469301030626\\
24583.3000000003	75.4488801386301\\
24600.3	75.4471472008585\\
24610.0000000001	75.4491042295643\\
24619.1999999998	75.4473116619888\\
24635.6000000001	75.4510730362847\\
24646.4999999996	75.453003659734\\
24650.7999999999	75.4552572117042\\
24668.3999999999	75.4569592800535\\
24682.1	75.4588326810414\\
24685.7000000001	75.4607912241046\\
24688.7000000004	75.4589125433395\\
24707.6	75.4606863447427\\
24740.2000000004	75.4590688067647\\
24751.7999999997	75.4572376262024\\
24755.7	75.4554488241139\\
24769.2	75.4579257387169\\
24790.4999999996	75.4604567860399\\
24804.4999999997	75.4622126541045\\
24811.6000000001	75.460367487919\\
24839.3000000001	75.4583444044542\\
24841.2000000002	75.4565984871967\\
24852.8	75.4583972091788\\
24863.3999999997	75.4563918997728\\
24865.9000000001	75.4584573312957\\
24882.2000000004	75.4604678827922\\
24887.5999999999	75.4621760950188\\
24890.3000000001	75.460016713271\\
24899.2000000003	75.4619607780138\\
24915.5999999999	75.4596499395963\\
24951.0999999996	75.4576934175511\\
24958.1999999997	75.4558603751057\\
24962.4000000001	75.4579869804707\\
24983.7000000001	75.459697884229\\
25000.5000000001	75.4573890226635\\
25003.6	75.4556325663802\\
25011.6000000002	75.4538875806123\\
25019.0000000003	75.4559516529372\\
25035.3999999998	75.454055241557\\
25042.8000000002	75.4561173027085\\
25067.1000000001	75.4543786302419\\
25074.3999999999	75.4563401064827\\
25078.6000000004	75.4580580333111\\
25085.8000000004	75.4599197158563\\
25087.3000000004	75.458198139305\\
25094.0000000002	75.4563821774839\\
25104.7000000004	75.4544947579746\\
25114.3000000003	75.4563119166693\\
25119.5999999996	75.4543246933682\\
25124.6999999996	75.4521429026301\\
25131.5000000001	75.4544079963074\\
25137.9	75.4527010899833\\
25144.1000000001	75.4507997868296\\
25152.2999999998	75.4528394904661\\
25172.4999999997	75.4510857042975\\
25192.6999999999	75.4493347305579\\
25196.6999999999	75.4516446532893\\
25227.7	75.4497023125282\\
25234.8000000003	75.4514580996953\\
25236.5000000003	75.449149251484\\
25238.8000000004	75.4474244123159\\
25242.3000000001	75.4456786993313\\
25278.7999999998	75.4439473236573\\
25291.6	75.4456995773317\\
25333.7000000004	75.447425968469\\
25337.1999999997	75.4492388691771\\
25338.6000000001	75.447438108506\\
25351.8000000002	75.4456273494294\\
25389.2000000001	75.4439862461746\\
25399.2	75.4414491737961\\
25420.1999999998	75.4432481127288\\
25425.2000000003	75.4413910553661\\
25442.1000000003	75.4435544096029\\
25451.4999999996	75.4455515566801\\
25461.1999999999	75.4478365205233\\
25520.5000000003	75.4461102011708\\
25579.4999999998	75.4484041970946\\
25589.1000000003	75.4466728150939\\
25598.1999999998	75.4487602692366\\
25603.3000000004	75.447010943859\\
25610.6000000002	75.4489334535956\\
25621.5999999998	75.4508873337835\\
25629.5999999997	75.4491859054144\\
25656.8000000004	75.4475404277212\\
25661.1	75.4497061711845\\
25665.0999999999	75.4515842463725\\
25667.2999999999	75.4497923435954\\
25675.3000000001	75.4480942848018\\
25692.6000000001	75.4498359456188\\
25703.0000000001	75.4480977002774\\
25709.7000000003	75.4463278594155\\
25735.8000000002	75.4494694182057\\
25739.8000000002	75.4474570608277\\
25743.3000000003	75.4457453172464\\
25750.7000000002	75.4438697050189\\
25756.0999999998	75.4420294919282\\
25760.2000000002	75.4439971584182\\
25765.3000000001	75.4422597747367\\
25775.8000000002	75.4398488510585\\
25786.7999999999	75.4425696768514\\
25805.6999999999	75.4442799680692\\
25823.9000000005	75.4406753407683\\
25827.6999999997	75.4388681962846\\
25838.3000000004	75.4419778314447\\
25858.9999999996	75.4399805097625\\
25884.4000000001	75.441673588441\\
25894.2999999997	75.4437252842313\\
25903.8000000004	75.4419218727682\\
25914.7000000003	75.4441477456897\\
25918.9999999998	75.4420485279196\\
25922.1000000002	75.4438280701484\\
25932.4999999999	75.4455781525956\\
25938.3999999999	75.4434527825433\\
25949.1	75.4412467436376\\
25954.7000000004	75.4430779663107\\
25971.9999999999	75.4448042322338\\
25975.3999999996	75.4430136089777\\
25982.5999999998	75.4405816177687\\
25986.4999999999	75.4388800381735\\
26028.3000000002	75.4372147346744\\
26050.5000000001	75.435498606558\\
26054.7999999999	75.4380174170693\\
26062.4999999999	75.4399023888637\\
26074.8000000004	75.4380649590219\\
26077.9999999997	75.4399285224\\
26102.1000000003	75.4380856786018\\
26112.7000000003	75.4400140927055\\
26117.2999999999	75.4382136047233\\
26126.5000000004	75.4399730542818\\
26157.0999999996	75.4381202881042\\
26160.2000000001	75.4402663578017\\
26164.3000000001	75.4383819235297\\
26183.0999999999	75.4365394604174\\
26233.9999999999	75.4380748720177\\
26259.1	75.4364184742871\\
26263.8999999998	75.4344349680171\\
26272.5	75.4325799502143\\
26284.2999999998	75.4302932537931\\
26293.8000000004	75.4285214441372\\
26300.6999999999	75.4304051587784\\
26304.5000000004	75.4286322544346\\
26317.7	75.4303171237717\\
26324.6999999997	75.4281134139671\\
26337.4000000002	75.4300901756051\\
26349.0999999999	75.4280964886979\\
26355.5000000003	75.4260954028745\\
26359.7999999996	75.4240342338173\\
26372.6000000003	75.425724328567\\
26387.7	75.4238701217987\\
26420.5000000003	75.4222084282719\\
26424.9000000004	75.4198675496689\\
26435.3000000002	75.4215937719876\\
26439.7999999998	75.4238858694624\\
26452.4999999999	75.421697678111\\
26473.1999999997	75.4235399440191\\
26478.0999999996	75.4216676360176\\
26487.1999999996	75.4199182249607\\
26496.6000000002	75.4176935240992\\
26501.1	75.4196036405899\\
26504.9	75.4178456894926\\
26508.6	75.415995503363\\
26529.6999999997	75.4182089574742\\
26570.1999999997	75.4165365088087\\
26581.3999999999	75.4144799954856\\
26607.2000000002	75.4161451932364\\
26628.9999999997	75.4178699242558\\
26639.2000000004	75.4160206912344\\
26642.9999999998	75.414272363201\\
26675.1999999997	75.4124602160051\\
26696.1000000004	75.4107326136304\\
26705.5000000002	75.4089030016176\\
26713.3000000002	75.4108424985213\\
26741.1999999998	75.4125640862636\\
26749.6000000004	75.4143037118173\\
26753.5999999999	75.4123728680519\\
26756.6000000002	75.4106448104587\\
26762.6000000004	75.412421018806\\
26764.4000000001	75.4107119505315\\
26771.8	75.4126528188138\\
26795.6000000004	75.41471206201\\
26799.8000000002	75.4129679588357\\
26811.2999999997	75.4149354379107\\
26813.7999999997	75.4168546910371\\
26832.5999999999	75.4187987045657\\
26837.5000000003	75.4206784511283\\
26857.6000000002	75.4189673724854\\
26864.6999999996	75.421369226646\\
26868.7000000001	75.4194456023343\\
26874.5	75.4176806352467\\
26893.0000000003	75.4200891678535\\
26902.6999999997	75.4218891713874\\
26928.8999999997	75.4201789891938\\
26970.1000000005	75.421761796353\\
26986.8000000004	75.4199259640789\\
27030.5000000004	75.421929220957\\
27034.3000000002	75.4198354688841\\
27049.6000000004	75.4222782507754\\
27053.6000000004	75.4244336264541\\
27084.7000000002	75.4279152882798\\
27094.3999999997	75.430068833158\\
27106.6000000001	75.4319042893454\\
27111.5999999998	75.4301648365835\\
27124.8000000003	75.4284808423257\\
27126.9000000004	75.4263280126811\\
27130.7000000002	75.424609668716\\
27133.3999999998	75.4263180201596\\
27135.5000000002	75.4245345597665\\
27142.3000000004	75.4262703371846\\
27146.3000000005	75.4243656617452\\
27153.6000000004	75.4225022740916\\
27159.0000000003	75.4207613654355\\
27162.8000000003	75.4190458308943\\
27166.6999999998	75.4222065167778\\
27182.9000000004	75.4247139756465\\
27186.9000000005	75.4228123735609\\
27195.1999999999	75.4211205612735\\
27204.2000000004	75.4230029811464\\
27214.1000000004	75.4212874161284\\
27238.5999999999	75.4235701410126\\
27256.6000000005	75.425858596235\\
27266.8000000004	75.4240489384565\\
27276.2999999998	75.4223431244592\\
27306.2	75.4239864060675\\
27310.7000000002	75.4258388622816\\
27317.3000000003	75.4240886760819\\
27321.0999999997	75.4220166024918\\
27324.5999999998	75.4200412081377\\
27340.2999999999	75.4217202381823\\
27348.4000000001	75.4198584931532\\
27361.8000000004	75.4223939127034\\
27387.1000000003	75.4206344569726\\
27390.9	75.4189332262422\\
27398.4999999996	75.4224668413715\\
27421.6999999996	75.424297456768\\
27425.8999999999	75.4225917013053\\
27472.8	75.4241452485904\\
27476.3000000004	75.4221804894382\\
27488.0999999998	75.4203621917768\\
27508.6000000002	75.4226844598254\\
27534.7000000004	75.4209218879382\\
27536.3999999997	75.419170918599\\
27560.2999999997	75.4172653517365\\
27564.5999999999	75.4192862610513\\
27573.0999999996	75.4214237012751\\
27579.7	75.4196912232866\\
27583.0000000002	75.4215443514326\\
27587.0999999999	75.419759888644\\
27600.4000000004	75.4214597561638\\
27613.1000000001	75.4193646516883\\
27619.9000000003	75.4174511223751\\
27626.9000000003	75.415716509212\\
27637.7000000001	75.4137449435194\\
27657.4999999998	75.4114601411547\\
27664.2	75.4134389809249\\
27667.4999999996	75.4116728592288\\
27683.3000000001	75.4137858789022\\
27687.3000000003	75.4119202236396\\
27712.3000000003	75.4138941412509\\
27725.7999999999	75.4157664854883\\
27727.4	75.4139392300063\\
27735.5000000003	75.4121057413577\\
27764.3000000005	75.4138393050093\\
27776.2000000003	75.4121319254184\\
27778.7	75.4100249110833\\
27786.9999999999	75.4080130708134\\
27795.2000000003	75.4062737225358\\
27805.4000000003	75.4081027134919\\
27808.6000000003	75.4062577538684\\
27810.6999999998	75.4045191076848\\
27836.1000000001	75.4064850805785\\
27847.4000000005	75.4046144178113\\
27851.4000000003	75.4063515430049\\
27868.5999999996	75.4082537039761\\
27889.9000000003	75.4062387952671\\
27896.0999999997	75.4091955176691\\
27903.5000000004	75.4074743043908\\
27909.6999999999	75.4050548552838\\
27955.9000000003	75.4067105451424\\
27992.8	75.4084071318084\\
28028.7000000003	75.4067245119306\\
28038.3000000001	75.4087251769002\\
28055.8999999997	75.4070430567436\\
28061.4999999996	75.4051800324999\\
28094.9999999998	75.4035401190955\\
28112.0000000004	75.4052525424995\\
28114.9000000001	75.4035212520007\\
28123.3000000001	75.4016228478776\\
28139.8999999998	75.3997867803838\\
28185.6000000004	75.3981628982072\\
28196.5999999997	75.3999581511312\\
28204.3999999996	75.4017975854917\\
28244.0999999997	75.3999051132622\\
28250.6999999996	75.3978648392258\\
28309.3000000002	75.3996905621454\\
28330.0000000005	75.4017811444365\\
28334.4000000004	75.3999541195362\\
28337.5999999999	75.40167338916\\
28356.9000000004	75.4043093416088\\
28367.5000000002	75.402571948279\\
28378.2	75.4047987370632\\
28397.4000000002	75.4066379082666\\
28407.1999999999	75.4049135257487\\
28420.1000000003	75.4030583880479\\
28430.9000000005	75.405367380676\\
28441.4999999998	75.4036341134113\\
28445.9	75.4014624200239\\
28457.8000000004	75.4036664687134\\
28479.3000000004	75.4053807313356\\
28502.1	75.408214102771\\
28508.9000000003	75.4063629029429\\
28521.7999999999	75.4090716256631\\
28529.5000000002	75.4072962817565\\
28534.3	75.4054754962431\\
28544.7999999996	75.403662300446\\
28586.4999999997	75.4052598070425\\
28602.2999999997	75.4034626464912\\
28612.0000000003	75.4055102561504\\
28632.8999999998	75.4035553382461\\
28657.1000000004	75.4016442639197\\
28672.5999999999	75.4034325333854\\
28701.5000000003	75.4052038910723\\
28716.5000000002	75.4069075029774\\
28735.6	75.4096820331504\\
28748.1000000004	75.4116779485324\\
28766.7	75.4098474630477\\
28773.9999999997	75.4077451597096\\
28797.3999999996	75.4058512023613\\
28805.8999999998	75.4040824828161\\
28817.3000000003	75.4023610735181\\
28844.1000000003	75.4040673688298\\
28872.3000000001	75.4062703481526\\
28895.9	75.4080149501661\\
28902.7999999999	75.4062741109024\\
28926.5000000005	75.4042991571771\\
28966.3999999999	75.4026202682409\\
28977.3000000002	75.4008296120425\\
28979.3000000003	75.3990765854365\\
29014.4000000003	75.397129021696\\
29020.9000000002	75.3988491092657\\
29034.3999999997	75.4013328970707\\
29055.1999999997	75.3993247359346\\
29064.7000000003	75.3970438468526\\
29080.0999999999	75.3952861397102\\
29091.4999999998	75.3935844023704\\
29106.3999999996	75.3917509834573\\
29131.3999999999	75.3936460532413\\
29148.4999999999	75.3953877716254\\
29187.6000000004	75.397170726025\\
29192.3999999997	75.3950500984842\\
29212.3000000001	75.3967493256288\\
29237.5000000001	75.3988015432183\\
29240.4999999998	75.3968796809915\\
29281.6999999996	75.3952967372224\\
29296.4999999999	75.397486397739\\
29307.6999999998	75.4000641467459\\
29339.3999999997	75.3983537551765\\
29362.2000000005	75.400087867776\\
29371.5999999996	75.4018323760627\\
29375.1	75.3999972766143\\
29380.4999999998	75.3980517756615\\
29392.7	75.3963555700716\\
29402.8	75.3946039336256\\
29434.1000000005	75.3963076964891\\
29452.6999999996	75.3945295523685\\
29460.3	75.3964644064575\\
29481.9999999997	75.3982925232599\\
29492.1000000004	75.4002753270356\\
29503.8000000001	75.3985066381055\\
29523.8999999998	75.4003522557919\\
29534.0999999998	75.4020762370404\\
29543.4000000004	75.4003418687698\\
29579.4999999996	75.4019662199624\\
29585.8000000002	75.4038241189215\\
29605.7999999997	75.4055779422345\\
29608.9000000002	75.4034246344017\\
29616.9999999998	75.401710498327\\
29635.7999999998	75.4034802384945\\
29655.1	75.4016159054736\\
29658.7999999999	75.4036730964398\\
29664.7999999999	75.4019059561974\\
29671.1999999998	75.400134136354\\
29715.8999999997	75.398438551622\\
29719.8999999997	75.3967025572005\\
29725	75.3985688862275\\
29736.8000000003	75.4002602826791\\
29746.6000000003	75.4019773621948\\
29763.8000000004	75.4037609318671\\
29798.3999999998	75.4054734298706\\
29805.1999999997	75.4073939869755\\
29807.0999999996	75.4052712096406\\
29834.9000000002	75.4070722306016\\
29844.3000000005	75.4051011244991\\
29860.6000000003	75.4068056006725\\
29876.7999999996	75.4087606143877\\
29885.6999999998	75.4070495017701\\
29891.6999999998	75.4052950976522\\
29925.4999999999	75.4036677627182\\
29948.3000000001	75.4053638925619\\
29953.4999999997	75.4072966187704\\
29971.5000000001	75.4093875535507\\
29976	75.4114110908357\\
30040.0999999999	75.4099506661074\\
30048.1999999997	75.4079265715531\\
30051.5000000003	75.4096287718458\\
30066.6000000003	75.4113354641514\\
30068.4999999998	75.409563464877\\
30071.7999999999	75.4115968728281\\
30086.9999999998	75.4133831442711\\
30117.3999999998	75.4116377521375\\
30145.9000000002	75.4136535527101\\
30159.6000000004	75.4118907018306\\
30163.2000000001	75.4141622435211\\
30206.8000000001	75.4125712999348\\
30213.0000000004	75.4143070389996\\
30250.7999999998	75.4126323514342\\
30259.4000000002	75.4143326889076\\
30282.7999999996	75.4125265413814\\
30296.3999999998	75.4106910039113\\
30302.7999999998	75.4089542585033\\
30352.8000000003	75.4072922191949\\
30363.5999999996	75.4094527346799\\
30394.1000000003	75.4078080686447\\
30403.2	75.4056303098677\\
30407.1	75.4074692835907\\
30416.0999999998	75.4094857345757\\
30422.4999999997	75.407756076075\\
30429.6999999997	75.4106172239055\\
30457.0999999999	75.4127102951026\\
30468.5000000001	75.4150174277781\\
30478.5999999996	75.4133214343131\\
30486.1000000005	75.4154338684388\\
30492.5000000002	75.4137069321737\\
30501.9999999999	75.4118568885421\\
30523.4999999999	75.4101744224141\\
30541.5999999998	75.4083760890848\\
30547.9999999995	75.4102546475886\\
30561.4999999998	75.4119548714727\\
30581.5000000003	75.4136474219792\\
30599.1	75.4153703364794\\
30636.9000000003	75.4137154421125\\
30700.4999999997	75.4115554744858\\
30735.9000000005	75.4131962519521\\
30742.3000000003	75.4153872176538\\
30753.2000000003	75.4136954408145\\
30765.2999999996	75.4119887926047\\
30772.4999999996	75.4138421842808\\
30781.2000000003	75.4120196352981\\
30788.9999999996	75.4094793287235\\
30799.8	75.4073876863237\\
30805.8000000005	75.405685274574\\
30833.7999999998	75.4075871038046\\
30846.4999999995	75.4092833569988\\
30869.0000000003	75.4110097152168\\
30874.8999999996	75.4092307692308\\
30880.7999999996	75.4110145753524\\
30896.3000000003	75.4129930995197\\
30901.0000000001	75.4112313153901\\
30910.8999999996	75.4129597877778\\
30920.8000000002	75.4111296889806\\
30925.0999999996	75.4093748787397\\
31000.0000000004	75.4078212650927\\
31008.0000000004	75.4061035664875\\
31031.8000000002	75.4043419835717\\
31040.8999999999	75.4060758351857\\
31044.3000000003	75.4042597054541\\
31050.8999999997	75.4024024991144\\
31066.9999999999	75.4006650121833\\
31081.4000000004	75.4027315284011\\
31111.0000000002	75.4010626432367\\
31127.5	75.3993240725273\\
31140.0999999999	75.4015709597241\\
31142.8999999998	75.3996082586777\\
31153.0999999996	75.401563884288\\
31194.6000000003	75.403193491202\\
31229.3000000005	75.4013845927235\\
31235.5999999998	75.3996228674241\\
31241.9	75.3978618526343\\
31271.8999999998	75.3962010744436\\
31314.7000000004	75.3982142629044\\
31323.3999999996	75.399939342666\\
31330.3000000005	75.401846130276\\
31341	75.4000338214038\\
31415.0999999998	75.3985331941226\\
31424.2000000003	75.3967471033563\\
31432.1	75.3949771253683\\
31444.7999999998	75.3972822301868\\
31452.4999999996	75.3953568226474\\
31458.3000000002	75.3935355898583\\
31501.6000000005	75.3952961268757\\
31509.9999999997	75.3932865969959\\
31555.9000000003	75.39105083027\\
31585.5000000004	75.3891013626463\\
31589.0999999999	75.3909564028212\\
31603.2	75.3927596168755\\
31631.0000000005	75.3909917770801\\
31641.1999999998	75.3929200127681\\
31659.9000000003	75.3910296904612\\
31663.0000000002	75.3928074004125\\
31670	75.3909839249008\\
31691.9	75.3890571753124\\
31702.2000000002	75.3913753891674\\
31708.1000000002	75.3934313521423\\
31711.8000000002	75.3912569098666\\
31718.5	75.38951908344\\
31748.8000000002	75.3868637968559\\
31771.2000000005	75.3887942891855\\
31783.5999999998	75.3870694727171\\
31795.5999999998	75.3853508493287\\
31808.4000000004	75.3836238741217\\
31815.8000000002	75.3818059523697\\
31819.9000000003	75.3837209302326\\
31848.0999999999	75.3819682117043\\
31867.3999999999	75.3849533223504\\
31881.5999999997	75.3871343121603\\
31901.9999999998	75.3853194617282\\
31919.4999999998	75.387536184664\\
31934.2999999998	75.3857908712861\\
31946.0000000003	75.3876059988543\\
31987.8999999999	75.3892084531699\\
32004.7999999999	75.3875187861859\\
32011.1000000001	75.389551157095\\
32048.7999999996	75.3876108072352\\
32061.2999999997	75.3894090713444\\
32098.2000000002	75.3874815800214\\
32108.1000000003	75.3857270105456\\
32125.9000000002	75.3838012824503\\
32148.1999999996	75.3856346991909\\
32161.9000000004	75.3877246439898\\
32165.6000000003	75.3858924257827\\
32173.4000000003	75.3834677607348\\
32184.8999999995	75.3854279944073\\
32212.6999999997	75.3874236328416\\
32234.8999999996	75.3891732588801\\
32245.3999999996	75.3909847885752\\
32263.1999999999	75.3927837511972\\
32286.9000000001	75.3910242512466\\
32320.2000000002	75.3928645464306\\
32333.6000000003	75.395021293573\\
32335.7000000003	75.3932174246501\\
32344.4000000002	75.3914885065467\\
32363.6	75.3897113123654\\
32377.7	75.3914719344736\\
32384.3000000004	75.3896938031892\\
32391.8	75.3879827981687\\
32398.5	75.3899859870488\\
32418.6000000001	75.3929059462594\\
32426.5000000002	75.3911911825477\\
32471.9000000002	75.3892584380389\\
32484.5999999999	75.3911841574649\\
32510.0000000003	75.3931855023516\\
32518.8000000005	75.3949241825523\\
32524.4999999999	75.3930870787035\\
32555.0000000003	75.3912597411773\\
32583.1999999998	75.3895400404502\\
32589.0000000004	75.3877830317499\\
32613.7999999996	75.3859550682378\\
32618.7999999997	75.3842097679566\\
32627.3999999997	75.3861006819401\\
32639.6999999999	75.3880232109265\\
32651.2999999997	75.3860477651801\\
32690.2999999999	75.3842106551159\\
32723.7000000005	75.3821377712857\\
32743.5000000005	75.3838918139728\\
32787.6000000005	75.3858916605922\\
32823.1000000005	75.3841794828049\\
32827.2999999995	75.3867196305525\\
32834.3	75.3846575542724\\
32843.4000000002	75.3829524867934\\
32848.4000000005	75.3848729774571\\
32865.5000000002	75.3830753127891\\
32874.1000000004	75.3813020544987\\
32879.8000000004	75.3794871638904\\
32893.6999999997	75.377730757772\\
32905.8999999996	75.3759192852367\\
32919.2999999999	75.3780445573127\\
32932.8000000003	75.3759917893656\\
32941.5	75.374298759016\\
32986.0000000004	75.3759917055972\\
33004.6999999996	75.3741879968974\\
33050.4000000005	75.3725359676858\\
33059.4000000004	75.3707708828022\\
33071.5000000001	75.3734321895524\\
33090.2999999998	75.3717090153035\\
33094.6000000003	75.3694700359876\\
33108.2	75.3717345801506\\
33155.6000000002	75.3701475161134\\
33168.4999999999	75.3718878698528\\
33176.5999999996	75.370063930409\\
33200.9999999995	75.3682859905244\\
33226.4999999997	75.3700348515948\\
33234.1000000002	75.3678439679607\\
33248.2999999995	75.3660326511952\\
33259.4000000002	75.3643319953698\\
33285.5000000005	75.3662244333886\\
33321.0999999998	75.3685341464293\\
33359.2	75.3666893489971\\
33363.2000000006	75.3648470025447\\
33384.1999999996	75.3629700188412\\
33404.2	75.3612558862182\\
33407.5000000001	75.3630910331781\\
33411.1999999997	75.3613298494821\\
33423.1000000001	75.3596304363436\\
33433.9000000003	75.3577196865466\\
33440.1000000003	75.3595971315961\\
33447.2999999998	75.3613135849124\\
33452.7000000005	75.3596111536254\\
33466.6000000001	75.3578930698276\\
33486.0000000001	75.359626830237\\
33489.7000000006	75.3578701574808\\
33491.7000000005	75.3560573035788\\
33500.6999999997	75.3584989015188\\
33514.9999999997	75.3567794814875\\
33570.2	75.3585758840404\\
33602.3999999996	75.3568930883119\\
33620.5999999999	75.3550639933136\\
33642.4999999999	75.3568392454804\\
33658.1000000001	75.3596449007968\\
33691.1000000003	75.3579569739279\\
33695.2000000004	75.3562069487436\\
33704.2000000006	75.3544799921672\\
33740.3999999997	75.3527659637528\\
33750.1999999996	75.354885734349\\
33778.6999999998	75.3531801011285\\
33800.7000000004	75.3514709711013\\
33839.3000000004	75.3497402436213\\
33864.9000000004	75.3479994094198\\
33873.3999999996	75.34975718482\\
33898.5999999998	75.3518571508642\\
33931.5999999999	75.3498940518748\\
33963.9000000002	75.3515486986221\\
33974.4000000005	75.3497476048213\\
33981.5000000002	75.3516608988394\\
33994.5000000003	75.3496143505145\\
34023.1000000001	75.3515248418726\\
34043.0000000003	75.3494834489221\\
34053.3999999999	75.3514323050494\\
34062.8000000004	75.3497206638307\\
34071.1999999995	75.3516889581554\\
34082.9000000006	75.349881172432\\
34124.7000000005	75.3522364966242\\
34128.1000000001	75.3497107963502\\
34161.9	75.3477548152919\\
34182.5000000002	75.349446794568\\
34199.1999999996	75.3477410356352\\
34216.7999999997	75.3501924487607\\
34231.5000000003	75.3482162680097\\
34240.0000000002	75.3464505068618\\
34267.5000000001	75.3484340893439\\
34275.9000000001	75.3465982028241\\
34286.9000000001	75.3489660804387\\
34315.4	75.3472920400402\\
34321.2000000004	75.3491272183746\\
34332.6000000001	75.3474093211428\\
34356.0999999999	75.3491364004168\\
34369.4000000004	75.3514016788141\\
34394.0999999995	75.353693355275\\
34402.9000000005	75.3518588495189\\
34415.0000000004	75.3500643612833\\
34434.6	75.3524787496334\\
34473.7999999995	75.3541084704661\\
34478.2999999997	75.3523945426702\\
34533.6999999999	75.3508157225675\\
34549.8	75.3486985490551\\
34569.3000000005	75.3469831700868\\
34591.9000000003	75.3445883441258\\
34606.1999999996	75.3463964653835\\
34654.7000000006	75.3480037397417\\
34660.3999999996	75.346287560768\\
34674.2	75.3483127272937\\
34695.5999999998	75.3465126802457\\
34707.1000000005	75.3445970864835\\
34737.4000000002	75.3427851745232\\
34775.1	75.3444983781545\\
34787.2	75.3424381886493\\
34801.8000000003	75.3441622440154\\
34815.7999999997	75.3460344267992\\
34828.0000000004	75.3483537718107\\
34835.6999999995	75.3466261719266\\
34866.5000000003	75.3483276258654\\
34892.0999999998	75.3466390769284\\
34903.7	75.3493888917539\\
34916.8999999999	75.3512615631354\\
34950.1000000004	75.349497284708\\
34955.1000000001	75.3513068155811\\
34959.0000000006	75.3531984519052\\
34975.4	75.3550342382525\\
34986.9000000001	75.3531311630034\\
35001.3000000003	75.354985800568\\
35025.7000000003	75.3533109879003\\
35046.0999999997	75.3553880306567\\
35089.5999999995	75.3537362815869\\
35129.6000000004	75.3556107794829\\
35135.8000000003	75.3573979889515\\
35153.2000000001	75.3556565101996\\
35169.1000000002	75.3574150108618\\
35197.0000000003	75.3550718667163\\
35205.4000000002	75.3532828677337\\
35280.2999999997	75.3517533814809\\
35327.8000000005	75.3540402911014\\
35335.3999999995	75.3522661346238\\
35342.3000000006	75.3505704196659\\
35371.0999999995	75.3488713981997\\
35380.9000000002	75.3506119103474\\
35388	75.3524489870889\\
35395.2999999998	75.3544245862457\\
35403.7999999995	75.3527153788142\\
35411.5999999998	75.3508021360172\\
35433.9000000005	75.3524863125811\\
35449.5999999995	75.3543753543754\\
35459.3000000005	75.3566050186974\\
35481.4000000003	75.3584262221158\\
35499.2999999998	75.3601469320608\\
35550.3999999997	75.3620905472497\\
35570.7999999996	75.3644130454951\\
35610.9999999999	75.366107758536\\
35633.5999999998	75.3679803107732\\
35652.6999999995	75.3699569178297\\
35674.9000000002	75.3718290119131\\
35754.0000000005	75.3703771036049\\
35773.8	75.3683551415976\\
35785.9000000002	75.3666238193707\\
35798.3000000005	75.3645414320193\\
35850.4000000001	75.3665918188031\\
35855.1000000005	75.3648006425846\\
35875.7000000003	75.3666817185958\\
35929.0000000005	75.3684339435165\\
35939.5000000002	75.3667263965097\\
35943.6000000002	75.3687016083486\\
35952.2000000004	75.370977656506\\
35974.9000000002	75.3690062543433\\
35988.3999999995	75.3707434319296\\
35998.6999999996	75.3727902041179\\
36008	75.3705416281337\\
36018.9000000001	75.3724423221078\\
36021.2	75.3706834567326\\
36041.0999999997	75.3687446588904\\
36075.6	75.3670753443454\\
36118.5999999997	75.3653924421422\\
36126.1000000006	75.3635865383018\\
36165.0000000002	75.361882035443\\
36172.8000000005	75.3636009277664\\
36186.9000000003	75.3618702849089\\
36202.6000000001	75.3598488510525\\
36216.2000000006	75.3580570074801\\
36227.4999999997	75.3560821031479\\
36236.1999999998	75.3542718213504\\
36298.6000000005	75.352560835512\\
36327.2000000004	75.3543478320712\\
36340.0999999995	75.3526397763358\\
36364.8000000001	75.3548064204769\\
36377.3	75.3531038501927\\
36381.8000000002	75.3512048573604\\
36414.2000000003	75.3539131604892\\
36424.0999999999	75.3521010756585\\
36447.1000000001	75.3503698500845\\
36470.8000000005	75.3526784367811\\
36513.6000000003	75.3508956912063\\
36515.6999999995	75.349026996533\\
36521.9000000006	75.3510213022288\\
36538.1999999996	75.3486068043669\\
36551.3000000004	75.3506021657173\\
36581.4999999997	75.3526362980296\\
36592.2000000006	75.3508251736021\\
36598.4999999998	75.3526091161957\\
36600.7999999996	75.3508793499613\\
36645.2000000003	75.3490897877763\\
36658.4000000006	75.3473273592755\\
36674.4999999995	75.3456070413856\\
36686	75.3473386377947\\
36696.4	75.3496927499898\\
36728.1999999997	75.3514320020257\\
36736.5999999997	75.353529304483\\
36751.1	75.3518252465226\\
36764.6999999999	75.3541430933937\\
36783.4000000005	75.3560699770277\\
36801.2000000002	75.3541315116586\\
36807.8999999997	75.3520973701369\\
36827.8999999997	75.353806886065\\
36854.2000000003	75.3556572774412\\
36871.2000000003	75.3575273993594\\
36882.8999999998	75.3555838733292\\
36889.0999999998	75.3580993895232\\
36901.1000000002	75.3563569748409\\
36917.4000000002	75.3542357960317\\
36924.0999999994	75.3524788621013\\
36961.4999999997	75.35415133544\\
36969.8000000001	75.3523812615127\\
37020.3000000002	75.3546693174574\\
37061.1000000002	75.352929748632\\
37074.1999999997	75.3511192389337\\
37086.1000000006	75.3530962999714\\
37096.3000000004	75.3550209723855\\
37114.0000000001	75.3573439743925\\
37130.8999999995	75.3591338773532\\
37140.0999999997	75.3574294161044\\
37207.5999999998	75.3591326526498\\
37256.7000000004	75.357518627472\\
37279.7000000004	75.3593098675422\\
37289.3000000005	75.3576083283721\\
37323.7	75.3559391058788\\
37359.0000000003	75.3578110821728\\
37424.4000000004	75.3562505845101\\
37438.1000000004	75.3583238510398\\
37440.6000000003	75.3562299850163\\
37446.6999999999	75.3589091724794\\
37455.8000000002	75.3571533456678\\
37462.9000000003	75.3588874356031\\
37518.4000000004	75.357223769607\\
37559.5000000006	75.359162504393\\
37572.1000000003	75.3573120551897\\
37616.4000000003	75.3589515239323\\
37624.9999999996	75.3608628282716\\
37669.3	75.359044741886\\
37672.8000000003	75.3608031237308\\
37697.5999999995	75.3629531775148\\
37702.2999999999	75.3612502121881\\
37710.6999999999	75.3632911526671\\
37717.1	75.3613735908286\\
37722.0999999996	75.3596025682489\\
37737.1000000004	75.3614470601952\\
37746.2000000001	75.3597041299412\\
37757.7999999998	75.3614475381311\\
37766.6000000003	75.3595098327363\\
37817.2999999996	75.357904033593\\
37846.2000000005	75.356111429651\\
37854.4999999996	75.3578164872961\\
37881.1999999996	75.3596101506548\\
37887.6999999997	75.3577668801039\\
37939.1000000004	75.3595226045884\\
37945.7000000002	75.3577471024461\\
37961.2999999998	75.3560195356336\\
37974.8999999995	75.3582620144832\\
37987.2999999997	75.3565655980667\\
38037.1	75.3596479236116\\
38054.5000000002	75.361454331408\\
38080.1	75.3638373748037\\
38129.7000000002	75.3620527776175\\
38133.4000000005	75.3602475513656\\
38145.0999999996	75.3620376875728\\
38161.9999999998	75.3637771506285\\
38186.4999999994	75.3620379923848\\
38215.7000000001	75.3599296626003\\
38229.8000000002	75.3619549096388\\
38293.9000000002	75.3635034209015\\
38302.9000000005	75.3617210140198\\
38339.0000000001	75.3637930989512\\
38350.9000000003	75.3620505332325\\
38363.7000000005	75.3637543726117\\
38419.4000000001	75.3619906557868\\
38427.1000000005	75.3638048049298\\
38437.2999999998	75.362017201996\\
38449.8000000004	75.3601439795682\\
38457.5999999994	75.3620211297088\\
38476.7000000006	75.3638556220892\\
38488.9999999996	75.3660127152882\\
38505.7999999997	75.3642948223519\\
38517.2000000004	75.3661341786678\\
38525.5999999999	75.3678713170689\\
38541.7000000002	75.3695987213882\\
38545.2999999994	75.3677481619078\\
38597.4999999998	75.3658258544573\\
38640.3999999998	75.3641904219666\\
38662.0999999997	75.3624987714098\\
38665.2000000007	75.3642154593395\\
38680.1999999999	75.362393776677\\
38691.5999999996	75.3606070552599\\
38711.5000000001	75.3624236662912\\
38726.3000000006	75.3604776070071\\
38746.1999999996	75.3622926576215\\
38768.8999999998	75.3643374861358\\
38780.2000000005	75.3661008295449\\
38809.0000000006	75.3678905205223\\
38822.5999999998	75.3657009945212\\
38854.2000000006	75.3638593411797\\
38860.6000000006	75.3656007225809\\
38901.7000000004	75.3677207738459\\
38914.7999999996	75.3698454833496\\
38928.5000000004	75.3679813812981\\
38936.0999999997	75.369707367437\\
38944.1000000001	75.3678339778452\\
38980.9999999997	75.3696021918314\\
39016.9999999998	75.3718241488988\\
39038.2999999997	75.3698922086971\\
39066.6000000005	75.371608044703\\
39128.6999999996	75.373382265748\\
39176.6999999997	75.3716485266791\\
39204.8000000003	75.369915495257\\
39277.0000000003	75.3683444042457\\
39364.9	75.3669503365934\\
39383.0999999996	75.3686851246217\\
39423.2999999998	75.3669140662652\\
39444.8000000005	75.3686788406106\\
39454.0999999995	75.3668810925073\\
39469.3999999996	75.3685757356946\\
39485.2000000003	75.3703276915713\\
39512.5999999999	75.3722220956806\\
39573.6999999997	75.3738584619117\\
39607.2000000003	75.3719642591139\\
39645.7999999998	75.3737460872373\\
39686.8999999997	75.3755637866304\\
39707.1000000005	75.3775133980739\\
39765.3999999996	75.3758911619369\\
39801.6000000002	75.3739161895095\\
39821.7	75.3757991853708\\
39832.2000000007	75.3740055181348\\
39876.6000000005	75.3723352233259\\
39882.2999999999	75.3705895332277\\
39922.4999999999	75.3723454885203\\
39942.0000000007	75.3741040155625\\
39968.4999999999	75.3724173476179\\
40016.6	75.3705328025549\\
40033.9999999999	75.3687481422088\\
40051.8999999997	75.3670228702687\\
40075.2999999994	75.3649370935786\\
40104.5000000002	75.3666661679707\\
40115.3999999995	75.3646346175419\\
40180.6000000005	75.3630474332204\\
40198.1000000007	75.3613345871208\\
40217.2000000006	75.3596089245176\\
40223.7999999999	75.3614144824346\\
40247.3000000002	75.3596505612785\\
40265.2	75.3614154122781\\
40293.6000000001	75.3631460997625\\
40321.8999999996	75.3613412033133\\
40352.0999999998	75.3631772245379\\
40359.6000000003	75.3650299680126\\
40414.9999999999	75.3634161489146\\
40432	75.3616557141479\\
40452.7000000001	75.3599256417356\\
40470.9999999997	75.3582185806662\\
40523.2000000004	75.3603482440966\\
40534.5000000004	75.3622830865483\\
40581.9000000003	75.3642008772362\\
40587.8000000002	75.3623616890748\\
40606.5000000006	75.364595903129\\
40627.5000000007	75.3628075495476\\
40656.8000000001	75.3611318128042\\
40695.8999999995	75.3629349321801\\
40704.4000000003	75.3612008500289\\
40712.4	75.3630948725821\\
40731.3999999996	75.3655033573524\\
40744.7000000007	75.3674088472639\\
40750.4000000002	75.3657010343431\\
40762.8999999999	75.3639329784363\\
40767.4999999997	75.3657316103965\\
40834.0000000007	75.3639727580625\\
40865.4999999995	75.3658333659606\\
40893.4999999997	75.3641156562396\\
40897.9000000006	75.3623649078195\\
40922.1999999996	75.3640435654887\\
40989.6000000005	75.3623471262293\\
41001.7999999998	75.3606540184723\\
41031.7999999996	75.3623400329987\\
41038.6000000004	75.3605742871972\\
41057.0999999997	75.3624212074861\\
41083.3000000004	75.3642590437987\\
41166.9999999996	75.3657653806073\\
41196.6000000005	75.3640461493275\\
41224.0999999996	75.3659258396767\\
41231.0000000007	75.3639849531058\\
41294.1999999996	75.3656073598535\\
41363.0999999995	75.3672346433545\\
41412.6000000006	75.3689085715252\\
41454.5000000002	75.3706464421319\\
41470.0999999996	75.368819055611\\
41488.8999999994	75.370821181518\\
41505.7	75.3689845756497\\
41565.8999999997	75.370735697445\\
41601.5	75.368495442483\\
41681.0999999995	75.3701908774219\\
41705.4000000005	75.3682368032993\\
41715.7999999999	75.3700627338737\\
41726.9999999998	75.367806533404\\
41760.9999999996	75.3660703381855\\
41808.4999999999	75.3677473055783\\
41815.4	75.3658332436537\\
41874.7999999996	75.3675829673623\\
41901.4999999998	75.3658571510396\\
41920.4000000001	75.3676602139764\\
41930.2999999994	75.3658443515922\\
41939.3999999995	75.3640362903706\\
41955.6999999997	75.3666954270923\\
41986.6000000007	75.3683904665049\\
41996.8999999999	75.3665738028907\\
42015.6999999995	75.3683138247992\\
42034.7000000006	75.3706452748675\\
42060.4999999996	75.3688725315378\\
42076.3000000005	75.3707541519712\\
42098.4999999996	75.3687771089775\\
42117.1000000002	75.3670709353898\\
42123.2999999995	75.368797390524\\
42170.0000000004	75.3704639068914\\
42208.9999999997	75.3723723083411\\
42243.8999999998	75.3707035318625\\
42264.4000000004	75.3689266405612\\
42283.7	75.37070934968\\
42334.4000000003	75.3723322585598\\
42364.4	75.3706523150279\\
42380.4000000006	75.3724000424724\\
42399.6	75.3705804522202\\
42411.4000000001	75.3729530905533\\
42460.9000000004	75.3745790254587\\
42470.9000000003	75.3763744672836\\
42492.8999999996	75.3745322759043\\
42512.8999999999	75.3762378566556\\
42583.9000000005	75.3745538230321\\
42607.7000000002	75.3763395434639\\
42625.6000000006	75.3782342577365\\
42642.5	75.3765014328395\\
42678.6999999999	75.3746590813237\\
42707.4999999996	75.3765137820903\\
42741.7000000007	75.3784351618322\\
42794.2000000006	75.3766272611072\\
42802.5000000004	75.3783648656857\\
42850.3999999995	75.3767167244256\\
42868.1999999995	75.379009664484\\
42879.6999999995	75.3772172444834\\
42906.1999999997	75.3789070602685\\
42918.4000000006	75.3807798501812\\
42926.5000000006	75.3826298845005\\
42944.8999999998	75.3843287926418\\
42972.4000000006	75.3826284251556\\
43017.2000000001	75.384322121565\\
43041.4000000003	75.3870102110753\\
43080.5999999994	75.3889792877089\\
43098.2000000007	75.3906766624206\\
43125.2000000005	75.3889248306678\\
43151.3999999995	75.390658493911\\
43198.4000000002	75.3931270761716\\
43226.6999999996	75.3914238389147\\
43259.2999999998	75.3930937553457\\
43285.2000000002	75.3953420676304\\
43302.0999999995	75.3933980259663\\
43316.1000000002	75.3916548543039\\
43342.8000000003	75.3899715985779\\
43354.5000000002	75.392230582222\\
43385.1000000001	75.393913131667\\
43404.1999999996	75.3920694493403\\
43465.3000000001	75.3937614746442\\
43479.5000000005	75.3919079292358\\
43507.3999999995	75.3902200769982\\
43519.0000000001	75.388507574835\\
43627.1000000004	75.387143800198\\
43639.1000000006	75.3888705567472\\
43682.3999999996	75.3869398500544\\
43693.0999999998	75.3851857955014\\
43740.4000000007	75.3868840090991\\
43760.3000000007	75.3850513249422\\
43777.0999999998	75.3833045512276\\
43790.2999999998	75.3852442544485\\
43797.4000000007	75.3835264569896\\
43822.6999999994	75.3856440026653\\
43840.6000000003	75.3874824078995\\
43877.8999999999	75.3892611331419\\
43964.4000000007	75.3908266897156\\
43972.8000000003	75.3887053162289\\
43988.3999999995	75.3906134557896\\
44031.8999999995	75.3924418604651\\
44072.7999999999	75.3905461179092\\
44109.6000000004	75.3888600466564\\
44121.1999999994	75.390570994055\\
44155.8000000006	75.3924164154734\\
44191.9000000006	75.3905684286749\\
44213.4000000006	75.3923575378561\\
44263.9	75.3939996385324\\
44302.9	75.3957971243482\\
44332.1000000007	75.3977921240092\\
44374.6000000002	75.3961153540193\\
44384.6000000007	75.3978285310028\\
44403.3000000007	75.3996315597454\\
44423.5999999998	75.4014186121372\\
44449.7000000008	75.399664340447\\
44488.3000000007	75.3978565199018\\
44512.2999999997	75.3996639138757\\
44543.7000000005	75.3979229432603\\
44567.8999999997	75.3996140728774\\
44605.1000000004	75.4012985033135\\
44612	75.3994992389957\\
44633.8000000006	75.4012084984731\\
44655.6000000006	75.4033639602559\\
44680.6999999997	75.4015147445883\\
44688.3000000001	75.4032366341153\\
44780.2000000007	75.4016833294998\\
44802.1999999998	75.3999236646333\\
44820.4000000002	75.4016577235863\\
44833.4000000003	75.3998684019762\\
44849.7999999997	75.4017288778793\\
44854.8000000002	75.4000120388185\\
44937.4000000002	75.4016133518776\\
44952.9000000001	75.3998620781705\\
44975.1000000001	75.4015546345542\\
44982.6000000007	75.4034328753054\\
44998.1000000002	75.401682733976\\
45011.0999999994	75.403455140054\\
45044.0999999999	75.4017165362022\\
45071.1999999997	75.3998664338504\\
45143.0999999995	75.3982881142675\\
45174.6999999997	75.4002231332513\\
45188.6000000004	75.3984956416095\\
45238.8000000005	75.3968376773087\\
45255.2000000004	75.3951470877444\\
45328.8000000001	75.3969321999872\\
45363.2000000001	75.3986151801126\\
45382.1	75.3967414536977\\
45499.6000000006	75.3954421677506\\
45521.5999999997	75.3937133279293\\
45539.7000000004	75.3918550366932\\
45571.5999999999	75.3937202254908\\
45581.4000000006	75.3955003674736\\
45632.5999999998	75.3972480260866\\
45640.7000000002	75.3954794832694\\
45651.7000000006	75.393741320167\\
45654.7999999994	75.3919075499015\\
45747.8999999998	75.3934598233803\\
45763.7000000006	75.3916851310424\\
45830.0999999996	75.3935178113995\\
45920.6	75.3951921464612\\
45945.9000000003	75.3935054194054\\
45951.5000000006	75.3917165017105\\
45994.3000000007	75.3900474840415\\
46010.1000000002	75.3917609573529\\
46028.7999999997	75.3900267006163\\
46039.5999999996	75.3881975773083\\
46081.3000000002	75.3898536068783\\
46107.4000000008	75.3881689529903\\
46121.1999999994	75.3862098423071\\
46193.3999999995	75.3846320369749\\
46214.6000000005	75.3829409257228\\
46232.1999999996	75.3847418363352\\
46292.0000000003	75.3865130335414\\
46301.3000000002	75.384761583883\\
46321.1999999994	75.3830311325459\\
46328.9000000002	75.3847482138617\\
46338.0999999996	75.38294538847\\
46358.9	75.3847149420824\\
46398	75.3830437022206\\
46441.1999999994	75.3813954389735\\
46450.4999999999	75.3796506396042\\
46490.4000000008	75.3779804476183\\
46496.7000000008	75.3798110837735\\
46558.6000000001	75.3781785144345\\
46581.1000000004	75.3801963023709\\
46618.4000000006	75.3784441798857\\
46723.4000000002	75.3798409793787\\
46729.8999999997	75.378129681147\\
46839.8000000008	75.3767621194751\\
46861.1999999996	75.3747762012578\\
46878.4999999993	75.3768243932199\\
46891.3999999997	75.3750679760724\\
46980.8999999993	75.3768544730849\\
47053.2000000002	75.3785600584869\\
47057.4999999993	75.3767722960797\\
47071.0000000002	75.378523127779\\
47102.3000000002	75.3802354020177\\
47111.2000000002	75.3783062662249\\
47132.2999999993	75.3802055486247\\
47221.1000000003	75.3820317992766\\
47250.3000000004	75.380102602306\\
47289.1999999997	75.3783625471301\\
47328.2	75.3802693103281\\
47354.4999999993	75.3785271124664\\
47373.0000000004	75.380542966367\\
47467.0000000005	75.3789466809643\\
47503.5	75.3772345674854\\
47518.5999999995	75.3791665176025\\
47563.8999999998	75.381170633252\\
47588.9999999999	75.3794461336735\\
47618.0999999999	75.3812617864598\\
47633.0999999995	75.3795671926303\\
47687	75.3778275466531\\
47766.2000000001	75.3761543180024\\
47782.3000000005	75.3744056388963\\
47831.1999999997	75.3763748842285\\
47865.3999999997	75.3780906916255\\
47922.4999999995	75.3763360084803\\
47974.9000000007	75.3780093798854\\
48037.4999999997	75.3763718420571\\
48043.8999999996	75.3744484222796\\
48126.3999999997	75.3728195484816\\
48183.3	75.3711859271035\\
48199.2999999994	75.3729299534849\\
48233.9	75.3748393249575\\
48263.2000000007	75.3765283351946\\
48293.2000000003	75.3748449577892\\
48354.3999999995	75.3729228923885\\
48451.3999999993	75.371453928155\\
48549.2000000008	75.3699847371641\\
48560.3999999995	75.3682519743413\\
48588.4000000005	75.3703036726798\\
48607.7999999998	75.3686129209449\\
48628.0000000006	75.3704134029501\\
48664.0000000001	75.3682899714574\\
48695.6999999996	75.3664587089646\\
48730.0000000005	75.364712980273\\
48742.4000000003	75.3629789198338\\
48786.2999999996	75.3646508043225\\
48831.1999999994	75.3664145742587\\
48885.4999999995	75.3647290817746\\
48968.9999999994	75.3663024233649\\
48977.3999999996	75.3641978459497\\
48993.2000000006	75.3664276543936\\
49014.5999999994	75.3647375175203\\
49039.7000000006	75.3667429312518\\
49054.3999999994	75.3686206158456\\
49092.8000000008	75.370369238729\\
49097.4000000007	75.3686032893732\\
49132.4999999997	75.3668643629688\\
49145.2999999997	75.3686001131337\\
49167.3999999997	75.3668581888442\\
49184.6	75.3689663655568\\
49198.6000000007	75.3672353131282\\
49236.4999999997	75.3689328670136\\
49261.9000000003	75.3672201697048\\
49355.5000000006	75.3687524819879\\
49380.6999999996	75.3707918867252\\
49397.6	75.3690961320061\\
49415.6000000005	75.367342767582\\
49452.4000000007	75.365653910318\\
49542.4999999995	75.3672596916593\\
49607.6000000008	75.3655178530752\\
49613.8999999999	75.3672350546217\\
49640.7000000008	75.3690512642826\\
49670.9000000007	75.3707394656842\\
49716.5000000005	75.3724108245536\\
49730.1000000002	75.3707002988124\\
49757.9999999997	75.3688344209285\\
49768.1999999995	75.3706676740013\\
49784.1000000003	75.3686912715279\\
49803.8000000003	75.3704027194657\\
49843.9999999997	75.3685992925943\\
49886.0999999994	75.3669351443886\\
49914.1999999995	75.3651759115123\\
49939.2000000006	75.366895411029\\
49949.6000000001	75.3686208325576\\
49978.6	75.370307751102\\
50026.1999999994	75.3719543520108\\
50097.2000000004	75.3741219586684\\
50109.1000000008	75.372386707431\\
50116.4999999995	75.3706356775998\\
50189.5999999998	75.3688505808961\\
50256.8999999999	75.3672125276081\\
50286.6000000005	75.3690339592775\\
50373.2000000004	75.3706824845702\\
50405.5999999997	75.3724281182485\\
50447.4000000008	75.3706328361168\\
50496.0999999999	75.3724042601225\\
50507.1999999994	75.3706889895124\\
50516.6000000002	75.3725005790164\\
50527.7000000002	75.3707859831618\\
50613.6000000004	75.3689218531741\\
50631.4999999994	75.3707171015729\\
50655.6000000001	75.3690107924676\\
50675.8999999997	75.367037650959\\
50711.3999999997	75.3651538605642\\
50721.2999999999	75.3632589005824\\
50737.2000000004	75.3650667260575\\
50828.7999999996	75.3634251380612\\
50908.8	75.3618718927339\\
50945.3000000006	75.3636245863218\\
50988.0000000003	75.3618981683962\\
51031.1999999993	75.360220100213\\
51116.1	75.3618226706993\\
51130.3999999998	75.360107959046\\
51145.4999999997	75.3621034849528\\
51214.5999999993	75.3639091901349\\
51235.7000000008	75.3619539462641\\
51257.5	75.3642386689974\\
51273.4999999998	75.3660753292143\\
51323.1000000009	75.3678648252642\\
51371.5000000007	75.366155619058\\
51456.1000000001	75.367982866982\\
51491.5000000001	75.3697690497091\\
51598.1999999997	75.3716692216604\\
51652.2000000005	75.3699254437847\\
51727.1999999999	75.3683258163484\\
51753.2999999994	75.3666425780722\\
51814.9000000001	75.3683296342758\\
51856.2999999993	75.3665892734551\\
51882.6000000003	75.3688609112479\\
51907.1999999992	75.3670485654234\\
51931.0999999993	75.3687571248113\\
51959.5999999993	75.367063320227\\
51999.8	75.3653372410332\\
52054.2999999994	75.3636580193029\\
52107.9999999998	75.3619878675292\\
52127.8000000006	75.3602197671496\\
52141.5000000002	75.3584469981742\\
52176.6000000005	75.3566247002973\\
52247.9999999997	75.3583383893386\\
52319.8999999993	75.3566513761468\\
52349.3999999996	75.3548744496127\\
52370.7000000008	75.3568782604047\\
52407.1000000007	75.3587293349005\\
52487.7000000006	75.3571306093988\\
52502.6999999995	75.355409616249\\
52644.1000000004	75.3541700700172\\
52675.7000000002	75.3524388808523\\
52692.9999999994	75.3544581738406\\
52728.8999999993	75.3562555709382\\
52744.5000000003	75.3544438672395\\
52776.3999999995	75.3562665201368\\
52857.3999999995	75.3546800359457\\
52900.7000000007	75.356705380637\\
52925.8000000002	75.3549774307097\\
52946.0000000003	75.3532365934413\\
52962.2000000006	75.3558663426626\\
53003.3999999998	75.3576650598545\\
53037.5999999999	75.3594141525745\\
53061.7	75.3611826210193\\
53103.1000000007	75.3594886937134\\
53111.9	75.3577345985841\\
53127.5999999995	75.355981907743\\
53171.5999999997	75.3577560995793\\
53250.2000000004	75.3603266084886\\
53292.1000000006	75.3620604891523\\
53314.0999999992	75.3637867584996\\
53341.2999999992	75.3656634434042\\
53396.9999999998	75.364017896103\\
53485.5000000001	75.365705909628\\
53524.2999999996	75.3639461628715\\
53591.5000000005	75.3659155539301\\
53603.0999999993	75.3677019282431\\
53626.2000000007	75.3660051131628\\
53685.9999999998	75.3677395079918\\
53700.1000000007	75.369737915315\\
53749.2000000003	75.3678652559196\\
53778.9000000003	75.3695680470072\\
53862.7000000001	75.3679719583832\\
53950.9000000002	75.3696873088543\\
54014.5000000001	75.3679560711363\\
54049.5000000005	75.3698454752672\\
54093.6999999999	75.3681567943091\\
54108.6000000001	75.3664382992014\\
54125.0999999995	75.3647099687391\\
54137.6000000003	75.3630095109323\\
54159.7999999997	75.3612912874654\\
54265.6000000003	75.3597576369604\\
54346.7000000006	75.3613828229077\\
54454.7000000003	75.3630533947421\\
54524.3000000009	75.3613061308332\\
54652.0999999998	75.3627484346467\\
54713.6000000003	75.3644882360359\\
54745.7999999995	75.3661918061444\\
54761.0999999998	75.3644916473708\\
54836.5000000003	75.3628051337975\\
54861.3999999997	75.3610455419556\\
54874.5999999993	75.3593185930857\\
54890.3999999996	75.3574844463067\\
54963.4000000006	75.3591019494756\\
55022.5000000005	75.3615786967537\\
55083.2000000008	75.3633133817328\\
55092.2999999996	75.3650231247867\\
55152.7000000006	75.363172857951\\
55196.0000000004	75.3614838729548\\
55221.8999999995	75.363261019159\\
55266.4000000002	75.3650041164177\\
55281.6999999999	75.3669381243013\\
55298.5999999993	75.3652436675727\\
55327.0999999999	75.3670888821411\\
55360.0999999996	75.3687667313341\\
55430.8999999993	75.3708213815374\\
55523.1999999994	75.3725012742398\\
55565.5999999994	75.3743766388258\\
55598.4999999999	75.3761785368696\\
55710.5000000002	75.3747042753085\\
55719.3000000007	75.3764398037308\\
55750.5000000007	75.3783815779561\\
55880.4000000003	75.3767414393214\\
55894.5999999999	75.3785242607975\\
55911.6000000001	75.3804659847581\\
55943.4999999993	75.3821706146905\\
55976.1000000007	75.3804295396972\\
56062.8000000006	75.3821154453302\\
56070.6000000004	75.3803679996861\\
56133.5000000002	75.3787392934\\
56152.3000000007	75.3807495316318\\
56191.3999999997	75.3790163992775\\
56222.2000000007	75.3807652835263\\
56263.5000000005	75.3824853013316\\
56284.2999999998	75.3805672619767\\
56335.3999999994	75.3823077810617\\
56353.9999999992	75.3840448166149\\
56394.0000000002	75.3857229745665\\
56423.5999999994	75.3839255490158\\
56511.7000000008	75.3819556269664\\
56533.4000000006	75.3802612610222\\
56596.3000000003	75.3784693019344\\
56639.0000000007	75.3765508279616\\
56700.2	75.3782607852163\\
56717.5999999993	75.3764697792753\\
56767.600000001	75.3784282259102\\
56781.7999999999	75.3801827695093\\
56800.7999999992	75.3822562670662\\
56833.8000000006	75.3803627764415\\
56894.9999999994	75.378723299546\\
56910.4000000003	75.3804658191371\\
56967.9999999992	75.3821875751517\\
56990.6	75.3838784222689\\
57031.7999999997	75.3821983837116\\
57054.4000000008	75.3840626068058\\
57062.000000001	75.3820837298312\\
57092.8	75.3803012283489\\
57135.5000000002	75.3820735233375\\
57179.9999999993	75.3837436450793\\
57264.2000000001	75.3820443103295\\
57416.3000000005	75.3833399516514\\
57444.6000000002	75.3813667753509\\
57562.4000000004	75.383105320304\\
57624.6999999991	75.3849037220086\\
57647.7999999999	75.3867877234036\\
57664.8000000008	75.388494560816\\
57715.9999999992	75.386763831929\\
57868.3000000002	75.3881220147784\\
57921.7000000009	75.3864693431489\\
57999.5999999991	75.3848726803759\\
58024.4000000002	75.3866039345449\\
58100.5000000003	75.3849357837957\\
58252.9999999992	75.386202622693\\
58268.9000000004	75.3845097736361\\
58280.9999999995	75.3827570172835\\
58327.4	75.3846813252754\\
58357.0000000006	75.3863711527817\\
58375.2000000009	75.3846232910152\\
58383.2000000008	75.3828577692594\\
58502.5999999992	75.3845549008849\\
58527.8000000009	75.3828515972724\\
58541.1999999991	75.3809703576791\\
58613.0999999994	75.3826783045457\\
58640.8000000006	75.3844159963438\\
58710.9999999999	75.3826789142087\\
58765.5000000002	75.3844085655554\\
58796.6999999999	75.3825038097311\\
58975.9999999991	75.3813833061189\\
58985.2000000001	75.379628483707\\
59021.8999999995	75.3813832130392\\
59098.1999999991	75.3794948416452\\
59123.9000000005	75.3815709356606\\
59168.1999999991	75.383271109699\\
59240.5999999995	75.3816210814524\\
59265.7000000002	75.3795612309291\\
59293.4999999995	75.3777810758665\\
59354.0999999999	75.3795013663734\\
59383.4999999997	75.3814184387609\\
59423.1999999999	75.3796911312566\\
59451.3000000008	75.3815721749193\\
59498.9000000002	75.3797542815846\\
59529.5999999995	75.3780045926655\\
59582.1	75.3762029599444\\
59609.8000000005	75.3782509281176\\
59622.3000000006	75.3765363353371\\
59686.0000000006	75.3781868810326\\
59780.7999999999	75.3764162132052\\
59788.2000000009	75.3781258206037\\
59873.3000000003	75.3763774898369\\
59913.000000001	75.3746676436372\\
59939.1999999992	75.3727521008754\\
59998.8000000003	75.3745485333898\\
60013.4000000001	75.37270780741\\
60050.9000000003	75.3745982581472\\
60096.1	75.3769789104802\\
60149.3000000002	75.3753154644934\\
60164.6000000001	75.3735994694563\\
60207.0000000006	75.3753294877182\\
60233.3000000005	75.377116350729\\
60295.900000001	75.3754809605944\\
60313.4999999995	75.3737133913413\\
60321.0000000007	75.3754490551399\\
60334.9999999996	75.373704526884\\
60351.4000000002	75.3719460162548\\
60411.7999999993	75.3738915677209\\
60428.5	75.3720920226515\\
60434.7999999992	75.3703571942702\\
60485.6999999999	75.3720708000886\\
60527.5999999999	75.3737544958754\\
60604.8999999993	75.3716690042076\\
60624.9999999995	75.3697725859421\\
60766.900000001	75.3683742820939\\
60864.1000000002	75.3668001879594\\
60953.5999999994	75.3690095925268\\
61151.8000000008	75.37002120948\\
61158.3999999996	75.3682644276756\\
61253.1000000008	75.3701031129802\\
61280.3999999996	75.3680208222844\\
61378.3000000004	75.3696740221316\\
61415.1000000009	75.3678242519767\\
61461.5000000004	75.3694990042563\\
61509.1999999999	75.3677899114443\\
61531.9000000002	75.365988428785\\
61588.5000000003	75.3676816813501\\
61621.2999999999	75.3694333462076\\
61791.9999999993	75.3707998271624\\
61831.4000000002	75.3725851709889\\
61897.8999999998	75.3709328249701\\
61935.5000000002	75.3690930579505\\
61993.8999999999	75.3708423395812\\
62125.1999999996	75.3723523266688\\
62205.0999999997	75.370708558127\\
62385.9000000005	75.3694739204309\\
62413.2000000002	75.3677501429984\\
62499.5000000006	75.365762340879\\
62586.0000000001	75.3640185280757\\
62633.4999999999	75.3622656210085\\
62722.2999999998	75.36063671033\\
62893.8000000011	75.361998540399\\
62946.3999999991	75.3636818568149\\
62998.0999999995	75.3619944696833\\
63137.7000000006	75.3634114587458\\
63187.6000000006	75.365142266612\\
63287.4000000003	75.3635394035157\\
63321.3000000006	75.3617260515402\\
63381.6000000008	75.3633935347269\\
63421.2999999995	75.3616287246891\\
63435.8999999996	75.363358345419\\
63543.8000000008	75.3649052072661\\
63610.7999999996	75.3669575497281\\
63792.9999999992	75.3655175873253\\
63833.0000000007	75.3637219561638\\
63874.2999999996	75.3619603471813\\
63894.8999999999	75.3602003286642\\
63906.8000000006	75.3619718684524\\
63980.9000000001	75.3603413513387\\
64005.1999999994	75.3623528051583\\
64036.6000000004	75.364127133347\\
64075.5999999998	75.3658563230679\\
64136.1000000004	75.363991006639\\
64173.3999999993	75.3622601229479\\
64268.1999999991	75.3606365813317\\
64323.9	75.3623841800883\\
64351.1000000009	75.3640957744378\\
64422.9999999995	75.3658237495557\\
64459.4999999999	75.3675170184115\\
64531.0999999994	75.3658695328771\\
64618.599999999	75.3642521437293\\
64774.5000000008	75.3658069675459\\
64873.5999999996	75.367521815466\\
64889.500000001	75.3695507446494\\
64927.6999999994	75.3678701573132\\
64935.7999999993	75.3658607950302\\
64961.4999999996	75.3676017832073\\
65070.9	75.3692120914078\\
65083.4000000003	75.3674894558529\\
65129.3000000009	75.3656566773224\\
65231.0999999998	75.3673088951299\\
65296.8000000007	75.3691216581492\\
65310.8000000011	75.3673582816957\\
65360.1000000002	75.3692614159687\\
65398.9999999998	75.3675509295999\\
65476.9	75.3695190677643\\
65526.5000000002	75.3712232894733\\
65606.8000000005	75.3728647444095\\
65620.4999999993	75.3711486941601\\
65664.2999999997	75.3694543771054\\
65718.9000000002	75.3715059571813\\
65755.2999999992	75.369779516207\\
65868.4000000004	75.3681957232972\\
65915.6000000004	75.3700559957643\\
65939.4000000008	75.3683300601309\\
65959.0000000008	75.3700399186769\\
66137.0999999998	75.3686276407226\\
66172.7999999998	75.3705822171916\\
66336.0000000004	75.3692182687858\\
66384.0000000001	75.370909600341\\
66431.1999999997	75.3691407514229\\
66561.3999999995	75.3707473539509\\
66619.9000000011	75.3724106874812\\
66637.4000000009	75.3706246482836\\
66728.6999999996	75.3689561328842\\
66884.4999999993	75.3675734025471\\
67000.8000000002	75.365853294508\\
67196.6999999997	75.3669817610362\\
67384.3999999993	75.365551425031\\
67396.5999999993	75.3673399439438\\
67430.3000000003	75.3694179479879\\
67459.5999999995	75.3712216330639\\
67505.2000000007	75.3694895067498\\
67539.5999999996	75.3712261084962\\
67620.0000000001	75.3728551126071\\
67652.7000000005	75.3711598041766\\
67721.4000000004	75.3694173932946\\
67803.7000000003	75.3677522498739\\
67847.6999999993	75.3694298120204\\
67916.7999999992	75.3676919882975\\
67954.5999999991	75.3659423115693\\
68019.900000001	75.367832990297\\
68131.8999999994	75.3694299301356\\
68234.1000000001	75.3711481925486\\
68325.8000000008	75.3695158058657\\
68360.8999999999	75.3678266848057\\
68419.0999999998	75.3696330854497\\
68463.8999999994	75.3715821453611\\
68534.3999999997	75.3732791513763\\
68561.3000000006	75.3750652699624\\
68582.6999999992	75.376770852167\\
68618.5000000002	75.3785416782039\\
68666.9999999997	75.3803495414835\\
68693.6999999992	75.3785640043206\\
68790.8999999994	75.3802096204445\\
68864.9999999994	75.3785299084732\\
68876.5999999996	75.3803535883688\\
68973.7000000002	75.3787670100821\\
69018.5999999992	75.3769630549402\\
69054.8000000006	75.3752449138294\\
69111.7000000001	75.3735252156651\\
69167.9000000006	75.3753180661578\\
69241.8999999998	75.3736171687704\\
69283.2000000011	75.3753068921371\\
69335.6000000002	75.3735809979563\\
69381.0999999994	75.3754619406986\\
69494.3000000003	75.3770087949533\\
69557.7999999997	75.3749034976617\\
69576.4000000003	75.3731504171667\\
69630.0000000004	75.3714270121686\\
69799.1999999995	75.3728189251182\\
69873.0000000008	75.3745003441954\\
69997.199999999	75.3729072407079\\
70055.8000000001	75.3710965100727\\
70095.1000000012	75.3693548203015\\
70243.2999999992	75.3707821660113\\
70344.299999999	75.3687571434258\\
70360.6999999996	75.3669656968084\\
70408.9999999994	75.3652581839563\\
70441.9000000004	75.3635615115982\\
70458.0999999997	75.36184574684\\
70594.7999999995	75.3602597354766\\
70646.4999999992	75.3584744347216\\
70708.9000000005	75.360279455232\\
70747.399999999	75.3585639068518\\
70797.9	75.360462159948\\
70898.3000000011	75.3623495029507\\
71015.299999999	75.3639351464612\\
71032.8000000007	75.365781208426\\
71049.3999999992	75.3640771574747\\
71133.5000000009	75.3657905687326\\
71219.7999999992	75.3640485313796\\
71231.1000000011	75.3623412212626\\
71325.4999999994	75.3639646914993\\
71379.4999999995	75.3656506901131\\
71455.7999999999	75.3639657467053\\
71602.0999999999	75.3624888620741\\
71628.9000000008	75.3606779377068\\
71759.2999999995	75.3622521927441\\
71782.2000000011	75.3641217960416\\
71908.4000000008	75.3624397672042\\
72008.599999999	75.3643656947008\\
72150.5999999993	75.3660047650265\\
72257.7999999993	75.3678144535061\\
72300.3999999991	75.3697415647195\\
72383.7000000005	75.368107228413\\
72540.8000000005	75.3696190700694\\
72622.4000000003	75.3712692347413\\
72767.2000000009	75.3697058981163\\
72795.9999999989	75.3714828129529\\
72835.7999999991	75.3734078936349\\
73018.9000000003	75.3747654720004\\
73051.2000000005	75.3764820064804\\
73064.4	75.3747716059099\\
73096.6000000005	75.3730332559473\\
73131.8999999997	75.3709730350599\\
73176.6999999993	75.3726590941391\\
73283.299999999	75.3709571335391\\
73338.6999999993	75.3692452017213\\
73480.9000000003	75.3709122085981\\
73549.8000000004	75.3691031530974\\
73567.7999999996	75.3673816977241\\
73650.8999999999	75.3691056469023\\
73732.5999999999	75.3709005637933\\
73789.9999999997	75.3725770801232\\
73843.2000000008	75.3743399875141\\
73944.8999999991	75.372641828386\\
74030.1000000005	75.3743742418634\\
74043.799999999	75.3760944520751\\
74215.7000000007	75.3746506808523\\
74326.1999999992	75.3762799977935\\
74441.8999999997	75.3746540931195\\
74489.1999999993	75.3764634652225\\
74533.5000000002	75.3747303229684\\
74591.2999999995	75.3730322798607\\
74663.3000000011	75.3749494397523\\
74737.099999999	75.3733080714825\\
74773.9999999998	75.3750295891224\\
74790.4000000003	75.3730754574445\\
74885.3000000012	75.371434218152\\
74954.7000000009	75.3732916370933\\
74977.1	75.3715796268732\\
75095.2999999999	75.3731919664853\\
75317.9999999998	75.3720553226914\\
75410.3999999997	75.3737211661506\\
75491.7000000012	75.3720271605658\\
75587.3000000005	75.3703659604643\\
75713.7000000009	75.3688231207521\\
75802.5000000011	75.3705018033682\\
75834	75.3726885398521\\
75917.4999999993	75.3710602021139\\
76033.8999999995	75.3694399873741\\
76101.2000000006	75.3711171819667\\
76131.5999999998	75.3728079105025\\
76237.4000000009	75.3744548286604\\
76398.4999999995	75.3729780388646\\
76459.0999999989	75.3713091426539\\
76505.3999999998	75.3731431073583\\
76550.2999999993	75.3749163949502\\
76569.9000000009	75.376648818075\\
76690.9999999997	75.3782381527974\\
76823.399999999	75.3766751059246\\
76837.9000000004	75.3749446888258\\
76880.3999999994	75.3767210150819\\
77020.1000000007	75.3751613213157\\
77125.3999999996	75.3735145963397\\
77335.7000000002	75.3750526922848\\
77375.7000000011	75.3769266359766\\
77451.6000000005	75.3752338554222\\
77503.9	75.3735291081751\\
77720.5000000004	75.3722951186687\\
77816.4999999992	75.3706792638075\\
77966.6999999988	75.3722097097739\\
78181.700000001	75.3709942723243\\
78208.4000000004	75.3693012907804\\
78232.5000000001	75.3673021221332\\
78259.4000000006	75.3655466748446\\
78323.8000000012	75.3672889118136\\
78362.2999999998	75.3690545465683\\
78433.3999999995	75.3709830620844\\
78573.599999999	75.3693665946748\\
78804.6000000011	75.3705045511245\\
78823.6999999996	75.3686069435881\\
78882.5000000012	75.3669377023577\\
78893.5999999997	75.3652066007805\\
78935.6000000002	75.3634920574594\\
78988.0999999994	75.3617628962301\\
79084.5999999989	75.3600886138533\\
79210.199999999	75.3616638240229\\
79224.4999999997	75.359926083565\\
79349.6999999998	75.3582239652778\\
79394.9000000013	75.3599093141886\\
79435.1999999995	75.3618353553143\\
79470.1000000013	75.3600720773321\\
79529.9999999989	75.3582605831\\
79553.2000000001	75.3600416324653\\
79629.7999999994	75.3583766901629\\
79681.2000000012	75.3564512627179\\
79731.7000000001	75.3583889991196\\
79909.5999999995	75.3598123882332\\
79950.6000000004	75.3615665653959\\
79970.6000000006	75.3633518276069\\
80027.2000000007	75.3650316829382\\
80134.3999999989	75.3632954595087\\
80163.6999999992	75.3615721809595\\
80249.7000000009	75.3599385917473\\
80401.7000000002	75.3614968819106\\
80584.7999999989	75.3628781570741\\
80617.7	75.3611485304734\\
80656.7000000005	75.3594241279099\\
80753.5999999994	75.3577854637001\\
80829.8999999997	75.3595199802054\\
80930.1	75.3579010060521\\
80974.6000000014	75.3595876242826\\
81103.7999999991	75.3580284055391\\
81130.699999999	75.3600358926573\\
81157.4999999993	75.3583151793547\\
81231.3000000014	75.3565739356948\\
81295.6000000004	75.3549080701685\\
81371.5999999996	75.3567886623974\\
81462.6999999995	75.358445818214\\
81504.3999999999	75.3565754038121\\
81713.8999999997	75.3552634799422\\
81797.8999999996	75.3535538766229\\
81886.7000000007	75.3518271565136\\
81959.3999999998	75.3501424484044\\
82032.8999999993	75.3519924908269\\
82066.5000000007	75.350264297534\\
82104.5000000005	75.3483970447454\\
82271.0000000013	75.3468715016573\\
82323.8	75.348714042945\\
82468.4999999991	75.3502545211147\\
82521.2999999995	75.3519693073554\\
82573.2000000004	75.3502645528276\\
82619.500000001	75.3485855656527\\
82693.4000000005	75.3465508171743\\
82830.3999999999	75.3483318342881\\
82897.7999999988	75.3466613750167\\
83045.9999999997	75.3451396272673\\
83182.0000000014	75.3434933717711\\
83243.2	75.3451629140123\\
83319.0000000007	75.3434686644479\\
83457.0999999995	75.3418518713784\\
83544.8000000011	75.3402062842855\\
83739.3000000011	75.3387294391887\\
83789.6000000002	75.3404058016677\\
83817.4999999992	75.3387116786928\\
83921.6000000013	75.3370105705676\\
83958.2000000007	75.3387098118947\\
84124.6000000006	75.3402983903657\\
84168.1999999987	75.3385775879993\\
84228.1000000013	75.3404441742789\\
84362.7	75.3387749102685\\
84442.6999999999	75.3404671564657\\
84588.3999999996	75.3420382203255\\
84638.1000000002	75.3403309616698\\
84660.0999999988	75.3385888528494\\
84770.2000000006	75.3403019689679\\
84915.7000000014	75.3419269441023\\
84938.2000000003	75.3402175461482\\
85013.2999999999	75.3384760520106\\
85111.2999999999	75.3404361812871\\
85222.3000000006	75.3420462225894\\
85279.199999999	75.3437235061732\\
85292.3000000007	75.3419999906205\\
85381.8	75.3437203903872\\
85504.4999999998	75.3453030597184\\
85578.8999999989	75.3473398847848\\
85792.3000000004	75.3487488402236\\
85847.1999999989	75.3470406174685\\
86005.6999999992	75.3454999546542\\
86179.0999999999	75.3470675058483\\
86244.3000000011	75.3487762683722\\
86367.9999999992	75.3471478474113\\
86409.7999999995	75.3454175968263\\
86469.1000000008	75.3471756417314\\
86602.4999999992	75.3487770575017\\
86745.1000000013	75.3503363874889\\
86816.8000000004	75.3521491783282\\
87023.4000000003	75.3505662263642\\
87073.5000000003	75.352575292626\\
87109.4999999989	75.3543811474281\\
87157.9	75.3562495697469\\
87183.1000000003	75.3579818130098\\
87272.9000000013	75.3596186678583\\
87515.2000000012	75.3583659085897\\
87572.6999999991	75.3603858732392\\
87735.8999999993	75.3619950761375\\
87818.0999999993	75.3602328446723\\
87884.2999999988	75.3620665328545\\
87936.5999999987	75.3603444295726\\
87955.7000000003	75.3586460472192\\
87977.4999999996	75.3604326555851\\
88136.4999999998	75.3619949033659\\
88366.0999999994	75.3607148434582\\
88388.7000000004	75.358982133483\\
88413.0000000002	75.360777984258\\
88500.0000000008	75.359010893773\\
88544.6000000002	75.3571924688886\\
88626.4999999988	75.3588651714045\\
88648.6000000015	75.3606087850132\\
88760.0000000014	75.3622404661554\\
88804.9999999998	75.3642527287284\\
88855.9	75.3659854146034\\
89117.6000000012	75.3648265159447\\
89259.1000000006	75.3664608242064\\
89311.299999999	75.3647350730142\\
89341.8000000011	75.3664294132988\\
89375.9999999994	75.3683591027131\\
89577.6000000003	75.3697627869436\\
89696.7000000011	75.3681290748388\\
89864.1999999989	75.3666361391565\\
89979.0999999995	75.3649732382595\\
90104.099999999	75.3633015997035\\
90177.999999999	75.3649722050032\\
90379.8000000003	75.3634381095797\\
90471.6000000006	75.3617982197748\\
90666.4999999999	75.3632539435691\\
90776.2	75.3650457222865\\
90846.2000000001	75.363333454417\\
90986.4999999997	75.3617565663515\\
91197.0000000011	75.3631420297356\\
91270.2000000002	75.3649325136436\\
91505.3000000009	75.3636397414797\\
91658.7000000016	75.3621038023627\\
91757.3	75.3604613905799\\
91845.4000000014	75.3622115400319\\
91870.4000000002	75.3640178294447\\
91936.4	75.3657143789463\\
92155.8000000005	75.3643554020958\\
92273.7000000014	75.3661386005562\\
92340.4000000016	75.3644392222264\\
92395.2000000011	75.3661712229951\\
92462.7000000011	75.3644709007298\\
92527.599999999	75.3627292151431\\
92752.000000001	75.3613125740549\\
92813.5999999991	75.3630121415265\\
92836.3000000014	75.3647276283871\\
92877.7000000002	75.3630038609872\\
92994.3000000014	75.3646456130692\\
93094.7999999999	75.3663197446906\\
93324.5000000014	75.3676951200434\\
93342.6999999992	75.3659628809078\\
93395.8000000002	75.3642290507399\\
93455.9999999988	75.3624429009984\\
93620.4000000015	75.3607382998382\\
93946.1000000016	75.359940050795\\
94107.5	75.3617136129282\\
94181.0999999986	75.3600506258149\\
94332.0999999991	75.3584672041996\\
94456.6999999997	75.3567768546044\\
94691.2999999986	75.3554177042477\\
94879.5999999997	75.357004712283\\
95024.1000000005	75.355435773203\\
95106.3999999989	75.3570996724724\\
95172.7000000007	75.3589260797203\\
95271.3999999995	75.3571634749112\\
95319.8	75.3554084718931\\
95414.9000000008	75.3536655662108\\
95652.7000000012	75.3523158757506\\
95754.8000000011	75.3539505550108\\
95840.7000000003	75.3522508159364\\
96000.5999999995	75.3538255450221\\
96027.9000000005	75.3520848085975\\
96202.199999999	75.3536038119671\\
96308.3000000007	75.3553168778632\\
96532.0999999998	75.3539233540725\\
96711.7000000003	75.3554374957347\\
96754.0999999991	75.3537314142435\\
96844.3999999994	75.3554409388246\\
96914.9000000003	75.3537636072847\\
97140.6000000001	75.3522467925391\\
97340.6999999986	75.3537057431211\\
97571.6999999993	75.3550718547777\\
97763.4999999989	75.3535058037961\\
97911.4000000012	75.3519249526358\\
97946.0000000009	75.350116033206\\
98116.0000000014	75.3485921270821\\
98139.2000000017	75.3503438479793\\
98387.5000000006	75.3489260841813\\
98439.6000000012	75.3472430330446\\
98493.9999999994	75.3455283108328\\
98683.6999999988	75.3470174435926\\
98778.7000000016	75.3453170113425\\
99071.9000000013	75.3440931847545\\
99367.1000000005	75.3452849632474\\
99681.5000000004	75.3440956003917\\
99808.1000000009	75.3458132698516\\
99827.2000000003	75.344019121022\\
99914.6000000008	75.3457699417603\\
99990.0000000015	75.3440590618471\\
100000	75.3439246560753\\
};
\addlegendentry{PSK 16}

\addplot [color=mycolor3, forget plot]
  table[row sep=crcr]{%
0.1	50\\
0.2	66.6666666666667\\
0.3	50\\
0.4	60\\
0.5	50\\
0.6	57.1428571428571\\
0.699999999999989	50\\
0.8	55.5555555555556\\
0.900000000000001	60\\
1	63.6363636363636\\
1.10000000000001	66.6666666666667\\
1.2	69.2307692307692\\
1.29999999999999	64.2857142857143\\
1.39999999999998	60\\
1.6	64.7058823529412\\
1.8	68.4210526315789\\
1.89999999999998	70\\
2	66.6666666666667\\
2.09999999999997	68.1818181818182\\
2.3	62.5\\
2.5	65.3846153846154\\
2.7	67.8571428571429\\
2.79999999999996	65.5172413793103\\
3	67.741935483871\\
3.10000000000004	65.625\\
3.39999999999996	68.5714285714286\\
3.50000000000006	66.6666666666667\\
3.79999999999997	69.2307692307692\\
4	65.8536585365854\\
4.30000000000002	68.1818181818182\\
4.40000000000004	66.6666666666667\\
4.5	67.3913043478261\\
4.8	63.265306122449\\
5.29999999999999	66.6666666666667\\
5.50000000000005	64.2857142857143\\
5.69999999999995	65.5172413793103\\
5.80000000000007	64.4067796610169\\
5.90000000000006	65\\
6.10000000000005	62.9032258064516\\
6.2999999999999	64.0625\\
6.40000000000001	63.0769230769231\\
6.99999999999989	66.1971830985916\\
7.20000000000001	64.3835616438356\\
7.3000000000001	64.8648648648649\\
7.5	63.1578947368421\\
7.69999999999994	64.1025641025641\\
8.1999999999999	60.2409638554217\\
8.39999999999987	61.1764705882353\\
8.4999999999999	60.4651162790698\\
8.80000000000008	61.7977528089888\\
8.90000000000012	61.1111111111111\\
9.39999999999986	63.1578947368421\\
9.80000000000001	60.6060606060606\\
10.7	63.8888888888889\\
10.8	64.2201834862385\\
11.0000000000001	63.0630630630631\\
11.5	64.6551724137931\\
11.8000000000001	63.0252100840336\\
11.9000000000001	63.3333333333333\\
12.0999999999998	62.2950819672131\\
12.2000000000001	62.6016260162602\\
12.5	61.1111111111111\\
12.5999999999998	61.4173228346457\\
12.6999999999999	60.9375\\
12.9000000000001	61.5384615384615\\
13.1000000000001	60.6060606060606\\
13.4000000000001	61.4814814814815\\
13.5999999999998	60.5839416058394\\
13.8	61.1510791366907\\
13.8999999999999	60.7142857142857\\
14.3	61.8055555555556\\
14.5000000000002	60.958904109589\\
14.6000000000002	61.2244897959184\\
14.7	60.8108108108108\\
14.8	61.0738255033557\\
14.9000000000002	60.6666666666667\\
15.0999999999999	61.1842105263158\\
15.3999999999999	60\\
15.7000000000002	60.7594936708861\\
15.9	60\\
16.2	60.7361963190184\\
16.5000000000002	59.6385542168675\\
16.5999999999998	59.8802395209581\\
16.6999999999998	59.5238095238095\\
17.0999999999999	60.4651162790698\\
17.3	59.7701149425287\\
17.4000000000002	60\\
17.7000000000002	58.9887640449438\\
17.8000000000002	59.2178770949721\\
17.9	58.8888888888889\\
18.4	60\\
18.6000000000003	59.3582887700535\\
18.8999999999997	60\\
18.9999999999998	59.6858638743455\\
19.1000000000001	59.8958333333333\\
19.2	59.5854922279793\\
19.2999999999999	59.7938144329897\\
19.6	58.8832487309645\\
19.6999999999998	59.0909090909091\\
19.8000000000002	58.7939698492462\\
19.9000000000003	59\\
20.1000000000002	58.4158415841584\\
20.3999999999998	59.0243902439024\\
20.7	58.1730769230769\\
21.0999999999999	58.9622641509434\\
21.2	58.6854460093897\\
21.2999999999998	58.8785046728972\\
21.4000000000001	58.6046511627907\\
21.5000000000001	58.7962962962963\\
22.1000000000003	57.2072072072072\\
22.9000000000001	58.695652173913\\
23	58.4415584415584\\
23.0999999999998	58.6206896551724\\
23.2000000000003	58.3690987124464\\
23.2999999999999	58.5470085470086\\
23.4999999999996	58.0508474576271\\
23.6000000000002	58.2278481012658\\
23.8000000000001	57.7405857740586\\
23.9000000000004	57.9166666666667\\
24.1000000000002	57.4380165289256\\
24.4000000000002	57.9591836734694\\
24.5	57.7235772357724\\
24.5999999999997	57.8947368421053\\
24.8999999999997	57.2\\
25.3999999999997	58.0392156862745\\
25.6	57.5875486381323\\
26.2000000000003	58.5551330798479\\
26.2999999999998	58.3333333333333\\
26.5	58.6466165413534\\
26.7000000000004	58.2089552238806\\
26.8000000000003	58.364312267658\\
27	57.9335793357934\\
27.1999999999997	58.2417582417582\\
27.3000000000002	58.029197080292\\
27.5000000000002	58.3333333333333\\
27.7999999999999	57.7060931899642\\
27.9999999999996	58.0071174377224\\
28.2999999999996	57.3943661971831\\
28.3999999999997	57.5438596491228\\
28.6	57.1428571428571\\
28.9000000000003	57.5862068965517\\
29.2000000000004	56.9965870307167\\
29.4	57.2881355932203\\
29.5000000000003	57.0945945945946\\
29.8999999999998	57.6666666666667\\
30	57.4750830564784\\
30.0999999999997	57.6158940397351\\
30.1999999999997	57.4257425742574\\
30.3999999999998	57.7049180327869\\
30.5999999999997	57.328990228013\\
30.7999999999998	57.6051779935275\\
31.3	56.687898089172\\
31.4000000000003	56.8253968253968\\
31.6000000000003	56.4668769716088\\
31.7	56.6037735849057\\
31.8	56.4263322884013\\
32.7000000000004	57.6219512195122\\
32.7999999999996	57.4468085106383\\
33.0000000000003	57.7039274924471\\
33.1000000000001	57.5301204819277\\
33.3	57.7844311377246\\
33.5000000000003	57.4404761904762\\
33.6999999999995	57.6923076923077\\
33.7999999999995	57.5221238938053\\
33.9999999999996	57.7712609970675\\
34.1999999999997	57.4344023323615\\
34.6	57.9250720461095\\
34.7000000000005	57.7586206896552\\
34.8000000000004	57.8796561604585\\
34.9000000000003	57.7142857142857\\
35.0000000000006	57.8347578347578\\
35.0999999999998	57.6704545454545\\
35.3000000000001	57.909604519774\\
35.4999999999996	57.5842696629214\\
35.8999999999998	58.0555555555556\\
36.2000000000004	57.5757575757576\\
36.4000000000003	57.8082191780822\\
36.5000000000005	57.6502732240437\\
36.6000000000003	57.7656675749319\\
36.7000000000004	57.6086956521739\\
37.2000000000005	58.1769436997319\\
37.5	57.7127659574468\\
37.5999999999994	57.8249336870027\\
37.7999999999994	57.5197889182058\\
37.9999999999997	57.742782152231\\
38.0999999999996	57.5916230366492\\
38.4	57.9220779220779\\
38.4999999999997	57.7720207253886\\
38.5999999999998	57.8811369509044\\
38.7999999999997	57.5835475578406\\
39.3999999999996	58.2278481012658\\
39.5000000000004	58.0808080808081\\
39.6999999999994	58.2914572864322\\
40.3000000000002	57.4257425742574\\
40.5000000000001	57.6354679802956\\
40.5999999999998	57.4938574938575\\
40.9	57.8048780487805\\
40.9999999999995	57.6642335766423\\
41.1999999999994	57.8692493946731\\
41.2999999999997	57.7294685990338\\
41.4999999999995	57.9326923076923\\
41.5999999999997	57.7937649880096\\
41.6999999999998	57.8947368421053\\
41.8	57.7565632458234\\
41.9999999999993	57.957244655582\\
42.0999999999999	57.8199052132701\\
42.4999999999995	58.2159624413146\\
42.6999999999997	57.9439252336449\\
43.0999999999998	58.3333333333333\\
43.2000000000001	58.1986143187067\\
43.5000000000005	58.4862385321101\\
43.8999999999993	57.9545454545455\\
44.2000000000006	58.2392776523702\\
44.4	57.9775280898876\\
44.7999999999993	58.3518930957684\\
44.8999999999996	58.2222222222222\\
45.3999999999994	58.6813186813187\\
45.5000000000004	58.5526315789474\\
46.1	59.0909090909091\\
46.1999999999997	58.963282937365\\
46.2999999999994	59.051724137931\\
46.4000000000005	58.9247311827957\\
46.7999999999997	59.275053304904\\
46.8999999999997	59.1489361702128\\
47.2000000000005	59.4080338266385\\
47.3000000000007	59.2827004219409\\
47.4000000000005	59.3684210526316\\
47.6000000000002	59.1194968553459\\
47.8999999999998	59.375\\
48	59.2515592515593\\
48.0999999999996	59.3360995850622\\
48.2000000000005	59.2132505175984\\
48.8000000000004	59.7137014314928\\
48.8999999999993	59.5918367346939\\
49.0999999999992	59.7560975609756\\
49.3	59.5141700404858\\
49.6000000000007	59.7585513078471\\
49.7000000000003	59.6385542168675\\
49.9	59.8\\
50	59.6806387225549\\
50.0999999999994	59.7609561752988\\
50.1999999999998	59.6421471172962\\
50.3000000000004	59.7222222222222\\
50.5000000000003	59.4861660079051\\
50.6000000000005	59.5660749506903\\
50.6999999999992	59.4488188976378\\
50.7999999999994	59.5284872298625\\
50.8999999999993	59.4117647058824\\
51.9	60.1923076923077\\
51.9999999999996	60.0767754318618\\
52.0999999999994	60.1532567049808\\
52.4000000000006	59.8095238095238\\
52.5999999999997	59.9620493358634\\
52.9	59.622641509434\\
53.2000000000005	59.8499061913696\\
53.3000000000007	59.7378277153558\\
53.4000000000007	59.8130841121495\\
53.5000000000002	59.7014925373134\\
53.6000000000005	59.7765363128492\\
53.6999999999999	59.6654275092937\\
53.9000000000005	59.8148148148148\\
54.0000000000001	59.7042513863216\\
54.0999999999999	59.7785977859779\\
54.2000000000006	59.6685082872928\\
54.5000000000007	59.8901098901099\\
54.6999999999992	59.6715328467153\\
55.5	60.2517985611511\\
55.5999999999998	60.1436265709156\\
55.7000000000004	60.2150537634409\\
55.8000000000008	60.1073345259392\\
56.9999999999995	60.9457092819615\\
57.0999999999993	60.8391608391609\\
57.4000000000002	61.0434782608696\\
57.5	60.9375\\
57.6999999999999	61.0726643598616\\
57.8999999999997	60.8620689655173\\
58.3000000000004	61.1301369863014\\
58.4999999999996	60.9215017064846\\
58.7000000000004	61.0544217687075\\
59.0000000000005	60.7445008460237\\
59.0999999999995	60.8108108108108\\
59.2	60.70826306914\\
59.6000000000009	60.9715242881072\\
59.7999999999996	60.7679465776294\\
60.1000000000002	60.9634551495017\\
60.1999999999994	60.8623548922056\\
60.3000000000006	60.9271523178808\\
60.4999999999991	60.7260726072607\\
60.6000000000003	60.7907742998353\\
60.6999999999994	60.6907894736842\\
60.8999999999997	60.8196721311475\\
61.1000000000006	60.6209150326797\\
61.4000000000001	60.8130081300813\\
61.4999999999993	60.7142857142857\\
61.7999999999991	60.9046849757674\\
61.8999999999991	60.8064516129032\\
62.0000000000008	60.8695652173913\\
62.1000000000001	60.7717041800643\\
62.2999999999998	60.8974358974359\\
62.3999999999996	60.8\\
62.6	60.9250398724083\\
62.8000000000007	60.731319554849\\
62.8999999999993	60.7936507936508\\
62.999999999999	60.6973058637084\\
63.1000000000001	60.7594936708861\\
63.2000000000006	60.6635071090047\\
63.7999999999992	61.0328638497653\\
63.8999999999994	60.9375\\
64.8999999999991	61.5384615384615\\
64.9999999999995	61.4439324116744\\
65.4000000000009	61.6793893129771\\
65.5000000000007	61.5853658536585\\
65.5999999999992	61.6438356164384\\
65.700000000001	61.5501519756839\\
66.1000000000007	61.7824773413897\\
66.2000000000003	61.6892911010558\\
67.0000000000006	62.1460506706408\\
67.099999999999	62.0535714285714\\
67.399999999999	62.2222222222222\\
67.5000000000001	62.1301775147929\\
67.599999999999	62.1861152141802\\
67.9000000000006	61.9117647058824\\
68.8000000000004	62.4092888243832\\
68.8999999999993	62.3188405797101\\
69.0000000000001	62.3733719247467\\
69.2999999999996	62.1037463976945\\
69.4999999999997	62.2126436781609\\
69.8000000000006	61.9456366237482\\
70.0000000000012	62.0542082738944\\
70.4000000000007	61.7021276595745\\
70.6999999999992	61.864406779661\\
70.8000000000007	61.7771509167842\\
70.8999999999995	61.830985915493\\
70.9999999999993	61.7440225035162\\
71.1000000000007	61.7977528089888\\
71.200000000001	61.711079943899\\
71.4000000000004	61.8181818181818\\
71.5000000000001	61.731843575419\\
71.5999999999999	61.7852161785216\\
71.7000000000011	61.6991643454039\\
71.7999999999995	61.7524339360223\\
72.0000000000001	61.5811373092926\\
72.1000000000001	61.6343490304709\\
72.2000000000012	61.5491009681881\\
72.6999999999998	61.8131868131868\\
72.9000000000002	61.6438356164384\\
73.0000000000011	61.6963064295486\\
73.0999999999995	61.6120218579235\\
73.4000000000008	61.7687074829932\\
73.5000000000001	61.6847826086957\\
73.7999999999992	61.8403247631935\\
73.8999999999997	61.7567567567568\\
74.2000000000011	61.9111709286676\\
74.3000000000001	61.8279569892473\\
74.400000000001	61.8791946308725\\
74.5000000000011	61.7962466487936\\
74.7999999999998	61.9492656875834\\
74.8999999999991	61.8666666666667\\
75.5999999999989	62.21928665786\\
75.8000000000003	62.0553359683795\\
75.9999999999994	62.1550591327201\\
76.1000000000009	62.0734908136483\\
76.1999999999992	62.1231979030144\\
76.4000000000004	61.9607843137255\\
76.4999999999991	62.0104438642298\\
76.5999999999992	61.9295958279009\\
76.7000000000001	61.9791666666667\\
76.8000000000001	61.8985695708713\\
77.1000000000009	62.0466321243523\\
77.1999999999996	61.9663648124192\\
77.500000000001	62.1134020618557\\
77.6999999999988	61.9537275064267\\
77.9999999999994	62.0998719590269\\
78.1999999999991	61.941251596424\\
78.300000000001	61.9897959183673\\
78.4000000000001	61.9108280254777\\
78.6000000000009	62.007623888183\\
78.8999999999996	61.7721518987342\\
79.3000000000001	61.9647355163728\\
79.3999999999989	61.8867924528302\\
80.4000000000008	62.360248447205\\
80.8000000000005	62.0519159456119\\
80.8999999999994	62.0987654320988\\
81.0000000000001	62.0221948212084\\
81.4000000000007	62.2085889570552\\
81.4999999999987	62.1323529411765\\
81.999999999999	62.3629719853837\\
82.1000000000012	62.2871046228711\\
82.8999999999988	62.6506024096386\\
82.9999999999989	62.575210589651\\
83.1999999999994	62.6650660264106\\
83.299999999999	62.589928057554\\
83.3999999999997	62.6347305389222\\
83.499999999999	62.5598086124402\\
83.7999999999989	62.6936829558999\\
83.9999999999987	62.5445897740785\\
84.1999999999998	62.6334519572954\\
84.4999999999987	62.4113475177305\\
84.5999999999988	62.4557260920897\\
84.7000000000001	62.3820754716981\\
84.8999999999987	62.4705882352941\\
84.999999999999	62.3971797884841\\
85.4999999999993	62.6168224299065\\
85.6000000000004	62.5437572928821\\
86.1000000000004	62.7610208816705\\
86.1999999999996	62.6882966396292\\
86.2999999999991	62.7314814814815\\
86.4000000000001	62.6589595375723\\
86.5999999999986	62.7450980392157\\
86.7000000000008	62.6728110599078\\
86.7999999999998	62.7157652474108\\
86.8999999999987	62.6436781609195\\
87.3999999999993	62.8571428571429\\
87.6000000000013	62.7137970353478\\
87.6999999999998	62.7562642369021\\
87.7999999999986	62.6848691695108\\
88.0000000000008	62.7695800227015\\
88.0999999999999	62.6984126984127\\
88.3000000000001	62.7828054298643\\
88.5000000000009	62.6410835214447\\
89.1000000000009	62.8923766816144\\
89.3000000000008	62.751677852349\\
89.4000000000013	62.7932960893855\\
89.4999999999998	62.7232142857143\\
89.6999999999994	62.8062360801782\\
89.7999999999993	62.7363737486096\\
90.0000000000001	62.819089900111\\
90.1999999999997	62.6799557032115\\
90.7999999999989	62.9262926292629\\
91.0000000000008	62.788144895719\\
91.1999999999993	62.8696604600219\\
91.3999999999987	62.7322404371585\\
91.6000000000003	62.8135223555071\\
91.6999999999995	62.7450980392157\\
91.799999999999	62.7856365614799\\
92.0000000000001	62.6492942453854\\
92.1000000000001	62.6898047722343\\
92.5	62.4190064794817\\
93.0999999999993	62.6609442060086\\
93.1999999999998	62.593783494105\\
93.4000000000005	62.6737967914438\\
93.4999999999997	62.6068376068376\\
93.9	62.7659574468085\\
94.1000000000004	62.6326963906582\\
94.2000000000011	62.6723223753977\\
94.2999999999989	62.6059322033898\\
94.6000000000014	62.7243928194298\\
94.8000000000009	62.5922023182297\\
94.9000000000006	62.6315789473684\\
95.2000000000005	62.4344176285415\\
95.2999999999988	62.4737945492662\\
95.3999999999999	62.4083769633508\\
95.6000000000015	62.4869383490073\\
95.6999999999988	62.4217118997912\\
95.9000000000006	62.5\\
96.0000000000001	62.4349635796046\\
96.0999999999994	62.4740124740125\\
96.1999999999993	62.4091381100727\\
96.400000000001	62.4870466321244\\
96.5999999999985	62.3578076525336\\
96.6999999999994	62.396694214876\\
96.7999999999986	62.3323013415893\\
96.9000000000012	62.3711340206186\\
96.9999999999993	62.3069001029866\\
97.2000000000002	62.3843782117163\\
97.3000000000012	62.3203285420945\\
97.7000000000011	62.4744376278119\\
97.899999999999	62.3469387755102\\
98.3	62.5\\
98.3999999999989	62.4365482233503\\
98.4999999999991	62.474645030426\\
98.7000000000002	62.3481781376518\\
99.000000000001	62.4621594349142\\
99.3000000000005	62.2736418511066\\
99.5000000000015	62.3493975903614\\
99.5999999999988	62.2868605817452\\
100.500000000001	62.624254473161\\
100.600000000001	62.5620655412115\\
101.000000000001	62.7101879327399\\
101.099999999999	62.6482213438735\\
101.200000000001	62.685093780849\\
101.299999999999	62.6232741617357\\
101.699999999999	62.770137524558\\
101.799999999999	62.7085377821394\\
101.899999999999	62.7450980392157\\
101.999999999999	62.6836434867777\\
102.4	62.8292682926829\\
102.499999999999	62.7680311890838\\
102.599999999999	62.8042843232717\\
102.7	62.7431906614786\\
102.800000000001	62.779397473275\\
102.899999999998	62.7184466019417\\
102.999999999998	62.7546071774976\\
103.1	62.6937984496124\\
103.3	62.7659574468085\\
103.399999999999	62.7053140096618\\
103.900000000001	62.8846153846154\\
104.400000000001	62.5837320574163\\
104.5	62.6195028680688\\
104.800000000001	62.4404194470925\\
105.299999999999	62.618595825427\\
105.600000000001	62.4408703878903\\
106.1	62.6177024482109\\
106.299999999999	62.5\\
106.499999999999	62.5703564727955\\
106.800000000001	62.3947614593078\\
107.200000000001	62.5349487418453\\
107.300000000001	62.4767225325885\\
107.4	62.5116279069767\\
107.500000000001	62.453531598513\\
107.800000000001	62.5579240037071\\
108	62.4421831637373\\
108.099999999998	62.4768946395564\\
108.400000000001	62.3041474654378\\
108.600000000001	62.3735050597976\\
108.799999999999	62.2589531680441\\
108.899999999998	62.2935779816514\\
109.000000000001	62.2364802933089\\
109.300000000001	62.3400365630713\\
109.500000000002	62.2262773722628\\
109.899999999999	62.3636363636364\\
110.000000000001	62.3069936421435\\
110.100000000001	62.3411978221416\\
110.4	62.1719457013575\\
110.500000000001	62.2061482820977\\
110.699999999999	62.0938628158845\\
110.8	62.1280432822362\\
111.300000000002	61.8491921005386\\
111.400000000001	61.8834080717489\\
111.499999999999	61.8279569892473\\
111.600000000002	61.8621307072516\\
111.699999999999	61.8067978533095\\
112.1	61.9429590017825\\
112.2	61.8878005342832\\
112.799999999998	62.0903454384411\\
113.1	61.9257950530035\\
113.299999999999	61.9929453262787\\
113.399999999998	61.9383259911894\\
113.800000000001	62.071992976295\\
113.9	62.0175438596491\\
113.999999999999	62.0508326029799\\
114.1	61.9964973730298\\
114.199999999999	62.0297462817148\\
114.4	61.9213973799127\\
114.600000000001	61.9877942458588\\
114.700000000002	61.9337979094077\\
115	62.0330147697654\\
115.100000000001	61.9791666666667\\
115.499999999999	62.1107266435986\\
115.600000000001	62.057044079516\\
117.1	62.542662116041\\
117.299999999999	62.4361158432709\\
117.499999999998	62.5\\
117.6	62.446898895497\\
118.199999999999	62.6373626373626\\
118.4	62.5316455696203\\
118.599999999999	62.5947767481045\\
118.700000000001	62.5420875420875\\
119.000000000001	62.6364399664148\\
119.200000000002	62.5314333612741\\
119.599999999999	62.65664160401\\
119.700000000001	62.6043405676127\\
119.899999999999	62.6666666666667\\
120.099999999999	62.5623960066556\\
120.299999999999	62.624584717608\\
120.500000000001	62.5207296849088\\
120.799999999999	62.6137303556659\\
120.900000000001	62.5619834710744\\
120.999999999998	62.5928984310487\\
121.399999999999	62.3868312757202\\
121.599999999999	62.4486442070666\\
122.000000000001	62.2440622440622\\
122.100000000001	62.2749590834697\\
122.200000000001	62.2240392477514\\
122.3	62.2549019607843\\
122.5	62.1533442088091\\
122.7	62.2149837133551\\
123.599999999998	61.7623282134196\\
123.799999999998	61.8240516545601\\
123.899999999999	61.7741935483871\\
124.3	61.8971061093248\\
124.400000000001	61.8473895582329\\
124.499999999998	61.8780096308186\\
124.6	61.8283881315156\\
124.799999999999	61.8895116092874\\
124.9	61.84\\
125	61.8705035971223\\
125.5	61.6242038216561\\
125.799999999999	61.7156473391581\\
125.999999999998	61.6177636796194\\
126.400000000001	61.7391304347826\\
126.6	61.6416732438832\\
126.699999999999	61.6719242902208\\
126.899999999998	61.5748031496063\\
127.700000000001	61.8153364632238\\
127.799999999999	61.7670054730258\\
127.9	61.796875\\
128.099999999999	61.7004680187207\\
128.299999999999	61.7601246105919\\
128.400000000001	61.7120622568093\\
128.600000000002	61.7715617715618\\
128.900000000001	61.6279069767442\\
129.299999999999	61.7465224111283\\
129.399999999998	61.6988416988417\\
129.500000000002	61.7283950617284\\
129.6	61.6808018504241\\
129.699999999999	61.7103235747304\\
129.799999999998	61.662817551963\\
129.899999999998	61.6923076923077\\
130.2	61.5502686108979\\
130.3	61.579754601227\\
130.399999999998	61.5325670498084\\
130.7	61.6207951070336\\
130.800000000002	61.5737203972498\\
130.900000000002	61.6030534351145\\
131.400000000002	61.3688212927757\\
131.900000000001	61.5151515151515\\
132.000000000001	61.4685844057532\\
132.099999999998	61.4977307110439\\
132.200000000001	61.4512471655329\\
132.5	61.5384615384615\\
132.600000000002	61.4920874152223\\
132.7	61.5210843373494\\
132.9	61.4285714285714\\
133.099999999999	61.4864864864865\\
133.299999999998	61.3943028485757\\
133.400000000002	61.4232209737828\\
133.500000000002	61.377245508982\\
133.699999999999	61.4349775784753\\
133.799999999999	61.3890963405527\\
134.399999999998	61.5613382899628\\
134.499999999999	61.5156017830609\\
134.699999999999	61.5727002967359\\
134.900000000002	61.4814814814815\\
135.100000000001	61.5384615384615\\
135.199999999998	61.4929785661493\\
135.399999999998	61.549815498155\\
135.500000000002	61.5044247787611\\
135.800000000001	61.5894039735099\\
135.899999999999	61.5441176470588\\
135.999999999998	61.5723732549596\\
136.099999999999	61.5271659324523\\
136.299999999999	61.58357771261\\
136.399999999999	61.5384615384615\\
137.699999999999	61.9013062409289\\
137.799999999999	61.8564176939812\\
137.900000000001	61.8840579710145\\
138.099999999998	61.794500723589\\
138.200000000001	61.822125813449\\
138.4	61.7328519855596\\
138.5	61.7604617604618\\
138.599999999999	61.7159336697909\\
138.700000000001	61.7435158501441\\
138.899999999998	61.6546762589928\\
139.200000000002	61.7372577171572\\
139.3	61.6929698708752\\
139.400000000001	61.7204301075269\\
139.500000000002	61.676217765043\\
139.7	61.7310443490701\\
139.8	61.68691922802\\
140.000000000002	61.7416131334761\\
140.100000000001	61.69757489301\\
140.800000000001	61.8878637331441\\
140.899999999999	61.8439716312057\\
140.999999999998	61.8710134656272\\
141.100000000001	61.8271954674221\\
141.2	61.8542108987969\\
141.399999999998	61.7667844522968\\
141.499999999998	61.7937853107345\\
141.600000000001	61.7501764290755\\
141.799999999999	61.8040873854827\\
141.900000000002	61.7605633802817\\
142.300000000002	61.8679775280899\\
142.499999999999	61.781206171108\\
142.7	61.8347338935574\\
142.800000000001	61.7914625612316\\
143.400000000002	61.9512195121951\\
143.599999999999	61.8649965205289\\
143.7	61.8915159944367\\
143.899999999998	61.8055555555556\\
144.099999999998	61.8585298196949\\
144.400000000002	61.7301038062284\\
144.500000000001	61.7565698478562\\
144.899999999998	61.5862068965517\\
145.000000000002	61.6126809097174\\
145.199999999998	61.5278733654508\\
145.8	61.6860863605209\\
146.000000000002	61.6016427104723\\
146.199999999999	61.6541353383459\\
146.299999999998	61.6120218579235\\
146.900000000002	61.7687074829932\\
147.2	61.642905634759\\
147.299999999998	61.6689280868385\\
147.700000000002	61.5020297699594\\
147.799999999999	61.5280594996619\\
148	61.4449696151249\\
148.299999999999	61.5229110512129\\
148.400000000002	61.4814814814815\\
148.6	61.5332885003363\\
148.7	61.491935483871\\
149.100000000001	61.5951742627346\\
149.200000000002	61.5539182853315\\
149.399999999998	61.6053511705686\\
149.6	61.5230460921844\\
150	61.6255829447035\\
150.199999999999	61.5435795076514\\
150.399999999998	61.5946843853821\\
150.6	61.5129396151294\\
150.800000000001	61.5639496355202\\
151.100000000001	61.441798941799\\
151.3	61.4927344782034\\
151.399999999999	61.4521452145215\\
151.600000000001	61.5029663810152\\
151.699999999998	61.4624505928854\\
151.800000000001	61.4878209348255\\
152.099999999998	61.3666228646518\\
152.3	61.4173228346457\\
152.699999999998	61.2565445026178\\
153.100000000002	61.3577023498695\\
153.199999999999	61.3176777560339\\
153.3	61.3428943937419\\
153.5	61.2630208333333\\
153.700000000001	61.3133940182055\\
153.999999999999	61.1940298507463\\
154.200000000002	61.2443292287751\\
154.399999999999	61.1650485436893\\
154.600000000002	61.2152553329024\\
154.7	61.1757105943152\\
155.000000000002	61.250805931657\\
155.199999999999	61.1719253058596\\
155.300000000002	61.1969111969112\\
155.399999999998	61.1575562700965\\
155.500000000001	61.1825192802057\\
155.7	61.1039794608472\\
155.800000000002	61.1289288005132\\
155.9	61.0897435897436\\
155.999999999999	61.11467008328\\
156.100000000001	61.0755441741357\\
156.2	61.1004478566859\\
156.299999999998	61.0613810741688\\
156.399999999998	61.0862619808307\\
156.5	61.0472541507024\\
157.4	61.2698412698413\\
157.499999999998	61.2309644670051\\
157.599999999999	61.2555485098288\\
157.700000000002	61.2167300380228\\
158.000000000001	61.2903225806452\\
158.1	61.251580278129\\
158.5	61.3493064312736\\
158.900000000001	61.1949685534591\\
159.100000000001	61.2437185929648\\
159.200000000003	61.2052730696798\\
159.399999999998	61.2539184952978\\
160.200000000002	60.9482220835933\\
160.299999999998	60.9725685785536\\
160.500000000001	60.8966376089664\\
160.800000000002	60.969546302051\\
161.1	60.8560794044665\\
161.200000000001	60.8803471791692\\
161.399999999999	60.8049535603715\\
161.500000000002	60.8292079207921\\
161.600000000001	60.7915893630179\\
162.099999999999	60.9124537607892\\
162.199999999998	60.8749229821318\\
162.600000000002	60.9711124769514\\
162.699999999999	60.9336609336609\\
162.800000000001	60.9576427255985\\
162.999999999997	60.8828939301042\\
163.100000000002	60.906862745098\\
163.199999999998	60.8695652173913\\
163.299999999998	60.8935128518972\\
163.899999999998	60.6707317073171\\
164.200000000002	60.7425441265977\\
164.399999999998	60.6686930091186\\
164.6	60.7164541590771\\
164.799999999998	60.6428138265616\\
164.900000000002	60.6666666666667\\
165.000000000002	60.6299212598425\\
165.100000000002	60.6537530266344\\
165.199999999999	60.6170598911071\\
165.300000000001	60.6408706166868\\
165.399999999998	60.6042296072508\\
165.500000000001	60.6280193236715\\
166.2	60.3728202044498\\
166.5	60.4441776710684\\
166.699999999998	60.3717026378897\\
167.100000000001	60.4665071770335\\
167.2	60.430364614465\\
167.400000000002	60.4776119402985\\
167.500000000002	60.4415274463007\\
167.599999999998	60.4651162790698\\
167.7	60.4290822407628\\
167.8	60.4526503871352\\
167.900000000002	60.4166666666667\\
168.199999999998	60.4872251931075\\
168.499999999998	60.3795966785291\\
168.8	60.4499703966844\\
168.999999999997	60.3784742755766\\
169.6	60.5185621685327\\
169.7	60.4829210836278\\
169.999999999998	60.5526161081717\\
170.3	60.4460093896714\\
170.600000000002	60.5155243116579\\
170.700000000001	60.480093676815\\
170.799999999999	60.5032182562902\\
170.899999999998	60.4678362573099\\
170.999999999999	60.4909409701929\\
171.099999999998	60.4556074766355\\
171.299999999998	60.5017502917153\\
171.500000000003	60.4312354312354\\
172.1	60.5691056910569\\
172.200000000001	60.5339524085897\\
172.300000000002	60.5568445475638\\
172.399999999999	60.5217391304348\\
172.5	60.5446118192352\\
172.699999999998	60.474537037037\\
173	60.5430387059503\\
173.1	60.5080831408776\\
173.200000000003	60.530871321408\\
173.400000000002	60.4610951008646\\
173.500000000003	60.4838709677419\\
173.6	60.4490500863558\\
173.799999999997	60.4945370902818\\
174.000000000002	60.4250430786904\\
174.300000000001	60.4931192660551\\
174.400000000002	60.458452722063\\
174.900000000001	60.5714285714286\\
175.000000000003	60.5368360936608\\
175.8	60.7163160886868\\
176.000000000002	60.6473594548552\\
176.299999999999	60.7142857142857\\
176.4	60.6798866855524\\
176.5	60.7021517553794\\
176.6	60.6677985285795\\
177.000000000002	60.7566346696781\\
177.299999999998	60.6538895152198\\
177.6	60.7203151378728\\
177.800000000001	60.6520517144463\\
178.000000000002	60.6962380685008\\
178.100000000003	60.662177328844\\
178.200000000002	60.6842400448682\\
178.299999999999	60.6502242152466\\
178.399999999998	60.672268907563\\
178.600000000002	60.6043648573027\\
178.700000000002	60.6263982102908\\
178.800000000003	60.5925097820011\\
179.100000000003	60.6584821428571\\
179.199999999997	60.6246514221974\\
179.299999999999	60.6465997770346\\
179.499999999999	60.5790645879733\\
179.700000000002	60.622914349277\\
179.799999999999	60.5892162312396\\
180.100000000001	60.6548279689234\\
180.199999999998	60.6211869107044\\
180.300000000001	60.6430155210643\\
180.399999999999	60.6094182825485\\
180.500000000003	60.6312292358804\\
180.599999999998	60.5976757055894\\
181.000000000002	60.6847045831033\\
181.100000000003	60.6512141280353\\
181.599999999998	60.7594936708861\\
181.800000000001	60.6926882902694\\
181.899999999999	60.7142857142857\\
182.000000000002	60.6809445359693\\
182.199999999998	60.7240811848601\\
182.399999999999	60.6575342465753\\
182.699999999999	60.7221006564552\\
182.799999999998	60.688901038819\\
183.000000000002	60.7318405243037\\
183.099999999999	60.6986899563319\\
183.899999999999	60.8695652173913\\
184	60.8365019011407\\
184.100000000002	60.8577633007601\\
184.300000000003	60.7917570498915\\
184.4	60.8130081300813\\
184.700000000002	60.7142857142857\\
184.799999999999	60.7355327203894\\
184.900000000002	60.7027027027027\\
185	60.7239330091842\\
185.099999999998	60.6911447084233\\
185.199999999997	60.7123583378306\\
185.3	60.6796116504854\\
185.399999999997	60.7008086253369\\
185.500000000003	60.6681034482759\\
185.600000000002	60.6892837910609\\
185.899999999999	60.5913978494624\\
186.100000000002	60.6337271750806\\
186.3	60.568669527897\\
186.500000000002	60.6109324758843\\
186.700000000001	60.5460385438972\\
186.800000000001	60.5671482075976\\
186.9	60.5347593582888\\
186.999999999999	60.555852485302\\
187.199999999999	60.4911906033102\\
187.299999999999	60.5122732123799\\
187.700000000003	60.3833865814697\\
187.899999999999	60.4255319148936\\
187.999999999997	60.3934077618288\\
188.400000000002	60.4774535809019\\
188.500000000001	60.4453870625663\\
188.800000000002	60.5082053996824\\
188.900000000002	60.4761904761905\\
189.099999999998	60.5179704016913\\
189.200000000003	60.486001056524\\
189.699999999999	60.5900948366702\\
189.800000000001	60.5581885202738\\
189.9	60.5789473684211\\
190.400000000001	60.4199475065617\\
190.8	60.5028810895757\\
191.000000000001	60.4395604395604\\
191.3	60.5015673981191\\
191.399999999998	60.4699738903394\\
191.499999999998	60.490605427975\\
191.699999999998	60.4275286757039\\
192	60.4893284747527\\
192.199999999999	60.4264170566823\\
192.399999999999	60.4675324675325\\
192.600000000001	60.4047742605086\\
192.800000000002	60.4458268532919\\
192.900000000003	60.4145077720207\\
193.099999999998	60.4554865424431\\
193.300000000002	60.3929679420889\\
193.599999999997	60.4543107898813\\
193.800000000003	60.3919546157813\\
193.999999999999	60.4327666151468\\
194.200000000001	60.3705609881626\\
194.299999999998	60.3909465020576\\
194.4	60.3598971722365\\
194.699999999997	60.4209445585216\\
194.800000000002	60.3899435608004\\
194.899999999999	60.4102564102564\\
194.999999999999	60.3792926704254\\
195.099999999999	60.3995901639344\\
195.3	60.3377686796315\\
195.400000000002	60.3580562659847\\
195.499999999998	60.3271983640082\\
195.700000000002	60.367722165475\\
195.799999999998	60.3369065849923\\
195.9	60.3571428571429\\
196.1	60.2956167176351\\
196.500000000002	60.3763987792472\\
196.699999999999	60.3150406504065\\
196.799999999998	60.3351955307263\\
196.900000000001	60.3045685279188\\
197.299999999999	60.3850050658561\\
197.4	60.3544303797468\\
197.500000000002	60.3744939271255\\
197.600000000003	60.3439554881133\\
197.899999999998	60.4040404040404\\
198.000000000002	60.3735487127713\\
198.199999999999	60.4135148764498\\
198.300000000002	60.383064516129\\
198.800000000001	60.4826546003017\\
198.900000000003	60.4522613065327\\
199.000000000003	60.4721245605224\\
199.199999999998	60.4114400401405\\
199.500000000002	60.4709418837675\\
199.6	60.4406609914872\\
200.6	60.6377678126557\\
200.699999999998	60.6075697211155\\
201.500000000001	60.7638888888889\\
201.800000000001	60.6736007924715\\
201.9	60.6930693069307\\
202.000000000001	60.6630380999505\\
202.099999999998	60.6824925816024\\
202.199999999997	60.6524962926347\\
202.400000000002	60.6913580246914\\
202.5	60.6614017769003\\
202.899999999997	60.7389162561576\\
202.999999999999	60.7090103397341\\
203.199999999998	60.7476635514019\\
203.299999999999	60.7177974434612\\
203.599999999997	60.7756504663721\\
203.700000000002	60.7458292443572\\
203.899999999997	60.7843137254902\\
203.999999999998	60.7545320921117\\
204.199999999998	60.7929515418502\\
204.4	60.7334963325183\\
204.599999999998	60.771861260381\\
204.700000000003	60.7421875\\
204.899999999999	60.780487804878\\
204.999999999997	60.7508532423208\\
205.100000000003	60.7699805068226\\
205.299999999997	60.7108081791626\\
205.499999999998	60.7490272373541\\
205.600000000002	60.719494409334\\
205.700000000001	60.7385811467444\\
205.9	60.6796116504854\\
205.999999999997	60.6986899563319\\
206.100000000001	60.6692531522794\\
206.2	60.6883179835192\\
206.300000000003	60.6589147286822\\
206.400000000001	60.6779661016949\\
206.499999999999	60.648596321394\\
206.699999999998	60.6866537717602\\
206.900000000002	60.6280193236715\\
207	60.6470304200869\\
207.200000000004	60.5885190545104\\
207.399999999999	60.6265060240964\\
207.499999999997	60.597302504817\\
207.6	60.6162734713529\\
207.699999999998	60.5871029836381\\
207.999999999998	60.6439211917347\\
208.1	60.6147934678194\\
208.299999999997	60.6525911708253\\
208.6	60.5654048873982\\
208.800000000003	60.6031594064146\\
208.899999999998	60.5741626794258\\
209	60.5930176948828\\
209.100000000002	60.5640535372849\\
209.200000000002	60.5828953655041\\
209.300000000002	60.5539637058262\\
209.499999999997	60.5916030534351\\
209.600000000002	60.5627086313782\\
209.800000000003	60.600285850405\\
209.900000000002	60.5714285714286\\
210.199999999999	60.6276747503566\\
210.700000000001	60.4838709677419\\
210.899999999998	60.521327014218\\
210.999999999999	60.4926575082899\\
211.100000000003	60.5113636363636\\
211.3	60.4541154210028\\
211.6	60.5101558809636\\
211.8	60.4530438886267\\
211.900000000002	60.4716981132076\\
212.099999999998	60.4147031102733\\
212.200000000001	60.4333490343853\\
212.699999999999	60.2913533834587\\
212.9	60.3286384976526\\
213.1	60.2720450281426\\
213.300000000002	60.3092783505155\\
213.4	60.2810304449649\\
213.799999999998	60.3553062178588\\
214.000000000001	60.2989257356376\\
215.299999999998	60.5385329619313\\
215.399999999999	60.5104408352668\\
215.600000000002	60.5470560964302\\
215.800000000003	60.4909680407596\\
216	60.5275335492828\\
216.100000000003	60.4995374653099\\
216.800000000003	60.6270170585523\\
216.900000000002	60.5990783410138\\
217.200000000002	60.6534744592729\\
217.299999999997	60.6255749770009\\
217.400000000001	60.6436781609195\\
217.500000000002	60.6158088235294\\
217.699999999998	60.6519742883379\\
217.799999999997	60.6241395135383\\
217.9	60.6422018348624\\
218.000000000003	60.6143970655663\\
218.099999999999	60.6324472960587\\
218.300000000002	60.5769230769231\\
218.400000000002	60.5949656750572\\
218.499999999998	60.5672461116194\\
218.799999999997	60.6212882594792\\
219.000000000003	60.5659516202647\\
219.199999999998	60.6019151846785\\
219.399999999997	60.5466970387244\\
220.000000000002	60.6542480690595\\
220.100000000001	60.6267029972752\\
220.400000000001	60.6802721088435\\
220.5	60.6527651858568\\
221.099999999997	60.7594936708861\\
221.3	60.7046070460705\\
221.699999999999	60.775473399459\\
221.900000000004	60.7207207207207\\
222	60.7384061233679\\
222.299999999997	60.6564748201439\\
222.600000000003	60.7094746295465\\
222.7	60.6822262118492\\
222.9	60.7174887892377\\
222.999999999998	60.690273419991\\
223.300000000001	60.7430617726052\\
223.500000000003	60.6887298747764\\
223.700000000001	60.7238605898123\\
223.800000000003	60.6967396159\\
224.2	60.7668301382078\\
224.300000000003	60.7397504456328\\
224.600000000003	60.7921673342234\\
224.800000000003	60.738105824811\\
224.899999999998	60.7555555555556\\
225	60.7285650821857\\
225.999999999997	60.9022556390977\\
226.099999999999	60.8753315649867\\
226.399999999997	60.9271523178808\\
226.499999999998	60.9002647837599\\
226.599999999999	60.917512130569\\
226.799999999997	60.8638166593213\\
227.500000000002	60.9841827768014\\
227.600000000001	60.9574000878349\\
227.700000000002	60.9745390693591\\
227.900000000001	60.9210526315789\\
228.2	60.9724047306176\\
228.300000000003	60.9457092819615\\
228.599999999998	60.9969392216878\\
228.8	60.9436435124508\\
229.099999999997	60.9947643979058\\
229.300000000001	60.9415867480384\\
229.400000000003	60.958605664488\\
229.499999999997	60.9320557491289\\
229.699999999998	60.9660574412533\\
229.900000000002	60.9130434782609\\
230.299999999997	60.9809027777778\\
230.4	60.9544468546638\\
230.600000000001	60.9882964889467\\
230.699999999999	60.9618717504333\\
230.8	60.9787786920745\\
230.999999999998	60.9260060579836\\
231.200000000002	60.9597924773022\\
231.300000000003	60.933448573898\\
231.499999999997	60.9671848013817\\
231.599999999999	60.9408718170047\\
231.8	60.9745579991376\\
231.900000000002	60.948275862069\\
232.100000000001	60.9819121447028\\
232.500000000003	60.8770421324162\\
232.999999999999	60.960960960961\\
233.099999999996	60.9348198970841\\
233.300000000003	60.968294772922\\
233.400000000003	60.9421841541756\\
233.600000000003	60.9756097560976\\
233.700000000003	60.9495295124038\\
234.099999999999	61.0162254483348\\
234.199999999999	60.9901835253948\\
234.3	61.0068259385666\\
234.599999999997	60.9288453344695\\
234.699999999998	60.9454855195911\\
234.800000000002	60.9195402298851\\
234.999999999996	60.95278604849\\
235.300000000001	60.875106202209\\
235.400000000003	60.8917197452229\\
235.500000000003	60.8658743633277\\
235.9	60.9322033898305\\
236.000000000002	60.9063955950868\\
236.200000000001	60.9394837071519\\
236.300000000004	60.9137055837564\\
236.399999999998	60.9302325581395\\
236.500000000003	60.9044801352494\\
236.699999999999	60.9375\\
237.000000000002	60.8603964571911\\
237.099999999999	60.8768971332209\\
237.199999999998	60.8512431521281\\
237.700000000003	60.9335576114382\\
237.9	60.8823529411765\\
238.000000000001	60.8987820243595\\
238.099999999997	60.8732157850546\\
238.400000000004	60.9224318658281\\
238.6	60.8713866778383\\
239.000000000004	60.9368465077374\\
239.099999999997	60.9113712374582\\
239.199999999998	60.9277058086085\\
239.300000000003	60.9022556390977\\
239.600000000003	60.9511889862328\\
239.700000000001	60.9257714762302\\
239.799999999997	60.9420591913297\\
239.900000000004	60.9166666666667\\
240.199999999999	60.9654598418643\\
240.300000000003	60.9400998336107\\
240.400000000001	60.956340956341\\
240.499999999998	60.9310058187864\\
240.7	60.9634551495017\\
240.799999999998	60.9381486093815\\
241.200000000002	61.0029009531703\\
241.300000000003	60.9776304888153\\
241.400000000003	60.9937888198758\\
241.499999999996	60.9685430463576\\
241.800000000001	61.0169491525424\\
241.899999999999	60.9917355371901\\
241.999999999996	61.0078479966956\\
242.099999999998	60.9826589595376\\
242.499999999998	61.0469909315746\\
242.600000000001	61.0218376596621\\
242.699999999998	61.0378912685338\\
243	60.9625668449198\\
243.299999999997	61.0106820049302\\
243.500000000003	60.9605911330049\\
244.400000000002	61.1042944785276\\
244.499999999996	61.07931316435\\
244.699999999999	61.1111111111111\\
245	61.0363117095063\\
245.199999999998	61.0680799021606\\
245.500000000004	60.9934853420196\\
245.6	61.009361009361\\
245.700000000002	60.9845402766477\\
245.8	61.0004066693778\\
245.899999999999	60.9756097560976\\
246.200000000001	61.0231425091352\\
246.300000000004	60.9983766233766\\
246.399999999998	61.0141987829615\\
246.5	60.9894566098946\\
246.9	61.0526315789474\\
247.000000000004	61.0279239174423\\
247.099999999996	61.0436893203884\\
247.199999999997	61.0190052567732\\
247.900000000002	61.1290322580645\\
248.099999999997	61.0797743755036\\
248.500000000002	61.1423974255833\\
248.599999999999	61.1178126256534\\
248.800000000002	61.1490558457212\\
248.999999999997	61.0999598554797\\
249.100000000004	61.115569823435\\
249.199999999999	61.0910549538708\\
249.599999999998	61.1533840608731\\
250.099999999999	61.031175059952\\
250.4	61.0778443113772\\
250.499999999997	61.0534716679968\\
250.600000000004	61.0690067810132\\
250.700000000003	61.0446570972887\\
250.8	61.0601833399761\\
251.100000000004	60.9872611464968\\
251.500000000002	61.0492845786963\\
251.799999999998	60.9765780071457\\
251.900000000003	60.9920634920635\\
251.999999999996	60.9678698928996\\
252.099999999997	60.9833465503569\\
252.500000000001	60.8867775138559\\
253.100000000004	60.9794628751975\\
253.300000000001	60.9313338595107\\
253.900000000004	61.0236220472441\\
253.999999999997	60.9996064541519\\
254.2	61.0302791977979\\
254.300000000002	61.0062893081761\\
254.4	61.0216110019646\\
254.499999999997	60.9976433621367\\
254.999999999997	61.0740885927087\\
255.3	61.0023492560689\\
255.499999999999	61.0328638497653\\
255.599999999998	61.0089949159171\\
255.8	61.0394685423994\\
255.900000000004	61.015625\\
256	61.0308473252636\\
256.100000000004	61.0070257611241\\
256.300000000002	61.0374414976599\\
256.400000000002	61.0136452241715\\
256.499999999998	61.0288386593921\\
256.700000000003	60.981308411215\\
256.800000000001	60.9964966913196\\
256.899999999999	60.9727626459144\\
257.3	61.033411033411\\
257.400000000001	61.0097087378641\\
257.499999999996	61.0248447204969\\
257.599999999996	61.0011641443539\\
257.9	61.046511627907\\
258.000000000002	61.0228593568384\\
258.099999999998	61.0379550735864\\
258.199999999997	61.0143244289586\\
258.700000000002	61.0896445131376\\
258.799999999996	61.0660486674392\\
259.200000000001	61.1261087543386\\
259.399999999999	61.0789980732177\\
259.5	61.0939907550077\\
259.599999999996	61.0704659222179\\
260.100000000003	61.1452728670254\\
260.399999999999	61.0748560460653\\
260.499999999997	61.0897927858787\\
260.600000000001	61.066359800537\\
260.799999999996	61.0962054426983\\
261.000000000003	61.0494063577174\\
261.2	61.0792192881745\\
261.3	61.0558530986993\\
261.500000000002	61.085626911315\\
261.699999999996	61.038961038961\\
261.899999999998	61.0687022900763\\
262.199999999998	60.9988562714449\\
262.699999999997	61.0730593607306\\
262.800000000004	61.0498288322556\\
262.899999999998	61.06463878327\\
262.999999999998	61.0414291144052\\
263.099999999999	61.0562310030395\\
263.5	60.9635811836115\\
263.599999999998	60.9783845278726\\
263.699999999999	60.9552691432904\\
263.800000000003	60.9700644183403\\
264.000000000003	60.9238924649754\\
264.099999999997	60.9386828160485\\
264.3	60.89258698941\\
264.700000000002	60.9516616314199\\
264.800000000001	60.9286523216308\\
265.4	61.0169491525424\\
265.500000000003	60.9939759036145\\
265.599999999997	61.0086563793752\\
265.7	60.9857035364936\\
266.000000000002	61.0296880871853\\
266.199999999998	60.9838527975967\\
266.299999999997	60.9984984984985\\
266.4	60.9756097560976\\
266.500000000003	60.9902475618905\\
266.599999999996	60.9673790776153\\
266.700000000001	60.9820089955023\\
266.9	60.936329588015\\
267.900000000003	61.0820895522388\\
268.000000000003	61.059306229019\\
268.100000000002	61.0738255033557\\
268.200000000004	61.051062243757\\
268.300000000001	61.0655737704918\\
268.399999999996	61.0428305400372\\
268.599999999999	61.0718273167101\\
268.699999999996	61.0491071428571\\
269.2	61.1214259190494\\
269.399999999998	61.0760667903525\\
269.699999999998	61.119347664937\\
269.800000000004	61.0967024824009\\
270.599999999999	61.2116734392316\\
270.899999999997	61.1439114391144\\
271.099999999998	61.1725663716814\\
271.199999999997	61.1500184297825\\
271.400000000003	61.1786372007367\\
271.699999999998	61.1111111111111\\
271.900000000004	61.139705882353\\
271.999999999997	61.1172363101801\\
272.299999999999	61.1600587371512\\
272.500000000003	61.1151870873074\\
272.599999999999	61.1294462779611\\
272.799999999997	61.0846463906193\\
272.9	61.0989010989011\\
273.099999999997	61.0541727672035\\
273.299999999998	61.0826627651792\\
273.400000000004	61.0603290676417\\
273.599999999998	61.0887833394227\\
273.8	61.0441767068273\\
273.999999999997	61.0726012404232\\
274.100000000004	61.0503282275711\\
274.299999999997	61.0787172011662\\
274.399999999996	61.0564663023679\\
274.600000000002	61.0848198034219\\
274.699999999999	61.0625909752547\\
275.300000000001	61.1474219317357\\
275.500000000001	61.1030478955007\\
275.800000000002	61.1453425154041\\
276	61.1010503440782\\
276.100000000002	61.1151339608979\\
276.199999999997	61.0930148389432\\
276.300000000001	61.1070911722142\\
276.400000000001	61.0849909584087\\
276.900000000004	61.1552346570397\\
277.100000000001	61.1111111111111\\
277.299999999998	61.1391492429704\\
277.5	61.0951008645533\\
278.099999999996	61.1790079079799\\
278.500000000002	61.0911701363963\\
278.6	61.1051309651956\\
278.699999999997	61.0832137733142\\
278.899999999998	61.1111111111111\\
279.000000000004	61.0892153350054\\
279.200000000002	61.1170784103115\\
279.700000000002	61.0078627591136\\
279.799999999999	61.0217934976777\\
279.900000000003	61\\
280.499999999999	61.0833927298646\\
280.700000000002	61.039886039886\\
280.799999999998	61.0537557849769\\
281.000000000003	61.0103166133049\\
281.199999999998	61.0380376821898\\
281.300000000001	61.0163468372424\\
281.600000000003	61.0578629747959\\
281.799999999998	61.0145441645974\\
281.9	61.0283687943262\\
281.999999999996	61.0067352002836\\
282.099999999997	61.020552799433\\
282.400000000001	60.9557522123894\\
282.600000000003	60.9833746020516\\
282.699999999997	60.961810466761\\
282.799999999997	60.9756097560976\\
282.899999999997	60.9540636042403\\
282.999999999996	60.967855881314\\
283.200000000003	60.9248146840805\\
283.400000000001	60.952380952381\\
283.499999999996	60.9308885754584\\
283.800000000004	60.9721733004579\\
283.999999999997	60.9292502639916\\
284.299999999996	60.9704641350211\\
284.400000000003	60.9490333919156\\
284.500000000002	60.9627547434996\\
284.600000000004	60.9413417632596\\
284.700000000002	60.9550561797753\\
284.899999999998	60.9122807017544\\
284.999999999998	60.9259908803928\\
285.499999999997	60.8193277310924\\
285.600000000002	60.8330416520826\\
286	60.7479902132122\\
286.2	60.7754104086623\\
287.000000000001	60.6060606060606\\
287.099999999997	60.6197771587744\\
287.199999999998	60.5986773407588\\
287.300000000002	60.6123869171886\\
287.399999999999	60.5913043478261\\
287.6	60.6187000347584\\
287.700000000002	60.597637248089\\
288.5	60.7068607068607\\
288.700000000004	60.6648199445983\\
288.900000000002	60.6920415224914\\
289.000000000003	60.6710480802491\\
289.200000000003	60.6982371240926\\
289.300000000001	60.6772633033863\\
289.499999999998	60.7044198895028\\
289.600000000003	60.683465654125\\
289.699999999997	60.6970324361629\\
289.799999999996	60.6760952052432\\
289.900000000004	60.6896551724138\\
290.000000000003	60.6687349189935\\
290.200000000001	60.6958318980365\\
290.399999999996	60.6540447504303\\
290.700000000004	60.6946354883081\\
290.799999999999	60.6737710553455\\
291.100000000003	60.7142857142857\\
291.399999999999	60.6518010291595\\
291.500000000002	60.6652949245542\\
291.600000000001	60.6444977716832\\
291.900000000004	60.6849315068493\\
292.000000000004	60.6641561109209\\
292.099999999997	60.6776180698152\\
292.399999999998	60.6153846153846\\
292.499999999998	60.6288448393712\\
292.599999999995	60.6081311923471\\
292.699999999999	60.6215846994536\\
292.800000000003	60.6008876749744\\
293.100000000003	60.6412005457026\\
293.200000000001	60.6205250596659\\
293.699999999997	60.6875425459496\\
293.800000000004	60.6668935011909\\
294	60.6936416184971\\
294.099999999996	60.6730115567641\\
294.4	60.7130730050934\\
294.599999999995	60.671869697998\\
295.100000000004	60.7384823848239\\
295.499999999997	60.6562922868742\\
296.300000000003	60.7624831309042\\
296.399999999995	60.7419898819562\\
296.499999999997	60.7552258934592\\
296.899999999997	60.6734006734007\\
297.000000000003	60.6866374957927\\
297.299999999998	60.6254203093477\\
297.600000000004	60.6650990930467\\
297.699999999999	60.6447280053727\\
297.799999999996	60.657938905673\\
297.900000000002	60.6375838926175\\
298.099999999996	60.6639839034205\\
298.200000000002	60.6436473348978\\
298.500000000005	60.6831882116544\\
298.699999999999	60.6425702811245\\
298.900000000003	60.6688963210702\\
298.999999999998	60.6486125041792\\
299.500000000003	60.7142857142857\\
299.599999999996	60.694027360694\\
300.299999999996	60.7856191744341\\
300.399999999999	60.7653910149751\\
300.899999999999	60.8305647840532\\
300.999999999997	60.8103620059781\\
301.100000000004	60.8233731739708\\
301.199999999999	60.803186193163\\
301.600000000001	60.8551541266158\\
301.700000000004	60.8349900596422\\
301.800000000003	60.8479629016231\\
301.999999999997	60.8076795762992\\
302.099999999997	60.8206485771013\\
302.6	60.7201850016518\\
302.899999999997	60.7590759075908\\
303.200000000001	60.6989779096604\\
303.899999999999	60.7894736842105\\
303.999999999998	60.7694837224597\\
304.100000000005	60.7823800131492\\
304.199999999996	60.7624055208676\\
304.400000000004	60.7881773399015\\
304.499999999999	60.7682206172029\\
305.000000000003	60.8325139298591\\
305.299999999999	60.7727570399476\\
305.600000000002	60.8112528622833\\
305.699999999996	60.7913669064748\\
305.900000000003	60.8169934640523\\
306.100000000001	60.7772697583279\\
306.499999999997	60.8284409654273\\
306.800000000001	60.7689801238188\\
307.099999999996	60.8072916666667\\
307.3	60.7677293428757\\
307.499999999996	60.7932379713914\\
307.599999999999	60.7734806629834\\
307.7	60.7862248213125\\
307.799999999998	60.7664826242286\\
307.900000000004	60.7792207792208\\
308.100000000001	60.7397793640493\\
308.200000000003	60.7525137852741\\
308.299999999997	60.7328145265888\\
308.699999999995	60.7836787564767\\
308.799999999998	60.7640012949174\\
308.999999999996	60.7893885473957\\
309.099999999996	60.7697283311772\\
309.200000000003	60.7824118978338\\
309.600000000002	60.7039070067808\\
309.9	60.741935483871\\
310.099999999996	60.7027724049001\\
310.199999999998	60.7154366741863\\
310.300000000005	60.6958762886598\\
310.5	60.7211848036059\\
310.700000000002	60.6821106821107\\
311.000000000003	60.7200257152041\\
311.200000000004	60.6810150979762\\
311.4	60.7062600321027\\
311.499999999999	60.6867779204108\\
312.200000000002	60.7748959333974\\
312.300000000005	60.7554417413572\\
313	60.8431810923028\\
313.200000000004	60.8043408873284\\
313.300000000001	60.8168474792597\\
313.400000000001	60.7974481658692\\
313.700000000004	60.8349267049076\\
313.800000000003	60.8155463523415\\
314.000000000003	60.8404966571156\\
314.2	60.8017817371938\\
314.300000000003	60.8142493638677\\
314.400000000004	60.7949125596184\\
314.899999999998	60.8571428571429\\
314.999999999995	60.8378292605522\\
315.199999999997	60.8626704725658\\
315.299999999999	60.8433734939759\\
315.400000000004	60.8557844690967\\
315.500000000001	60.8365019011407\\
316.000000000003	60.89844985764\\
316.200000000001	60.8599430920013\\
316.3	60.8723135271808\\
316.499999999999	60.8338597599495\\
316.600000000005	60.8462267129776\\
316.900000000001	60.788643533123\\
317.700000000001	60.8873505349276\\
317.800000000001	60.8681975463982\\
318	60.892801005973\\
318.400000000005	60.8163265306122\\
318.499999999998	60.8286252354049\\
318.600000000003	60.8095387511767\\
318.799999999995	60.8341172781436\\
318.899999999998	60.8150470219436\\
319.200000000003	60.8518634512997\\
319.299999999999	60.832811521603\\
319.600000000002	60.8695652173913\\
319.800000000004	60.8315098468271\\
320	60.855982505467\\
320.299999999995	60.7990012484395\\
321.400000000001	60.9331259720062\\
321.500000000005	60.9141791044776\\
321.900000000004	60.9627329192547\\
321.999999999995	60.9438062713443\\
322.1	60.9559279950341\\
322.2	60.9370152032268\\
322.400000000002	60.9612403100775\\
322.500000000002	60.9423434593924\\
322.899999999998	60.9907120743034\\
323.000000000004	60.9718353450944\\
323.099999999998	60.9839108910891\\
323.200000000002	60.9650479430869\\
323.499999999995	61.0012360939431\\
323.700000000004	60.9635577516986\\
324.199999999998	61.0237434474252\\
324.399999999995	60.9861325115562\\
324.599999999999	61.0101632275947\\
324.699999999998	60.9913793103448\\
324.900000000005	61.0153846153846\\
324.999999999997	60.9966164257152\\
325.700000000004	61.0804174340086\\
325.999999999995	61.0242256976388\\
326.100000000002	61.0361741263029\\
326.200000000005	61.0174685871897\\
326.899999999996	61.1009174311927\\
327.000000000004	61.082237847753\\
327.400000000003	61.1297709923664\\
327.500000000003	61.1111111111111\\
327.799999999997	61.1466910643489\\
328.100000000005	61.0907982937233\\
328.799999999997	61.173608999696\\
329	61.1364326952294\\
329.099999999995	61.1482381530984\\
329.300000000004	61.1111111111111\\
330.099999999999	61.2053301029679\\
330.300000000004	61.1682808716707\\
330.600000000001	61.2035077109162\\
330.699999999998	61.1850060459492\\
330.900000000003	61.2084592145015\\
331.000000000001	61.1899728178798\\
331.099999999999	61.201690821256\\
331.200000000001	61.1832176275279\\
331.300000000003	61.1949305974653\\
331.400000000002	61.1764705882353\\
331.599999999995	61.1998794091046\\
331.699999999995	61.1814345991561\\
331.799999999998	61.1931304609822\\
332.099999999996	61.1378687537628\\
332.299999999995	61.1612515042118\\
332.500000000003	61.1244738424534\\
332.799999999998	61.1595073595674\\
332.899999999996	61.1411411411411\\
333	61.1528069648754\\
333.099999999998	61.1344537815126\\
333.300000000005	61.1577684463107\\
333.599999999999	61.1027869343722\\
333.900000000005	61.1377245508982\\
334.100000000001	61.1011370436864\\
334.900000000004	61.1940298507463\\
335.000000000003	61.1757684273351\\
335.300000000005	61.2104949314252\\
335.700000000004	61.1375818939845\\
335.999999999995	61.1722701576912\\
336.3	61.1177170035672\\
336.799999999999	61.1754229741763\\
336.900000000005	61.1572700296736\\
337.5	61.2263033175356\\
337.599999999999	61.2081729345573\\
337.999999999995	61.254066844129\\
338.1	61.2359550561798\\
338.200000000002	61.2474135382796\\
338.300000000001	61.2293144208038\\
338.499999999995	61.2522150029533\\
338.900000000006	61.1799410029499\\
339.099999999999	61.2028301886792\\
339.2	61.184792219275\\
339.599999999995	61.2304974977922\\
339.800000000001	61.1944689614593\\
339.999999999996	61.2172890326375\\
340.199999999995	61.1813106082868\\
340.300000000003	61.192714453584\\
340.4	61.1747430249633\\
340.499999999996	61.1861421021726\\
340.600000000001	61.1681831523334\\
341.200000000005	61.2364488719601\\
341.499999999998	61.1826697892272\\
341.599999999998	61.1940298507463\\
341.699999999999	61.1761263897016\\
341.999999999997	61.2101724641918\\
342.099999999995	61.1922852133255\\
342.199999999996	61.2036225533158\\
342.3	61.1857476635514\\
342.400000000002	61.1970802919708\\
342.599999999996	61.1613656259119\\
342.700000000005	61.1726954492415\\
342.799999999998	61.1548556430446\\
343.1	61.1888111888112\\
343.200000000001	61.1709874745121\\
343.600000000003	61.216176898458\\
343.7	61.1983711460151\\
343.800000000002	61.2096539691771\\
343.900000000001	61.1918604651163\\
344.000000000002	61.2031386224935\\
344.199999999999	61.167586407203\\
344.400000000002	61.1901306240929\\
344.600000000004	61.1546272120685\\
344.699999999997	61.1658932714617\\
345.100000000005	61.0950173812283\\
345.500000000001	61.1400462962963\\
345.600000000001	61.1223604281169\\
345.700000000001	61.1336032388664\\
345.8	61.1159294593813\\
346	61.1383993065588\\
346.100000000003	61.1207394569613\\
346.2	61.1319665030321\\
346.299999999995	61.1143187066975\\
346.700000000001	61.159169550173\\
346.9	61.1239193083573\\
347.100000000002	61.1463133640553\\
347.299999999997	61.1111111111111\\
347.499999999998	61.1334867663982\\
347.7	61.0983323749281\\
347.899999999997	61.1206896551724\\
348.000000000004	61.1031312841138\\
348.099999999995	61.1143021252154\\
348.300000000002	61.0792192881745\\
348.400000000001	61.090387374462\\
348.500000000003	61.0728628800918\\
348.700000000003	61.0951834862385\\
348.800000000005	61.0776726855833\\
349.199999999998	61.122244488978\\
349.599999999997	61.0523305690592\\
349.800000000003	61.0745927407831\\
350.000000000006	61.0397029420166\\
350.100000000005	61.0508280982296\\
350.500000000002	60.9811751283514\\
350.700000000002	61.0034207525656\\
350.899999999999	60.968660968661\\
351.099999999996	60.9908883826879\\
351.199999999995	60.9735269000854\\
351.499999999997	61.0068259385666\\
351.599999999999	60.9894796701734\\
352.4	61.0780141843972\\
352.499999999995	61.0606920022689\\
352.599999999998	61.0717323504395\\
352.800000000001	61.0371209974497\\
352.899999999999	61.0481586402266\\
353.000000000001	61.0308694420844\\
353.400000000002	61.0749646393211\\
353.499999999996	61.0576923076923\\
353.799999999995	61.0907035885843\\
353.900000000005	61.0734463276836\\
354.199999999998	61.1064069997178\\
354.300000000002	61.0891647855531\\
354.599999999997	61.1220749929518\\
355.2	61.018857303687\\
355.399999999998	61.0407876230661\\
355.600000000002	61.0064661231375\\
355.699999999995	61.017425519955\\
355.799999999997	61.0002809778028\\
355.900000000002	61.0112359550562\\
356.100000000003	60.9769792251544\\
356.300000000001	60.9988776655443\\
356.7	60.9304932735426\\
356.799999999997	60.9414401793219\\
356.899999999997	60.9243697478992\\
357.099999999999	60.946248600224\\
357.200000000003	60.9291911558914\\
357.400000000004	60.9510489510489\\
357.5	60.9340044742729\\
357.700000000005	60.9558412520961\\
357.799999999999	60.9388097233864\\
357.900000000001	60.9497206703911\\
357.999999999999	60.9327003630271\\
358.1	60.9436069235064\\
358.200000000006	60.9265978230533\\
358.300000000002	60.9375\\
358.399999999995	60.9205020920502\\
358.500000000005	60.9313998884551\\
358.599999999998	60.9144131586284\\
358.899999999997	60.9470752089136\\
358.999999999998	60.9301030353662\\
359.400000000004	60.9735744089013\\
359.599999999997	60.9396719488463\\
359.699999999996	60.9505280711506\\
359.799999999998	60.9335926646291\\
359.899999999995	60.9444444444444\\
360	60.9275201332963\\
360.200000000002	60.9492089925062\\
360.299999999999	60.9322974472808\\
360.399999999995	60.9431345353675\\
360.500000000001	60.9262340543539\\
360.600000000001	60.9370668145273\\
360.699999999997	60.920177383592\\
361.000000000006	60.9526446967599\\
361.100000000004	60.9357696566999\\
361.299999999999	60.9573879358052\\
361.399999999996	60.9405255878285\\
361.500000000004	60.9513274336283\\
361.599999999996	60.9344760851534\\
361.700000000005	60.9452736318408\\
362.000000000004	60.8947804473902\\
362.200000000006	60.9163676511179\\
362.299999999998	60.8995584988963\\
363.199999999996	60.9964216900633\\
363.300000000006	60.9796367638965\\
363.499999999999	61.001100110011\\
363.700000000002	60.9675645959318\\
364.599999999997	61.0638881272279\\
364.799999999997	61.030419292957\\
365.000000000005	61.0517666392769\\
365.099999999994	61.0350492880613\\
365.199999999999	61.0457158499863\\
365.300000000002	61.0290093048714\\
365.599999999995	61.06097894449\\
366.000000000003	60.9942638623327\\
366.099999999997	61.004915346805\\
366.199999999998	60.988260988261\\
366.300000000004	60.9989082969432\\
366.500000000002	60.9656301145663\\
366.900000000001	61.008174386921\\
367.099999999994	60.9749455337691\\
367.299999999998	60.9961894393032\\
367.700000000005	60.9298531810767\\
367.799999999997	60.9404729546072\\
367.900000000004	60.9239130434783\\
368.1	60.9451385116784\\
368.400000000001	60.8955223880597\\
368.499999999995	60.9061313076506\\
368.8	60.8566007047981\\
369.300000000002	60.9095831077423\\
369.400000000004	60.893098782138\\
369.599999999998	60.9142548011902\\
369.7	60.8977825851812\\
370	60.9294785193191\\
370.299999999994	60.8801295896328\\
370.500000000006	60.9012412304371\\
370.6	60.88481251686\\
371.099999999997	60.9375\\
371.200000000004	60.9210880689469\\
372.200000000004	61.0260542573194\\
372.300000000001	61.0096670247046\\
372.500000000005	61.0305958132045\\
372.600000000001	61.0142205527234\\
372.799999999999	61.0351300616787\\
372.899999999999	61.0187667560322\\
373.099999999999	61.0396570203644\\
373.300000000002	61.0069630423139\\
373.400000000002	61.0174029451138\\
373.600000000002	60.984747123361\\
373.899999999997	61.0160427807487\\
373.999999999999	60.9997326917936\\
374.099999999994	61.0101549973276\\
374.300000000006	60.9775641025641\\
374.599999999997	61.0088070456365\\
374.7	60.9925293489861\\
375.199999999998	61.0444977351452\\
375.299999999999	61.0282365476825\\
375.499999999998	61.04898828541\\
375.6	61.0327388874102\\
375.799999999999	61.0534716679968\\
375.900000000006	61.0372340425532\\
375.999999999994	61.0475937250731\\
376.200000000001	61.0151474887058\\
376.700000000004	61.0668789808917\\
376.800000000004	61.050676572035\\
376.900000000004	61.0610079575597\\
377.000000000001	61.0448156987537\\
377.399999999996	61.0860927152318\\
377.499999999996	61.0699152542373\\
377.800000000003	61.1008203228367\\
377.899999999998	61.0846560846561\\
378.299999999994	61.1257928118393\\
378.400000000006	61.10964332893\\
378.600000000001	61.1301822022709\\
378.800000000001	61.0979150171549\\
378.9	61.1081794195251\\
379.200000000004	61.0598470867387\\
379.299999999996	61.070110701107\\
379.399999999998	61.0540184453228\\
379.999999999997	61.1154959221258\\
380.099999999998	61.0994213571804\\
380.200000000006	61.1096502760978\\
380.4	61.0775295663601\\
380.500000000005	61.0877561744614\\
380.600000000003	61.0717100078802\\
380.699999999995	61.0819327731093\\
380.899999999996	61.0498687664042\\
380.999999999996	61.060089215429\\
381.099999999994	61.0440713536202\\
381.399999999998	61.0747051114024\\
381.499999999999	61.0587002096436\\
381.699999999997	61.0790990047145\\
381.799999999995	61.0631055250065\\
382.000000000002	61.0834859984297\\
382.100000000001	61.0675039246468\\
382.499999999996	61.1082070047047\\
382.699999999995	61.0762800417973\\
382.799999999995	61.0864455471402\\
383.100000000002	61.0386221294363\\
383.199999999999	61.0487868510305\\
383.399999999997	61.0169491525424\\
384.200000000004	61.0981004423627\\
384.299999999997	61.0822060353798\\
384.399999999998	61.0923276983095\\
384.499999999998	61.0764430577223\\
385.100000000006	61.1370716510904\\
385.400000000002	61.0894941634241\\
385.600000000004	61.1096707285455\\
385.699999999995	61.0938310005184\\
385.800000000006	61.1039129308111\\
385.999999999998	61.0722610722611\\
386.100000000001	61.0823407560849\\
386.3	61.0507246376812\\
386.399999999994	61.0608020698577\\
386.500000000004	61.0450077599586\\
386.600000000004	61.055081458495\\
386.799999999998	61.0235202894805\\
386.900000000005	61.0335917312661\\
387.000000000002	61.0178248514596\\
387.100000000004	61.0278925619835\\
387.299999999996	60.9963861641714\\
387.400000000003	61.0064516129032\\
387.600000000005	60.9749806551457\\
387.999999999997	61.0152022674568\\
388.100000000004	60.9994848016486\\
388.199999999995	61.0095287149112\\
388.300000000004	60.9938208032956\\
388.400000000002	61.003861003861\\
388.500000000007	60.9881626351004\\
388.599999999995	60.9981991252894\\
388.699999999994	60.9825102880658\\
389.000000000005	61.0125931637111\\
389.099999999999	60.9969167523124\\
389.399999999995	61.0269576379974\\
389.500000000005	61.0112936344969\\
389.699999999994	61.0312981015906\\
389.899999999996	61\\
390.299999999996	61.0399590163934\\
390.400000000004	61.0243277848912\\
390.599999999999	61.0442794983363\\
390.700000000002	61.028659160696\\
390.999999999996	61.0585527997955\\
391.100000000002	61.0429447852761\\
391.400000000004	61.0727969348659\\
391.599999999996	61.0416134797039\\
391.700000000001	61.0515569167943\\
391.8	61.0359785659607\\
392	61.0558530986993\\
392.199999999999	61.0247259750191\\
392.700000000006	61.0743380855397\\
392.799999999994	61.0587935861542\\
393.099999999995	61.088504577823\\
393.2	61.0729722857869\\
393.700000000001	61.1223971559167\\
393.800000000003	61.1068799187611\\
393.899999999999	61.1167512690355\\
393.999999999996	61.101243339254\\
394.499999999998	61.1505321844906\\
394.599999999999	61.1350392703319\\
394.700000000002	61.144883485309\\
394.800000000001	61.1293998480628\\
395.099999999994	61.1589068825911\\
395.199999999994	61.1434353655452\\
395.299999999994	61.1532625189681\\
395.600000000002	61.1068991660349\\
395.900000000002	61.1363636363636\\
396.000000000004	61.1209290583186\\
396.200000000001	61.1405500883169\\
396.399999999997	61.109709962169\\
396.5	61.1195158850227\\
396.699999999995	61.0887096774194\\
396.900000000001	61.1083123425693\\
396.999999999994	61.0929236968018\\
397.3	61.1222949169602\\
397.499999999999	61.0915492957747\\
397.600000000003	61.1013326628112\\
397.700000000005	61.0859728506787\\
398.000000000006	61.1152976639035\\
398.199999999995	61.0846095907607\\
398.800000000006	61.1431436450238\\
399.000000000003	61.1125031320471\\
399.399999999997	61.1514392991239\\
399.700000000002	61.1055527763882\\
399.799999999999	61.1152788197049\\
399.899999999995	61.1\\
400.099999999998	61.1194402798601\\
400.200000000005	61.1041718710967\\
400.699999999996	61.1526946107785\\
400.799999999995	61.1374407582938\\
400.999999999997	61.1568187484418\\
401.099999999996	61.1415752741775\\
401.499999999996	61.1802788844622\\
401.599999999999	61.1650485436893\\
401.799999999995	61.1843742224434\\
401.900000000003	61.1691542288557\\
402.300000000006	61.2077534791253\\
402.400000000003	61.1925465838509\\
402.599999999994	61.2118202135585\\
402.899999999999	61.166253101737\\
403.000000000002	61.1758868767055\\
403.300000000005	61.1303916707982\\
403.399999999995	61.1400247831475\\
403.499999999996	61.1248761149653\\
403.699999999995	61.1441307578009\\
403.799999999999	61.1289923248329\\
403.900000000006	61.1386138613861\\
404.000000000002	61.1234842860678\\
404.099999999997	61.1331024245423\\
404.199999999996	61.1179816967598\\
404.300000000001	61.1275964391691\\
404.399999999994	61.1124845488257\\
404.499999999997	61.1220958971824\\
404.599999999997	61.1069928341982\\
404.800000000004	61.1262040009879\\
404.899999999994	61.1111111111111\\
405.599999999994	61.1782105003697\\
405.899999999999	61.1330049261084\\
405.999999999998	61.142575720266\\
406.099999999996	61.1275233874939\\
406.799999999996	61.1943966576554\\
407.100000000004	61.1493123772102\\
407.699999999997	61.2064737616479\\
407.799999999994	61.1914684971807\\
408.499999999999	61.2579539892315\\
408.599999999995	61.242965500367\\
409.199999999996	61.299780112387\\
409.299999999998	61.2848070346849\\
409.600000000001	61.3131559677813\\
409.699999999999	61.2981942410932\\
409.799999999998	61.3076360087826\\
409.999999999995	61.2777371372836\\
410.100000000006	61.2871769868357\\
410.299999999995	61.2573099415205\\
410.500000000006	61.2761811982465\\
410.599999999995	61.2612612612613\\
410.7	61.2706913339825\\
410.800000000001	61.2557799951326\\
410.899999999996	61.2652068126521\\
410.999999999996	61.250304062272\\
411.400000000002	61.2879708383961\\
411.5	61.2730806608358\\
411.599999999994	61.2824872479961\\
411.699999999999	61.2676056338028\\
412.399999999999	61.3333333333333\\
412.500000000004	61.3184682501212\\
412.600000000006	61.3278410467652\\
412.700000000001	61.312984496124\\
412.999999999997	61.3410796417332\\
413.099999999996	61.3262342691191\\
413.300000000001	61.3449443638123\\
413.399999999996	61.3301088270859\\
413.499999999994	61.3394584139265\\
413.599999999998	61.3246313753928\\
413.700000000003	61.3339777670372\\
413.800000000004	61.3191592172022\\
414.299999999995	61.3658301158301\\
414.600000000002	61.3214371835061\\
414.700000000005	61.3307618129219\\
414.9	61.3012048192771\\
414.999999999995	61.3105275837148\\
415.100000000003	61.2957610789981\\
415.299999999996	61.3143957631199\\
415.399999999996	61.2996389891697\\
415.999999999997	61.3554434030281\\
416.400000000007	61.296518607443\\
416.799999999996	61.3336531542336\\
416.999999999998	61.3042435866699\\
417.299999999999	61.3320555821754\\
417.600000000005	61.2880057457505\\
417.9	61.3157894736842\\
418.099999999995	61.2864658058345\\
418.299999999999	61.3049713193117\\
418.400000000004	61.2903225806452\\
418.500000000006	61.2995699952222\\
418.800000000003	61.2556696108857\\
419.099999999999	61.2833969465649\\
419.299999999998	61.2541726275632\\
419.699999999999	61.2910909957122\\
419.800000000005	61.2764944034294\\
419.900000000002	61.2857142857143\\
420.300000000003	61.2274024738344\\
420.999999999999	61.29185466635\\
421.100000000005	61.2773029439696\\
421.500000000005	61.3140417457305\\
421.600000000001	61.2995020156509\\
421.799999999997	61.3178478312396\\
421.999999999999	61.288794124615\\
422.099999999997	61.2979630506869\\
422.300000000002	61.2689393939394\\
422.699999999998	61.3055818353832\\
422.800000000003	61.29108536297\\
422.900000000004	61.3002364066194\\
422.999999999994	61.2857480501064\\
423.100000000003	61.2948960302458\\
423.200000000001	61.2804157807701\\
423.300000000004	61.2895606991025\\
423.399999999997	61.2750885478158\\
423.500000000001	61.2842304060434\\
423.700000000006	61.2553091080699\\
423.800000000004	61.2644491625383\\
423.999999999999	61.2355576514973\\
424.099999999999	61.2446958981613\\
424.199999999996	61.2302616073533\\
424.499999999994	61.2576542628356\\
424.700000000001	61.228813559322\\
424.900000000003	61.2470588235294\\
424.999999999995	61.232651140908\\
425.100000000002	61.241768579492\\
425.300000000005	61.212976022567\\
425.5	61.2312030075188\\
425.600000000004	61.2168193563542\\
426.500000000006	61.2986404125645\\
426.699999999993	61.269915651359\\
427.099999999996	61.3061797752809\\
427.299999999996	61.2774918109499\\
427.399999999998	61.2865497076023\\
427.499999999997	61.2722170252572\\
427.699999999997	61.2903225806452\\
427.800000000006	61.2759990652022\\
428.000000000002	61.2940901658491\\
428.199999999997	61.2654681298155\\
428.599999999995	61.3016095171449\\
428.700000000002	61.2873134328358\\
428.800000000004	61.2963394730706\\
428.899999999994	61.2820512820513\\
429.100000000004	61.3000931966449\\
429.400000000005	61.2572759022119\\
429.6	61.2753083546661\\
429.700000000004	61.2610516519311\\
429.799999999994	61.2700628053036\\
429.899999999996	61.2558139534884\\
430.299999999999	61.2918215613383\\
430.500000000002	61.263353460288\\
430.999999999998	61.3082811412665\\
431.099999999999	61.2940630797774\\
431.200000000004	61.3030373290053\\
431.300000000005	61.2888270746407\\
431.800000000004	61.3336420467701\\
431.900000000001	61.3194444444445\\
432.000000000001	61.3283962045823\\
432.099999999997	61.3142063859324\\
432.699999999995	61.3678373382625\\
433.000000000007	61.3253290233203\\
433.4	61.361014994233\\
433.500000000004	61.3468634686347\\
433.800000000005	61.3735883844204\\
433.899999999996	61.3594470046083\\
433.999999999999	61.3683483068418\\
434.099999999999	61.3542146476278\\
434.200000000006	61.3631130554916\\
434.400000000005	61.3348676639816\\
435.000000000005	61.3881866237647\\
435.099999999998	61.3740808823529\\
435.199999999995	61.3829542844016\\
435.399999999997	61.3547646383467\\
435.699999999995	61.3813675998164\\
435.800000000001	61.3672860747878\\
435.900000000001	61.3761467889908\\
436.199999999999	61.3339445335778\\
436.499999999997	61.3605130554283\\
436.600000000004	61.3464621021296\\
436.800000000004	61.3641565575647\\
436.9	61.350114416476\\
437.300000000007	61.3854595336077\\
437.500000000007	61.3574040219379\\
438.100000000005	61.4103149246919\\
438.200000000007	61.3963039014374\\
438.300000000005	61.4051094890511\\
438.499999999999	61.3771089831281\\
438.699999999996	61.3947128532361\\
438.799999999994	61.3807245386193\\
438.899999999993	61.3895216400911\\
438.999999999993	61.3755408790708\\
439.800000000002	61.4457831325301\\
440.099999999994	61.4039073148569\\
440.899999999997	61.4739229024943\\
441.000000000001	61.4599863976423\\
441.299999999999	61.4861803352968\\
441.5	61.4583333333333\\
441.699999999994	61.4757808963332\\
441.800000000001	61.4618692011767\\
442.099999999994	61.4880144730891\\
442.300000000003	61.4602169981917\\
442.399999999997	61.4689265536723\\
442.500000000004	61.455038409399\\
442.799999999995	61.4811469857756\\
442.899999999996	61.4672686230248\\
443.000000000001	61.4759647935003\\
443.100000000006	61.4620938628159\\
443.5	61.4968440036069\\
443.7	61.4691302388463\\
443.800000000007	61.4778103176391\\
444	61.4501238459806\\
444.200000000003	61.4674769300022\\
444.500000000002	61.4260008996851\\
444.599999999993	61.4346750618394\\
444.699999999998	61.4208633093525\\
444.899999999998	61.438202247191\\
445.299999999996	61.38302649304\\
445.500000000005	61.4003590664273\\
445.699999999995	61.3728129205922\\
445.800000000001	61.38147566719\\
445.900000000004	61.3677130044843\\
446.100000000002	61.3850291349171\\
446.500000000004	61.3300492610837\\
447.000000000007	61.3732945649743\\
447.100000000007	61.3595706618962\\
447.300000000003	61.3768439874832\\
447.400000000002	61.3631284916201\\
447.499999999999	61.3717605004468\\
447.699999999993	61.3443501563198\\
448.600000000007	61.4218854468464\\
448.699999999997	61.4081996434938\\
448.999999999996	61.4339790692496\\
449.100000000001	61.420302760463\\
449.300000000005	61.4374721851357\\
449.399999999995	61.4238042269188\\
449.499999999996	61.432384341637\\
449.699999999997	61.4050689195198\\
449.900000000004	61.4222222222222\\
450	61.4085758720284\\
450.199999999995	61.4257161892072\\
450.3	61.4120781527531\\
450.7	61.4463176574978\\
450.899999999995	61.4190687361419\\
450.999999999998	61.4276213699845\\
451.100000000003	61.4140070921986\\
451.400000000004	61.4396456256922\\
451.600000000004	61.4124418862077\\
451.700000000002	61.4209827357238\\
451.799999999999	61.4073910157114\\
451.899999999994	61.4159292035398\\
451.999999999995	61.4023446140234\\
452.1	61.4108801415303\\
452.300000000003	61.3837312113174\\
452.599999999998	61.4093218466976\\
452.700000000004	61.3957597173145\\
453.099999999996	61.4298323036187\\
453.199999999997	61.4162806088683\\
453.500000000004	61.441798941799\\
453.7	61.4147201410313\\
453.999999999994	61.440211407179\\
454.100000000003	61.4266842800529\\
454.600000000003	61.469100505828\\
454.800000000002	61.4420751813585\\
454.899999999996	61.4505494505495\\
455.000000000004	61.4370468029005\\
455.099999999995	61.4455184534271\\
455.200000000003	61.4320228420821\\
455.400000000005	61.4489571899012\\
455.599999999999	61.4219881500988\\
456.400000000001	61.4895947426068\\
456.499999999998	61.4761279018835\\
456.700000000004	61.4929947460596\\
457.100000000007	61.4391951006124\\
457.500000000004	61.4729020979021\\
457.600000000001	61.4594712693904\\
457.700000000007	61.4678899082569\\
457.799999999999	61.4544660406202\\
457.900000000001	61.4628820960699\\
458.000000000001	61.4494651822746\\
458.499999999997	61.491495856956\\
458.600000000002	61.4780902550687\\
458.700000000002	61.4864864864865\\
458.899999999997	61.4596949891068\\
458.999999999995	61.4680897407972\\
459.200000000002	61.441323753538\\
459.699999999993	61.4832535885168\\
459.800000000004	61.469884757556\\
459.9	61.4782608695652\\
460.100000000004	61.451542807475\\
460.3	61.4682884448306\\
460.400000000002	61.4549402823018\\
460.599999999993	61.4716735402648\\
461	61.4183474300586\\
461.200000000002	61.4350747886408\\
461.399999999998	61.4084507042253\\
461.6	61.4251678579164\\
461.899999999998	61.3852813852814\\
462.400000000004	61.427027027027\\
462.5	61.4137483787289\\
462.999999999994	61.4554091988771\\
463.100000000003	61.4421416234888\\
463.6	61.4837179210697\\
464.099999999999	61.4174924601465\\
464.700000000001	61.4672977624785\\
464.800000000002	61.4540761454076\\
465.100000000001	61.4789337919175\\
465.199999999994	61.4657210401891\\
465.299999999997	61.4740008594757\\
465.399999999996	61.4607948442535\\
465.499999999996	61.4690721649485\\
465.800000000005	61.4294913071475\\
465.999999999999	61.4460416219695\\
466.099999999993	61.4328614328614\\
466.199999999993	61.4411323182501\\
466.400000000004	61.4147909967846\\
466.699999999993	61.439588688946\\
467.000000000003	61.4001284521516\\
467.200000000007	61.4166488337257\\
467.499999999999	61.377245508982\\
467.9	61.4102564102564\\
467.999999999997	61.3971373638112\\
468.399999999999	61.4300960512273\\
468.500000000001	61.4169867690995\\
468.6	61.4252186899936\\
468.999999999997	61.3728416115967\\
469.099999999993	61.381074168798\\
469.199999999995	61.3679948860004\\
469.5	61.3926746166951\\
469.600000000004	61.3796040025548\\
469.899999999995	61.4042553191489\\
469.999999999993	61.3911933631142\\
470.100000000002	61.3994045087197\\
470.2	61.3863491388475\\
470.400000000001	61.4027630180659\\
470.500000000002	61.3897152571186\\
471.499999999994	61.4715860899067\\
471.700000000005	61.4455277660025\\
472.000000000005	61.4700275365389\\
472.099999999993	61.4570097416349\\
472.899999999994	61.5221987315011\\
473.100000000006	61.4961961115807\\
473.200000000001	61.504331290936\\
473.300000000007	61.4913392479932\\
473.700000000004	61.5238497256226\\
473.799999999993	61.5108672715763\\
474.300000000001	61.5514333895447\\
474.500000000003	61.5254951538137\\
474.600000000003	61.5336001685275\\
474.699999999995	61.5206402695872\\
474.800000000003	61.5287428932407\\
474.900000000008	61.5157894736842\\
474.999999999996	61.52388970743\\
475.400000000006	61.472134595163\\
475.5	61.4802354920101\\
475.6	61.4673113306706\\
476.000000000002	61.4996849401386\\
476.100000000001	61.4867702645947\\
476.399999999994	61.511017838405\\
476.599999999995	61.4852108244179\\
477.7	61.5738802846379\\
478.000000000007	61.5352436728718\\
478.499999999994	61.5754283326369\\
478.700000000006	61.5497076023392\\
478.999999999999	61.5737841786683\\
479.099999999995	61.5609348914858\\
479.400000000003	61.584984358707\\
479.500000000003	61.5721434528774\\
479.599999999995	61.5801542630811\\
479.700000000006	61.5673197165486\\
479.800000000008	61.5753281933736\\
480	61.5496771505936\\
480.200000000001	61.5656881115969\\
480.299999999997	61.5528726061615\\
480.499999999997	61.5688722430295\\
480.600000000007	61.5560640732266\\
481.799999999996	61.6517949782113\\
482.000000000005	61.6262186268409\\
482.300000000003	61.6500829187396\\
482.400000000005	61.6373056994819\\
482.600000000006	61.6532007458049\\
482.700000000002	61.6404308202154\\
482.800000000005	61.6483744046386\\
482.899999999997	61.6356107660455\\
483.199999999996	61.6594247879164\\
483.499999999997	61.6211745244003\\
483.600000000003	61.6291089518297\\
483.699999999994	61.6163704009922\\
483.799999999999	61.6243025418475\\
483.900000000001	61.6115702479339\\
484.100000000001	61.6274266831888\\
484.199999999996	61.6147016312203\\
484.600000000001	61.6463792036311\\
484.800000000003	61.6209527737678\\
485.700000000008	61.6920543433512\\
485.899999999997	61.6666666666667\\
486.000000000001	61.6745525612014\\
486.100000000002	61.6618675442205\\
486.199999999995	61.6697511823977\\
486.299999999996	61.6570723684211\\
486.399999999996	61.6649537512847\\
486.599999999993	61.6396137250873\\
486.899999999999	61.6632443531828\\
487.100000000005	61.6379310344828\\
487.199999999998	61.6458034065258\\
487.300000000005	61.6331555190808\\
487.400000000005	61.6410256410256\\
487.499999999996	61.6283839212469\\
487.599999999999	61.6362517941357\\
487.800000000006	61.6109858577577\\
488.000000000004	61.6267158369187\\
488.099999999997	61.6140925850061\\
488.199999999993	61.6219537169773\\
488.300000000001	61.6093366093366\\
488.500000000005	61.6250511665985\\
488.700000000006	61.5998363338789\\
488.800000000005	61.6076907343015\\
488.900000000002	61.5950920245399\\
489.099999999996	61.6107931316435\\
489.200000000002	61.5982015123646\\
489.499999999995	61.6217320261438\\
489.599999999995	61.6091484582397\\
489.799999999995	61.6248213921209\\
489.899999999996	61.6122448979592\\
490.099999999998	61.6279069767442\\
490.199999999999	61.6153375484397\\
490.500000000007	61.6388096208724\\
490.599999999999	61.6262482168331\\
490.699999999994	61.6340668296659\\
490.8	61.6215115094724\\
490.899999999999	61.6293279022403\\
491.000000000008	61.6167786601507\\
491.100000000005	61.6245928338762\\
491.200000000001	61.6120496641563\\
491.5	61.6354759967453\\
491.799999999998	61.597885749136\\
491.999999999994	61.6134931924406\\
492.100000000004	61.6009752133279\\
492.499999999995	61.63215590743\\
492.600000000007	61.6196468439213\\
492.699999999998	61.6274350649351\\
493	61.5899411883999\\
493.199999999995	61.6055138860734\\
493.300000000004	61.5930279691934\\
493.699999999999	61.624139327663\\
493.8	61.6116622798137\\
494.100000000005	61.6349656009713\\
494.199999999993	61.6224964596399\\
494.399999999993	61.6380182002022\\
494.599999999998	61.6130988477865\\
494.799999999996	61.6286118407759\\
494.900000000003	61.6161616161616\\
495.000000000005	61.6239143607352\\
495.300000000007	61.5865966895438\\
495.499999999996	61.6020984665053\\
495.599999999997	61.5896711720799\\
495.800000000005	61.6051623311152\\
496.000000000007	61.5803265470671\\
496.300000000005	61.6035455278002\\
496.400000000001	61.5911379657603\\
497.000000000003	61.6374974854154\\
497.100000000002	61.6251005631537\\
497.300000000002	61.6405307599518\\
497.399999999993	61.6281407035176\\
497.500000000008	61.6358520900322\\
497.600000000005	61.6234679525819\\
498.3	61.677367576244\\
498.399999999999	61.6649949849549\\
498.6	61.6803689592942\\
498.800000000008	61.6556424133093\\
499.6	61.7170302181309\\
499.699999999993	61.7046818727491\\
500.600000000005	61.7735170760935\\
500.699999999997	61.7611821086262\\
500.8	61.7688161309643\\
500.900000000002	61.7564870259481\\
501.100000000006	61.7717478052674\\
501.200000000008	61.7594254937163\\
501.300000000004	61.7670522536897\\
501.400000000007	61.7547357926221\\
501.499999999999	61.762360446571\\
501.600000000001	61.750049830576\\
501.700000000005	61.757672379434\\
501.799999999993	61.7453676031082\\
501.9	61.7529880478088\\
502.400000000006	61.6915422885572\\
502.500000000005	61.6991643454039\\
502.699999999997	61.6746221161496\\
502.799999999994	61.6822429906542\\
503.000000000004	61.6577221228384\\
503.399999999999	61.6881827209533\\
503.599999999996	61.6636887035934\\
503.700000000008	61.6712981341802\\
503.900000000006	61.6468253968254\\
504.100000000007	61.6620388734629\\
504.199999999994	61.6498116200674\\
504.300000000004	61.6574147501983\\
504.399999999992	61.6451932606541\\
504.500000000001	61.6527942925089\\
504.599999999995	61.6405785615217\\
504.899999999999	61.6633663366337\\
505.1	61.6389548693587\\
505.499999999993	61.6693037974684\\
505.600000000005	61.6571089578802\\
505.800000000007	61.6722672464914\\
505.900000000008	61.6600790513834\\
506.000000000005	61.6676546137127\\
506.099999999997	61.6554721453971\\
506.799999999997	61.7084237522194\\
506.999999999992	61.6840859790968\\
507.499999999997	61.7218282111899\\
507.599999999993	61.7096710655899\\
507.800000000008	61.724748966332\\
507.900000000009	61.7125984251969\\
508.099999999992	61.7276662731208\\
508.200000000001	61.7155223293331\\
508.300000000007	61.7230527143981\\
508.499999999994	61.6987809673614\\
508.600000000005	61.7063102024769\\
508.700000000004	61.6941823899371\\
508.900000000007	61.7092337917485\\
508.999999999993	61.6971125515616\\
509.399999999993	61.7271835132483\\
509.499999999994	61.7150706436421\\
509.599999999996	61.722581910928\\
509.700000000001	61.7104746959592\\
510.299999999993	61.7554858934169\\
510.399999999994	61.743388834476\\
510.499999999994	61.7508813160987\\
510.6	61.7387898962209\\
511.100000000008	61.7762128325509\\
511.300000000001	61.7520531873289\\
511.400000000002	61.7595307917889\\
511.499999999995	61.7474589523065\\
511.900000000005	61.77734375\\
512.000000000001	61.7652802187073\\
512.299999999999	61.7876658860265\\
512.399999999996	61.7756097560976\\
512.699999999994	61.7979719188768\\
512.899999999995	61.7738791423002\\
513.400000000006	61.811100292113\\
513.700000000007	61.775009731413\\
513.799999999998	61.7824479470714\\
513.899999999994	61.7704280155642\\
514.099999999995	61.7852975495916\\
514.199999999997	61.7732840754423\\
514.499999999992	61.7955693742713\\
514.999999999992	61.7355853232382\\
515.400000000005	61.7652764306498\\
515.499999999999	61.7532971295578\\
515.600000000004	61.7607135931743\\
515.700000000003	61.7487398216363\\
516.100000000006	61.778380472685\\
516.5	61.7305458768873\\
516.699999999995	61.7453560371517\\
516.800000000007	61.7334107177404\\
516.900000000006	61.7408123791103\\
516.999999999997	61.7288725584993\\
517.400000000004	61.7584541062802\\
517.5	61.7465224111283\\
517.599999999993	61.7539115317752\\
517.799999999995	61.7300637188647\\
518.199999999999	61.7595986880185\\
518.300000000002	61.7476851851852\\
519.199999999993	61.8139803581745\\
519.3	61.802079322295\\
519.999999999996	61.8534897135166\\
520.099999999992	61.841599384852\\
520.700000000005	61.8855606758833\\
520.799999999999	61.8736801689384\\
520.999999999994	61.88831318365\\
521.099999999998	61.8764389869532\\
521.200000000001	61.8837521580664\\
521.300000000009	61.8718833908707\\
521.399999999993	61.8791946308725\\
521.499999999999	61.8673312883436\\
521.699999999992	61.8819471061709\\
521.800000000006	61.8700900555662\\
521.899999999994	61.8773946360153\\
522.099999999994	61.8536959019533\\
522.800000000002	61.9047619047619\\
522.900000000002	61.8929254302103\\
523.799999999997	61.9583890055354\\
524.000000000006	61.9347452776188\\
524.100000000002	61.9420068676078\\
524.200000000009	61.9301926378028\\
524.600000000007	61.9592147894035\\
524.799999999994	61.9356067822442\\
525.799999999995	62.0079863091843\\
525.9	61.9961977186312\\
525.999999999997	62.0034214027751\\
526.100000000002	61.9916381603953\\
526.199999999999	61.998859965799\\
526.400000000001	61.9753086419753\\
527.000000000001	62.0185922974768\\
527.499999999998	61.9598180439727\\
527.799999999993	61.9814358780072\\
528.100000000006	61.9462324876941\\
528.199999999993	61.9534355479841\\
528.300000000004	61.9417108251325\\
528.599999999999	61.9633062228107\\
528.699999999999	61.9515885022693\\
528.800000000006	61.9587823785215\\
528.899999999996	61.9470699432892\\
529	61.954261954262\\
529.100000000001	61.9425547996977\\
529.200000000001	61.9497449461553\\
529.299999999992	61.9380430676237\\
529.400000000004	61.9452313503305\\
529.600000000003	61.9218425523881\\
529.699999999999	61.9290298225746\\
529.900000000006	61.9056603773585\\
530.800000000001	61.9702392164249\\
531.000000000005	61.9469026548673\\
531.499999999997	61.9826937547028\\
531.800000000003	61.947734536567\\
532.000000000004	61.9620372110506\\
532.200000000001	61.9387563404095\\
532.399999999997	61.9530516431925\\
532.499999999995	61.9414194517462\\
532.800000000001	61.9628448114093\\
532.899999999998	61.9512195121951\\
533.199999999993	61.9726232889556\\
533.500000000001	61.9377811094453\\
533.799999999999	61.9591683835924\\
534.000000000007	61.9359670473694\\
534.100000000007	61.9430924747286\\
534.300000000008	61.9199101796407\\
534.499999999995	61.9341563786008\\
534.799999999994	61.899420452421\\
535.100000000008	61.9207772795217\\
535.199999999995	61.9092097889034\\
535.600000000006	61.9376516707112\\
535.699999999993	61.9260918253079\\
536.300000000007	61.9686800894855\\
536.500000000006	61.9455833022736\\
536.600000000001	61.9526737469723\\
536.699999999999	61.941132637854\\
537.099999999999	61.9694713328369\\
537.199999999999	61.9579378373348\\
537.500000000003	61.9791666666667\\
537.599999999992	61.9676399479264\\
537.899999999997	61.9888475836431\\
538.100000000003	61.965811965812\\
538.600000000007	62.0011137924633\\
538.700000000007	61.9896065330364\\
539.399999999996	62.0389249304912\\
539.600000000008	62.0159347785807\\
539.900000000005	62.037037037037\\
540.000000000001	62.0255508239215\\
540.300000000003	62.0466321243523\\
540.400000000006	62.0351526364477\\
540.499999999992	62.0421753607103\\
540.699999999992	62.0192307692308\\
541.099999999997	62.0473022912047\\
541.199999999998	62.0358396452984\\
542.099999999994	62.0988565105127\\
542.199999999996	62.0874054951134\\
542.6	62.1153491800258\\
542.700000000006	62.1039056742815\\
542.800000000005	62.1108859826856\\
542.899999999994	62.0994475138122\\
543.300000000009	62.1273463378727\\
543.399999999996	62.1159153633855\\
543.7	62.1368150055167\\
543.800000000008	62.1253906968193\\
543.999999999994	62.1393126263554\\
544.200000000006	62.1164798824178\\
544.399999999997	62.1303948576676\\
544.599999999999	62.1075821553149\\
544.799999999993	62.1214901816847\\
545.000000000007	62.0986974866997\\
545.100000000005	62.1056493030081\\
545.199999999998	62.0942600403448\\
545.400000000004	62.1081576535289\\
545.5	62.0967741935484\\
545.599999999994	62.10371999267\\
545.699999999996	62.0923415170392\\
545.800000000001	62.0992855834402\\
546.700000000009	61.9970738844184\\
546.800000000008	62.0040226732492\\
547.100000000007	61.9700292397661\\
547.500000000008	61.9978086194302\\
547.699999999992	61.9751734209566\\
547.799999999998	61.9821135243658\\
547.999999999994	61.9594964422551\\
548.099999999998	61.9664356074425\\
548.300000000006	61.9438366156091\\
548.699999999995	61.9715743440233\\
548.799999999992	61.9602842047732\\
549.000000000005	61.9741395010016\\
549.200000000004	61.9515747314764\\
549.600000000002	61.9792614153175\\
549.700000000008	61.9679883594034\\
549.900000000006	61.9818181818182\\
550.100000000004	61.9592875318066\\
550.300000000002	61.9731104651163\\
550.799999999994	61.9168633145761\\
551.000000000002	61.9306840863727\\
551.300000000008	61.8969894813203\\
551.400000000003	61.9038984587489\\
551.600000000004	61.8814573137575\\
552.100000000001	61.9159724737414\\
552.200000000003	61.9047619047619\\
552.600000000001	61.9323321874435\\
552.800000000003	61.9099294628323\\
552.999999999996	61.9237027662267\\
553.100000000007	61.9125090383225\\
553.2	61.9193927345021\\
553.4	61.8970189701897\\
553.800000000009	61.9245351146416\\
554	61.9021837213499\\
554.300000000006	61.9227994227994\\
554.399999999997	61.9116321009919\\
555.200000000009	61.9665045921124\\
555.399999999998	61.9441944194419\\
555.800000000001	61.9715776218744\\
555.900000000001	61.9604316546763\\
556.099999999998	61.9741100323625\\
556.199999999993	61.9629696207083\\
556.399999999998	61.9766397124888\\
556.500000000008	61.9655048508804\\
556.899999999992	61.9928186714542\\
557.000000000004	61.9816908992999\\
557.2	61.9953346492015\\
557.399999999994	61.9730941704036\\
557.699999999997	61.9935460738616\\
557.900000000001	61.9713261648746\\
558.699999999995	62.0257695060845\\
558.799999999999	62.0146716765074\\
558.900000000001	62.0214669051878\\
558.999999999999	62.0103738150599\\
559.100000000009	62.0171673819742\\
559.199999999999	62.0060790273556\\
559.299999999994	62.0128709331427\\
559.500000000006	61.9907076483202\\
559.899999999998	62.0178571428572\\
559.999999999991	62.0067845027674\\
560.100000000002	62.0135665833631\\
560.299999999993	61.9914346895075\\
560.400000000003	61.9982158786798\\
560.5	61.9871566179094\\
560.599999999999	61.9939361512395\\
560.699999999999	61.9828815977176\\
560.800000000003	61.9896594758424\\
560.999999999999	61.967563714133\\
561.099999999995	61.9743406985032\\
561.299999999994	61.9522622016388\\
561.400000000005	61.9590382902939\\
561.500000000008	61.9480056980057\\
562.799999999996	62.0358855924676\\
562.900000000005	62.0248667850799\\
563.300000000009	62.0518281860135\\
563.699999999998	62.0078041858815\\
563.899999999997	62.0212765957447\\
564.100000000007	61.9992910315491\\
564.800000000001	62.0463798902461\\
564.900000000005	62.0353982300885\\
565.100000000005	62.0488322717622\\
565.300000000005	62.0268836222144\\
565.399999999993	62.0335985853227\\
565.599999999994	62.0116669612869\\
565.899999999996	62.0318021201414\\
565.999999999991	62.0208443737856\\
566.300000000005	62.0409604519774\\
566.400000000006	62.0300088261253\\
566.600000000009	62.0434092112229\\
566.999999999992	61.9996473285135\\
567.399999999997	62.0264317180617\\
567.499999999993	62.015503875969\\
567.900000000002	62.0422535211268\\
567.999999999995	62.0313325118817\\
568.300000000001	62.0513722730471\\
568.400000000007	62.0404573438874\\
568.700000000002	62.0604781997187\\
568.800000000006	62.0495693443487\\
569.099999999998	62.0695713281799\\
569.200000000008	62.0586685403127\\
569.699999999999	62.0919620919621\\
569.799999999996	62.081066853834\\
570.5	62.1275849982475\\
570.6	62.1166987909585\\
570.900000000005	62.1366024518389\\
571.099999999994	62.1148459383754\\
571.200000000003	62.1214773323998\\
571.400000000009	62.0997375328084\\
571.499999999994	62.1063680895731\\
571.799999999993	62.0737891239727\\
571.900000000008	62.0804195804196\\
572.099999999997	62.0587207270185\\
572.300000000001	62.0719776380154\\
572.4	62.061135371179\\
572.799999999999	62.0876243672543\\
572.900000000004	62.0767888307155\\
573.000000000003	62.0834060373408\\
573.299999999996	62.0509243111266\\
573.399999999996	62.0575414123801\\
573.500000000008	62.0467224546722\\
573.600000000009	62.0533379815235\\
573.700000000002	62.0425235273615\\
573.800000000008	62.0491374803973\\
574.000000000003	62.027521337746\\
574.399999999996	62.0539599651871\\
574.499999999994	62.04316045945\\
574.699999999999	62.0563674321503\\
574.900000000004	62.0347826086957\\
575.599999999992	62.0809449365989\\
575.699999999998	62.0701632511289\\
576.500000000003	62.1227887617066\\
576.599999999997	62.1120166464366\\
576.900000000006	62.1317157712305\\
577.099999999997	62.1101871101871\\
577.299999999998	62.1233113959127\\
577.400000000008	62.1125541125541\\
577.600000000009	62.125670763372\\
577.700000000006	62.1149186569747\\
577.800000000003	62.1214743035127\\
577.900000000001	62.1107266435986\\
578.100000000003	62.123832583881\\
578.400000000006	62.0916162489196\\
578.70000000001	62.1112646855563\\
578.799999999999	62.100535498359\\
578.999999999997	62.1136245898808\\
579.299999999993	62.0814635830169\\
579.700000000006	62.1076233183856\\
579.800000000009	62.0969132609071\\
580.300000000001	62.1295658166782\\
580.599999999998	62.0974685724126\\
580.900000000006	62.1170395869191\\
581.000000000002	62.1063500258131\\
581.100000000004	62.1128699242946\\
581.300000000003	62.0915032679739\\
581.400000000008	62.0980223559759\\
581.599999999998	62.0766718239642\\
581.700000000009	62.0831900996906\\
581.799999999996	62.0725210517271\\
582.400000000006	62.1115879828326\\
582.499999999999	62.1009268795057\\
583.000000000004	62.1334247984908\\
583.100000000003	62.1227709190672\\
583.400000000004	62.1422450728363\\
583.599999999995	62.1209525441151\\
583.800000000007	62.1339270423018\\
583.900000000001	62.1232876712329\\
584.000000000009	62.1297722992638\\
584.199999999994	62.1085059045011\\
584.500000000003	62.1279507355457\\
584.699999999996	62.1067031463748\\
584.899999999993	62.1196581196581\\
585.099999999996	62.0984278879016\\
585.299999999997	62.1113768363512\\
585.399999999998	62.1007685738685\\
585.899999999999	62.1331058020478\\
586.099999999993	62.1119071989082\\
586.499999999993	62.1377429253324\\
586.600000000002	62.1271518663712\\
587.000000000004	62.1529552035428\\
587.100000000006	62.1423705722071\\
587.399999999994	62.1617021276596\\
587.499999999991	62.1511232130701\\
587.7	62.1640013610072\\
587.800000000003	62.1534274536486\\
587.899999999996	62.1598639455782\\
588.000000000001	62.1492943376977\\
588.100000000003	62.1557293437606\\
588.399999999998	62.1240441801189\\
588.600000000008	62.1369118396467\\
588.700000000004	62.1263586956522\\
588.900000000009	62.1392190152801\\
589.000000000003	62.1286708538449\\
589.199999999991	62.1415238418463\\
589.300000000006	62.1309806582966\\
589.400000000009	62.1374045801527\\
589.599999999992	62.1163303374597\\
589.8	62.1291744363452\\
590.000000000005	62.1081172682596\\
590.200000000007	62.1209554463832\\
590.300000000007	62.1104336043361\\
590.599999999999	62.1296766548163\\
590.9	62.0981387478849\\
590.999999999995	62.1045508374218\\
591.100000000007	62.0940460081191\\
591.899999999999	62.1452702702703\\
592	62.1347745313292\\
592.099999999992	62.1411685241472\\
592.300000000003	62.120189061445\\
593.199999999997	62.1776504297994\\
593.400000000004	62.156697556866\\
593.899999999991	62.1885521885522\\
594.000000000006	62.1780844975593\\
594.100000000007	62.1844496802423\\
594.200000000001	62.1739862022548\\
594.300000000001	62.1803499327053\\
594.499999999999	62.1594349142281\\
594.599999999996	62.1657978812847\\
594.900000000007	62.1344537815126\\
595.399999999997	62.1662468513854\\
595.499999999992	62.1558092679651\\
595.900000000009	62.1812080536913\\
596.000000000009	62.1707767153162\\
596.500000000001	62.2024807241033\\
596.700000000008	62.1816353887399\\
597.099999999999	62.2069658405894\\
597.200000000006	62.1965511468274\\
597.399999999998	62.2092050209205\\
597.599999999993	62.1883888238247\\
597.700000000001	62.1947139511542\\
597.800000000005	62.184311757819\\
598.200000000009	62.2095938492395\\
598.399999999999	62.1888053467001\\
598.500000000005	62.1951219512195\\
598.600000000001	62.1847335894438\\
599.499999999993	62.241494329553\\
599.699999999999	62.2207402467489\\
599.799999999991	62.2270378396399\\
599.900000000006	62.2166666666667\\
600.200000000008	62.2355488922206\\
600.300000000007	62.2251832111925\\
600.499999999997	62.2377622377622\\
601.1	62.1756487025948\\
601.599999999991	62.2070799401695\\
601.799999999999	62.1864097026084\\
601.999999999994	62.1989702707191\\
602.099999999994	62.1886416472933\\
602.500000000006	62.2137404580153\\
602.600000000007	62.2034179525469\\
603.000000000006	62.2284861548665\\
603.200000000002	62.2078567876678\\
603.299999999991	62.2141199867418\\
603.400000000007	62.2038111019056\\
603.500000000007	62.2100728959576\\
603.600000000005	62.1997680967368\\
604.100000000004	62.2310493214168\\
604.699999999999	62.1693121693122\\
604.999999999991	62.1880680879194\\
605.300000000002	62.1572514040304\\
605.400000000002	62.1635012386458\\
605.500000000009	62.1532364597094\\
606.000000000004	62.1844580102293\\
606.100000000007	62.1741999340152\\
606.599999999992	62.2053733311356\\
606.799999999993	62.1848739495798\\
606.899999999996	62.1911037891269\\
606.999999999994	62.1808598253994\\
607.400000000005	62.2057613168724\\
607.500000000001	62.1955233706386\\
607.600000000008	62.201744281718\\
607.699999999997	62.1915103652517\\
608.10000000001	62.2163761920421\\
608.20000000001	62.2061482820977\\
608.300000000001	62.2123602892834\\
608.399999999991	62.202136400986\\
608.50000000001	62.2083470259612\\
608.599999999992	62.1981271562346\\
608.999999999997	62.2229518962404\\
609.400000000005	62.1821164889254\\
609.499999999999	62.1883202099738\\
609.599999999994	62.1781203870756\\
609.799999999996	62.1905230365634\\
609.899999999998	62.1803278688525\\
610.000000000005	62.1865267988854\\
610.199999999994	62.1661477961658\\
610.700000000002	62.1971185330714\\
610.799999999992	62.1869373056147\\
611.000000000006	62.1993127147766\\
611.100000000006	62.1891361256545\\
611.699999999992	62.2262177182086\\
611.800000000005	62.2160483739173\\
611.900000000005	62.2222222222222\\
611.999999999993	62.2120568534553\\
612.099999999996	62.218229336818\\
612.400000000009	62.1877551020408\\
612.5	62.1939275220372\\
612.599999999993	62.183776725967\\
612.799999999994	62.1961168216675\\
612.899999999993	62.1859706362153\\
612.999999999995	62.1921383134888\\
613.099999999992	62.1819960861057\\
613.199999999999	62.1881624001305\\
613.399999999992	62.1678891605542\\
613.799999999994	62.1925395015475\\
613.900000000009	62.1824104234528\\
614.299999999995	62.20703125\\
614.400000000001	62.1969080553295\\
614.5	62.2030589000976\\
614.600000000001	62.1929396453555\\
615.199999999998	62.2298065984073\\
615.500000000003	62.1994801819363\\
616.400000000006	62.2546634225466\\
616.599999999994	62.2344738122264\\
616.700000000009	62.2405966277562\\
616.899999999998	62.2204213938412\\
616.999999999992	62.226543509966\\
617.100000000004	62.2164614387557\\
617.199999999991	62.2225822128625\\
617.299999999993	62.2125040492387\\
617.399999999991	62.2186234817814\\
617.50000000001	62.2085492227979\\
617.599999999997	62.2146673142302\\
617.700000000001	62.204596956944\\
618.199999999993	62.2351609251173\\
618.300000000002	62.2250970245796\\
618.400000000007	62.2312045270817\\
618.499999999995	62.2211445198836\\
618.999999999991	62.2516556291391\\
619.100000000007	62.2416020671835\\
619.200000000004	62.247699015017\\
619.299999999993	62.2376493380691\\
619.700000000003	62.2620200064537\\
619.800000000001	62.2519761251815\\
619.9	62.258064516129\\
620.000000000008	62.2480245121755\\
620.199999999991	62.2601966790263\\
620.299999999993	62.2501611863314\\
620.700000000006	62.2744845360825\\
620.799999999996	62.2644548236431\\
621.000000000001	62.2766060215746\\
621.200000000009	62.2565588282633\\
621.300000000003	62.2626327647248\\
621.499999999998	62.2425997425998\\
621.699999999991	62.2547442907687\\
621.80000000001	62.244733880045\\
621.900000000003	62.2508038585209\\
622.199999999998	62.2207938293428\\
622.400000000009	62.2329317269076\\
622.499999999992	62.2229360745262\\
622.899999999995	62.247191011236\\
623.099999999994	62.2272143774069\\
623.200000000008	62.2332745066581\\
623.300000000007	62.223291626564\\
623.400000000004	62.2293504410585\\
623.50000000001	62.2193713919179\\
623.999999999996	62.2496394808524\\
624.099999999991	62.23966677347\\
624.60000000001	62.2698895469826\\
624.899999999991	62.24\\
625	62.2460406334987\\
625.099999999996	62.236084452975\\
625.699999999998	62.2722914669223\\
626	62.2424532822233\\
626.199999999991	62.2545106179147\\
626.300000000011	62.2445721583653\\
626.500000000009	62.2566230450048\\
626.600000000002	62.2466890059039\\
626.800000000003	62.2587334503111\\
626.899999999999	62.2488038277512\\
627.200000000001	62.2668579626973\\
627.400000000009	62.2470119521912\\
627.900000000006	62.2770700636943\\
628.000000000007	62.2671549116383\\
628.299999999996	62.2851686823679\\
628.400000000001	62.2752585521082\\
628.499999999992	62.2812599427299\\
628.699999999997	62.2614503816794\\
628.800000000007	62.2674511051042\\
628.899999999997	62.2575516693164\\
628.999999999992	62.2635511047528\\
629.099999999999	62.2536554354736\\
629.600000000001	62.2836271240273\\
629.700000000007	62.2737376945062\\
629.799999999996	62.2797269407843\\
629.899999999995	62.2698412698413\\
630.099999999993	62.2818152967312\\
630.199999999994	62.2719339996827\\
630.300000000009	62.2779187817259\\
630.49999999999	62.2581668252458\\
630.800000000008	62.276113488667\\
630.900000000004	62.2662440570523\\
631.099999999994	62.2782002534854\\
631.300000000009	62.258473234083\\
631.400000000008	62.264449722882\\
631.699999999995	62.2348844571067\\
632.699999999995	62.2945638432364\\
632.800000000003	62.2847211249803\\
632.999999999998	62.2966356025904\\
633.100000000002	62.2867972204675\\
633.300000000008	62.2987053994316\\
633.4	62.2888713496448\\
633.600000000007	62.3007732365473\\
633.700000000004	62.2909435153045\\
634.499999999991	62.3384809328711\\
634.600000000008	62.3286592090752\\
634.700000000005	62.3345935727788\\
634.899999999996	62.3149606299213\\
635.099999999996	62.3268261964735\\
635.199999999991	62.317015583189\\
636.400000000004	62.3880597014925\\
636.5	62.378259503613\\
637.200000000007	62.4195826141535\\
637.400000000006	62.4\\
637.499999999992	62.4058971141782\\
637.599999999991	62.3961110239925\\
637.999999999992	62.4196834351983\\
638.099999999996	62.4099028517706\\
638.400000000006	62.4275646045419\\
638.7	62.3982467125861\\
638.999999999994	62.4158973556564\\
639.399999999998	62.3768569194683\\
639.499999999999	62.3827392120075\\
639.600000000008	62.3729873378146\\
639.800000000002	62.3847476168151\\
639.900000000007	62.375\\
640.099999999999	62.3867541393315\\
640.300000000006	62.367270455965\\
640.399999999996	62.3731459797034\\
640.499999999995	62.3634093037777\\
640.599999999991	62.3692835960668\\
640.800000000009	62.3498205648307\\
641.30000000001	62.3791705643904\\
641.399999999995	62.369446609509\\
642.000000000003	62.4046098738514\\
642.299999999996	62.3754669987547\\
643.099999999993	62.422263681592\\
643.200000000007	62.4125602362817\\
643.600000000002	62.4359173528041\\
643.699999999999	62.426219322771\\
643.800000000008	62.4320546668737\\
644.100000000008	62.4029804408569\\
644.299999999995	62.4146492861577\\
644.4	62.4049650892164\\
644.500000000005	62.4107973937326\\
644.600000000004	62.4011167985109\\
644.69999999999	62.4069478908189\\
644.900000000001	62.3875968992248\\
645.200000000002	62.405082907175\\
645.30000000001	62.3954136969321\\
645.400000000011	62.4012393493416\\
645.70000000001	62.3722514710437\\
645.799999999996	62.3780771017185\\
646.000000000008	62.3587679925708\\
646.099999999999	62.3645930052615\\
646.300000000008	62.345297029703\\
646.900000000002	62.3802163833076\\
646.999999999991	62.3705764178643\\
647.600000000011	62.4054346147908\\
647.700000000006	62.3958011732016\\
648.000000000001	62.4132078382966\\
648.200000000002	62.3939534166281\\
648.499999999997	62.4113475177305\\
648.699999999995	62.3921085080148\\
648.900000000002	62.4036979969183\\
649.100000000008	62.3844731977819\\
649.800000000011	62.424988459763\\
649.900000000001	62.4153846153846\\
650.1	62.4269455552138\\
650.300000000004	62.4077490774908\\
650.6	62.4250806823421\\
650.799999999995	62.4058995237364\\
650.999999999998	62.4174473967133\\
651.200000000006	62.3982803623522\\
651.300000000009	62.4040528093337\\
651.99999999999	62.3370648673516\\
652.100000000005	62.3428396197485\\
652.200000000003	62.3332822321018\\
652.799999999993	62.3678970745903\\
652.899999999995	62.3583460949464\\
654.500000000009	62.4503513596089\\
654.699999999999	62.4312767257178\\
654.800000000006	62.4370132844709\\
654.900000000006	62.4274809160305\\
655.200000000007	62.4446818251183\\
655.30000000001	62.4351541043637\\
655.4	62.4408848207475\\
655.499999999995	62.431360585723\\
655.999999999992	62.4599908550526\\
656.100000000002	62.4504724169461\\
656.499999999998	62.4733475479744\\
656.600000000007	62.4638343231308\\
656.899999999991	62.4809741248097\\
657.09999999999	62.46195982958\\
657.200000000008	62.46767077438\\
657.399999999994	62.4486692015209\\
657.700000000009	62.4657950744907\\
657.800000000001	62.4563003495972\\
657.899999999996	62.4620060790274\\
658.000000000001	62.4525148153776\\
658.49999999999	62.4810203461889\\
658.600000000009	62.4715348413542\\
659.199999999991	62.5056878507508\\
659.300000000004	62.4962086745526\\
659.799999999991	62.5246249431732\\
660.199999999998	62.4867484476753\\
660.400000000009	62.4981074943225\\
660.600000000008	62.479188739216\\
660.899999999991	62.4962178517398\\
661.000000000007	62.4867644834367\\
661.200000000002	62.4981097837593\\
661.399999999995	62.4792139077853\\
661.700000000008	62.496222423693\\
661.800000000006	62.4867804804351\\
662.199999999999	62.5094368111128\\
662.299999999999	62.5\\
662.89999999999	62.5339366515837\\
663.000000000009	62.5245061076761\\
663.19999999999	62.5358058193879\\
663.39999999999	62.5169555388094\\
663.500000000001	62.5226039783002\\
663.599999999996	62.5131836673196\\
663.999999999992	62.5357626863424\\
664.199999999993	62.5169351196748\\
664.500000000001	62.5338549503461\\
664.599999999991	62.5244471190011\\
664.700000000006	62.5300842358604\\
665.099999999998	62.49248346362\\
665.199999999997	62.4981211483541\\
665.399999999997	62.4793388429752\\
665.499999999996	62.4849759615385\\
665.700000000011	62.4662060678883\\
665.799999999993	62.4718426190119\\
665.900000000004	62.4624624624625\\
666.199999999997	62.4793636500075\\
666.300000000005	62.4699879951981\\
666.60000000001	62.4868756562172\\
666.699999999998	62.4775044991002\\
666.89999999999	62.4887556221889\\
667.299999999997	62.4513035660773\\
667.400000000005	62.4569288389513\\
667.500000000009	62.4475733972439\\
667.699999999999	62.4588200059898\\
667.80000000001	62.4494684833059\\
667.900000000002	62.4550898203593\\
667.999999999992	62.4457416554408\\
668.099999999999	62.4513618677043\\
668.300000000003	62.4326750448833\\
668.599999999995	62.4495289367429\\
668.700000000001	62.4401913875598\\
668.800000000001	62.445806548064\\
668.999999999993	62.4271409355851\\
669.299999999996	62.4439796832985\\
669.400000000011	62.4346527259149\\
669.500000000007	62.4402628434887\\
669.800000000008	62.4123003433348\\
670.09999999999	62.4291256341391\\
670.199999999994	62.4198120244667\\
670.399999999992	62.4310216256525\\
670.50000000001	62.4217118997912\\
670.600000000009	62.427314745788\\
670.699999999995	62.4180083482409\\
671.199999999998	62.4460002979294\\
671.299999999993	62.4366994340185\\
671.400000000008	62.4422933730454\\
671.60000000001	62.4237010570195\\
671.700000000004	62.4292944328669\\
671.99999999999	62.4014283588752\\
672.099999999991	62.4070217197263\\
672.299999999992	62.3884592504462\\
672.600000000001	62.4052326445667\\
672.799999999993	62.3866844999257\\
672.999999999999	62.3978606447779\\
673.299999999991	62.3700623700624\\
673.400000000006	62.3756495916852\\
673.499999999995	62.3663895486936\\
673.599999999999	62.3719756568206\\
673.99999999999	62.3349651387035\\
674.900000000005	62.3851851851852\\
675.100000000011	62.3667061611374\\
675.400000000007	62.3834196891192\\
675.500000000007	62.3741859088218\\
675.700000000006	62.3853211009174\\
675.799999999995	62.3760911377423\\
675.899999999997	62.3816568047337\\
676.200000000001	62.3539849179358\\
676.400000000004	62.3651145602365\\
676.600000000003	62.346682429437\\
676.70000000001	62.3522458628842\\
676.99999999999	62.3246197016689\\
677.100000000006	62.3301831069108\\
677.200000000011	62.3209803632068\\
677.400000000006	62.3321033210332\\
677.500000000008	62.3229043683589\\
678.399999999999	62.3728813559322\\
678.600000000003	62.3545012523943\\
678.699999999997	62.3600471420153\\
679.000000000006	62.3324988955971\\
679.40000000001	62.3546725533481\\
679.699999999992	62.3271550456016\\
679.9	62.3382352941177\\
679.999999999992	62.3290692545214\\
680.199999999993	62.3401440540938\\
680.399999999991	62.3218221895665\\
680.600000000005	62.332892610548\\
680.699999999998	62.3237367802585\\
680.999999999992	62.3403318161797\\
681.099999999997	62.3311802701116\\
681.299999999997	62.342236571764\\
681.599999999994	62.3148012322136\\
681.700000000008	62.3203285420945\\
681.900000000005	62.3020527859238\\
681.999999999993	62.3075795337927\\
682.09999999999	62.2984462034594\\
682.500000000006	62.320539115148\\
682.699999999996	62.3022847100176\\
682.800000000004	62.3078049494802\\
682.999999999997	62.2895622895623\\
683.199999999995	62.3006000292697\\
683.299999999989	62.2914837576822\\
683.399999999998	62.2970007315289\\
683.899999999997	62.2514619883041\\
684.100000000007	62.2624963460976\\
684.19999999999	62.2533976326173\\
684.5	62.2699386503067\\
684.600000000001	62.2608441653279\\
684.700000000006	62.2663551401869\\
684.800000000003	62.2572638341364\\
684.999999999993	62.2682820026274\\
685.099999999996	62.2591943957969\\
685.300000000004	62.2702071782901\\
685.40000000001	62.2611232676878\\
685.499999999991	62.2666277712952\\
685.700000000008	62.2484689413823\\
685.999999999989	62.2649759510276\\
686.100000000005	62.2559020693675\\
686.199999999999	62.2614017193647\\
686.400000000001	62.2432629278951\\
686.500000000005	62.2487620157297\\
686.699999999998	62.2306348281887\\
686.799999999996	62.2361333527442\\
686.900000000002	62.2270742358079\\
687.000000000002	62.2325716780672\\
687.100000000007	62.2235157159488\\
687.200000000007	62.2290120762404\\
687.600000000004	62.1928166351607\\
688.300000000009	62.2312608948286\\
688.399999999999	62.2222222222222\\
688.600000000006	62.2331929722666\\
688.700000000011	62.2241579558653\\
689.500000000005	62.2679814385151\\
689.599999999997	62.2589531680441\\
689.800000000003	62.2698941875634\\
689.900000000002	62.2608695652174\\
690.099999999996	62.2718052738337\\
690.399999999993	62.2447501810282\\
690.699999999992	62.2611464968153\\
690.89999999999	62.2431259044862\\
691.200000000002	62.2595110661073\\
691.300000000002	62.2505062192653\\
691.400000000001	62.2559652928417\\
691.5	62.246963562753\\
692.099999999998	62.2796879514591\\
692.200000000005	62.2706918965766\\
692.4	62.2815884476534\\
692.59999999999	62.2636061787209\\
692.999999999995	62.2853845044005\\
693.099999999995	62.2763993075591\\
693.400000000003	62.2927180966114\\
693.500000000004	62.2837370242215\\
693.700000000006	62.2946093975209\\
693.8	62.2856319354374\\
693.900000000009	62.2910662824208\\
694.200000000004	62.2641509433962\\
694.399999999998	62.2750179985601\\
694.49999999999	62.2660524042614\\
694.799999999997	62.2823427831343\\
694.999999999997	62.2644223852683\\
696.099999999998	62.3240448147084\\
696.19999999999	62.3150940686486\\
696.300000000004	62.3205054566341\\
696.399999999992	62.3115577889447\\
697.300000000002	62.3601950100373\\
697.799999999992	62.315517982519\\
698.000000000005	62.3263142816215\\
698.199999999993	62.3084634111413\\
698.299999999991	62.3138602520046\\
698.399999999996	62.3049391553329\\
698.499999999998	62.3103349556255\\
698.700000000008	62.2925014310246\\
698.800000000002	62.2978966948061\\
698.999999999999	62.2800743813474\\
699.400000000011	62.301644031451\\
699.499999999998	62.2927387078331\\
699.800000000009	62.3089012716102\\
699.900000000008	62.3\\
700.299999999999	62.3215305539692\\
700.400000000004	62.3126338329765\\
700.899999999993	62.339514978602\\
701.000000000003	62.3306233062331\\
701.2	62.3413660345073\\
701.299999999992	62.3324779013402\\
701.799999999998	62.3593104430831\\
701.999999999995	62.3415467882068\\
702.199999999991	62.3522711092126\\
702.39999999999	62.3345195729537\\
702.599999999999	62.3452397893838\\
702.700000000001	62.3363688104724\\
702.9	62.3470839260313\\
702.999999999994	62.3382164699189\\
703.099999999993	62.3435722411832\\
703.199999999999	62.3347078060572\\
703.400000000004	62.3454157782516\\
703.499999999995	62.3365548607163\\
703.7	62.3472577436772\\
703.799999999993	62.3384003409575\\
704.099999999994	62.3544447600114\\
704.199999999994	62.3455913673151\\
704.400000000005	62.3562810503904\\
704.799999999999	62.3208965810753\\
704.900000000005	62.3262411347518\\
705.099999999989	62.3085649461146\\
705.399999999997	62.3245924875975\\
705.500000000006	62.3157596371882\\
705.900000000002	62.3371104815864\\
706.000000000001	62.3282821130152\\
706.400000000001	62.3496107572541\\
706.800000000004	62.3143301739992\\
706.900000000001	62.3196605374823\\
706.999999999992	62.3108471220478\\
707.299999999994	62.3268306474413\\
707.400000000008	62.3180212014134\\
707.700000000002	62.3339926532919\\
707.80000000001	62.3251871733296\\
707.900000000007	62.3305084745763\\
708.000000000007	62.3217059737325\\
708.100000000005	62.3270262637673\\
708.300000000001	62.3094297007341\\
708.500000000003	62.3200677392041\\
708.800000000002	62.2936944561997\\
709.199999999994	62.3149584096997\\
709.299999999991	62.3061742317451\\
709.900000000011	62.3380281690141\\
710.200000000005	62.3116992819935\\
710.300000000004	62.3170045045045\\
710.400000000001	62.3082336382829\\
710.599999999992	62.3188405797101\\
710.69999999999	62.3100731570062\\
710.900000000002	62.3206751054852\\
711.000000000007	62.3119111236113\\
711.100000000008	62.3172103487064\\
711.299999999997	62.2996907506325\\
711.699999999996	62.3208766507446\\
711.800000000004	62.3121224891136\\
712.099999999989	62.32799775344\\
712.200000000005	62.3192475080724\\
713.100000000009	62.366797532249\\
713.599999999993	62.3231049460558\\
714.199999999998	62.3547529049419\\
714.299999999992	62.3460246360582\\
714.800000000007	62.3723597705973\\
714.899999999993	62.3636363636364\\
715.000000000001	62.3688994546217\\
715.1	62.3601789709172\\
715.200000000011	62.3654410736754\\
715.400000000011	62.3480083857442\\
715.69999999999	62.3637887678122\\
715.800000000001	62.355077524794\\
716.499999999992	62.3918504046888\\
716.699999999995	62.3744419642857\\
716.799999999989	62.3796903333798\\
717.000000000011	62.3622925672849\\
717.099999999997	62.3675404350251\\
717.199999999995	62.3588456712672\\
717.399999999989	62.3693379790941\\
717.500000000003	62.360646599777\\
717.599999999995	62.3658910408249\\
717.799999999995	62.3485165064772\\
717.90000000001	62.3537604456825\\
718.300000000006	62.3190423162584\\
718.599999999997	62.3347711145123\\
718.699999999998	62.3260990539789\\
719.100000000005	62.3470522803115\\
719.199999999994	62.3383845405255\\
719.799999999989	62.3697735796638\\
719.9	62.3611111111111\\
720.300000000001	62.3820099944475\\
720.400000000004	62.3733518390007\\
720.49999999999	62.3785734110463\\
720.599999999997	62.3699181351464\\
721.100000000007	62.396006655574\\
721.29999999999	62.3787080676462\\
721.700000000011	62.3995566638958\\
721.900000000007	62.3822714681441\\
722.000000000012	62.387480958316\\
722.099999999999	62.3788424259208\\
722.399999999993	62.3944636678201\\
722.500000000006	62.3858289510102\\
722.599999999999	62.3910336239103\\
722.799999999991	62.3737723059898\\
722.899999999995	62.3789764868603\\
723.000000000008	62.3703498824506\\
723.90000000001	62.4171270718232\\
724.300000000012	62.3826615129763\\
724.699999999996	62.4034216335541\\
724.799999999994	62.3948130776659\\
725.600000000007	62.4362684304809\\
725.699999999998	62.427666023698\\
725.90000000001	62.4380165289256\\
725.99999999999	62.4294174356149\\
726.099999999991	62.4345910217571\\
726.199999999998	62.4259947680022\\
726.900000000002	62.4621733149931\\
726.999999999997	62.4535827258974\\
727.099999999991	62.4587458745875\\
727.200000000005	62.4501581190705\\
727.900000000006	62.4862637362637\\
728.099999999995	62.4691018950838\\
728.20000000001	62.4742551146506\\
728.499999999998	62.44853143014\\
728.799999999994	62.4639868294691\\
728.899999999993	62.4554183813443\\
729.1	62.4657158529896\\
729.299999999992	62.4485878804497\\
729.499999999993	62.4588815789474\\
729.599999999995	62.4503220501576\\
729.800000000001	62.4606110426086\\
729.89999999999	62.4520547945205\\
730.199999999989	62.4674791181706\\
730.299999999992	62.4589266155531\\
730.500000000008	62.4692033944703\\
730.799999999997	62.4435627308797\\
731.100000000008	62.4589715536105\\
731.19999999999	62.4504307397785\\
731.300000000006	62.455564670495\\
731.399999999999	62.447026657553\\
731.499999999988	62.452159650082\\
731.599999999993	62.4436244362444\\
731.799999999989	62.4538871430523\\
732.000000000007	62.4368255702773\\
732.100000000002	62.4419557497951\\
732.199999999994	62.4334289225727\\
732.299999999989	62.4385581649372\\
732.499999999998	62.4215124215124\\
733.100000000001	62.4522640480087\\
733.200000000008	62.4437474430656\\
733.299999999996	62.4488682847014\\
733.39999999999	62.4403544648943\\
733.699999999991	62.4557100027255\\
733.800000000003	62.4471998909933\\
734.299999999999	62.4727668845316\\
734.399999999993	62.4642614023145\\
734.50000000001	62.4693710863055\\
734.599999999995	62.4608683816524\\
735	62.4812950618963\\
735.099999999998	62.4727965179543\\
735.200000000004	62.4779001767986\\
735.499999999998	62.452419793366\\
735.599999999994	62.4575234470572\\
735.70000000001	62.4490350638761\\
735.899999999999	62.4592391304348\\
736.000000000001	62.4507539736449\\
736.100000000004	62.4558543873947\\
736.200000000001	62.4473719951107\\
736.400000000008	62.4575695858792\\
736.499999999988	62.4490904154222\\
736.6	62.4541875933216\\
736.699999999995	62.4457111834962\\
736.800000000001	62.4508074365586\\
736.999999999989	62.4338624338624\\
737.199999999989	62.4440526244405\\
737.4	62.4271186440678\\
737.6	62.4373051375898\\
737.799999999989	62.4203821656051\\
737.899999999997	62.4254742547426\\
738.200000000007	62.4001083570364\\
738.50000000001	62.4153804494991\\
738.700000000001	62.3984840281538\\
738.800000000008	62.4035728786033\\
739.100000000004	62.3782467532468\\
739.199999999995	62.3833355877181\\
739.299999999988	62.3748985664052\\
739.799999999993	62.4003243681579\\
739.899999999993	62.3918918918919\\
740.39999999999	62.4172856178258\\
740.599999999989	62.4004320237613\\
740.700000000001	62.4055075593952\\
740.79999999999	62.3970846268052\\
741.100000000007	62.4123043712898\\
741.200000000001	62.4038850667746\\
741.899999999998	62.4393530997305\\
742.000000000011	62.4309392265193\\
742.599999999996	62.4612898882456\\
742.699999999991	62.4528809908455\\
743.000000000001	62.4680392948459\\
743.100000000011	62.45963401507\\
743.299999999994	62.4697336561743\\
743.500000000002	62.4529316837009\\
743.599999999995	62.4579803684281\\
743.700000000007	62.4495832212961\\
743.799999999998	62.4546309987902\\
743.900000000003	62.4462365591398\\
744.199999999988	62.4613731022437\\
744.400000000009	62.4445936870383\\
744.600000000002	62.4546797368068\\
744.799999999994	62.4379111290106\\
745.00000000001	62.4479935579117\\
745.10000000001	62.4396135265701\\
745.200000000002	62.4446531598014\\
745.299999999992	62.4362758250604\\
745.500000000005	62.4463519313305\\
745.700000000008	62.4296057924377\\
745.900000000009	62.4396782841823\\
746.000000000008	62.4313094759416\\
746.400000000007	62.4514400535834\\
746.700000000002	62.4263524370648\\
746.899999999989	62.4364123159304\\
746.999999999991	62.4280551465667\\
747.099999999993	62.4330835117773\\
747.200000000004	62.4247290244882\\
747.400000000004	62.4347826086957\\
747.499999999994	62.426431246656\\
747.599999999998	62.4314564664973\\
747.799999999994	62.4147613317288\\
747.999999999998	62.4248095174442\\
748.10000000001	62.4164661855119\\
748.600000000011	62.441565379992\\
748.700000000007	62.4332264957265\\
748.900000000011	62.4432576769025\\
749.099999999997	62.4265883609183\\
749.199999999995	62.4316028293074\\
749.400000000001	62.4149432955304\\
749.499999999995	62.4199573105656\\
749.600000000002	62.4116313191944\\
749.899999999991	62.4266666666667\\
750	62.4183442207706\\
750.299999999997	62.433368869936\\
750.399999999995	62.4250499666889\\
750.5	62.4300559552358\\
750.599999999998	62.4217397096044\\
750.699999999998	62.4267448055408\\
750.900000000007	62.4101198402131\\
750.999999999996	62.41512448409\\
751.2	62.3985092506322\\
751.300000000003	62.4035134415757\\
751.400000000001	62.3952095808383\\
751.599999999998	62.4052148463483\\
751.699999999995	62.3969140728917\\
752.10000000001	62.4169103961712\\
752.199999999997	62.408613585006\\
752.300000000007	62.4136097820308\\
752.400000000001	62.4053156146179\\
752.999999999998	62.4352675607489\\
753.099999999993	62.4269782262347\\
753.499999999999	62.4469214437367\\
753.699999999994	62.4303528787477\\
753.900000000008	62.4403183023873\\
754.000000000003	62.4320381912213\\
754.100000000002	62.4370193582604\\
754.199999999991	62.4287418798886\\
754.399999999991	62.4387011265739\\
754.7	62.413884472708\\
754.799999999992	62.4188634256193\\
754.89999999999	62.4105960264901\\
755.099999999999	62.4205508474576\\
755.200000000008	62.4122865086721\\
755.700000000009	62.4371526858957\\
755.799999999996	62.428892710676\\
755.900000000002	62.4338624338624\\
755.999999999989	62.4256050786933\\
756.099999999994	62.4305739222428\\
756.200000000006	62.4223191855084\\
756.500000000001	62.4372191382501\\
756.599999999989	62.4289678868772\\
756.899999999993	62.443857331572\\
756.999999999996	62.4356095628054\\
757.099999999998	62.4405705229794\\
757.200000000002	62.4323253664334\\
757.300000000012	62.4372854502245\\
757.500000000004	62.4208025343189\\
757.600000000001	62.4257621750033\\
757.9	62.401055408971\\
758.200000000008	62.4159303705657\\
758.400000000008	62.3994726433751\\
758.49999999999	62.4044292117058\\
758.599999999991	62.3962040332147\\
758.799999999997	62.4061141125313\\
759.000000000007	62.3896719799763\\
759.200000000005	62.3995785591993\\
759.399999999994	62.38314680711\\
759.799999999992	62.4029477562837\\
759.999999999994	62.3865280884094\\
760.499999999989	62.4112542729424\\
760.600000000002	62.4030498225319\\
761.700000000012	62.457337883959\\
761.900000000012	62.4409448818898\\
761.999999999992	62.445873244981\\
762.199999999989	62.429489702217\\
762.300000000002	62.4344176285415\\
762.500000000006	62.4180435352741\\
762.600000000001	62.4229710239937\\
762.700000000004	62.4147876245412\\
762.999999999997	62.4295636220679\\
763.2	62.4132058168479\\
763.30000000001	62.4181294210113\\
763.399999999993	62.4099541584807\\
763.49999999999	62.4148768988999\\
763.699999999988	62.3985336475517\\
763.800000000001	62.4034559497317\\
763.899999999999	62.3952879581152\\
764.000000000004	62.4002093966758\\
764.200000000003	62.3838806751276\\
764.299999999996	62.388801674516\\
764.399999999995	62.380640941792\\
764.600000000001	62.3904799267687\\
764.800000000012	62.37416655772\\
764.899999999991	62.3790849673203\\
764.999999999991	62.3709319043262\\
765.199999999999	62.3807657127924\\
765.399999999989	62.3644676681907\\
765.699999999998	62.3792112823191\\
765.900000000001	62.3629242819843\\
766.100000000002	62.3727486296006\\
766.399999999998	62.3483365949119\\
767.000000000001	62.3777864685178\\
767.100000000003	62.3696558915537\\
768.300000000012	62.4284226965122\\
768.400000000007	62.4202992843201\\
768.500000000007	62.4251886546969\\
768.599999999995	62.417067776766\\
768.700000000004	62.4219562955255\\
768.799999999995	62.4138379503186\\
769.100000000009	62.4284971398856\\
769.200000000007	62.4203821656051\\
769.400000000006	62.4301494476933\\
769.799999999993	62.3977139888297\\
769.89999999999	62.4025974025974\\
769.999999999994	62.3944942215297\\
770.100000000009	62.3993767852506\\
770.699999999996	62.3508043591074\\
770.800000000004	62.3556881567\\
770.900000000012	62.3476005188067\\
771.299999999995	62.3671247083225\\
771.39999999999	62.3590408295528\\
771.800000000006	62.3785464438399\\
772.399999999991	62.3300970873786\\
772.600000000001	62.3398472887278\\
772.89999999999	62.3156532988357\\
773.000000000008	62.3205277454404\\
773.200000000008	62.3044096728307\\
773.3	62.3092836824412\\
773.400000000007	62.3012281835811\\
773.80000000001	62.3207132704484\\
773.900000000002	62.312661498708\\
774.000000000005	62.3175300348792\\
774.099999999996	62.3094807543271\\
774.399999999989	62.3240800516462\\
774.500000000002	62.3160340821069\\
774.900000000004	62.3354838709677\\
775.00000000001	62.3274416204361\\
775.099999999989	62.3323013415893\\
775.400000000005	62.3081882656351\\
775.600000000012	62.3179064071162\\
775.799999999997	62.3018430210079\\
775.899999999993	62.3067010309278\\
776.100000000007	62.2906467405308\\
776.200000000009	62.295504315342\\
776.300000000009	62.2874806800618\\
777.600000000002	62.3505207663624\\
777.699999999998	62.3425044998714\\
778.00000000001	62.3570235188279\\
778.100000000008	62.349010537137\\
778.5	62.3683534549191\\
778.599999999993	62.3603441633492\\
778.799999999989	62.370008987033\\
779.099999999999	62.3459958932238\\
779.300000000006	62.3556581986143\\
779.599999999995	62.3316660253944\\
779.699999999991	62.3364965375737\\
779.900000000001	62.3205128205128\\
780.200000000004	62.3349993592208\\
780.500000000005	62.3110427875993\\
780.900000000011	62.3303457106274\\
781.199999999998	62.3064123896071\\
781.999999999991	62.3449686740826\\
782.200000000005	62.3290297839704\\
782.399999999988	62.3386581469649\\
782.5	62.3306925632507\\
783.099999999994	62.3595505617977\\
783.299999999996	62.3436303293337\\
783.400000000001	62.3484365028717\\
783.500000000003	62.3404798366514\\
784.000000000001	62.3644943247035\\
784.1	62.3565416985463\\
784.599999999999	62.3805275901618\\
784.699999999988	62.3725790010194\\
784.800000000011	62.377372913747\\
784.900000000004	62.3694267515924\\
785.099999999988	62.3790117167601\\
785.200000000012	62.3710683815103\\
785.299999999996	62.375859434683\\
785.400000000011	62.3679185232336\\
785.500000000001	62.3727087576375\\
785.799999999997	62.3488993510625\\
786.099999999998	62.3632663444416\\
786.199999999989	62.3553351138242\\
786.799999999997	62.384038632609\\
786.900000000011	62.3761118170267\\
787	62.3808918815907\\
787.100000000005	62.3729674796748\\
787.199999999991	62.3777467293281\\
787.300000000001	62.3698247396495\\
787.499999999987	62.3793803961402\\
787.600000000006	62.3714612161991\\
787.899999999995	62.3857868020305\\
787.999999999993	62.3778708285751\\
788.199999999997	62.3874159583915\\
788.300000000011	62.3795027904617\\
788.400000000012	62.3842739378567\\
788.500000000009	62.3763631752473\\
788.699999999993	62.3859026369168\\
788.899999999988	62.3700887198986\\
789.199999999998	62.3843912327379\\
789.299999999998	62.3764884722574\\
789.500000000007	62.3860182370821\\
789.800000000002	62.3623243448538\\
789.900000000001	62.3670886075949\\
790.000000000007	62.3591950386027\\
790.099999999994	62.3639584915211\\
790.199999999988	62.356067316209\\
790.399999999988	62.3655913978495\\
790.599999999989	62.3498166181864\\
790.799999999995	62.359337463649\\
791.099999999999	62.3356926188069\\
791.200000000005	62.3404524200682\\
791.300000000011	62.3325751832196\\
792.400000000001	62.384858044164\\
792.499999999999	62.3769871309614\\
792.599999999989	62.3817333165132\\
792.700000000003	62.3738647830474\\
792.799999999995	62.3786101652163\\
793	62.362879838608\\
793.100000000009	62.3676248108926\\
793.200000000009	62.3597630152528\\
793.399999999991	62.3692501575299\\
793.500000000001	62.3613911290323\\
793.800000000002	62.3756140571861\\
793.999999999989	62.3599042941695\\
794.100000000007	62.3646436665827\\
794.199999999994	62.3567921440262\\
794.400000000004	62.3662680931403\\
794.500000000003	62.3584193304807\\
794.600000000001	62.3631559078898\\
794.700000000001	62.3553095118269\\
794.900000000007	62.3647798742138\\
794.999999999999	62.3569362344359\\
795.500000000004	62.3805932629462\\
795.600000000011	62.372753550333\\
795.699999999995	62.3774817793416\\
796.100000000009	62.3461441848782\\
796.500000000008	62.3650514687422\\
796.599999999992	62.3572235471319\\
796.699999999994	62.3619477911647\\
796.900000000013	62.3462986198243\\
797.300000000011	62.3651868572862\\
797.600000000012	62.3417324808825\\
798.100000000012	62.3653219744425\\
798.199999999995	62.3575097081298\\
798.799999999994	62.3857804481162\\
798.89999999999	62.377972465582\\
799.599999999998	62.4109040890334\\
800	62.3797025371829\\
800.299999999991	62.3938030984508\\
800.40000000001	62.3860087445347\\
800.799999999988	62.4047946060682\\
801.199999999987	62.3736428304006\\
801.700000000006	62.3971065103517\\
801.799999999991	62.3893253522883\\
802.500000000004	62.4221280837279\\
802.700000000007	62.406576980568\\
802.899999999998	62.4159402241594\\
803.199999999998	62.3926303996016\\
803.300000000008	62.3973114264376\\
803.700000000008	62.3662602637472\\
803.899999999995	62.3756218905473\\
804.000000000008	62.3678646934461\\
804.100000000002	62.3725441432479\\
804.199999999988	62.3647892577397\\
804.300000000006	62.3694679264048\\
804.399999999988	62.3617153511498\\
804.600000000013	62.3710699639617\\
804.700000000008	62.3633200795229\\
804.900000000002	62.3726708074534\\
805.199999999988	62.3494349931702\\
805.899999999992	62.3821339950372\\
806.099999999989	62.36665839742\\
806.300000000011	62.3759920634921\\
806.399999999988	62.3682579045257\\
806.60000000001	62.3775877029875\\
806.999999999993	62.3466732746871\\
807.099999999991	62.3513379583746\\
807.200000000002	62.3436145175276\\
807.39999999999	62.3529411764706\\
807.50000000001	62.3452204061417\\
807.599999999999	62.3498823820726\\
807.700000000009	62.3421639019559\\
807.800000000012	62.3468251021166\\
808.099999999998	62.3236822568671\\
808.299999999999	62.3330034636319\\
808.399999999992	62.3252937538652\\
808.500000000008	62.3299530051942\\
808.600000000001	62.3222455793248\\
808.999999999994	62.3408725744655\\
809.300000000012	62.3177662466024\\
809.500000000013	62.3270750988142\\
809.600000000008	62.3193775472397\\
809.799999999988	62.3286825534017\\
809.899999999991	62.320987654321\\
810.09999999999	62.3302888175759\\
810.300000000012	62.314906219151\\
810.399999999996	62.3195558297347\\
810.800000000013	62.288814897028\\
811.099999999993	62.3027613412229\\
811.300000000013	62.2874044860735\\
811.499999999998	62.2966978807294\\
811.599999999989	62.2890230380683\\
812.100000000003	62.3122383649348\\
812.199999999993	62.3045672780992\\
812.600000000003	62.3231204626554\\
812.699999999992	62.3154527559055\\
812.900000000004	62.3247232472325\\
813.100000000014	62.3093949827841\\
813.199999999997	62.3140292634944\\
813.39999999999	62.2987092808851\\
813.499999999994	62.303343166175\\
813.599999999991	62.2956863708984\\
813.799999999993	62.3049514682394\\
813.900000000001	62.2972972972973\\
814.000000000007	62.3019285100111\\
814.100000000004	62.2942765905183\\
814.299999999988	62.303536345776\\
814.39999999999	62.2958870472683\\
814.500000000009	62.3005155904739\\
814.599999999989	62.2928685405671\\
814.700000000007	62.2974963181149\\
814.899999999992	62.2822085889571\\
814.999999999987	62.2868359710465\\
815.099999999994	62.2791952894995\\
815.599999999989	62.3023170283192\\
815.999999999991	62.2717804190663\\
816.499999999992	62.2948812147931\\
816.599999999996	62.2872535814865\\
817.300000000008	62.3195497920235\\
817.399999999988	62.3119266055046\\
817.50000000001	62.3165362035225\\
817.599999999999	62.3089152500917\\
818.399999999992	62.3457544288332\\
818.500000000007	62.3381382848766\\
818.700000000002	62.3473375671715\\
818.800000000012	62.3397240200269\\
819.100000000005	62.353515625\\
819.200000000001	62.3459050408886\\
819.399999999999	62.3550945698597\\
819.499999999992	62.3474865788189\\
819.700000000012	62.356672359112\\
819.80000000001	62.3490669593853\\
819.899999999993	62.3536585365854\\
819.99999999999	62.3460553591026\\
820.100000000004	62.3506461838576\\
820.200000000013	62.3430452273558\\
820.299999999995	62.3476352998537\\
820.400000000011	62.3400365630713\\
820.499999999989	62.3446258834999\\
820.699999999997	62.3294346978557\\
820.799999999994	62.3340236325984\\
820.899999999995	62.326431181486\\
821.000000000012	62.3310193642674\\
821.100000000004	62.3234291281052\\
821.199999999989	62.3280165591136\\
821.300000000009	62.3204285366448\\
821.4	62.3250152160682\\
821.799999999993	62.2946830514661\\
822.300000000013	62.3176070038911\\
822.500000000001	62.302455628495\\
822.700000000007	62.311618862421\\
822.800000000005	62.3040466642362\\
822.899999999991	62.3086269744836\\
823	62.3010569797109\\
823.100000000013	62.3056365403304\\
823.199999999988	62.2980687477226\\
823.600000000001	62.3163773218405\\
823.800000000006	62.3012501517175\\
824.699999999996	62.3423860329777\\
825.099999999991	62.3121667474552\\
825.400000000002	62.3258631132647\\
825.600000000005	62.3107666222599\\
825.79999999999	62.3198934495702\\
826.100000000003	62.2972645848463\\
826.400000000001	62.3109497882638\\
826.600000000002	62.295875166324\\
827.199999999995	62.3232201136226\\
827.300000000013	62.3156876963984\\
827.699999999995	62.3338970765886\\
827.799999999999	62.3263679188308\\
827.899999999999	62.3309178743961\\
828.000000000001	62.3233908948195\\
828.099999999987	62.3279401110843\\
828.199999999994	62.3204153084631\\
828.700000000007	62.3431467181467\\
828.799999999987	62.3356255278079\\
828.999999999988	62.3447111325534\\
829.299999999994	62.3221605980227\\
829.40000000001	62.3267028330319\\
829.499999999994	62.3191899710704\\
829.699999999991	62.3282718727404\\
829.8	62.3207615375347\\
829.9	62.3253012048193\\
830.1	62.3102866779089\\
830.700000000013	62.3375060182956\\
830.799999999992	62.3300036105428\\
831.300000000005	62.3526581669473\\
831.499999999996	62.3376623376623\\
831.599999999995	62.3421906937598\\
831.700000000004	62.3346958403462\\
831.799999999988	62.3392234643587\\
831.999999999994	62.324239875015\\
832.200000000005	62.3332932836718\\
832.600000000009	62.3033505464153\\
832.800000000013	62.3124024492736\\
832.899999999989	62.3049219687875\\
832.99999999999	62.3094466450606\\
833.099999999997	62.3019683149304\\
833.199999999988	62.3064922596904\\
833.300000000001	62.2990160787137\\
833.999999999997	62.3306557966671\\
834.100000000007	62.3231838887557\\
834.199999999998	62.3276998681529\\
834.399999999999	62.3127621330138\\
834.599999999998	62.3217922606925\\
834.700000000004	62.3143267848587\\
834.899999999988	62.3233532934132\\
835.099999999988	62.3084291187739\\
835.200000000011	62.3129414581587\\
835.500000000006	62.2905696505505\\
835.699999999998	62.2995932041158\\
835.8	62.2921402081589\\
836.000000000001	62.3011601483076\\
836.099999999991	62.2937096388424\\
836.30000000001	62.3027259684362\\
836.400000000008	62.2952779438135\\
836.699999999995	62.3087954110899\\
836.800000000007	62.3013502210539\\
837.09999999999	62.3148590539895\\
837.200000000008	62.3074166965245\\
837.99999999999	62.3433957761604\\
838.09999999999	62.3359580052494\\
838.199999999998	62.3404509125611\\
838.299999999994	62.3330152671756\\
838.40000000001	62.3375074537865\\
838.799999999999	62.3077840028609\\
838.9	62.3122765196663\\
839.100000000006	62.2974261201144\\
839.200000000013	62.301918265221\\
839.800000000004	62.2574115966187\\
839.999999999987	62.266396857517\\
840.099999999993	62.2589859557248\\
841.300000000013	62.3128119800333\\
841.399999999989	62.3054070112894\\
841.599999999996	62.3143637875728\\
841.699999999995	62.3069612734616\\
842.200000000011	62.3293363409712\\
842.399999999995	62.3145400593472\\
842.499999999987	62.3190125801092\\
842.599999999999	62.311617420197\\
842.699999999997	62.3160892263882\\
842.800000000005	62.3086961679915\\
842.899999999992	62.3131672597865\\
843.200000000002	62.2909996442547\\
843.499999999995	62.3044096728307\\
843.599999999993	62.2970250088894\\
843.799999999992	62.3059604218509\\
843.999999999997	62.291197725388\\
844.199999999995	62.3001302854436\\
844.299999999991	62.2927522501184\\
844.500000000011	62.3016812692399\\
844.700000000008	62.2869318181818\\
844.800000000009	62.291395431412\\
845.100000000001	62.2692853762423\\
845.699999999999	62.2960510759045\\
845.800000000008	62.2886866059818\\
845.900000000004	62.2931442080378\\
845.999999999988	62.2857818224796\\
846.400000000001	62.3036030714708\\
846.500000000014	62.2962437987243\\
846.900000000004	62.3140495867769\\
847.200000000002	62.2919863094536\\
847.49999999999	62.3053327041057\\
847.7	62.2906345836282\\
847.80000000001	62.2950819672131\\
847.999999999999	62.2803914632708\\
848.09999999999	62.2848384814902\\
848.299999999991	62.2701555869873\\
848.399999999991	62.2746022392457\\
848.69999999999	62.2525918944392\\
848.999999999987	62.2659286303144\\
849.099999999989	62.2585963259538\\
849.200000000003	62.2630401507124\\
};
\addplot [color=mycolor3, forget plot]
  table[row sep=crcr]{%
849.200000000003	62.2630401507124\\
849.600000000009	62.2337295516065\\
849.800000000005	62.2426167784445\\
849.99999999999	62.2279731796259\\
850.200000000005	62.2368575796778\\
850.299999999989	62.2295390404516\\
850.399999999996	62.2339800117578\\
850.499999999988	62.2266635316247\\
850.600000000009	62.2311037968732\\
850.699999999996	62.2237893747062\\
851.099999999996	62.2415413533835\\
851.300000000005	62.2269203664553\\
851.6	62.24022543149\\
851.700000000008	62.2329185254755\\
851.799999999998	62.2373518018547\\
851.999999999992	62.222743809412\\
852.800000000011	62.258177981006\\
853.399999999987	62.2144112478032\\
853.500000000005	62.2188378631678\\
853.600000000002	62.2115497247277\\
853.699999999997	62.2159756383228\\
853.900000000013	62.2014051522248\\
854.100000000006	62.2102552095528\\
854.599999999993	62.1738621738622\\
854.999999999993	62.1915565430944\\
855.199999999993	62.1770139132468\\
855.69999999999	62.1991119420425\\
855.799999999991	62.1918448416871\\
856.699999999988	62.2315592903828\\
856.800000000005	62.2242968841172\\
857.100000000013	62.2375174988334\\
857.300000000006	62.2229997667366\\
857.69999999999	62.2406155280951\\
857.799999999989	62.2333605315305\\
858.000000000001	62.2421629180748\\
858.099999999998	62.2349102773246\\
858.200000000007	62.2393102644763\\
858.300000000008	62.2320596458527\\
858.6	62.2452544544078\\
858.700000000011	62.2380065207266\\
858.800000000011	62.2424030736989\\
859.000000000008	62.2279129321383\\
859.599999999987	62.2542747470048\\
859.700000000003	62.2470341939986\\
859.900000000003	62.2558139534884\\
860.000000000005	62.2485757470062\\
860.700000000013	62.2792750929368\\
860.799999999993	62.2720408874434\\
861.099999999991	62.2851834649327\\
861.199999999993	62.2779519331243\\
861.399999999992	62.2867092280905\\
861.500000000009	62.2794800371402\\
861.69999999999	62.2882339289858\\
861.900000000005	62.2737819025522\\
862.8	62.3131301425426\\
862.899999999998	62.305909617613\\
863.200000000011	62.3190084559249\\
863.499999999989	62.2973598888374\\
863.699999999993	62.3060893725399\\
863.800000000003	62.2988771848594\\
864.000000000002	62.3076032866566\\
864.100000000002	62.3003934274474\\
864.899999999995	62.3352601156069\\
864.999999999999	62.3280545601665\\
865.39999999999	62.3454650491046\\
865.499999999999	62.3382624768946\\
865.799999999994	62.3513107749163\\
865.900000000007	62.3441108545035\\
866.199999999999	62.3571511023895\\
866.400000000014	62.3427582227351\\
866.499999999986	62.3471036233556\\
866.700000000005	62.332718043378\\
867.100000000003	62.3500922509225\\
867.300000000009	62.3357159326724\\
867.799999999992	62.3574144486692\\
867.900000000003	62.3502304147465\\
868.300000000009	62.3675725472133\\
868.400000000011	62.3603914795625\\
868.499999999995	62.3647248445775\\
868.600000000009	62.3575457580292\\
868.80000000001	62.3662101507653\\
868.900000000011	62.3590333716916\\
869.300000000007	62.3763515067863\\
869.399999999987	62.3691776883266\\
869.600000000004	62.3778314361274\\
869.700000000013	62.3706599218211\\
869.799999999986	62.3749856305323\\
869.900000000013	62.367816091954\\
870.00000000001	62.3721411332031\\
870.100000000012	62.3649735692944\\
870.199999999995	62.369297943238\\
870.299999999995	62.3621323529412\\
870.399999999991	62.3664560597358\\
870.600000000002	62.352130469737\\
870.799999999994	62.3607762085199\\
870.899999999997	62.353616532721\\
871.39999999999	62.3752151462995\\
871.899999999987	62.3394495412844\\
872.100000000012	62.3480853015364\\
872.9	62.290950744559\\
873.100000000014	62.2995877233165\\
873.200000000008	62.2924539104546\\
873.30000000001	62.2967712388367\\
873.399999999996	62.2896393817974\\
874.199999999999	62.32414503031\\
874.400000000012	62.3098913664951\\
874.899999999992	62.3314285714286\\
875.100000000006	62.3171846435101\\
875.300000000014	62.3257939227782\\
875.599999999993	62.3044421605573\\
875.899999999992	62.3173515981735\\
876.000000000013	62.3102385572423\\
876.100000000008	62.3145400593472\\
876.299999999991	62.3003194888179\\
876.400000000014	62.3046206503138\\
876.5	62.2975131188684\\
876.900000000008	62.3147092360319\\
876.999999999998	62.3076046060882\\
877.099999999999	62.311901504788\\
877.199999999995	62.3047988145446\\
877.399999999993	62.3133903133903\\
877.499999999994	62.306289881495\\
877.799999999987	62.3191707483768\\
877.999999999987	62.3049766541396\\
878.300000000001	62.3178506375228\\
878.400000000009	62.3107569721116\\
878.600000000005	62.319335381814\\
878.700000000015	62.3122439690487\\
878.800000000008	62.3165320286722\\
879.099999999992	62.2952684258417\\
879.300000000011	62.3038435296793\\
879.40000000001	62.2967595224559\\
879.499999999997	62.3010459299682\\
879.900000000004	62.2727272727273\\
880.000000000008	62.2770139756846\\
880.199999999987	62.2628649324094\\
880.800000000011	62.2885685094789\\
881.099999999991	62.2673626872447\\
881.300000000007	62.2759246653052\\
881.400000000012	62.2688598979013\\
881.499999999994	62.2731397459165\\
881.600000000003	62.2660768969037\\
882.000000000001	62.2831878471829\\
882.100000000014	62.2761278621628\\
882.599999999998	62.2974963181149\\
882.700000000001	62.2904395106479\\
883.600000000002	62.3288446305307\\
883.699999999992	62.3217922606925\\
884.100000000013	62.3388373671115\\
884.300000000013	62.3247399366802\\
884.399999999988	62.3289994347089\\
884.499999999988	62.321953425277\\
884.799999999994	62.3347270878065\\
884.900000000009	62.3276836158192\\
885.099999999988	62.336195210122\\
885.300000000011	62.3221142986221\\
885.59999999999	62.3348763689737\\
885.799999999992	62.3208037024495\\
885.899999999998	62.3250564334086\\
886.200000000012	62.3039602843281\\
886.299999999999	62.3082129963899\\
886.499999999992	62.2941574554478\\
886.700000000012	62.3026612539468\\
886.90000000001	62.2886133032694\\
887.199999999996	62.3013636875916\\
887.4	62.287323943662\\
887.600000000014	62.2958206601329\\
887.799999999995	62.2817884896948\\
888.300000000002	62.3030166591625\\
888.499999999994	62.288993923025\\
888.800000000013	62.3017212284846\\
888.90000000001	62.2947131608549\\
889.000000000005	62.2989539984254\\
889.099999999997	62.2919478182636\\
889.300000000005	62.3004272543288\\
889.399999999996	62.2934232715009\\
889.499999999992	62.2976618705036\\
889.699999999986	62.2836592492695\\
889.799999999995	62.2878975165749\\
890.000000000012	62.2739018087855\\
890.599999999991	62.2993151453913\\
890.699999999992	62.2923215087562\\
890.799999999999	62.2965540464699\\
890.9	62.2895622895623\\
891.099999999986	62.2980251346499\\
891.200000000007	62.2910355660272\\
891.499999999993	62.3037236428892\\
891.800000000008	62.2827671263595\\
892.399999999994	62.3081232492997\\
892.500000000004	62.301142729106\\
892.600000000015	62.305365744371\\
892.800000000013	62.2914100123194\\
892.900000000006	62.2956326987682\\
893.200000000004	62.2747117429755\\
893.500000000009	62.2873769024172\\
893.700000000005	62.2734392481539\\
893.899999999989	62.2818791946309\\
894.099999999997	62.2679490046969\\
894.200000000014	62.2721681762272\\
894.299999999989	62.2652057245081\\
894.399999999998	62.2694242593628\\
894.600000000007	62.2555046384263\\
894.800000000004	62.2639401050397\\
894.900000000003	62.2569832402235\\
894.999999999998	62.2611998659368\\
895.20000000001	62.2472914107003\\
895.600000000005	62.2641509433962\\
895.700000000014	62.2572002679169\\
895.800000000009	62.2614131041411\\
895.899999999991	62.2544642857143\\
895.999999999986	62.2586764869992\\
896.199999999995	62.2447841124624\\
896.300000000011	62.2489959839357\\
896.499999999994	62.2351104171314\\
897.10000000001	62.2603655818101\\
897.200000000013	62.2534269475092\\
897.300000000014	62.2576331624694\\
897.399999999993	62.2506963788301\\
897.499999999994	62.2549019607843\\
897.599999999998	62.2479670268464\\
897.699999999991	62.2521719759412\\
897.799999999987	62.2452388907451\\
898.100000000006	62.2578490313961\\
898.299999999997	62.2439893143366\\
898.50000000001	62.2523926107278\\
898.60000000001	62.2454656726383\\
898.700000000005	62.2496662216288\\
898.79999999999	62.2427411280454\\
898.899999999987	62.2469410456062\\
898.999999999992	62.2400177955734\\
899.100000000002	62.2442170818505\\
899.399999999995	62.2234574763758\\
899.599999999992	62.2318550627987\\
899.800000000008	62.2180242249139\\
900.600000000001	62.2515821028089\\
900.800000000012	62.2377622377622\\
900.899999999987	62.2419533851276\\
901.099999999996	62.2281402574345\\
901.300000000002	62.2365209673841\\
901.399999999999	62.2296173044925\\
901.500000000003	62.2338065661047\\
901.600000000001	62.2269047354996\\
902.500000000014	62.264569022823\\
902.600000000009	62.257671430154\\
902.699999999998	62.2618520159504\\
902.800000000008	62.2549562520767\\
902.900000000006	62.2591362126246\\
903.200000000008	62.2384589837263\\
903.400000000005	62.2468179302712\\
903.499999999989	62.2399291722001\\
903.599999999998	62.2441075578179\\
903.700000000006	62.2372206240319\\
903.799999999988	62.2413983847771\\
903.90000000001	62.2345132743363\\
904.099999999993	62.2428666224287\\
904.299999999987	62.2291021671827\\
904.500000000009	62.2374530179085\\
904.600000000006	62.2305736708301\\
905.100000000012	62.2514361467079\\
905.400000000008	62.2308117062396\\
905.700000000005	62.2433208213734\\
905.800000000003	62.2364499392869\\
906.100000000001	62.2489516662988\\
906.199999999992	62.2420831954099\\
906.5	62.2545775424664\\
906.600000000005	62.2477114811956\\
906.900000000009	62.2601984564498\\
907.000000000009	62.2533348032191\\
907.499999999987	62.2741295724989\\
907.599999999998	62.2672689214498\\
907.70000000001	62.2714254241022\\
907.800000000002	62.2645665822227\\
908.200000000006	62.2811846306287\\
908.399999999996	62.2674738580077\\
908.799999999992	62.2840796567279\\
908.899999999991	62.2772277227723\\
909.000000000005	62.2813771862281\\
909.200000000006	62.2676784339602\\
909.399999999999	62.2759758108851\\
909.499999999993	62.2691292875989\\
909.699999999996	62.2774236095845\\
909.899999999988	62.2637362637363\\
910.700000000006	62.2968818620993\\
910.80000000001	62.2900428147986\\
910.999999999992	62.2983207112282\\
911.100000000008	62.2914837576822\\
911.199999999998	62.2956216394162\\
911.400000000013	62.2819528250137\\
911.499999999992	62.2860903905222\\
911.600000000004	62.2792585280246\\
911.899999999995	62.2916666666667\\
911.999999999993	62.2848371889047\\
912.300000000015	62.2972380534853\\
912.500000000013	62.2835853605084\\
912.599999999987	62.2877177604908\\
912.699999999987	62.2808939526731\\
912.800000000001	62.2850257421404\\
912.899999999988	62.2782037239869\\
913.200000000012	62.2905945472463\\
913.300000000001	62.2837749069411\\
913.400000000008	62.287903667214\\
913.499999999996	62.2810858143608\\
914.100000000008	62.3058411726099\\
914.200000000015	62.2990265777097\\
914.300000000009	62.3031496062992\\
914.399999999991	62.2963367960634\\
914.600000000011	62.3045807368536\\
914.800000000006	62.2909607607389\\
915.10000000001	62.3033216783217\\
915.200000000002	62.2965148038894\\
915.299999999991	62.3006336027966\\
915.499999999994	62.2870249017038\\
915.800000000002	62.2993776613167\\
916.000000000003	62.2857766619365\\
916.399999999989	62.3022367703219\\
916.599999999986	62.2886440493073\\
916.699999999999	62.292757417103\\
916.800000000006	62.2859635729087\\
917.200000000004	62.302409244522\\
917.300000000001	62.2956180510137\\
917.50000000001	62.3038360941587\\
917.600000000012	62.2970469652392\\
918.099999999998	62.3175778697452\\
918.199999999992	62.3107916802788\\
918.7	62.3313016978668\\
918.79999999999	62.324518445968\\
918.89999999999	62.3286180631121\\
919.199999999992	62.3082780376373\\
919.600000000009	62.3246710883984\\
919.799999999999	62.3111207739972\\
920.200000000009	62.3275019015538\\
920.400000000002	62.3139598044541\\
920.50000000001	62.3180534434065\\
920.6	62.3112848919301\\
920.699999999991	62.3153779322328\\
920.800000000005	62.3086111412748\\
920.900000000001	62.3127035830619\\
921.000000000001	62.3059385517316\\
921.199999999987	62.3141213502659\\
921.299999999989	62.3073583677013\\
921.600000000002	62.3196267766084\\
921.699999999986	62.3128661314819\\
922.099999999992	62.3292127521145\\
922.200000000009	62.3224547327334\\
922.400000000005	62.3306233062331\\
922.70000000001	62.310359774599\\
922.900000000006	62.3185265438787\\
923.000000000014	62.3117755389449\\
923.099999999999	62.3158578856153\\
923.199999999999	62.3091086320806\\
923.700000000012	62.3295085516345\\
923.799999999995	62.3227622037017\\
924.100000000009	62.3349924258819\\
924.400000000014	62.3147647376961\\
924.600000000009	62.3229155401752\\
924.700000000002	62.3161764705882\\
925	62.3283969300616\\
925.200000000012	62.3149248892251\\
925.799999999987	62.339345501674\\
925.900000000008	62.3326133909287\\
926.500000000001	62.3570041010145\\
926.600000000008	62.3502751699579\\
926.699999999992	62.3543375053949\\
926.799999999989	62.3476103139497\\
927.2	62.3638520435673\\
927.299999999989	62.3571274530947\\
927.500000000013	62.3652436394998\\
927.60000000001	62.3585210736229\\
927.800000000005	62.3666343355965\\
928.000000000011	62.3531946988471\\
928.199999999998	62.3613056124098\\
928.300000000005	62.3545885394227\\
928.400000000006	62.3586429725364\\
928.599999999997	62.3452137396361\\
929.300000000006	62.3735743490424\\
929.800000000012	62.3400365630713\\
930.000000000013	62.3481346091818\\
930.100000000001	62.3414319501183\\
930.999999999993	62.3778326710343\\
931.199999999994	62.364436808762\\
931.300000000005	62.3684775606614\\
931.400000000004	62.361782071927\\
931.500000000002	62.3658222413053\\
931.60000000001	62.359128474831\\
931.70000000001	62.3631680618158\\
931.800000000014	62.35647601674\\
931.999999999998	62.3645531595322\\
932.100000000007	62.3578631195023\\
932.199999999987	62.3619006757482\\
932.300000000015	62.3552123552124\\
932.50000000001	62.3632854385589\\
932.700000000016	62.3499142367067\\
932.900000000008	62.3579849946409\\
933.000000000009	62.3513021112421\\
933.399999999987	62.3674343867167\\
933.500000000004	62.3607540702656\\
933.600000000013	62.3647852629324\\
933.899999999994	62.3447537473234\\
934.400000000014	62.3649010165864\\
934.499999999998	62.3582281189814\\
934.900000000001	62.3743315508021\\
934.999999999997	62.3676612127045\\
935.100000000006	62.3716852010265\\
935.300000000001	62.3583493692538\\
935.699999999992	62.3744389826886\\
935.900000000005	62.3611111111111\\
936.100000000009	62.3691518906217\\
936.399999999995	62.349172450614\\
936.600000000006	62.3572114871357\\
936.699999999992	62.3505550811273\\
937.10000000001	62.3666239863423\\
937.299999999994	62.35331768722\\
937.5	62.3613481228669\\
937.599999999998	62.3546976644982\\
937.999999999994	62.370749387059\\
938.199999999986	62.3574549717574\\
938.299999999991	62.3614663256607\\
938.39999999999	62.3548215237081\\
938.699999999999	62.3668512995313\\
938.799999999992	62.360208754926\\
939	62.3682248961772\\
939.099999999993	62.3615843270869\\
939.600000000012	62.3816111524955\\
939.699999999996	62.3749733985954\\
939.799999999989	62.3789764868603\\
939.999999999986	62.3657057759813\\
940.300000000006	62.37771161208\\
940.700000000001	62.3511904761905\\
940.899999999986	62.3591923485654\\
941.000000000004	62.3525661459994\\
941.100000000013	62.3565660858479\\
941.299999999994	62.3433184618653\\
941.900000000008	62.3673036093418\\
942.000000000011	62.3606835792379\\
942.300000000008	62.3726655348048\\
942.400000000005	62.3660477453581\\
942.500000000003	62.370040314025\\
942.600000000002	62.3634242070648\\
942.800000000003	62.3714073602715\\
942.900000000008	62.3647932131495\\
942.999999999989	62.3687837981126\\
943.400000000011	62.3423423423423\\
943.499999999989	62.3463331920305\\
943.600000000002	62.3397266080322\\
944.000000000009	62.3556826607351\\
944.100000000015	62.3490785850455\\
944.500000000008	62.3650222316324\\
944.800000000005	62.3452217165838\\
946.200000000012	62.4009299376519\\
946.300000000014	62.394336432798\\
946.400000000001	62.3983095615425\\
946.500000000002	62.3917177266005\\
946.700000000001	62.3996620194339\\
946.899999999996	62.3864836325238\\
947.100000000013	62.3944256756757\\
947.200000000001	62.3878391217144\\
947.500000000003	62.3997467285775\\
947.699999999995	62.3865794471407\\
947.900000000015	62.3945147679325\\
948.000000000009	62.3879337622614\\
948.200000000005	62.39586628704\\
948.399999999996	62.3827095413811\\
948.600000000002	62.3906398229156\\
948.699999999997	62.3840640809444\\
949.099999999992	62.3999157184998\\
949.399999999991	62.3802001053186\\
949.499999999998	62.3841617523168\\
949.700000000007	62.3710254790482\\
949.899999999994	62.3789473684211\\
950.100000000001	62.3658177225847\\
950.599999999994	62.3856106027138\\
950.700000000004	62.379049221708\\
951.4	62.4067262217551\\
951.800000000016	62.3805021535876\\
952.200000000002	62.3963036858133\\
952.299999999986	62.3897522049559\\
952.399999999989	62.3937007874016\\
952.499999999989	62.3871509552803\\
952.70000000001	62.3950461796809\\
952.799999999987	62.3884982684437\\
953.199999999991	62.4042798699255\\
953.299999999997	62.3977344241661\\
953.499999999994	62.4056208053691\\
953.600000000014	62.39907727797\\
953.699999999986	62.4030195009436\\
954.100000000012	62.3768601970237\\
954.299999999993	62.384744341995\\
954.499999999988	62.371673999581\\
954.800000000006	62.3834956539952\\
955.000000000005	62.3704324154539\\
955.299999999986	62.3822482729747\\
955.4	62.3757195185767\\
955.599999999994	62.3835931777755\\
955.80000000001	62.3705408515535\\
955.899999999994	62.3744769874477\\
956.000000000015	62.3679531429767\\
956.100000000009	62.3718887262079\\
956.200000000012	62.3653665167834\\
956.499999999998	62.3771691407067\\
956.599999999999	62.3706491063029\\
956.799999999994	62.3785139513011\\
957.200000000013	62.3524495978272\\
957.400000000013	62.3603133159269\\
957.600000000009	62.3472903832098\\
957.699999999987	62.351221549384\\
957.899999999999	62.3382045929019\\
957.999999999997	62.3421354764638\\
958.10000000001	62.3356293049468\\
958.400000000011	62.3474178403756\\
958.499999999992	62.3409138326727\\
958.600000000008	62.3448419735058\\
958.699999999986	62.3383395911556\\
958.800000000006	62.3422671811451\\
958.899999999995	62.3357664233577\\
959.000000000006	62.3396934626212\\
959.099999999997	62.3331943286072\\
960.000000000001	62.3685032809082\\
960.200000000012	62.3555139019057\\
960.400000000001	62.3633524206143\\
960.500000000009	62.3568602956486\\
960.699999999989	62.3646960865945\\
960.899999999986	62.3517169614984\\
961.200000000014	62.3634661396026\\
961.400000000001	62.3504940197608\\
961.600000000004	62.3583238016013\\
961.699999999994	62.3518402994386\\
961.999999999993	62.363579669473\\
962.599999999994	62.3247117482082\\
962.700000000012	62.3286248442044\\
962.799999999999	62.3221518330045\\
962.900000000001	62.3260643821392\\
963.000000000005	62.3195929809989\\
963.300000000009	62.3313265517957\\
963.599999999991	62.3119227975511\\
963.799999999986	62.3197427118996\\
964.00000000001	62.3068146457836\\
964.09999999999	62.3107239162\\
964.400000000005	62.2913426645931\\
964.80000000001	62.3069748160431\\
965.100000000003	62.2876087857439\\
965.200000000013	62.291515591008\\
965.299999999991	62.285063186244\\
965.700000000012	62.3006833712984\\
965.799999999994	62.2942333574904\\
966.900000000006	62.3371251292658\\
967.100000000011	62.3242349048801\\
967.500000000006	62.3398098387763\\
967.599999999997	62.3333677792704\\
967.800000000003	62.3411509453456\\
967.899999999988	62.3347107438017\\
967.999999999986	62.3386013841545\\
968.200000000003	62.325725498296\\
968.399999999992	62.3335054207538\\
968.500000000006	62.3270699979352\\
968.599999999999	62.3309590172396\\
968.699999999994	62.3245251857969\\
969.000000000012	62.3361882158704\\
969.200000000003	62.3233261116269\\
969.900000000007	62.3505154639175\\
970.20000000001	62.3312377615171\\
970.700000000009	62.3506386485373\\
970.799999999994	62.3442167061489\\
971.30000000001	62.3635989293803\\
971.400000000016	62.3571796191457\\
971.499999999988	62.3610539316591\\
971.599999999999	62.3546362045899\\
971.699999999997	62.3585099814777\\
972.000000000002	62.3392655076638\\
972.100000000009	62.3431392717548\\
972.39999999999	62.3239074550129\\
972.700000000012	62.3355263157895\\
972.799999999993	62.3291191283791\\
973.599999999985	62.3600698367053\\
973.799999999998	62.3472635794229\\
974.29999999999	62.3665845648604\\
974.499999999993	62.3537861686846\\
974.999999999992	62.3730899394934\\
975.100000000016	62.3666940114848\\
975.199999999999	62.3705526504665\\
975.300000000009	62.3641582940332\\
975.399999999985	62.3680164018452\\
975.499999999995	62.3616236162362\\
976.000000000009	62.380903595943\\
976.099999999993	62.3745134193813\\
976.399999999986	62.3860727086534\\
976.499999999998	62.3796846201106\\
976.80000000001	62.3912375882895\\
976.900000000004	62.3848515864893\\
977.100000000013	62.3925501432665\\
977.199999999992	62.3861659674614\\
977.299999999997	62.390014323716\\
977.400000000011	62.383631713555\\
977.999999999985	62.4067068806871\\
978.199999999991	62.393948686497\\
978.600000000015	62.4093184837029\\
978.700000000012	62.4029423784226\\
978.99999999999	62.4144622612603\\
979.099999999991	62.4080882352941\\
979.300000000014	62.415764753931\\
979.399999999997	62.409392547218\\
979.5	62.4132298897509\\
979.699999999998	62.4004898958971\\
979.900000000003	62.4081632653061\\
980.000000000001	62.4017957351291\\
980.199999999996	62.40946648985\\
980.299999999998	62.4031007751938\\
980.399999999998	62.4069352371239\\
980.600000000004	62.3942082186194\\
980.69999999999	62.3980424143556\\
981.000000000013	62.378962389155\\
981.100000000014	62.3827965756217\\
981.300000000009	62.3700835541064\\
981.399999999989	62.3739174732552\\
981.600000000001	62.3612101456657\\
981.799999999998	62.3688766676851\\
981.900000000002	62.3625254582485\\
982.200000000009	62.3740201567749\\
982.299999999986	62.367671009772\\
982.90000000001	62.3906408952187\\
983	62.3842945783745\\
983.1	62.3881204231082\\
983.299999999989	62.3754321740899\\
983.700000000014	62.3907298231348\\
983.800000000013	62.3843886573839\\
984.300000000015	62.403494514425\\
984.500000000007	62.3908186065407\\
984.90000000001	62.4060913705584\\
984.999999999991	62.3997563699117\\
985.200000000015	62.4073886126053\\
985.299999999997	62.401055408971\\
985.599999999994	62.4124987318657\\
985.800000000012	62.3998377117355\\
986.099999999991	62.4112756033259\\
986.30000000001	62.3986212489862\\
986.499999999997	62.4062436651125\\
986.600000000008	62.3999189216581\\
986.79999999999	62.4075387577262\\
987.000000000002	62.3948941343329\\
987.099999999987	62.3987034035656\\
987.199999999987	62.3923832674972\\
987.299999999986	62.396192019445\\
987.399999999998	62.3898734177215\\
987.500000000009	62.3936816524909\\
987.600000000001	62.387364584388\\
987.699999999992	62.3911723020854\\
987.900000000014	62.3785425101215\\
988.099999999998	62.3861566484517\\
988.20000000001	62.3798441768694\\
988.399999999986	62.3874557410218\\
988.499999999994	62.3811450536112\\
989.499999999989	62.4191592562652\\
989.599999999992	62.4128523795089\\
989.699999999987	62.4166498282481\\
989.800000000006	62.4103444792403\\
990.099999999994	62.4217329832357\\
990.200000000001	62.4154296677774\\
990.500000000001	62.4268120331112\\
990.600000000014	62.4205107499748\\
990.999999999993	62.4356775300172\\
991.199999999994	62.423080802986\\
991.40000000001	62.4306606152295\\
991.700000000014	62.4117765678564\\
992.499999999986	62.4420713278259\\
992.60000000001	62.4357812027803\\
992.70000000001	62.4395648670427\\
992.900000000006	62.4269889224572\\
994.399999999998	62.483660130719\\
994.500000000014	62.4773778403378\\
994.600000000004	62.4811500955062\\
994.799999999986	62.4685898080209\\
995.099999999985	62.4799035369775\\
995.299999999991	62.467349809122\\
995.800000000015	62.4861933929109\\
996.099999999998	62.4673760289099\\
996.400000000016	62.4786753637732\\
996.500000000006	62.4724061810155\\
996.799999999998	62.4836994683519\\
997.000000000015	62.4711663825093\\
997.2	62.4786924696681\\
997.300000000013	62.4724283136154\\
997.400000000011	62.4761904761905\\
997.500000000008	62.4699278267843\\
997.799999999996	62.4812105421385\\
998.199999999998	62.4561754983472\\
998.499999999992	62.4674544362107\\
998.600000000006	62.4611995594273\\
998.69999999999	62.4649579495395\\
998.799999999996	62.4587045750325\\
999.000000000001	62.4662195976379\\
999.100000000011	62.4599679743795\\
999.200000000001	62.4637246072251\\
999.399999999986	62.4512256128064\\
999.599999999989	62.4587376212864\\
999.799999999987	62.4462446244625\\
999.900000000015	62.45\\
1000	62.4437556244376\\
1000.1	62.4475104979004\\
1000.19999999999	62.4412676197141\\
1000.4	62.4487756121939\\
1000.6	62.4362945937844\\
1001.29999999999	62.4625524266028\\
1001.39999999999	62.4563155267099\\
1001.50000000002	62.4600638977636\\
1001.6	62.4538284915644\\
1001.70000000002	62.4575763625474\\
1001.8	62.4513424493463\\
1001.9	62.4550898203593\\
1001.99999999999	62.4488573994611\\
1002.40000000002	62.4638403990025\\
1002.60000000001	62.4513812705695\\
1003	62.4663543016648\\
1003.3	62.4476778951565\\
1003.70000000002	62.4626419605499\\
1004	62.4439796832985\\
1004.40000000002	62.4589347934296\\
1004.49999999999	62.4527174995023\\
1004.60000000001	62.4564546630835\\
1004.70000000001	62.4502388535032\\
1004.80000000001	62.4539755199522\\
1004.9	62.4477611940299\\
1005.20000000001	62.4589674723963\\
1005.4	62.4465440079562\\
1005.59999999999	62.4540121308541\\
1005.79999999999	62.4415945919077\\
1006.00000000001	62.4490607295497\\
1006.1	62.4428543033194\\
1006.2	62.4465865050184\\
1006.29999999999	62.4403815580286\\
1006.69999999999	62.4553039332539\\
1006.89999999999	62.4428997020854\\
1006.99999999999	62.4466289345646\\
1007.19999999999	62.4342301201231\\
1007.50000000001	62.4454148471616\\
1007.60000000001	62.4392180212365\\
1007.80000000001	62.4466712967556\\
1007.89999999999	62.4404761904762\\
1007.99999999998	62.4442019640909\\
1008.10000000001	62.4380083316802\\
1008.3	62.4454581515272\\
1008.7	62.4206978588422\\
1009.1	62.4355925485533\\
1009.19999999999	62.4294065193699\\
1009.5	62.4405705229794\\
1009.69999999998	62.4282036046742\\
1009.8	62.4319239528666\\
1010.10000000001	62.4133834884181\\
1010.3	62.4208234362629\\
1010.4	62.4146462147452\\
1010.49999999999	62.4183653275282\\
1010.8	62.3998417251954\\
1011.10000000001	62.410996835443\\
1011.29999999999	62.3986553292466\\
1011.40000000001	62.4023727137914\\
1011.49999999999	62.3962040332147\\
1012.00000000001	62.4147811481079\\
1012.09999999999	62.4086148982415\\
1012.99999999999	62.44200967328\\
1013.10000000001	62.4358468219503\\
1013.30000000001	62.4432603118216\\
1013.40000000001	62.4370991613222\\
1013.69999999999	62.4482146379957\\
1013.89999999999	62.4358974358974\\
1014.4	62.4544110399211\\
1014.80000000002	62.4297960390186\\
1015.19999999999	62.4445976558653\\
1015.40000000001	62.4322993599212\\
1015.49999999999	62.4359984245766\\
1016.30000000001	62.3868555686738\\
1017.50000000001	62.4312106918239\\
1017.6	62.4250761521077\\
1017.80000000001	62.4324589841831\\
1017.9	62.426326129666\\
1018.4	62.4447717231222\\
1018.50000000001	62.4386412723346\\
1018.7	62.4460149195131\\
1018.79999999999	62.4398861517323\\
1018.9	62.4435721295388\\
1018.99999999999	62.437444804239\\
1019.4	62.4521824423737\\
1019.49999999999	62.4460572773637\\
1019.69999999999	62.4534222396548\\
1019.90000000001	62.4411764705882\\
1020.99999999999	62.481637449809\\
1021.10000000001	62.4755189972581\\
1021.2	62.4791931851562\\
1021.30000000001	62.4730761699628\\
1021.39999999998	62.4767498776309\\
1021.6	62.4645199177841\\
1021.69999999999	62.4681933842239\\
1021.8	62.4620804383991\\
1021.9	62.4657534246575\\
1022	62.4596419137071\\
1022.39999999999	62.4743276283619\\
1022.5	62.4682182671621\\
1022.89999999999	62.4828934506354\\
1023.09999999999	62.470680218921\\
1023.2	62.4743476986221\\
1023.3	62.468243111198\\
1024	62.4938970803633\\
1024.40000000002	62.4694973157638\\
1024.69999999999	62.4804839968775\\
1024.79999999999	62.4743877451459\\
1025.1	62.4853687085447\\
1025.29999999999	62.4731811975814\\
1025.50000000001	62.4804992199688\\
1025.79999999999	62.4622282873574\\
1026.19999999999	62.4768586183377\\
1026.29999999999	62.4707716289946\\
1026.49999999999	62.4780829924021\\
1026.70000000001	62.465913517725\\
1027	62.4768766429754\\
1027.1	62.4707943925234\\
1027.20000000001	62.4744475810377\\
1027.30000000002	62.468366751022\\
1027.49999999999	62.4756714674971\\
1027.6	62.4695922934709\\
1027.69999999999	62.4732438217552\\
1028.10000000002	62.4489398949621\\
1028.39999999999	62.4598930481284\\
1028.6	62.4477495868572\\
1028.90000000001	62.4586977648202\\
1028.99999999998	62.4526285103489\\
1029.3	62.4635710122401\\
1029.40000000001	62.4575036425449\\
1029.70000000001	62.4684404738784\\
1029.8	62.4623749878629\\
1030.39999999998	62.4842309558467\\
1030.80000000001	62.4599864196333\\
1030.89999999999	62.4636275460718\\
1031	62.4575695858792\\
1031.1	62.4612102404965\\
1031.20000000001	62.4551536895181\\
1031.3	62.4587938724064\\
1031.40000000001	62.4527387300048\\
1031.50000000002	62.4563784412563\\
1031.6	62.4503247067946\\
1031.7	62.4539639465013\\
1031.8	62.447911619343\\
1032.20000000001	62.4624624624625\\
1032.30000000001	62.4564122433165\\
1032.49999999999	62.4636839047066\\
1032.6	62.4576353248765\\
1033.20000000001	62.4794348204781\\
1033.70000000001	62.4492164828787\\
1033.9	62.4564796905223\\
1033.99999999999	62.4504399961319\\
1034.09999999999	62.4540707793464\\
1034.20000000001	62.4480324857391\\
1034.29999999998	62.4516627996907\\
1034.50000000001	62.4395901797796\\
1035.19999999999	62.4649859943978\\
1035.30000000002	62.4589530616187\\
1035.4	62.4625784645099\\
1035.59999999999	62.4505165588491\\
1036.4	62.4794983116257\\
1036.7	62.4614197530864\\
1036.8	62.4650400231459\\
1037	62.4529939253688\\
1037.1	62.4566139606633\\
1037.2	62.4505928853755\\
1038.20000000001	62.486757199268\\
1038.30000000001	62.4807395993837\\
1038.49999999999	62.4879645676873\\
1038.6	62.4819485895831\\
1039.19999999998	62.5036081978255\\
1039.30000000001	62.4975947662113\\
1039.89999999999	62.5192307692308\\
1039.99999999999	62.5132198827036\\
1040.59999999999	62.534832324397\\
1041.20000000001	62.4987995774513\\
1041.40000000001	62.5060009601536\\
1041.6	62.4940001919939\\
1042.4	62.5227817745803\\
1042.5	62.5167849606752\\
1042.60000000002	62.520379783255\\
1042.7	62.5143843498274\\
1042.79999999999	62.5179787132036\\
1042.90000000001	62.5119846596357\\
1043.19999999998	62.5227643055689\\
1043.39999999998	62.5107810253953\\
1044.00000000001	62.5323244899914\\
1044.09999999999	62.5263359509673\\
1044.19999999999	62.5299243512401\\
1044.29999999999	62.5239371888165\\
1044.49999999999	62.5311123875168\\
1044.8	62.513159153986\\
1044.90000000001	62.5167464114833\\
1045.2	62.4988041710514\\
1045.39999999999	62.5059780009565\\
1045.50000000001	62.5\\
1045.7	62.5071715433161\\
1045.90000000001	62.4952198852773\\
1046.10000000001	62.5023896004588\\
1046.2	62.4964159418905\\
1046.30000000001	62.5\\
1046.40000000001	62.494027711419\\
1046.69999999999	62.5047764615972\\
1046.79999999998	62.4988059986627\\
1046.89999999999	62.5023877745941\\
1047.00000000001	62.4964186801643\\
1047.29999999999	62.507160588123\\
1047.49999999999	62.4952271859488\\
1047.89999999998	62.5095419847328\\
1048.00000000001	62.5035779028719\\
1048.10000000001	62.5071551230681\\
1048.2	62.5011924067538\\
1048.3	62.5047691720717\\
1048.40000000002	62.4988078206962\\
1048.5	62.5023841312226\\
1048.7	62.4904652936689\\
1048.79999999999	62.4940413766803\\
1048.90000000001	62.4880838894185\\
1049.10000000002	62.4952344643538\\
1049.20000000001	62.4892785666635\\
1049.30000000001	62.4928530588908\\
1049.70000000001	62.4690417222328\\
1049.79999999999	62.4726164396609\\
1049.89999999999	62.4666666666667\\
1050.10000000002	62.4738145115216\\
1050.19999999999	62.4678663239075\\
1050.30000000002	62.4714394516375\\
1050.4	62.4654926225607\\
1050.9	62.4833491912464\\
1051	62.4774046237275\\
1051.59999999999	62.4988114481316\\
1051.69999999998	62.4928693667998\\
1051.89999999999	62.5\\
1052.30000000001	62.4762447738502\\
1052.4	62.479809976247\\
1052.5	62.4738742162265\\
1052.59999999999	62.4774389664672\\
1052.7	62.4715045592705\\
1052.99999999999	62.4821954230367\\
1053.09999999999	62.4762628180782\\
1053.2	62.4798253109276\\
1053.4	62.467963929758\\
1053.9	62.4857685009488\\
1054	62.4798406223318\\
1054.30000000001	62.4905159332322\\
1054.39999999999	62.4845898530109\\
1054.70000000001	62.4952597648843\\
1054.8	62.4893354820362\\
1055	62.4964458345181\\
1055.30000000001	62.4786810687891\\
1055.50000000001	62.485790071997\\
1055.59999999999	62.4798711755234\\
1055.79999999999	62.4869779335164\\
1055.90000000002	62.4810606060606\\
1056.00000000001	62.4846131995076\\
1056.29999999999	62.4668686103749\\
1056.39999999999	62.4704212020823\\
1056.5	62.4645088018172\\
1056.79999999999	62.4751632131706\\
1057.1	62.4574347332577\\
1057.4	62.468085106383\\
1057.60000000001	62.4562730452869\\
1057.69999999999	62.4598222726413\\
1057.79999999999	62.4539181397108\\
1058.80000000001	62.4893757673057\\
1059	62.4775752997828\\
1059.50000000001	62.4952812382031\\
1059.6	62.4893837878645\\
1060.3	62.5141456054319\\
1060.39999999999	62.5082508250825\\
1060.49999999999	62.511785781633\\
1060.70000000001	62.5\\
1060.80000000001	62.5035347346593\\
1061	62.4917538403543\\
1061.19999999999	62.4988221991897\\
1061.29999999999	62.4929338609384\\
1061.70000000002	62.5070634771143\\
1061.80000000001	62.5011771353235\\
1062.00000000001	62.5082383956313\\
1062.20000000001	62.4964699237504\\
1062.29999999999	62.5\\
1062.39999999999	62.4941176470588\\
1062.59999999999	62.5011762491766\\
1062.99999999999	62.4776596745367\\
1063.3	62.4882452510815\\
1063.39999999999	62.4823695345557\\
1064.00000000001	62.5035241048774\\
1064.09999999999	62.4976508175155\\
1064.29999999999	62.5046974821496\\
1064.4	62.4988257397839\\
1064.89999999999	62.5164319248826\\
1065.30000000001	62.4929603904637\\
1065.70000000002	62.5070369675361\\
1065.8	62.501172717891\\
1065.89999999999	62.5046904315197\\
1066.00000000001	62.4988275021105\\
1066.50000000001	62.5164072754547\\
1066.70000000002	62.5046869141357\\
1066.90000000001	62.5117150890347\\
1067	62.5058569955956\\
1067.49999999999	62.5234170101161\\
1067.6	62.5175611126721\\
1067.70000000001	62.5210713616782\\
1067.89999999999	62.5093632958801\\
1068.39999999999	62.5269068788021\\
1068.5	62.521055586749\\
1068.60000000002	62.5245625526341\\
1068.79999999999	62.5128636916456\\
1068.99999999999	62.5198765316621\\
1069.10000000002	62.5140291806959\\
1069.20000000001	62.5175348358739\\
1069.40000000001	62.5058438522674\\
1069.59999999999	62.5128540712349\\
1069.7	62.5070106561974\\
1070	62.5175217269414\\
1070.1	62.5116800598019\\
1070.39999999999	62.5221858944418\\
1070.6	62.5105071448585\\
1070.69999999999	62.5140082181547\\
1070.8	62.5081706975441\\
1071.20000000001	62.5221693269859\\
1071.39999999999	62.5104993000467\\
1071.7	62.5209927225229\\
1072.00000000001	62.5034978080403\\
1072.5	62.5209770650755\\
1072.69999999999	62.5093214019389\\
1072.80000000002	62.5128157330599\\
1072.90000000001	62.5069897483691\\
1073.00000000001	62.510483645513\\
1073.10000000002	62.5046589638464\\
1073.4	62.5151374010247\\
1073.59999999998	62.5034925956971\\
1074	62.5174564751885\\
1074.10000000001	62.5116365667473\\
1074.4	62.5221033038623\\
1074.60000000002	62.5104680375919\\
1074.69999999999	62.513956084853\\
1074.90000000001	62.5023255813954\\
1075.00000000001	62.5058134127058\\
1075.19999999998	62.4941876685576\\
1075.40000000001	62.5011622501162\\
1075.50000000002	62.495351431759\\
1075.89999999999	62.5092936802974\\
1075.99999999999	62.5034848062448\\
1076.09999999998	62.5069689648764\\
1076.29999999999	62.4953548866592\\
1076.49999999999	62.502322125209\\
1076.60000000001	62.4965171356924\\
1077.40000000001	62.5243619489559\\
1077.49999999999	62.518559762435\\
1077.59999999999	62.5220376728218\\
1077.9	62.5046382189239\\
1078.1	62.5115933964014\\
1078.20000000001	62.5057961606232\\
1079.59999999999	62.5544132629434\\
1079.80000000001	62.5428280396333\\
1079.90000000001	62.5462962962963\\
1080	62.5405055087492\\
1080.29999999999	62.550907071455\\
1080.50000000001	62.5393300018508\\
1080.80000000001	62.5497270792858\\
1080.9	62.5439407955597\\
1080.99999999998	62.5474054204051\\
1081.1	62.5416204217536\\
1081.69999999999	62.5623960066556\\
1081.89999999999	62.550831792976\\
1082	62.5542925792441\\
1082.09999999999	62.5485122897801\\
1082.69999999999	62.5692648688585\\
1082.8	62.5634869332348\\
1082.90000000001	62.5669436749769\\
1083.00000000002	62.5611670205891\\
1083.10000000001	62.564623338257\\
1083.19999999998	62.5588479645528\\
1083.30000000001	62.5623038582241\\
1083.59999999999	62.544984774384\\
1084.1	62.5622578859989\\
1084.29999999999	62.5507192917743\\
1084.39999999999	62.5541724296911\\
1084.50000000001	62.5484049419141\\
1084.60000000001	62.5518576564949\\
1084.69999999999	62.5460914454277\\
1085	62.5564464104691\\
1085.10000000001	62.5506819019536\\
1085.3	62.55758245808\\
1085.40000000001	62.551819438047\\
1085.60000000001	62.5587178778668\\
1085.70000000001	62.5529563455517\\
1086.00000000001	62.5632998803057\\
1086.10000000001	62.5575400478733\\
1086.30000000001	62.5644329896907\\
1086.70000000002	62.5414059624586\\
1086.79999999999	62.5448523323213\\
1087.1	62.5275938189846\\
1087.19999999999	62.5310401912996\\
1087.3	62.5252896818098\\
1087.4	62.5287356321839\\
1087.60000000002	62.517238209065\\
1087.90000000002	62.5275735294118\\
1087.99999999999	62.5218270379561\\
1088.40000000001	62.5355994487827\\
1088.49999999999	62.5298548594525\\
1088.60000000002	62.533296592266\\
1088.79999999999	62.5218110019286\\
1088.89999999999	62.5252525252525\\
1088.99999999998	62.5195115232761\\
1089.2	62.526393096484\\
1089.3	62.5206535707729\\
1089.59999999999	62.5309718271084\\
1089.7	62.5252339878877\\
1089.79999999999	62.5286723552619\\
1089.90000000002	62.5229357798165\\
1090.1	62.5298110438452\\
1090.20000000001	62.5240759424012\\
1090.3	62.527512839325\\
1090.5	62.5160462130937\\
1091.10000000001	62.5366568914956\\
1091.19999999999	62.5309264180335\\
1091.39999999999	62.5377920293175\\
1091.50000000001	62.5320630267497\\
1091.7	62.538926543323\\
1091.90000000001	62.5274725274725\\
1092.09999999999	62.5343343709943\\
1092.19999999999	62.5286093564039\\
1092.30000000001	62.5320395459539\\
1092.39999999999	62.5263157894737\\
1092.8	62.5400311098911\\
1092.90000000001	62.5343092406222\\
1093.60000000002	62.5582883788973\\
1093.7	62.552569025416\\
1093.99999999998	62.5628370350059\\
1094.20000000001	62.551402723202\\
1094.39999999999	62.5582457743262\\
1094.49999999999	62.5525306047871\\
1094.90000000002	62.5662100456621\\
1095.00000000002	62.5604967582869\\
1095.70000000001	62.5844132140902\\
1095.80000000001	62.5787024363537\\
1095.90000000001	62.5821167883212\\
1096.10000000001	62.5706987775953\\
1096.2	62.5741129252942\\
1096.29999999999	62.5684056913535\\
1096.60000000001	62.5786450259871\\
1096.69999999999	62.572939460248\\
1096.89999999999	62.5797629899727\\
1097.1	62.5683558148013\\
1097.19999999999	62.5717670646131\\
1097.39999999999	62.5603644646925\\
1097.49999999998	62.5637755102041\\
1097.59999999998	62.5580759770429\\
1097.89999999999	62.568306010929\\
1098.00000000001	62.562608141335\\
1098.2	62.5694254757352\\
1098.29999999999	62.5637290604516\\
1098.50000000001	62.5705443291462\\
1098.59999999999	62.5648493674342\\
1099.29999999999	62.5886847371293\\
1099.50000000001	62.5773008366679\\
1099.59999999999	62.5807038283168\\
1099.69999999998	62.5750136388434\\
1099.80000000001	62.5784162196563\\
1099.90000000001	62.5727272727273\\
1100.00000000001	62.5761294427779\\
1100.09999999998	62.5704417378658\\
1100.20000000001	62.573843497228\\
1100.30000000001	62.5681570338059\\
1100.80000000001	62.5851575983286\\
1101.00000000001	62.5737898465171\\
1101.09999999999	62.5771885216128\\
1101.20000000001	62.5715064015255\\
1101.40000000001	62.578302315025\\
1101.59999999999	62.5669419987292\\
1101.7	62.5703394445453\\
1101.79999999998	62.5646610400218\\
1101.89999999999	62.568058076225\\
1102	62.5623809091734\\
1102.5	62.5793578813713\\
1102.60000000002	62.5736827786343\\
1102.70000000001	62.5770765324628\\
1102.80000000001	62.5714026656995\\
1102.90000000001	62.5747960108794\\
1103.00000000001	62.5691233795667\\
1103.20000000001	62.5759086377232\\
1103.30000000001	62.5702374478883\\
1103.50000000002	62.5770206596593\\
1103.59999999999	62.5713509105735\\
1104	62.5849107870664\\
1104.09999999998	62.5792428907806\\
1104.2	62.5826315312868\\
1104.9	62.5429864253394\\
1105.00000000001	62.5463758935843\\
1105.2	62.5350583551977\\
1105.60000000001	62.5486117391698\\
1105.7	62.5429553264605\\
1106.1	62.5564997288013\\
1106.29999999999	62.5451916124367\\
1106.4	62.5485765928604\\
1106.8	62.5259734393351\\
1106.89999999999	62.5293586269196\\
1106.99999999999	62.5237105952489\\
1107.09999999999	62.5270953757225\\
1107.2	62.5214485685903\\
1107.30000000001	62.5248329420264\\
1107.39999999999	62.5191873589165\\
1107.5	62.5225713253882\\
1108	62.4943597148272\\
1108.20000000001	62.5011278534693\\
1108.40000000001	62.489851150203\\
1108.69999999999	62.5\\
1108.90000000001	62.4887285843102\\
1109.1	62.4954922466643\\
1109.29999999998	62.4842257075897\\
1109.40000000001	62.4876070301938\\
1109.50000000002	62.4819754866619\\
1109.69999999999	62.488736709317\\
1109.90000000001	62.4774774774775\\
1110.19999999998	62.4876159596506\\
1110.3	62.4819884726225\\
1111.1	62.5089992800576\\
1111.29999999999	62.4977505848479\\
1111.6	62.5078708284609\\
1111.7	62.5022486058644\\
1111.8	62.5056210090836\\
1111.90000000002	62.5\\
1112	62.503371998921\\
1112.39999999999	62.4808988764045\\
1112.50000000001	62.4842710767572\\
1112.59999999999	62.4786555226027\\
1113.29999999999	62.5022453745285\\
1113.39999999999	62.4966322406825\\
1113.60000000001	62.5033671545299\\
1113.69999999999	62.4977554318549\\
1113.79999999998	62.5011221833199\\
1113.89999999999	62.4955116696589\\
1114.00000000001	62.4988780181312\\
1114.19999999998	62.4876604146101\\
1114.5	62.4977570428853\\
1114.7	62.48654467169\\
1115.7	62.5201649041047\\
1115.8	62.5145622367596\\
1116.00000000002	62.5212794552459\\
1116.09999999999	62.5156781938721\\
1116.29999999998	62.5223934073809\\
1116.40000000001	62.5167935512763\\
1116.60000000001	62.5235067609922\\
1116.80000000001	62.5123108604172\\
1116.9	62.515666965085\\
1117.19999999999	62.4988812315403\\
1117.70000000001	62.5156557523707\\
1117.90000000001	62.5044722719141\\
1118	62.5078257758698\\
1118.20000000002	62.4966466958777\\
1118.59999999999	62.5100563153661\\
1118.80000000001	62.4988828313522\\
1119.60000000001	62.5256765204966\\
1119.7	62.5200928737274\\
1119.8	62.5234395928208\\
1119.90000000001	62.5178571428572\\
1119.99999999998	62.5212034639764\\
1120.10000000001	62.5156222103196\\
1120.20000000001	62.5189681335357\\
1120.39999999999	62.5078090138331\\
1120.49999999999	62.5111547385329\\
1120.70000000001	62.5\\
1120.80000000001	62.5033455259167\\
1120.9	62.4977698483497\\
1121.10000000001	62.5044595076704\\
1121.3	62.4933119315142\\
1121.4	62.4966562639322\\
1121.50000000002	62.4910841654779\\
1121.69999999999	62.4977714387591\\
1122.29999999998	62.4643620812545\\
1122.7	62.477734235839\\
1122.90000000002	62.46660730187\\
1123.19999999999	62.4766313540461\\
1123.3	62.4710699661741\\
1123.49999999999	62.4777500889996\\
1123.7	62.4666310731447\\
1123.90000000001	62.4733096085409\\
1124.00000000001	62.4677519793613\\
1124.1	62.4710905532823\\
1124.6	62.4433182181915\\
1125.3	62.466678514306\\
1125.39999999999	62.4611283873834\\
1125.59999999999	62.4677978146931\\
1125.70000000001	62.4622490673299\\
1125.9	62.4689165186501\\
1126.00000000001	62.4633691501643\\
1126.09999999999	62.4667021843367\\
1126.29999999999	62.4556107954545\\
1126.40000000001	62.4589436307146\\
1126.49999999999	62.4533996094443\\
1126.60000000002	62.4567320493477\\
1126.70000000002	62.4511892083777\\
1126.8	62.454521252995\\
1127	62.4434389140272\\
1127.19999999999	62.450102013661\\
1127.5	62.4334870521462\\
1127.6	62.4368183027401\\
1127.70000000002	62.4312821422238\\
1128.09999999999	62.4446020209183\\
1128.20000000001	62.4390676238589\\
1128.39999999999	62.4457244129375\\
1128.50000000001	62.4401913875598\\
1128.6	62.4435190927616\\
1128.69999999998	62.43798724309\\
1128.80000000001	62.4413145539906\\
1128.89999999999	62.4357838795394\\
1129.70000000001	62.4623827226058\\
1129.90000000002	62.4513274336283\\
1129.99999999999	62.4546500309707\\
1130.10000000001	62.4491240488409\\
1130.79999999999	62.472367141215\\
1130.9	62.4668435013263\\
1131.40000000002	62.4834290764472\\
1131.49999999999	62.4779073877695\\
1131.59999999999	62.4812229389414\\
1131.79999999999	62.4701828783462\\
1131.89999999999	62.4734982332156\\
1131.99999999998	62.4679798604364\\
1132.40000000001	62.4812362030905\\
1132.60000000001	62.4702039374945\\
1132.80000000001	62.4768293759379\\
1132.89999999999	62.4713150926743\\
1132.99999999999	62.4746271291148\\
1133.09999999999	62.4691140134133\\
1133.20000000002	62.4724256595782\\
1133.3	62.4669137109582\\
1133.69999999999	62.4801552301993\\
1133.80000000002	62.474645030426\\
1134.30000000001	62.4911847672779\\
1134.39999999999	62.4856765094755\\
1135.10000000001	62.5088090204369\\
1135.20000000002	62.5033030916938\\
1135.9	62.5264084507042\\
1135.99999999999	62.5209048499252\\
1136.10000000002	62.5242034853019\\
1136.19999999998	62.5187010472587\\
1136.30000000001	62.5219992960225\\
1136.4	62.5164980202376\\
1136.6	62.5230931644233\\
1136.89999999999	62.5065963060686\\
1137.19999999998	62.5164864151939\\
1137.30000000001	62.5109899771409\\
1137.4	62.5142857142857\\
1137.50000000001	62.5087904360056\\
1137.69999999999	62.5153805589735\\
1137.8	62.5098866332718\\
1138.40000000002	62.5296442687747\\
1138.60000000001	62.5186616316853\\
1138.80000000001	62.5252436561595\\
1139.20000000002	62.5032914947775\\
1139.29999999999	62.5065824117957\\
1139.4	62.5010969723563\\
1139.69999999999	62.5109668362871\\
1139.9	62.5\\
1139.99999999999	62.5032891851592\\
1140.09999999999	62.4978074022101\\
1140.20000000002	62.5010962027537\\
1140.49999999998	62.484657197966\\
1140.60000000002	62.4879459980714\\
1140.70000000001	62.4824684431978\\
1140.80000000002	62.4857568586204\\
1140.9	62.4802804557406\\
1141.1	62.4868559411146\\
1141.2	62.481380881451\\
1141.29999999999	62.4846679516383\\
1141.40000000001	62.479194042926\\
1142.19999999999	62.5054714173159\\
1142.40000000001	62.4945295404814\\
1142.69999999999	62.5043752187609\\
1142.80000000002	62.4989062910141\\
1142.99999999999	62.5054675881375\\
1143.10000000001	62.5\\
1143.20000000001	62.5032799790081\\
1143.39999999998	62.4923480542195\\
1143.50000000001	62.4956278419028\\
1143.59999999999	62.4901635044155\\
1143.7	62.4934429095996\\
1143.80000000002	62.4879797185069\\
1143.9	62.4912587412587\\
1144	62.485796696093\\
1144.40000000001	62.4989078200087\\
1144.5	62.4934474925738\\
1145.29999999999	62.5196437925616\\
1145.49999999999	62.5087290502793\\
1145.6	62.5120013965261\\
1145.7	62.5065456449642\\
1146.2	62.5228997644596\\
1146.70000000001	62.4956400418556\\
1147.4	62.5185185185185\\
1147.49999999999	62.5130707563611\\
1147.7	62.5196027182436\\
1148.19999999999	62.4923800400592\\
1148.30000000001	62.4956461163358\\
1148.89999999999	62.4630113141862\\
1149.20000000001	62.4728095362395\\
1149.30000000001	62.4673742822342\\
1149.4	62.4706394084385\\
1149.50000000001	62.4652052887961\\
1150	62.481523345796\\
1150.09999999999	62.4760911145888\\
1150.29999999998	62.4826147426982\\
1150.4	62.477183833116\\
1150.59999999998	62.4837055705223\\
1150.70000000001	62.4782759819256\\
1150.80000000001	62.4815361890694\\
1151.00000000001	62.470680218921\\
1151.1	62.4739402362752\\
1151.19999999998	62.4685138539043\\
1151.3	62.4717734931388\\
1151.4	62.4663482414242\\
1151.49999999998	62.4696075026051\\
1151.79999999999	62.4533379633649\\
1152	62.4598559152851\\
1152.1	62.4544349939247\\
1152.60000000001	62.47072091611\\
1152.79999999999	62.4598837713592\\
1152.89999999999	62.4631396357329\\
1153.00000000001	62.4577226606539\\
1153.09999999999	62.4609781477627\\
1153.19999999999	62.4555622994884\\
1153.30000000002	62.4588174093983\\
1153.5	62.4479889042996\\
1153.70000000002	62.4544981799272\\
1153.89999999999	62.4436741767764\\
1154.10000000001	62.4501819442038\\
1154.19999999999	62.4447717231222\\
1154.39999999999	62.4512776093547\\
1154.49999999999	62.4458686991166\\
1156.20000000001	62.5010810343337\\
1156.50000000001	62.4848694449248\\
1156.60000000002	62.4881127345033\\
1156.7	62.4827109266943\\
1157.00000000001	62.4924379915306\\
1157.09999999998	62.4870376771517\\
1157.2	62.4902790979003\\
1157.30000000001	62.4848799032314\\
1157.49999999998	62.4913614374568\\
1157.6	62.485963548415\\
1157.8	62.4924432161672\\
1157.99999999999	62.4816509800535\\
1158.1	62.4848903470903\\
1158.20000000001	62.4794958128291\\
1159	62.5053921145716\\
1159.10000000001	62.5\\
1159.29999999999	62.5064688632051\\
1159.60000000002	62.4902992153143\\
1159.70000000002	62.493533367822\\
1159.89999999999	62.4827586206897\\
1160.30000000001	62.4956911409859\\
1160.39999999999	62.4903059026282\\
1160.50000000001	62.4935378252628\\
1160.6	62.4881537003532\\
1161.39999999999	62.5139905294877\\
1161.59999999998	62.5032280278902\\
1161.7	62.5064555000861\\
1161.80000000001	62.5010758240812\\
1162	62.5075294725067\\
1162.20000000001	62.4967736384754\\
1162.3	62.5\\
1162.50000000002	62.4892482367108\\
1162.60000000001	62.4924744130042\\
1162.80000000002	62.4817267176885\\
1163.3	62.49785112601\\
1163.7	62.476370510397\\
1163.99999999999	62.4860407181514\\
1164.40000000001	62.4645770717046\\
1165	62.4839069607759\\
1165.10000000002	62.4785444558874\\
1165.29999999998	62.4849836965849\\
1165.39999999999	62.4796224796225\\
1165.49999999998	62.4828414550446\\
1165.6	62.4774813416831\\
1166.09999999998	62.4935688561139\\
1166.20000000001	62.4882105804681\\
1166.4	62.4946420917274\\
1166.59999999999	62.4839290305991\\
1167.10000000002	62.5\\
1167.19999999999	62.4946457637283\\
1167.3	62.4978584889498\\
1167.69999999999	62.4764514471656\\
1167.8	62.4796643548249\\
1168.09999999999	62.4636192432803\\
1168.6	62.4796782750064\\
1168.70000000002	62.4743326488706\\
1168.80000000001	62.4775429891351\\
1168.90000000001	62.4721984602224\\
1169.00000000001	62.475408433838\\
1169.3	62.4593808790833\\
1169.59999999999	62.4690091476447\\
1169.69999999999	62.4636690032484\\
1169.79999999999	62.4668775108984\\
1169.90000000002	62.4615384615385\\
1169.99999999999	62.4647466028545\\
1170.19999999999	62.4540716055712\\
1170.49999999999	62.4636938322228\\
1170.6	62.4583582472025\\
1170.8	62.4647706892134\\
1170.89999999999	62.4594363791631\\
1171.30000000001	62.4722554208639\\
1171.39999999998	62.4669227486129\\
1171.60000000002	62.4733293505164\\
1171.70000000001	62.4679979518689\\
1172	62.4776042999744\\
1172.10000000001	62.472274355912\\
1172.19999999999	62.4754755608633\\
1172.40000000001	62.4648187633262\\
1172.49999999999	62.468019785093\\
1172.80000000001	62.4520419473101\\
1172.9	62.4552429667519\\
1172.99999999999	62.4499190179865\\
1173.10000000001	62.4531196726901\\
1173.30000000001	62.442474859383\\
1173.39999999999	62.4456753302088\\
1173.49999999999	62.4403544648943\\
1173.70000000001	62.4467541318794\\
1173.80000000001	62.4414345344578\\
1174.00000000001	62.4478323822502\\
1174.09999999999	62.4425140521206\\
1174.60000000002	62.4585000425641\\
1175.1	62.431926480599\\
1175.20000000002	62.4351229473326\\
1175.3	62.4298111281266\\
1175.4	62.4330072309655\\
1175.5	62.4276964954066\\
1175.60000000001	62.4308922344135\\
1175.79999999999	62.4202738328089\\
1175.9	62.4234693877551\\
1176	62.4181617209421\\
1176.10000000001	62.4213569120898\\
1176.20000000001	62.4160503272975\\
1176.30000000001	62.4192451547093\\
1176.8	62.392726654771\\
1176.9	62.3959218351742\\
1177.09999999999	62.3853211009174\\
1178.19999999999	62.4204362216753\\
1178.3	62.4151391717583\\
1178.80000000002	62.4310798201713\\
1178.89999999999	62.4257845631892\\
1179.00000000001	62.4289712492579\\
1179.1	62.4236770691995\\
1179.19999999998	62.4268633935386\\
1179.29999999998	62.421570289978\\
1179.39999999999	62.4247562526494\\
1179.70000000002	62.408882861502\\
1179.8	62.4120688193915\\
1179.89999999999	62.406779661017\\
1180.40000000001	62.4227022448115\\
1180.5	62.417414873793\\
1180.60000000001	62.4205979503684\\
1180.79999999999	62.4100262511644\\
1181.00000000001	62.4163914994497\\
1181.1	62.4111073484592\\
1181.60000000002	62.4270119319624\\
1181.89999999999	62.4111675126904\\
1182.20000000001	62.4207054047196\\
1182.3	62.4154262516915\\
1182.7	62.4281366249577\\
1182.90000000001	62.4175824175824\\
1183	62.4207590229059\\
1183.09999999998	62.4154834347532\\
1183.30000000002	62.4218353895555\\
1183.4	62.4165610477398\\
1183.49999999999	62.4197363974316\\
1183.6	62.4144631241024\\
1184.10000000001	62.4303327140686\\
1184.19999999998	62.4250612175969\\
1184.49999999998	62.4345770724295\\
1184.60000000001	62.4293069975521\\
1185.29999999998	62.4514931668635\\
1185.49999999999	62.4409581646424\\
1185.70000000001	62.4472929667735\\
1185.89999999998	62.4367622259696\\
1185.99999999999	62.4399291796644\\
1186.10000000001	62.4346653178216\\
1186.19999999999	62.4378319143556\\
1186.29999999998	62.4325691166554\\
1187.00000000001	62.4547215904305\\
1187.10000000001	62.449460916442\\
1187.20000000002	62.4526235997642\\
1187.3	62.4473639885464\\
1187.49999999999	62.4536881104749\\
1187.59999999999	62.4484297381494\\
1187.90000000002	62.4579124579125\\
1188.00000000001	62.4526555003788\\
1188.4	62.4652923853597\\
1188.6	62.454782535543\\
1188.69999999999	62.4579407806191\\
1188.8	62.4526873580621\\
1189.39999999999	62.4716267339218\\
1189.5	62.4663752521856\\
1189.9	62.4789915966387\\
1190.00000000001	62.4737417023779\\
1190.29999999998	62.4831989247312\\
1190.40000000002	62.4779504409912\\
1190.79999999999	62.4905533630028\\
1190.9	62.4853064651553\\
1191.10000000001	62.4916051040967\\
1191.19999999998	62.486359439268\\
1191.49999999999	62.4958039610608\\
1191.69999999999	62.485316328243\\
1192.09999999998	62.4979030364033\\
1192.29999999999	62.4874203287487\\
1193.10000000002	62.5125712370097\\
1193.2	62.5073326070561\\
1194.00000000002	62.5324512184909\\
1194.09999999999	62.5272148718808\\
1194.2	62.5303525077451\\
1194.3	62.5251172136638\\
1194.40000000001	62.5282544997907\\
1194.49999999999	62.5230202578269\\
1194.8	62.5324294920077\\
1194.89999999999	62.5271966527197\\
1195.30000000002	62.5397356533378\\
1195.49999999998	62.5292740046838\\
1195.70000000002	62.535541060378\\
1195.80000000001	62.5303118989882\\
1195.99999999999	62.5365772092634\\
1196.2	62.526122210148\\
1196.70000000001	62.5417780748663\\
1196.8	62.5365527613\\
1196.90000000002	62.5396825396825\\
1197.00000000001	62.5344582741626\\
1197.09999999999	62.5375877046442\\
1197.2	62.5323644867619\\
1197.70000000001	62.5480046752379\\
1197.79999999998	62.5427832039402\\
1198.50000000001	62.5646587685633\\
1198.7	62.5542208875542\\
1198.89999999999	62.5604670558799\\
1198.99999999999	62.5552497706613\\
1199.19999999999	62.5614942049529\\
1199.30000000001	62.5562781390695\\
1199.4	62.5593997498958\\
1199.50000000002	62.5541847282427\\
1199.90000000001	62.5666666666667\\
1200.10000000001	62.5562406265622\\
1200.19999999999	62.55936015996\\
1200.29999999999	62.5541486171276\\
1200.40000000002	62.5572678050812\\
1200.5	62.552057304681\\
1200.60000000002	62.5551761472474\\
1200.8	62.5447580980931\\
1201.19999999998	62.5572296678598\\
1201.29999999998	62.552022640253\\
1201.50000000002	62.5582556591212\\
1201.59999999999	62.5530498460514\\
1201.79999999998	62.5592811381979\\
1201.90000000001	62.5540765391015\\
1202.2	62.5634201114531\\
1202.49999999999	62.547813071678\\
1202.59999999999	62.550927080735\\
1202.69999999999	62.545726637845\\
1202.99999999999	62.5550660792952\\
1203.19999999998	62.5446688273913\\
1203.5	62.5540046527085\\
1203.80000000001	62.5384168120276\\
1203.99999999999	62.5446391495723\\
1204.10000000001	62.5394452748713\\
1204.19999999999	62.5425558415677\\
1204.40000000002	62.5321710253217\\
1204.70000000002	62.5415006640106\\
1204.8	62.5363100672255\\
1205.70000000001	62.5642726820368\\
1205.89999999999	62.5538971807629\\
1206.29999999998	62.5663129973475\\
1206.40000000001	62.5611272275176\\
1206.69999999999	62.5704342061651\\
1207.10000000001	62.5497017892644\\
1207.29999999999	62.5559052509525\\
1207.49999999998	62.5455448824114\\
1207.60000000001	62.548646186967\\
1207.79999999998	62.5382895935094\\
1208.20000000001	62.5506910535463\\
1208.39999999999	62.5403392635499\\
1208.5	62.5434386893927\\
1208.6	62.538264250848\\
1208.80000000001	62.5444619075192\\
1209.09999999999	62.528944756864\\
1209.5	62.541335978836\\
1209.59999999998	62.5361659915682\\
1209.99999999998	62.5485497066358\\
1210.19999999998	62.5382136660332\\
1210.29999999998	62.5413086582948\\
1210.49999999999	62.5309763753511\\
1210.69999999999	62.5371655104064\\
1211.00000000002	62.5216745107753\\
1211.19999999999	62.5278626269297\\
1211.4	62.5175402393727\\
1211.8	62.5299117088869\\
1211.9	62.5247524752475\\
1212.4	62.540206185567\\
1212.59999999999	62.5298919765812\\
1212.69999999998	62.532981530343\\
1212.80000000001	62.5278258718773\\
1213.1	62.5370919881306\\
1213.19999999998	62.5319376905959\\
1213.49999999999	62.5411997363217\\
1213.7	62.5308947108255\\
1213.79999999999	62.5339813823214\\
1214.5	62.4979417092047\\
1214.80000000001	62.5072022388674\\
1214.89999999999	62.5020576131687\\
1215.1	62.5082290980909\\
1215.30000000002	62.4979430640118\\
1215.49999999999	62.50411319513\\
1215.80000000001	62.4886915042355\\
1216.49999999998	62.510274535591\\
1216.7	62.5\\
1217.1	62.5123233651002\\
1217.19999999998	62.5071880391029\\
1217.50000000001	62.5164257555848\\
1217.60000000002	62.5112917795845\\
1217.7	62.5143701757267\\
1217.99999999999	62.4989738116739\\
1218.09999999999	62.502052208176\\
1218.50000000001	62.4815361890694\\
1218.59999999998	62.4846147534258\\
1218.9	62.4692370795734\\
1219.1	62.4753937007874\\
1219.19999999999	62.4702698269499\\
1219.3	62.4733475479744\\
1219.39999999998	62.4682246822468\\
1219.70000000001	62.4774553205444\\
1219.8	62.4723337978523\\
1221.10000000001	62.5122830003276\\
1221.19999999999	62.5071644968476\\
1221.40000000001	62.5133033155956\\
1221.49999999999	62.5081859855927\\
1221.8	62.5173909485228\\
1222.00000000001	62.5071598068898\\
1222.09999999999	62.5102274586811\\
1222.20000000001	62.5051133109711\\
1222.60000000001	62.5173795698045\\
1222.69999999999	62.5122669283611\\
1222.89999999998	62.5183973834832\\
1223	62.5132859128444\\
1223.09999999999	62.5163505559189\\
1223.3	62.5061304561059\\
1223.49999999999	62.5122589081399\\
1223.60000000001	62.5071504453706\\
1223.9	62.5163398692811\\
1223.99999999999	62.511232742423\\
1224.09999999999	62.5142950498285\\
1224.29999999999	62.5040836327997\\
1224.49999999999	62.510207414666\\
1224.6	62.5051032906018\\
1224.89999999999	62.5142857142857\\
1225.19999999999	62.4989798416714\\
1225.29999999999	62.5020401501551\\
1225.5	62.4918407310705\\
1225.69999999999	62.4979605155817\\
1225.79999999999	62.4928623868178\\
1225.90000000002	62.4959216965742\\
1225.99999999999	62.4908245656961\\
1226.19999999998	62.4969420207127\\
1226.4	62.4867509172442\\
1226.59999999999	62.4928670416565\\
1226.7	62.4877730681448\\
1227	62.4969440143428\\
1227.10000000002	62.49185136897\\
1227.2	62.4949075205736\\
1227.29999999998	62.4898158709467\\
1227.59999999999	62.4989818359534\\
1227.9	62.4837133550489\\
1228	62.4867681784871\\
1228.1	62.4816805080606\\
1228.7	62.5\\
1228.9	62.4898291293735\\
1229.1	62.4959323137\\
1229.2	62.4908484503376\\
1229.60000000001	62.5030495242742\\
1229.70000000002	62.4979671491299\\
1229.99999999999	62.5071132428258\\
1230.09999999998	62.5020321898878\\
1230.19999999999	62.5050800617736\\
1230.50000000001	62.4898423533236\\
1230.7	62.49593760156\\
1230.8	62.4908603460882\\
1231.50000000002	62.5121792789867\\
1231.60000000002	62.507104002598\\
1231.79999999998	62.5131910057635\\
1231.90000000002	62.5081168831169\\
1231.99999999999	62.5111598084571\\
1232.30000000001	62.4959428756897\\
1232.69999999999	62.5081116158339\\
1232.80000000001	62.5030416092141\\
1233.10000000001	62.5121634771327\\
1233.19999999999	62.5070947863456\\
1233.30000000001	62.5101345873196\\
1233.40000000002	62.5050668828537\\
1233.49999999999	62.5081063553826\\
1233.60000000002	62.5030396368647\\
1233.7	62.5060787810018\\
1233.8	62.501013048059\\
1233.89999999998	62.5040518638574\\
1233.99999999998	62.498987116117\\
1234.5	62.5141746314596\\
1234.59999999999	62.5091115250668\\
1234.7	62.5121477162294\\
1234.79999999998	62.5070855939752\\
1234.9	62.5101214574899\\
1235.00000000002	62.5050603190025\\
1235.19999999999	62.5111308993767\\
1235.29999999999	62.5060709082079\\
1235.49999999998	62.5121398510845\\
1235.70000000001	62.5020229810649\\
1235.79999999998	62.5050570434501\\
1235.89999999998	62.5\\
1236.29999999999	62.5121319961178\\
1236.39999999999	62.5070764253943\\
1236.7	62.51617076326\\
1236.80000000001	62.5111165009297\\
1236.99999999999	62.5171772694204\\
1237.2	62.507071849996\\
1237.6	62.5191888179688\\
1237.70000000002	62.5141379867507\\
1237.80000000002	62.5171661685112\\
1237.99999999998	62.5070672805105\\
1238.1	62.5100952996285\\
1238.20000000001	62.5050472421869\\
1238.3	62.5080749354005\\
1238.50000000002	62.4979815921201\\
1238.59999999999	62.5010091224671\\
1238.69999999999	62.4959638359703\\
1238.80000000001	62.4989910404391\\
1238.90000000002	62.4939467312349\\
1239.10000000002	62.5\\
1239.2	62.4949568304688\\
1239.29999999999	62.4979828949492\\
1239.40000000001	62.4929407018959\\
1239.50000000001	62.4959664407874\\
1239.6	62.4909252238445\\
1239.70000000001	62.4939506371995\\
1239.8	62.4889103959997\\
1239.9	62.491935483871\\
1240.00000000002	62.4868962180469\\
1240.9	62.5141015310234\\
1241.10000000001	62.504028359652\\
1241.20000000002	62.5070490614678\\
1241.29999999999	62.5020138553246\\
1242.10000000001	62.5261632587345\\
1242.29999999999	62.5160978750805\\
1242.50000000001	62.5221310156124\\
1242.89999999999	62.5020112630732\\
1243.10000000001	62.5080437580438\\
1243.39999999998	62.4929634097306\\
1243.69999999999	62.5020099694485\\
1243.80000000001	62.4969852882065\\
1244.19999999999	62.5090412279997\\
1244.3	62.5040180006429\\
1244.80000000002	62.5190778375773\\
1244.90000000001	62.5140562248996\\
1245.19999999999	62.523086806392\\
1245.30000000001	62.5180664846636\\
1245.39999999999	62.5210758731433\\
1245.6	62.5110379706189\\
1246.09999999999	62.5260792810143\\
1246.19999999999	62.5210623445398\\
1246.3	62.5240693196406\\
1246.5	62.5140381838601\\
1246.80000000001	62.5230571818109\\
1247.00000000002	62.5130302301339\\
1247.09999999998	62.5160359204618\\
1247.2	62.5110238114327\\
1247.59999999999	62.5230423980124\\
1247.69999999998	62.5180317358551\\
1247.8	62.5210353393701\\
1247.89999999998	62.5160256410256\\
1248.30000000001	62.528035885934\\
1248.39999999999	62.5230276331598\\
1248.6	62.5290301913991\\
1248.7	62.5240230621397\\
1248.80000000001	62.5270237809272\\
1248.90000000001	62.5220176140913\\
1249.20000000002	62.5310173697271\\
1249.39999999999	62.5210084033613\\
1249.49999999999	62.5240076824584\\
1249.70000000002	62.5140022403585\\
1249.79999999998	62.5170013601088\\
1249.90000000001	62.512\\
1250	62.514998800096\\
1250.30000000001	62.5\\
1250.59999999998	62.5089949628208\\
1251	62.4890096714891\\
1251.10000000001	62.4920076726343\\
1251.19999999999	62.4870135059538\\
1252	62.5109815509943\\
1252.09999999998	62.5059894585529\\
1252.30000000002	62.511977004152\\
1252.39999999998	62.5069860279441\\
1252.59999999998	62.5129719805221\\
1252.70000000001	62.5079821200511\\
1252.90000000001	62.5139664804469\\
1253.00000000002	62.5089777352167\\
1253.30000000001	62.51795117281\\
1253.60000000001	62.5029911462072\\
1253.7	62.5059818152815\\
1253.9	62.4960127591707\\
1254.49999999998	62.5139486688985\\
1254.60000000001	62.5089662867618\\
1254.80000000002	62.514941429596\\
1255	62.5049796828938\\
1255.20000000001	62.5109535569187\\
1255.3	62.5059741914928\\
1255.40000000001	62.5089605734767\\
1255.50000000002	62.5039821599236\\
1256.29999999998	62.5278573702643\\
1256.40000000001	62.5228810187027\\
1256.59999999999	62.5288453887165\\
1256.90000000001	62.5139220365951\\
1257.60000000001	62.534785719965\\
1257.8	62.5248429922887\\
1258.80000000002	62.5546111684804\\
1258.9	62.549642573471\\
1259.2	62.5585642817438\\
1259.40000000001	62.5486304088924\\
1259.50000000001	62.5516036837091\\
1259.59999999999	62.5466380884338\\
1259.79999999999	62.5525835383761\\
1259.9	62.5476190476191\\
1260.39999999999	62.5624752082507\\
1260.49999999998	62.5575122957322\\
1260.60000000002	62.5604822717538\\
1260.79999999999	62.5505591244349\\
1260.90000000001	62.5535289452815\\
1260.99999999998	62.5485687098565\\
1261.10000000001	62.5515382175706\\
1261.60000000002	62.5267496235238\\
1261.80000000001	62.5326888025993\\
1262.09999999998	62.5178260180637\\
1262.19999999999	62.5207953735245\\
1262.3	62.5158428390368\\
1262.39999999999	62.5188118811881\\
1262.50000000001	62.513860288294\\
1262.70000000001	62.5197972758948\\
1263.09999999999	62.5\\
1263.30000000002	62.5059363621972\\
1263.39999999999	62.5009893153938\\
1264.39999999999	62.5306445235271\\
1264.50000000002	62.525699826032\\
1264.90000000001	62.5375494071146\\
1265.2	62.5227218841382\\
1265.39999999999	62.5286448044251\\
1265.60000000001	62.5187643201391\\
1265.8	62.5246859941544\\
1265.89999999999	62.519747235387\\
1266	62.5227075270516\\
1266.2	62.5128326620864\\
1266.60000000002	62.5246704034104\\
1266.8	62.5147999052806\\
1266.90000000001	62.5177584846093\\
1266.99999999999	62.5128245600189\\
1267.20000000001	62.5187406296852\\
1267.3	62.5138077954868\\
1267.8	62.5285905828535\\
1267.90000000002	62.5236593059937\\
1268	62.5266146202981\\
1268.09999999999	62.5216842769279\\
1268.70000000001	62.5394073139975\\
1268.90000000002	62.5295508274232\\
1269.20000000002	62.538406995982\\
1269.40000000001	62.5285545490351\\
1269.50000000002	62.5315059861374\\
1269.6	62.5265810821454\\
1269.70000000001	62.5295322097968\\
1269.79999999999	62.5246082368691\\
1270.1	62.5334592977484\\
1270.3	62.5236146095718\\
1270.60000000002	62.5324624222869\\
1270.7	62.527541706012\\
1270.89999999999	62.5334382376082\\
1271.10000000002	62.5235997482694\\
1271.30000000001	62.5294950448325\\
1271.50000000001	62.5196602705253\\
1272.19999999999	62.5402813801776\\
1272.3	62.5353662370324\\
1272.59999999999	62.5441973756581\\
1272.80000000001	62.5343703354545\\
1273.20000000001	62.5461399513076\\
1273.49999999998	62.5314070351759\\
1273.60000000002	62.5343487477428\\
1273.7	62.5294394724447\\
1273.89999999998	62.5353218210361\\
1274.1	62.5255061999686\\
1274.3	62.5313873195229\\
1274.40000000001	62.5264809729306\\
1274.49999999999	62.5294209948219\\
1274.60000000002	62.5245155722915\\
1274.80000000001	62.5303945407483\\
1274.99999999998	62.5205866206572\\
1275.1	62.5235257214555\\
1275.19999999998	62.5186230690818\\
1275.50000000001	62.52743806836\\
1275.59999999999	62.522536646547\\
1275.99999999998	62.5342841470104\\
1276.10000000001	62.5293841090738\\
1276.19999999999	62.5323199874638\\
1276.5	62.5176249412502\\
1277.00000000001	62.5322997416021\\
1277.10000000002	62.5274036955841\\
1277.2	62.5303374305175\\
1277.30000000001	62.5254423046814\\
1277.5	62.5313087038197\\
1277.59999999999	62.5264146513266\\
1277.89999999999	62.5352112676056\\
1277.99999999999	62.5303184414365\\
1278.5	62.5449710620992\\
1278.6	62.5400797685149\\
1279.10000000001	62.5547217010632\\
1279.3	62.5449429420041\\
1279.6	62.5537235289521\\
1279.70000000002	62.5488357555868\\
1280.2	62.5634616886667\\
1280.4	62.5536899648575\\
1280.90000000001	62.568306010929\\
1281.10000000001	62.5585388698096\\
1281.19999999998	62.5614610161555\\
1281.29999999999	62.5565787420009\\
1281.70000000001	62.5682633796224\\
1281.79999999999	62.5633824791325\\
1281.89999999999	62.5663026521061\\
1282.00000000001	62.5614226659387\\
1282.3	62.5701809107923\\
1282.39999999998	62.5653021442495\\
1283.40000000002	62.5944682508765\\
1283.50000000001	62.5895917731381\\
1283.60000000002	62.5925060372361\\
1283.69999999999	62.5876304720361\\
1283.80000000001	62.5905444349248\\
1284.00000000001	62.5807958881707\\
1284.09999999998	62.5837097025386\\
1284.19999999999	62.5788367203924\\
1284.40000000001	62.5846632931102\\
1285.1	62.5505757858699\\
1285.3	62.5564026762097\\
1285.39999999999	62.5515363671723\\
1285.49999999999	62.5544492843808\\
1285.79999999998	62.5398553542266\\
1285.90000000002	62.542768273717\\
1286.10000000001	62.533043072617\\
1286.40000000001	62.5417800233191\\
1286.5	62.5369190113477\\
1286.8	62.5456523428394\\
1286.90000000002	62.5407925407926\\
1287	62.5437028979877\\
1287.10000000001	62.538844002486\\
1287.60000000002	62.5533897646967\\
1287.69999999999	62.5485323808045\\
1287.9	62.554347826087\\
1287.99999999998	62.5494914991072\\
1288.29999999998	62.5582117354859\\
1288.8	62.5339436728994\\
1288.89999999998	62.5368502715283\\
1289.00000000001	62.531999069118\\
1289.60000000001	62.5494301000233\\
1289.69999999999	62.5445805551248\\
1289.89999999999	62.5503875968992\\
1290.1	62.5406913656797\\
1290.3	62.5464972101674\\
1290.4	62.5416505230531\\
1290.99999999999	62.5590581674541\\
1291.30000000001	62.5445253213567\\
1291.40000000002	62.5474254742548\\
1291.50000000001	62.5425828429854\\
1291.79999999999	62.5512810589055\\
1292.00000000002	62.5415989474499\\
1292.09999999999	62.5444977557654\\
1292.29999999999	62.5348189415042\\
1292.80000000001	62.5493077577539\\
1292.89999999999	62.5444702242846\\
1293.7	62.5676302365126\\
1293.8	62.5627946518278\\
1294.39999999999	62.5801467748165\\
1294.49999999999	62.5753128379422\\
1294.60000000002	62.5782034448135\\
1294.69999999999	62.5733704046957\\
1294.80000000002	62.5762607151132\\
1294.90000000002	62.5714285714286\\
1295.09999999998	62.577208153181\\
1295.30000000001	62.5675467037209\\
1295.40000000001	62.5704361250482\\
1295.6	62.5607779578606\\
1295.89999999999	62.5694444444444\\
1296	62.5646169277062\\
1296.20000000001	62.570392656021\\
1296.29999999999	62.5655661832768\\
1296.4	62.5684535287312\\
1296.5	62.5636279500231\\
1297.09999999998	62.5809435707678\\
1297.39999999999	62.5664739884393\\
1297.5	62.5693588162762\\
1297.59999999998	62.5645372582261\\
1297.69999999998	62.5674217907228\\
1297.9	62.557781201849\\
1298.09999999998	62.5635495301186\\
1298.20000000002	62.5587306477702\\
1298.4	62.5644974971121\\
1298.69999999999	62.5500461964891\\
1298.99999999998	62.5586944807944\\
1299.10000000002	62.5538793103448\\
1299.39999999999	62.5625240477107\\
1299.49999999998	62.5577100646353\\
1299.9	62.5692307692308\\
1299.99999999999	62.564418121683\\
1300.10000000001	62.5672973388709\\
1300.2	62.5624855802507\\
1300.3	62.5653645032298\\
1300.70000000001	62.5461254612546\\
1300.80000000002	62.5490045353217\\
1300.90000000001	62.5441967717141\\
1300.99999999999	62.5470755514565\\
1301.10000000001	62.5422686750692\\
1301.2	62.5451471605318\\
1301.30000000001	62.5403411710466\\
1301.79999999998	62.554727705661\\
1302.1	62.5403163876517\\
1302.49999999999	62.551819438047\\
1302.69999999999	62.5422167638932\\
1302.90000000001	62.5479662317728\\
1303	62.5431662957563\\
1303.20000000001	62.5489142944832\\
1303.29999999999	62.5441153905171\\
1303.6	62.552734524814\\
1303.79999999998	62.5431398113352\\
1303.89999999999	62.5460122699387\\
1303.99999999998	62.5412161644046\\
1304.09999999998	62.5440883300107\\
1304.20000000001	62.5392931074139\\
1304.50000000002	62.5479074045684\\
1304.6	62.543113359393\\
1304.80000000002	62.5488543183386\\
1304.90000000001	62.544061302682\\
1305.10000000001	62.5498007968127\\
1305.40000000002	62.5354270394485\\
1305.49999999999	62.5382965686275\\
1305.59999999999	62.533506931148\\
1306.1	62.5478487214822\\
1306.19999999999	62.5430605527061\\
1306.60000000002	62.554526670238\\
1306.79999999998	62.5449537072462\\
1306.89999999998	62.5478194338179\\
1307	62.5430341978426\\
1307.4	62.5544933078394\\
1307.7	62.5401437528674\\
1307.8	62.5430078752198\\
1308.20000000001	62.5238859588779\\
1308.40000000001	62.5296140619029\\
1308.6	62.5200580728968\\
1308.70000000001	62.5229217603912\\
1308.8	62.518145007258\\
1309.60000000001	62.541039932809\\
1309.70000000001	62.536265078638\\
1309.90000000002	62.5419847328244\\
1310.19999999999	62.5276654201328\\
1310.3	62.530525030525\\
1310.40000000001	62.5257535291873\\
1310.60000000002	62.5314717326619\\
1310.7	62.5267012511443\\
1311.10000000001	62.5381330079317\\
1311.19999999999	62.5333638374133\\
1311.89999999998	62.5533536585366\\
1312.09999999999	62.5438195397043\\
1312.89999999999	62.5666412795126\\
1313	62.561876475516\\
1313.09999999999	62.5647273834907\\
1313.20000000001	62.559963450849\\
1313.7	62.5742122088598\\
1314.1	62.5551666413027\\
1314.30000000001	62.5608642726719\\
1314.90000000002	62.532319391635\\
1315.19999999999	62.5408652018551\\
1315.40000000002	62.5313568985177\\
1315.7	62.5398996808026\\
1316	62.5256439480283\\
1316.30000000001	62.5341841385597\\
1316.5	62.5246847941668\\
1316.6	62.5275309485836\\
1316.69999999998	62.5227825030377\\
1316.8	62.525628369656\\
1316.99999999998	62.5161339306051\\
1317.20000000002	62.5218249449632\\
1317.5	62.5075895567699\\
1317.69999999999	62.5132797086053\\
1317.79999999998	62.5085363077624\\
1317.90000000001	62.5113808801214\\
1318.29999999999	62.4924150485437\\
1319.10000000002	62.5151607034566\\
1319.20000000002	62.5104221935875\\
1319.3	62.5132636046688\\
1319.49999999998	62.5037890269779\\
1319.9	62.5151515151515\\
1320.00000000001	62.510415877585\\
1320.20000000001	62.5160948269333\\
1320.80000000002	62.4876977818154\\
1321.29999999999	62.5018919327986\\
1321.39999999999	62.4971623155505\\
1321.59999999999	62.5028372550503\\
1321.69999999999	62.4981086397337\\
1321.79999999998	62.5009456085937\\
1322.20000000002	62.482038871663\\
1322.4	62.4877126654064\\
1322.5	62.4829880538334\\
1322.6	62.4858244499887\\
1322.89999999999	62.4716553287982\\
1323.40000000002	62.4858330185115\\
1323.60000000001	62.4763919317066\\
1323.80000000002	62.4820605785935\\
1323.9	62.4773413897281\\
1324.09999999999	62.4830086089715\\
1324.20000000002	62.4782904175791\\
1324.6	62.4896202913867\\
1324.70000000002	62.4849033816425\\
1324.8	62.4877349233904\\
1324.89999999999	62.4830188679245\\
1325.20000000001	62.4915113559194\\
1325.3	62.4867964388109\\
1325.79999999998	62.5009427558639\\
1325.9	62.4962292609352\\
1326.2	62.5047123576868\\
1326.29999999998	62.5\\
1326.59999999998	62.50847968644\\
1326.70000000001	62.5037684654809\\
1326.79999999998	62.5065943175823\\
1326.90000000001	62.5018839487566\\
1327.1	62.5075346594334\\
1327.19999999999	62.502825284412\\
1327.60000000001	62.514122166152\\
1327.69999999998	62.5094140683838\\
1328.09999999999	62.5207047131456\\
1328.59999999999	62.4971776924814\\
1328.69999999999	62.5\\
1328.8	62.4952968620664\\
1329.19999999998	62.5065824117957\\
1329.30000000002	62.5018805476155\\
1329.69999999999	62.5131598736652\\
1329.79999999998	62.5084592826528\\
1329.90000000002	62.5112781954887\\
1330.2	62.4971810869729\\
1330.29999999999	62.5\\
1330.39999999999	62.4953025178504\\
1330.7	62.5037571385633\\
1331.19999999999	62.4802824307068\\
1331.40000000002	62.485918137439\\
1331.49999999998	62.4812255932712\\
1331.59999999999	62.484042952617\\
1331.69999999999	62.479351253942\\
1331.9	62.484984984985\\
1332	62.4802942722018\\
1332.30000000002	62.4887421194836\\
1332.39999999999	62.484052532833\\
1332.99999999998	62.5009376640912\\
1333.20000000002	62.4915622890572\\
1333.50000000001	62.5\\
1333.59999999999	62.4953137887081\\
1333.69999999999	62.4981256560204\\
1334.00000000002	62.4840716587962\\
1334.09999999999	62.4868835257083\\
1334.20000000001	62.4822004047066\\
1334.30000000002	62.4850119904077\\
1334.4	62.4803297115024\\
1334.59999999999	62.4859518993032\\
1334.80000000001	62.4765900067421\\
1335.09999999999	62.4850209706411\\
1335.19999999999	62.480341496293\\
1335.60000000002	62.4915774500262\\
1335.70000000002	62.4868992364126\\
1336.10000000001	62.4981290226014\\
1336.2	62.4934520691462\\
1336.80000000001	62.510284987658\\
1337.09999999999	62.4962608435537\\
1337.99999999999	62.521485688663\\
1338.2	62.5121422700441\\
1338.5	62.5205438517854\\
1338.59999999999	62.5158736087249\\
1338.69999999999	62.5186734389005\\
1338.80000000002	62.5140040331616\\
1339.00000000001	62.5196027182436\\
1339.10000000002	62.5149342891278\\
1339.60000000002	62.5289243860566\\
1340.00000000001	62.5102604283262\\
1340.19999999998	62.5158546594046\\
1340.29999999999	62.5111906893465\\
1340.50000000001	62.5167835297628\\
1340.59999999999	62.5121205340494\\
1340.7	62.5149164677804\\
1340.79999999998	62.5102543068089\\
1340.90000000001	62.5130499627144\\
1341.00000000002	62.5083886361942\\
1341.20000000002	62.5139789756207\\
1341.30000000002	62.5093186223349\\
1341.39999999999	62.5121133060008\\
1341.5	62.5074537865236\\
1341.70000000001	62.5130421821434\\
1341.80000000002	62.5083836351442\\
1342.5	62.5279308803814\\
1342.59999999999	62.523274000149\\
1342.7	62.5260649389336\\
1342.80000000002	62.5214088912056\\
1343	62.5269897997171\\
1343.30000000002	62.5130266488016\\
1343.60000000002	62.5213961449728\\
1343.80000000001	62.5120916734876\\
1343.90000000001	62.514880952381\\
1344.19999999998	62.5009298519676\\
1344.59999999999	62.5120844798096\\
1344.80000000001	62.5027883113986\\
1344.9	62.5055762081784\\
1344.99999999999	62.5009292989369\\
1345.40000000001	62.512077294686\\
1345.59999999999	62.5027866537861\\
1345.90000000001	62.5111441307578\\
1346.09999999999	62.5018570791859\\
1346.19999999998	62.504642353116\\
1346.30000000001	62.5\\
1346.4	62.5027849981433\\
1346.50000000002	62.4981434724491\\
1347.3	62.5204096778982\\
1347.39999999999	62.5157699443414\\
1347.50000000001	62.5185514989611\\
1347.70000000001	62.5092743730524\\
1348.09999999998	62.5203975671265\\
1348.19999999999	62.5157605874064\\
1348.39999999998	62.5213199851687\\
1348.49999999999	62.51668396856\\
1348.6	62.5194631867725\\
1348.80000000002	62.5101934909927\\
1348.89999999999	62.5129725722758\\
1349.00000000002	62.5083388925951\\
1349.49999999999	62.5222288085359\\
1349.59999999999	62.5175965029266\\
1350.20000000002	62.534251647782\\
1350.4	62.5249907441688\\
1350.50000000002	62.5277654375833\\
1350.59999999999	62.5231361516251\\
1350.7	62.5259105715132\\
1350.9	62.5166543301258\\
1351.10000000002	62.5222024866785\\
1351.20000000002	62.5175756678754\\
1351.49999999998	62.5258952352767\\
1351.79999999999	62.5120201198313\\
1352.1	62.5203372282207\\
1352.30000000002	62.511091393079\\
1352.5	62.5166346296023\\
1352.69999999999	62.5073920756949\\
1353.09999999999	62.5184747265741\\
1353.20000000001	62.5138550210596\\
1353.40000000002	62.5193941632804\\
1353.59999999998	62.5101573465317\\
1353.99999999998	62.5212318144893\\
1354.10000000002	62.5166149756314\\
1354.30000000002	62.5221500295334\\
1354.49999999999	62.5129189428614\\
1354.60000000002	62.5156861297704\\
1354.70000000001	62.511071744907\\
1355.00000000002	62.5193712641134\\
1355.2	62.5101453552719\\
1355.80000000002	62.5267350099565\\
1355.89999999999	62.5221238938053\\
1356.19999999999	62.5304136253041\\
1356.39999999999	62.5211942499079\\
1356.6	62.5267192452274\\
1356.7	62.5221108490566\\
1357.00000000001	62.5303956967062\\
1357.20000000001	62.5211817579017\\
1357.50000000001	62.5294637595757\\
1357.6	62.5248582161008\\
1357.70000000001	62.5276182059213\\
1357.90000000002	62.5184094256259\\
1358.00000000001	62.5211692806126\\
1358.10000000001	62.5165660432926\\
1358.29999999999	62.5220848056537\\
1358.6	62.5082799735041\\
1358.80000000002	62.5137979247921\\
1358.89999999999	62.5091979396615\\
1358.99999999999	62.5119564417629\\
1359.1	62.5073572689818\\
1359.2	62.5101155006253\\
1359.29999999999	62.5055171399147\\
1359.39999999998	62.5082751011401\\
1359.50000000002	62.5036775522212\\
1359.8	62.5119494080447\\
1359.99999999998	62.5027571502095\\
1360.10000000001	62.5055138950154\\
1360.3	62.4963246104087\\
1360.79999999998	62.5101036079065\\
1360.9	62.5055106539309\\
1360.99999999999	62.5082653735949\\
1361.1	62.5036732295034\\
1361.20000000001	62.5064276794241\\
1361.6	62.4880663876037\\
1361.7	62.4908209722426\\
1362.19999999999	62.4678851941569\\
1362.30000000001	62.4706400469759\\
1362.59999999999	62.4568870624496\\
1362.79999999999	62.4623963607014\\
1362.9	62.4578136463683\\
1362.99999999999	62.4605678233439\\
1363.10000000001	62.455985915493\\
1363.29999999998	62.4614933255098\\
1363.59999999999	62.4477524382196\\
1363.70000000001	62.4505059392873\\
1363.89999999999	62.441348973607\\
1363.99999999999	62.4441023385382\\
1364.19999999998	62.4349483251484\\
1364.30000000001	62.4377015537965\\
1364.5	62.4285504909864\\
1364.6	62.4313035832051\\
1364.80000000002	62.4221554692651\\
1364.9	62.4249084249084\\
1365.10000000002	62.4157632581307\\
1365.29999999998	62.4212684927494\\
1365.39999999999	62.41669718052\\
1365.49999999999	62.4194493263035\\
1365.70000000002	62.4103089764241\\
1366.20000000001	62.4240649930469\\
1366.29999999998	62.4194964871194\\
1366.89999999998	62.4359912216533\\
1367.00000000002	62.4314241825763\\
1367.10000000002	62.4341720304271\\
1367.40000000001	62.4204753199269\\
1367.59999999998	62.4259706075894\\
1367.79999999999	62.4168433365012\\
1368.10000000002	62.4250840520392\\
1368.20000000001	62.4205218153914\\
1368.79999999998	62.436993206224\\
1369.1	62.4233128834356\\
1369.3	62.4288009347159\\
1369.5	62.4196845794392\\
1369.99999999999	62.4333990219692\\
1370.10000000001	62.4288425047438\\
1370.39999999998	62.4370667639548\\
1370.50000000002	62.4325113089158\\
1370.60000000001	62.4352520609907\\
1370.7	62.4306974029764\\
1370.89999999998	62.436177972283\\
1371.1	62.4270711785298\\
1371.19999999999	62.4298111281266\\
1371.40000000002	62.4207072548305\\
1371.59999999999	62.4261864839251\\
1371.69999999999	62.421635806969\\
1371.99999999998	62.4298520515997\\
1372.1	62.4253024340475\\
1372.20000000001	62.4280405159222\\
1372.30000000002	62.4234916933839\\
1372.4	62.4262295081967\\
1372.59999999998	62.4171341152473\\
1372.70000000002	62.4198717948718\\
1372.89999999999	62.4107793153678\\
1373.10000000002	62.4162540052432\\
1373.30000000002	62.4071647007427\\
1373.5	62.4126383226558\\
1373.59999999999	62.408094926112\\
1373.8	62.4135672174103\\
1373.90000000001	62.4090247452693\\
1374	62.4117604250055\\
1374.1	62.4072187454519\\
1374.29999999998	62.4126891734575\\
1374.40000000001	62.4081484176064\\
1374.70000000001	62.4163514693046\\
1374.8	62.4118117681286\\
1375.00000000001	62.4172787433641\\
1375.39999999999	62.3991275899673\\
1375.7	62.4073266463149\\
1376.20000000002	62.3846545084647\\
1376.40000000002	62.3901198692336\\
1376.60000000002	62.3810561487615\\
1376.90000000002	62.3892519970951\\
1376.99999999999	62.3847215162298\\
1377.20000000001	62.3901836927322\\
1377.29999999999	62.3856541309714\\
1377.5	62.3911149825784\\
1377.90000000002	62.3730043541364\\
1377.99999999999	62.3757347072056\\
1378.09999999998	62.3712088231026\\
1378.29999999999	62.3766686012768\\
1378.40000000002	62.3721436343852\\
1378.6	62.3776020889243\\
1378.79999999999	62.3685546450069\\
1379.90000000001	62.3985507246377\\
1380.19999999999	62.3849887705571\\
1380.30000000001	62.3877137061721\\
1380.40000000002	62.3831944947483\\
1380.50000000001	62.3859191655802\\
1380.6	62.3814007387557\\
1380.99999999998	62.3922959959453\\
1381.10000000002	62.3877787431219\\
1381.29999999998	62.3932242652382\\
1381.5	62.3841922408801\\
1381.89999999998	62.3950795947902\\
1382.00000000001	62.3905650821214\\
1382.4	62.4014466546112\\
1382.49999999999	62.396933314046\\
1382.6	62.3996528531135\\
1382.90000000002	62.3861171366594\\
1383.30000000001	62.3969929160041\\
1383.69999999999	62.3789564966036\\
1383.8	62.3816749765157\\
1384.2	62.3636494979412\\
1384.40000000001	62.3690863127483\\
1384.50000000002	62.364581828687\\
1384.8	62.3727344934652\\
1385	62.3637282506678\\
1385.30000000001	62.3718781579327\\
1385.50000000001	62.3628752886836\\
1386.10000000001	62.3791660655028\\
1386.29999999998	62.370167339873\\
1386.49999999999	62.3755949805279\\
1386.60000000002	62.3710968486334\\
1386.80000000001	62.3765231811955\\
1386.90000000001	62.3720259552992\\
1387.00000000001	62.3747386633985\\
1387.09999999999	62.3702422145329\\
1387.70000000002	62.3865110246433\\
1387.80000000002	62.3820159953887\\
1387.9	62.3847262247839\\
1388.09999999999	62.3757383662297\\
1388.40000000001	62.3838674828952\\
1388.60000000001	62.3748829840858\\
1388.69999999998	62.3775921658986\\
1388.89999999998	62.3686105111591\\
1388.99999999998	62.371319559427\\
1389.19999999999	62.3623407471388\\
1390.19999999998	62.3894123570452\\
1390.30000000002	62.3849252013809\\
1390.59999999998	62.3930394765226\\
1390.7	62.3885533505896\\
1390.8	62.3912574591991\\
1390.89999999999	62.3867721063983\\
1390.99999999999	62.3894759542808\\
1391.10000000002	62.3849913743531\\
1391.40000000002	62.3931009701761\\
1391.49999999998	62.3886174187985\\
1391.59999999999	62.391319968384\\
1391.70000000001	62.3868371892513\\
1391.80000000001	62.3895394784108\\
1392.00000000002	62.3805761080382\\
1392.39999999998	62.3913824057451\\
1392.50000000001	62.3869021973288\\
1393.30000000001	62.4084972010909\\
1393.49999999998	62.3995407577497\\
1393.80000000001	62.4076332592008\\
1393.89999999999	62.403156384505\\
1394.1	62.4085497059245\\
1394.19999999999	62.4040737287528\\
1394.49999999999	62.4121611931737\\
1394.6	62.4076862407686\\
1394.70000000002	62.4103814166906\\
1394.80000000001	62.4059072334934\\
1395.3	62.4193779561416\\
1395.50000000001	62.4104327887647\\
1395.7	62.4158188852271\\
1395.8	62.4113475177305\\
1396.00000000001	62.4167323257646\\
1396.19999999999	62.4077920217718\\
1396.49999999999	62.4158671058284\\
1396.69999999999	62.4069301260023\\
1396.79999999999	62.4096213043167\\
1396.90000000002	62.4051539012169\\
1397.19999999998	62.4132255063336\\
1397.29999999999	62.4087591240876\\
1397.40000000002	62.4114490161002\\
1397.6	62.4025184231237\\
1398.2	62.4186512193378\\
1398.50000000001	62.4052624052624\\
1398.90000000002	62.4160114367405\\
1399.09999999998	62.4070897655803\\
1399.30000000002	62.4124624839217\\
1399.50000000001	62.4035438696771\\
1399.60000000002	62.4062299064085\\
1399.69999999999	62.4017716816688\\
1399.80000000002	62.4044574612472\\
1399.89999999999	62.4\\
1400.00000000002	62.4026855224627\\
1400.19999999998	62.3937727629794\\
1400.69999999999	62.407195888064\\
1400.80000000001	62.4027410950104\\
1401.10000000002	62.410790750785\\
1401.2	62.406336972811\\
1401.60000000002	62.4170649925091\\
1401.70000000002	62.4126123555429\\
1402.30000000001	62.4286936679977\\
1402.59999999999	62.4153418407357\\
1402.70000000001	62.4180211006558\\
1402.80000000001	62.4135718868059\\
1403.1	62.4216077537058\\
1403.2	62.4171595524834\\
1403.50000000001	62.4251923624964\\
1403.6	62.4207451734701\\
1403.80000000002	62.4260987249804\\
1403.99999999999	62.4172067516559\\
1404.39999999998	62.4279102883588\\
1404.49999999999	62.4234657553752\\
1404.79999999998	62.4314897857499\\
1404.90000000001	62.4270462633452\\
1405.1	62.4323939652719\\
1405.2	62.4279513271188\\
1405.4	62.4332977588047\\
1405.5	62.4288560045532\\
1405.7	62.4342011665955\\
1405.8	62.4297602958959\\
1406.09999999998	62.4377755653534\\
1406.19999999999	62.4333357036194\\
1406.30000000001	62.4360068259386\\
1406.4	62.431567721294\\
1406.80000000001	62.4422489160566\\
1406.90000000001	62.4378109452736\\
1406.99999999999	62.4404804207235\\
1407.1	62.4360432063673\\
1407.7	62.452052848416\\
1407.79999999998	62.4476170182541\\
1408.00000000001	62.4529507847454\\
1408.09999999999	62.4485158358188\\
1408.60000000001	62.4618442535671\\
1408.90000000001	62.4485450674237\\
1409.5	62.4645289443814\\
1409.6	62.4600978931688\\
1410.6	62.4867087261643\\
1410.7	62.4822795576978\\
1411.10000000001	62.4929138321995\\
1411.2	62.4884857932403\\
1412	62.5097372707315\\
1412.10000000001	62.5053108624841\\
1412.59999999998	62.5185814397961\\
1412.69999999998	62.5141562853907\\
1412.8	62.5168093991082\\
1412.90000000001	62.5123849964614\\
1413.30000000002	62.5229941983869\\
1413.40000000001	62.5185709232402\\
1413.70000000001	62.5265242608573\\
1413.9	62.5176803394625\\
1413.99999999998	62.5203309525493\\
1414.10000000001	62.5159100551549\\
1414.20000000001	62.5185604185816\\
1414.29999999999	62.5141402714932\\
1414.40000000002	62.5167903852952\\
1414.59999999999	62.5079522160175\\
1414.70000000001	62.5106022052587\\
1414.80000000002	62.5061841826277\\
1414.99999999998	62.5114832874002\\
1415.10000000002	62.5070661390616\\
1415.4	62.5150123631226\\
1415.50000000001	62.5105962136197\\
1415.80000000001	62.5185394448761\\
1415.89999999999	62.5141242937853\\
1416.10000000001	62.5194181612767\\
1416.20000000001	62.515003883358\\
1416.29999999998	62.5176503812482\\
1416.39999999999	62.5132368513943\\
1417.09999999999	62.5317527519052\\
1417.20000000001	62.5273407182671\\
1417.4	62.5326278659612\\
1417.6	62.523806164915\\
1417.70000000001	62.5264494286923\\
1417.80000000002	62.5220396360815\\
1417.99999999999	62.527325294408\\
1418.20000000002	62.5185080730452\\
1418.29999999999	62.5211505922166\\
1418.39999999999	62.5167430384209\\
1418.79999999998	62.5273098879414\\
1418.9	62.522903453136\\
1419.50000000001	62.5387433079741\\
1419.6	62.5343382404733\\
1419.70000000001	62.5369770390196\\
1419.80000000002	62.5325727163885\\
1419.99999999999	62.537849447222\\
1420.19999999999	62.5290431598958\\
1420.3	62.5316812165587\\
1420.40000000001	62.5272791270679\\
1420.5	62.5299169365057\\
1420.60000000001	62.5255155909059\\
1420.90000000001	62.533427163969\\
1421.00000000002	62.5290268102174\\
1421.30000000001	62.5369354157873\\
1421.39999999998	62.5325360534647\\
1421.7	62.5404416936278\\
1421.89999999998	62.5316455696203\\
1422.9	62.5579761068166\\
1423.00000000002	62.5535802122128\\
1423.10000000002	62.5562113546936\\
1423.30000000001	62.5474216664325\\
1423.50000000001	62.5526833380163\\
1423.60000000001	62.5482896677671\\
1424.40000000001	62.5693225693226\\
1424.60000000001	62.5605390608549\\
1424.69999999998	62.5631667602471\\
1424.80000000002	62.55877605446\\
1424.99999999999	62.5640305943443\\
1425.09999999999	62.5596407521751\\
1425.20000000001	62.5622675927875\\
1425.30000000002	62.5578784902484\\
1425.39999999998	62.5605050859348\\
1425.49999999998	62.5561167227834\\
1425.89999999999	62.5666199158485\\
1426.10000000001	62.5578460244005\\
1426.20000000002	62.5604711491271\\
1426.39999999999	62.5516999649492\\
1426.5	62.5543249684565\\
1426.60000000001	62.5499404219528\\
1426.70000000001	62.5525651808242\\
1427	62.539415598066\\
1427.3	62.547288776797\\
1427.49999999998	62.5385261978145\\
1428.00000000001	62.5516420418738\\
1428.10000000001	62.5472622881949\\
1428.4	62.5551277563878\\
1428.50000000002	62.5507489850203\\
1429.00000000001	62.5638513749913\\
1429.10000000002	62.5594738315141\\
1429.29999999999	62.5647124667693\\
1429.40000000001	62.5603357817419\\
1429.79999999999	62.5708091474928\\
1429.89999999998	62.5664335664336\\
1430	62.5690511153066\\
1430.10000000001	62.5646762690533\\
1430.50000000001	62.5751432965189\\
1430.69999999998	62.5663964215823\\
1430.80000000002	62.5690125096093\\
1430.89999999999	62.5646401118099\\
1431.1	62.5698714365567\\
1431.2	62.5654998952002\\
1431.39999999998	62.5707300034928\\
1431.50000000002	62.5663593182453\\
1431.6	62.568973947056\\
1431.70000000002	62.5646039949714\\
1431.79999999999	62.5672183811719\\
1431.90000000001	62.5628491620112\\
1432.50000000001	62.5785285494904\\
1432.59999999999	62.5741606756474\\
1432.70000000001	62.5767727526522\\
1432.80000000002	62.5724056109987\\
1432.9	62.5750174459177\\
1432.99999999999	62.5706510362152\\
1433.1	62.5732626290818\\
1433.20000000001	62.5688969510919\\
1433.39999999999	62.5741192884548\\
1433.49999999999	62.5697544642857\\
1433.59999999998	62.5723652089001\\
1433.70000000002	62.5680011159157\\
1433.9	62.5732217573222\\
1434.00000000002	62.5688585175371\\
1434.19999999999	62.5740779474308\\
1434.39999999999	62.5653537818055\\
1434.79999999998	62.5757892536065\\
1435.40000000001	62.5496342737722\\
1435.89999999998	62.5626740947075\\
1436.19999999999	62.5496066281418\\
1436.30000000001	62.5522138680033\\
1436.50000000001	62.5435054990951\\
1436.79999999999	62.5513257707565\\
1437.1	62.5382688561091\\
1437.4	62.5460869565217\\
1437.5	62.5417362270451\\
1437.70000000002	62.5469467241619\\
1437.79999999998	62.5425968426177\\
1438	62.5478061330923\\
1438.10000000001	62.5434570991517\\
1438.79999999998	62.5616790603934\\
1439.10000000002	62.5486381322957\\
1439.30000000002	62.5538418785605\\
1439.39999999998	62.5494963529003\\
1439.59999999998	62.5546988956033\\
1439.8	62.546010139593\\
1440.30000000002	62.5590113857262\\
1440.39999999999	62.5546685178757\\
1440.50000000002	62.5572678050812\\
1440.80000000001	62.544243181345\\
1440.89999999999	62.5468424705066\\
1440.99999999998	62.5425022552217\\
1442.30000000002	62.5762617859124\\
1442.40000000001	62.5719237435009\\
1442.69999999998	62.5797061269753\\
1442.89999999998	62.5710325710326\\
1442.99999999999	62.5736262213291\\
1443.1	62.5692904656319\\
1443.40000000002	62.5770696224454\\
1443.50000000002	62.5727348295927\\
1444.1	62.5882841711674\\
1444.30000000001	62.5796178343949\\
1444.79999999999	62.5925669596512\\
1445.00000000001	62.5839042280811\\
1445.30000000001	62.5916701259167\\
1445.69999999998	62.5743532992115\\
1446.3	62.5898783185841\\
1446.39999999998	62.5855513307985\\
1446.5	62.5881377021983\\
1446.69999999998	62.579485761681\\
1446.90000000001	62.5846579129233\\
1447	62.580333079953\\
1447.20000000002	62.5855040420093\\
1447.40000000001	62.5768566493955\\
1447.69999999998	62.5846111341345\\
1447.89999999999	62.5759668508287\\
1448.00000000002	62.5785512050273\\
1448.2	62.5699095491266\\
1448.60000000002	62.5802443570097\\
1448.69999999999	62.5759249033683\\
1448.80000000002	62.5785078335289\\
1448.99999999998	62.5698709543855\\
1449.19999999999	62.5750362243842\\
1449.3	62.570718918173\\
1449.39999999999	62.5733011383236\\
1449.59999999999	62.5646685521142\\
1450.20000000001	62.5801558298283\\
1450.30000000002	62.5758411472697\\
1450.4	62.5784212340572\\
1450.59999999999	62.5697938926036\\
1451.20000000002	62.585268380073\\
1451.4	62.576644850155\\
1451.7	62.5843780135005\\
1452.19999999998	62.5628313709289\\
1452.3	62.5654089782429\\
1452.7	62.5481828193833\\
1453.09999999999	62.5584916047344\\
1453.20000000002	62.5541870226381\\
1453.70000000002	62.5670656211308\\
1453.8	62.5627622257377\\
1454.40000000001	62.578205568924\\
1454.49999999999	62.5739034786196\\
1454.7	62.5790486664834\\
1454.80000000001	62.57474740532\\
1455.09999999998	62.5824628916987\\
1455.19999999999	62.5781625781626\\
1455.30000000002	62.5807338188814\\
1455.60000000001	62.5678367795562\\
1455.80000000001	62.5729789133869\\
1455.9	62.5686813186813\\
1456.10000000002	62.5738222771597\\
1456.19999999999	62.5695255098537\\
1456.40000000002	62.5746652935118\\
1456.50000000001	62.5703693532885\\
1456.90000000002	62.5806451612903\\
1457.10000000002	62.572055997804\\
1457.4	62.5797598627787\\
1457.50000000001	62.5754665203074\\
1457.59999999999	62.5780338890032\\
1457.79999999999	62.5694492077646\\
1457.89999999999	62.5720164609054\\
1458	62.5677251217338\\
1458.50000000001	62.5805566982038\\
1458.59999999999	62.5762665386988\\
1458.69999999998	62.5788319166438\\
1458.80000000001	62.5745424634999\\
1459.19999999999	62.5848009319537\\
1459.4	62.5762247344981\\
1459.50000000002	62.5787887092354\\
1459.6	62.5745016099198\\
1459.70000000002	62.577065351418\\
1459.79999999998	62.5727789574628\\
1460.00000000002	62.5779056229025\\
1460.3	62.565050671049\\
1460.49999999998	62.5701766397371\\
1460.9	62.5530458590007\\
1461.00000000002	62.5556087878995\\
1461.09999999998	62.5513276758828\\
1461.30000000002	62.5564527165732\\
1461.39999999998	62.5521724255901\\
1461.69999999999	62.559857709673\\
1461.99999999999	62.5470214075645\\
1462.10000000002	62.5495828204076\\
1462.20000000002	62.5453053409013\\
1462.8	62.5606671679541\\
1462.90000000001	62.5563909774436\\
1463.19999999999	62.5640675186223\\
1463.40000000002	62.555517594807\\
1463.59999999998	62.5606340097014\\
1463.69999999999	62.5563601584916\\
1463.80000000001	62.558917958877\\
1464.1	62.5461002595274\\
1464.29999999999	62.5512155148866\\
1464.39999999999	62.5469443496074\\
1464.70000000002	62.5546149645003\\
1464.90000000001	62.5460750853242\\
1465.20000000001	62.5537432607657\\
1465.3	62.549474546199\\
1465.89999999999	62.5648021828104\\
1466.00000000001	62.5605347520633\\
1466.89999999999	62.5835037491479\\
1466.99999999998	62.5792379524232\\
1467.10000000002	62.5817884405671\\
1467.19999999999	62.5775233421931\\
1467.39999999998	62.5826235093697\\
1467.50000000001	62.5783592259471\\
1467.60000000001	62.5809089050896\\
1467.7	62.5766453195258\\
1467.79999999998	62.579194768036\\
1467.90000000002	62.574931880109\\
1468.29999999999	62.5851266684827\\
1468.39999999998	62.5808648280558\\
1469.1	62.598693166349\\
1469.19999999999	62.5944327230654\\
1469.50000000001	62.6020685900925\\
1469.6	62.5978090766823\\
1469.69999999999	62.6003537896313\\
1469.80000000002	62.5960949724471\\
1470.80000000002	62.6215242368618\\
1471.09999999998	62.6087547580207\\
1471.50000000002	62.6189181842892\\
1471.69999999999	62.6104090229651\\
1472	62.6180286665308\\
1472.50000000001	62.5967676218932\\
1473.30000000001	62.6170761504004\\
1473.39999999999	62.6128266033254\\
1473.50000000002	62.6153637350706\\
1473.6	62.611114880912\\
1473.9	62.6187245590231\\
1474.30000000001	62.6017362995117\\
1474.5	62.6068086260681\\
1474.8	62.5940741745203\\
1475.00000000001	62.5991458206223\\
1475.09999999999	62.5949023861171\\
1475.3	62.5999728887082\\
1475.50000000002	62.5914882081865\\
1475.79999999999	62.5990920794092\\
1475.9	62.5948509485095\\
1476.1	62.5999187102019\\
1476.30000000001	62.5914386345164\\
1476.60000000001	62.5990383964245\\
1476.80000000002	62.5905613108538\\
1477.1	62.5981586785811\\
1477.20000000001	62.5939213429906\\
1477.49999999999	62.6015159718462\\
1477.60000000002	62.5972795560669\\
1477.80000000002	62.6023411597537\\
1477.99999999999	62.5938705094378\\
1478.20000000001	62.5989312047622\\
1478.39999999999	62.5904633074062\\
1478.50000000002	62.5929933721088\\
1478.59999999998	62.5887603976466\\
1478.70000000002	62.5912902353259\\
1478.90000000002	62.5828262339419\\
1479.09999999998	62.5878853434289\\
1479.20000000002	62.5836544311499\\
1479.5	62.5912408759124\\
1479.59999999999	62.5870108805839\\
1479.70000000001	62.589539126909\\
1479.79999999999	62.585309818231\\
1480	62.5903655158435\\
1480.10000000001	62.5861370085124\\
1480.4	62.5937183383992\\
1480.5	62.5894907469945\\
1481.4	62.6122173472832\\
1481.59999999998	62.6037659445232\\
1481.7	62.6062896477257\\
1481.90000000001	62.5978407557355\\
1482.69999999999	62.6180199622336\\
1482.79999999999	62.6137972890957\\
1482.89999999999	62.6163182737694\\
1482.99999999999	62.6120962848089\\
1483.19999999998	62.6171374637632\\
1483.40000000001	62.6086956521739\\
1483.50000000001	62.6112159611755\\
1483.69999999999	62.6027766545357\\
1484.2	62.6153742504884\\
1484.30000000001	62.6111560226354\\
1484.39999999999	62.6136746379252\\
1484.60000000002	62.6052401158483\\
1484.70000000001	62.6077586206897\\
1484.80000000002	62.6035423260826\\
1485.00000000001	62.6085785468992\\
1485.09999999999	62.6043630487476\\
1485.19999999999	62.6068807648287\\
1485.3	62.602665948566\\
1485.39999999998	62.6051834399192\\
1485.5	62.6009693053312\\
1485.59999999998	62.6034865719863\\
1485.80000000002	62.5950602328555\\
1486	62.6000942063118\\
1486.3	62.5874596340151\\
1486.49999999999	62.592492936903\\
1486.59999999999	62.5882827739288\\
1487.40000000001	62.6084033613445\\
1487.50000000001	62.6041946759882\\
1487.6	62.6067083417356\\
1488.09999999998	62.5856739685526\\
1488.49999999999	62.5957275292221\\
1488.59999999998	62.591522805132\\
1488.90000000001	62.5990597716588\\
1488.99999999998	62.5948559532603\\
1489.09999999998	62.597367714209\\
1489.50000000001	62.5805585392052\\
1489.8	62.5880931606148\\
1490.00000000002	62.579692638078\\
1490.10000000001	62.5822037310428\\
1490.4	62.569607514257\\
1490.89999999998	62.5821596244132\\
1491.10000000001	62.5737660944206\\
1491.2	62.5762757325823\\
1491.30000000001	62.5720799249028\\
1491.5	62.5770984178064\\
1491.70000000002	62.5687089422175\\
1492.09999999998	62.5787427958719\\
1492.4	62.5661641541039\\
1492.80000000002	62.5761939848617\\
1492.9	62.5720026791695\\
1493.09999999999	62.5770158049826\\
1493.4	62.5644459323736\\
1493.9	62.5769745649264\\
1494.1	62.5685985811806\\
1494.19999999999	62.5711035267349\\
1494.29999999998	62.5669164882227\\
1494.69999999998	62.5769333690126\\
1494.80000000001	62.5727473409593\\
1494.9	62.5752508361204\\
1495.29999999999	62.5585127725023\\
1495.8	62.5710274750986\\
1495.90000000002	62.5668449197861\\
1496.5	62.5818521983162\\
1496.69999999999	62.5734901122395\\
1496.79999999999	62.5759903801189\\
1496.90000000002	62.5718102872411\\
1497.20000000002	62.5793094236292\\
1497.3	62.5751302257246\\
1497.90000000001	62.5901201602136\\
1497.99999999998	62.585942193445\\
1498.2	62.5909363945805\\
1498.30000000002	62.5867592098238\\
1498.8	62.5992394422577\\
1499.2	62.5825385179751\\
1499.69999999998	62.5950126683558\\
1499.90000000002	62.5866666666667\\
1500	62.5891607226185\\
1500.10000000001	62.5849886681776\\
1500.4	62.592469176941\\
1500.50000000002	62.5882980141277\\
1500.59999999999	62.5907909642167\\
1500.70000000002	62.5866204690832\\
1501.49999999998	62.6065530101225\\
1501.69999999999	62.5982154747636\\
1501.80000000001	62.6007057726879\\
1501.99999999999	62.5923706810465\\
1502.3	62.5998402555911\\
1502.4	62.5956738768719\\
1502.5	62.5981631838147\\
1502.60000000001	62.5939974712185\\
1502.70000000001	62.5964865584243\\
1502.89999999999	62.5881570192947\\
1503.30000000002	62.5981109485167\\
1503.50000000001	62.5897845171588\\
1503.90000000001	62.5997340425532\\
1503.99999999998	62.5955721029187\\
1504.20000000002	62.6005451040351\\
1504.3	62.5963839404414\\
1504.39999999999	62.5988700564972\\
1504.5	62.5947095573574\\
1504.60000000001	62.5971954542434\\
1504.7	62.5930356193514\\
1504.99999999998	62.6004916616836\\
1505.19999999998	62.5921743174118\\
1505.8	62.607078823295\\
1506	62.5987650222429\\
1506.1	62.6012481742133\\
1506.19999999999	62.5970922127066\\
1506.3	62.5995751460435\\
1506.50000000002	62.5912651002257\\
1506.60000000002	62.5937479259308\\
1506.69999999999	62.589593841253\\
1507.20000000002	62.6020035825648\\
1507.30000000002	62.5978506036885\\
1507.39999999999	62.6003316749585\\
1507.7	62.5878763761772\\
1508.00000000001	62.5953186128241\\
1508.09999999999	62.591168280069\\
1508.2	62.5936484784194\\
1508.39999999998	62.5853496851177\\
1509.3	62.6076586723201\\
1509.50000000001	62.5993640699523\\
1509.59999999998	62.6018414254488\\
1509.69999999999	62.5976950589482\\
1509.90000000003	62.6026490066225\\
1509.99999999999	62.5985034103702\\
1510.2	62.6034562669668\\
1510.70000000001	62.5827376224517\\
1510.9	62.5876902713435\\
1511.10000000002	62.5794070937004\\
1511.40000000002	62.5868342705921\\
1511.59999999999	62.5785539458887\\
1511.70000000002	62.5810292366715\\
1511.90000000002	62.5727513227513\\
1512.09999999999	62.5777013622537\\
1512.19999999999	62.5735634464061\\
1512.40000000001	62.5785123966942\\
1512.59999999998	62.5702386461294\\
1512.79999999998	62.575186727477\\
1512.90000000001	62.571050892267\\
1513.10000000002	62.5759978852762\\
1513.19999999998	62.5718628163616\\
1513.3	62.5743359323378\\
1513.60000000002	62.5619343330911\\
1513.69999999998	62.5644074514467\\
1513.79999999999	62.560274786974\\
1514.09999999998	62.5676925108968\\
1514.2	62.5635607211253\\
1514.30000000001	62.5660327522451\\
1514.49999999998	62.5577710286544\\
1515.00000000001	62.5701273843311\\
1515.2	62.5618689368442\\
1515.3	62.5643394483305\\
1515.40000000002	62.5602111514352\\
1515.6	62.5651514151877\\
1515.69999999999	62.5610238817786\\
1516.1	62.5709009365519\\
1516.2	62.5667743850162\\
1516.29999999998	62.5692429438143\\
1516.69999999998	62.5527426160338\\
1516.80000000002	62.5552112861757\\
1516.90000000002	62.5510876730389\\
1517.19999999998	62.5584920582614\\
1517.40000000002	62.5502471169687\\
1517.49999999998	62.5527148128624\\
1517.59999999999	62.5485932661264\\
1518.89999999999	62.5806451612903\\
1519.00000000002	62.5765255743532\\
1519.1	62.5789889415482\\
1519.3	62.5707516124786\\
1519.99999999999	62.5879876323926\\
1520.20000000002	62.5797539959219\\
1520.30000000002	62.5822152065246\\
1520.39999999999	62.5780993094377\\
1520.80000000002	62.5879413505161\\
1520.89999999999	62.5838264299803\\
1520.99999999998	62.5862862402209\\
1521.1	62.5821719694978\\
1521.30000000001	62.5870908373866\\
1521.70000000001	62.5706400315416\\
1521.90000000001	62.5755584756899\\
1522.00000000002	62.5714473424874\\
1522.29999999998	62.5788229111929\\
1522.40000000001	62.5747126436782\\
1522.49999999999	62.5771706291869\\
1522.69999999999	62.5689519306541\\
1522.90000000001	62.5738673670387\\
1523	62.5697590440549\\
1523.20000000001	62.5746734064203\\
1523.40000000002	62.5664588119462\\
1523.59999999998	62.5713723173853\\
1523.80000000002	62.5631603123565\\
1523.90000000001	62.5656167979003\\
1524.10000000002	62.5574071644141\\
1524.2	62.5598635439218\\
1524.49999999999	62.5475534566444\\
1524.6	62.5500098380009\\
1524.79999999998	62.5418060200669\\
1525.00000000001	62.5467182479837\\
1525.09999999999	62.5426173616575\\
1525.49999999998	62.5524383848978\\
1525.70000000002	62.5442390876917\\
1525.80000000002	62.5466937545055\\
1525.99999999999	62.5384968219645\\
1526.2	62.5434056214375\\
1526.3	62.5393081761006\\
1526.70000000002	62.5491223473932\\
1526.79999999999	62.5450258694086\\
1526.90000000001	62.5474787164375\\
1526.99999999998	62.5433828825879\\
1527.3	62.5507398193008\\
1527.50000000001	62.5425504058654\\
1527.6	62.5450022910257\\
1527.8	62.5368152365993\\
1528.19999999998	62.5466204279265\\
1528.49999999998	62.5343451524271\\
1529.00000000001	62.5465960368844\\
1529.1	62.5425058854303\\
1529.3	62.5474042108016\\
1529.39999999999	62.543314808761\\
1529.60000000002	62.5482120677257\\
1529.70000000002	62.5441234148255\\
1529.79999999998	62.5465716713511\\
1529.90000000002	62.5424836601307\\
1530.10000000001	62.5473794275258\\
1530.29999999999	62.5392054364872\\
1530.70000000002	62.5489939900706\\
1530.79999999998	62.5449082239206\\
1531.00000000002	62.5498007968127\\
1531.3	62.5375473423012\\
1531.70000000001	62.5473299386343\\
1531.8	62.5432469482342\\
1531.99999999998	62.548136544612\\
1532.09999999998	62.5440543010051\\
1532.30000000001	62.548942834769\\
1532.40000000001	62.5448613376835\\
1532.8	62.5546350055451\\
1532.89999999998	62.5505544683627\\
1533	62.5529971952254\\
1533.10000000001	62.5489172971563\\
1533.20000000002	62.5513598121698\\
1533.30000000002	62.5472805530194\\
1533.7	62.5570478550006\\
1533.9	62.5488917861799\\
1534.7	62.5684128225176\\
1534.8	62.5643364388559\\
1535	62.5692137320044\\
1535.10000000002	62.5651380927567\\
1535.8	62.5821993619376\\
1535.9	62.578125\\
1536.09999999998	62.5829970055982\\
1536.30000000001	62.5748502994012\\
1536.39999999998	62.5772860397006\\
1536.60000000002	62.569141667209\\
1537.19999999999	62.5837507318025\\
1537.40000000001	62.5756097560976\\
1537.59999999999	62.5804773362815\\
1537.7	62.5764078553778\\
1537.80000000001	62.5788412770661\\
1537.99999999999	62.5707041154671\\
1538.30000000001	62.5780031201248\\
1538.40000000001	62.5739356516087\\
1538.60000000002	62.5788002859557\\
1538.69999999999	62.5747335586171\\
1538.99999999999	62.5820284581899\\
1539.10000000001	62.5779625779626\\
1539.59999999999	62.5901149574592\\
1539.79999999998	62.5819858432366\\
1539.89999999998	62.5844155844156\\
1539.99999999999	62.5803519251997\\
1540.30000000002	62.5876395741366\\
1540.40000000002	62.583576760792\\
1540.69999999998	62.5908618899273\\
1540.80000000001	62.5867999221234\\
1541.10000000002	62.5940825330911\\
1541.30000000002	62.5859608148436\\
1541.39999999999	62.5883879338307\\
1541.49999999998	62.5843279709393\\
1541.60000000001	62.5867548809755\\
1541.69999999998	62.5826955506551\\
1542.00000000002	62.5899747098113\\
1542.20000000002	62.5818582636322\\
1542.40000000002	62.5867098865478\\
1542.70000000002	62.5745397977703\\
1543.60000000001	62.5963593962557\\
1543.7	62.5923047026817\\
1543.8	62.5947276378004\\
1543.89999999999	62.5906735751295\\
1544.19999999999	62.5979408146086\\
1544.29999999998	62.5938875938876\\
1544.4	62.5963094852703\\
1544.49999999999	62.592256894989\\
1544.70000000001	62.5970999482134\\
1544.79999999998	62.5930480937278\\
1544.9	62.5954692556634\\
1544.99999999998	62.5914180311954\\
1545.09999999999	62.5938389852446\\
1545.2	62.5897883906038\\
1545.29999999999	62.5922091367931\\
1545.39999999998	62.5881591717891\\
1545.79999999998	62.5978394462772\\
1546.00000000002	62.589741931311\\
1546.40000000002	62.5994180407372\\
1546.7	62.5872769588829\\
1547	62.5945317044794\\
1547.10000000001	62.5904860392968\\
1547.7	62.6049877245122\\
1547.8	62.6009432133859\\
1547.9	62.6033591731266\\
1548.10000000002	62.5952719286914\\
1548.40000000002	62.6025185663545\\
1548.50000000001	62.5984760428774\\
1548.9	62.6081342801808\\
1549	62.6040926989865\\
1549.10000000002	62.6065065840434\\
1549.19999999999	62.6024656296392\\
1549.30000000002	62.6048793081193\\
1549.4	62.6008389803162\\
1549.60000000001	62.6056656126992\\
1549.7	62.6016260162602\\
1549.9	62.6064516129032\\
1550.09999999998	62.5983744033028\\
1550.29999999998	62.6031991744066\\
1550.40000000002	62.5991615607868\\
1550.70000000002	62.6063966984782\\
1550.80000000001	62.6023599200464\\
1550.99999999999	62.6071819998711\\
1551.09999999999	62.6031459515214\\
1551.50000000002	62.6127868007218\\
1551.79999999999	62.6006830337006\\
1552.5	62.6175447636223\\
1552.60000000002	62.6135119469312\\
1552.7	62.6159196290572\\
1552.79999999998	62.6118874364093\\
1552.89999999999	62.6142949130715\\
1553.1	62.6062322946176\\
1553.20000000002	62.6086396703792\\
1553.3	62.6046092442384\\
1553.60000000001	62.6118298255777\\
1553.80000000001	62.6037711564451\\
1554.10000000001	62.6109895766311\\
1554.19999999999	62.606961333076\\
1554.69999999999	62.6189863648058\\
1554.79999999998	62.6149591613609\\
1555.49999999999	62.6317819490872\\
1555.69999999998	62.6237305566268\\
1556.60000000002	62.6453395002248\\
1556.69999999999	62.6413155190134\\
1556.80000000002	62.6437150748282\\
1556.90000000001	62.6396917148362\\
1557	62.6420910667266\\
1557.19999999999	62.6340461054389\\
1557.30000000002	62.6364453576474\\
1557.4	62.6324237560193\\
1557.5	62.6348228043143\\
1557.59999999998	62.6308018232009\\
1557.79999999999	62.6355992040567\\
1558.09999999999	62.6235399820306\\
1558.29999999999	62.6283367556468\\
1558.50000000001	62.6203002694726\\
1558.79999999998	62.6274937455898\\
1558.90000000002	62.6234765875561\\
1559	62.6258739016099\\
1559.29999999998	62.6138258304476\\
1559.39999999998	62.616223148445\\
1559.69999999998	62.6041800230799\\
1559.90000000001	62.6089743589744\\
1559.99999999999	62.6049612204346\\
1560.09999999999	62.6073580310217\\
1560.40000000001	62.5953220121756\\
1560.89999999999	62.6073030108905\\
1561.00000000001	62.6032925501249\\
1561.19999999999	62.60808300775\\
1561.29999999998	62.6040732675804\\
1561.4	62.6064681396093\\
1561.50000000002	62.6024590163934\\
1561.70000000002	62.6072480471251\\
1561.80000000002	62.6032396440233\\
1562	62.6080276550797\\
1562.19999999998	62.6000128016386\\
1562.29999999999	62.6024065540195\\
1562.5	62.5943939587866\\
1562.99999999998	62.6063591580833\\
1563.09999999998	62.6023541453429\\
1563.20000000002	62.6047463698586\\
1563.40000000001	62.5967380876239\\
1563.69999999999	62.6039135439314\\
1563.80000000001	62.5999104802097\\
1564.5	62.6166432314969\\
1564.59999999999	62.6126414009075\\
1564.89999999998	62.6198083067093\\
1565	62.6158072966584\\
1565.79999999999	62.6349064435788\\
1565.89999999998	62.6309067688378\\
1566.39999999998	62.6428343440792\\
1566.5	62.638835695136\\
1566.59999999999	62.641220399566\\
1566.70000000001	62.6372223640541\\
1566.8	62.6396068670624\\
1566.89999999998	62.635609444799\\
1567.00000000001	62.6379937464106\\
1567.2	62.6300006380399\\
1567.29999999998	62.6323848411382\\
1567.39999999998	62.6283891547049\\
1567.5	62.6307731564175\\
1567.69999999998	62.6227835183059\\
1568.50000000002	62.6418462323091\\
1568.70000000001	62.6338602753697\\
1569.2	62.645765627987\\
1569.3	62.6417739263413\\
1569.49999999998	62.6465341488277\\
1569.60000000002	62.6425431611136\\
1569.80000000001	62.6473023759475\\
1569.90000000002	62.6433121019108\\
1570.00000000002	62.6456913572384\\
1570.1	62.6417016940517\\
1570.30000000002	62.6464595007641\\
1570.40000000002	62.64247055078\\
1570.50000000001	62.6448491022539\\
1570.59999999999	62.6408607627173\\
1570.70000000003	62.6432391138274\\
1570.80000000002	62.6392513845566\\
1571	62.644007383362\\
1571.19999999998	62.6360338573156\\
1571.7	62.6479195826441\\
1571.79999999998	62.6439340924995\\
1572.00000000002	62.6486864703263\\
1572.09999999999	62.6447016918967\\
1572.2	62.6470775297335\\
1572.30000000001	62.6430933604681\\
1572.39999999998	62.6454689984102\\
1572.49999999998	62.6414854381279\\
1572.90000000001	62.6509853782581\\
1573.00000000002	62.6470027334562\\
1573.70000000002	62.6636167238531\\
1573.90000000001	62.6556543837357\\
1574	62.6580268089702\\
1574.10000000001	62.6540464998094\\
1574.30000000001	62.6587906504065\\
1574.39999999998	62.6548110511274\\
1574.69999999998	62.6619253238506\\
1574.89999999998	62.6539682539683\\
1575.29999999998	62.6634505522407\\
1575.6	62.6515199593831\\
1575.70000000001	62.6538900875746\\
1575.90000000001	62.6459390862944\\
1576.19999999999	62.6530482776121\\
1576.3	62.6490738391271\\
1576.89999999999	62.6632847178187\\
1577.10000000002	62.6553385746893\\
1577.6	62.6671737339165\\
1577.9	62.6552598225602\\
1578.10000000003	62.659992396401\\
1578.30000000001	62.6520527116067\\
1578.50000000002	62.6567844925884\\
1578.7	62.6488472257411\\
1580.29999999999	62.686661604657\\
1580.49999999999	62.6787295963558\\
1581.10000000002	62.6928914748293\\
1581.59999999999	62.6730732755896\\
1582.10000000001	62.6848691695108\\
1582.2	62.6809075396575\\
1582.5	62.6879818020978\\
1582.70000000002	62.6800606520091\\
1582.89999999999	62.6847757422615\\
1583.1	62.6768569984841\\
1583.19999999998	62.6792142992484\\
1583.30000000002	62.6752557787041\\
1583.5	62.6799696893155\\
1583.69999999998	62.6720545523425\\
1584.19999999999	62.6838351322351\\
1584.30000000002	62.6798788184802\\
1584.59999999998	62.6869439010538\\
1584.69999999999	62.6829883897022\\
1585	62.6900511008769\\
1585.19999999998	62.6821421812906\\
1585.29999999999	62.6844960262394\\
1585.40000000001	62.6805424156418\\
1585.9	62.6923076923077\\
1586.20000000002	62.6804513648112\\
1586.40000000002	62.6851560037819\\
1586.59999999998	62.6772546795235\\
1587.20000000002	62.6913626913627\\
1587.6	62.6755684323235\\
1587.69999999998	62.6779191333921\\
1587.79999999999	62.673971912589\\
1587.89999999998	62.676322418136\\
1587.99999999998	62.6723757949751\\
1588.09999999998	62.6747261050246\\
1588.20000000001	62.6707800793301\\
1588.30000000002	62.6731301939058\\
1588.39999999999	62.6691847655021\\
1588.6	62.6738843079247\\
1588.70000000003	62.6699395770393\\
1588.80000000001	62.672289004972\\
1588.89999999999	62.6683448709881\\
1589.00000000001	62.6706941035807\\
1589.1	62.6667505663227\\
1589.2	62.6690996035991\\
1589.49999999998	62.6572722697534\\
1589.80000000001	62.6643185105982\\
1589.89999999999	62.6603773584906\\
1590	62.6627256147412\\
1590.10000000002	62.6587850584832\\
1590.19999999999	62.6611331195372\\
1590.30000000001	62.6571931589537\\
1590.49999999999	62.6618885954986\\
1590.59999999998	62.6579493304834\\
1590.9	62.6649905719673\\
1591.20000000002	62.6531766480236\\
1591.70000000001	62.664907651715\\
1591.79999999998	62.6609711665306\\
1592.20000000002	62.6703510644979\\
1592.4	62.6624803767661\\
1592.60000000003	62.6671689583726\\
1592.70000000001	62.6632345554998\\
1592.89999999998	62.6679221594476\\
1593.00000000002	62.6639884501915\\
1593.39999999999	62.6733605271415\\
1593.50000000001	62.6694277108434\\
1593.59999999998	62.6717700947481\\
1593.69999999999	62.6678378717531\\
1593.80000000003	62.6701800614844\\
1593.99999999998	62.6623172950254\\
1594.09999999998	62.6646593902898\\
1594.2	62.6607288465157\\
1594.40000000001	62.6654123549702\\
1594.49999999999	62.6614825034491\\
1594.8	62.6685058624365\\
1594.9	62.6645768025078\\
1594.99999999998	62.6669174346436\\
1595.29999999998	62.6551335088379\\
1595.40000000001	62.6574741460357\\
1595.80000000002	62.641769534432\\
1595.90000000002	62.6441102756892\\
1596.20000000002	62.6323372799599\\
1596.29999999998	62.6346780255575\\
1596.6	62.6229097513622\\
1596.99999999998	62.6322709911715\\
1597.10000000002	62.6283496118207\\
1597.19999999999	62.6306892881738\\
1597.39999999998	62.622848200313\\
1597.59999999999	62.6275270701634\\
1597.69999999999	62.6236074602579\\
1597.90000000002	62.6282853566959\\
1598.00000000001	62.6243664351417\\
1598.09999999999	62.6267050431736\\
1598.19999999998	62.6227867108803\\
1598.80000000001	62.6368128088061\\
1598.89999999998	62.6328955597248\\
1599.29999999999	62.6422408403151\\
1599.40000000001	62.6383244763989\\
1599.59999999998	62.6429955616678\\
1599.79999999998	62.6351646977936\\
1600.39999999999	62.6491721337082\\
1600.59999999998	62.6413444118198\\
1600.69999999999	62.6436781609195\\
1600.80000000002	62.6397651321132\\
1601.09999999999	62.6467649263053\\
1601.19999999998	62.6428526821957\\
1601.59999999998	62.6521820565649\\
1601.70000000001	62.6482706954676\\
1601.80000000002	62.6506024096386\\
1601.90000000002	62.6466916354557\\
1602.19999999999	62.6536853273419\\
1602.30000000002	62.6497753369945\\
1602.59999999999	62.6567667061833\\
1602.79999999999	62.6489487803356\\
1602.90000000002	62.6512788521522\\
1603.10000000001	62.6434630738523\\
1603.19999999998	62.6457930518306\\
1603.3	62.6418859922664\\
1603.40000000001	62.6442157779857\\
1603.49999999999	62.6403093040659\\
1603.70000000001	62.6449682005238\\
1603.90000000003	62.6371571072319\\
1604.50000000002	62.6511280069799\\
1604.60000000002	62.6472237801458\\
1604.9	62.6542056074766\\
1605.00000000001	62.6503021618591\\
1605.30000000002	62.6572816743491\\
1605.50000000001	62.6494768310912\\
1605.69999999999	62.654128783161\\
1605.99999999998	62.6424257518212\\
1606.1	62.6447515875981\\
1606.2	62.6408516466414\\
1606.80000000002	62.6548011699546\\
1607	62.6470039201045\\
1607.40000000002	62.656298600311\\
1607.50000000003	62.6524010947997\\
1607.7	62.6570468963802\\
1607.79999999999	62.6531500715219\\
1608.2	62.6624385997637\\
1608.39999999998	62.65464718682\\
1608.70000000001	62.661611138737\\
1608.79999999998	62.6577164522345\\
1609.00000000001	62.6623578397862\\
1609.1	62.6584638329605\\
1609.29999999999	62.6631042624581\\
1609.40000000002	62.659210935073\\
1609.89999999998	62.6708074534161\\
1610.1	62.6630232269283\\
1610.59999999999	62.6746135220712\\
1610.69999999998	62.6707226222995\\
1610.80000000002	62.6730399155751\\
1610.89999999999	62.6691495965239\\
1611	62.6714666997703\\
1611.10000000003	62.6675769612711\\
1611.3	62.6722105001862\\
1611.39999999998	62.6683214396525\\
1611.79999999998	62.677585458155\\
1611.89999999998	62.6736972704715\\
1612.00000000001	62.6760126543018\\
1612.1	62.6721250465203\\
1612.19999999998	62.67444024065\\
1612.40000000001	62.6666666666667\\
1612.50000000001	62.6689817685725\\
1612.70000000002	62.6612103174603\\
1612.90000000002	62.665840049597\\
1613	62.6619552414605\\
1613.10000000001	62.6642697743615\\
1613.20000000002	62.6603855451559\\
1613.29999999999	62.6626998884344\\
1613.4	62.6588162379919\\
1613.50000000003	62.6611303916708\\
1613.59999999998	62.657247319824\\
1613.69999999999	62.6595612839261\\
1613.90000000002	62.6517967781908\\
1613.99999999999	62.6541106498978\\
1614.1	62.6502292157106\\
1614.30000000001	62.6548562933598\\
1614.40000000001	62.6509755342211\\
1614.49999999999	62.6532887402453\\
1614.59999999999	62.6494085588654\\
1614.90000000002	62.656346749226\\
1615.1	62.648588410104\\
1615.2	62.6509007614685\\
1615.29999999998	62.6470224093104\\
1615.69999999999	62.6562693402649\\
1615.80000000002	62.6523918559317\\
1615.89999999998	62.654702970297\\
1616.00000000001	62.6508260627436\\
1616.2	62.6554476272969\\
1616.30000000001	62.6515713932195\\
1616.69999999998	62.6608114794656\\
1616.89999999999	62.6530612244898\\
1617.30000000002	62.6622975145295\\
1617.59999999998	62.6506768869383\\
1617.7	62.652985535913\\
1617.80000000001	62.649113047778\\
1617.9	62.6514215080346\\
1618.19999999999	62.6398072050918\\
1618.50000000002	62.646731743482\\
1618.80000000002	62.6351226141207\\
1619.00000000003	62.6397381261195\\
1619.09999999999	62.6358695652174\\
1619.30000000002	62.6404841299247\\
1619.70000000001	62.6250154340042\\
1619.79999999998	62.6273226742392\\
1620	62.619591383248\\
1620.10000000001	62.6218985310456\\
1620.29999999999	62.6141693409035\\
1620.70000000001	62.6233958538993\\
1620.79999999999	62.6195323585662\\
1621.29999999998	62.6310595781423\\
1621.39999999999	62.627197039778\\
1621.49999999998	62.6295017266897\\
1621.60000000003	62.6256397607449\\
1622.29999999999	62.6417652859961\\
1622.39999999998	62.6379044684129\\
1622.49999999998	62.6402070750647\\
1622.60000000003	62.6363468293585\\
1622.80000000001	62.6409513833262\\
1622.90000000001	62.6370918052988\\
1623	62.6393937526955\\
1623.10000000001	62.6355347461804\\
1623.49999999999	62.6447400837645\\
1623.69999999999	62.6370242640719\\
1623.89999999997	62.6416256157636\\
1623.99999999999	62.6377686103072\\
1624.9	62.6584615384615\\
1624.99999999999	62.654605870408\\
1625.20000000001	62.6592013782071\\
1625.30000000001	62.6553463762766\\
1625.69999999999	62.6645343830729\\
1625.80000000001	62.6606802386371\\
1626.00000000002	62.665272738454\\
1626.20000000003	62.6575662546886\\
1626.39999999999	62.6621580079926\\
1626.5	62.6583056682651\\
1626.7	62.6628964838948\\
1626.90000000001	62.6551936078672\\
1627.19999999998	62.6620782891907\\
1627.29999999998	62.6582278481013\\
1627.8	62.669697155845\\
1627.90000000001	62.6658476658477\\
1628.00000000002	62.6681407775935\\
1628.10000000002	62.6642918560374\\
1628.30000000003	62.6688774256939\\
1628.40000000002	62.665029167946\\
1629.19999999998	62.6833609525563\\
1629.30000000002	62.6795139315085\\
1629.40000000001	62.6818042344277\\
1629.5	62.6779577810506\\
1629.90000000003	62.6871165644172\\
1630.19999999999	62.6755811813777\\
1630.50000000001	62.6824481785846\\
1630.60000000001	62.6786042803704\\
1630.69999999999	62.6808928133431\\
1630.79999999999	62.6770494818812\\
1630.9	62.6793378295524\\
1631.1	62.6716527709662\\
1631.19999999998	62.6739410286275\\
1631.30000000001	62.6700993012137\\
1631.40000000003	62.6723873735826\\
1631.49999999998	62.6685462123069\\
1631.7	62.6731217060914\\
1631.80000000001	62.6692812059563\\
1632.1	62.6761426295797\\
1632.29999999998	62.6684636118598\\
1632.80000000001	62.6798946659318\\
1632.90000000002	62.6760563380282\\
1633.09999999999	62.6806269899584\\
1633.19999999999	62.6767893222311\\
1633.9	62.6927784577723\\
1634.20000000003	62.6812702686165\\
1634.49999999998	62.6881194175945\\
1634.60000000002	62.6842845782101\\
1634.90000000001	62.691131498471\\
1635.00000000002	62.6872974130023\\
1635.40000000002	62.6964231121981\\
1635.50000000001	62.6925898752751\\
1636.10000000001	62.7062706270627\\
1636.20000000001	62.7024384281611\\
1636.30000000002	62.7047176729406\\
1636.59999999998	62.6932241705872\\
1636.70000000003	62.6955034213099\\
1636.79999999998	62.6916732848677\\
1636.90000000001	62.6939523518632\\
1637.00000000002	62.6901227780832\\
1637.50000000002	62.7015144113337\\
1637.70000000003	62.6938576138723\\
1637.89999999999	62.6984126984127\\
1638.00000000002	62.6945851901593\\
1638.1	62.6968624099622\\
1638.4	62.6853829722307\\
1638.5	62.6876601977298\\
1638.59999999998	62.6838347470556\\
1638.99999999998	62.692941248246\\
1639.30000000002	62.6814688300598\\
1639.99999999998	62.6973965002134\\
1640.40000000003	62.6821091130753\\
1640.50000000002	62.6843837620383\\
1640.69999999999	62.6767430521697\\
1640.9	62.6812918951859\\
1641.10000000002	62.6736534243237\\
1641.30000000002	62.6782015352748\\
1641.39999999999	62.6743831861103\\
1641.69999999998	62.6812035570715\\
1641.79999999999	62.6773859552957\\
1642.00000000002	62.6819316728579\\
1642.10000000001	62.6781147241505\\
1642.29999999998	62.6826595226498\\
1642.50000000002	62.6750273955924\\
1642.60000000002	62.6772995677848\\
1642.8	62.669669486883\\
1643.00000000002	62.674213377153\\
1643.30000000001	62.6627723013265\\
1643.49999999998	62.6673156485763\\
1643.70000000001	62.6596909599708\\
1643.99999999998	62.666504470531\\
1644.19999999998	62.6588821991121\\
1644.29999999999	62.6611530041353\\
1644.40000000001	62.6573426573427\\
1645.09999999998	62.673231218089\\
1645.3	62.6656132247478\\
1645.40000000001	62.6678821027043\\
1645.49999999998	62.6640738940204\\
1645.79999999998	62.6708791542621\\
1646.09999999998	62.6594581460333\\
1646.29999999998	62.6639941690962\\
1646.39999999998	62.6601882781658\\
1646.89999999999	62.6715239829994\\
1647.2	62.660110483822\\
1647.8	62.6737059287578\\
1647.99999999998	62.666100357988\\
1648.2	62.6706303464175\\
1648.30000000001	62.6668284396991\\
1648.39999999999	62.669093114953\\
1648.59999999999	62.6614908715958\\
1648.9	62.6682838083687\\
1649.00000000002	62.6644836577527\\
1649.19999999998	62.6690110956163\\
1649.39999999999	62.6614125492574\\
1649.49999999998	62.663676042677\\
1649.89999999998	62.6484848484849\\
1650.00000000002	62.6507484394885\\
1650.10000000003	62.6469518846201\\
1650.30000000001	62.6514784294716\\
1650.49999999999	62.643887071368\\
1651.10000000002	62.6574612403101\\
1651.20000000001	62.6536668079695\\
1651.80000000001	62.6672316726194\\
1652.10000000002	62.6558528023242\\
1652.60000000001	62.6671507230592\\
1653.00000000001	62.6519871756095\\
1653.7	62.6677953803362\\
1653.99999999998	62.6564294782661\\
1654.1	62.6586869785999\\
1654.20000000001	62.6548993532008\\
1654.39999999999	62.6594137201571\\
1654.50000000002	62.6556267375801\\
1655.39999999999	62.6759287224404\\
1655.5	62.6721430297173\\
1655.6	62.6743975357855\\
1655.70000000003	62.6706123928011\\
1656.09999999998	62.6796280642435\\
1656.19999999998	62.6758437481133\\
1656.3	62.6780970780005\\
1656.39999999999	62.6743133111983\\
1657.89999999998	62.708082026538\\
1657.99999999998	62.704300102527\\
1658.3	62.7110467920888\\
1658.40000000001	62.7072656014471\\
1658.99999999999	62.7207522150564\\
1659.39999999998	62.7056342271769\\
1659.50000000002	62.707881417209\\
1659.70000000002	62.7003253404025\\
1660.60000000003	62.7205395315229\\
1660.70000000002	62.7167630057804\\
1660.9	62.7212522576761\\
1661	62.7174763710794\\
1661.59999999998	62.7309381958236\\
1661.69999999997	62.7271633168853\\
1661.79999999998	62.7294061014502\\
1662.2	62.7143114961198\\
1662.30000000002	62.7165543792108\\
1662.40000000001	62.7127819548872\\
1662.50000000001	62.7150246601708\\
1662.60000000001	62.7112527816203\\
1662.69999999998	62.7134953091171\\
1662.90000000002	62.705953096813\\
1663.19999999999	62.7126796128179\\
1663.49999999999	62.70137052176\\
1663.59999999998	62.7036124301256\\
1663.7	62.6998437312177\\
1663.99999999999	62.706568114897\\
1664.20000000002	62.6990326263294\\
1664.70000000002	62.7102354637194\\
1665.09999999999	62.695171751141\\
1665.49999999999	62.7041306436119\\
1665.89999999998	62.6890756302521\\
1666.50000000002	62.702508100324\\
1666.79999999998	62.691223228748\\
1666.90000000001	62.6934613077385\\
1666.99999999998	62.6897006778238\\
1667.1	62.6919385796545\\
1667.29999999999	62.6844188557035\\
1667.40000000001	62.6866566716642\\
1667.80000000001	62.6716229989808\\
1667.90000000003	62.673860911271\\
1667.99999999999	62.6701037108087\\
1668.89999999998	62.690233672858\\
1669.2	62.6789672317738\\
1669.89999999998	62.6946107784431\\
1669.99999999998	62.6908568349201\\
1670.09999999998	62.6930906478266\\
1670.19999999998	62.6893372448063\\
1670.30000000001	62.6915708812261\\
1670.40000000002	62.6878180185573\\
1670.50000000001	62.6900514785107\\
1670.80000000001	62.6787958585194\\
1671.19999999999	62.687728115838\\
1671.4	62.6802273407119\\
1671.49999999999	62.6824599186408\\
1671.6	62.6787102949094\\
1671.69999999997	62.6809426964948\\
1671.80000000001	62.6771936120581\\
1672.1	62.6838894869035\\
1672.30000000002	62.6763932073667\\
1673	62.692008845855\\
1673.7	62.6657904170152\\
1673.79999999999	62.6680207897724\\
1673.89999999998	62.6642771804062\\
1674.00000000002	62.6665073770982\\
1674.10000000001	62.6627643053399\\
1674.3	62.6672240802676\\
1674.40000000002	62.6634816363094\\
1674.80000000001	62.6723983521404\\
1675.00000000002	62.6649155274312\\
1675.40000000001	62.6738287078484\\
1675.49999999998	62.6700883265696\\
1675.6	62.6723160470251\\
1675.79999999999	62.6648368041053\\
1675.99999999998	62.6692918083647\\
1676.09999999998	62.6655530366305\\
1676.19999999998	62.667780230269\\
1676.4	62.6603042051894\\
1676.50000000002	62.6625313133723\\
1676.59999999999	62.6587940597603\\
1677.19999999999	62.6721516723305\\
1677.49999999997	62.6609442060086\\
1678	62.6720696025267\\
1678.09999999999	62.6683351209629\\
1678.30000000002	62.6727836034318\\
1678.40000000003	62.6690497467977\\
1678.9	62.6801667659321\\
1679.10000000001	62.6727012863268\\
1679.20000000002	62.6749240755077\\
1679.3	62.671192092414\\
1679.50000000002	62.675637056442\\
1679.69999999999	62.6681747827122\\
1680.19999999999	62.6792834612867\\
1680.30000000001	62.6755534396572\\
1680.59999999999	62.6822157434402\\
1681.10000000002	62.6635736378777\\
1681.20000000001	62.6657943258193\\
1681.29999999999	62.6620673248483\\
1681.8	62.6731672513229\\
1681.90000000002	62.6694411414982\\
1682.1	62.6738794435858\\
1682.20000000002	62.6701539558937\\
1682.5	62.6768096992749\\
1682.60000000003	62.6730849230404\\
1682.79999999998	62.6775209459861\\
1682.89999999997	62.6737967914438\\
1683	62.676014497059\\
1683.1	62.6722908745247\\
1683.19999999999	62.6745084061071\\
1683.29999999999	62.6707853154331\\
1683.60000000002	62.6774365979688\\
1683.7	62.6737142178406\\
1683.89999999998	62.6781472684086\\
1684	62.674425509174\\
1684.09999999999	62.6766417290108\\
1684.20000000001	62.6729205010984\\
1684.29999999999	62.6751365471385\\
1684.70000000002	62.6602564102564\\
1684.99999999998	62.6669040413032\\
1685.10000000002	62.6631853785901\\
1685.29999999999	62.6676159962027\\
1685.39999999999	62.6638979531296\\
1685.60000000001	62.6683276976923\\
1685.90000000001	62.6571767497034\\
1686.20000000001	62.6638201980668\\
1686.30000000001	62.6601043643264\\
1686.6	62.6667457164878\\
1686.69999999999	62.6630305904672\\
1686.80000000002	62.6652439385856\\
1686.89999999999	62.6615293420273\\
1687.69999999998	62.679227396611\\
1687.80000000001	62.6755139522484\\
1687.90000000003	62.6777251184834\\
1688	62.6740122030685\\
1688.09999999999	62.6762231963038\\
1688.20000000002	62.6725108096902\\
1688.39999999999	62.6769321883328\\
1688.59999999998	62.6695090898324\\
1688.90000000002	62.6761397276495\\
1689.00000000002	62.6724291042567\\
1689.09999999999	62.6746388823112\\
1689.20000000001	62.6709287870716\\
1689.29999999997	62.6731383923286\\
1689.60000000002	62.6620110078712\\
1690.10000000001	62.6730564430245\\
1690.2	62.6693486363367\\
1690.60000000002	62.6781806352398\\
1690.89999999998	62.6670609107037\\
1691.1	62.6714758751183\\
1691.20000000001	62.6677703541654\\
1691.30000000003	62.6699775334043\\
1691.50000000001	62.6625679829747\\
1692.30000000002	62.6802174426849\\
1692.40000000001	62.6765140324963\\
1692.49999999997	62.678719130332\\
1692.7	62.6713137996219\\
1692.8	62.6735188138697\\
1693.20000000002	62.6587137542078\\
1693.50000000001	62.6653282947567\\
1693.69999999998	62.6579289172275\\
1693.80000000001	62.6601334199185\\
1693.89999999999	62.6564344746163\\
1694.19999999999	62.663046685947\\
1694.50000000003	62.651953263307\\
1694.60000000001	62.6541570779489\\
1694.70000000001	62.6504602312957\\
1695.4	62.6658802713064\\
1695.60000000002	62.6584891195377\\
1695.7	62.6606911192358\\
1695.90000000002	62.6533018867925\\
1696.19999999998	62.6599068560986\\
1696.29999999998	62.6562131572742\\
1696.4	62.6584143825523\\
1696.90000000001	62.6399528579847\\
1696.99999999999	62.6421542631548\\
1697.10000000002	62.6384633514023\\
1697.20000000001	62.640664584929\\
1697.29999999999	62.6369741958289\\
1697.39999999998	62.639175257732\\
1697.50000000002	62.6354853911404\\
1697.69999999999	62.639886912475\\
1697.79999999999	62.6361976559279\\
1697.90000000003	62.6383981154299\\
1697.99999999997	62.634709381073\\
1698.40000000001	62.6435089785105\\
1698.50000000002	62.6398210290828\\
1699.20000000002	62.6552109692226\\
1699.3	62.6515240673179\\
1699.49999999999	62.6559190397741\\
1699.89999999998	62.6411764705882\\
1700.2	62.647768040934\\
1700.30000000003	62.6440837450012\\
1700.59999999998	62.6506732521903\\
1700.70000000003	62.6469896519285\\
1700.90000000002	62.6513815402704\\
1700.99999999998	62.6476985479984\\
1701.3	62.6542847067121\\
1701.39999999999	62.6506024096386\\
1701.50000000001	62.6527973671838\\
1701.70000000001	62.6454342460924\\
1701.8	62.6476291203949\\
1701.89999999998	62.6439482961222\\
1702	62.6461429998238\\
1702.19999999999	62.6387828232392\\
1703.00000000001	62.6563325700194\\
1703.1	62.6526538280883\\
1703.70000000001	62.6658058457565\\
1703.8	62.6621280591584\\
1703.99999999998	62.6665101813274\\
1704.09999999999	62.6628330008215\\
1704.49999999999	62.6715945089757\\
1704.6	62.6679181087581\\
1704.69999999999	62.670107930549\\
1704.8	62.6664320488005\\
1705.09999999998	62.6730002345766\\
1705.20000000002	62.6693250454465\\
1705.50000000001	62.6758911819887\\
1705.7	62.6685426192989\\
1706.30000000002	62.6816690107829\\
1706.40000000002	62.6779958980369\\
1706.50000000001	62.6801828196414\\
1706.60000000003	62.6765102244097\\
1707.49999999999	62.6961817755915\\
1707.60000000002	62.6925103940973\\
1707.80000000003	62.6968792083846\\
1707.90000000002	62.6932084309134\\
1707.99999999999	62.6953925414203\\
1708.20000000001	62.6880524498039\\
1708.5	62.6946037691677\\
1708.7	62.687265917603\\
1708.99999999998	62.6938154584284\\
1709.1	62.6901474373976\\
1709.19999999999	62.6923301936465\\
1709.29999999999	62.6886626886627\\
1709.39999999999	62.6908452763966\\
1709.49999999998	62.6871782873187\\
1709.59999999999	62.6893607065567\\
1709.80000000001	62.682028188783\\
1710.10000000001	62.6885744357385\\
1710.30000000003	62.6812441534144\\
1710.39999999999	62.68342589886\\
};
\addplot [color=mycolor3, forget plot]
  table[row sep=crcr]{%
1710.39999999999	62.68342589886\\
1710.69999999999	62.6724339490297\\
1711.1	62.6811594202899\\
1711.29999999998	62.6738342877177\\
1711.70000000002	62.6825563734081\\
1711.8	62.6788947952567\\
1712.19999999998	62.687613151901\\
1712.29999999999	62.6839523475823\\
1712.4	62.6861313868613\\
1712.49999999998	62.6824710965783\\
1712.70000000002	62.6868285847735\\
1712.99999999998	62.6758507968011\\
1713.09999999998	62.6780294186318\\
1713.29999999998	62.6707132018209\\
1713.50000000003	62.6750700280112\\
1713.60000000001	62.6714127326837\\
1714.20000000003	62.6844776293531\\
1714.30000000001	62.6808212785814\\
1714.49999999998	62.6851743846961\\
1714.60000000001	62.681518632997\\
1714.80000000001	62.6858708962622\\
1714.90000000002	62.6822157434402\\
1715.00000000003	62.6843915806659\\
1715.09999999997	62.6807369402985\\
1715.29999999998	62.6850880261164\\
1715.39999999998	62.6814339842611\\
1715.59999999998	62.6857842280119\\
1715.70000000002	62.6821307844737\\
1715.80000000002	62.6843056122152\\
1715.90000000001	62.6806526806527\\
1716.69999999999	62.6980428704567\\
1716.80000000002	62.6943910536432\\
1717.30000000002	62.7052521253057\\
1717.40000000002	62.7016011644833\\
1717.50000000001	62.7037727061015\\
1717.60000000002	62.7001222565058\\
1717.79999999997	62.7044647534781\\
1718.09999999998	62.6935164707252\\
1718.20000000001	62.6956875982075\\
1718.3	62.6920391061452\\
1718.4	62.6942100669188\\
1718.5	62.6905620854184\\
1719.3	62.707921367919\\
1719.40000000001	62.7042744984007\\
1719.7	62.7107803232934\\
1719.89999999999	62.703488372093\\
1720.00000000001	62.7056566478693\\
1720.19999999999	62.6983665639714\\
1720.3	62.7005347593583\\
1720.4	62.6968904388259\\
1720.50000000002	62.6990584679763\\
1720.60000000003	62.6954146568257\\
1720.70000000001	62.6975825197583\\
1720.80000000003	62.6939392178511\\
1721	62.6982743594213\\
1721.19999999999	62.6909893685006\\
1721.30000000002	62.6931567328918\\
1721.50000000001	62.6858736059479\\
1721.7	62.6902079219422\\
1721.89999999998	62.6829268292683\\
1722.29999999999	62.6915931258709\\
1722.39999999999	62.6879535558781\\
1722.69999999998	62.6944508938937\\
1722.79999999998	62.6908120030182\\
1722.9	62.6929773650609\\
1723.00000000002	62.6893389820672\\
1723.39999999998	62.697998259356\\
1723.49999999998	62.6943606405198\\
1723.80000000001	62.700852717675\\
1723.89999999998	62.6972157772622\\
1723.99999999999	62.6993793863465\\
1724.2	62.6921069419475\\
1724.29999999999	62.6942704708884\\
1724.4	62.690634966657\\
1724.50000000001	62.6927983300476\\
1724.60000000003	62.6891633327535\\
1725.29999999999	62.7043004520691\\
1725.50000000003	62.6970329160872\\
1726.20000000001	62.7121589526733\\
1726.29999999999	62.7085264133457\\
1726.50000000001	62.7128460558323\\
1726.69999999998	62.7055825804957\\
1726.99999999998	62.7120606797522\\
1727.1	62.7084298286244\\
1727.20000000002	62.7105887801772\\
1727.39999999999	62.7033285094067\\
1727.60000000001	62.7076460033571\\
1727.80000000003	62.7003877539209\\
1727.90000000001	62.7025462962963\\
1728	62.6989178866964\\
1728.2	62.7032343921773\\
1728.29999999997	62.6996065725527\\
1728.39999999999	62.7017645357246\\
1728.60000000003	62.6945103256783\\
1728.90000000002	62.700983227299\\
1729	62.6973570065352\\
1729.50000000001	62.7081406105458\\
1729.7	62.7008902763325\\
1729.79999999999	62.7030464188682\\
1729.9	62.6994219653179\\
1730.09999999998	62.7037336724078\\
1730.50000000001	62.6892407257599\\
1730.70000000002	62.6935521146291\\
1730.79999999998	62.6899300941707\\
1731.29999999999	62.7007046320897\\
1731.40000000002	62.6970834536529\\
1731.49999999997	62.6992376992377\\
1731.70000000002	62.6919967663703\\
1732.00000000003	62.6984585185613\\
1732.10000000002	62.6948389331486\\
1732.4	62.7012987012987\\
1732.6	62.6940612916258\\
1733.6	62.7155793966661\\
1733.70000000001	62.7119621640328\\
1734.30000000001	62.7248616236162\\
1734.40000000002	62.7212453156529\\
1734.5	62.7233944425228\\
1734.60000000002	62.7197786360754\\
1734.69999999998	62.7219275997233\\
1734.80000000001	62.7183122946568\\
1734.90000000002	62.7204610951009\\
1735.00000000003	62.71684629128\\
1735.29999999999	62.7232914601821\\
1735.40000000002	62.7196773264189\\
1735.50000000001	62.7218253053699\\
1735.59999999998	62.718211672524\\
1735.80000000001	62.7225070568581\\
1735.89999999997	62.7188940092166\\
1736.09999999997	62.7231885727451\\
1736.19999999998	62.7195761101192\\
1736.49999999998	62.7260163537948\\
1736.60000000002	62.7224045603731\\
1737.09999999998	62.7331337784941\\
1737.20000000002	62.729522822771\\
1737.30000000001	62.7316680096696\\
1737.39999999998	62.7280575539568\\
1737.80000000002	62.7366361700903\\
1737.99999999997	62.729417179679\\
1738.19999999998	62.7337053443019\\
1738.5	62.722880478546\\
1738.79999999998	62.7293116337915\\
1739.10000000002	62.7184912603496\\
1739.20000000001	62.720634738113\\
1739.49999999998	62.7098183490458\\
1740.1	62.722675554534\\
1740.20000000002	62.719071424467\\
1740.39999999999	62.7233553576558\\
1740.49999999998	62.7197518097208\\
1740.7	62.7240349264706\\
1740.79999999998	62.7204319604802\\
1740.99999999998	62.7247142610993\\
1741.09999999999	62.7211118768665\\
1741.20000000001	62.7232527422041\\
1741.40000000002	62.7160493827161\\
1741.79999999999	62.7246110568919\\
1742.19999999999	62.7102106411066\\
1742.50000000001	62.7166303225066\\
1742.6	62.7130315028404\\
1742.79999999998	62.7173102300763\\
1742.90000000003	62.7137119908204\\
1743.30000000001	62.7222668349203\\
1743.6	62.7114755978666\\
1743.7	62.7136139465535\\
1743.90000000001	62.7064220183486\\
1744.09999999998	62.7106983144135\\
1744.30000000001	62.70350836964\\
1744.40000000002	62.7056463169963\\
1744.60000000002	62.698458187654\\
1744.89999999999	62.7048710601719\\
1745.00000000001	62.7012778637327\\
1745.10000000003	62.703415081366\\
1745.39999999999	62.6926382125466\\
1745.6	62.6969124133585\\
1745.8	62.6897302250988\\
1745.89999999998	62.6918671248568\\
1745.99999999999	62.6882767310005\\
1746.20000000003	62.6925499627784\\
1746.40000000002	62.685370741483\\
1746.70000000002	62.6917792534921\\
1747	62.6810142521893\\
1747.30000000001	62.687421311663\\
1747.39999999999	62.6838340486409\\
1747.5	62.685969329366\\
1747.69999999997	62.6787962009383\\
1748.09999999999	62.6873355451321\\
1748.30000000002	62.6801647220316\\
1748.50000000003	62.6844332608944\\
1748.69999999998	62.6772644098811\\
1749.00000000001	62.6836658853125\\
1749.50000000002	62.665752171925\\
1749.69999999998	62.6700194307921\\
1749.79999999998	62.6664380821761\\
1750.20000000001	62.674970005142\\
1750.50000000002	62.6642294070604\\
1750.70000000002	62.6684944025588\\
1750.79999999998	62.6649151864755\\
1751.09999999997	62.6713111009593\\
1751.7	62.6498458728165\\
1752.40000000002	62.6647646219686\\
1752.50000000001	62.6611890904941\\
1752.59999999998	62.6633194499914\\
1752.90000000001	62.6525955504849\\
1753	62.654725914095\\
1753.30000000001	62.6440059313334\\
1753.39999999999	62.6461362988309\\
1753.50000000001	62.6425638686132\\
1753.70000000001	62.6468240392291\\
1753.80000000002	62.6432521808541\\
1754.3	62.6538987688099\\
1754.39999999999	62.6503277286976\\
1754.6	62.6545848293156\\
1754.69999999998	62.6510143606109\\
1755.29999999999	62.6637803349664\\
1755.39999999998	62.6602107661635\\
1755.49999999998	62.6623376623377\\
1755.70000000001	62.6551999088735\\
1755.89999999999	62.6594533029613\\
1756.2	62.6487502135171\\
1756.30000000001	62.6508767934411\\
1756.39999999999	62.6473099914603\\
1756.7	62.6536885245902\\
1757.20000000001	62.6358618334946\\
1757.3	62.6379879367247\\
1757.39999999997	62.6344238975818\\
1757.7	62.6408010012516\\
1757.80000000003	62.637237613061\\
1757.89999999997	62.6393629124005\\
1758	62.6358000113759\\
1758.10000000002	62.6379251507223\\
1758.19999999998	62.6343627367344\\
1758.50000000001	62.6407369498465\\
1758.60000000002	62.6371751862171\\
1759.59999999998	62.6584076831278\\
1759.90000000002	62.6477272727273\\
1760.00000000002	62.6498494403727\\
1760.09999999998	62.6462901942961\\
1760.20000000002	62.6484122024655\\
1760.3	62.6448534423995\\
1760.39999999997	62.6469752911105\\
1760.70000000001	62.6363016810541\\
1761.50000000001	62.6532697547684\\
1761.60000000002	62.6497133450644\\
1762	62.6581919300834\\
1762.10000000001	62.6546362501419\\
1762.49999999997	62.6631113128333\\
1762.60000000002	62.6595563623986\\
1762.99999999999	62.6680279053939\\
1763.1	62.6644736842105\\
1763.59999999999	62.6750581164597\\
1763.69999999999	62.671504705749\\
1763.79999999999	62.6736209535688\\
1763.89999999998	62.6700680272109\\
1764.1	62.6742999659903\\
1764.20000000003	62.6707476052826\\
1764.30000000001	62.6728632963047\\
1764.4	62.6693114196656\\
1764.49999999999	62.6714269522838\\
1764.59999999998	62.6678755595852\\
1764.9	62.6742209631728\\
1765	62.6706702169849\\
1765.30000000003	62.6770137079415\\
1765.4	62.6734636080431\\
1765.50000000002	62.675577707295\\
1765.60000000003	62.6720280908422\\
1765.70000000001	62.6741420319402\\
1765.80000000003	62.6705928988052\\
1766.20000000003	62.6790465945762\\
1766.70000000003	62.6613085804845\\
1767.00000000001	62.6676475581461\\
1767.10000000003	62.6641014033499\\
1767.29999999999	62.6683263550979\\
1767.39999999999	62.6647807637907\\
1767.49999999998	62.6668929622086\\
1767.59999999997	62.6633478531425\\
1768.20000000003	62.6760165130351\\
1768.30000000002	62.6724722913368\\
1768.60000000003	62.6788036410923\\
1768.69999999999	62.6752600633198\\
1769.20000000001	62.6858079466456\\
1769.29999999998	62.6822651746355\\
1769.59999999999	62.6885912866588\\
1770.00000000002	62.674425173719\\
1770.19999999998	62.6786420380726\\
1770.29999999999	62.6751016719385\\
1770.60000000002	62.6814254249732\\
1770.69999999998	62.6778857013779\\
1771.09999999999	62.6863143631436\\
1771.19999999998	62.6827753627279\\
1771.50000000001	62.689094603748\\
1771.7	62.6820182864883\\
1771.89999999998	62.686230248307\\
1772	62.6826928502906\\
1772.20000000001	62.6869040230209\\
1772.49999999999	62.676294708338\\
1772.99999999998	62.6868196943207\\
1773.20000000003	62.6797496193537\\
1773.30000000002	62.6818540656366\\
1773.49999999997	62.6747857465043\\
1773.80000000002	62.6810981453295\\
1774.1	62.6704993800023\\
1774.30000000002	62.6747069431921\\
1774.39999999999	62.6711749788673\\
1774.59999999999	62.6753817546628\\
1774.69999999997	62.6718503493351\\
1774.90000000002	62.6760563380282\\
1775.30000000001	62.6619353385153\\
1775.59999999999	62.6682435096019\\
1775.7	62.6647144948756\\
1776.19999999999	62.6752237797669\\
1776.29999999998	62.6716955640621\\
1776.6	62.6779985366128\\
1776.69999999999	62.6744709590275\\
1777.10000000001	62.6828719333784\\
1777.19999999998	62.6793450739886\\
1777.60000000003	62.6877425887383\\
1777.80000000002	62.6806907025142\\
1777.90000000002	62.6827896512936\\
1778.00000000001	62.6792643833305\\
1778.1	62.6813631762456\\
1778.3	62.6743139901035\\
1778.49999999997	62.6785111885753\\
1778.90000000001	62.6644182124789\\
1779.89999999999	62.685393258427\\
1780.00000000002	62.6818718049548\\
1780.09999999999	62.6839680934726\\
1780.20000000001	62.6804471156547\\
1780.29999999999	62.6825432487082\\
1780.39999999999	62.6790227464195\\
1781.19999999998	62.6957839779936\\
1781.29999999998	62.6922645110587\\
1781.69999999997	62.700639802447\\
1782.19999999997	62.6830499915839\\
1782.30000000002	62.6851436265709\\
1782.40000000001	62.6816269284712\\
1782.70000000002	62.6879066636751\\
1782.9	62.6808749298934\\
1783.3	62.6892452618594\\
1783.40000000002	62.6857303055789\\
1783.5	62.6878223816999\\
1783.79999999999	62.6772801165985\\
1784.09999999999	62.6835556551956\\
1784.19999999999	62.6800425937342\\
1784.9	62.6946778711485\\
1785.09999999998	62.6876540443648\\
1786.10000000003	62.7085432762289\\
1786.19999999999	62.7050327492582\\
1786.40000000001	62.7092079485027\\
1786.49999999998	62.705697973805\\
1786.69999999997	62.7098723975823\\
1786.79999999998	62.7063629749846\\
1786.89999999999	62.7084499160604\\
1787.00000000002	62.7049409658105\\
1787.09999999997	62.7070277529096\\
1787.30000000001	62.7000111894372\\
1787.89999999999	62.7125279642058\\
1788.00000000003	62.7090207482803\\
1788.19999999999	62.713191298999\\
1788.30000000002	62.70968463431\\
1788.40000000003	62.7117696393626\\
1788.50000000003	62.7082634462708\\
1788.59999999998	62.7103482976463\\
1788.70000000002	62.7068425760286\\
1789	62.7130959700408\\
1789.20000000001	62.7060861789527\\
1789.40000000002	62.7102542609667\\
1789.5	62.7067501117568\\
1789.80000000001	62.7130007262976\\
1790.20000000003	62.6989889962576\\
1790.89999999999	62.713567839196\\
1791.00000000003	62.7100664396181\\
1791.09999999999	62.7121482804824\\
1791.3	62.7051468125488\\
1791.79999999998	62.7155533232881\\
1792.30000000001	62.6980584690917\\
1792.50000000003	62.7022202387594\\
1792.60000000002	62.6987225971998\\
1792.99999999999	62.7070436673917\\
1793.1	62.703546732099\\
1793.49999999997	62.7118644067797\\
1793.6	62.7083681775102\\
1794.30000000001	62.7229157378511\\
1794.40000000003	62.7194204513792\\
1794.49999999999	62.7214978268138\\
1794.60000000003	62.7180030088594\\
1794.7	62.7200802317807\\
1794.79999999999	62.7165858822219\\
1794.90000000001	62.7186629526462\\
1794.99999999999	62.7151690713609\\
1795.09999999999	62.7172459893048\\
1795.30000000002	62.7102595521889\\
1795.39999999998	62.7123363965469\\
1795.50000000002	62.708843840499\\
1795.59999999997	62.7109205323829\\
1795.7	62.7074284441475\\
1795.8	62.7095049835737\\
1795.99999999999	62.7025221312845\\
1796.4	62.710826607292\\
1796.59999999999	62.7038459397785\\
1797.29999999999	62.7183709803049\\
1797.40000000001	62.7148817802504\\
1797.59999999998	62.7190298715025\\
1797.69999999998	62.7155412170431\\
1797.99999999999	62.7217618597408\\
1798.1	62.7182738293849\\
1798.2	62.7203469943836\\
1798.59999999998	62.7063990659921\\
1799.39999999997	62.7229786051681\\
1799.49999999997	62.7194932207157\\
1799.60000000002	62.7215647052287\\
1799.70000000002	62.718079786643\\
1800.10000000001	62.7263637373625\\
1800.20000000002	62.72287952008\\
1800.69999999997	62.7332296756997\\
1800.90000000002	62.7262631871183\\
1801.00000000001	62.728332685581\\
1801.09999999997	62.7248500999334\\
1801.30000000001	62.7289885644499\\
1801.60000000002	62.7185435977133\\
1801.69999999999	62.7206127206127\\
1801.79999999997	62.7171319163106\\
1802.9	62.7398779811425\\
1803.1	62.7329192546584\\
1803.2	62.734985859258\\
1803.40000000002	62.7280288328251\\
1803.69999999999	62.7342277414348\\
1803.79999999998	62.7307500415766\\
1803.89999999999	62.7328159645233\\
1804.10000000003	62.7258618778406\\
1804.69999999999	62.7382535460993\\
1804.79999999997	62.7347775500028\\
1804.89999999998	62.7368421052632\\
1805.00000000003	62.7333665724891\\
1805.09999999999	62.735430977177\\
1805.20000000002	62.7319559076054\\
1805.30000000003	62.734020161737\\
1805.4	62.7305455552479\\
1805.70000000002	62.7367371801972\\
1805.80000000001	62.7332631928678\\
1806.19999999998	62.7415158057909\\
1806.40000000002	62.7345696097426\\
1806.89999999998	62.7448810182623\\
1806.99999999998	62.7414088871673\\
1807.10000000003	62.7434705621957\\
1807.2	62.7399988933769\\
1807.80000000002	62.752364621937\\
1807.90000000001	62.7488938053097\\
1808.29999999998	62.7571333775713\\
1808.4	62.7536632568427\\
1808.59999999997	62.7577818322552\\
1808.70000000001	62.7543122512163\\
1808.79999999999	62.7563712753607\\
1808.90000000002	62.7529021558872\\
1809.20000000001	62.7590780965014\\
1809.29999999997	62.7556095943407\\
1809.60000000001	62.7617837210587\\
1809.70000000001	62.758315836004\\
1810.30000000001	62.7706584180292\\
1810.39999999999	62.7671913835957\\
1810.50000000002	62.7692477631724\\
1810.60000000002	62.7657811895952\\
1810.70000000002	62.7678374199249\\
1810.80000000002	62.7643713070849\\
1811.00000000003	62.7684832422285\\
1811.19999999997	62.7615524761221\\
1811.30000000002	62.7636082588053\\
1811.40000000001	62.7601435274634\\
1811.69999999997	62.7663097472127\\
1811.8	62.7628456316574\\
1812.2	62.7710643933124\\
1812.30000000001	62.7676009710881\\
1812.79999999999	62.777869711512\\
1813	62.77094479069\\
1813.20000000001	62.7750510119671\\
1813.29999999999	62.7715892798059\\
1813.50000000001	62.7756947507719\\
1813.60000000002	62.7722335557148\\
1813.89999999998	62.7783902976847\\
1814.00000000002	62.7749297172151\\
1814.90000000002	62.7933884297521\\
1814.99999999997	62.7899289295356\\
1815.29999999998	62.796077999339\\
1815.40000000002	62.792619113192\\
1815.69999999998	62.798766383963\\
1815.80000000002	62.7953081116802\\
1815.90000000001	62.7973568281938\\
1816.09999999998	62.7904415813236\\
1816.20000000001	62.7924902273853\\
1816.29999999998	62.7890332525875\\
1816.40000000001	62.7910817506193\\
1816.50000000003	62.7876252339535\\
1816.70000000003	62.7917217084985\\
1816.79999999999	62.7882657273378\\
1818.09999999998	62.8148718512815\\
1818.19999999999	62.8114172578782\\
1818.40000000001	62.8155072862249\\
1818.50000000001	62.8120532277576\\
1818.6	62.8140979820751\\
1818.8	62.807191159492\\
1819.39999999999	62.8194558944765\\
1819.50000000001	62.8160035172565\\
1820.00000000002	62.8262183396517\\
1820.19999999998	62.8193154974455\\
1820.3	62.8213579433092\\
1820.39999999998	62.8179071683603\\
1820.90000000001	62.8281164195497\\
1821.09999999997	62.821216780145\\
1821.40000000001	62.8273401043096\\
1821.7	62.8169941815787\\
1822.1	62.8251564043464\\
1822.4	62.8148148148148\\
1822.5	62.8168550422473\\
1822.70000000002	62.8099626947553\\
1823.20000000001	62.8201612460923\\
1823.39999999999	62.8132711817933\\
1823.50000000001	62.8153103750823\\
1823.70000000001	62.8084219760939\\
1823.79999999999	62.810461099841\\
1823.99999999999	62.8035743654405\\
1824.19999999997	62.8076522501782\\
1824.39999999998	62.800767333516\\
1824.60000000003	62.8048446319943\\
1824.79999999998	62.7979615321387\\
1824.9	62.8\\
1825.00000000003	62.7965590926525\\
1825.19999999997	62.8006355119706\\
1825.39999999997	62.7937551355793\\
1825.70000000001	62.7998685507723\\
1825.79999999998	62.7964291582233\\
1825.89999999997	62.7984665936473\\
1825.99999999999	62.7950276545644\\
1826.29999999999	62.8011388523872\\
1826.40000000002	62.7977005201205\\
1826.70000000001	62.8038099408802\\
1826.89999999997	62.7969348659004\\
1827.10000000002	62.8010070052539\\
1827.19999999998	62.7975701855196\\
1827.3	62.7996059975922\\
1827.40000000003	62.796169630643\\
1827.60000000001	62.8002407397275\\
1827.89999999998	62.7899343544858\\
1828.20000000002	62.796040037193\\
1828.3	62.792605556771\\
1828.89999999998	62.8048113723346\\
1828.99999999998	62.8013777267509\\
1829.09999999997	62.8034113273562\\
1829.20000000002	62.7999781337124\\
1829.30000000001	62.802011588499\\
1829.39999999999	62.7985788466794\\
1829.69999999999	62.8046781068969\\
1829.79999999998	62.8012459697251\\
1830.00000000002	62.8053111851811\\
1830.10000000001	62.8018795760026\\
1830.7	62.8140703517588\\
1830.90000000003	62.807209175314\\
1831.2	62.8133020258833\\
1831.29999999999	62.8098722288959\\
1833.09999999997	62.8463888282784\\
1833.60000000001	62.8292523313519\\
1833.7	62.8312793107209\\
1833.90000000001	62.824427480916\\
1834.80000000001	62.8426617254346\\
1835.20000000003	62.8289652917779\\
1835.3	62.8309905197777\\
1835.40000000002	62.8275674203214\\
1835.70000000002	62.8336420089334\\
1835.8	62.8302195108666\\
1835.9	62.8322440087146\\
1835.99999999998	62.8288219595883\\
1836.1	62.8308463130378\\
1836.2	62.8274247127376\\
1836.29999999999	62.8294489218035\\
1836.39999999998	62.8260277702151\\
1836.49999999999	62.8280518349123\\
1836.60000000001	62.8246311319214\\
1836.69999999999	62.8266550522648\\
1836.80000000001	62.8232347977571\\
1837.20000000003	62.8313285799815\\
1837.29999999997	62.8279090018504\\
1837.4	62.8299319727891\\
1837.79999999998	62.8162576854018\\
1838.20000000003	62.8243485829299\\
1838.30000000001	62.8209312445605\\
1838.69999999998	62.8290189253861\\
1838.79999999997	62.825602262222\\
1839.6	62.841767679513\\
1839.8	62.8349366813414\\
1840.40000000002	62.8470524314045\\
1840.5	62.8436379441487\\
1840.70000000001	62.8476749239461\\
1840.8	62.8442609593134\\
1841.00000000002	62.8482972136223\\
1841.10000000001	62.8448837714534\\
1841.60000000001	62.854970950752\\
1841.8	62.8481459362615\\
1841.90000000001	62.8501628664495\\
1842.09999999999	62.8433394853979\\
1842.2	62.8453563480432\\
1842.30000000003	62.8419452887538\\
1842.8	62.8520266970536\\
1842.89999999998	62.8486163863266\\
1843.2	62.8546628329626\\
1843.39999999997	62.8478437754272\\
1843.5	62.8498589715773\\
1843.59999999999	62.8464500732223\\
1843.8	62.8504799609523\\
1844	62.843663575728\\
1844.09999999999	62.8456783429129\\
1844.19999999998	62.8422707802418\\
1844.49999999997	62.8483139976147\\
1844.59999999999	62.8449070309536\\
1844.80000000001	62.8489349016207\\
1845.20000000002	62.8353113314908\\
1845.59999999999	62.8433656607249\\
1845.80000000001	62.836556693212\\
1846.3	62.8466204506066\\
1846.4	62.8432168968318\\
1846.50000000001	62.8452290696415\\
1846.70000000002	62.8384232185402\\
1847.09999999999	62.8464703334777\\
1847.30000000001	62.8396665584064\\
1847.89999999999	62.8517316017316\\
1847.99999999999	62.8483307180347\\
1848.20000000002	62.8523508088514\\
1848.30000000003	62.8489504436269\\
1848.6	62.8549791745551\\
1848.70000000001	62.8515794028559\\
1849.10000000003	62.8596149686351\\
1849.20000000002	62.8562158654626\\
1849.4	62.860232495269\\
1849.60000000002	62.8534356922744\\
1850.2	62.865481273307\\
1850.40000000002	62.8586868413942\\
1850.99999999998	62.8707255145589\\
1851.20000000001	62.8639334521687\\
1851.30000000001	62.8659392891866\\
1851.49999999997	62.8591488442428\\
1851.59999999997	62.8611546146784\\
1851.7	62.8577600172805\\
1851.89999999998	62.8617710583153\\
1851.99999999997	62.858376977485\\
1852.19999999997	62.8623873022729\\
1852.30000000002	62.8589937378536\\
1852.50000000003	62.863003346648\\
1852.59999999998	62.8596102984833\\
1852.7	62.8616148531952\\
1852.79999999999	62.8582222462087\\
1853.20000000002	62.8662386014137\\
1853.3	62.8628466601921\\
1853.49999999999	62.8668536901165\\
1853.69999999998	62.8600712050922\\
1853.80000000001	62.8620745455526\\
1853.99999999997	62.8552936734804\\
1854.4	62.8633054731734\\
1855.20000000002	62.8361989974667\\
1855.29999999999	62.8382020049585\\
1855.49999999998	62.8314291873249\\
1855.70000000001	62.8354348528936\\
1856.00000000002	62.8252788104089\\
1856.80000000001	62.8412946308363\\
1856.89999999998	62.8379106085084\\
1857.10000000001	62.8419125565367\\
1857.30000000002	62.835145902875\\
1857.39999999999	62.8371467025572\\
1857.5	62.8337639965547\\
1857.79999999997	62.8397653264438\\
1857.89999999998	62.8363832077503\\
1858.69999999999	62.852377878201\\
1858.80000000001	62.8489967184894\\
1858.89999999998	62.8509951586875\\
1859.10000000001	62.8442340791738\\
1859.20000000001	62.8462324530737\\
1859.30000000002	62.8428525330752\\
1859.80000000001	62.852841550621\\
1859.90000000003	62.8494623655914\\
1860.00000000003	62.8514595989463\\
1860.10000000003	62.8480808515213\\
1860.29999999998	62.8520748226188\\
1860.50000000003	62.8453187143932\\
1860.59999999997	62.8473155264148\\
1860.69999999999	62.8439380911436\\
1861.10000000001	62.8519234902214\\
1861.29999999998	62.8451703019233\\
1861.59999999998	62.8511575441801\\
1861.69999999998	62.8477817166183\\
1861.80000000003	62.8497771094044\\
1861.89999999998	62.8464017185822\\
1861.99999999999	62.8483969711616\\
1862.10000000002	62.8450220169692\\
1862.19999999999	62.8470171293562\\
1862.29999999999	62.8436426116838\\
1862.60000000001	62.8496268857035\\
1862.69999999997	62.8462529525446\\
1863	62.8522355214428\\
1863.10000000003	62.8488621726063\\
1863.30000000001	62.8528496297091\\
1863.49999999998	62.8461043142305\\
1864	62.8560699533287\\
1864.5	62.839214845007\\
1864.69999999999	62.8432003432004\\
1865.09999999997	62.8297233540639\\
1865.29999999997	62.8337085879704\\
1865.39999999997	62.830340391316\\
1865.49999999999	62.832332761578\\
1865.60000000001	62.828964999732\\
1866.2	62.8409151797675\\
1866.40000000002	62.8341816233592\\
1867.79999999999	62.8620375823117\\
1867.89999999999	62.8586723768737\\
1868.19999999999	62.86463630038\\
1868.50000000001	62.854543508509\\
1868.90000000002	62.8624933119315\\
1869	62.859130062597\\
1869.1	62.8611170554248\\
1869.29999999998	62.8543917834599\\
1869.39999999999	62.8563787108853\\
1869.60000000002	62.8496550248703\\
1870.20000000001	62.8615730096776\\
1870.30000000002	62.8582121471343\\
1870.40000000001	62.8601978080727\\
1870.49999999997	62.8568373783813\\
1870.6	62.8588229005185\\
1870.70000000002	62.8554629035707\\
1871.09999999999	62.8634031637452\\
1871.20000000001	62.8600438198044\\
1871.50000000003	62.8659970079077\\
1871.80000000001	62.855921790694\\
1872.20000000002	62.8638572878278\\
1872.49999999998	62.8537861796433\\
1872.70000000002	62.8577530969671\\
1872.99999999999	62.8476856547969\\
1873.20000000001	62.8516521646293\\
1873.6	62.8382345092597\\
1874.30000000003	62.8521126760563\\
1874.49999999997	62.8454070201643\\
1875.09999999999	62.857295221843\\
1875.29999999999	62.8505918737336\\
1875.40000000001	62.8525726472941\\
1875.5	62.849221582427\\
1875.89999999997	62.8571428571429\\
1876.09999999999	62.8504423835412\\
1876.90000000002	62.8662759722962\\
1877.09999999999	62.8595780950352\\
1877.30000000002	62.8635346756152\\
1877.4	62.8601864181092\\
1878.30000000003	62.8779812606474\\
1878.69999999998	62.8645944219715\\
1879.1	62.8724989357173\\
1879.20000000002	62.8691534081839\\
1879.39999999999	62.8731045490822\\
1879.59999999998	62.8664148534341\\
1880.09999999999	62.8762897564089\\
1880.40000000001	62.8662589736772\\
1880.60000000001	62.8702079013133\\
1880.9	62.8601807549176\\
1881	62.8621551220031\\
1881.09999999998	62.858813523283\\
1881.30000000002	62.8627617731477\\
1881.4	62.8594206749934\\
1881.79999999997	62.8673149476593\\
1882.29999999997	62.8506162345941\\
1882.89999999997	62.8624535315985\\
1883.00000000001	62.8591152886198\\
1883.70000000002	62.8729164454825\\
1883.90000000002	62.8662420382166\\
1884.10000000001	62.8701836323108\\
1884.19999999999	62.8668471050257\\
1884.50000000002	62.8727581449645\\
1884.60000000002	62.8694221892078\\
1885.30000000003	62.8832078073618\\
1885.4	62.8798727128083\\
1885.60000000001	62.8838097258313\\
1885.70000000002	62.8804751299183\\
1885.89999999998	62.8844114528102\\
1885.99999999998	62.8810773553894\\
1886.09999999999	62.8830452762167\\
1886.29999999999	62.8763782866836\\
1886.6	62.8822812317804\\
1886.69999999999	62.8789484842061\\
1886.80000000002	62.8809157878001\\
1886.99999999998	62.874251497006\\
1887.10000000002	62.8762187367529\\
1887.2	62.872887193345\\
1887.39999999999	62.876821192053\\
1887.49999999998	62.8734901462174\\
1887.60000000002	62.8754569052286\\
1887.69999999998	62.872126284564\\
1887.90000000002	62.8760593220339\\
1888.00000000002	62.8727291986653\\
1888.39999999997	62.8805930632777\\
1888.7	62.8706056755612\\
1889.00000000002	62.8765020380075\\
1889.10000000002	62.8731738301927\\
1889.30000000001	62.8771038424897\\
1889.39999999999	62.8737761312517\\
1889.60000000001	62.8777054558925\\
1889.69999999999	62.8743782410837\\
1889.79999999997	62.876342663633\\
1889.90000000003	62.8730158730159\\
1890.09999999998	62.8769442387049\\
1890.50000000001	62.8636411721147\\
1890.7	62.8675692828432\\
1890.90000000003	62.8609201480698\\
1891.19999999998	62.8668111880717\\
1891.29999999997	62.8634873638575\\
1891.39999999999	62.8654507005023\\
1891.49999999997	62.8621272996405\\
1891.8	62.8680162799302\\
1892.2	62.854727051736\\
1892.60000000003	62.8625772705659\\
1892.70000000001	62.8592561284869\\
1892.8	62.8612182365682\\
1892.9	62.8578975171685\\
1893	62.8598594897258\\
1893.20000000003	62.8532192468177\\
1893.29999999998	62.8551811555931\\
1893.4	62.8518616318986\\
1893.59999999999	62.8557849712204\\
1893.90000000003	62.8458289334741\\
1894	62.847790507365\\
1894.30000000001	62.8378378378378\\
1894.80000000002	62.847643675128\\
1895.00000000001	62.8410110284418\\
1895.39999999999	62.8488525455025\\
1895.70000000001	62.8389070577065\\
1896.00000000002	62.8447866673699\\
1896.19999999999	62.8381585192217\\
1896.49999999998	62.8440366972477\\
1896.59999999998	62.8407233616281\\
1896.9	62.8465998945704\\
1896.99999999999	62.8432871224501\\
1897.09999999997	62.8452456251318\\
1897.2	62.8419332735993\\
1897.30000000003	62.8438916411932\\
1897.39999999999	62.8405797101449\\
1897.59999999998	62.8444959688043\\
1897.8	62.8378734390642\\
1897.9	62.8398314014752\\
1898.00000000001	62.8365207312576\\
1898.09999999999	62.8384785586345\\
1898.30000000001	62.8318584070797\\
1898.40000000001	62.8338161706611\\
1898.49999999998	62.8305066891394\\
1898.6	62.832464317691\\
1899	62.8192301616555\\
1899.69999999999	62.8329297820823\\
1899.79999999999	62.8296226117164\\
1900.1	62.8354910009473\\
1900.30000000001	62.8288781309198\\
1900.49999999999	62.8327896453751\\
1900.80000000002	62.8228733757694\\
1901.40000000001	62.8346042597949\\
1901.50000000001	62.8312999579302\\
1901.80000000002	62.8371628371629\\
1901.90000000002	62.8338590956888\\
1902.3	62.8416736753575\\
1902.60000000001	62.8317653860304\\
1902.8	62.8356718692522\\
1902.90000000001	62.8323699421965\\
1903.10000000003	62.8362757461118\\
1903.2	62.8329743077812\\
1903.70000000003	62.8427355814686\\
1903.80000000002	62.839434844267\\
1904.00000000001	62.8433380599758\\
1904.10000000002	62.8400378111543\\
1904.2	62.8419891823767\\
1904.29999999997	62.8386893509767\\
1904.4	62.8406405880809\\
1904.49999999998	62.8373411739998\\
1904.79999999998	62.8431938684445\\
1904.99999999998	62.8365965041205\\
1905.10000000001	62.8385471341591\\
1905.19999999999	62.8352490421456\\
1905.29999999999	62.8371995381547\\
1905.49999999997	62.830604534005\\
1905.59999999997	62.8325549666789\\
1905.69999999998	62.8292580543604\\
1905.99999999998	62.8351083363937\\
1906.20000000001	62.8285159733515\\
1906.29999999999	62.8304657994125\\
1906.39999999998	62.8271702071859\\
1906.5	62.8291198992972\\
1906.6	62.825824723344\\
1906.80000000001	62.8297236352195\\
1907.20000000003	62.8165469511875\\
1908.20000000002	62.8360320704292\\
1908.40000000001	62.8294472098507\\
1909.40000000003	62.8489133280964\\
1909.5	62.8456221198157\\
1909.70000000001	62.8495130380145\\
1909.80000000001	62.8462223153045\\
1910.09999999998	62.852057376191\\
1910.2	62.8487672093389\\
1910.29999999997	62.8507118927973\\
1910.40000000003	62.8474221408008\\
1910.70000000001	62.853255181076\\
1910.8	62.8499659846146\\
1910.90000000003	62.8519099947671\\
1911.19999999999	62.8420446816303\\
1911.4	62.8459325137327\\
1911.5	62.8426449047918\\
1911.60000000002	62.8445885860752\\
1911.70000000001	62.8413013913589\\
1912.00000000003	62.8471314261806\\
1912.1	62.8438447861102\\
1912.20000000002	62.8457877948021\\
1912.29999999998	62.8425015687095\\
1912.40000000002	62.8444444444444\\
1912.59999999999	62.8378731635907\\
1912.79999999997	62.8417585864394\\
1913	62.8351889603262\\
1913.2	62.8390738514608\\
1913.30000000003	62.8357896937389\\
1913.49999999997	62.8396739130435\\
1913.69999999999	62.8331069077229\\
1914.29999999999	62.8447555369829\\
1914.5	62.8381907448031\\
1914.69999999999	62.8420722790892\\
1914.80000000003	62.8387905373649\\
1914.90000000002	62.8407310704961\\
1914.99999999998	62.8374497415279\\
1915.1	62.8393901420217\\
1915.20000000002	62.8361092257088\\
1915.3	62.8380494935784\\
1915.39999999997	62.8347689898199\\
1915.6	62.8386490577857\\
1915.69999999997	62.8353690364339\\
1916.1	62.8431270222315\\
1916.20000000002	62.8398476230235\\
1916.50000000002	62.8456641970156\\
1916.70000000002	62.8391068447412\\
1916.89999999998	62.8429838288993\\
1916.99999999998	62.8397058056439\\
1917.20000000002	62.8435821206906\\
1917.70000000002	62.8271978308479\\
1918.30000000001	62.8388240200167\\
1918.39999999998	62.8355486056815\\
1918.5	62.8374856666319\\
1918.60000000002	62.8342106634701\\
1918.7	62.8361475922452\\
1919.29999999999	62.8165051578618\\
1919.80000000003	62.8261888640033\\
1920	62.8196448101661\\
1920.2	62.8235171587773\\
1920.70000000001	62.8071636817993\\
1920.8	62.809099901088\\
1920.90000000002	62.8058302967205\\
1921.19999999999	62.811637953469\\
1921.30000000001	62.8083688976788\\
1921.70000000002	62.8161098969716\\
1921.79999999997	62.8128414589729\\
1922.8	62.8321805606116\\
1922.90000000003	62.8289131565263\\
1923.20000000001	62.8347111735039\\
1923.70000000002	62.8183802890113\\
1923.79999999999	62.8203129060762\\
1923.90000000003	62.8170478170478\\
1924.19999999999	62.8228446707894\\
1924.30000000002	62.8195801288713\\
1924.40000000001	62.82151208106\\
1924.60000000002	62.8149841533746\\
1924.69999999997	62.8169160432253\\
1924.99999999999	62.8071269024986\\
1925.10000000003	62.8090587990858\\
1925.19999999999	62.8057964992469\\
1925.40000000003	62.8096598286159\\
1925.6	62.8031365217843\\
1925.8	62.8069993249909\\
1926.10000000003	62.7972173190738\\
1926.19999999999	62.7991486269013\\
1926.29999999999	62.7958887043189\\
1926.5	62.799750856431\\
1926.80000000001	62.7899735326172\\
1927.59999999997	62.8054157804638\\
1927.89999999997	62.795643153527\\
1928.00000000002	62.7975727400031\\
1928.09999999999	62.7943159423296\\
1928.19999999998	62.7962453975004\\
1928.30000000002	62.7929890064302\\
1928.4	62.7949183303085\\
1928.49999999997	62.791662345743\\
1928.90000000002	62.7993779160187\\
1929.00000000003	62.7961225441916\\
1929.10000000002	62.7980510055982\\
1929.29999999998	62.7915414118379\\
1929.40000000001	62.7934698108318\\
1929.50000000002	62.7902155887231\\
1929.7	62.7940719245518\\
1929.79999999997	62.7908181771076\\
1929.99999999999	62.7946738510958\\
1930.10000000003	62.7914205781784\\
1930.20000000001	62.7933481842201\\
1930.40000000003	62.7868427868428\\
1930.89999999999	62.7964785085448\\
1931.1	62.7899751449876\\
1931.79999999999	62.8034577359077\\
1932.09999999998	62.7937066556257\\
1932.20000000001	62.7956321482172\\
1932.50000000002	62.7858843009417\\
1932.69999999998	62.7897350993378\\
1932.89999999998	62.7832384893947\\
1933.09999999999	62.7870887647424\\
1933.2	62.7838411007086\\
1933.30000000002	62.7857660080687\\
1933.50000000002	62.7792718245759\\
1933.89999999997	62.7869700103413\\
1934.20000000002	62.7772320736184\\
1934.40000000001	62.7810803825278\\
1934.50000000002	62.7778352114132\\
1934.90000000001	62.7855297157623\\
1935.00000000001	62.782285153222\\
1935.10000000002	62.7842083505581\\
1935.19999999999	62.7809641915982\\
1935.29999999999	62.7828872584479\\
1935.40000000002	62.7796435029708\\
1936.20000000003	62.7950214326292\\
1936.40000000001	62.7885360185902\\
1936.6	62.7923787886611\\
1936.70000000003	62.7891367203635\\
1937.00000000001	62.7948995921739\\
1937.79999999998	62.7689767273853\\
1938.00000000003	62.7728187400031\\
1938.10000000002	62.7695800227015\\
1938.3	62.7734213784565\\
1938.40000000001	62.7701831312871\\
1938.59999999999	62.7740238304018\\
1938.80000000002	62.7675486100366\\
1939.20000000001	62.7752281751147\\
1939.69999999998	62.7590473244665\\
1939.90000000002	62.7628865979381\\
1940.1	62.7564168642408\\
1940.19999999999	62.7583363397413\\
1940.50000000002	62.7486344429558\\
1940.60000000003	62.7505539238419\\
1940.9	62.7408552292633\\
1941.10000000002	62.744694003709\\
1941.2	62.7414619069696\\
1941.59999999999	62.7491373538652\\
1941.70000000002	62.7459058605418\\
1941.79999999998	62.7478242957928\\
1942.00000000001	62.7413624427166\\
1942.09999999997	62.74328081557\\
1942.39999999998	62.7335907335907\\
1942.60000000002	62.7374272919133\\
1942.69999999998	62.734198064649\\
1943.30000000002	62.7457034064012\\
1943.39999999999	62.742474916388\\
1943.49999999997	62.7443918501749\\
1943.59999999999	62.7411637598395\\
1943.8	62.7449971706364\\
1943.90000000002	62.7417695473251\\
1944.20000000002	62.7475183870802\\
1944.30000000001	62.7442912980868\\
1944.40000000001	62.7462072512214\\
1944.49999999999	62.7429805615551\\
1944.60000000001	62.7448963850465\\
1944.90000000003	62.7352185089974\\
1945.29999999997	62.7428806415133\\
1945.40000000002	62.739655615523\\
1945.49999999999	62.7415707236842\\
1945.59999999999	62.7383460965205\\
1945.69999999999	62.7402610751362\\
1945.79999999997	62.7370368467033\\
1946.00000000002	62.7408663480808\\
1946.09999999999	62.7376425855513\\
1946.70000000001	62.7491267721389\\
1946.79999999998	62.7459037444142\\
1946.99999999997	62.74973036824\\
1947.19999999997	62.7432855748986\\
1947.50000000002	62.7490244403368\\
1947.6	62.7458027416953\\
1947.89999999999	62.7515400410678\\
1948.00000000002	62.7483188748011\\
1948.79999999999	62.7636102416748\\
1948.90000000002	62.7603899435608\\
1949.39999999997	62.7699410105155\\
1949.60000000002	62.7635020772426\\
1950.09999999998	62.7730489180597\\
1950.20000000003	62.7698302825206\\
1950.4	62.7736477826198\\
1950.70000000001	62.7639942587656\\
1950.79999999997	62.7659029166026\\
1950.90000000001	62.7626858021527\\
1950.99999999999	62.7645943314028\\
1951.10000000002	62.7613776137761\\
1951.29999999999	62.7651942195347\\
1951.40000000002	62.7619779656674\\
1951.59999999999	62.7657939232464\\
1951.70000000002	62.7625781330054\\
1951.8	62.7644858855474\\
1952.10000000002	62.754840692552\\
1952.39999999999	62.7605633802817\\
1952.90000000003	62.7444956477215\\
1953.09999999999	62.7483104648782\\
1953.20000000001	62.7450980392157\\
1953.40000000003	62.7489122088559\\
1953.60000000002	62.7424886113528\\
1953.90000000003	62.7482088024565\\
1954.09999999998	62.741786920479\\
1954.59999999999	62.7513173376989\\
1954.7	62.7481072232453\\
1954.89999999998	62.7519181585678\\
1954.99999999998	62.7487085059588\\
1955.10000000002	62.7506137479542\\
1955.4	62.7409869598568\\
1955.80000000003	62.7486067794877\\
1955.99999999997	62.7421910945248\\
1957.00000000002	62.761228348066\\
1957.09999999999	62.7580216636011\\
1957.20000000003	62.7599243856333\\
1957.3	62.7567180954327\\
1957.40000000002	62.7586206896552\\
1957.59999999999	62.7522092251111\\
1957.69999999998	62.7541117580958\\
1957.79999999998	62.7509065835845\\
1957.89999999998	62.752808988764\\
1958.29999999997	62.7399918300654\\
1958.79999999999	62.7495022716831\\
1958.89999999998	62.7462991322103\\
1959.5	62.7577056542151\\
1960	62.7416968522014\\
1960.09999999999	62.7435975920825\\
1960.19999999997	62.7403968780289\\
1960.30000000002	62.7422974903081\\
1960.50000000001	62.7358971743344\\
1960.6	62.7377977253022\\
1960.70000000002	62.734598123215\\
1960.8	62.7364985465858\\
1961.20000000001	62.7237036659359\\
1961.39999999998	62.7275044608718\\
1961.5	62.7243066884176\\
1961.79999999997	62.7300066262297\\
1961.90000000003	62.7268093781855\\
1962.00000000003	62.7287090362367\\
1962.20000000003	62.7223156500026\\
1962.60000000002	62.729912875121\\
1962.69999999998	62.7267169349908\\
1962.79999999998	62.7286158235264\\
1963.10000000001	62.7190301548492\\
1963.6	62.7285226867648\\
1963.70000000001	62.7253284448518\\
1964.49999999999	62.7405069734297\\
1964.70000000002	62.7341205211726\\
1964.89999999998	62.7379134860051\\
1965.00000000002	62.7347208793446\\
1965.19999999999	62.738513204091\\
1965.29999999999	62.7353210542383\\
1965.60000000002	62.7410082922114\\
1965.69999999997	62.737816664971\\
1965.80000000002	62.7397120911542\\
1965.90000000003	62.736520854527\\
1966.2	62.7422061740324\\
1966.30000000003	62.7390154597234\\
1966.69999999998	62.7465934512914\\
1966.79999999998	62.7434033250292\\
1966.99999999999	62.7471912968329\\
1967.09999999999	62.7440016266775\\
1967.40000000003	62.7496823379924\\
1967.60000000003	62.7433043655029\\
1967.79999999997	62.7470908074597\\
1967.90000000001	62.7439024390244\\
1967.99999999998	62.7457954372237\\
1968.30000000001	62.7362324730746\\
1968.40000000002	62.738125476251\\
1968.5	62.7349385349995\\
1968.90000000003	62.7425088877603\\
1969.00000000001	62.739322533137\\
1969.09999999998	62.7412147064798\\
1969.19999999997	62.7380287411771\\
1969.30000000001	62.7399207880573\\
1969.40000000002	62.7367352119827\\
1969.60000000001	62.7405188607402\\
1969.80000000002	62.7341489415706\\
1969.90000000001	62.7360406091371\\
1970.09999999999	62.7296721145061\\
1970.20000000003	62.7315637212607\\
1970.29999999998	62.7283800243605\\
1970.49999999999	62.732162793058\\
1970.59999999999	62.7289795504136\\
1971.19999999999	62.7403236442957\\
1971.4	62.7339589145321\\
1971.50000000002	62.7358490566038\\
1971.60000000002	62.7326672414668\\
1971.70000000002	62.7345572573283\\
1971.8	62.7313758304174\\
1972.19999999998	62.7389342392131\\
1972.3	62.7357533968769\\
1972.79999999998	62.7451974251102\\
1972.89999999998	62.7420172326406\\
1973	62.74390552937\\
1973.4	62.7311882442361\\
1974.39999999997	62.7500633071664\\
1974.5	62.7468854451535\\
1974.8	62.7525444326295\\
1974.9	62.7493670886076\\
1975.00000000002	62.7512531011088\\
1975.2	62.7448995089353\\
1975.29999999997	62.7467854611724\\
1975.39999999998	62.7436092128575\\
1975.99999999997	62.7549213096503\\
1976.09999999998	62.7517457747192\\
1976.2	62.7536305216819\\
1976.40000000002	62.7472805464204\\
1976.69999999997	62.7529340348037\\
1976.99999999999	62.7434120681807\\
1977.19999999998	62.7471804986598\\
1977.30000000002	62.7440072822899\\
1977.59999999997	62.749658694443\\
1977.70000000003	62.7464859945394\\
1977.8	62.7483694827848\\
1977.90000000001	62.7451971688574\\
1978.19999999999	62.7508466865491\\
1978.29999999999	62.747674888799\\
1978.60000000001	62.7533228887654\\
1978.70000000003	62.7501516070346\\
1979.19999999998	62.7595614611226\\
1979.50000000001	62.7500505152556\\
1979.60000000001	62.7519321109259\\
1979.69999999999	62.7487625012628\\
1979.89999999999	62.7525252525252\\
1980.09999999999	62.7461872538127\\
1980.19999999999	62.7480684744736\\
1980.4	62.7417318858874\\
1980.70000000001	62.7473747980614\\
1980.79999999997	62.7442071785552\\
1980.90000000001	62.7460878344271\\
1981	62.7429205996669\\
1981.80000000002	62.7579595337807\\
1981.99999999999	62.7516270622068\\
1982.09999999999	62.7535062052265\\
1982.19999999999	62.7503405135449\\
1982.30000000001	62.7522195318806\\
1982.59999999999	62.742724567509\\
1982.69999999997	62.7446035908816\\
1982.80000000002	62.7414393060669\\
1983.00000000002	62.7451969139227\\
1983.1	62.742033077854\\
1983.20000000002	62.7439116623809\\
1983.40000000003	62.7375850768843\\
1983.49999999998	62.7394636015326\\
1983.79999999998	62.7299763092898\\
1983.9	62.7318548387097\\
1984.1	62.7255317004334\\
1984.40000000001	62.7311665406904\\
1984.5	62.7280056434546\\
1984.89999999998	62.735516372796\\
1984.99999999997	62.7323560525918\\
1985.29999999997	62.7379873073436\\
1985.40000000002	62.7348274993704\\
1985.49999999998	62.7367042707494\\
1985.80000000001	62.7272269499975\\
1986.09999999999	62.732856711308\\
1986.19999999997	62.7296984342748\\
1986.30000000003	62.7315747080145\\
1986.4	62.7284168134911\\
1986.5	62.7302929628511\\
1986.6	62.7271354507475\\
1986.7	62.7290114757399\\
1986.80000000001	62.7258543459661\\
1986.9	62.7277302466029\\
1987.29999999999	62.7151051625239\\
1987.4	62.7169811320755\\
1987.5	62.7138257194607\\
1987.59999999999	62.7157015646224\\
1988.00000000001	62.7030833459082\\
1988.10000000001	62.7049592596318\\
1988.20000000001	62.7018055625409\\
1988.50000000003	62.7074323644775\\
1988.59999999999	62.704279177352\\
1988.8	62.7080295640806\\
1988.90000000001	62.7048768225239\\
1989.00000000003	62.7067517972953\\
1989.39999999999	62.6941442573511\\
1989.5	62.6960193003619\\
1989.59999999997	62.6928682715987\\
1989.80000000001	62.6966179204985\\
1989.89999999997	62.6934673366834\\
1990.00000000003	62.695341942616\\
1990.1	62.6921917395237\\
1990.89999999998	62.707182320442\\
1991.09999999997	62.7008838891121\\
1991.40000000002	62.7065026362039\\
1991.60000000003	62.7002058542953\\
1991.7	62.70207852194\\
1991.79999999997	62.6989306692103\\
1991.99999999998	62.7026755684956\\
1992.2	62.6963810671084\\
1992.49999999997	62.7019973903443\\
1992.70000000002	62.6957045363308\\
1992.80000000003	62.6975763962065\\
1992.89999999998	62.6944305067737\\
1993.10000000001	62.698173790889\\
1993.2	62.6950283449556\\
1993.79999999997	62.7062540749285\\
1993.9	62.703109327984\\
1994.00000000003	62.7049796900858\\
1994.10000000002	62.7018353224351\\
1994.19999999999	62.7037055608484\\
1994.30000000002	62.7005615724027\\
1994.60000000003	62.7061713540883\\
1994.69999999998	62.7030278724684\\
1994.89999999999	62.7067669172932\\
1995.1	62.7004811547715\\
1995.3	62.7042197053222\\
1995.59999999999	62.6947938066844\\
1996	62.7022694253795\\
1996.10000000001	62.6991283438533\\
1996.29999999998	62.7028651572831\\
1996.4	62.6997245179063\\
1996.49999999999	62.7015927076029\\
1996.79999999999	62.6921728679453\\
1996.89999999998	62.6940410615924\\
1996.99999999998	62.690901807621\\
1997.59999999999	62.7021074235371\\
1997.70000000001	62.6989688657523\\
1998	62.7045693408738\\
1998.1	62.7014312881594\\
1998.30000000001	62.7051641313051\\
1998.59999999999	62.6957522389553\\
1999.19999999998	62.7069474316011\\
1999.39999999998	62.7006751687922\\
1999.49999999997	62.7025405081016\\
1999.59999999998	62.6994049107366\\
1999.9	62.705\\
2000	62.7018649067547\\
2000.20000000001	62.7055941608759\\
2000.29999999999	62.7024595080984\\
2000.49999999999	62.7061881435569\\
2000.79999999999	62.6967864460993\\
2001.00000000002	62.7005147169057\\
2001.2	62.6942487383201\\
2001.30000000001	62.6961127210952\\
2001.39999999999	62.6929802648014\\
2001.49999999998	62.6948441247003\\
2001.59999999999	62.691712044762\\
2002.40000000002	62.7066167290886\\
2002.50000000003	62.7034854688904\\
2003.00000000003	62.7127951674904\\
2003.2	62.7065342185394\\
2003.49999999998	62.7121181872629\\
2003.70000000001	62.7058588681505\\
2003.80000000001	62.7077199461051\\
2003.90000000003	62.7045908183633\\
2003.99999999998	62.7064517738636\\
2004.1	62.7033230216545\\
2004.40000000003	62.7089049638314\\
2004.49999999998	62.7057767135588\\
2004.90000000001	62.713216957606\\
2004.99999999998	62.7100892723555\\
2005.49999999998	62.719385719984\\
2005.60000000003	62.716258662811\\
2006	62.7236927371517\\
2006.10000000001	62.7205662446416\\
2006.20000000001	62.7224243632558\\
2006.29999999998	62.719298245614\\
2006.49999999998	62.723014053623\\
2006.6	62.7198883739473\\
2006.70000000002	62.7217460633845\\
2006.80000000002	62.7186207583836\\
2007.19999999997	62.7260499178\\
2007.3	62.7229251768457\\
2007.40000000003	62.7247820672478\\
2007.49999999998	62.7216577007372\\
2007.6	62.7235144692932\\
2007.79999999997	62.7172667961552\\
2007.89999999999	62.7191235059761\\
2007.99999999999	62.7160001991933\\
2008.60000000001	62.727136954249\\
2008.7	62.7240143369176\\
2008.89999999999	62.727725236436\\
2008.99999999998	62.7246030560948\\
2009.30000000002	62.7301682094158\\
2009.60000000002	62.7208041001144\\
2009.80000000001	62.7245136573959\\
2010.00000000002	62.7182727227501\\
2010.49999999997	62.7275440167114\\
2010.89999999998	62.7150671307807\\
2011.19999999998	62.7206284492617\\
2011.3	62.7175101919061\\
2011.50000000003	62.7212169417379\\
2011.59999999998	62.7180991201471\\
2011.80000000003	62.7218052587107\\
2011.9	62.7186878727634\\
2012.00000000002	62.720540728592\\
2012.09999999999	62.7174237153365\\
2012.20000000001	62.7192764498335\\
2012.29999999998	62.7161598091831\\
2012.79999999998	62.7254210343286\\
2013.20000000003	62.7129588238216\\
2013.30000000001	62.7148107678554\\
2013.39999999997	62.7116960516514\\
2013.50000000002	62.7135478744537\\
2013.69999999999	62.7073194954812\\
2013.99999999998	62.7128742366317\\
2014.1	62.7097606990368\\
2014.20000000002	62.7116119743832\\
2014.30000000001	62.7084988085782\\
2014.39999999998	62.7103499627699\\
2014.6	62.7041246835757\\
2014.80000000003	62.7078266911509\\
2014.89999999998	62.7047146401985\\
2015.20000000002	62.710266461569\\
2015.30000000002	62.7071549072144\\
2015.40000000002	62.7090052096254\\
2015.60000000002	62.7027831522548\\
2015.79999999998	62.7064834565207\\
2015.89999999999	62.703373015873\\
2015.99999999997	62.7052229552106\\
2016.10000000001	62.7021128856264\\
2016.40000000003	62.7076617902306\\
2016.89999999998	62.6921170054536\\
2017.1	62.6958159825501\\
2017.20000000002	62.69270807515\\
2017.59999999997	62.7001040789017\\
2017.8	62.6938896872987\\
2017.89999999999	62.6957383548067\\
2018.10000000002	62.6895253195917\\
2018.3	62.6932223543401\\
2018.60000000002	62.6839054837272\\
2018.70000000003	62.6857539132158\\
2018.79999999997	62.6826489672594\\
2019	62.6863454014165\\
2019.10000000003	62.6832408874802\\
2019.29999999997	62.6869367138754\\
2019.39999999997	62.6838326318396\\
2019.50000000003	62.6856803327392\\
2019.69999999997	62.6794732151698\\
2019.8	62.6813208574682\\
2020.00000000001	62.6751150933122\\
2020.49999999998	62.684351182817\\
2020.9	62.6719445818902\\
2021.40000000003	62.6811773435568\\
2021.6	62.6749765049216\\
2021.99999999997	62.6823599228525\\
2022.10000000003	62.6792602116507\\
2022.69999999997	62.6903302353174\\
2022.9	62.68413247652\\
2022.99999999999	62.6859769660422\\
2023.20000000003	62.6797805565166\\
2023.4	62.6834692364715\\
2023.50000000003	62.6803716149437\\
2023.69999999999	62.6840596896927\\
2023.89999999997	62.6778656126482\\
2024.49999999997	62.688926207646\\
2024.80000000003	62.6796385006667\\
2025.6	62.6943772523078\\
2025.70000000003	62.6912824563136\\
2025.80000000003	62.6931240436349\\
2025.89999999998	62.6900296150049\\
2026.10000000003	62.6937123679795\\
2026.40000000001	62.6844312854676\\
2026.60000000002	62.6881136823408\\
2026.69999999999	62.6850207223209\\
2027.00000000003	62.6905431404469\\
2027.29999999999	62.681266646937\\
2027.39999999999	62.6831072749692\\
2027.49999999997	62.6800157822056\\
2027.79999999999	62.6855367621678\\
2028.20000000002	62.6731745796973\\
2028.29999999999	62.6750147899823\\
2028.40000000002	62.6719250677841\\
2028.90000000002	62.6811237062592\\
2029.10000000002	62.6749457914449\\
2029.49999999999	62.6823019314151\\
2029.60000000003	62.6792136768981\\
2030.49999999998	62.6957549492761\\
2030.70000000002	62.6895804609021\\
2030.99999999997	62.6950913298213\\
2031.10000000003	62.6920047262702\\
2031.59999999997	62.7011861987498\\
2031.69999999997	62.6981002067133\\
2031.90000000003	62.7017716535433\\
2031.99999999998	62.6986860882831\\
2032.10000000002	62.7005216022045\\
2032.20000000002	62.697436402106\\
2032.30000000001	62.6992717968904\\
2032.39999999997	62.6961869618696\\
2032.8	62.7035269811599\\
2032.99999999999	62.697358713295\\
2033.6	62.7083640654964\\
2033.69999999997	62.7052807552365\\
2033.89999999997	62.708947885939\\
2034.09999999998	62.7027824206076\\
2034.19999999997	62.7046158383719\\
2034.40000000002	62.6984517080364\\
2034.89999999999	62.7076167076167\\
2035.00000000002	62.7045354036657\\
2035.29999999999	62.7100324260588\\
2035.40000000002	62.7069516089413\\
2035.50000000002	62.708783651012\\
2035.70000000003	62.7026230474506\\
2035.99999999997	62.7081184617651\\
2036.09999999999	62.7050387977605\\
2036.19999999999	62.7068703039827\\
2036.30000000002	62.7037910037321\\
2036.4	62.7056223913577\\
2036.49999999997	62.7025434547776\\
2037.00000000002	62.7116980020618\\
2037.09999999999	62.7086196740624\\
2037.19999999999	62.7104501055318\\
2037.3	62.707372140964\\
2037.69999999999	62.7146923152419\\
2037.80000000001	62.7116148976888\\
2038.10000000003	62.7171033264645\\
2038.30000000002	62.7109497645212\\
2038.49999999998	62.7146080643579\\
2038.59999999999	62.7115318585373\\
2038.70000000001	62.7133608004709\\
2038.8	62.7102849575752\\
2038.89999999998	62.7121137812653\\
2038.99999999997	62.7090383012113\\
2039.19999999998	62.7126955327809\\
2039.30000000003	62.7096204766108\\
2039.39999999999	62.7114488845305\\
2039.7	62.7022257084028\\
2039.99999999998	62.7077104063526\\
2040.1	62.7046368003137\\
2040.39999999999	62.7101200686106\\
2040.59999999998	62.7039741265252\\
2041.00000000001	62.7112831316447\\
2041.09999999999	62.708210856359\\
2041.19999999997	62.7100377210601\\
2041.49999999999	62.7008228840125\\
2042.29999999999	62.7154328241285\\
2042.4	62.7123623011016\\
2042.79999999997	62.7196632238485\\
2042.90000000001	62.7165932452276\\
2043.49999999997	62.7275396359366\\
2043.70000000002	62.7214013112829\\
2043.8	62.7232252067127\\
2043.89999999999	62.720156555773\\
2044.1	62.723803933079\\
2044.19999999999	62.720735704153\\
2044.60000000003	62.7280285616472\\
2044.79999999998	62.7218934911243\\
2045	62.7255390934429\\
2045.10000000003	62.7224721298651\\
2045.3	62.7261171409015\\
2045.49999999997	62.719984356668\\
2045.60000000001	62.7218067165274\\
2045.70000000001	62.7187408348812\\
2045.99999999998	62.7242070280045\\
2046.09999999999	62.7211416283843\\
2046.49999999999	62.7284276360794\\
2046.69999999998	62.7222982216142\\
2047.00000000003	62.7277612231938\\
2047.09999999998	62.7246971473232\\
2047.40000000003	62.7301587301587\\
2047.50000000002	62.7270951357687\\
2047.60000000002	62.7289153684622\\
2047.80000000001	62.7227891986913\\
2048.09999999999	62.7282491944146\\
2048.29999999997	62.722124585042\\
2048.5	62.7257639363468\\
2048.70000000001	62.719640765326\\
2048.90000000003	62.7232796486091\\
2048.99999999999	62.7202186325704\\
2049.2	62.7238569267555\\
2049.39999999999	62.7177360331788\\
2049.7	62.723192506586\\
2049.89999999997	62.7170731707317\\
2050.10000000001	62.7207101746171\\
2050.3	62.7145922746781\\
2050.69999999999	62.72186463819\\
2050.79999999998	62.7188063776878\\
2051.00000000003	62.7224416166935\\
2051.39999999997	62.7102120399708\\
2051.49999999997	62.7120296354065\\
2051.69999999999	62.7059167560191\\
2051.8	62.7077342950436\\
2051.9	62.7046783625731\\
2051.99999999999	62.7064957848058\\
2052.19999999997	62.7003849339765\\
2052.30000000002	62.7022022997466\\
2052.39999999998	62.6991473812424\\
2052.59999999997	62.7027817021484\\
2052.7	62.6997272018706\\
2053.09999999997	62.7069939606468\\
2053.19999999998	62.703939999026\\
2053.40000000003	62.7075724373022\\
2053.60000000002	62.7014656473682\\
2054.10000000002	62.7105442508032\\
2054.19999999999	62.7074916029791\\
2054.29999999999	62.7093068535826\\
2054.40000000001	62.7062545631541\\
2054.49999999998	62.7080696972647\\
2054.80000000003	62.6989147890408\\
2054.89999999997	62.7007299270073\\
2054.99999999998	62.6976789450635\\
2055.19999999999	62.7013088113657\\
2055.29999999999	62.69825824657\\
2055.69999999999	62.705516100788\\
2055.79999999998	62.7024660732526\\
2056.20000000003	62.7097213441618\\
2056.49999999997	62.7005737625207\\
2056.79999999999	62.7060139044193\\
2056.90000000002	62.7029654837142\\
2057.39999999998	62.7120291616039\\
2057.5	62.7089813374806\\
2057.90000000002	62.7162293488824\\
2057.99999999997	62.7131820611243\\
2058.1	62.7149936838014\\
2058.19999999997	62.7119467521741\\
2058.5	62.7173807441951\\
2058.70000000001	62.7112881290072\\
2058.80000000002	62.713099227743\\
2058.89999999998	62.7100534239922\\
2059	62.7118644067797\\
2059.09999999997	62.708818958819\\
2059.40000000003	62.7142510318038\\
2059.60000000001	62.7081613827256\\
2059.7	62.7099718419264\\
2059.90000000002	62.7038834951456\\
2060.3	62.7111240535818\\
2060.39999999999	62.7080805629702\\
2061.00000000001	62.7189364902237\\
2061.10000000002	62.715893654182\\
2061.30000000002	62.7195110119336\\
2061.40000000002	62.7164685908319\\
2061.49999999998	62.7182770663562\\
2061.60000000002	62.7152350002425\\
2061.70000000002	62.7170433601707\\
2061.79999999999	62.7140016489646\\
2061.89999999998	62.7158098933075\\
2062	62.7127685369284\\
2062.09999999998	62.7145766656968\\
2062.2	62.7115356640644\\
2062.40000000002	62.7151515151515\\
2062.70000000001	62.7060306379678\\
2063.3	62.7168750605796\\
2063.40000000001	62.7138357160165\\
2063.80000000002	62.7210620669606\\
2063.89999999998	62.7180232558139\\
2064.00000000001	62.7198294656267\\
2064.30000000003	62.7107149777175\\
2064.40000000002	62.7125211915718\\
2064.49999999997	62.7094836772256\\
2064.60000000003	62.7112897757544\\
2064.69999999999	62.7082526152654\\
2064.99999999999	62.7136700401918\\
2065.1	62.7106333527019\\
2065.29999999999	62.7142442141958\\
2065.39999999998	62.7112079399661\\
2065.49999999998	62.7130131680868\\
2065.59999999998	62.7099772474222\\
2066	62.7171966506945\\
2066.20000000003	62.7111261675459\\
2066.3	62.7129307007356\\
2066.70000000002	62.7007934971937\\
2067.79999999999	62.7206344600803\\
2067.99999999999	62.7145689280016\\
2068.79999999999	62.7289864179032\\
2068.90000000001	62.7259545674239\\
2069.40000000001	62.7349601352984\\
2069.60000000002	62.7288979079094\\
2069.89999999999	62.7342995169082\\
2070	62.7312690208202\\
2070.10000000001	62.7330692686697\\
2070.20000000003	62.7300391247645\\
2070.49999999998	62.7354390031875\\
2070.60000000003	62.7324093301782\\
2070.69999999999	62.7342090013521\\
2070.8	62.7311796803322\\
2071.30000000003	62.7401757265617\\
2071.39999999998	62.7371469949312\\
2071.49999999998	62.7389457424213\\
2071.60000000002	62.7359173625525\\
2071.69999999998	62.7377159957525\\
2071.80000000001	62.734687967566\\
2071.89999999998	62.7364864864865\\
2072.09999999998	62.7304314255381\\
2072.20000000002	62.7322298894948\\
2072.60000000003	62.7201235103971\\
2072.89999999998	62.7255185721177\\
2073.00000000001	62.7224928850514\\
2073.20000000001	62.7260888438721\\
2073.3	62.7230635670879\\
2073.60000000001	62.7284563823118\\
2073.70000000002	62.7254315748867\\
2073.99999999999	62.7308230075696\\
2074.09999999999	62.7277986693665\\
2074.4	62.7331887201735\\
2074.5	62.7301648510556\\
2074.59999999999	62.7319612474093\\
2075.10000000002	62.7168465690054\\
2075.30000000003	62.7204394333622\\
2075.50000000002	62.7143958373482\\
2075.89999999999	62.7215799614644\\
2076.10000000003	62.7155380021193\\
2076.29999999999	62.7191292621846\\
2076.70000000001	62.7070493066256\\
2076.79999999997	62.7088449130916\\
2076.89999999998	62.7058257101589\\
2077.09999999998	62.7094165222415\\
2077.2	62.7063977278198\\
2077.6	62.7135775135968\\
2077.70000000001	62.7105592453557\\
2078.00000000001	62.7159424474279\\
2078.30000000001	62.7068899153195\\
2079.19999999998	62.7230317895446\\
2079.29999999999	62.7200153890545\\
2079.49999999997	62.7236006924409\\
2079.69999999997	62.7175689970189\\
2079.90000000003	62.7211538461538\\
2080.1	62.7151235458129\\
2080.39999999997	62.7204998798366\\
2080.50000000001	62.7174853407671\\
2080.59999999998	62.7192771663383\\
2080.70000000002	62.7162629757785\\
2080.80000000002	62.7180546878754\\
2081.10000000002	62.7090140303671\\
2081.29999999997	62.7125972902854\\
2081.59999999998	62.7035595907191\\
2082.09999999997	62.712515608491\\
2082.30000000002	62.7064925086439\\
2082.99999999997	62.7190245307474\\
2083.2	62.7130034080545\\
2083.30000000003	62.71479312662\\
2083.40000000002	62.7117830573554\\
2083.49999999997	62.7135726626992\\
2083.99999999998	62.6985269420853\\
2084.1	62.700316668266\\
2084.20000000003	62.6973084488797\\
2084.89999999999	62.7098321342926\\
2084.99999999999	62.7068246127284\\
2085.10000000002	62.7086130826779\\
2085.20000000003	62.7056059080228\\
2085.4	62.7091824502517\\
2085.59999999997	62.703169199789\\
2086.10000000002	62.7121081392005\\
2086.2	62.7091022384125\\
2086.30000000002	62.7108895705521\\
2086.59999999998	62.7018737719845\\
2086.79999999997	62.7054482725574\\
2087.00000000003	62.6994394135403\\
2087.20000000003	62.7030134623677\\
2087.29999999998	62.7000095812973\\
2087.39999999998	62.7017964071856\\
2087.59999999998	62.6957896249461\\
2087.70000000001	62.6975763962065\\
2087.89999999998	62.6915708812261\\
2088.09999999997	62.6951441432813\\
2088.19999999998	62.6921419336302\\
2088.39999999998	62.6957146277232\\
2088.59999999997	62.6897113036817\\
2088.90000000001	62.6950694112015\\
2088.99999999998	62.6920683547939\\
2089.5	62.7009954058193\\
2089.6	62.6979949275016\\
2089.9	62.7033492822967\\
2090.10000000003	62.6973495359296\\
2090.70000000001	62.7080543332696\\
2090.79999999999	62.7050552393706\\
2090.90000000003	62.7068388330942\\
2091.00000000002	62.7038400841662\\
2091.09999999998	62.705623565417\\
2091.20000000001	62.7026251613829\\
2091.80000000003	62.7133228165782\\
2091.90000000001	62.7103250478011\\
2092.10000000001	62.7138896854985\\
2092.29999999998	62.7078952399159\\
2092.60000000002	62.7132412672624\\
2093.00000000002	62.7012565094835\\
2093.29999999997	62.7066017005828\\
2093.50000000002	62.7006113870845\\
2093.90000000003	62.7077363896848\\
2094.00000000002	62.7047418938924\\
2094.10000000003	62.7065227771942\\
2094.19999999997	62.7035286253163\\
2094.5	62.7088704287215\\
2094.99999999997	62.6939048255453\\
2095.19999999999	62.6974657566935\\
2095.6	62.6854988786563\\
2095.7	62.6872793205459\\
2095.79999999997	62.6842883725369\\
2096.10000000002	62.6896288522088\\
2096.30000000002	62.6836481587483\\
2096.79999999997	62.6925461395393\\
2097.10000000002	62.6835781041388\\
2097.3	62.6871364546581\\
2097.59999999999	62.6781713305048\\
2097.70000000001	62.679950424254\\
2097.89999999997	62.67397521449\\
2098.20000000003	62.6793118238574\\
2098.30000000003	62.6763248189096\\
2098.40000000003	62.6781034071956\\
2098.5	62.6751167444963\\
2098.7	62.6786735277301\\
2099.00000000002	62.6697155923968\\
2099.09999999999	62.671493902439\\
2099.3	62.6655234828999\\
2099.40000000003	62.6673017385092\\
2099.50000000001	62.6643170127643\\
2099.79999999998	62.6696509357589\\
2099.99999999997	62.6636826817771\\
2100.29999999997	62.6690154256332\\
2100.39999999998	62.6660318971673\\
2101.30000000002	62.6820215094699\\
2101.39999999997	62.6790387818225\\
2101.59999999998	62.6825902840558\\
2101.69999999997	62.6796079550861\\
2101.89999999998	62.6831588962893\\
2101.99999999999	62.6801769658913\\
2102.1	62.6819522405099\\
2102.40000000003	62.6730083234245\\
2102.6	62.6765587102297\\
2102.80000000003	62.6705977459699\\
2103.79999999997	62.6883407006036\\
2104.09999999997	62.6794030985648\\
2104.60000000001	62.6882691119875\\
2104.70000000002	62.6852907639681\\
2104.8	62.6870635184569\\
2104.89999999998	62.6840855106888\\
2105.39999999999	62.6929470434576\\
2105.50000000003	62.6899696048632\\
2105.70000000003	62.6935131541457\\
2106.10000000002	62.6816066850252\\
2106.30000000003	62.6851500189897\\
2106.69999999997	62.6732485285741\\
2107.69999999999	62.6909573963374\\
2107.8	62.6879833009156\\
2107.89999999999	62.6897533206831\\
2108.1	62.6838060905038\\
2108.30000000002	62.6873458546765\\
2108.40000000003	62.6843727768556\\
2108.60000000002	62.6879119836866\\
2108.99999999998	62.6760229481769\\
2109.29999999998	62.681331184223\\
2109.40000000003	62.6783598009007\\
2109.6	62.6818979001754\\
2109.69999999998	62.6789269125036\\
2109.80000000003	62.6806957675719\\
2109.89999999998	62.6777251184834\\
2110.09999999997	62.6812624395792\\
2110.30000000002	62.6753222137983\\
2110.70000000001	62.6823953003601\\
2111.1	62.6705191360364\\
2111.19999999997	62.672287216407\\
2111.29999999998	62.6693189353036\\
2111.50000000001	62.6728547073309\\
2111.59999999997	62.6698868210447\\
2112.00000000002	62.6769565834951\\
2112.20000000002	62.671022108602\\
2112.29999999998	62.6727892444613\\
2112.50000000004	62.6668560068162\\
2112.69999999997	62.6703900037865\\
2112.79999999997	62.6674239197312\\
2112.90000000002	62.669190724089\\
2113	62.6662249775212\\
2113.40000000002	62.6732907499409\\
2113.49999999999	62.6703255109765\\
2113.59999999997	62.6720915929413\\
2113.70000000001	62.6691266912669\\
2113.9	62.6726584673605\\
2114.10000000003	62.6667297322864\\
2114.20000000001	62.6684954831386\\
2114.4	62.6625679829747\\
2114.50000000002	62.6643336801286\\
2114.60000000001	62.66137040715\\
2114.99999999997	62.6684317526358\\
2115.09999999997	62.6654689863843\\
2115.39999999998	62.6707634129047\\
2115.50000000002	62.6678010966156\\
2116	62.676622087803\\
2116.29999999998	62.6677376677377\\
2116.8	62.6765553403562\\
2116.99999999998	62.6706343583203\\
2117.30000000002	62.675923302163\\
2117.39999999999	62.6729634002361\\
2118	62.6835371323356\\
2118.3	62.6746601208459\\
2118.40000000001	62.6764219966958\\
2118.50000000003	62.6734636080431\\
2118.60000000003	62.6752253740501\\
2118.69999999998	62.6722673211252\\
2118.8	62.6740289772995\\
2118.89999999997	62.6710712600283\\
2119.00000000002	62.6728328063801\\
2119.09999999997	62.6698754246886\\
2119.2	62.6716368612278\\
2119.60000000002	62.6598103505213\\
2120.10000000001	62.668616168286\\
2120.40000000001	62.6597500589484\\
2120.49999999999	62.6615108931434\\
2120.90000000002	62.6496935407826\\
2120.99999999998	62.6514544340201\\
2121.10000000003	62.6485008485763\\
2121.30000000001	62.6520222494579\\
2121.50000000002	62.6461161387632\\
2121.69999999997	62.649637100575\\
2122.00000000001	62.6407803590783\\
2122.30000000003	62.6460610629476\\
2122.39999999997	62.643109540636\\
2122.49999999997	62.6448694996702\\
2122.59999999998	62.6419183115843\\
2122.69999999999	62.6436781609195\\
2122.90000000003	62.637776731041\\
2123.10000000001	62.6412961567445\\
2123.2	62.6383459708944\\
2123.40000000003	62.6418648457735\\
2123.90000000002	62.6271186440678\\
2124.00000000002	62.6288781130832\\
2124.09999999999	62.6259297617927\\
2124.40000000001	62.6312073429042\\
2124.50000000001	62.6282594370705\\
2125.19999999999	62.6405683903449\\
2125.29999999998	62.6376211536652\\
2125.40000000002	62.6393789696542\\
2125.50000000001	62.6364320662401\\
2125.80000000001	62.6417046897785\\
2125.89999999998	62.6387582314205\\
2125.99999999999	62.6405154978599\\
2126.40000000001	62.6287326592993\\
2126.50000000002	62.630489984012\\
2126.59999999999	62.6275450228053\\
2126.69999999997	62.6293022381042\\
2126.99999999999	62.6204691833952\\
2127.40000000001	62.6274970622797\\
2127.5	62.6245534874976\\
2127.7	62.6280665476079\\
2127.89999999997	62.6221804511278\\
2128.50000000003	62.632716339378\\
2128.90000000003	62.6209488022546\\
2129.59999999999	62.633234727896\\
2129.70000000004	62.6302939243121\\
2129.99999999998	62.6355570161025\\
2130.19999999997	62.6296765713749\\
2130.39999999997	62.6331846984276\\
2130.60000000002	62.6273055803257\\
2130.79999999999	62.6308132713877\\
2131.1	62.621996996997\\
2131.2	62.6237507624455\\
2131.40000000004	62.6178747361013\\
2131.69999999998	62.6231353785533\\
2131.79999999998	62.6201979454946\\
2132.00000000003	62.6237043290652\\
2132.19999999997	62.6178305116541\\
2132.59999999997	62.6248417498945\\
2132.89999999998	62.6160337552743\\
2133.1	62.6195387211701\\
2133.30000000002	62.6136683228649\\
2133.40000000003	62.6154206702601\\
2133.89999999999	62.6007497656982\\
2134.19999999997	62.6060066532353\\
2134.79999999998	62.5884116352054\\
2135.2	62.5954198473283\\
2135.29999999999	62.5924885267397\\
2135.60000000001	62.5977431287166\\
2135.70000000001	62.5948122483379\\
2135.79999999997	62.5965635095276\\
2135.90000000001	62.5936329588015\\
2136.29999999997	62.6006365849092\\
2136.40000000003	62.5977065293705\\
2136.60000000002	62.6012074694623\\
2136.69999999998	62.5982777985773\\
2136.90000000002	62.6017781937295\\
2136.99999999999	62.5988489073979\\
2137.10000000003	62.6005989144675\\
2137.20000000003	62.5976699574229\\
2137.29999999999	62.5994198558997\\
2137.59999999998	62.5906347944052\\
2137.99999999998	62.5976334128432\\
2138.09999999997	62.5947058273314\\
2138.40000000002	62.5999532382511\\
2138.49999999999	62.5970260918358\\
2138.60000000002	62.5987749567494\\
2138.69999999997	62.5958481391434\\
2138.80000000002	62.5975968956005\\
2138.90000000002	62.5946704067321\\
2139.10000000002	62.598167539267\\
2139.19999999998	62.595241434114\\
2139.30000000003	62.5969898102272\\
2139.40000000001	62.5940640336527\\
2139.59999999999	62.5975604056644\\
2139.70000000002	62.594635012618\\
2139.79999999998	62.5963830085518\\
2139.90000000001	62.5934579439252\\
2140.00000000001	62.5952058315032\\
2140.10000000003	62.5922810952247\\
2140.5	62.5992712323648\\
2140.70000000002	62.593423019432\\
2140.89999999997	62.5969173283512\\
2140.99999999999	62.5939937415347\\
2141.5	62.6027269331341\\
2141.6	62.5998038941028\\
2142.00000000001	62.6067877316652\\
2142.09999999997	62.6038651853235\\
2142.6	62.6125915900499\\
2142.99999999997	62.6009052307405\\
2143.20000000002	62.6043950916811\\
2143.4	62.5985537672032\\
2143.79999999997	62.6055319744391\\
2144.20000000003	62.5938534719955\\
2145	62.6078038319892\\
2145.1	62.6048853253776\\
2145.80000000001	62.6170837410877\\
2145.90000000002	62.614165890028\\
2146.10000000001	62.6176497996459\\
2146.20000000003	62.6147323300564\\
2147.50000000002	62.6373626373626\\
2147.59999999997	62.6344461516972\\
2148.4	62.6483593204561\\
2148.50000000002	62.6454435446337\\
2148.6	62.6471820170336\\
2149.09999999999	62.6326074818537\\
2149.30000000004	62.6360844886945\\
2149.39999999997	62.6331705047686\\
2149.80000000001	62.6401227964091\\
2149.89999999997	62.6372093023256\\
2150.09999999999	62.6406845874802\\
2150.29999999998	62.6348586309524\\
2150.49999999999	62.6383334883288\\
2150.80000000002	62.6295969129202\\
2151.09999999997	62.6348084789885\\
2151.2	62.6318969925162\\
2151.29999999999	62.633633912801\\
2151.39999999999	62.6307227515687\\
2151.49999999999	62.6324595649749\\
2151.70000000003	62.6266381633981\\
2151.99999999998	62.6318479624553\\
2152.10000000001	62.6289378310566\\
2152.19999999997	62.6306741625238\\
2152.3	62.6277643560677\\
2152.40000000001	62.6295005807201\\
2152.59999999997	62.6236818878618\\
2153.49999999998	62.6393016344725\\
2153.70000000003	62.6334850032501\\
2153.8	62.6352198337899\\
2153.99999999999	62.6294043916253\\
2154.19999999997	62.6328737873091\\
2154.40000000003	62.6270596426085\\
2154.80000000003	62.6339969372129\\
2155.1	62.6252783964365\\
2155.29999999999	62.6287464043797\\
2155.49999999999	62.6229356095751\\
2155.59999999997	62.6246694809111\\
2155.70000000002	62.6217645421653\\
2155.79999999999	62.6234983069716\\
2156.00000000002	62.6176893465053\\
2156.2	62.6211566108612\\
2156.29999999997	62.6182526432944\\
2156.60000000003	62.6234524968702\\
2156.69999999997	62.6205489614243\\
2156.90000000003	62.6240148354196\\
2157	62.6211116777154\\
2157.2	62.6245770175683\\
2157.59999999998	62.6129675117023\\
2158.00000000003	62.6198971317363\\
2158.10000000001	62.6169956445186\\
2158.19999999998	62.6187277023583\\
2158.49999999997	62.6100250162142\\
2158.6	62.611757076018\\
2158.79999999999	62.6059567372273\\
2159.19999999998	62.6128838049368\\
2159.29999999998	62.6099842548856\\
2159.60000000002	62.6151780339862\\
2159.80000000002	62.6093800638918\\
2159.9	62.6111111111111\\
2160	62.6082125827508\\
2160.10000000003	62.6099435237478\\
2160.29999999999	62.6041473801148\\
2160.69999999998	62.6110699740837\\
2161.20000000001	62.5965853884236\\
2161.49999999997	62.6017764618801\\
2161.60000000002	62.5988805107092\\
2161.70000000003	62.6006106022759\\
2161.79999999999	62.5977149729405\\
2161.89999999997	62.5994449583719\\
2161.99999999997	62.5965496508025\\
2162.5	62.6051974475169\\
2162.70000000003	62.5994081745885\\
2162.90000000003	62.6028663892742\\
2163	62.5999722620313\\
2163.10000000003	62.601701183432\\
2163.39999999998	62.5930205685232\\
2163.89999999997	62.6016635859519\\
2164.09999999998	62.5958783846225\\
2164.19999999998	62.597606616458\\
2164.30000000001	62.594714470523\\
2164.39999999999	62.5964425964426\\
2164.49999999998	62.5935507715051\\
2164.70000000002	62.5970066518847\\
2164.79999999999	62.594115201626\\
2165.10000000003	62.5992979863292\\
2165.20000000004	62.5964069643929\\
2165.29999999997	62.5981342938949\\
2165.5	62.592353158478\\
2165.60000000001	62.5940804358868\\
2165.70000000002	62.5911903222828\\
2165.90000000004	62.5946445060019\\
2166.00000000004	62.5917547666313\\
2166.60000000003	62.6021138136337\\
2166.79999999997	62.5963357792238\\
2167.10000000003	62.6015134736065\\
2167.29999999998	62.5957368275353\\
2168.29999999999	62.6129865338498\\
2168.39999999998	62.6100991468757\\
2168.50000000002	62.6118232961358\\
2168.69999999998	62.6060494282553\\
2169.69999999997	62.6232832519126\\
2169.90000000001	62.6175115207373\\
2169.99999999999	62.6192341366757\\
2170.4	62.6076940797051\\
2170.6	62.6111392638319\\
2170.70000000001	62.6082550211903\\
2170.89999999998	62.6116996775679\\
2171.00000000003	62.6088158076551\\
2171.29999999998	62.6139817629179\\
2171.40000000002	62.6110983191342\\
2172.00000000002	62.6214262695088\\
2172.20000000001	62.6156608203287\\
2172.40000000003	62.6191024165708\\
2172.7	62.6104565537555\\
2172.80000000002	62.61217727461\\
2172.99999999997	62.6064147991349\\
2173.20000000004	62.6098559793862\\
2173.30000000003	62.6069752461581\\
2173.40000000003	62.6086956521739\\
2173.70000000002	62.6000552028705\\
2174.09999999999	62.6069358844633\\
2174.2	62.6040564779469\\
2174.49999999997	62.6092154879058\\
2174.6	62.6063365061848\\
2174.70000000001	62.6080559131874\\
2174.8	62.6051772495287\\
2175.30000000001	62.613772179829\\
2175.39999999999	62.6108940473454\\
2175.80000000003	62.6177673606324\\
2175.89999999999	62.6148897058823\\
2175.99999999998	62.6166076926612\\
2176.10000000001	62.6137303556659\\
2176.20000000001	62.6154482378349\\
2176.3	62.612571218526\\
2176.39999999997	62.6142889960946\\
2176.50000000001	62.6114122944041\\
2176.80000000002	62.61656483991\\
2176.9	62.6136885622416\\
2177.09999999999	62.6171229101598\\
2177.20000000003	62.6142470031691\\
2177.5	62.6193975018369\\
2177.59999999999	62.6165220186435\\
2177.70000000003	62.6182385894022\\
2177.90000000001	62.6124885215794\\
2178.1	62.6159214029933\\
2178.19999999998	62.6130468714135\\
2178.29999999997	62.614763128902\\
2178.49999999998	62.6090149637382\\
2178.69999999999	62.6124472186525\\
2178.80000000003	62.6095736380743\\
2179.39999999999	62.6198669419592\\
2179.5	62.6169939438429\\
2180.00000000003	62.6255676345122\\
2180.2	62.6198229601431\\
2180.29999999997	62.6215373325995\\
2180.40000000002	62.6186654437056\\
2180.50000000003	62.6203797120059\\
2180.6	62.6175081395882\\
2180.90000000003	62.6226501604769\\
2180.99999999999	62.619779010591\\
2181.19999999999	62.6232063448402\\
2181.50000000001	62.6145947928126\\
2181.80000000003	62.6197350932673\\
2182	62.6139956922231\\
2182.20000000001	62.6174219859781\\
2182.30000000001	62.6145527859237\\
2182.59999999997	62.6196912081367\\
2182.70000000002	62.6168224299065\\
2182.79999999998	62.6185349764075\\
2182.90000000003	62.6156665139716\\
2183.00000000002	62.6173789565297\\
2183.10000000003	62.6145108098204\\
2183.2	62.616223148445\\
2183.30000000002	62.6133553173949\\
2183.49999999999	62.6167796299689\\
2183.70000000002	62.6110449674879\\
2184.00000000002	62.6161805778124\\
2184.19999999999	62.6104472828824\\
2184.29999999999	62.6121589452481\\
2184.39999999999	62.6092927443351\\
2184.60000000003	62.6127157046734\\
2184.79999999998	62.606984301341\\
2184.99999999998	62.6104068463686\\
2185.59999999999	62.5932195635266\\
2185.69999999997	62.5949309177418\\
2185.80000000003	62.5920673406835\\
2186.89999999999	62.6108824874257\\
2187.20000000003	62.6022950669776\\
2187.4	62.6057142857143\\
2187.7	62.5971295365207\\
2188.40000000003	62.6090929860635\\
2188.60000000003	62.6033718645772\\
2188.69999999998	62.6050804093567\\
2188.79999999999	62.602220293298\\
2189.4	62.612468600137\\
2189.5	62.6096090610157\\
2189.60000000001	62.6113166187149\\
2189.69999999997	62.6084573933693\\
2189.80000000004	62.6101648477099\\
2190.00000000003	62.6044472855121\\
2190.09999999999	62.6061546890695\\
2190.20000000003	62.6032963520979\\
2190.59999999997	62.6101246177021\\
2191.00000000004	62.5986947195473\\
2191.40000000002	62.6055213324207\\
2191.50000000002	62.6026647198394\\
2191.70000000001	62.6060771968245\\
2191.99999999998	62.5975092377173\\
2192.09999999997	62.5992154000547\\
2192.39999999999	62.5906499429875\\
2192.59999999998	62.5940621152004\\
2192.70000000004	62.5912075884714\\
2192.80000000003	62.5929134935474\\
2192.89999999999	62.5900592795258\\
2193.30000000002	62.5968815537522\\
2193.39999999998	62.594027809437\\
2193.59999999997	62.5974381182477\\
2193.79999999997	62.5917316194904\\
2193.99999999997	62.5951415158835\\
2194.09999999998	62.5922887612797\\
2194.29999999997	62.5956981407218\\
2194.39999999998	62.5928457507405\\
2194.60000000002	62.5962546133868\\
2194.70000000002	62.5934025879351\\
2195.10000000003	62.6002186588921\\
2195.19999999997	62.5973671024461\\
2195.30000000001	62.5990707843673\\
2195.39999999997	62.5962195399681\\
2195.49999999999	62.5979231189652\\
2196.1	62.5808214188143\\
2196.20000000001	62.5825251559441\\
2196.30000000001	62.5796758331816\\
2196.89999999997	62.5898953117888\\
2197.00000000002	62.5870465613764\\
2197.09999999997	62.5887493173129\\
2197.19999999999	62.5859008783507\\
2197.30000000001	62.5876035314463\\
2197.39999999998	62.584755403868\\
2197.49999999999	62.5864579541318\\
2197.59999999999	62.5836101378714\\
2197.90000000001	62.5887170154686\\
2197.99999999999	62.5858696146672\\
2198.29999999999	62.5909752547307\\
2198.50000000002	62.5852815428\\
2198.69999999998	62.5886847371293\\
2198.80000000003	62.5858383737323\\
2199.09999999997	62.5909421607857\\
2199.19999999999	62.5880962124312\\
2199.49999999997	62.5931987634115\\
2199.60000000002	62.5903532299859\\
2199.80000000002	62.5937542615574\\
2200.00000000002	62.588064178901\\
2200.09999999998	62.5897645668576\\
2200.19999999997	62.5869199654593\\
2200.69999999997	62.5954198473283\\
2200.79999999999	62.59257576446\\
2201.20000000001	62.599373097715\\
2201.50000000001	62.5908430232558\\
2201.79999999997	62.5959398701122\\
2202.00000000003	62.590254756823\\
2202.19999999999	62.5936520909958\\
2202.29999999999	62.5908100254268\\
2202.40000000001	62.5925085130533\\
2202.5	62.5896667574684\\
2202.59999999997	62.5913651427793\\
2202.9	62.5828415796641\\
2202.99999999997	62.584539966411\\
2203.10000000003	62.5816993464052\\
2203.19999999998	62.5833976308265\\
2203.50000000003	62.5748774732256\\
2203.59999999997	62.5765757589509\\
2203.69999999997	62.573736273709\\
2203.79999999999	62.5754344570988\\
2203.89999999997	62.5725952813067\\
2204.19999999998	62.5776890622873\\
2204.39999999999	62.5720117940576\\
2204.49999999998	62.5737095164656\\
2204.70000000003	62.5680333817126\\
2204.90000000004	62.5714285714286\\
2205.60000000001	62.5515709298635\\
2205.80000000003	62.5549662269369\\
2205.9	62.5521305530372\\
2206.09999999998	62.5555253376847\\
2206.20000000003	62.5526900240221\\
2206.60000000001	62.5594779535052\\
2206.70000000003	62.5566431031358\\
2206.79999999998	62.5583397525941\\
2206.89999999998	62.5555052106933\\
2207.00000000004	62.5572017579629\\
2207.19999999999	62.5515335477733\\
2207.3	62.5532300443961\\
2207.40000000002	62.5503963759909\\
2207.60000000003	62.5537890111881\\
2207.79999999999	62.5481226504824\\
2208.19999999997	62.5458497486755\\
2209.00000000001	62.5594133357476\\
2209.59999999998	62.542426573743\\
2210.29999999999	62.5542888165038\\
2210.40000000001	62.5514589459398\\
2210.69999999999	62.5565406187805\\
2210.90000000003	62.550881953867\\
2211.1	62.5542691751085\\
2211.30000000002	62.5486117391698\\
2211.59999999997	62.5536917303432\\
2211.89999999997	62.5452079566004\\
2212.29999999998	62.5519797504972\\
2212.40000000003	62.5491525423729\\
2212.70000000002	62.5451916124367\\
2213.09999999999	62.5474426170251\\
2213.19999999997	62.5446166357927\\
2213.49999999998	62.5496928080954\\
2213.79999999998	62.54121685713\\
2214.29999999999	62.5496748554913\\
2214.4	62.5468503048092\\
2214.79999999998	62.5400695290984\\
2215.09999999998	62.5361141206212\\
2216	62.5513289111502\\
2216.1	62.5485064524863\\
2216.6	62.5569540307665\\
2217.20000000002	62.5400261579398\\
2218.09999999997	62.5552249571725\\
2218.29999999997	62.5495852866931\\
2218.5	62.552961326963\\
2218.69999999999	62.5473228772309\\
2218.90000000003	62.5506985128436\\
2219.00000000004	62.5478797710784\\
2219.20000000001	62.5512549001938\\
2219.29999999998	62.5484365143733\\
2219.89999999997	62.5540540540541\\
2220	62.5512364307914\\
2220.29999999998	62.547288776797\\
2220.6	62.5523483586257\\
2220.90000000002	62.5438991445295\\
2221.09999999999	62.5472717450027\\
2221.19999999997	62.5444559492189\\
2221.59999999999	62.5466984741414\\
2221.70000000003	62.5438833378342\\
2221.89999999998	62.5472547254726\\
2222.00000000003	62.5444399441969\\
2222.2	62.5478108266211\\
2222.29999999998	62.544996400288\\
2222.59999999998	62.5500517388761\\
2222.90000000002	62.5416104363473\\
2223.1	62.5449802087082\\
2223.20000000001	62.5421670489813\\
2223.6	62.5399109592121\\
2224.19999999999	62.5455199388572\\
2224.30000000003	62.5427081460169\\
2224.59999999997	62.547759248438\\
2224.89999999998	62.5393258426966\\
2225.69999999999	62.552790008087\\
2225.90000000002	62.5471698113208\\
2226.20000000003	62.5522166823878\\
2226.39999999997	62.5465977992365\\
2226.79999999998	62.5443441555526\\
2227	62.5477077814198\\
2227.20000000003	62.5420913213308\\
2227.50000000002	62.5381576584665\\
2227.89999999998	62.5359066427289\\
2228.39999999997	62.5443123177025\\
2228.49999999998	62.5415058781298\\
2228.90000000003	62.5437415881561\\
2229	62.5409358036876\\
2229.30000000003	62.5459764959182\\
2229.4	62.5431711145997\\
2229.8	62.5364366115072\\
2229.99999999998	62.5397964216851\\
2230.10000000002	62.5369921980091\\
2230.6	62.5453893396692\\
2230.70000000003	62.5425856195087\\
2231.10000000002	62.5448189315167\\
2231.19999999999	62.5420158651907\\
2231.4	62.5453730674434\\
2231.50000000004	62.5425703531099\\
2231.80000000002	62.5476051794435\\
2232.4	62.5307950727884\\
2232.89999999997	62.5391849529781\\
2233.00000000001	62.53638439837\\
2233.4	62.5430937989702\\
2233.90000000002	62.5290957923008\\
2234.09999999999	62.5324500939934\\
2234.20000000002	62.5296513449402\\
2234.5	62.5346818222501\\
2234.7	62.5290853767675\\
2236.10000000003	62.5525444951257\\
2236.30000000001	62.5469504560901\\
2236.69999999999	62.5402360515021\\
2237.10000000003	62.533524047917\\
2237.5	62.5402216660708\\
2237.70000000003	62.5346322280812\\
2238.00000000003	62.5396541709486\\
2238.19999999999	62.5340660322566\\
2238.4	62.5374134465044\\
2238.49999999997	62.534619851693\\
2238.80000000001	62.5396400017866\\
2238.90000000001	62.5368468066101\\
2239.20000000002	62.5418657616219\\
2239.29999999997	62.539072965973\\
2239.70000000002	62.5323689615144\\
2239.90000000001	62.5357142857143\\
2239.99999999997	62.5329226373823\\
2240.79999999998	62.5418358695167\\
2240.99999999997	62.5362545178707\\
2241.40000000002	62.5429399955387\\
2241.49999999999	62.5401498929336\\
2241.69999999997	62.5434918369168\\
2241.90000000004	62.5379125780553\\
2242.49999999998	62.5300989922412\\
2242.89999999999	62.5323227819884\\
2243.10000000003	62.5267475035663\\
2243.39999999998	62.5228437708937\\
2243.80000000003	62.5295244886136\\
2243.89999999999	62.5267379679144\\
2245.10000000003	62.5467664350615\\
2245.19999999998	62.5439807598094\\
2245.60000000002	62.5372934942334\\
2246.00000000003	62.5395129335292\\
2246.10000000002	62.5367286973555\\
2246.39999999999	62.5328288448698\\
2246.70000000003	62.5378315826954\\
2246.90000000001	62.5322652425456\\
2247.19999999999	62.5372669425533\\
2247.29999999999	62.5344842929608\\
2247.69999999998	62.5322537592313\\
2247.9	62.5355871886121\\
2248.00000000003	62.5328054801833\\
2248.40000000002	62.5261285301312\\
2248.59999999999	62.5294614666252\\
2248.70000000001	62.5266808964781\\
2248.90000000003	62.5300133392619\\
2249.19999999999	62.5216734095052\\
2249.39999999997	62.5250055567904\\
2249.50000000002	62.522226173542\\
2250.29999999998	62.5355492356914\\
2250.40000000001	62.5327704954455\\
2250.79999999998	62.5305433382203\\
2250.99999999998	62.5338723290836\\
2251.30000000003	62.5255396642089\\
2252.19999999999	62.5405141411002\\
2252.39999999997	62.534961154273\\
2253.40000000003	62.5515864211227\\
2253.5	62.5488107916223\\
2253.80000000003	62.5537956431075\\
2254	62.5482454194579\\
2254.30000000004	62.5532292405962\\
2254.39999999999	62.550454646263\\
2254.90000000004	62.5587583148559\\
2254.99999999999	62.5559842135604\\
2255.2	62.5593047488139\\
2255.3	62.5565309922852\\
2255.80000000003	62.5603971807261\\
2255.99999999997	62.5548512920527\\
2256.79999999998	62.5681244184501\\
2256.90000000002	62.5653522374834\\
2257.10000000002	62.5686691476165\\
2257.30000000002	62.5631257198547\\
2257.49999999998	62.5664422395464\\
2257.60000000002	62.5636709926031\\
2257.89999999999	62.5686448184234\\
2258.00000000002	62.5658739648377\\
2258.29999999999	62.5619907899398\\
2258.59999999997	62.5581086465666\\
2258.99999999999	62.5647381700677\\
2259.09999999998	62.5619688385269\\
2259.40000000002	62.5669395884045\\
2259.49999999997	62.5641706496725\\
2259.90000000004	62.5619469026549\\
2260.5	62.5674599663806\\
2260.6	62.5646923519264\\
2261.40000000002	62.5779349988945\\
2261.50000000002	62.5751680226388\\
2261.70000000002	62.5784773189495\\
2261.8	62.5757106857067\\
2262.00000000001	62.5790194951594\\
2262.39999999998	62.567955801105\\
2262.59999999997	62.5712644186149\\
2262.70000000003	62.5684992045254\\
2263.09999999998	62.5618593142453\\
2263.59999999999	62.5657110041083\\
2263.70000000001	62.5629472568248\\
2263.89999999998	62.5662544169611\\
2264.1	62.5607278508966\\
2264.69999999997	62.5706464146945\\
2264.90000000002	62.5651214128035\\
2265.70000000003	62.5783387765911\\
2265.79999999998	62.5755770334084\\
2266.10000000002	62.5717059394581\\
2266.30000000002	62.5750088245676\\
2266.40000000004	62.5722479594088\\
2266.69999999998	62.5772013410976\\
2267.10000000003	62.5661609033169\\
2267.39999999999	62.5711135611907\\
2267.70000000001	62.5628362289443\\
2267.99999999997	62.5677880164014\\
2268.09999999998	62.5650295388414\\
2268.69999999997	62.5705218617772\\
2268.79999999999	62.5677641147693\\
2268.99999999999	62.5710634172139\\
2269.29999999998	62.5627919273817\\
2270.40000000003	62.5809293107245\\
2270.50000000001	62.5781731700872\\
2270.89999999996	62.5759577278732\\
2271.09999999999	62.5792532581895\\
2271.20000000002	62.5764980407696\\
2271.60000000001	62.5830875555751\\
2271.89999999996	62.574823943662\\
2272.29999999997	62.5814117232882\\
2272.50000000001	62.575904250638\\
2273.2	62.587427968152\\
2273.29999999997	62.5846749362189\\
2273.60000000003	62.5896116462154\\
2273.7	62.5868590025508\\
2274.00000000001	62.5917945560881\\
2274.09999999997	62.5890423005892\\
2274.70000000001	62.5945138034113\\
2275.10000000001	62.5835091420535\\
2275.69999999998	62.5933737586783\\
2275.79999999998	62.5906234896085\\
2276.00000000001	62.5939106366153\\
2276.10000000003	62.5911607064406\\
2276.3	62.5944473730452\\
2276.49999999998	62.5889484318721\\
2276.70000000001	62.59223471539\\
2276.80000000003	62.589485704247\\
2277.20000000003	62.5872744038994\\
2277.49999999998	62.583421145065\\
2277.99999999999	62.5740748869672\\
2278.29999999999	62.5790028089888\\
2278.40000000003	62.5762563089752\\
2278.89999999998	62.5800789820097\\
2279.09999999999	62.5745875745876\\
2279.29999999998	62.5778713696587\\
2279.39999999998	62.57512612415\\
2279.60000000004	62.5784094398386\\
2279.99999999998	62.5674312530152\\
2280.30000000003	62.563585335906\\
2280.69999999999	62.5657663977552\\
2280.99999999997	62.5575380298979\\
2281.40000000001	62.5641025641026\\
2281.49999999997	62.5613604488079\\
2281.69999999998	62.5646419493382\\
2281.80000000001	62.5619001709102\\
2282.30000000003	62.5569575885033\\
2282.49999999999	62.5602383247174\\
2282.59999999999	62.557497700092\\
2282.89999999997	62.5536574682435\\
2283.30000000004	62.5602172199352\\
2283.4	62.5574775563828\\
2283.99999999997	62.5673131649227\\
2284.29999999999	62.5590964804763\\
2284.90000000003	62.5645514223195\\
2284.99999999997	62.5618134873747\\
2285.20000000001	62.5650899225485\\
2285.30000000001	62.5623523234445\\
2285.70000000004	62.5645288301689\\
2285.90000000002	62.5590551181102\\
2286.09999999998	62.5623305047677\\
2286.40000000001	62.5541220205554\\
2286.69999999997	62.5590344586322\\
2287.00000000001	62.5508285601854\\
2287.50000000002	62.5546424200035\\
2287.70000000003	62.5491738788356\\
2288.10000000001	62.555720653789\\
2288.19999999997	62.5529869335314\\
2288.50000000004	62.5578956567334\\
2288.6	62.5551623192205\\
2288.80000000002	62.5584341823583\\
2288.89999999999	62.5557011795544\\
2289.2	62.5606080461276\\
2289.3	62.5578754258758\\
2289.80000000002	62.5616839163282\\
2290.10000000002	62.5534887782726\\
2290.59999999998	62.5616623739468\\
2290.79999999997	62.5562006198437\\
2291.09999999997	62.5611033519553\\
2291.49999999996	62.550183278059\\
2291.79999999998	62.5463589161831\\
2292.29999999997	62.5545280055837\\
2292.49999999999	62.5490709238419\\
2293.30000000001	62.5577744832999\\
2293.40000000002	62.5550468715936\\
2293.8	62.5615763546798\\
2294.00000000003	62.5561222265812\\
2294.20000000003	62.5593863051911\\
2294.30000000002	62.556659693166\\
2294.99999999997	62.56807982223\\
2295.20000000003	62.5626279789134\\
2295.6	62.5691510214749\\
2295.69999999997	62.5664256468334\\
2296.10000000002	62.5729466074384\\
2296.39999999998	62.5647724798607\\
2296.60000000002	62.5680323943049\\
2297.09999999999	62.5544140693018\\
2297.40000000001	62.550598476605\\
2297.79999999998	62.5484137690935\\
2298.20000000002	62.5549319061915\\
2298.30000000001	62.5522102332057\\
2298.8	62.5560050458915\\
2298.9	62.5532840365376\\
2299.09999999997	62.5565414057063\\
2299.30000000002	62.5511002870314\\
2299.80000000003	62.5592417061611\\
2299.90000000002	62.5565217391304\\
2300.19999999999	62.561405034126\\
2300.3	62.5586854460094\\
2300.59999999996	62.5548746033816\\
2300.8	62.5581294276153\\
2300.90000000002	62.5554106910039\\
2301.7	62.5640802849943\\
2301.80000000002	62.561362352839\\
2302.2	62.567866915693\\
2302.29999999997	62.565149409312\\
2302.5	62.56840093807\\
2302.59999999999	62.5656837625396\\
2302.9	62.5705601389492\\
2303.10000000003	62.565126780132\\
2303.39999999996	62.5700021706099\\
2303.59999999999	62.5645700395017\\
2303.79999999998	62.5678197838448\\
2303.89999999998	62.5651041666667\\
2304.09999999999	62.5683534415415\\
2304.2	62.5656381547542\\
2304.40000000004	62.5688869602951\\
2304.49999999999	62.5661720038185\\
2304.69999999999	62.5694203401597\\
2304.80000000003	62.5667057139138\\
2305.40000000004	62.576447625244\\
2305.70000000003	62.568306010929\\
2306.3	62.5780437044745\\
2306.39999999999	62.5753305874702\\
2306.70000000002	62.5715276573608\\
2306.99999999999	62.5763946079494\\
2307.09999999997	62.5736823855756\\
2307.50000000003	62.5715028601144\\
2307.79999999998	62.5763681268686\\
2308	62.5709457995754\\
2308.79999999998	62.5839144181212\\
2308.99999999998	62.5784937854575\\
2309.19999999999	62.5817347248084\\
2309.30000000001	62.5790248549407\\
2309.69999999997	62.5768464802147\\
2311.00000000003	62.593570161395\\
2311.39999999999	62.5827384815055\\
2312.5	62.6005361930295\\
2312.70000000004	62.5951227948807\\
2313.09999999998	62.6015908697908\\
2313.20000000003	62.5988847101543\\
2313.70000000003	62.6069668942865\\
2313.80000000002	62.6042612040278\\
2314.00000000001	62.6074931938983\\
2314.29999999999	62.5993778085033\\
2314.70000000001	62.6058406773803\\
2314.79999999996	62.6031362045877\\
2314.99999999997	62.6063668955985\\
2315.30000000002	62.5982551610953\\
2315.50000000002	62.6014855760926\\
2315.60000000001	62.5987822256769\\
2315.99999999999	62.6009239670135\\
2316.09999999998	62.5982212244193\\
2316.60000000001	62.6019769499719\\
2316.69999999998	62.5992748618785\\
2316.99999999997	62.6041172154849\\
2317.19999999997	62.5987140206275\\
2317.40000000001	62.6019417475728\\
2317.59999999997	62.5965396729516\\
2318	62.5943660756654\\
2318.59999999997	62.5997326087894\\
2318.80000000002	62.5943335202035\\
2319.00000000003	62.5975593980423\\
2319.09999999999	62.594860296654\\
2319.50000000001	62.5926883945508\\
2319.79999999997	62.5889046941679\\
2320.20000000002	62.5910442615179\\
2320.30000000001	62.5883468367523\\
2320.90000000004	62.5980180956484\\
2321.00000000001	62.5953211839214\\
2321.49999999999	62.5861474844935\\
2321.8	62.5909815237521\\
2322.00000000002	62.5855906291719\\
2322.29999999999	62.5904236996211\\
2322.49999999997	62.5850340136055\\
2322.79999999998	62.5812561883852\\
2323.1	62.577479338843\\
2323.60000000003	62.5812282136248\\
2323.69999999999	62.578535157931\\
2323.90000000001	62.5817555938038\\
2324.00000000001	62.5790628630438\\
2324.40000000002	62.58120025812\\
2324.50000000003	62.578508130431\\
2324.70000000001	62.5817274604267\\
2324.8	62.5790356574476\\
2325.29999999997	62.5741807861013\\
2325.89999999998	62.5838349097163\\
2325.99999999997	62.5811444047977\\
2326.39999999999	62.5875779067268\\
2326.49999999999	62.5848878191352\\
2327.19999999998	62.5918446268208\\
2327.29999999997	62.5891552805706\\
2327.49999999998	62.5923698229936\\
2327.60000000002	62.5896808007905\\
2327.80000000001	62.5928948838009\\
2327.89999999999	62.590206185567\\
2328.10000000002	62.5934198092947\\
2328.20000000003	62.5907314349525\\
2328.50000000001	62.5869621231641\\
2329.00000000002	62.5949937744193\\
2329.10000000003	62.5923063712863\\
2329.30000000002	62.5955181591826\\
2329.40000000004	62.5928310796308\\
2329.80000000003	62.5863770977295\\
2330.30000000001	62.5772399588054\\
2331.1	62.5643445435827\\
2331.29999999998	62.5675559749507\\
2331.39999999996	62.5648723997427\\
2331.70000000003	62.5696886525431\\
2332.00000000002	62.561639723854\\
2332.59999999996	62.5669824666695\\
2332.80000000001	62.5616185863089\\
2333.19999999998	62.5594651352162\\
2333.7	62.5503470734425\\
2334.20000000003	62.5583686758343\\
2334.30000000002	62.5556888279644\\
2334.49999999999	62.5588965989891\\
2334.59999999999	62.5562170728573\\
2334.90000000004	62.5610278372591\\
2335.00000000001	62.5583486788574\\
2335.60000000001	62.5679667765552\\
2335.70000000002	62.5652881239832\\
2336.19999999999	62.5690193896332\\
2336.69999999999	62.5556316330024\\
2337.09999999997	62.5620400479206\\
2337.20000000001	62.5593633679887\\
2337.80000000002	62.5475854399247\\
2338.19999999997	62.5497156053543\\
2338.40000000002	62.5443660466111\\
2338.89999999999	62.5523728088927\\
2339.09999999998	62.547024623803\\
2339.5	62.5534279363994\\
2339.59999999997	62.5507543702184\\
2340.19999999997	62.5603555099774\\
2340.39999999999	62.5550096133305\\
2340.60000000001	62.558209082753\\
2340.70000000002	62.5555365686945\\
2341.10000000002	62.5533914232018\\
2341.29999999999	62.5565900743145\\
2341.39999999997	62.5539184283579\\
2341.99999999999	62.5635113786772\\
2342.09999999999	62.5608402356759\\
2342.40000000002	62.5656350053362\\
2342.5	62.5629642277811\\
2343.29999999997	62.5757446445336\\
2343.6	62.5677347783419\\
2344.10000000002	62.5714529477007\\
2344.29999999999	62.5661149974407\\
2344.69999999999	62.5725008529512\\
2344.80000000003	62.5698324022346\\
2345.49999999996	62.581002728513\\
2345.60000000001	62.5783348254252\\
2345.89999999998	62.5831202046036\\
2345.99999999997	62.5804526661268\\
2346.29999999997	62.5767132628708\\
2346.50000000002	62.5799028381488\\
2346.60000000001	62.5772361188051\\
2346.99999999998	62.5750926675472\\
2347.40000000003	62.5729499467519\\
2347.70000000002	62.5692137320044\\
2348.00000000002	62.5654784719561\\
2348.19999999998	62.568666695056\\
2348.40000000002	62.5633383010432\\
2349.00000000003	62.5729002596739\\
2349.10000000003	62.5702366763153\\
2349.49999999997	62.576608784474\\
2349.59999999998	62.5739456100779\\
2349.79999999996	62.5771309417422\\
2349.89999999999	62.5744680851064\\
2350.09999999999	62.577652965705\\
2350.2	62.574990426754\\
2350.40000000003	62.5781748564135\\
2350.50000000001	62.5755126350719\\
2350.70000000001	62.5786966139187\\
2350.89999999999	62.573373032752\\
2351.09999999997	62.5765566519224\\
2351.20000000001	62.5738952919661\\
2351.49999999999	62.5786698418098\\
2351.59999999998	62.5760088446656\\
2352	62.5823731984184\\
2352.09999999998	62.579712609472\\
2352.60000000001	62.583414800017\\
2352.89999999998	62.5754356141097\\
2353.2	62.5802065185059\\
2353.30000000001	62.5775473782612\\
2353.8	62.5854964102128\\
2354.00000000003	62.5801792617136\\
2354.40000000003	62.5780420471438\\
2354.70000000001	62.5743162901308\\
2355.29999999997	62.5626220599474\\
2355.50000000003	62.5658006452708\\
2355.70000000003	62.5604890058579\\
2356.20000000003	62.5684335610915\\
2356.30000000001	62.5657783058903\\
2356.7	62.5678886625934\\
2356.79999999996	62.5652339938054\\
2357.49999999997	62.576348829318\\
2357.69999999998	62.5710408007465\\
2357.89999999999	62.5742154368109\\
2358.20000000001	62.5662553534326\\
2358.59999999997	62.5726035528045\\
2358.79999999999	62.5672983170122\\
2359.09999999998	62.5720583248559\\
2359.2	62.5694061797991\\
2359.60000000001	62.5757511548078\\
2359.70000000001	62.5730994152047\\
2360.09999999999	62.5709685619863\\
2360.30000000002	62.5741399762752\\
2360.39999999999	62.5714890912942\\
2360.89999999998	62.566709021601\\
2361.5	62.5550474254743\\
2361.80000000003	62.559803547991\\
2361.9	62.5571549534293\\
2362.20000000001	62.5534436777717\\
2362.4	62.5566137566138\\
2362.49999999996	62.5539659696944\\
2362.69999999999	62.5571356018283\\
2362.80000000002	62.554488128994\\
2363.10000000001	62.550778605281\\
2363.29999999999	62.5539477024626\\
2363.4	62.5513010365983\\
2363.69999999999	62.5560538116592\\
2363.79999999997	62.5534075045476\\
2364.40000000003	62.5586804821315\\
2364.79999999999	62.5480992853821\\
2365.30000000002	62.5517882810518\\
2365.60000000003	62.5438559411591\\
2366	62.5501880732006\\
2366.19999999997	62.5449013227401\\
2366.4	62.5480667652652\\
2366.50000000003	62.5454238147553\\
2366.90000000003	62.5517532741867\\
2366.99999999999	62.5491107262051\\
2367.39999999997	62.5427666314678\\
2367.9	62.5464527027027\\
2368	62.543811494447\\
2368.80000000001	62.5564608045929\\
2368.90000000002	62.55382017729\\
2369.20000000001	62.5585616004727\\
2369.59999999997	62.5480018567751\\
2369.90000000004	62.5527426160338\\
2370.2	62.5448255495085\\
2370.50000000001	62.5495655108411\\
2370.59999999997	62.5469270679546\\
2370.99999999999	62.5532453291721\\
2371.09999999997	62.5506072874494\\
2371.3	62.5537657080206\\
2371.40000000003	62.551127978073\\
2371.70000000002	62.5558647440762\\
2371.79999999997	62.5532273704625\\
2372.39999999999	62.5584826132771\\
2372.79999999999	62.5479371233512\\
2373.09999999998	62.5526714983988\\
2373.19999999999	62.5500358151098\\
2374.20000000003	62.5658088699827\\
2374.50000000004	62.5579044891771\\
2374.99999999998	62.5657867037177\\
2375.30000000003	62.5578849877915\\
2375.59999999996	62.5541945531843\\
2376.09999999998	62.5620738995034\\
2376.19999999998	62.5594411480032\\
2376.49999999998	62.5557519144997\\
2376.70000000002	62.5589027263548\\
2376.8	62.5562707728554\\
2377.2	62.5541580784924\\
2377.6	62.5520460949657\\
2378.39999999998	62.5646415808283\\
2378.7	62.5567513031781\\
2379.20000000001	62.560416929349\\
2379.50000000004	62.5525298369474\\
2380.00000000001	62.5477921095752\\
2380.19999999997	62.5509389572743\\
2380.40000000002	62.5456836798992\\
2380.59999999997	62.5488301759987\\
2380.70000000002	62.5462029569892\\
2380.99999999997	62.5509218428457\\
2381.09999999996	62.5482949773224\\
2381.40000000001	62.5530128070544\\
2381.49999999998	62.5503862949278\\
2381.69999999999	62.5535309429843\\
2381.80000000001	62.5509047399135\\
2382.39999999997	62.5561385099685\\
2382.69999999997	62.5482625482625\\
2382.90000000001	62.5514057910197\\
2383.20000000001	62.5435320773717\\
2383.70000000002	62.551388539307\\
2384.00000000004	62.5435174699048\\
2384.29999999996	62.5398423083375\\
2384.70000000002	62.5377390137538\\
2385.09999999999	62.5440214657052\\
2385.19999999998	62.541399404687\\
2385.50000000001	62.5461099932931\\
2385.60000000002	62.543488284361\\
2385.8	62.5466281067941\\
2385.89999999997	62.5440067057837\\
2386.70000000003	62.5523713759008\\
2386.80000000003	62.5497507226947\\
2387.4	62.5591623036649\\
2387.60000000002	62.553922184529\\
2388.09999999997	62.5575747424839\\
2388.30000000003	62.5523362920784\\
2388.90000000004	62.5617413143575\\
2388.99999999998	62.5591226821816\\
2389.50000000003	62.5502176096418\\
2389.70000000001	62.5533517449159\\
2389.9	62.5481171548117\\
2390.20000000003	62.5444504873865\\
2390.39999999997	62.5475841874085\\
2390.8	62.5371199130035\\
2391.69999999998	62.5470357053265\\
2391.99999999998	62.5391915053718\\
2392.30000000001	62.5355291757231\\
2393.00000000003	62.5464878191467\\
2393.10000000001	62.5438743105465\\
2393.50000000003	62.5501336898396\\
2393.6	62.5475205748423\\
2394.00000000002	62.5412472327806\\
2394.30000000004	62.5459405278984\\
2394.40000000001	62.5433284610566\\
2394.9	62.5511482254697\\
2394.99999999999	62.5485365955493\\
2395.49999999997	62.5563533144098\\
2395.59999999997	62.5537421213007\\
2395.90000000002	62.5584307178631\\
2396.00000000003	62.5558198739619\\
2396.79999999997	62.5683174099879\\
2396.90000000003	62.5657071339174\\
2398.20000000001	62.5818287953967\\
2398.29999999999	62.5792194796531\\
2398.69999999999	62.5812906453227\\
2399.10000000001	62.5708569523174\\
2399.50000000004	62.5729288214702\\
2399.80000000002	62.5651068794533\\
2400	62.5682263239032\\
2400.10000000003	62.5656195317057\\
2400.39999999998	62.5619662570298\\
2400.89999999997	62.5655976676385\\
2401	62.5629919620174\\
2401.29999999997	62.5593403847756\\
2401.6	62.5556897197818\\
2402.30000000002	62.5666000666001\\
2402.39999999997	62.5639958376691\\
2403.09999999997	62.5707390146471\\
2403.29999999998	62.5655321627694\\
2403.5	62.5686470294558\\
2403.59999999996	62.5660440154761\\
2404.00000000001	62.5722723680379\\
2404.09999999999	62.5696697446136\\
2404.30000000002	62.5727832307436\\
2404.79999999997	62.5597737951682\\
2404.99999999998	62.5628871980375\\
2405.10000000002	62.5602860468984\\
2405.50000000001	62.5665114732291\\
2405.70000000003	62.5613101670962\\
2406.19999999997	62.5566222000582\\
2406.79999999999	62.5493373218663\\
2407.2	62.547252108171\\
2407.39999999998	62.5503634475597\\
2407.60000000003	62.545167587324\\
2408.29999999997	62.5519016774622\\
2408.39999999998	62.5493045463982\\
2408.69999999999	62.5539687811358\\
2408.80000000003	62.5513719955166\\
2409.09999999999	62.5477336875311\\
2409.49999999996	62.5498007968127\\
2409.59999999999	62.5472050462713\\
2410.20000000003	62.5565282330001\\
2410.29999999997	62.5539329571855\\
2410.70000000001	62.5601460096234\\
2410.79999999997	62.5575511219876\\
2411.00000000001	62.5606569615528\\
2411.10000000002	62.5580623755806\\
2411.29999999998	62.5611677863482\\
2411.40000000002	62.5585735019697\\
2411.60000000001	62.5616784840569\\
2411.69999999999	62.5590845012024\\
2411.89999999998	62.5621890547264\\
2412.00000000002	62.5595953733261\\
2412.59999999997	62.568906204667\\
2412.70000000004	62.5663129973475\\
2413.00000000003	62.5626787120302\\
2413.39999999999	62.5605966438782\\
2413.60000000003	62.5636988855284\\
2414.00000000003	62.5533325048672\\
2414.59999999999	62.5626371806022\\
2414.70000000001	62.5600463806527\\
2414.89999999998	62.5631469979296\\
2414.99999999996	62.5605564986957\\
2415.40000000003	62.5667563651418\\
2415.79999999999	62.5563972018709\\
2416.20000000003	62.558457145222\\
2416.29999999997	62.5558682337361\\
2416.69999999997	62.5537901357167\\
2417.2	62.5573987506722\\
2417.4	62.5522233712513\\
2417.9	62.5475599669148\\
2418.19999999998	62.5439358226854\\
2418.49999999997	62.5485818241958\\
2418.60000000002	62.5459957828586\\
2418.8	62.5490925627351\\
2418.90000000003	62.5465068210004\\
2419.59999999999	62.5573418192338\\
2419.70000000003	62.5547565914538\\
2419.89999999997	62.5578512396694\\
2419.99999999996	62.5552663113095\\
2420.30000000003	62.5516443563047\\
2421.29999999997	62.5671099364004\\
2421.5	62.5619425173439\\
2421.69999999999	62.5650342720291\\
2421.89999999999	62.559867877787\\
2422.30000000004	62.5660501981506\\
2422.8	62.5531388006108\\
2423.19999999998	62.5551933314076\\
2423.29999999999	62.5526120326814\\
2423.80000000001	62.5603366475515\\
2424.2	62.5500144371571\\
2424.40000000003	62.5531037327284\\
2424.50000000001	62.5505237977398\\
2425.10000000001	62.5556655121227\\
2425.19999999998	62.5530862161382\\
2425.60000000001	62.5592612441769\\
2425.69999999999	62.5566823316019\\
2426.29999999999	62.5659413122321\\
2426.39999999997	62.5633628683289\\
2426.70000000002	62.5679907697379\\
2426.79999999998	62.5654126663645\\
2426.99999999998	62.568497383709\\
2427.09999999999	62.5659195781147\\
2427.50000000001	62.5679683638161\\
2427.59999999998	62.565391110928\\
2427.79999999998	62.5684748136249\\
2427.90000000004	62.5658978583196\\
2428.10000000001	62.5689811382917\\
2428.20000000004	62.5664044805008\\
2428.50000000004	62.5710285761344\\
2428.7	62.5658761528327\\
2429	62.5704993619036\\
2429.39999999996	62.5601975715168\\
2429.60000000002	62.563279417212\\
2429.90000000003	62.5555555555556\\
2430.29999999997	62.5617182356814\\
2430.40000000003	62.5591442090105\\
2430.60000000003	62.5622248734932\\
2430.99999999997	62.5519312245486\\
2431.40000000003	62.5457536500103\\
2431.70000000004	62.5503742084053\\
2431.80000000003	62.5478021300218\\
2432.30000000004	62.5513895740832\\
2432.50000000003	62.5462468141084\\
2432.99999999997	62.5539435288315\\
2433.10000000002	62.551372677955\\
2433.49999999998	62.5575279421433\\
2433.70000000002	62.5523872134111\\
2434.10000000004	62.550324541944\\
2435.20000000003	62.5672401757484\\
2435.30000000002	62.5646711012565\\
2435.50000000002	62.5677451141403\\
2435.60000000001	62.5651763353451\\
2436.10000000003	62.5605451112388\\
2436.50000000003	62.5584831322334\\
2436.70000000001	62.561556139199\\
2436.80000000001	62.5589888793139\\
2437.19999999997	62.5651335494194\\
2437.29999999998	62.5625666694018\\
2437.49999999998	62.5656383327863\\
2437.70000000004	62.5605053736976\\
2439.00000000003	62.5804600057398\\
2439.29999999996	62.5727637943757\\
2439.59999999999	62.5773660695987\\
2439.70000000003	62.5748012132142\\
2440.00000000002	62.5794024835048\\
2440.10000000002	62.5768379641013\\
2440.39999999997	62.5814382298709\\
2440.60000000004	62.5763100749785\\
2440.99999999996	62.5824423415673\\
2441.09999999999	62.5798787481566\\
2441.39999999998	62.5844767560926\\
2441.50000000001	62.5819134993447\\
2442.00000000002	62.5895745464969\\
2442.20000000003	62.584449084879\\
2442.69999999997	62.5921074177174\\
2442.80000000001	62.5895452126571\\
2443.19999999997	62.5956697908566\\
2443.29999999999	62.593107964312\\
2443.60000000001	62.5895158980235\\
2444.30000000004	62.6002290950745\\
2444.40000000003	62.5976682348128\\
2444.99999999996	62.6027565334751\\
2445.1	62.6001963029609\\
2445.39999999998	62.5966060110407\\
2445.90000000002	62.6042518397384\\
2446.30000000002	62.5940156965337\\
2446.50000000004	62.5970734897409\\
2446.59999999999	62.5945150611027\\
2446.90000000004	62.5991009399264\\
2447.20000000002	62.5914272872145\\
2447.39999999996	62.5944841675179\\
2447.70000000001	62.5868126480922\\
2447.99999999997	62.5913974102365\\
2448.29999999998	62.5837281489953\\
2448.70000000002	62.5816726559948\\
2449.40000000001	62.5923657889365\\
2449.49999999998	62.5898105813194\\
2450.30000000002	62.5775383610839\\
2450.59999999997	62.5739584608479\\
2450.90000000002	62.5703794369645\\
2451.19999999998	62.5668012891119\\
2451.39999999998	62.5698551906996\\
2451.70000000001	62.5621992005873\\
2451.99999999998	62.5667794951266\\
2452.09999999996	62.5642280401272\\
2452.39999999997	62.5688073394495\\
2452.50000000003	62.5662562178912\\
2454.30000000002	62.5896349413299\\
2454.49999999999	62.584535158478\\
2454.80000000001	62.5891074992872\\
2455.00000000004	62.5840087980123\\
2455.19999999998	62.5870565714984\\
2455.29999999998	62.5845076158671\\
2455.7	62.5783858620409\\
2456	62.5829567200033\\
2456.30000000003	62.5753134668621\\
2456.89999999996	62.5844525844526\\
2457.00000000002	62.5819054983517\\
2457.30000000002	62.5864735085863\\
2457.5	62.5813802083333\\
2457.8	62.5859473534318\\
2458	62.5808551320125\\
2458.30000000003	62.5772860397006\\
2458.79999999999	62.5848956850624\\
2459.10000000004	62.577260897853\\
2459.50000000003	62.5792811839324\\
2459.99999999997	62.5665623348644\\
2460.70000000003	62.577210663199\\
2460.79999999999	62.5746678044618\\
2461.00000000001	62.5777091544431\\
2461.09999999998	62.5751665854055\\
2461.6	62.5705813056018\\
2462.00000000001	62.5726006254823\\
2462.19999999998	62.5675181740649\\
2462.60000000003	62.5654769155805\\
2463.09999999998	62.5730756739201\\
2463.20000000003	62.570535460561\\
2463.99999999998	62.5826873909338\\
2464.09999999998	62.5801477152829\\
2464.40000000003	62.5765875431122\\
2464.89999999997	62.5720081135903\\
2465.20000000003	62.5765626901391\\
2465.39999999998	62.5714865138917\\
2465.60000000002	62.5745224479864\\
2465.70000000003	62.5719847513991\\
2466.09999999997	62.5740004865785\\
2466.20000000002	62.5714633256295\\
2466.50000000002	62.5760155679883\\
2466.69999999998	62.5709421112372\\
2467.1	62.5729571984436\\
2467.20000000003	62.5704211080939\\
2467.4	62.5734549138805\\
2467.5	62.5709191116875\\
2467.99999999997	62.5785016814554\\
2468.10000000002	62.5759662912244\\
2469	62.5855574905836\\
2469.10000000004	62.5830228414061\\
2469.30000000004	62.5860532922977\\
2469.4	62.5835189309577\\
2469.69999999996	62.5880638108349\\
2469.79999999999	62.5855297785335\\
2470.00000000004	62.588559167645\\
2470.09999999997	62.5860254230427\\
2470.50000000002	62.5839876952967\\
2471.40000000002	62.5976127857576\\
2471.49999999998	62.5950801100502\\
2471.69999999999	62.5981066429323\\
2471.79999999996	62.595574254622\\
2472.29999999999	62.6031386507038\\
2472.39999999996	62.6006066734075\\
2472.89999999999	62.6081682167408\\
2473.09999999997	62.6031052886948\\
2473.4	62.6076409945422\\
2473.50000000003	62.6051099611902\\
2473.79999999997	62.6015602894216\\
2474.20000000002	62.5995230974417\\
2474.50000000001	62.595975107088\\
2474.90000000004	62.5979797979798\\
2475.19999999999	62.5903930836666\\
2475.60000000004	62.5964373712485\\
2475.8	62.5913809119916\\
2475.99999999996	62.5944024877832\\
2476.1	62.591874646636\\
2476.40000000003	62.5964062184535\\
2476.50000000003	62.5938787046758\\
2476.70000000003	62.5968992248062\\
2476.80000000002	62.5943719972546\\
2477.00000000004	62.5973921117436\\
2477.19999999997	62.5923384329714\\
2477.39999999997	62.5953582240161\\
2477.49999999998	62.5928317726832\\
2477.89999999999	62.5988700564972\\
2477.99999999998	62.5963439732053\\
2478.30000000001	62.5928018076178\\
2478.70000000003	62.5867355171857\\
2479.09999999997	62.5927718618909\\
2479.2	62.5902472472069\\
2479.40000000002	62.5932647711232\\
2479.50000000001	62.5907404420068\\
2479.7	62.5937575610936\\
2479.80000000001	62.5912335174805\\
2480.50000000001	62.5977586067887\\
2480.59999999997	62.5952352158665\\
2480.99999999997	62.5972350973359\\
2481.10000000001	62.5947122360148\\
2481.30000000002	62.5977270895462\\
2481.39999999998	62.5952045133992\\
2481.8	62.5931745839881\\
2482.09999999997	62.5976955926195\\
2482.29999999999	62.5926522719949\\
2482.59999999998	62.597172433238\\
2482.69999999996	62.5946512002578\\
2483.09999999999	62.6006765463918\\
2483.19999999998	62.598155679942\\
2483.59999999996	62.6041792487015\\
2483.79999999996	62.5991384516285\\
2484.20000000002	62.5930845711066\\
2484.40000000001	62.5960957939223\\
2484.49999999999	62.5935764308138\\
2484.80000000004	62.5900438649443\\
2485.20000000001	62.5920412022693\\
2485.40000000004	62.5870046268357\\
2485.69999999998	62.5915198326495\\
2485.99999999999	62.5839668557178\\
2486.39999999996	62.5899859239895\\
2486.50000000001	62.5874688329446\\
2487.19999999996	62.5979978289712\\
2487.29999999997	62.5954812253759\\
2487.69999999998	62.5894364498754\\
2488.09999999998	62.595450526485\\
2488.20000000003	62.5929349354981\\
2488.6	62.5989472415317\\
2488.70000000002	62.5964320154291\\
2489.00000000001	62.6009400988309\\
2489.09999999997	62.5984251968504\\
2489.80000000002	62.608940118077\\
2490.1	62.6013974781142\\
2490.79999999997	62.6078927295355\\
2490.90000000004	62.6053793657166\\
2491.39999999997	62.6128838049368\\
2491.59999999998	62.607858088855\\
2491.99999999999	62.6018217567513\\
2492.19999999998	62.6048228543915\\
2492.29999999999	62.6023110255176\\
2493	62.6088002887971\\
2493.20000000003	62.603778125376\\
2493.4	62.6067776218167\\
2493.50000000003	62.6042669233237\\
2493.90000000002	62.6102646351243\\
2494.00000000004	62.6077543001484\\
2494.19999999996	62.6107525157359\\
2494.30000000003	62.6082424631174\\
2494.79999999999	62.6157361016474\\
2495	62.6107170053305\\
2495.39999999996	62.6167100781407\\
2495.6	62.6116921104299\\
2496.20000000002	62.6166726755598\\
2496.39999999996	62.6116563188464\\
2496.69999999998	62.608138417174\\
2496.90000000004	62.611133360032\\
2496.99999999999	62.6086260061672\\
2497.3	62.613117642348\\
2497.4	62.6106106106106\\
2497.90000000004	62.6020816653323\\
2498.10000000001	62.6050756544712\\
2498.19999999999	62.6025697474282\\
2498.59999999996	62.6005522871893\\
2498.90000000001	62.6050420168067\\
2498.99999999997	62.6025369132888\\
2499.40000000003	62.6045209041808\\
2499.49999999997	62.6020163226116\\
2499.9	62.604\\
2500	62.6014959401624\\
2500.60000000002	62.6104690686608\\
2500.69999999996	62.6079654510557\\
2501.10000000001	62.613945306253\\
2501.19999999996	62.6114420501339\\
2502.10000000004	62.6248900967149\\
2502.39999999997	62.6173826173826\\
2502.7	62.6218635128656\\
2502.79999999999	62.6193615406129\\
2503.1	62.6238414829019\\
2503.2	62.6213398314225\\
2503.9	62.6317891373802\\
2504	62.6292879677329\\
2504.30000000002	62.6337645743492\\
2504.40000000003	62.6312637252945\\
2504.60000000004	62.6342476144848\\
2504.69999999997	62.6317470456723\\
2505.19999999996	62.6392048856424\\
2505.50000000003	62.6317049808429\\
2505.90000000002	62.6376695929768\\
2506.00000000004	62.6351701847492\\
2506.40000000001	62.6411330540595\\
2506.50000000002	62.6386340062236\\
2506.90000000003	62.636617471081\\
2507.30000000001	62.6346015793252\\
2507.49999999999	62.6375817514755\\
2507.70000000002	62.6325863306484\\
2507.90000000001	62.6355661881978\\
2508.20000000001	62.6280747916916\\
2508.59999999997	62.6300474349265\\
2508.8	62.6250548048946\\
2509.20000000002	62.6270274578568\\
2509.40000000002	62.6220362622036\\
2509.90000000003	62.6175298804781\\
2510.30000000003	62.6195028680688\\
2510.59999999999	62.6120205520373\\
2511.00000000004	62.6179761857353\\
2511.09999999997	62.6154826377827\\
2511.29999999997	62.6184598232062\\
2511.39999999999	62.6159665538523\\
2511.80000000001	62.6179386122059\\
2512.00000000003	62.612953305999\\
2512.69999999998	62.6233683540274\\
2512.80000000002	62.6208762784034\\
2513.10000000001	62.6253382142289\\
2513.19999999998	62.6228464568496\\
2513.90000000004	62.6292760540971\\
2513.99999999997	62.626784932978\\
2514.29999999997	62.6312440343621\\
2514.40000000002	62.6287532312587\\
2514.6	62.6317254543285\\
2514.70000000002	62.629234929219\\
2515.00000000002	62.6257405272156\\
2515.40000000004	62.6316835619161\\
2515.59999999999	62.6267042970148\\
2515.99999999997	62.6326457612973\\
2516.10000000003	62.6301565853271\\
2516.50000000004	62.6360963204323\\
2516.60000000001	62.6336075018874\\
2517.30000000002	62.6400254230555\\
2517.70000000002	62.630073874017\\
2517.99999999998	62.6265835352051\\
2518.69999999999	62.6329998411942\\
2518.89999999998	62.6280269948392\\
2519.59999999998	62.6384093344446\\
2519.80000000001	62.6334378348347\\
2520.19999999997	62.6393683291672\\
2520.29999999998	62.636883034439\\
2520.70000000002	62.6388448111711\\
2520.79999999998	62.636360030148\\
2521.20000000004	62.6422877087217\\
2521.59999999998	62.6323511916564\\
2521.99999999996	62.6343126759447\\
2522.10000000003	62.6318293553247\\
2522.6	62.6352717326674\\
2522.80000000001	62.6303063934361\\
2522.99999999997	62.6332685981531\\
2523.09999999996	62.6307863031072\\
2523.30000000003	62.6337481176191\\
2523.39999999997	62.6312660986725\\
2524.3	62.6406274758358\\
2524.39999999998	62.6381461675579\\
2524.69999999998	62.6346641318124\\
2525.09999999998	62.6366228417551\\
2525.20000000003	62.6341424781214\\
2525.60000000003	62.6400601813359\\
2525.80000000003	62.6351003602676\\
2526	62.6380586675112\\
2526.10000000003	62.6355791307102\\
2526.60000000003	62.6310998535639\\
2527.19999999999	62.6360147192656\\
2527.30000000001	62.6335364406109\\
2527.49999999996	62.6364931159994\\
2527.60000000001	62.6340151125529\\
2527.8	62.6369713991851\\
2527.89999999998	62.6344936708861\\
2528.30000000001	62.640404999209\\
2528.69999999998	62.6304966782664\\
2529.10000000004	62.6245453107702\\
2529.40000000002	62.6210713579759\\
2529.70000000002	62.6255039924105\\
2530.00000000002	62.6180783368246\\
2530.29999999997	62.6146063863421\\
2530.60000000002	62.6111352590192\\
2530.79999999998	62.6140898494607\\
2530.90000000003	62.6116159620703\\
2531.20000000001	62.6160470904278\\
2531.30000000002	62.6135735166311\\
2531.59999999999	62.6180037129202\\
2531.70000000001	62.6155304526424\\
2531.99999999999	62.6199597172308\\
2532.1	62.6174867703973\\
2532.40000000001	62.6219151036525\\
2532.59999999999	62.6169700319817\\
2532.89999999997	62.6135017765495\\
2533.20000000003	62.6179291832787\\
2533.29999999999	62.6154574879609\\
2533.99999999999	62.6257843021191\\
2534.40000000003	62.615900572105\\
2534.69999999999	62.6203250749566\\
2534.99999999996	62.612914677922\\
2535.69999999998	62.6232352709204\\
2535.90000000004	62.6182965299685\\
2536.29999999997	62.6123639804447\\
2536.60000000004	62.6167855875744\\
2536.7	62.6143172500788\\
2537.30000000001	62.6231575628596\\
2537.40000000002	62.6206896551724\\
2537.99999999996	62.6295260234033\\
2538.19999999996	62.6245912618682\\
2538.60000000002	62.6304801670146\\
2538.69999999999	62.628013234599\\
2539.00000000004	62.6245520066165\\
2539.40000000001	62.6265012797795\\
2539.59999999998	62.6215694767099\\
2539.99999999997	62.623518759104\\
2540.09999999997	62.6210534603575\\
2540.30000000003	62.6239962210675\\
2540.39999999998	62.6215311946467\\
2540.90000000004	62.6170798898072\\
2541.20000000004	62.6136229488844\\
2541.70000000003	62.6170430403651\\
2541.79999999997	62.6145796451473\\
2542.20000000004	62.6086614482949\\
2542.80000000003	62.6174839749892\\
2543.00000000002	62.6125594746569\\
2543.19999999996	62.6154995478316\\
2543.30000000002	62.6130376661162\\
2543.50000000002	62.61597735493\\
2543.90000000003	62.6061320754717\\
2544.59999999999	62.6124887020081\\
2544.79999999998	62.6075680773311\\
2545.19999999997	62.605586767768\\
2545.69999999996	62.6011469871946\\
2546	62.6055535917678\\
2546.09999999997	62.6030948079491\\
2546.29999999999	62.6060320452403\\
2546.40000000001	62.6035735322992\\
2546.60000000002	62.6065103859897\\
2546.69999999998	62.6040521438668\\
2547.49999999997	62.6118699952897\\
2547.60000000001	62.6094124111944\\
2547.79999999997	62.6123474233683\\
2547.99999999998	62.6074329892861\\
2548.4	62.613301942319\\
2548.5	62.6108451698972\\
2549.10000000002	62.6196453789424\\
2549.20000000004	62.6171890322834\\
2549.40000000002	62.6201215924691\\
2549.50000000002	62.6176655161594\\
2550.09999999999	62.6225394086738\\
2550.19999999999	62.6200839116967\\
2550.69999999999	62.6274110083111\\
2550.80000000002	62.6249558979184\\
2551.30000000003	62.6322803166889\\
2551.39999999996	62.6298255927886\\
2551.59999999999	62.6327546341655\\
2551.80000000003	62.6278459187272\\
2552.20000000002	62.6258668651804\\
2552.7	62.6292698213726\\
2552.89999999997	62.6243634939287\\
2553.29999999999	62.6302185321532\\
2553.40000000004	62.6277658116311\\
2553.70000000001	62.6321560028193\\
2553.80000000002	62.629703590587\\
2554.29999999996	62.6370184779204\\
2554.60000000002	62.6296629741261\\
2554.8	62.6325883596227\\
2554.90000000003	62.6301369863014\\
2555.39999999999	62.6374486401878\\
2555.7	62.6300962516629\\
2555.89999999999	62.633020344288\\
2556.09999999997	62.6281198654252\\
2556.59999999996	62.6315171901279\\
2556.80000000002	62.6266181704408\\
2557.09999999997	62.6310026591585\\
2557.29999999997	62.6261046375225\\
2557.69999999997	62.6319493314567\\
2557.80000000003	62.6295007623441\\
2559.00000000004	62.6431167207221\\
2559.09999999999	62.6406689590497\\
2559.30000000004	62.6435883410174\\
2559.40000000004	62.6411408478218\\
2560.80000000001	62.661564293803\\
2560.89999999997	62.659117532214\\
2561.09999999999	62.6620334218335\\
2561.20000000003	62.6595869285129\\
2561.49999999999	62.656152404747\\
2561.70000000004	62.659067842923\\
2561.80000000002	62.6566220383309\\
2561.99999999998	62.6595370984739\\
2562.20000000002	62.6546462162901\\
2562.39999999996	62.6575609756098\\
2562.49999999997	62.6551158979162\\
2562.90000000001	62.6609442060086\\
2563.10000000002	62.6560549313358\\
2563.49999999997	62.6540801997191\\
2563.90000000003	62.6599063962558\\
2564.00000000002	62.6574626574627\\
2564.99999999998	62.6720205839928\\
2565.10000000002	62.6695774208639\\
2565.49999999997	62.6753975678204\\
2565.59999999998	62.6729547491913\\
2565.80000000003	62.6758642191824\\
2565.89999999997	62.6734216679657\\
2566.49999999997	62.6821475882491\\
2566.59999999997	62.6797054583707\\
2566.90000000001	62.6762758083366\\
2567.09999999997	62.6791835462761\\
2567.20000000003	62.6767421025981\\
2567.5	62.6811029755414\\
2567.7	62.6762208894774\\
2567.9	62.6791277258567\\
2568.00000000001	62.676687044897\\
2568.19999999996	62.6795935054316\\
2568.69999999998	62.6673933354095\\
2568.99999999999	62.6639679265112\\
2569.19999999999	62.6668742459036\\
2569.29999999997	62.6644352767183\\
2569.8	62.67169928791\\
2570.00000000003	62.6668223026341\\
2570.2	62.6697272691904\\
2570.40000000002	62.6648511962653\\
2570.69999999997	62.6692080286292\\
2570.79999999999	62.6667703916916\\
2570.99999999998	62.669674458403\\
2571.20000000001	62.664799906662\\
2571.49999999999	62.6613781303469\\
2571.90000000002	62.6671850699844\\
2572.00000000004	62.6647486489639\\
2572.80000000003	62.6763574176999\\
2572.90000000003	62.6739214924213\\
2573.39999999997	62.6772877404313\\
2573.89999999999	62.6651126651127\\
2574.10000000003	62.668013363375\\
2574.20000000001	62.6655789923474\\
2574.60000000002	62.6713791898085\\
2574.70000000001	62.6689451607892\\
2574.90000000003	62.6718446601942\\
2574.99999999996	62.6694108966642\\
2575.29999999998	62.6659936320572\\
2575.70000000004	62.6717912881435\\
2575.8	62.669358282542\\
2576.40000000003	62.6780516204153\\
2576.49999999997	62.675619032834\\
2577.30000000002	62.6833242802825\\
2577.39999999999	62.6808923375364\\
2577.70000000003	62.6774769183024\\
2578.50000000002	62.689056076941\\
2578.60000000004	62.6866250436266\\
2578.90000000002	62.6909654905002\\
2579	62.6885347601877\\
2579.39999999998	62.6943206047684\\
2579.50000000001	62.6918902155373\\
2579.79999999997	62.6884762975309\\
2580.1	62.6850631733974\\
2580.39999999998	62.6894012788219\\
2580.50000000003	62.6869720220104\\
2580.7	62.6898636081835\\
2580.80000000001	62.6874346158317\\
2580.99999999998	62.6903258300725\\
2581.29999999999	62.6830402107384\\
2581.79999999999	62.6902668577404\\
2581.89999999997	62.6878388845856\\
2582.3	62.6936183395291\\
2582.80000000004	62.6814820550544\\
2583.00000000001	62.6843714916186\\
2583.19999999999	62.6795184453993\\
2583.40000000001	62.6824075866073\\
2583.49999999997	62.6799814212726\\
2584.4	62.6891081447088\\
2584.59999999998	62.6842573606221\\
2584.90000000003	62.688588007737\\
2584.99999999998	62.6861630111021\\
2585.30000000004	62.6904927670767\\
2585.50000000003	62.6856435643564\\
2585.79999999998	62.6822382922773\\
2586.3	62.6894525208784\\
2586.50000000003	62.6846052733318\\
2586.89999999996	62.6826439891767\\
2587.60000000001	62.68887428991\\
2587.99999999996	62.6791855028786\\
2588.4	62.6849526752946\\
2588.5	62.6825310978907\\
2588.79999999998	62.6791301324887\\
2589.20000000003	62.6848955316109\\
2589.30000000002	62.6824747045648\\
2589.90000000002	62.6911196911197\\
2590.00000000004	62.6886992780202\\
2590.40000000004	62.6867400115808\\
2590.80000000003	62.6809216874445\\
2591	62.6838022461503\\
2591.09999999998	62.6813831429454\\
2591.50000000001	62.6871430776354\\
2591.60000000004	62.6847243122275\\
2591.89999999998	62.6813271604938\\
2592.59999999998	62.6914027847418\\
2592.90000000004	62.684149633629\\
2593.40000000003	62.6913437439753\\
2593.5	62.6889265885256\\
2593.99999999999	62.6806985081531\\
2594.49999999998	62.6724735990133\\
2594.79999999998	62.6690816601796\\
2595.19999999996	62.6748352791585\\
2595.29999999996	62.6724204361563\\
2595.60000000001	62.67673459953\\
2595.70000000004	62.6743200554742\\
2595.99999999996	62.6709294711298\\
2596.5	62.6627127782485\\
2597	62.6583496977398\\
2597.19999999999	62.6612251183922\\
2597.30000000002	62.6588126588127\\
2597.50000000001	62.6616877117339\\
2597.70000000002	62.6568634998845\\
2597.89999999997	62.6597382602002\\
2598.20000000004	62.65250356002\\
2598.6	62.6505560472544\\
2598.99999999997	62.6563041052672\\
2599.10000000003	62.6538935056941\\
2599.60000000001	62.6456898872947\\
2599.99999999998	62.6475904772893\\
2600.20000000003	62.6427719878476\\
2600.69999999997	62.638418948016\\
2601.09999999998	62.644164231893\\
2601.20000000001	62.6417560450544\\
2601.80000000001	62.6503708828164\\
2601.90000000003	62.6479631053036\\
2602.30000000004	62.6498616661543\\
2602.4	62.6474543707973\\
2602.80000000003	62.6493526451266\\
2603.00000000004	62.6445392032577\\
2603.30000000004	62.6488438196205\\
2603.70000000001	62.63921960212\\
2603.99999999999	62.635843477593\\
2604.20000000001	62.6387128978996\\
2604.40000000003	62.6339028604339\\
2604.7	62.6305282555282\\
2604.99999999997	62.6271544278531\\
2605.20000000003	62.6300234138103\\
2605.39999999999	62.6252158894646\\
2605.7	62.6295187658301\\
2605.80000000003	62.6271153919951\\
2606.00000000001	62.6299835002494\\
2606.09999999997	62.6275803852352\\
2606.89999999998	62.6352128883775\\
2606.99999999996	62.6328104023628\\
2607.29999999997	62.6371097645164\\
2607.59999999996	62.6299037465966\\
2607.79999999997	62.6327696614134\\
2607.90000000002	62.6303680981595\\
2608.30000000004	62.6284312222052\\
2608.59999999999	62.6327289454518\\
2608.90000000001	62.6255270218474\\
2609.60000000004	62.6355519791547\\
2609.80000000002	62.6307521360972\\
2610.39999999997	62.6393411223903\\
2610.59999999999	62.6345424598767\\
2610.80000000003	62.6374047263396\\
2610.99999999999	62.6326069472636\\
2611.40000000004	62.6383304614207\\
2611.49999999997	62.6359319957114\\
2612.5	62.6502334838858\\
2612.59999999999	62.6478355723964\\
2613.20000000004	62.6564114338193\\
2613.29999999998	62.6540139282161\\
2613.59999999996	62.6583004935532\\
2613.70000000003	62.6559032825771\\
2614.10000000004	62.6616173207865\\
2614.19999999997	62.6592204414184\\
2614.80000000001	62.6677884431527\\
2614.90000000004	62.6653919694073\\
2615.10000000001	62.6682471703885\\
2615.30000000004	62.6634549208534\\
2615.80000000002	62.6705913834627\\
2616.10000000001	62.6634049384604\\
2616.30000000004	62.6662589818071\\
2616.40000000003	62.6638639403784\\
2616.69999999999	62.6605013757261\\
2617.09999999996	62.6662081613939\\
2617.19999999999	62.6638138539717\\
2617.6	62.6618787485197\\
2617.80000000002	62.6647312731579\\
2617.89999999998	62.6623376623377\\
2618.90000000003	62.6765941198931\\
2618.99999999998	62.6742010614333\\
2619.40000000002	62.6760832219889\\
2619.49999999999	62.6736906397923\\
2619.80000000004	62.6779648078171\\
2620.00000000003	62.6731804129613\\
2620.29999999997	62.6774538238437\\
2620.50000000001	62.6726703808288\\
2621.4	62.6816707991608\\
2621.50000000001	62.679279829112\\
2621.9	62.6849733028223\\
2622.10000000003	62.6801922050187\\
2622.39999999998	62.6844613918017\\
2622.50000000004	62.6820712270266\\
2622.69999999999	62.6849168827208\\
2623.10000000002	62.6753583409576\\
2623.90000000003	62.6638719512195\\
2624.09999999999	62.6667174758021\\
2624.3	62.6619417771681\\
2624.90000000002	62.6666666666667\\
2625.1	62.6618924272436\\
2625.30000000002	62.6647368020111\\
2625.49999999997	62.6599634369287\\
2625.79999999999	62.656612970791\\
2625.99999999999	62.659456989452\\
2626.30000000003	62.6522997258605\\
2626.79999999998	62.6594084281853\\
2626.89999999997	62.6570232204035\\
2627.20000000004	62.6612872530735\\
2627.30000000004	62.658902336911\\
2627.49999999999	62.6617445577714\\
2627.59999999996	62.6593598964874\\
2628.09999999999	62.6626588539685\\
2628.2	62.660274702279\\
2628.39999999996	62.6631158455393\\
2628.50000000003	62.6607319485658\\
2628.70000000001	62.6635727328058\\
2629.10000000001	62.6540392514833\\
2629.29999999996	62.6568798965544\\
2629.39999999998	62.6544970526716\\
2629.80000000003	62.6487699152059\\
2630.20000000002	62.6544500627305\\
2630.70000000003	62.642542192489\\
2631.09999999996	62.6482213438735\\
2631.20000000001	62.6458404590887\\
2631.49999999998	62.6424988600091\\
2632	62.6495953801147\\
2632.09999999998	62.6472152571993\\
2632.40000000001	62.6438746438746\\
2632.70000000003	62.6481312670921\\
2632.79999999996	62.64575183258\\
2633.7	62.6547194168122\\
2633.89999999996	62.6499620349279\\
2634.20000000001	62.6466233914133\\
2634.59999999999	62.6447033817892\\
2634.90000000001	62.6413662239089\\
2635.10000000002	62.6442015786278\\
2635.29999999999	62.6394475221978\\
2636.89999999999	62.6621160409556\\
2636.99999999999	62.6597398657616\\
2637.20000000002	62.6625715694081\\
2637.39999999998	62.6578199052133\\
2637.70000000003	62.6620668739101\\
2638.00000000003	62.6549410560631\\
2638.20000000003	62.6577720501838\\
2638.29999999998	62.6553972104306\\
2638.70000000001	62.6496892526906\\
2638.89999999998	62.6525198938992\\
2639.10000000003	62.647772052137\\
2639.30000000001	62.6506024096386\\
2639.59999999999	62.6434822138879\\
2640.00000000003	62.6491420779516\\
2640.10000000001	62.6467691841527\\
2640.40000000001	62.6510130657073\\
2640.50000000003	62.6486404605014\\
2640.79999999999	62.6528834866902\\
2640.90000000003	62.6505111700114\\
2641.20000000001	62.6471813122326\\
2641.70000000002	62.6428949958362\\
2641.99999999996	62.6395670110897\\
2642.40000000003	62.6452223273415\\
2642.50000000002	62.6428517369258\\
2642.89999999997	62.6485054861899\\
2643	62.6461352200068\\
2643.29999999998	62.6503745176666\\
2643.39999999999	62.6480045394364\\
2643.79999999996	62.6536555845531\\
2643.90000000002	62.6512859304085\\
2644.20000000004	62.6555232008471\\
2644.29999999998	62.6531538345182\\
2644.49999999998	62.6559782197686\\
2644.80000000001	62.6488714129079\\
2645	62.6516955880685\\
2645.2	62.6469587570408\\
2645.5	62.6511944360448\\
2645.59999999998	62.6488263975507\\
2646.09999999999	62.6558839090016\\
2646.20000000004	62.6535162302082\\
2647.20000000003	62.6638461829033\\
2647.30000000001	62.661479187127\\
2647.79999999999	62.6685297783149\\
2648.1	62.6614304055585\\
2649.40000000003	62.6759765993584\\
2649.50000000003	62.6736111111111\\
2649.99999999999	62.680653560243\\
2650.19999999996	62.6759234803607\\
2650.70000000003	62.68296363362\\
2650.80000000001	62.6805990418349\\
2650.99999999996	62.6834144317453\\
2651.09999999998	62.681050090525\\
2651.39999999996	62.6852724872714\\
2651.5	62.6829084326445\\
2651.7	62.6857229051965\\
2651.90000000001	62.6809954751131\\
2652.10000000004	62.6838096674459\\
2652.59999999997	62.6719945715686\\
2652.79999999996	62.6748086999133\\
2652.90000000004	62.672446287222\\
2653.40000000003	62.664405502167\\
2653.69999999999	62.668626121034\\
2653.80000000002	62.6662647424545\\
2654	62.6690780302174\\
2654.1	62.6667169015146\\
2654.29999999997	62.6695298372514\\
2654.50000000001	62.6648082573646\\
2655.10000000001	62.6694787586623\\
2655.20000000003	62.6671185930027\\
2655.6	62.6727416500358\\
2655.69999999998	62.6703818058589\\
2656.09999999998	62.6722385362548\\
2656.19999999997	62.6698791552159\\
2656.40000000004	62.6726896292114\\
2656.79999999997	62.6632541683917\\
2657.10000000003	62.6599427969291\\
2657.49999999999	62.6655629139073\\
2657.6	62.663205026903\\
2658.00000000001	62.6688235958015\\
2658.09999999997	62.6664660296441\\
2658.49999999998	62.6645602948921\\
2659.2	62.6743879968413\\
2659.30000000002	62.6720312852523\\
2659.59999999997	62.6762416813926\\
2659.80000000004	62.6715290048498\\
2660.49999999997	62.6587987671954\\
2660.79999999999	62.6630087564358\\
2660.90000000002	62.6606538895152\\
2661.10000000002	62.663460093191\\
2661.20000000003	62.661105474768\\
2661.49999999999	62.6653140967839\\
2661.59999999999	62.6629597625578\\
2662.60000000002	62.676982010741\\
2662.70000000002	62.6746282109058\\
2663.19999999997	62.6816355649007\\
2663.39999999999	62.676928853013\\
2663.69999999997	62.6736241459569\\
2663.89999999998	62.6764264264264\\
2664	62.6740737960287\\
2664.19999999998	62.6768757272079\\
2664.29999999998	62.6745233448431\\
2665.20000000002	62.6871271526658\\
2665.49999999998	62.6800720288115\\
2665.7	62.6828719333784\\
2665.79999999996	62.6805206496868\\
2665.99999999996	62.6833202055437\\
2666.10000000004	62.6809691696047\\
2666.59999999998	62.6767165410432\\
2667.10000000003	62.6837132573485\\
2667.19999999997	62.6813631762456\\
2667.7	62.6846090411575\\
2667.90000000002	62.6799100449775\\
2668.40000000001	62.6831553307101\\
2668.49999999998	62.6808064153489\\
2669.50000000001	62.6947857356907\\
2669.60000000002	62.6924373525115\\
2670.00000000004	62.6942811130669\\
2670.09999999998	62.6919331885252\\
2670.50000000003	62.6975211562945\\
2670.59999999998	62.6951735500056\\
2671.09999999999	62.6984126984127\\
2671.20000000004	62.6960655860443\\
2671.79999999997	62.7044425315319\\
2671.99999999997	62.699749260881\\
2672.4	62.7053320860617\\
2672.49999999998	62.7029858564694\\
2672.90000000003	62.7048260381594\\
2673.29999999996	62.6954440038902\\
2673.60000000002	62.6996297265961\\
2673.69999999997	62.6972847632583\\
2673.99999999997	62.7014696533413\\
2674.09999999999	62.6991249719542\\
2674.29999999997	62.7019144481005\\
2674.69999999997	62.6925377598325\\
2675.2	62.6957724367361\\
2675.30000000001	62.6934290199596\\
2675.50000000004	62.6962176708028\\
2675.69999999998	62.6915315045968\\
2675.89999999997	62.6943198804185\\
2675.99999999997	62.6919771308994\\
2676.50000000001	62.6989464245685\\
2676.60000000001	62.6966040273471\\
2677	62.6947069590228\\
2677.29999999997	62.6914170463883\\
2677.79999999996	62.6946487919639\\
2677.89999999998	62.6923076923077\\
2678.29999999997	62.6904121863799\\
2678.49999999996	62.693197939222\\
2678.70000000002	62.6885172465283\\
2679.09999999998	62.6903553299492\\
2679.20000000003	62.6880155264435\\
2680.00000000003	62.6991530166785\\
2680.19999999997	62.6944744991232\\
2681.4	62.7074398657468\\
2681.50000000003	62.7051014319809\\
2681.80000000002	62.7092732764085\\
2681.90000000002	62.7069351230425\\
2682.2	62.7036498527383\\
2682.40000000004	62.7064305684995\\
2682.50000000003	62.7040930440617\\
2682.89999999997	62.7096533730898\\
2683.00000000001	62.7073161641385\\
2683.30000000001	62.7114854289334\\
2683.40000000001	62.7091485000932\\
2683.80000000002	62.7109802898767\\
2683.90000000004	62.7086438152012\\
2684.10000000004	62.7114223977349\\
2684.29999999999	62.7067501117568\\
2684.49999999996	62.7095284213663\\
2684.59999999999	62.7071926099751\\
2684.79999999999	62.7099705761853\\
2685.19999999997	62.7006293524001\\
2685.39999999999	62.7034071867436\\
2685.60000000003	62.6987377592434\\
2685.79999999998	62.7015153207491\\
2685.99999999999	62.6968467294591\\
2686.89999999998	62.7056196501675\\
2686.99999999996	62.7032860704849\\
2687.60000000003	62.7116121590951\\
2687.69999999997	62.7092789642087\\
2687.90000000004	62.7120535714286\\
2687.99999999996	62.7097206205126\\
2688.50000000002	62.7166555084431\\
2688.59999999999	62.7143229069811\\
2689.19999999997	62.7226415795932\\
2689.30000000003	62.7203093626831\\
2689.8	62.7272389308153\\
2689.89999999999	62.724907063197\\
2690.30000000001	62.7267320844484\\
2690.4	62.7244006690206\\
2690.99999999998	62.7327115306009\\
2691.10000000001	62.7303804994055\\
2691.4	62.7271038454393\\
2691.90000000004	62.7340267459138\\
2692.10000000004	62.7293663175098\\
2692.49999999998	62.7274753026814\\
2693.09999999997	62.7320659438586\\
2693.29999999999	62.7274077374322\\
2693.60000000002	62.7315588224375\\
2693.70000000001	62.7292300838964\\
2693.9	62.731997030438\\
2693.99999999998	62.7296685349467\\
2694.19999999996	62.7324351408529\\
2694.30000000002	62.730106888361\\
2694.6	62.7268341559357\\
2694.89999999999	62.7309833024119\\
2695.09999999999	62.7263282873256\\
2695.3	62.7290940120205\\
2695.40000000003	62.7267668336116\\
2695.89999999999	62.7299703264095\\
2695.99999999996	62.7276436333964\\
2696.59999999997	62.7359365149998\\
2696.69999999996	62.733610204687\\
2696.99999999998	62.7303399948092\\
2697.2	62.7331034738442\\
2697.30000000001	62.7307777860162\\
2698.09999999999	62.7418278852568\\
2698.30000000002	62.7371775867181\\
2698.79999999998	62.744080921857\\
2698.90000000004	62.7417562060022\\
2699.09999999996	62.7445168938945\\
2699.30000000004	62.7398681188412\\
2699.60000000002	62.7366003630033\\
2699.79999999998	62.7393607170636\\
2700.20000000002	62.7300670295893\\
2700.6	62.7281815825527\\
2700.90000000004	62.7249166975194\\
2701.29999999998	62.7304360701858\\
2701.50000000001	62.7257921231863\\
2702.20000000003	62.7317470303075\\
2702.40000000004	62.72710453284\\
2702.69999999996	62.7312416752997\\
2702.80000000002	62.7289207887824\\
2703.19999999998	62.7344356897126\\
2703.40000000001	62.7297947105604\\
2703.59999999999	62.732551688427\\
2703.80000000002	62.7279115351899\\
2704.39999999999	62.7361804400074\\
2704.50000000001	62.7338608296976\\
2704.8	62.730600022182\\
2705.20000000003	62.7361105977156\\
2705.49999999998	62.7291543465405\\
2705.80000000001	62.7332865220444\\
2705.89999999996	62.7309682187731\\
2706.19999999998	62.7350995824558\\
2706.30000000002	62.732781554833\\
2706.80000000004	62.72488824855\\
2707.10000000004	62.7290189125296\\
2707.29999999996	62.7243850188373\\
2707.69999999997	62.7225053549007\\
2707.90000000003	62.725258493353\\
2707.99999999996	62.7229422842583\\
2708.50000000005	62.7298235250683\\
2708.90000000003	62.7205610926541\\
2709.40000000001	62.7274404871748\\
2709.69999999998	62.7204959775629\\
2710.1	62.7259980813224\\
2710.20000000001	62.7236837250489\\
};
\addplot [color=mycolor3, forget plot]
  table[row sep=crcr]{%
2710.20000000001	62.7236837250489\\
2710.49999999997	62.7204309009076\\
2710.90000000003	62.7222427148654\\
2710.99999999998	62.7199291800376\\
2711.50000000002	62.7268033633279\\
2711.60000000005	62.7244901722167\\
2711.90000000004	62.7212389380531\\
2712.29999999998	62.7267364695473\\
2712.39999999998	62.7244239631336\\
2712.6	62.7271721900689\\
2713	62.7179241458111\\
2713.4	62.7123641053989\\
2713.80000000002	62.7141751722613\\
2713.89999999997	62.7118644067797\\
2714.30000000004	62.7063071028588\\
2714.79999999999	62.7131754392427\\
2714.90000000003	62.7108655616943\\
2715.19999999999	62.7076197841859\\
2715.59999999999	62.7094303494495\\
2715.69999999996	62.7071212902276\\
2716.10000000003	62.7052499815919\\
2716.30000000002	62.7079958768959\\
2716.39999999999	62.7056874654887\\
2716.69999999999	62.7024440518257\\
2716.99999999998	62.7065621434618\\
2717.10000000002	62.7042543795083\\
2717.50000000001	62.7097438916691\\
2717.60000000004	62.707436435221\\
2717.89999999997	62.7115526122149\\
2717.99999999998	62.7092454287922\\
2718.19999999999	62.7119891108413\\
2718.5	62.7050687854043\\
2718.89999999998	62.7031997057742\\
2719.5	62.7114281511987\\
2719.6	62.7091223296687\\
2719.80000000002	62.7118644067797\\
2719.99999999997	62.7072534098011\\
2720.6	62.7118021097512\\
2720.70000000003	62.7094972067039\\
2721.00000000003	62.713608467164\\
2721.30000000003	62.706695083413\\
2721.70000000004	62.7121757660372\\
2721.79999999999	62.7098717807414\\
2722.10000000001	62.7139813386232\\
2722.2	62.7116776255372\\
2722.60000000003	62.7171557644985\\
2722.70000000004	62.7148523578669\\
2722.99999999999	62.7116154382873\\
2723.40000000001	62.7060767394896\\
2723.59999999998	62.7088152145978\\
2723.80000000001	62.7042108741143\\
2724.20000000003	62.7060162243512\\
2724.39999999999	62.7014131033217\\
2724.90000000003	62.7045871559633\\
2725.10000000002	62.6999853221782\\
2725.30000000002	62.7027225361415\\
2725.40000000004	62.7004219409283\\
2725.80000000001	62.6985582743314\\
2726.30000000001	62.7053990610329\\
2726.39999999998	62.7030992114432\\
2726.89999999996	62.6952695269527\\
2727.40000000002	62.698441796517\\
2727.69999999998	62.6915463010485\\
2728.09999999997	62.6933509273514\\
2728.2	62.6910530366895\\
2728.60000000003	62.6965221534064\\
2728.70000000001	62.6942245675755\\
2729	62.6909970319886\\
2729.39999999998	62.6964645539476\\
2729.69999999996	62.6895743277896\\
2729.89999999996	62.6923076923077\\
2730.00000000003	62.6900113548954\\
2730.29999999997	62.6941107530032\\
2730.49999999998	62.6895187870798\\
2731.10000000003	62.6977152899824\\
2731.30000000002	62.693124405067\\
2731.69999999999	62.6949264221393\\
2731.89999999997	62.690336749634\\
2732.10000000002	62.6930678574043\\
2732.19999999998	62.6907733411412\\
2732.59999999997	62.6962344933582\\
2732.70000000001	62.6939402810305\\
2732.89999999997	62.6966703256495\\
2732.99999999998	62.6943763492005\\
2734.00000000004	62.7080209209612\\
2734.10000000003	62.7057274522712\\
2734.39999999997	62.7025050283416\\
2734.80000000001	62.7006471900252\\
2734.99999999996	62.7033746480933\\
2735.09999999999	62.7010821877742\\
2735.29999999998	62.7038093149082\\
2735.40000000003	62.7015170901115\\
2735.59999999999	62.7042438863911\\
2735.80000000003	62.6996600752952\\
2736.09999999998	62.7037497258972\\
2736.20000000004	62.7014581734459\\
2736.59999999997	62.7069097818541\\
2736.70000000003	62.7046185325928\\
2737.19999999996	62.7114309721258\\
2737.30000000004	62.7091400599109\\
2737.50000000004	62.7118644067797\\
2737.59999999997	62.7095737297732\\
2738.80000000002	62.7259118624265\\
2738.9	62.7236217597663\\
2739.29999999996	62.7254143243046\\
2739.40000000002	62.7231246577843\\
2739.79999999998	62.7176174312931\\
2740.39999999998	62.7257799671593\\
2740.49999999999	62.7234912063052\\
2740.79999999997	62.7275712357255\\
2740.90000000002	62.7252827435243\\
2741.40000000001	62.732080977567\\
2741.50000000003	62.7297928217099\\
2741.99999999996	62.7365887458517\\
2742.09999999997	62.7343009262636\\
2742.39999999998	62.7383773928897\\
2742.49999999996	62.736089841756\\
2742.80000000003	62.732873965511\\
2743.49999999997	62.7423822714681\\
2743.80000000003	62.7355224315755\\
2744.40000000002	62.7436691564948\\
2744.50000000003	62.7413830795016\\
2744.8	62.7381689679041\\
2745.00000000002	62.7408837565116\\
2745.10000000003	62.7385982806353\\
2745.40000000004	62.7426698233473\\
2745.5	62.7403846153846\\
2746.00000000002	62.7471687119915\\
2746.10000000004	62.7448838394873\\
2746.40000000001	62.7416712179137\\
2746.70000000001	62.7384592980923\\
2747.10000000002	62.7438846825859\\
2747.60000000001	62.7324671543473\\
2747.80000000003	62.7351795916882\\
2748.00000000001	62.7306138786798\\
2748.30000000002	62.7346819967981\\
2748.40000000003	62.7323994906313\\
2749.1	62.7382511276008\\
2749.29999999997	62.7336873499673\\
2749.6	62.7304796886933\\
2749.79999999996	62.733190297829\\
2749.90000000004	62.7309090909091\\
2750.40000000002	62.7340483548446\\
2750.69999999999	62.7272066307983\\
2750.90000000001	62.7299163940385\\
2750.99999999999	62.72763621824\\
2751.29999999997	62.7244311986625\\
2752.00000000002	62.7339122851641\\
2752.30000000001	62.7270745531173\\
2752.69999999998	62.7252252252252\\
2753.00000000001	62.7220224474229\\
2753.39999999998	62.7165425821681\\
2753.6	62.7192504630134\\
2753.80000000003	62.7146955227133\\
2754.29999999999	62.7105721754284\\
2754.7	62.7159866414985\\
2754.80000000004	62.7137101165197\\
2755.20000000001	62.7118644067797\\
2755.80000000004	62.7199825828223\\
2756.19999999996	62.7108805282444\\
2757.00000000002	62.6963113416271\\
2757.19999999999	62.6990171544627\\
2757.30000000004	62.6967433089142\\
2757.80000000002	62.7035062910185\\
2757.90000000003	62.7012327773749\\
2758.30000000002	62.7066415313225\\
2758.70000000004	62.6975496592721\\
2759.09999999996	62.7029573789504\\
2759.20000000001	62.7006849563295\\
2759.4	62.703388294981\\
2759.50000000004	62.7011161037832\\
2760.09999999998	62.7092239692776\\
2760.20000000003	62.706952142883\\
2760.39999999999	62.7096540481797\\
2760.80000000005	62.7005686551487\\
2761.10000000003	62.6973779516152\\
2761.50000000003	62.6955388180765\\
2761.90000000001	62.697320782042\\
2762.09999999997	62.6927811165013\\
2762.29999999999	62.6954821894005\\
2762.40000000002	62.6932126696833\\
2762.79999999997	62.6986137753809\\
2763.00000000001	62.6940754949151\\
2763.19999999997	62.696775594398\\
2763.3	62.6945067670261\\
2763.50000000003	62.6972065421913\\
2763.79999999996	62.6904012446181\\
2764.10000000002	62.6872151074452\\
2764.49999999996	62.6889965998698\\
2764.60000000001	62.6867291207003\\
2765.20000000001	62.6912089104256\\
2765.50000000003	62.68440844663\\
2766.29999999998	62.6951995373048\\
2766.4	62.6929333092355\\
2766.60000000003	62.6956301731305\\
2766.69999999996	62.6933641752205\\
2767.20000000002	62.6856502728291\\
2767.69999999998	62.6779391574536\\
2768.19999999999	62.6846801285988\\
2768.6	62.6756239390328\\
2768.80000000002	62.6783199104337\\
2769.00000000004	62.6737929291105\\
2769.3	62.677836354445\\
2769.6	62.6710474058562\\
2769.8	62.6737427343948\\
2769.89999999998	62.6714801444043\\
2770.39999999996	62.6746074715755\\
2770.50000000003	62.6723453403595\\
2771.10000000004	62.6623845265589\\
2771.29999999997	62.6650790214332\\
2771.39999999999	62.6628179686091\\
2771.80000000004	62.6645982899816\\
2771.99999999998	62.6600771977923\\
2772.20000000002	62.6627709843812\\
2772.49999999999	62.6559907667893\\
2772.90000000001	62.6541651640822\\
2773.80000000003	62.6662821298533\\
2774.00000000002	62.6617641757687\\
2774.40000000003	62.6635429807173\\
2774.50000000003	62.6612845094788\\
2774.89999999997	62.6666666666667\\
2775	62.6644084897842\\
2775.60000000004	62.668876319487\\
2775.70000000002	62.6666186324663\\
2775.90000000001	62.6693083573487\\
2776.10000000003	62.6647936027664\\
2776.80000000002	62.6742050487954\\
2776.99999999999	62.6696914047027\\
2777.30000000002	62.6737236264132\\
2777.50000000004	62.6692108294931\\
2778.09999999997	62.6772730544957\\
2778.29999999996	62.6727613014685\\
2778.49999999999	62.6754480673721\\
2778.89999999997	62.6664267722202\\
2779.10000000004	62.6691134139321\\
2779.19999999999	62.6668585615083\\
2779.70000000002	62.6627814950716\\
2780.30000000005	62.6708387282405\\
2780.39999999997	62.6685847869088\\
2780.79999999996	62.6667625588838\\
2781.19999999997	62.6649408549959\\
2781.79999999999	62.65861461591\\
2782.09999999998	62.6554525195888\\
2782.29999999999	62.6581368602645\\
2782.59999999998	62.6513817515363\\
2783.10000000004	62.6437194596148\\
2783.49999999997	62.6490875125736\\
2783.60000000002	62.6468369436362\\
2784.00000000003	62.6486117596351\\
2784.20000000002	62.6441116259024\\
2784.39999999999	62.6467947566888\\
2784.49999999996	62.6445449974862\\
2785.00000000002	62.6404796955226\\
2785.30000000004	62.6445034824442\\
2785.39999999999	62.6422545323999\\
2785.79999999999	62.6476183639039\\
2785.90000000004	62.6453697056712\\
2786.30000000003	62.6507321274763\\
2786.40000000002	62.6484837609905\\
2786.80000000005	62.653844773763\\
2786.99999999997	62.649348785476\\
2787.20000000001	62.6520288451189\\
2787.30000000003	62.6497811580684\\
2788.10000000002	62.6604978122086\\
2788.20000000001	62.6582505469282\\
2788.59999999996	62.6600207982214\\
2788.70000000003	62.6577739529547\\
2788.99999999998	62.6617905417518\\
2789.1	62.659543955256\\
2789.50000000001	62.6648981932894\\
2789.80000000001	62.6581597906735\\
2790.19999999997	62.6599290398882\\
2790.40000000004	62.6554380935316\\
2790.60000000001	62.6581144515713\\
2790.69999999997	62.6558692847929\\
2791.10000000001	62.6612209802236\\
2791.19999999997	62.6589761043242\\
2791.40000000001	62.6616514418771\\
2791.50000000003	62.659406791804\\
2791.90000000005	62.6647564469914\\
2792.00000000002	62.6625120876759\\
2792.29999999998	62.6665234207134\\
2792.99999999998	62.6508180874297\\
2793.39999999998	62.6454268838375\\
2793.80000000003	62.6507749024661\\
2793.99999999999	62.6462903976236\\
2794.50000000001	62.6493952622916\\
2794.59999999998	62.6471535406305\\
2794.99999999999	62.6524990161354\\
2795.20000000001	62.648016313097\\
2796	62.6587031937341\\
2796.09999999999	62.6564623417495\\
2796.4	62.6533166458073\\
2797.10000000002	62.6626626626627\\
2797.29999999996	62.6581825981268\\
2797.50000000003	62.6608521589934\\
2797.80000000004	62.6541334572358\\
2798.09999999996	62.6581373740262\\
2798.30000000004	62.6536592338479\\
2798.60000000004	62.6576624861543\\
2798.70000000002	62.655423753037\\
2799.00000000003	62.6522810903505\\
2799.30000000001	62.6562834893192\\
2799.39999999999	62.6540453652438\\
2799.69999999996	62.6580470033574\\
2799.79999999997	62.6558091360406\\
2800.19999999996	62.6504303110381\\
2800.70000000003	62.657097972008\\
2800.80000000004	62.6548609375558\\
2801.2	62.6530539392425\\
2801.69999999998	62.6454422157185\\
2801.90000000001	62.6481084939329\\
2802.10000000002	62.6436371422454\\
2802.6	62.6467335069754\\
2802.70000000003	62.6444983587841\\
2803.10000000005	62.6391267123288\\
2803.40000000004	62.6431246655966\\
2803.49999999999	62.6408902839207\\
2803.7	62.6435551751195\\
2803.8	62.6413210171547\\
2804.00000000001	62.6439855925252\\
2804.2	62.6395178832507\\
2804.7	62.6461779806047\\
2804.8	62.6439445256515\\
2805.10000000003	62.6479395408527\\
2805.19999999997	62.6457063415677\\
2805.49999999997	62.6497005988024\\
2805.69999999997	62.6452348706251\\
2806.00000000003	62.6421011368091\\
2806.30000000001	62.646094640821\\
2806.39999999999	62.6438624621415\\
2806.59999999997	62.6465243880714\\
2807.00000000002	62.6375975205728\\
2807.19999999999	62.6402593239055\\
2807.50000000004	62.6335660350477\\
2807.70000000002	62.6362276515421\\
2807.8	62.6339969372129\\
2808.10000000002	62.6308667473827\\
2808.50000000003	62.6361888485366\\
2808.60000000001	62.6339587709617\\
2808.79999999997	62.6366193171704\\
2808.89999999998	62.6343894624422\\
2809.49999999998	62.6388097949886\\
2809.7	62.6343511993736\\
2809.89999999996	62.6370106761566\\
2810.00000000003	62.6347816803673\\
2810.30000000003	62.6387702818104\\
2810.4	62.6365415406511\\
2810.70000000005	62.6334139746691\\
2811.10000000001	62.6387307911212\\
2811.40000000001	62.6320469500267\\
2811.6	62.6347049827506\\
2811.69999999997	62.6324774166015\\
2812.10000000002	62.627124671076\\
2812.90000000001	62.6377532883043\\
2813.00000000001	62.635526643205\\
2813.19999999997	62.6381829168592\\
2813.39999999997	62.6337302292518\\
2813.80000000002	62.6319343260244\\
2814.1	62.6359178452136\\
2814.2	62.6336922147603\\
2814.8	62.6381043731571\\
2814.89999999997	62.6358792184725\\
2815.19999999997	62.6398607608425\\
2815.29999999999	62.6376358599133\\
2815.49999999996	62.640289813894\\
2815.59999999996	62.63806513478\\
2816.00000000003	62.6433720393452\\
2816.19999999998	62.6389234101481\\
2816.60000000004	62.6371285546917\\
2817.09999999998	62.6402101377254\\
2817.29999999998	62.6357634698658\\
2817.69999999996	62.6410674994677\\
2817.89999999996	62.6366217175302\\
2818.29999999996	62.641924496168\\
2818.40000000001	62.6397019691325\\
2818.70000000002	62.6436781609195\\
2818.80000000004	62.6414558870481\\
2819.09999999999	62.6454313280363\\
2819.2	62.6432093072749\\
2819.60000000002	62.6485087066\\
2819.79999999998	62.6440653923898\\
2820.19999999998	62.6493635428855\\
2820.30000000004	62.6471422493263\\
2820.60000000002	62.651114971461\\
2820.70000000002	62.6488939307998\\
2820.90000000002	62.6515420063807\\
2820.99999999997	62.6493211867711\\
2821.2	62.6519689504838\\
2821.30000000004	62.649748351882\\
2822.00000000002	62.6590127918926\\
2822.1	62.6567925731699\\
2823.09999999999	62.6700198356475\\
2823.19999999999	62.6678000920908\\
2823.80000000001	62.6757321434895\\
2823.89999999998	62.6735127478754\\
2824.29999999998	62.6787990369636\\
2824.69999999997	62.6699235344095\\
2824.99999999997	62.6668082545751\\
2825.19999999996	62.6694510317488\\
2825.29999999998	62.6672329581652\\
2825.6	62.671196517677\\
2825.70000000002	62.6689786962984\\
2826.10000000002	62.6742622602788\\
2826.19999999999	62.6720447227824\\
2826.50000000003	62.6689308710111\\
2826.7	62.6715720956559\\
2826.99999999996	62.664921651162\\
2827.40000000002	62.6702033598585\\
2827.70000000002	62.6635547068392\\
2827.99999999997	62.6675152929529\\
2828.20000000002	62.6630838312767\\
2828.99999999997	62.6701071011983\\
2829.10000000003	62.6678919835996\\
2829.60000000003	62.6709545181468\\
2829.99999999996	62.662096745698\\
2830.20000000003	62.6647351870826\\
2830.59999999997	62.6558801709825\\
2830.80000000001	62.6585184923523\\
2830.9	62.6563051925115\\
2831.20000000004	62.6531981775156\\
2831.69999999997	62.6597923582174\\
2831.79999999998	62.6575797167979\\
2832.10000000003	62.6615352023162\\
2832.40000000002	62.6548984995587\\
2832.9	62.6614895870102\\
2833.09999999998	62.6570662148807\\
2833.90000000003	62.6676076217361\\
2834.19999999997	62.660974491056\\
2834.80000000001	62.6653497477865\\
2834.9	62.663139329806\\
2835.29999999997	62.6578260562884\\
2835.59999999996	62.6617766336354\\
2835.79999999996	62.6573574526605\\
2836.10000000001	62.6613073831183\\
2836.19999999999	62.6590981207912\\
2836.49999999997	62.6559966156667\\
2836.7	62.6586294416244\\
2836.79999999998	62.6564207409496\\
2837.19999999996	62.6581609276425\\
2837.30000000004	62.6559526326919\\
2837.69999999999	62.6541687222496\\
2838.00000000004	62.6510693774004\\
2838.80000000002	62.6615942794744\\
2838.90000000004	62.6593871081367\\
2839.40000000002	62.6659623173094\\
2839.50000000001	62.6637554585153\\
2839.89999999996	62.6584507042254\\
2840.29999999998	62.6637093367131\\
2840.50000000005	62.6592973315497\\
2840.69999999999	62.6619262179668\\
2840.80000000002	62.6597205111056\\
2841.10000000001	62.6636632408841\\
2841.20000000002	62.661457783409\\
2842.20000000003	62.6745945185237\\
2842.30000000005	62.6723895299747\\
2842.60000000002	62.6692932775179\\
2843.40000000001	62.6586952699138\\
2843.79999999996	62.6604310981399\\
2843.89999999998	62.6582278481013\\
2844.29999999998	62.6599634369287\\
2844.5	62.6555578991774\\
2844.70000000001	62.658183352081\\
2845.00000000002	62.6515763945028\\
2845.40000000001	62.6568265682657\\
2845.49999999999	62.6546246837222\\
2846.00000000004	62.6611854818875\\
2846.09999999999	62.6589839083691\\
2846.39999999998	62.6558932021781\\
2846.7	62.6598285794576\\
2846.90000000001	62.6554267650158\\
2847.10000000004	62.6580500140489\\
2847.20000000002	62.6558494011871\\
2848.09999999996	62.664138754301\\
2848.19999999997	62.6619387002774\\
2848.39999999996	62.6645602948921\\
2848.49999999999	62.6623604577687\\
2848.89999999996	62.6640926640927\\
2849.10000000002	62.6596939491787\\
2849.39999999996	62.6636251974031\\
2849.50000000005	62.6614261650758\\
2850.40000000001	62.6697070689353\\
2850.60000000004	62.6653102746694\\
2851.10000000001	62.6718574635241\\
2851.20000000003	62.6696594535826\\
2851.50000000004	62.6735867583111\\
2851.59999999996	62.6713889960375\\
2851.79999999998	62.6740068024826\\
2851.89999999998	62.671809256662\\
2852.10000000001	62.6744267582918\\
2852.20000000003	62.672229428882\\
2852.5	62.6761550865877\\
2852.59999999996	62.6739580046973\\
2853.30000000002	62.6796102894792\\
2853.40000000001	62.6774137024707\\
2853.6	62.6800294354697\\
2853.69999999997	62.6778330646857\\
2854.10000000001	62.6795599467451\\
2854.2	62.6773639771573\\
2854.60000000001	62.6755876274215\\
2854.99999999997	62.6738117754194\\
2855.39999999996	62.6685344072842\\
2855.70000000004	62.6724560543455\\
2855.90000000002	62.6680672268908\\
2856.39999999998	62.6711009977245\\
2856.60000000001	62.6667133405678\\
2856.79999999999	62.6693268927859\\
2856.89999999996	62.6671333566678\\
2857.39999999996	62.673665791776\\
2857.49999999997	62.6714725643897\\
2858.39999999996	62.6832254679027\\
2858.49999999996	62.6810326733366\\
2858.69999999999	62.6836434867777\\
2858.99999999996	62.6770662096464\\
2859.30000000001	62.6809820242009\\
2859.40000000003	62.6787899982514\\
2859.79999999996	62.6840099304171\\
2860	62.679626586483\\
2860.49999999998	62.6861497587919\\
2860.59999999998	62.6839584717027\\
2861.09999999996	62.6799944079407\\
2861.60000000004	62.6760317293916\\
2861.89999999997	62.6729559748428\\
2862.2	62.6698808650386\\
2862.89999999998	62.6755151938526\\
2863.10000000001	62.671137189159\\
2863.40000000004	62.6750480181596\\
2863.70000000001	62.6684824359243\\
2865.19999999999	62.6845356507172\\
2865.40000000002	62.6801605304484\\
2865.79999999999	62.6818800376845\\
2866.09999999997	62.6753192380155\\
2866.60000000003	62.6783409495239\\
2866.79999999998	62.6739683979211\\
2867.1	62.6778738839286\\
2867.29999999999	62.6735021273628\\
2867.50000000004	62.6761054540382\\
2867.60000000004	62.6739198660948\\
2868.00000000004	62.6791255535023\\
2868.09999999995	62.6769402412663\\
2868.60000000004	62.6695018649563\\
2868.89999999998	62.6734053677239\\
2869.00000000004	62.6712209403646\\
2869.39999999996	62.6764244641924\\
2869.59999999996	62.6720563125065\\
2870.39999999998	62.6789757881902\\
2870.49999999997	62.6767923082282\\
2870.80000000001	62.6737260092654\\
2871.69999999998	62.6819416393899\\
2872.10000000001	62.6732121718543\\
2872.49999999997	62.6749286360788\\
2872.60000000003	62.6727468931667\\
2872.89999999996	62.676644622346\\
2873.00000000002	62.6744631234555\\
2873.20000000003	62.6770612188076\\
2873.29999999996	62.6748799331802\\
2873.7	62.6765954485351\\
2873.8	62.6744145586137\\
2874.90000000005	62.6852173913043\\
2875	62.6830371117526\\
2875.40000000004	62.6812728221179\\
2876	62.6890580995098\\
2876.09999999996	62.6868785202698\\
2876.30000000002	62.6894729523015\\
2876.40000000002	62.6872935859552\\
2876.69999999999	62.6911846496107\\
2877.00000000002	62.6846477355671\\
2877.29999999999	62.6885382637103\\
2877.39999999997	62.6863596872285\\
2877.79999999998	62.681121651204\\
2877.99999999996	62.6837149508356\\
2878.10000000001	62.681537071781\\
2878.29999999999	62.6841300722624\\
2878.39999999999	62.6819524057669\\
2878.60000000004	62.6845451071664\\
2878.79999999996	62.6801903504811\\
2879.40000000004	62.6879666608786\\
2879.49999999998	62.6857896930129\\
2879.90000000001	62.6875\\
2880.10000000001	62.6831470036803\\
2880.3	62.6857380919317\\
2880.50000000005	62.6813858223981\\
2880.79999999998	62.6852719636225\\
2880.90000000001	62.6830961471711\\
2881.2	62.6869815708187\\
2881.30000000002	62.6848059970848\\
2881.60000000002	62.6886907034042\\
2881.70000000002	62.6865153723367\\
2882.59999999996	62.6981649148368\\
2882.69999999996	62.6959900097128\\
2883.09999999998	62.6907602663707\\
2883.49999999997	62.6889998612845\\
2883.79999999998	62.6859461146364\\
2884.00000000001	62.688533684685\\
2884.09999999999	62.6863601691977\\
2884.50000000003	62.6915343548499\\
2884.60000000004	62.6893611120741\\
2884.89999999996	62.6932409012132\\
2885.19999999996	62.6867223512286\\
2885.59999999998	62.6884291506394\\
2885.69999999998	62.6862568438561\\
2886.29999999998	62.6940133037694\\
2886.39999999999	62.6918413303309\\
2886.80000000004	62.6900827877654\\
2887.00000000004	62.6926673824945\\
2887.10000000001	62.6904959822666\\
2887.3	62.6930802798365\\
2887.60000000003	62.6865671641791\\
2888.2	62.694318457224\\
2888.80000000003	62.6812973796255\\
2889.09999999996	62.6851723660529\\
2889.20000000003	62.6830028034472\\
2889.7	62.6790781368953\\
2890.00000000003	62.6760319712121\\
2890.50000000002	62.6721095966235\\
2891.09999999997	62.6763973436635\\
2891.19999999997	62.6742295853077\\
2891.80000000001	62.6819737888585\\
2891.9	62.679806362379\\
2892.29999999997	62.6745954916333\\
2892.50000000002	62.6771762428265\\
2892.60000000001	62.6750095066892\\
2892.79999999997	62.6775899616302\\
2892.90000000003	62.6754234358797\\
2893.30000000003	62.6805833966959\\
2893.50000000005	62.6762510367708\\
2894.29999999999	62.6865671641791\\
2894.40000000003	62.6844014510278\\
2894.80000000002	62.6826487961588\\
2895.30000000001	62.689093044139\\
2895.39999999996	62.6869279917113\\
2895.60000000004	62.6895051282937\\
2895.69999999998	62.6873402859314\\
2896.1	62.6821352116567\\
2896.69999999999	62.6898646782657\\
2897.10000000004	62.6812094436007\\
2897.5	62.6794588625069\\
2898.1	62.6871851494031\\
2898.30000000002	62.6828595086945\\
2898.59999999999	62.6867216338359\\
2898.70000000003	62.684559127915\\
2899.00000000004	62.6884205443069\\
2899.19999999998	62.6840961611423\\
2899.39999999997	62.6866701155372\\
2899.60000000001	62.6823464496327\\
2900.10000000001	62.6784359699331\\
2900.70000000002	62.6827082184225\\
2900.90000000004	62.6783867631851\\
2902.40000000003	62.6976744186047\\
2902.50000000002	62.6955143664301\\
2902.99999999999	62.6916055251283\\
2903.3	62.6954604945926\\
2903.40000000001	62.6933011882211\\
2903.99999999996	62.6837918804449\\
2904.50000000003	62.6798870756731\\
2905.30000000003	62.686721277621\\
2905.39999999999	62.6845637583893\\
2905.89999999999	62.6909841706814\\
2906	62.6888269502082\\
2906.40000000005	62.6939618097368\\
2906.50000000002	62.6918048579096\\
2907.00000000004	62.6982215954044\\
2907.10000000002	62.6960649422124\\
2907.70000000004	62.6934452163147\\
2907.99999999999	62.6904164230941\\
2908.39999999996	62.6955475330927\\
2908.50000000005	62.6933920099017\\
2908.70000000003	62.6959570957096\\
2908.99999999998	62.6894915953388\\
2909.19999999997	62.6920565084385\\
2909.30000000001	62.6899016979446\\
2909.8	62.6928760438503\\
2909.9	62.6907216494845\\
2910.19999999997	62.6945675703536\\
2910.29999999998	62.6924134139637\\
2910.69999999999	62.6906692318263\\
2911.30000000001	62.6949234045477\\
2911.4	62.6927700498025\\
2911.9	62.6991758241758\\
2912.00000000003	62.6970227670753\\
2912.20000000005	62.6995845208255\\
2912.30000000005	62.6974316714737\\
2912.49999999999	62.699993133283\\
2912.60000000004	62.6978404916401\\
2913.19999999998	62.6917928122747\\
2913.90000000003	62.697323266987\\
2913.99999999999	62.695171751141\\
2914.19999999995	62.6977318738634\\
2914.29999999995	62.695580565468\\
2915.00000000002	62.6976776096875\\
2915.19999999998	62.6933763249065\\
2915.50000000001	62.697214981479\\
2915.90000000002	62.6886145404664\\
2916.09999999999	62.691173444894\\
2916.30000000003	62.6868742285009\\
2917.00000000002	62.6958280484042\\
2917.19999999997	62.6915298392349\\
2917.70000000004	62.6876413736377\\
2917.89999999997	62.6901987662783\\
2917.99999999997	62.6880504437819\\
2918.59999999996	62.6957206975708\\
2918.8	62.6914248518277\\
2919.10000000004	62.6952589750617\\
2919.2	62.6931113623129\\
2919.40000000003	62.6956670662785\\
2919.70000000004	62.6892252894034\\
2920.3	62.6934666484043\\
2920.59999999999	62.6870270825487\\
2921.19999999997	62.694690719885\\
2921.3	62.6925446703635\\
2921.79999999999	62.698928779219\\
2922.00000000003	62.6946374182951\\
2922.70000000003	62.7035719173395\\
2922.79999999996	62.7014266652982\\
2923.09999999998	62.7052545155993\\
2923.19999999999	62.7031094995382\\
2923.80000000003	62.6970826635658\\
2923.99999999998	62.699634075442\\
2924.20000000003	62.695345894744\\
2924.60000000003	62.7004479091873\\
2924.69999999997	62.6983041575492\\
2925.10000000002	62.6965677560509\\
2925.40000000005	62.6935566569817\\
2926.39999999997	62.7028874081668\\
2926.49999999999	62.7007448916832\\
2926.70000000001	62.7032936996037\\
2926.80000000004	62.7011513888414\\
2927.60000000001	62.7045120743246\\
2927.80000000001	62.700228832952\\
2928.00000000003	62.7027765445169\\
2928.10000000004	62.7006352025135\\
2928.49999999996	62.705729700198\\
2928.59999999998	62.7035886229385\\
2928.79999999998	62.706135409198\\
2928.89999999996	62.7039945373848\\
2929.29999999999	62.7022598484331\\
2930.29999999996	62.7149877149877\\
2930.89999999997	62.7021494370522\\
2931.49999999997	62.7097830536226\\
2931.59999999998	62.7076440290616\\
2931.89999999999	62.7046384720328\\
2932.20000000004	62.7016335299935\\
2932.59999999998	62.6999011150135\\
2933.19999999999	62.7075307673951\\
2933.49999999997	62.7011180801745\\
2933.79999999998	62.7049320017724\\
2933.90000000003	62.7027948193592\\
2934.1	62.7053370595052\\
2934.7	62.6925173776748\\
2934.90000000002	62.6950596252129\\
2935.00000000002	62.6929235801165\\
2935.20000000001	62.6954655401492\\
2935.30000000001	62.6933296995299\\
2936.10000000004	62.6966827872761\\
2936.19999999999	62.6945475598542\\
2936.39999999999	62.6970883705091\\
2936.50000000004	62.6949533474086\\
2936.9	62.6932243786176\\
2937.40000000001	62.696170212766\\
2937.59999999997	62.6919018279606\\
2937.90000000004	62.6957113682777\\
2938.19999999999	62.6893101453221\\
2938.59999999998	62.6943886752646\\
2938.69999999999	62.6922553423166\\
2938.89999999999	62.6947941476693\\
2939.00000000001	62.6926610186792\\
2939.19999999999	62.6951995373048\\
2939.29999999997	62.6930666122338\\
2939.60000000004	62.696873830663\\
2939.79999999996	62.692608592129\\
2940.39999999999	62.6968202686618\\
2940.50000000002	62.6946881588791\\
2941.4	62.7061023287438\\
2941.5	62.7039706282295\\
2941.99999999999	62.7069100302505\\
2942.3	62.7005165851006\\
2942.80000000004	62.6966597573822\\
2945.09999999997	62.7223957625968\\
2945.3	62.7181367556189\\
2946.10000000002	62.7214717262915\\
2946.29999999998	62.7172142275319\\
2947.00000000004	62.7226765294697\\
2947.10000000001	62.7205483170467\\
2947.69999999999	62.7247438767895\\
2947.89999999996	62.7204884667571\\
2948.19999999997	62.7174982193128\\
2948.40000000003	62.7200271324402\\
2948.60000000003	62.7157730525316\\
2949.10000000005	62.7220941272209\\
2949.3	62.7178409167966\\
2949.89999999997	62.7152542372881\\
2950.29999999998	62.7203091106291\\
2950.4	62.7181833587528\\
2950.70000000002	62.7219737020469\\
2950.79999999999	62.7198481819106\\
2951.30000000003	62.7159991868266\\
2951.59999999999	62.7197885964021\\
2951.69999999995	62.7176637983603\\
2951.89999999998	62.720189701897\\
2951.99999999997	62.7180651061956\\
2952.50000000001	62.7243785138522\\
2952.60000000003	62.722254208013\\
2953.10000000003	62.7150209941758\\
2953.30000000002	62.7175458793255\\
2953.60000000005	62.7111758133866\\
2953.79999999996	62.7137005315007\\
2953.89999999997	62.7115775220041\\
2954.60000000002	62.7204115477037\\
2954.8	62.7161663677282\\
2955.39999999999	62.7203518863137\\
2955.50000000003	62.7182298010556\\
2956.30000000002	62.7283182248681\\
2956.40000000002	62.7261965161509\\
2956.89999999998	62.7291173486642\\
2957.00000000004	62.7269960434209\\
2957.40000000004	62.7218934911243\\
2958.10000000002	62.7104320194713\\
2958.29999999997	62.7129529475392\\
2958.40000000003	62.7108331924962\\
2958.60000000004	62.7133538378342\\
2958.70000000002	62.7112342841693\\
2959.30000000001	62.7052780969115\\
2960	62.7073409682105\\
2960.10000000001	62.7052226200932\\
2960.49999999996	62.7035060460718\\
2960.69999999996	62.7060253985409\\
2961.20000000002	62.6954378144734\\
2961.80000000002	62.7029946993484\\
2961.89999999996	62.7008777852802\\
2962.09999999998	62.7033961245021\\
2962.2	62.7012794112683\\
2962.60000000005	62.7063151854727\\
2962.69999999999	62.7041987309302\\
2963.70000000004	62.7167825089412\\
2963.80000000003	62.7146664867236\\
2964.19999999998	62.7196977363965\\
2964.40000000003	62.71546635183\\
2964.60000000003	62.7179815832968\\
2964.8	62.7137508853587\\
2965.30000000003	62.7099210899036\\
2965.59999999998	62.7069494554406\\
2965.99999999998	62.7052358315633\\
2966.39999999996	62.710264621608\\
2966.59999999998	62.7060370108201\\
2966.80000000002	62.7085510128417\\
2966.90000000001	62.706437478935\\
2967.50000000002	62.700498719504\\
2967.79999999997	62.6975302402372\\
2968.30000000004	62.7004446840049\\
2968.60000000003	62.6941085323542\\
2968.9	62.6978780734254\\
2968.99999999997	62.6957663938567\\
2969.29999999997	62.6927998922341\\
2969.70000000004	62.6877230789952\\
2970.00000000003	62.6914918689606\\
2970.09999999998	62.6893811864521\\
2970.39999999998	62.6931493014644\\
2970.6	62.688928535362\\
2971.39999999996	62.6922429749285\\
2971.5	62.6901332615426\\
2971.70000000003	62.6926441887072\\
2971.80000000004	62.6905346747872\\
2972.29999999999	62.6833535190419\\
2972.70000000004	62.6883745963402\\
2973.29999999999	62.6757247595345\\
2974.50000000003	62.684058360788\\
2974.70000000004	62.6798440231276\\
2975.30000000003	62.6873697654097\\
2975.40000000001	62.6852629810116\\
2975.60000000001	62.6877709446517\\
2975.80000000004	62.6835579152525\\
2976.19999999996	62.6818533077983\\
2976.40000000004	62.684360826474\\
2976.49999999999	62.6822549217228\\
2976.79999999995	62.6792972555343\\
2977.19999999996	62.6843112887516\\
2977.40000000003	62.6801007556675\\
2977.60000000004	62.6826073815361\\
2977.70000000004	62.6805023843106\\
2977.99999999996	62.677546086431\\
2978.19999999997	62.6800523788739\\
2978.30000000004	62.6779478914854\\
2978.79999999998	62.670784517775\\
2979.10000000002	62.6745435016112\\
2979.19999999998	62.6724398348605\\
2979.50000000004	62.6761981474023\\
2979.59999999999	62.6740947075209\\
2979.9	62.6711409395973\\
2980.40000000004	62.6774031202818\\
2980.60000000005	62.6731975710404\\
2981.59999999996	62.6857162021666\\
2981.70000000003	62.6836139244752\\
2982.00000000002	62.6873679621743\\
2982.10000000001	62.6852659110724\\
2982.5	62.6835646751157\\
2983.00000000001	62.6864670979853\\
2983.09999999999	62.6843657817109\\
2983.80000000001	62.6897684238748\\
2984.09999999999	62.6834662556129\\
2984.50000000002	62.6817664008577\\
2985.89999999997	62.6992632283992\\
2986.00000000003	62.6971635243294\\
2986.40000000002	62.7021597187343\\
2986.60000000001	62.6979609602571\\
2986.99999999999	62.6929128586254\\
2987.69999999995	62.6882656134949\\
2988.29999999999	62.6957569267836\\
2988.39999999997	62.6936590262674\\
2988.80000000002	62.6986516778748\\
2989.00000000003	62.6944565253755\\
2991.19999999998	62.7085213786648\\
2991.29999999998	62.7064250852444\\
2991.90000000004	62.7105614973262\\
2992.20000000004	62.7042743040471\\
2992.59999999999	62.702576268921\\
2992.9	62.7063147343802\\
2993.00000000001	62.7042197053222\\
2993.39999999999	62.7025221312845\\
2993.90000000003	62.7087508350033\\
2994	62.7066564243011\\
2994.20000000001	62.709147380022\\
2994.59999999999	62.7007713627408\\
2994.80000000003	62.7032622124278\\
2994.89999999997	62.7011686143573\\
2995.19999999998	62.7049043501486\\
2995.30000000002	62.7028109768311\\
2995.49999999998	62.7053011082922\\
2995.80000000002	62.6990219967289\\
2996.39999999999	62.7064909060571\\
2996.49999999997	62.7043983180938\\
2996.79999999998	62.7081317361273\\
2997	62.7039471489106\\
2997.30000000003	62.7010075398679\\
2997.79999999999	62.6972213883052\\
2997.99999999998	62.699709816217\\
2998.19999999996	62.6955274655638\\
2998.70000000005	62.7017473656129\\
2999.09999999996	62.6933849026407\\
2999.70000000005	62.7008467231149\\
2999.90000000004	62.6966666666667\\
3000.1	62.699153389774\\
3000.20000000001	62.6970636269706\\
3000.49999999997	62.7007931746984\\
3000.59999999997	62.6987036358183\\
3000.8	62.7011896431071\\
3001.00000000004	62.6970110959315\\
3001.50000000004	62.7032249466951\\
3001.59999999998	62.7011360229204\\
3002	62.7061057259918\\
3002.09999999999	62.7040170541603\\
3002.30000000001	62.7065014654943\\
3002.49999999999	62.7023246519683\\
3002.70000000001	62.7048088450779\\
3002.79999999999	62.7027207033201\\
3002.99999999996	62.7052046218907\\
3003.19999999995	62.7010288682449\\
3004.49999999995	62.6872129401584\\
3004.90000000002	62.6821963394343\\
3005.29999999998	62.6771810740667\\
3005.6	62.6809062780717\\
3005.70000000001	62.6788209461707\\
3006.20000000003	62.6717227156305\\
3006.79999999998	62.6791712394825\\
3006.90000000001	62.6770867974726\\
3007.49999999997	62.6712328767123\\
3008.10000000004	62.6687055381956\\
3009.79999999996	62.6897903584837\\
3009.89999999999	62.687707641196\\
3010.09999999996	62.6901866985582\\
3010.20000000003	62.6881041756636\\
3010.59999999997	62.693061414289\\
3010.70000000004	62.6909791417563\\
3011.30000000001	62.6950919837949\\
3011.4	62.6930101278433\\
3011.89999999998	62.699203187251\\
3011.99999999999	62.6971216095083\\
3013.69999999997	62.7048908354901\\
3013.90000000002	62.7007299270073\\
3014.10000000004	62.7032048304691\\
3014.40000000004	62.696964670758\\
3015.39999999999	62.7093351019731\\
3015.49999999995	62.7072556041915\\
3016.19999999999	62.7026489407552\\
3016.79999999997	62.7001226424475\\
3017.70000000004	62.711246603486\\
3017.80000000004	62.7091686271911\\
3018.00000000003	62.7116397733674\\
3018.09999999996	62.7095619905904\\
3018.7	62.7169736319067\\
3018.8	62.7148961542284\\
3019.30000000004	62.7111346625157\\
3019.49999999996	62.7136044509207\\
3019.59999999996	62.7115276351956\\
3020.20000000004	62.7090024169784\\
3020.69999999996	62.7151747881356\\
3020.80000000003	62.713098745407\\
3021.09999999999	62.7168012710181\\
3021.19999999999	62.7147254493099\\
3021.40000000002	62.7171934469634\\
3021.5	62.7151178183744\\
3021.70000000002	62.7175855450394\\
3021.8	62.7155101095337\\
3022.00000000002	62.7179775652692\\
3022.1	62.7159023228112\\
3022.59999999998	62.708836470705\\
3022.80000000004	62.7113037149757\\
3023.29999999996	62.700932724747\\
3023.60000000001	62.7046333961703\\
3024.19999999999	62.6921932347981\\
3024.39999999998	62.6946602744255\\
3024.60000000004	62.6905147617946\\
3024.90000000001	62.6876033057851\\
3025.29999999999	62.692536524096\\
3025.39999999999	62.6904643860519\\
3025.90000000004	62.6966292134831\\
3026	62.6945573510459\\
3026.6	62.6920408365547\\
3026.89999999998	62.6957383548067\\
3027.09999999997	62.6915961945032\\
3027.39999999996	62.6952931461602\\
3027.50000000005	62.6932223543401\\
3027.70000000001	62.6956866371623\\
3027.80000000004	62.6936160375178\\
3028.30000000004	62.6898692378814\\
3028.69999999997	62.6947966191231\\
3029.20000000005	62.6844485524709\\
3029.4	62.6869120316884\\
3029.5	62.684842883549\\
3029.80000000004	62.6885375754975\\
3029.90000000005	62.6864686468647\\
3030.50000000004	62.6905563254801\\
3030.80000000005	62.684351182817\\
3031.10000000004	62.6880443388757\\
3031.30000000005	62.6839084251501\\
3031.6	62.6810040571297\\
3032.2	62.6784948718794\\
3032.80000000003	62.6759866794157\\
3033.89999999997	62.6829268292683\\
3033.99999999996	62.6808608813157\\
3034.30000000004	62.6779593988927\\
3034.69999999998	62.6729932779755\\
3035.30000000001	62.6803716149437\\
3035.49999999999	62.6762419291079\\
3035.7	62.6787008366823\\
3035.80000000002	62.676636252841\\
3036.39999999996	62.6840111971019\\
3036.49999999997	62.6819469143121\\
3036.80000000002	62.6856333761401\\
3036.90000000001	62.6835693118209\\
3037.10000000004	62.6860266034506\\
3037.20000000005	62.6839627300563\\
3038.19999999999	62.6764967251424\\
3039.30000000003	62.6900046061723\\
3039.60000000001	62.6838174819884\\
3039.90000000004	62.6809210526316\\
3040.69999999999	62.6841620626151\\
3040.79999999998	62.6821006938735\\
3041.29999999996	62.6849477214441\\
3041.39999999997	62.6828867335197\\
3041.60000000005	62.6853404346254\\
3041.99999999996	62.6770980572631\\
3042.40000000001	62.6820049301561\\
3042.50000000005	62.6799447840663\\
3042.80000000002	62.6770514969273\\
3043.09999999996	62.6741587802313\\
3043.59999999997	62.6802904359825\\
3043.70000000002	62.6782311584204\\
3044.00000000004	62.6819092671069\\
3044.50000000001	62.6716153189253\\
3045.1	62.669118612899\\
3045.5	62.6740215392698\\
3045.70000000003	62.6699061002036\\
3046.20000000002	62.6629025375045\\
3046.59999999997	62.6678045097975\\
3046.70000000004	62.6657476696862\\
3047.30000000004	62.6599724355188\\
3047.60000000005	62.6570856711619\\
3047.90000000002	62.6607611548556\\
3048.30000000003	62.6525390368718\\
3048.49999999995	62.6549891753592\\
3048.59999999996	62.6529340374586\\
3049.80000000004	62.6610708547821\\
3049.90000000002	62.6590163934426\\
3050.59999999996	62.6511948077491\\
3050.80000000002	62.6536431872562\\
3050.90000000005	62.6515896427401\\
3051.89999999999	62.653997378768\\
3051.99999999999	62.6519445627601\\
3052.20000000002	62.6543917701405\\
3052.30000000003	62.6523391429695\\
3052.50000000003	62.654786083994\\
3052.6	62.6527336456252\\
3052.8	62.6551803203511\\
3052.99999999999	62.6510759555861\\
3053.70000000002	62.6596371733578\\
3053.89999999997	62.6555337262606\\
3054.39999999996	62.6583728924538\\
3054.50000000001	62.6563216133045\\
3054.9	62.6612111292962\\
3055.00000000003	62.6591600929593\\
3055.2	62.6616044250974\\
3055.30000000002	62.659553577273\\
3055.89999999998	62.663612565445\\
3056.00000000002	62.6615621216583\\
3056.29999999997	62.6652270645204\\
3056.39999999996	62.6631768362506\\
3057.50000000001	62.6667974882261\\
3057.59999999998	62.6647480132125\\
3057.89999999999	62.6618705035971\\
3058.49999999996	62.6593866474858\\
3059.20000000005	62.6613931291472\\
3059.3	62.6593449696019\\
3059.99999999997	62.6678866703703\\
3060.09999999996	62.6658388340631\\
3060.8	62.674376817276\\
3061.00000000001	62.6702819247983\\
3061.20000000003	62.6727207395551\\
3061.50000000001	62.6665795662399\\
3061.69999999998	62.6690182245738\\
3061.79999999996	62.6669714882916\\
3062.59999999999	62.6767231527737\\
3062.70000000003	62.6746767663576\\
3063.30000000004	62.6819873343344\\
3063.40000000002	62.6799412436755\\
3064.09999999999	62.6852033157105\\
3064.40000000001	62.6790667319302\\
3064.60000000003	62.6815022677587\\
3064.69999999998	62.6794570608196\\
3064.90000000002	62.6818923327896\\
3064.99999999997	62.6798473133014\\
3065.29999999996	62.6834997064005\\
3065.40000000001	62.6814549013212\\
3065.69999999999	62.685106660578\\
3066.09999999995	62.6769290979062\\
3066.29999999999	62.6793634229063\\
3066.49999999996	62.6752755494685\\
3066.69999999998	62.6777096647972\\
3066.79999999995	62.6756659819362\\
3067.5	62.671143565002\\
3068.09999999998	62.6751841470569\\
3068.30000000005	62.6710989440751\\
3068.50000000001	62.6735319037998\\
3068.60000000004	62.671489555838\\
3068.99999999997	62.676354631651\\
3069.09999999995	62.6743125244363\\
3069.4	62.677960579899\\
3069.6	62.673876926084\\
3069.90000000002	62.671009771987\\
3070.20000000003	62.6746571996222\\
3070.39999999999	62.6705748249471\\
3070.70000000003	62.6742217011854\\
3070.80000000005	62.6721807939041\\
3070.99999999996	62.6746117026473\\
3071.19999999995	62.6705303942956\\
3071.60000000001	62.6656248982648\\
3071.90000000003	62.6627604166667\\
3072.19999999995	62.659896494483\\
3072.5	62.6635422768991\\
3072.60000000002	62.6615029127477\\
3073.09999999997	62.6675777691006\\
3073.20000000005	62.665538671786\\
3073.9	62.6740403383214\\
3074.09999999998	62.6699629171817\\
3074.30000000002	62.672391360916\\
3074.39999999998	62.670352902911\\
3075.49999999998	62.6641956041098\\
3075.70000000003	62.6666233175109\\
3075.79999999998	62.6645859748366\\
3076.10000000002	62.6682270333528\\
3076.19999999997	62.6661899034554\\
3076.50000000004	62.6698303321849\\
3076.70000000005	62.6657566302652\\
3077.09999999996	62.6706096451319\\
3077.39999999997	62.6645004061738\\
3077.99999999998	62.6620317728469\\
3078.39999999998	62.657138216664\\
3079.10000000005	62.6493894518057\\
3079.90000000005	62.6396103896104\\
3080.20000000003	62.6367561601143\\
3081.29999999999	62.6500941130655\\
3081.39999999996	62.6480610092487\\
3081.90000000005	62.6541207008436\\
3082.00000000003	62.6520878621719\\
3082.70000000002	62.6605683145193\\
3082.99999999997	62.654471149168\\
3083.69999999997	62.6434917958363\\
3084.00000000004	62.6471255795856\\
3084.20000000001	62.6430632558441\\
3084.49999999999	62.6402126693899\\
3085.19999999998	62.648688944349\\
3085.30000000002	62.6466584559539\\
3085.59999999996	62.6438085361506\\
3085.89999999996	62.6474400518471\\
3086.09999999997	62.6433802086709\\
3086.30000000003	62.645800933126\\
3086.39999999995	62.6437712619472\\
3086.90000000003	62.6368642695173\\
3087.10000000002	62.6392847888054\\
3087.20000000003	62.6372558546303\\
3087.50000000005	62.6408861251458\\
3087.6	62.6388574019497\\
3088.20000000004	62.6428779587475\\
3088.29999999999	62.6408496308768\\
3088.50000000001	62.6432687949233\\
3088.59999999996	62.6412406514067\\
3089.30000000002	62.6464685699489\\
3089.49999999996	62.6424132573796\\
3090.10000000001	62.6464306517378\\
3090.19999999998	62.6444034559752\\
3090.4	62.6468209027665\\
3090.69999999999	62.640740261421\\
3091.09999999999	62.6455745341615\\
3091.19999999996	62.6435480218678\\
3091.50000000001	62.6407038426705\\
3092.10000000003	62.6479529137831\\
3092.30000000004	62.6439011770793\\
3092.59999999996	62.6475248165034\\
3092.69999999998	62.6454992240041\\
3092.90000000001	62.6479146459748\\
3093.1	62.6438639596534\\
3093.89999999999	62.6470588235294\\
3093.99999999999	62.6450340971526\\
3094.70000000001	62.6534832622464\\
3094.8	62.6514588516592\\
3095.70000000003	62.6558563214678\\
3095.89999999999	62.6518087855297\\
3096.20000000003	62.6489681232439\\
3096.49999999996	62.6525867080023\\
3096.60000000004	62.6505635030839\\
3097.49999999998	62.6614152892562\\
3097.69999999997	62.6573697462716\\
3098.30000000001	62.6516911954557\\
3098.79999999999	62.6544903030107\\
3099.30000000005	62.6443827837646\\
3099.60000000002	62.6479981933736\\
3099.69999999995	62.6459771598168\\
3100.00000000004	62.6495919486468\\
3100.10000000001	62.6475711244436\\
3101.30000000003	62.6265557490166\\
3101.49999999996	62.6289656951251\\
3101.60000000002	62.6269465132024\\
3101.99999999998	62.6220947100351\\
3102.40000000005	62.6172441579372\\
3102.60000000004	62.6196538498727\\
3102.70000000005	62.6176356838984\\
3103.09999999999	62.6224542407837\\
3103.30000000003	62.6184185087324\\
3103.89999999995	62.6224226804124\\
3103.99999999998	62.6204052704488\\
3104.29999999999	62.624017523515\\
3104.49999999995	62.6199832506603\\
3105.30000000004	62.607071552779\\
3106.69999999999	62.6174842281447\\
3106.89999999995	62.6134534921146\\
3107.60000000002	62.6186568845127\\
3107.89999999999	62.6126126126126\\
3108.09999999997	62.6150183385882\\
3108.20000000003	62.6130038928031\\
3108.49999999995	62.6166119796693\\
3108.7	62.6125836335564\\
3108.90000000005	62.6149887423609\\
3109.00000000005	62.6129748158631\\
3109.59999999995	62.610541209763\\
3110.49999999998	62.6149295955764\\
3110.60000000003	62.6129167068506\\
3111.39999999997	62.6161015587337\\
3111.49999999997	62.614089214552\\
3111.69999999997	62.6164920624719\\
3112.00000000004	62.610455962212\\
3112.2	62.6128586575844\\
3112.30000000001	62.6108469348413\\
3112.79999999999	62.6040026984484\\
3113.20000000005	62.6088073748113\\
3113.49999999997	62.6027749229188\\
3114.10000000005	62.6099800911952\\
3114.19999999996	62.6079696882124\\
3114.49999999997	62.605149938997\\
3115.19999999996	62.6103425031297\\
3115.29999999996	62.6083327983566\\
3115.49999999997	62.61073308512\\
3115.70000000003	62.6067141665062\\
3116.00000000004	62.6038958955104\\
3116.50000000004	62.6098953988321\\
3116.59999999999	62.607886546668\\
3117.10000000002	62.6106762479148\\
3117.20000000003	62.6086677573541\\
3117.69999999995	62.6146641862852\\
3117.80000000005	62.6126559543282\\
3118.49999999998	62.6146347720131\\
3118.79999999997	62.6086120106448\\
3118.99999999997	62.6110095860986\\
3119.10000000004	62.6090023082842\\
3119.80000000002	62.6141863521267\\
3119.90000000002	62.6121794871795\\
3120.30000000002	62.6169721830535\\
3120.40000000001	62.6149655503926\\
3120.70000000001	62.6121507305819\\
3121.20000000003	62.6181398776151\\
3121.39999999997	62.614127823162\\
3121.89999999998	62.6169122357463\\
3122.1	62.6129011594389\\
3122.59999999996	62.6188875012009\\
3122.79999999999	62.6148771974767\\
3123.00000000005	62.617271300951\\
3123.30000000002	62.6112569635654\\
3123.70000000004	62.6160445611115\\
3123.79999999997	62.61404014213\\
3125.10000000003	62.6167925252784\\
3125.19999999999	62.6147889802579\\
3125.90000000005	62.6071657069738\\
3126.30000000004	62.6119498464688\\
3126.49999999995	62.6079447322971\\
3126.90000000001	62.6127278541733\\
3127.09999999995	62.6087234586851\\
3128.49999999999	62.6062775682414\\
3128.70000000002	62.6086678598824\\
3128.80000000002	62.6066668797341\\
3129.39999999996	62.6138360760505\\
3129.59999999995	62.6098348084481\\
3130.29999999998	62.6118067978533\\
3130.40000000005	62.6098067401374\\
3130.89999999998	62.6125838390291\\
3131.20000000003	62.6065851243892\\
3131.59999999998	62.6113612414982\\
3131.79999999996	62.6073629426227\\
3132.1	62.6109443841389\\
3132.20000000001	62.6089455033043\\
3132.79999999997	62.6033387596157\\
3133.10000000002	62.6005361930295\\
3133.79999999997	62.6088898816172\\
3134.00000000001	62.6048945470789\\
3134.4	62.6000957090445\\
3134.79999999996	62.6048677788765\\
3134.89999999999	62.6028708133971\\
3135.69999999998	62.6092225269469\\
3135.90000000003	62.6052295918367\\
3136.19999999998	62.6088065554953\\
3136.29999999998	62.606810355822\\
3136.80000000001	62.6127705696707\\
3137.09999999999	62.6067831187046\\
3137.40000000002	62.6103585657371\\
3137.6	62.6063677215795\\
3138.70000000005	62.6194724098382\\
3138.99999999996	62.6134879424039\\
3139.40000000002	62.6182513139035\\
3139.50000000003	62.6162568480061\\
3139.89999999999	62.6114649681529\\
3140.49999999999	62.6154238043686\\
3140.60000000003	62.6134301270418\\
3141.19999999999	62.6173877057269\\
3141.40000000005	62.6134012414452\\
3142.09999999998	62.6185475144803\\
3142.20000000002	62.616554752888\\
3142.79999999998	62.6077826211461\\
3143.90000000004	62.6145038167939\\
3144.00000000004	62.6125123246716\\
3144.49999999999	62.618457037461\\
3144.60000000001	62.6164657995993\\
3144.79999999996	62.6188432064612\\
3145.69999999997	62.600928221756\\
3146.39999999998	62.6028921023359\\
3146.70000000002	62.5969238591585\\
3147.00000000005	62.6004893393918\\
3147.30000000005	62.5945224629853\\
3147.80000000002	62.5972870802757\\
3148.10000000003	62.5913220252843\\
3148.40000000002	62.5885342226457\\
3148.99999999998	62.5924867422438\\
3149.2	62.588511732766\\
3149.89999999998	62.5936507936508\\
3149.99999999995	62.5916637567061\\
3150.30000000005	62.5888776028441\\
3150.70000000005	62.5936270153612\\
3150.80000000001	62.5916404836713\\
3151.30000000001	62.5944024877832\\
3151.50000000005	62.5904302576469\\
3152.09999999995	62.5880337542034\\
3152.39999999997	62.5915939730373\\
3152.49999999995	62.5896085770475\\
3153.10000000002	62.5840416085247\\
3153.39999999998	62.5876010781671\\
3153.60000000005	62.5836319244063\\
3154.10000000002	62.5800519941665\\
3154.30000000001	62.5824245498352\\
3154.40000000003	62.5804406403551\\
3155.2	62.589928057554\\
3155.30000000005	62.5879444761362\\
3155.69999999996	62.5926864820331\\
3155.8	62.5907031274755\\
3155.99999999998	62.5930737302367\\
3156.20000000005	62.5891074992872\\
3156.69999999996	62.5823618854536\\
3157.30000000004	62.5894723506683\\
3157.69999999997	62.5815441129901\\
3158.09999999999	62.5767842441897\\
3158.40000000001	62.5803387684027\\
3158.69999999998	62.5743953400025\\
3159.50000000001	62.580706418534\\
3159.69999999996	62.5767453636306\\
3160.30000000004	62.5711935198076\\
3160.50000000002	62.5735619819022\\
3160.59999999997	62.5715822444395\\
3161.1	62.577502214349\\
3161.40000000002	62.5715641309505\\
3161.80000000003	62.5762990606914\\
3161.89999999996	62.5743200506009\\
3162.09999999996	62.5766871165644\\
3162.20000000003	62.5747082819467\\
3162.50000000003	62.5782583949915\\
3162.60000000004	62.5762797609637\\
3162.80000000004	62.5786461791394\\
3162.90000000003	62.5766677205185\\
3163.40000000004	62.5825825825826\\
3163.49999999998	62.5806043747629\\
3164.00000000002	62.5833570367561\\
3164.29999999998	62.5774238402225\\
3164.59999999997	62.5746516257465\\
3164.80000000002	62.577016651395\\
3164.99999999999	62.5730624624814\\
3165.50000000001	62.5694970937579\\
3166.00000000005	62.5659328511418\\
3166.79999999996	62.5753891818498\\
3167.39999999996	62.5635359116022\\
3168.00000000003	62.5674694611913\\
3168.10000000001	62.5654946026135\\
3168.39999999998	62.5690389774341\\
3168.60000000004	62.5650897844542\\
3169.69999999997	62.5717710896587\\
3169.80000000003	62.5697971544844\\
3170.29999999996	62.5662376987131\\
3170.89999999998	62.5733207190161\\
3171.00000000002	62.5713474819463\\
3171.30000000001	62.5748880620546\\
3171.40000000005	62.5729150244364\\
3171.70000000002	62.5764550097736\\
3171.79999999999	62.5744821715691\\
3172.49999999995	62.5827397087562\\
3172.59999999999	62.5807671699184\\
3172.89999999996	62.5780018909549\\
3173.09999999997	62.5803605193496\\
3173.19999999996	62.5783884284499\\
3173.50000000002	62.5819258885808\\
3173.70000000002	62.5779822295041\\
3174.49999999998	62.5842625842626\\
3174.70000000003	62.5803200201588\\
3174.9	62.5826771653543\\
3174.99999999996	62.5807061194923\\
3175.39999999995	62.5854196189577\\
3175.49999999998	62.5834487970777\\
3175.79999999996	62.5869832173557\\
3175.90000000005	62.5850125944584\\
3176.90000000004	62.5873465533522\\
3177.00000000001	62.5853766013031\\
3177.29999999995	62.5826147164348\\
3178.30000000003	62.5943871130128\\
3178.50000000003	62.5904486251809\\
3178.69999999997	62.5928023153391\\
3178.90000000003	62.5888644227745\\
3179.40000000004	62.5916024532159\\
3179.70000000002	62.5856972136612\\
3180.20000000004	62.5821463383958\\
3180.39999999998	62.5844992925641\\
3180.49999999997	62.5825315978117\\
3181.19999999996	62.590764781693\\
3181.29999999999	62.5887973847992\\
3181.49999999997	62.5911491073674\\
3181.60000000004	62.5891818838985\\
3181.8	62.5915333605707\\
3181.90000000001	62.5895663104966\\
3182.40000000005	62.5954438334643\\
3182.5	62.593477031358\\
3182.79999999998	62.5970027333564\\
3183.1	62.5911032922845\\
3183.59999999996	62.5969783585137\\
3183.70000000005	62.5950122495132\\
3184.50000000001	62.5981284933744\\
3184.59999999995	62.5961629038842\\
3184.90000000004	62.5996860282575\\
3184.99999999998	62.5977206367147\\
3185.70000000001	62.5933831376734\\
3186.20000000004	62.5992530521294\\
3186.29999999997	62.5972884760231\\
3186.60000000002	62.6008096149622\\
3186.69999999999	62.598845236601\\
3187.10000000002	62.6035391566265\\
3187.39999999998	62.5976470588235\\
3187.90000000005	62.5909661229611\\
3188.10000000003	62.5933128411016\\
3188.29999999999	62.5893865261573\\
3189.30000000001	62.5823038816078\\
3190.49999999998	62.5901084435529\\
3190.59999999996	62.5881468016423\\
3190.80000000002	62.5904917108026\\
3190.90000000002	62.5885302413037\\
3191.10000000001	62.5908749059915\\
3191.20000000003	62.5889136088741\\
3191.80000000003	62.5959459882828\\
3192.09999999996	62.5900632792432\\
3192.29999999996	62.5924069665456\\
3192.50000000001	62.5884858735827\\
3192.79999999998	62.5920010022237\\
3192.89999999997	62.590040714062\\
3193.99999999996	62.5966625966626\\
3194.10000000001	62.5947028990045\\
3194.60000000003	62.588036435346\\
3195.50000000004	62.5923144323445\\
3195.59999999996	62.5903557905936\\
3195.90000000003	62.5876095118899\\
3196.10000000004	62.5899505662975\\
3196.19999999999	62.5879923661734\\
3196.39999999996	62.5903331769123\\
3196.5	62.5883751485954\\
3197.10000000005	62.595395971475\\
3197.19999999998	62.5934382134926\\
3198.20000000005	62.5863740111935\\
3198.49999999996	62.583630338273\\
3198.80000000002	62.5871393291444\\
3199.00000000001	62.5832265324623\\
3199.29999999999	62.5804838407201\\
3200	62.5824192993969\\
3200.19999999996	62.5785082648502\\
3200.5	62.5757670436793\\
3201.09999999999	62.5827814569536\\
3201.19999999996	62.5808265392184\\
3201.39999999997	62.5831641418085\\
3201.50000000003	62.5812093953024\\
3201.90000000002	62.5858838226109\\
3202.09999999996	62.5819748922616\\
3202.39999999997	62.5792349726776\\
3202.69999999998	62.5827401024104\\
3202.89999999995	62.5788323446769\\
3203.69999999996	62.5881765403583\\
3204.09999999999	62.5803632732039\\
3204.30000000004	62.5826987891649\\
3204.39999999998	62.5807458261819\\
3204.7	62.5842486270594\\
3204.90000000003	62.5803432137286\\
3205.19999999998	62.5838455058809\\
3205.29999999995	62.5818930554689\\
3205.49999999999	62.584227601697\\
3205.59999999997	62.5822753220825\\
3206.20000000003	62.5892773601971\\
3206.30000000003	62.5873253493014\\
3206.50000000005	62.5896588286659\\
3206.80000000002	62.5838036733294\\
3207.10000000005	62.5810675979047\\
3207.40000000001	62.5845674201091\\
3207.49999999997	62.58261628632\\
3207.90000000003	62.5872817955112\\
3208.1	62.5833800885232\\
3208.29999999997	62.5857125046752\\
3208.40000000004	62.5837618824996\\
3209.30000000002	62.5724434473733\\
3210.69999999999	62.5887629251277\\
3210.89999999996	62.5848645281844\\
3211.49999999998	62.5918545273384\\
3211.59999999999	62.58990565744\\
3212.40000000001	62.5961089494163\\
3212.70000000001	62.5902639442231\\
3213.30000000005	62.5816891765731\\
3213.79999999997	62.5875105012602\\
3213.89999999999	62.5855631611699\\
3214.10000000002	62.5878912326551\\
3214.30000000004	62.5839970134395\\
3214.80000000003	62.5898161684656\\
3214.89999999996	62.5878693623639\\
3216.10000000002	62.5987189851377\\
3216.20000000005	62.5967726891148\\
3216.40000000001	62.5990983988808\\
3216.60000000003	62.5952062672926\\
3216.90000000005	62.5986944358098\\
3217.00000000001	62.5967486245376\\
3217.59999999995	62.5943997265127\\
3218.09999999997	62.5908893170095\\
3218.49999999999	62.5862176101411\\
3219.00000000004	62.5889223696064\\
3219.20000000003	62.5850340136055\\
3219.70000000002	62.581526802907\\
3219.99999999995	62.5788019005621\\
3220.69999999998	62.5838301043219\\
3220.89999999996	62.5799441167339\\
3221.29999999997	62.5752778295151\\
3221.89999999998	62.5822470515208\\
3222	62.5803047701809\\
3222.20000000005	62.5826273158924\\
3222.30000000002	62.580685203575\\
3222.79999999996	62.577182041019\\
3223.00000000002	62.5795042040272\\
3223.10000000005	62.5775626706379\\
3224.00000000002	62.5818057752551\\
3224.09999999996	62.5798647726568\\
3224.39999999996	62.5833462552334\\
3224.5	62.5814054456367\\
3225.10000000004	62.5883666129232\\
3225.20000000004	62.5864260688928\\
3226.59999999998	62.5964607803638\\
3226.69999999996	62.5945208875666\\
3226.90000000003	62.5968391695073\\
3227.00000000005	62.5948994453224\\
3227.89999999997	62.5898389095415\\
3228.50000000005	62.5844019079477\\
3228.89999999996	62.589036853515\\
3228.99999999998	62.5870985723576\\
3229.19999999998	62.5894156628372\\
3229.3	62.5874775500093\\
3229.89999999996	62.5913312693499\\
3230.00000000004	62.5893935172286\\
3230.30000000004	62.592867756315\\
3230.7	62.5851182369692\\
3231.10000000003	62.5897499381035\\
3231.29999999996	62.5858760908584\\
3231.90000000003	62.5835396039604\\
3232.49999999999	62.5873909546495\\
3232.79999999998	62.5815830987658\\
3233.2	62.5769337828225\\
3233.90000000005	62.5819418676562\\
3234.00000000003	62.5800068025108\\
3234.19999999999	62.5823207494667\\
3234.30000000001	62.58038585209\\
3234.50000000001	62.5826995609967\\
3234.60000000004	62.5807648313599\\
3234.80000000004	62.5830783022659\\
3234.99999999995	62.5792092980124\\
3235.5	62.581901347509\\
3235.60000000002	62.5799672404735\\
3236.30000000001	62.5726115436905\\
3236.50000000004	62.5749243032812\\
3236.70000000002	62.5710578348987\\
3236.90000000001	62.5733704046957\\
3237.09999999998	62.5695045100704\\
3238.1	62.5810635538262\\
3238.30000000003	62.5771986166008\\
3238.50000000003	62.5795096646699\\
3238.60000000004	62.5775774230401\\
3238.99999999999	62.5729369269242\\
3239.59999999997	62.5798685063432\\
3239.70000000003	62.5779369096858\\
3240.1	62.5825566323066\\
3240.20000000003	62.5806252507484\\
3241.40000000002	62.5944778651859\\
3241.50000000004	62.5925468904245\\
3241.90000000005	62.5971622455275\\
3241.99999999998	62.5952314857654\\
3242.29999999999	62.5925240562546\\
3242.59999999996	62.5959848274586\\
3242.69999999999	62.5940545207845\\
3243.09999999998	62.5986679822398\\
3243.20000000005	62.5967378904203\\
3243.40000000003	62.5990442423308\\
3243.49999999998	62.5971143174251\\
3244.10000000003	62.6009493865976\\
3244.19999999995	62.5990198193755\\
3244.50000000001	62.5963138753621\\
3244.70000000003	62.5986193293886\\
3244.79999999995	62.5966901907609\\
3245.40000000002	62.5943614235095\\
3245.80000000002	62.5989710095813\\
3245.99999999999	62.5951141369644\\
3246.39999999996	62.5997227783767\\
3246.49999999999	62.5977946159059\\
3246.79999999997	62.6012504234809\\
3247.19999999999	62.5935392479906\\
3247.69999999996	62.5869819570171\\
3248.30000000002	62.5785001847063\\
3248.79999999997	62.5719474283604\\
3249.70000000005	62.5761585328328\\
3249.90000000004	62.5723076923077\\
3250.19999999995	62.5757622373319\\
3250.5	62.5699870793084\\
3250.70000000001	62.5722898978713\\
3250.79999999997	62.5703651296564\\
3251.29999999996	62.5761210555453\\
3251.39999999997	62.5741965246809\\
3252.09999999995	62.5822520140213\\
3252.19999999997	62.5803277680411\\
3252.79999999999	62.5841556764733\\
3252.89999999996	62.5822317860436\\
3253.50000000004	62.5891320383575\\
3253.59999999996	62.5872084088883\\
3253.90000000001	62.5845113706208\\
3254.59999999997	62.5925584539282\\
3254.70000000004	62.5906353693007\\
3254.99999999999	62.5940831310866\\
3255.19999999997	62.5902374589132\\
3255.69999999995	62.5867682290067\\
3256.20000000004	62.5925129748488\\
3256.50000000004	62.5867469139593\\
3256.69999999997	62.5890444608204\\
3256.80000000003	62.5871227240628\\
3257.30000000003	62.5928654755326\\
3257.39999999997	62.5909439754413\\
3257.80000000002	62.5955370023635\\
3257.89999999997	62.5936157151627\\
3258.19999999998	62.5970598164687\\
3258.30000000002	62.5951387183894\\
3258.70000000004	62.5905241193077\\
3258.99999999999	62.5878309962873\\
3259.29999999996	62.585138369025\\
3259.59999999997	62.5824462373838\\
3259.80000000005	62.5847418632473\\
3259.89999999996	62.5828220858896\\
3260.30000000004	62.5874125874126\\
3260.39999999998	62.5854930225426\\
3260.69999999999	62.5889352306182\\
3260.90000000003	62.5850965961362\\
3261.10000000004	62.587391144364\\
3261.20000000003	62.5854720510226\\
3261.39999999998	62.5877663651694\\
3261.70000000004	62.5820099331657\\
3262.29999999995	62.588891613536\\
3262.39999999996	62.5869731800766\\
3262.89999999996	62.5835121054245\\
3263.60000000001	62.5915372123663\\
3263.70000000005	62.5896194619769\\
3263.99999999997	62.586930547471\\
3264.90000000005	62.5911179173048\\
3265.49999999998	62.5796178343949\\
3265.80000000005	62.5830552068343\\
3265.9	62.5811390079608\\
3266.09999999999	62.5834302859592\\
3266.19999999998	62.5815142515997\\
3266.69999999998	62.5841802375413\\
3266.80000000003	62.5822645321253\\
3267.1	62.5857002938296\\
3267.29999999997	62.5818693762625\\
3267.89999999999	62.5795593635251\\
3268.30000000004	62.5841390282707\\
3268.50000000001	62.5803096126782\\
3268.79999999996	62.5837437670164\\
3268.90000000004	62.581829305598\\
3269.39999999996	62.5844930417495\\
3269.5	62.582578908735\\
3269.89999999997	62.5871559633028\\
3270.00000000004	62.5852420415278\\
3270.4	62.5806451612903\\
3271.10000000005	62.5886524822695\\
3271.19999999999	62.5867392168251\\
3271.89999999998	62.588630806846\\
3272	62.5867180098408\\
3272.20000000001	62.5890046756104\\
3272.29999999995	62.5870920425376\\
3272.79999999996	62.5836414189251\\
3273.09999999997	62.5870707564463\\
3273.19999999996	62.5851587083372\\
3273.39999999995	62.5874446311288\\
3273.70000000001	62.581709328609\\
3274.00000000003	62.5851379004917\\
3274.19999999998	62.5813150902483\\
3274.39999999999	62.5836005497023\\
3274.80000000001	62.5759565177563\\
3275.00000000003	62.5782418857439\\
3275.10000000001	62.5763312164143\\
3275.40000000005	62.579758815448\\
3275.49999999998	62.5778483331298\\
3276.59999999998	62.5904110843226\\
3276.69999999996	62.5885009765625\\
3277	62.5858228311617\\
3277.79999999997	62.5919033527563\\
3277.89999999998	62.5899938987187\\
3278.20000000001	62.5934173199524\\
3278.3	62.5915080527086\\
3278.49999999999	62.5937900323309\\
3278.60000000003	62.5918809284168\\
3278.89999999997	62.5892040256176\\
3279.49999999995	62.586900841566\\
3279.70000000003	62.5891822672114\\
3279.80000000005	62.5872740022562\\
3280.10000000002	62.5845985000915\\
3280.29999999997	62.5868796488233\\
3280.40000000002	62.5849718030788\\
3281.1	62.5929537973912\\
3281.19999999998	62.5910462316765\\
3281.39999999998	62.593326222764\\
3281.50000000004	62.5914188200878\\
3282.20000000004	62.5872101879779\\
3282.39999999996	62.5894897182026\\
3282.49999999999	62.5875830134649\\
3282.90000000005	62.5921413341456\\
3283.00000000003	62.5902348390241\\
3283.3	62.5936529207529\\
3283.49999999997	62.5898404190523\\
3283.70000000001	62.5921188866557\\
3283.90000000001	62.5883069427527\\
3284.40000000002	62.5940021312224\\
3284.59999999995	62.5901908850123\\
3284.79999999996	62.5924685682974\\
3284.90000000002	62.5905631659056\\
3285.49999999995	62.5973946919893\\
3285.80000000003	62.5916796007182\\
3286.00000000004	62.5939563616445\\
3286.19999999995	62.5901469738003\\
3286.40000000001	62.59242355089\\
3286.70000000002	62.5867104782767\\
3286.99999999997	62.5901250342247\\
3287.19999999997	62.5863170382989\\
3287.89999999995	62.5942822384428\\
3287.99999999997	62.5923785772939\\
3289.00000000001	62.5976710954365\\
3289.09999999995	62.5957679678949\\
3289.30000000002	62.5980421961452\\
3289.39999999998	62.5961392308862\\
3289.60000000004	62.5984132291698\\
3289.80000000004	62.5946077388371\\
3290.39999999996	62.5923111989059\\
3290.6	62.5945847388094\\
3290.69999999996	62.5926826303634\\
3291.80000000001	62.5991068987515\\
3292	62.5953039093588\\
3292.30000000004	62.5926375896003\\
3292.59999999997	62.589971755702\\
3293.49999999996	62.5880495506437\\
3294.10000000001	62.5857567846518\\
3294.49999999996	62.5902992776058\\
3294.6	62.5883995507937\\
3294.89999999998	62.5918057663126\\
3295.20000000003	62.58610748642\\
3295.40000000005	62.5883780913367\\
3295.69999999996	62.582680987924\\
3296.30000000001	62.5894915665575\\
3296.60000000002	62.5837959171293\\
3296.89999999998	62.5872004852897\\
3296.99999999998	62.5853022352977\\
3297.80000000001	62.5792170775342\\
3298.00000000005	62.5814863102999\\
3298.10000000002	62.5795888666545\\
3298.50000000002	62.5841265991633\\
3298.70000000005	62.5803322420274\\
3300.29999999996	62.5863531693128\\
3300.39999999997	62.5844569004696\\
3301.09999999996	62.5863322428208\\
3301.29999999998	62.582540740292\\
3301.60000000001	62.5859405760669\\
3301.7	62.5840450663275\\
3302.40000000002	62.5919757759273\\
3302.49999999995	62.5900805426028\\
3303.19999999996	62.5919535010444\\
3303.29999999998	62.5900587273718\\
3303.80000000003	62.5957202094494\\
3303.89999999995	62.5938256658596\\
3304.70000000001	62.5968288549988\\
3304.79999999997	62.594934793791\\
3305.00000000004	62.5971982693413\\
3305.30000000002	62.5915169117202\\
3305.50000000004	62.5937802516941\\
3305.6	62.5918867410836\\
3305.90000000001	62.5952813067151\\
3306.00000000001	62.5933879797949\\
3306.40000000001	62.5979132012702\\
3306.5	62.59602008105\\
3306.90000000002	62.6005442999698\\
3306.99999999998	62.598651386411\\
3307.29999999996	62.6020439015541\\
3307.70000000005	62.5944736682992\\
3307.90000000004	62.5967351874244\\
3308.40000000002	62.5872752002418\\
3308.79999999998	62.591797878449\\
3309.00000000003	62.588014868091\\
3309.30000000005	62.591406297214\\
3309.4	62.5895150324823\\
3309.69999999995	62.5868632545773\\
3310.19999999998	62.5834516509078\\
3310.79999999998	62.5872119363315\\
3310.89999999995	62.5853216550891\\
3311.69999999999	62.5943595627755\\
3311.8	62.5924695793955\\
3312.00000000001	62.5947284200356\\
3312.19999999997	62.59094888748\\
3312.49999999999	62.5943367747389\\
3312.79999999998	62.5886685381388\\
3313.09999999999	62.5920560183508\\
3313.30000000005	62.5882779018531\\
3313.60000000002	62.5916649062981\\
3313.90000000002	62.5859987929994\\
3314.29999999996	62.5814627081825\\
3314.49999999995	62.5837205092621\\
3314.7	62.579944491372\\
3315.29999999995	62.5837003076552\\
3315.40000000003	62.581812697934\\
3315.60000000001	62.5840697288657\\
3316.20000000004	62.57274673582\\
3316.39999999996	62.5750037690336\\
3316.49999999995	62.5731170475788\\
3317.69999999999	62.577611670384\\
3317.89999999998	62.5738396624473\\
3318.39999999999	62.5794786801266\\
3318.49999999996	62.5775929608871\\
3318.79999999996	62.5809756244539\\
3318.89999999998	62.5790900873757\\
3319.09999999996	62.5813449023861\\
3319.2	62.5794595245986\\
3319.5	62.5768164839137\\
3319.89999999997	62.5813253012048\\
3319.99999999996	62.5794403783019\\
3320.59999999995	62.5831902912037\\
3320.69999999997	62.5813057094676\\
3321.40000000004	62.5831702544031\\
3321.6	62.5794021133757\\
3321.8	62.5816550769138\\
3322	62.5778874808103\\
3322.19999999996	62.5801402642748\\
3322.50000000001	62.5744898573406\\
3322.80000000005	62.5778687291222\\
3322.90000000002	62.5759855552212\\
3323.50000000003	62.5737152485257\\
3323.90000000005	62.5782190132371\\
3324.4	62.5688073394495\\
3325.20000000002	62.562776290861\\
3325.39999999997	62.5650278153661\\
3325.49999999999	62.5631464998797\\
3325.69999999999	62.5653977990258\\
3325.90000000002	62.5616355983163\\
3326.09999999996	62.5638867175756\\
3326.19999999998	62.5620058323062\\
3326.39999999998	62.5642567262889\\
3326.50000000004	62.562375999519\\
3328.30000000004	62.573608941233\\
3328.50000000005	62.5698491858439\\
3328.70000000001	62.5720980533526\\
3328.80000000002	62.5702183904593\\
3329.50000000002	62.5750840941855\\
3329.60000000001	62.5732047932246\\
3330.10000000004	62.5698156266891\\
3330.40000000004	62.5731872091278\\
3330.90000000001	62.5637946562594\\
3331.59999999996	62.5686586427349\\
3332.10000000001	62.5592701518516\\
3332.30000000003	62.5615172248229\\
3332.39999999999	62.5596399099775\\
3332.90000000001	62.5532553255326\\
3333.3	62.5577488450231\\
3333.49999999999	62.5539956803456\\
3333.89999999998	62.5584883023395\\
3334.10000000003	62.5547357687001\\
3334.29999999999	62.5569817658349\\
3334.39999999997	62.5551057130004\\
3334.69999999999	62.5524769101595\\
3335.20000000004	62.5550924954277\\
3335.60000000001	62.5475912102407\\
3335.80000000005	62.5498366257981\\
3335.90000000005	62.5479616306955\\
3336.59999999999	62.5528216501334\\
3336.89999999996	62.5471980821097\\
3337.19999999996	62.5445719593684\\
3337.6	62.5490607304431\\
3337.69999999999	62.5471867697286\\
3337.99999999998	62.5445612773734\\
3338.49999999999	62.5411849278141\\
3338.70000000002	62.5434287768061\\
3338.80000000002	62.5415556021444\\
3339.09999999995	62.5449209391471\\
3339.20000000004	62.5430479441799\\
3339.39999999998	62.5452912112592\\
3339.59999999995	62.5415456478127\\
3340.19999999996	62.5482741071161\\
3340.29999999995	62.5464016285475\\
3340.60000000002	62.5497650193073\\
3340.69999999995	62.5478927203065\\
3341.09999999997	62.5523763917156\\
3341.20000000002	62.5505042947356\\
3341.69999999998	62.54413788976\\
3342.00000000003	62.5475000748033\\
3342.09999999996	62.5456286278499\\
3342.29999999995	62.5478697941599\\
3342.39999999998	62.5459985041137\\
3342.99999999998	62.5437468218121\\
3344	62.5489668371161\\
3344.1	62.5470964655224\\
3344.99999999997	62.5452153896745\\
3345.29999999998	62.5485741615352\\
3345.39999999995	62.5467045284711\\
3345.60000000003	62.5489434199121\\
3345.7	62.5470739434515\\
3346.09999999998	62.5425856195087\\
3346.69999999999	62.5373491095972\\
3347.00000000006	62.5407068805832\\
3347.09999999999	62.5388384321224\\
3347.59999999999	62.544433491651\\
3347.79999999997	62.5406971534395\\
3348.10000000005	62.5380801624754\\
3348.39999999996	62.5414364640884\\
3348.60000000001	62.5377011974796\\
3349.00000000004	62.5421755098385\\
3349.3	62.5365737146952\\
3349.50000000001	62.5388106042513\\
3349.60000000003	62.5369436068902\\
3350.30000000002	62.544770773639\\
3350.40000000003	62.5429040441725\\
3350.60000000004	62.5451398215298\\
3350.70000000001	62.5432732481795\\
3350.99999999997	62.546626480857\\
3351.2	62.542893802405\\
3352.30000000006	62.5343037823649\\
3352.69999999997	62.5387735623956\\
3352.79999999999	62.5369083479973\\
3353.29999999999	62.5424941850063\\
3353.40000000001	62.5406291933801\\
3353.70000000002	62.5380165782098\\
3354.39999999998	62.5458339543896\\
3354.50000000004	62.5439694747511\\
3354.69999999997	62.5462024561822\\
3354.80000000004	62.5443381322841\\
3355.50000000002	62.546191441173\\
3355.6	62.5443275620586\\
3357.29999999999	62.5632930243641\\
3357.69999999995	62.5558401334207\\
3357.89999999999	62.5580702799285\\
3358.00000000005	62.5562073791727\\
3358.50000000002	62.5588042636813\\
3358.80000000004	62.5532168269374\\
3359.3	62.5558135381318\\
3359.39999999999	62.5539514808751\\
3359.70000000002	62.5572950770879\\
3359.80000000005	62.5554331974166\\
3360.10000000001	62.5587762633177\\
3360.19999999999	62.5569145611999\\
3360.39999999997	62.5591429846749\\
3360.49999999995	62.5572814378385\\
3360.80000000002	62.5606236424767\\
3360.90000000005	62.558762273133\\
3361.19999999997	62.5561538690388\\
3361.79999999996	62.5539129658824\\
3362.10000000003	62.5572541788115\\
3362.20000000004	62.5553936293609\\
3362.80000000004	62.5531535282048\\
3363.1	62.5564938154139\\
3363.19999999999	62.5546338417626\\
3363.69999999995	62.5572269457162\\
3363.9	62.5535077288942\\
3364.19999999999	62.5509021193116\\
3364.99999999999	62.5538616980179\\
3365.09999999998	62.5520028527279\\
3365.30000000005	62.5542283235277\\
3365.4	62.5523696330412\\
3366.59999999997	62.5568063682538\\
3366.89999999995	62.5512325512326\\
3367.40000000001	62.5478841870824\\
3367.59999999998	62.5501083825756\\
3367.90000000004	62.5445368171021\\
3368.19999999997	62.5478728141793\\
3368.40000000003	62.5441591212706\\
3368.89999999999	62.546749777382\\
3369.00000000005	62.5448932949452\\
3369.40000000005	62.5493396646387\\
3369.49999999999	62.5474833808167\\
3369.9	62.5519287833828\\
3369.99999999995	62.5500726981395\\
3371.5	62.560801993119\\
3371.59999999997	62.5589465254916\\
3372.00000000004	62.5633877998873\\
3372.09999999995	62.5615325306921\\
3372.80000000001	62.5574431498117\\
3373.39999999998	62.5552097228398\\
3373.89999999997	62.5607587433314\\
3373.99999999998	62.5589045967814\\
3374.20000000004	62.5611237886376\\
3374.29999999996	62.5592697961119\\
3374.60000000004	62.5566717041515\\
3375.30000000002	62.5614741956509\\
3375.7	62.5540612595533\\
3376.70000000006	62.5621890547264\\
3376.90000000003	62.5584838614155\\
3377.10000000003	62.560701172569\\
3377.19999999997	62.558848784532\\
3377.69999999999	62.5643910237433\\
3378.19999999999	62.5551312790457\\
3379.29999999995	62.5614014322069\\
3379.49999999999	62.5576991359924\\
3379.99999999995	62.5632377740304\\
3380.10000000005	62.5613869001834\\
3380.70000000005	62.5591575958353\\
3381.20000000005	62.5646940525833\\
3381.39999999995	62.5609936418749\\
3381.59999999998	62.5632078540379\\
3381.70000000005	62.5613578567627\\
3382.00000000002	62.5646787498891\\
3382.49999999999	62.5554307337551\\
3383.00000000001	62.5609647956017\\
3383.10000000005	62.5591156301726\\
3383.50000000005	62.5635417898097\\
3383.79999999996	62.5579952126245\\
3384.89999999999	62.5701624815362\\
3384.99999999995	62.5683140823019\\
3385.50000000003	62.5708884688091\\
3385.60000000001	62.5690403756978\\
3385.80000000001	62.5712513659588\\
3385.90000000002	62.5694034258712\\
3386.50000000005	62.5760349613181\\
3386.60000000005	62.5741872619364\\
3386.89999999996	62.577502214349\\
3387.09999999998	62.573807274445\\
3387.60000000002	62.5793311096024\\
3387.70000000006	62.5774839128638\\
3388.00000000001	62.5748944836339\\
3388.29999999998	62.5782080037776\\
3388.49999999999	62.5745145487812\\
3389.00000000006	62.5770853618955\\
3389.20000000001	62.5733927359632\\
3389.50000000006	62.5767052159547\\
3389.70000000005	62.5730131571184\\
3390.30000000004	62.5796366210477\\
3390.40000000003	62.5777908863\\
3390.8	62.5822053142234\\
3390.99999999999	62.5785143463773\\
3391.4	62.5740822644847\\
3391.80000000003	62.5696512279253\\
3392	62.5718581409746\\
3392.30000000002	62.5663247258578\\
3392.59999999996	62.5696348041383\\
3392.7	62.5677906154209\\
3392.89999999998	62.5699970527557\\
3393.00000000001	62.5681530164157\\
3393.7	62.5699805527727\\
3393.89999999996	62.5662934590454\\
3394.10000000004	62.5684992045254\\
3394.29999999996	62.5648126325713\\
3394.59999999998	62.5681208943353\\
3394.69999999996	62.5662778366914\\
3395.09999999997	62.5706880301602\\
3395.30000000001	62.5670024150321\\
3395.80000000006	62.5695691863718\\
3395.90000000003	62.567726737338\\
3396.20000000005	62.5651444218709\\
3396.50000000001	62.5684508037449\\
3396.59999999996	62.566608767333\\
3397.49999999997	62.5706380974806\\
3397.70000000005	62.5669550885867\\
3398.00000000001	62.5702598510933\\
3398.09999999998	62.5684185745395\\
3398.69999999996	62.566199835236\\
3398.99999999998	62.5695036921538\\
3399.09999999997	62.5676629795246\\
3399.29999999998	62.5698652703418\\
3399.39999999999	62.5680247095161\\
3399.69999999999	62.565445026178\\
3400.30000000002	62.572050347018\\
3400.4	62.5702102631966\\
3400.80000000002	62.5746126025464\\
3400.89999999996	62.5727727139077\\
3401.30000000003	62.5771741047804\\
3401.40000000005	62.5753344112891\\
3401.99999999995	62.5731166044502\\
3402.79999999998	62.5789767551206\\
3402.90000000003	62.577137819571\\
3403.19999999998	62.574559985896\\
3403.49999999999	62.577858737807\\
3403.59999999999	62.5760202132973\\
3403.79999999997	62.5782191016187\\
3403.90000000003	62.5763807285546\\
3404.59999999995	62.5693893735131\\
3405.39999999996	62.5781823520775\\
3405.49999999999	62.5763448437867\\
3405.80000000005	62.5737690478288\\
3406.10000000001	62.577065351418\\
3406.20000000001	62.5752282535302\\
3406.4	62.5774255100543\\
3406.49999999998	62.575588563377\\
3406.90000000003	62.5799823891987\\
3407.09999999996	62.5763089927213\\
3407.30000000003	62.578505605447\\
3407.80000000004	62.5693242172599\\
3408.59999999995	62.5575732684015\\
3408.90000000003	62.5608682898211\\
3408.99999999998	62.5590331759115\\
3409.29999999996	62.5564615474864\\
3409.50000000002	62.5586579070859\\
3409.6	62.556823180925\\
3409.80000000001	62.5590193260799\\
3410.1	62.5535159228198\\
3410.49999999995	62.557907699525\\
3410.69999999996	62.5542394746101\\
3411.00000000002	62.5516695494122\\
3411.4	62.5560603839953\\
3411.49999999999	62.5542267557744\\
3411.70000000004	62.556421830119\\
3411.79999999998	62.5545883525309\\
3412.09999999995	62.5520192251333\\
3412.40000000006	62.5553113553114\\
3412.50000000003	62.5534782863506\\
3413.09999999997	62.5512715340443\\
3413.4	62.5487036765783\\
3413.59999999995	62.5508978527697\\
3413.69999999996	62.5490655574433\\
3414.79999999999	62.5552724823567\\
3414.90000000002	62.5534407027819\\
3415.20000000005	62.5567300090768\\
3415.30000000003	62.5548984013586\\
3415.50000000004	62.5570909942616\\
3415.60000000005	62.5552595368446\\
3416.49999999995	62.5621963355383\\
3416.79999999997	62.5567034446428\\
3417.40000000001	62.5515727871251\\
3417.99999999996	62.5493695327814\\
3419.19999999999	62.5566636446056\\
3419.29999999996	62.5548341814353\\
3420.20000000003	62.5617635879894\\
3420.29999999997	62.5599345105836\\
3420.79999999997	62.5624835569587\\
3420.89999999998	62.5606547793043\\
3421.3	62.5650318583036\\
3421.39999999998	62.5632032734181\\
3421.7	62.5664854754807\\
3421.99999999996	62.5610005552146\\
3422.2	62.5631884989627\\
3422.40000000005	62.5595325054785\\
3422.89999999998	62.5620800467426\\
3423	62.5602524027928\\
3423.70000000003	62.5562240785093\\
3424.00000000002	62.5536637364563\\
3424.39999999996	62.5580376697328\\
3424.90000000006	62.5489051094891\\
3425.49999999998	62.5525455394675\\
3425.59999999998	62.5507195609656\\
3425.80000000004	62.5529058057737\\
3425.89999999998	62.5510799766491\\
3426.10000000005	62.5532660089896\\
3426.19999999995	62.5514403292181\\
3427.00000000005	62.5601820781419\\
3427.09999999997	62.5583566760037\\
3427.80000000001	62.5543335569882\\
3428.30000000001	62.5597946563995\\
3428.60000000002	62.5543208796337\\
3428.90000000002	62.551764362788\\
3429.09999999995	62.5539484427855\\
3429.20000000002	62.5521243402444\\
3429.40000000001	62.5543082081936\\
3429.50000000003	62.5524842547236\\
3429.70000000004	62.5546679106653\\
3429.80000000004	62.5528441062422\\
3430.19999999995	62.5572107395855\\
3430.40000000004	62.5535636204635\\
3431.49999999996	62.5597388973074\\
3431.70000000001	62.5560930124133\\
3432.00000000001	62.5535386498062\\
3432.80000000003	62.5418742171342\\
3433.10000000001	62.5393219154142\\
3433.39999999997	62.5425950196592\\
3433.60000000004	62.5389521507412\\
3433.80000000001	62.5411339875943\\
3433.90000000004	62.5393127548049\\
3434.1	62.5414943800594\\
3434.29999999996	62.5378523177265\\
3434.90000000002	62.5414847161572\\
3435.00000000001	62.5396640563593\\
3435.50000000004	62.5422051461171\\
3435.59999999995	62.5403847833047\\
3435.80000000003	62.5425652667423\\
3435.89999999997	62.5407450523865\\
3436.6	62.536735822155\\
3436.8	62.5389158834997\\
3436.90000000006	62.5370963049171\\
3437.50000000003	62.5349080754014\\
3437.69999999998	62.5370876723486\\
3437.90000000003	62.5334496800465\\
3438.6	62.5381684939076\\
3438.80000000002	62.5345313908517\\
3439.00000000001	62.5367101858044\\
3439.10000000005	62.5348918353105\\
3439.30000000005	62.5370704192592\\
3439.50000000001	62.5334341202466\\
3439.89999999995	62.5290697674419\\
3441.1	62.5363245379519\\
3441.40000000002	62.5308731657707\\
3441.60000000006	62.5330505273557\\
3441.80000000005	62.529416891833\\
3442.10000000001	62.5326825867178\\
3442.29999999995	62.5290495003486\\
3442.70000000005	62.5334030440339\\
3443.00000000003	62.5279544596439\\
3443.19999999997	62.5301309790027\\
3443.29999999999	62.528315037463\\
3443.50000000005	62.5304913462655\\
3443.69999999998	62.5268598641036\\
3444.50000000003	62.5326598153632\\
3444.59999999997	62.5308444857317\\
3444.90000000003	62.5341074020319\\
3444.99999999997	62.5322922411541\\
3445.59999999997	62.535914328003\\
3445.79999999999	62.5322847441888\\
3446.39999999998	62.5388074858552\\
3446.69999999998	62.5333642799118\\
3447.20000000003	62.5387984799698\\
3447.30000000003	62.5369843940361\\
3447.79999999996	62.539516807332\\
3447.99999999999	62.5358893303558\\
3448.50000000003	62.5413211158151\\
3448.59999999999	62.5395076405602\\
3448.90000000004	62.542766019136\\
3449.09999999996	62.5391395106112\\
3449.30000000002	62.5413115324404\\
3449.40000000002	62.5394984780403\\
3449.70000000005	62.542756101803\\
3450	62.5373177589055\\
3450.29999999998	62.5347785763969\\
3450.69999999996	62.5391213631622\\
3450.80000000001	62.537309107769\\
3451.50000000001	62.5449067099316\\
3451.59999999999	62.543094706956\\
3452.10000000003	62.5398296738312\\
3452.59999999998	62.5423581544878\\
3452.70000000002	62.540546802595\\
3453.49999999996	62.5434329395413\\
3453.59999999995	62.5416220285491\\
3454.40000000003	62.544507164568\\
3454.49999999995	62.5426966942627\\
3455.60000000005	62.5488323639205\\
3455.69999999996	62.5470223971295\\
3456.20000000004	62.5524404710239\\
3456.40000000001	62.5488210617677\\
3456.69999999999	62.5520712797963\\
3457.00000000001	62.5466431402042\\
3457.20000000005	62.5488097648454\\
3457.29999999995	62.5470006363163\\
3458.10000000005	62.555664796715\\
3458.59999999999	62.546621563015\\
3459.19999999995	62.5531176827682\\
3459.29999999998	62.5513094756316\\
3459.70000000004	62.5469680328343\\
3459.90000000005	62.5491329479769\\
3460.10000000003	62.5455176001387\\
3460.3	62.5476823488614\\
3460.39999999995	62.5458748735732\\
3460.59999999995	62.5480394139914\\
3460.70000000002	62.546232085067\\
3460.89999999999	62.5483964172205\\
3461.20000000003	62.5429751827348\\
3461.40000000004	62.5451393904377\\
3461.49999999996	62.5433325629767\\
3462.20000000005	62.5480172139907\\
3462.30000000003	62.5462107208872\\
3462.5	62.548374054179\\
3462.59999999998	62.5465677072804\\
3463.30000000006	62.5541375526939\\
3463.49999999997	62.5505254648343\\
3463.79999999999	62.5479950344987\\
3463.99999999995	62.5501573280217\\
3464.09999999997	62.5483517117949\\
3464.99999999998	62.5523072927188\\
3465.10000000001	62.5505021355189\\
3465.30000000001	62.5526634731921\\
3465.69999999995	62.545444053321\\
3466.20000000004	62.5421919626114\\
3467.00000000001	62.5479507369271\\
3467.09999999999	62.5461467466544\\
3467.59999999996	62.5486633791851\\
3467.70000000001	62.5468596804891\\
3468.19999999999	62.5493757748753\\
3468.30000000002	62.5475723676623\\
3468.50000000003	62.5497318802975\\
3468.60000000002	62.5479286187909\\
3469.30000000005	62.5526027555197\\
3469.39999999996	62.5507998270644\\
3469.80000000005	62.5551168621574\\
3470.00000000005	62.5515114838189\\
3470.59999999998	62.5551041576627\\
3471.00000000002	62.5478954798191\\
3471.60000000001	62.5543681769738\\
3471.79999999995	62.5507647109652\\
3472.19999999995	62.5550787662356\\
3472.29999999995	62.5532772722037\\
3472.8	62.5586685478995\\
3472.99999999997	62.5550660792952\\
3473.20000000004	62.5572222382173\\
3473.30000000005	62.5554212011286\\
3473.49999999999	62.5575771533855\\
3473.60000000005	62.5557762616231\\
3473.90000000003	62.5532527345999\\
3474.09999999998	62.5554084393529\\
3474.19999999996	62.5536079210201\\
3474.79999999995	62.5485625485626\\
3475.30000000002	62.5539506243886\\
3475.40000000005	62.5521507696734\\
3475.70000000003	62.5553829334254\\
3475.90000000003	62.5517836593786\\
3476.20000000005	62.5550153899261\\
3476.30000000001	62.5532159705442\\
3476.70000000001	62.5575241601473\\
3477	62.5521267723103\\
3477.20000000003	62.5542806200213\\
3477.30000000003	62.5524817392305\\
3477.70000000003	62.5567887745126\\
3477.80000000005	62.5549900802208\\
3478.30000000004	62.557497700092\\
3478.40000000002	62.5556992956734\\
3478.69999999999	62.5531792572151\\
3478.89999999995	62.5553319919517\\
3479.09999999995	62.5517360312716\\
3479.69999999996	62.5581929995977\\
3479.80000000004	62.5563952987155\\
3480.40000000005	62.5628501652062\\
3480.49999999999	62.5610526920646\\
3480.79999999998	62.5585337125456\\
3481.50000000002	62.5660615808823\\
3481.70000000001	62.5624676891263\\
3482.10000000005	62.55815289185\\
3482.40000000001	62.5556353194544\\
3482.59999999997	62.5577856260947\\
3482.70000000001	62.5559894337889\\
3483.00000000002	62.5592144928368\\
3483.09999999999	62.5574184657786\\
3483.80000000003	62.5505898561956\\
3484.19999999997	62.5462790230462\\
3484.69999999995	62.551652892562\\
3485.00000000003	62.546268399759\\
3485.40000000004	62.5419595466934\\
3485.59999999996	62.5441087873311\\
3485.69999999997	62.5423145332492\\
3485.90000000005	62.5444635685599\\
3486.00000000002	62.5426694587074\\
3486.20000000002	62.5448182887302\\
3486.50000000001	62.5394367005105\\
3486.80000000002	62.542659669047\\
3486.9	62.5408660739891\\
3488.10000000006	62.5508858437016\\
3488.19999999996	62.5490926812488\\
3488.99999999996	62.5576796308504\\
3489.20000000003	62.5540939443441\\
3489.59999999999	62.5583861076883\\
3489.9	62.5530085959885\\
3490.40000000002	62.5583727259705\\
3490.79999999998	62.5512045604285\\
3491.20000000001	62.5554950877897\\
3491.6	62.5483288942349\\
3491.99999999998	62.5440279487987\\
3492.40000000006	62.5483178239084\\
3492.49999999999	62.5465269426788\\
3493.10000000004	62.5500973319592\\
3493.20000000004	62.5483067586523\\
3493.9	62.5386376645678\\
3494.19999999998	62.5361302692957\\
3494.40000000004	62.5382744312491\\
3494.49999999996	62.5364848623591\\
3494.79999999998	62.5397007067441\\
3494.90000000002	62.5379113018598\\
3495.2	62.535404686293\\
3495.69999999998	62.5379026260084\\
3495.89999999995	62.5343249427918\\
3496.19999999998	62.5318193518863\\
3496.40000000002	62.5339625339625\\
3496.49999999995	62.5321741119945\\
3497.09999999995	62.5386023104198\\
3497.19999999995	62.5368141137449\\
3498.29999999995	62.5457351932312\\
3498.49999999999	62.5421597210313\\
3498.7	62.5443009031668\\
3498.79999999999	62.5425133613421\\
3499.39999999996	62.5489355622232\\
3499.50000000004	62.5471482455138\\
3500.09999999999	62.5449974287184\\
3500.60000000004	62.5389207872711\\
3500.79999999995	62.541060870062\\
3501.1	62.5357020450131\\
3501.40000000004	62.5332000571184\\
3501.70000000003	62.5364098463647\\
3501.80000000001	62.5346240612239\\
3502.29999999997	62.5399725902239\\
3502.39999999997	62.5381870092791\\
3502.80000000005	62.5424648148677\\
3502.90000000001	62.540679417642\\
3503.50000000001	62.5385317958671\\
3503.79999999996	62.541739204886\\
3503.9	62.5399543378995\\
3504.09999999995	62.5420923463273\\
3504.19999999998	62.5403076220643\\
3504.40000000003	62.5424454273078\\
3504.49999999996	62.5406608457456\\
3504.69999999996	62.542798447843\\
3504.80000000004	62.5410140089589\\
3505.29999999995	62.5463570491242\\
3505.39999999995	62.5445728141492\\
3505.69999999995	62.5420731359461\\
3506.70000000004	62.5470514429109\\
3506.90000000003	62.5434844596521\\
3507.10000000001	62.5456204379562\\
3507.19999999997	62.5438371396801\\
3507.60000000003	62.5395558343074\\
3507.99999999999	62.5438271428979\\
3508.19999999998	62.540261665194\\
3508.4	62.5423970357703\\
3508.50000000001	62.54061449011\\
3508.79999999998	62.5381173587164\\
3509.30000000001	62.5349062517809\\
3510.10000000004	62.5377471369153\\
3510.20000000001	62.5359655869869\\
3510.50000000005	62.5391670939441\\
3510.59999999998	62.5373857065542\\
3510.90000000002	62.5405867274281\\
3511.09999999995	62.5370243791296\\
3511.49999999998	62.5412917188746\\
3511.70000000001	62.5377299390626\\
3512.09999999996	62.5419964694493\\
3512.30000000003	62.5384352579433\\
3512.79999999998	62.5437672578212\\
3512.90000000001	62.5419869057785\\
3513.79999999999	62.5345058197445\\
3514.29999999999	62.5284543592078\\
3514.99999999997	62.535916474638\\
3515.29999999997	62.5305797348808\\
3515.9	62.5227531285552\\
3516.79999999998	62.5295004122949\\
3516.90000000004	62.5277224907592\\
3517.40000000002	62.5216773276475\\
3517.60000000004	62.5238081701112\\
3517.70000000005	62.5220308147137\\
3517.99999999999	62.5252266848583\\
3518.29999999996	62.5198954070032\\
3518.7	62.5241559622599\\
3518.80000000004	62.5223791525761\\
3519.19999999995	62.5266388202199\\
3519.30000000004	62.5248621924192\\
3519.49999999995	62.5269917036027\\
3519.60000000002	62.525215217206\\
3520.40000000004	62.5223689816788\\
3520.79999999995	62.5266267147604\\
3520.9	62.5248508946322\\
3521.70000000005	62.5305241637799\\
3521.79999999998	62.5287486867884\\
3522.10000000005	62.5262619953438\\
3522.29999999997	62.5283897342721\\
3522.40000000002	62.5266146202981\\
3523.4	62.5202213707961\\
3524	62.5180897250362\\
3524.29999999995	62.5212802179094\\
3524.39999999997	62.5195063129522\\
3524.59999999995	62.5216330467841\\
3524.99999999995	62.5145385946498\\
3525.79999999995	62.5202076065685\\
3525.99999999998	62.5166614673435\\
3526.2	62.5187873975555\\
3526.29999999996	62.5170145190563\\
3527.1	62.5255159900204\\
3527.19999999997	62.5237433731183\\
3527.50000000003	62.5269304909854\\
3527.79999999995	62.5216134244168\\
3528.20000000001	62.5258623132954\\
3528.30000000003	62.5240902392019\\
3528.60000000002	62.5272763340607\\
3528.69999999999	62.5255044207663\\
3528.99999999999	62.5230228670199\\
3529.19999999996	62.5251466296433\\
3529.29999999998	62.5233750779169\\
3530.09999999997	62.5262024814458\\
3530.29999999995	62.5226603217766\\
3530.59999999994	62.520180134251\\
3531.20000000006	62.5265482966613\\
3531.29999999997	62.5247777085575\\
3531.80000000001	62.5300829581811\\
3531.90000000001	62.5283125707814\\
3533.00000000002	62.5371486796298\\
3533.09999999999	62.5353786935356\\
3533.59999999995	62.5378498457707\\
3533.70000000004	62.5360801403588\\
3534.09999999995	62.5318318148379\\
3534.79999999997	62.5364225296331\\
3534.9	62.5346534653465\\
3535.10000000003	62.5367730255714\\
3535.50000000001	62.52969792963\\
3536.09999999999	62.5275719699112\\
3536.29999999998	62.5296912114014\\
3536.40000000005	62.5279230877987\\
3536.79999999996	62.5321609318895\\
3536.89999999996	62.5303929884083\\
3537.30000000001	62.5261491490926\\
3537.59999999995	62.5236735732256\\
3538.50000000001	62.5275532696547\\
3538.59999999995	62.5257863057055\\
3539.80000000004	62.5384897878471\\
3539.89999999996	62.5367231638418\\
3540.09999999999	62.5388396135812\\
3540.39999999995	62.533540460387\\
3540.59999999999	62.53565679103\\
3540.8	62.5321246010901\\
3541.10000000002	62.5352987687789\\
3541.20000000005	62.5335328834044\\
3541.39999999996	62.5356487364111\\
3541.59999999996	62.5321173447779\\
3541.80000000005	62.5342330387645\\
3541.89999999996	62.5324675324675\\
3542.19999999999	62.5299946362533\\
3542.80000000004	62.5335177397048\\
3542.89999999997	62.5317527519052\\
3543.10000000003	62.5338676902235\\
3543.19999999997	62.5321028419835\\
3543.4	62.5342175814872\\
3543.70000000003	62.5289237541622\\
3544.20000000003	62.534209858082\\
3544.30000000004	62.5324455479066\\
3544.50000000006	62.5345596118039\\
3544.99999999997	62.5257397534625\\
3545.69999999996	62.5331377968301\\
3545.90000000004	62.5296108291032\\
3546.60000000004	62.5341867087715\\
3546.80000000005	62.5306605768418\\
3547.1	62.5281912494362\\
3547.40000000006	62.5257223396758\\
3547.9	62.5225479143179\\
3548.2	62.5257165403151\\
3548.40000000005	62.522192475694\\
3549.00000000001	62.5200755121017\\
3549.6	62.5179592641632\\
3550.20000000001	62.5242937216573\\
3550.30000000005	62.522532672375\\
3550.49999999998	62.5246437221878\\
3550.60000000002	62.5228828118399\\
3550.8	62.5249936635782\\
3551.00000000003	62.5214722198755\\
3551.30000000005	62.5246381708622\\
3551.50000000002	62.5211172429328\\
3551.8	62.5242827782314\\
3552.10000000005	62.5190023084286\\
3552.50000000004	62.5232224286438\\
3552.59999999995	62.5214625496101\\
3553.29999999996	62.5232172004278\\
3553.60000000003	62.5179390494414\\
3554.20000000003	62.5214528880511\\
3554.39999999995	62.5179350119567\\
3554.80000000006	62.5221525218712\\
3554.90000000005	62.5203938115331\\
3555.20000000005	62.5179309762889\\
3556.2	62.5200348676996\\
3556.30000000006	62.5182769092341\\
3556.49999999998	62.5203846370129\\
3556.6	62.5186268169933\\
3557.39999999995	62.5242445537597\\
3557.50000000005	62.5224870699348\\
3558.19999999995	62.5298597644943\\
3558.39999999999	62.5263453702403\\
3558.89999999995	62.5231806687272\\
3559.10000000001	62.5252865812542\\
3559.19999999998	62.5235299075661\\
3559.39999999997	62.5256356229808\\
3559.60000000001	62.5221226507852\\
3560.30000000002	62.5266823952365\\
3560.59999999998	62.5214143286432\\
3561.40000000005	62.5186017127615\\
3561.60000000006	62.5207064042452\\
3561.79999999997	62.5171958786041\\
3562.00000000005	62.5193004126779\\
3562.10000000005	62.5175453371512\\
3562.39999999997	62.520701754386\\
3562.79999999997	62.5136826742261\\
3563.49999999995	62.5210461331238\\
3563.6	62.5192917473412\\
3564.29999999997	62.5210414094939\\
3564.39999999995	62.5192874175901\\
3564.69999999997	62.5224416517056\\
3564.80000000003	62.5206878173301\\
3565.10000000004	62.523841579715\\
3565.20000000001	62.5220879028413\\
3565.40000000005	62.5241901556584\\
3565.50000000004	62.5224366165582\\
3565.99999999997	62.5248871316004\\
3566.2	62.5213807026891\\
3566.50000000002	62.5245331688443\\
3566.60000000001	62.5227801609331\\
3567.20000000003	62.5150674179351\\
3567.59999999998	62.5192701180032\\
3567.69999999997	62.5175177980829\\
3568.19999999997	62.511560126671\\
3568.5	62.5091072129126\\
3568.99999999997	62.5143593623042\\
3569.20000000005	62.5108564704564\\
3570.00000000002	62.5136550796896\\
3570.09999999996	62.5119040950087\\
3570.40000000006	62.509452457639\\
3570.79999999999	62.513652020499\\
3570.89999999995	62.5119014281714\\
3571.20000000005	62.5094503402123\\
3571.49999999996	62.5069996640161\\
3572.00000000003	62.5122476974329\\
3572.10000000001	62.5104977324898\\
3572.39999999999	62.5080475857243\\
3572.70000000001	62.5111957008509\\
3572.80000000002	62.5094461082034\\
3573.10000000004	62.5069965297213\\
3573.59999999995	62.5122422139519\\
3573.69999999995	62.5104930326263\\
3574.1	62.5062951149908\\
3574.29999999995	62.5083930170099\\
3574.39999999997	62.5066442859141\\
3574.69999999997	62.5041960389392\\
3574.9	62.5062937062937\\
3575	62.5045453274034\\
3575.30000000004	62.5076914471108\\
3575.79999999999	62.4989513129562\\
3576.09999999995	62.5020971981433\\
3576.29999999995	62.498601946091\\
3576.69999999999	62.5027957951241\\
3576.80000000006	62.5010483938606\\
3577.00000000005	62.5031450057309\\
3577.19999999996	62.4996505744556\\
3577.5	62.5027951699463\\
3577.59999999999	62.501048159432\\
3577.79999999997	62.5031443025238\\
3578.09999999998	62.4979039740652\\
3578.7	62.5041913490555\\
3578.80000000005	62.5024448853\\
3579.79999999995	62.5101259811727\\
3579.90000000001	62.5083798882682\\
3580.30000000005	62.5041894760362\\
3580.89999999996	62.5076794191567\\
3580.99999999999	62.5059339309151\\
3581.20000000005	62.50802781113\\
3581.3	62.5062824593734\\
3582.19999999998	62.5101191971638\\
3582.40000000002	62.506629448709\\
3582.69999999996	62.5097688958357\\
3582.80000000006	62.5080242261855\\
3583.69999999997	62.5062782521346\\
3584.19999999998	62.5115085232821\\
3584.29999999999	62.5097645352081\\
3584.79999999998	62.5122039666379\\
3584.90000000002	62.510460251046\\
3586.39999999999	62.5066220549282\\
3586.59999999999	62.5087127443054\\
3586.80000000003	62.5052273550977\\
3586.99999999995	62.5073178891026\\
3587.10000000005	62.5055753791258\\
3587.69999999999	62.5118456993143\\
3587.80000000002	62.5101034031049\\
3587.99999999998	62.51219308269\\
3588.10000000006	62.5104509224681\\
3588.99999999997	62.514279345797\\
3589.20000000006	62.5107959769314\\
3590.10000000004	62.5146231407721\\
3590.19999999999	62.512881931872\\
3591.1	62.51113833816\\
3591.60000000001	62.5052203691845\\
3591.80000000002	62.507308109914\\
3591.89999999998	62.5055679287305\\
3592.30000000003	62.5097427903352\\
3592.40000000003	62.5080027835769\\
3592.99999999998	62.5059141131613\\
3593.49999999999	62.5027827248442\\
3593.69999999998	62.5048694974679\\
3593.79999999998	62.5031303041264\\
3594.20000000004	62.5073032301144\\
3594.40000000004	62.5038252886354\\
3595.39999999997	62.511472674176\\
3595.50000000002	62.5097341194794\\
3596.29999999998	62.5180736291848\\
3596.40000000001	62.5163353260114\\
3597.30000000002	62.5090343025519\\
3597.59999999998	62.5121605470162\\
3597.70000000005	62.5104230363\\
3597.99999999998	62.5079903282288\\
3598.70000000003	62.4986106479938\\
3598.99999999995	62.5017365452474\\
3599.09999999995	62.5\\
3599.30000000003	62.5020836806134\\
3599.40000000004	62.5003472704542\\
3599.80000000002	62.5045140142782\\
3600	62.5010416377323\\
3600.40000000003	62.5052076100542\\
3600.49999999998	62.503471643615\\
3600.8	62.5065955733289\\
3600.89999999998	62.5048597611775\\
3601.39999999995	62.5017353880328\\
3601.70000000003	62.5048586817702\\
3601.89999999998	62.5013881177124\\
3602.19999999994	62.4989589984177\\
3602.60000000003	62.5031226580065\\
3602.69999999997	62.5013878094815\\
3603.20000000005	62.5065911803069\\
3603.30000000003	62.5048565243936\\
3603.59999999995	62.5079779115909\\
3603.70000000003	62.5062434097342\\
3604.29999999995	62.5124847408723\\
3604.39999999999	62.5107504508254\\
3604.6	62.5128304713291\\
3604.70000000003	62.5110963160231\\
3605.3	62.5090142563932\\
3606.00000000001	62.5162918388286\\
3606.1	62.5145582607731\\
3606.49999999995	62.5187156879055\\
3606.7	62.5152489741599\\
3606.99999999997	62.512821934518\\
3607.49999999997	62.5152455926378\\
3607.59999999996	62.5135127643651\\
3608.50000000005	62.5200908939755\\
3609.09999999997	62.5096974398759\\
3609.29999999995	62.5117748102178\\
3609.50000000003	62.5083111702128\\
3610.29999999998	62.5138488810104\\
3610.5	62.5103860854152\\
3611.09999999997	62.5055383252105\\
3611.29999999995	62.5076147754334\\
3611.40000000004	62.505883981725\\
3611.60000000002	62.50796024033\\
3611.69999999996	62.5062295808184\\
3612.30000000005	62.5096888495183\\
3612.39999999997	62.5079584775087\\
3612.89999999997	62.5103791862718\\
3613.09999999995	62.5069190745046\\
3613.79999999997	62.511414261601\\
3613.99999999996	62.5079549542071\\
3614.30000000005	62.5055334218681\\
3614.70000000002	62.5013832023902\\
3615.00000000004	62.5044950347155\\
3615.09999999998	62.5027660986944\\
3615.50000000006	62.5069144816905\\
3615.60000000004	62.5051857178416\\
3616.10000000006	62.5020740003318\\
3616.7	62.5055297500553\\
3616.90000000002	62.5020735416091\\
3617.3	62.5062199369713\\
3617.40000000002	62.5044920525225\\
3617.59999999999	62.506564944578\\
3617.80000000005	62.5031095386827\\
3618.00000000004	62.5051822779912\\
3618.10000000001	62.5034547565088\\
3618.59999999995	62.5086356979026\\
3618.69999999999	62.5069083674146\\
3619.79999999995	62.5127765960386\\
3619.89999999998	62.5110497237569\\
3620.79999999998	62.5065591427546\\
3621.00000000004	62.5086299743172\\
3621.09999999999	62.5069037887993\\
3622.39999999995	62.520358868185\\
3622.49999999994	62.5186330260034\\
3622.69999999996	62.5207022192779\\
3622.99999999998	62.5155253788192\\
3623.29999999995	62.5186289120715\\
3623.39999999995	62.516903546295\\
3623.99999999997	62.5231091857289\\
3624.20000000005	62.5196589686284\\
3624.79999999997	62.5175866920467\\
3625.39999999996	62.5155151013653\\
3626.20000000006	62.5127540468246\\
3626.59999999998	62.5168886315383\\
3626.70000000001	62.515164883644\\
3627.00000000002	62.512751233768\\
3627.29999999997	62.5103379831284\\
3627.50000000001	62.5124048957989\\
3627.7	62.5089585974971\\
3628.09999999996	62.5048233283722\\
3628.60000000004	62.5099898035109\\
3628.70000000004	62.5082671957672\\
3629.29999999996	62.5144652008596\\
3629.39999999998	62.5127428020388\\
3629.90000000005	62.5096418732782\\
3630.39999999999	62.5037873571134\\
3630.69999999995	62.5013771069737\\
3630.99999999999	62.5044752278924\\
3631.30000000004	62.4993115602798\\
3631.60000000005	62.5024093399785\\
3631.70000000003	62.5006883638967\\
3632.09999999997	62.4965585595507\\
3632.30000000003	62.4986234996146\\
3632.40000000001	62.4969029593944\\
3632.59999999996	62.4989677099678\\
3632.69999999998	62.4972473023563\\
3633.09999999998	62.5013761972916\\
3633.29999999996	62.4979358176914\\
3633.69999999998	62.493808134735\\
3634.29999999995	62.4972485141977\\
3634.40000000001	62.4955289585913\\
3634.99999999999	62.4934664796017\\
3635.30000000004	62.4910601309347\\
3636.00000000002	62.4955309259921\\
3636.19999999996	62.4920936116382\\
3636.39999999998	62.4941564691324\\
3636.49999999996	62.4924379915306\\
3636.80000000002	62.4955319090434\\
3636.89999999998	62.493813582623\\
3637.3	62.4896904382251\\
3637.89999999997	62.4958768554151\\
3637.99999999997	62.4941590390589\\
3638.50000000005	62.4993129225526\\
3638.70000000001	62.4958777618995\\
3639.00000000001	62.4989695254321\\
3639.1	62.4972521433282\\
3640.10000000005	62.4938190209329\\
3640.70000000001	62.4972533509119\\
3640.89999999994	62.4938203790168\\
3641.60000000002	62.5010297388582\\
3641.90000000003	62.495881383855\\
3642.59999999999	62.486617069756\\
3643.30000000004	62.4828456935829\\
3643.70000000002	62.4869641582963\\
3643.89999999996	62.4835345773875\\
3644.09999999995	62.4855935459086\\
3644.29999999995	62.4821644166392\\
3644.79999999999	62.4763367993635\\
3645.19999999999	62.480454283598\\
3645.60000000005	62.473599034479\\
3645.79999999999	62.4756575879755\\
3645.99999999997	62.4722306025616\\
3646.49999999996	62.4773761860363\\
3646.60000000006	62.4756629281268\\
3647.40000000003	62.4784098697738\\
3647.70000000004	62.4732715609408\\
3648.69999999998	62.4780749835562\\
3648.89999999998	62.4746505892025\\
3649.9	62.4767123287671\\
3650.00000000005	62.4750006849127\\
3651.49999999996	62.4904151604776\\
3651.59999999995	62.4887038913383\\
3652.19999999995	62.4948662486652\\
3652.39999999995	62.4914442162902\\
3653.00000000002	62.4893925706934\\
3653.2	62.4914460898366\\
3653.49999999995	62.4863148675279\\
3653.99999999999	62.4832380066227\\
3654.29999999998	62.4808450087566\\
3654.50000000006	62.4828982652\\
3654.60000000001	62.481188606452\\
3654.80000000005	62.4832416755588\\
3654.89999999997	62.4815321477428\\
3655.20000000001	62.4846113862063\\
3655.29999999998	62.4829020079882\\
3655.69999999996	62.4870069478637\\
3655.79999999995	62.4852977379031\\
3656.20000000006	62.4811968383338\\
3656.70000000001	62.4781229490265\\
3657	62.4812009515737\\
3657.30000000001	62.4760759009132\\
3657.50000000006	62.4781277340333\\
3657.69999999998	62.4747115752638\\
3657.99999999996	62.4777890161559\\
3658.19999999995	62.4743733428095\\
3658.79999999998	62.4777938724753\\
3659.10000000003	62.4726716222125\\
3659.50000000002	62.4767734178599\\
3659.59999999996	62.4750662622619\\
3660.20000000003	62.481217386553\\
3660.40000000004	62.4778035787461\\
3660.90000000006	62.4747336793226\\
3661.19999999996	62.4778084287002\\
3661.29999999999	62.476102037472\\
3661.49999999995	62.4781516277037\\
3661.60000000006	62.4764453669061\\
3662.19999999995	62.4825929060973\\
3662.30000000001	62.4808868501529\\
3662.59999999997	62.4784994676059\\
3663.20000000001	62.4737258755767\\
3664.10000000001	62.4774848534469\\
3664.20000000002	62.4757798215212\\
3664.40000000002	62.4778278073407\\
3664.49999999997	62.4761229056377\\
3664.89999999998	62.4802182810368\\
3665.10000000003	62.4768089053803\\
3665.29999999999	62.4788563321875\\
3665.40000000001	62.477151821034\\
3665.89999999996	62.4822695035461\\
3666.00000000004	62.4805651782548\\
3666.19999999995	62.4826118975534\\
3666.29999999997	62.4809077023784\\
3667.30000000005	62.4747777717184\\
3667.49999999994	62.476824081143\\
3667.80000000002	62.4717140598163\\
3667.99999999998	62.4737602573539\\
3668.2	62.4703541149851\\
3668.80000000002	62.4628635285781\\
3669.49999999995	62.4591236102028\\
3669.69999999995	62.4611695460243\\
3669.79999999995	62.459467560424\\
3670.09999999996	62.4625361015749\\
3670.19999999995	62.4608342642291\\
3670.49999999998	62.4584536588024\\
3670.70000000006	62.4604990737714\\
3670.9	62.4570961590847\\
3671.70000000001	62.4598289667193\\
3671.80000000001	62.4581279446608\\
3672.39999999999	62.4642614023145\\
3672.7	62.4591592245698\\
3673.29999999999	62.4652910110524\\
3673.39999999998	62.4635905811896\\
3673.99999999999	62.4697204757628\\
3674.10000000002	62.468020249306\\
3674.30000000002	62.4700631395602\\
3674.50000000004	62.4666630381538\\
3674.89999999998	62.4625850340136\\
3675.10000000005	62.4646277753592\\
3675.19999999996	62.4629281963377\\
3675.70000000002	62.4680341694325\\
3676.20000000006	62.4595381225689\\
3676.49999999999	62.4626013164337\\
3676.79999999997	62.4575049634203\\
3677.30000000001	62.4598901397727\\
3677.39999999996	62.4581917063222\\
3677.89999999997	62.452419793366\\
3678.29999999995	62.448347107438\\
3678.59999999996	62.4459727621171\\
3678.89999999998	62.4490350638761\\
3679.09999999997	62.4456403565993\\
3679.5	62.4497227959561\\
3680.09999999999	62.4395413292756\\
3681	62.4351416696096\\
3681.19999999995	62.4371825170456\\
3681.50000000006	62.4320947414168\\
3681.80000000005	62.4351557619707\\
3682.00000000004	62.4317644822248\\
3682.20000000002	62.4338049588572\\
3682.50000000006	62.42871883995\\
3683	62.4311042328473\\
3683.49999999998	62.4226300358345\\
3683.80000000002	62.4256901653139\\
3684.10000000003	62.4233212094892\\
3684.4	62.4263807843669\\
3684.60000000005	62.4229923738703\\
3684.89999999996	62.4206241519674\\
3685.09999999999	62.4226636274829\\
3685.30000000005	62.4192760622999\\
3685.50000000004	62.4213153896245\\
3686.1	62.419293581466\\
3686.79999999995	62.4264287070439\\
3687.79999999998	62.4339054746604\\
3688.59999999998	62.4393417735245\\
3688.80000000004	62.4359565182033\\
3689.19999999997	62.440029273846\\
3689.49999999998	62.4376626192541\\
3689.99999999996	62.4427522289369\\
3690.29999999994	62.4403858660308\\
3691.00000000004	62.4475088726938\\
3691.50000000002	62.4444685231336\\
3691.70000000004	62.4465030608375\\
3692.4	62.4509140148951\\
3692.8	62.4468574832787\\
3693.19999999998	62.4509246473344\\
3693.40000000003	62.4475429809124\\
3694.00000000004	62.4509352751685\\
3694.50000000005	62.4478969306556\\
3695.30000000001	62.4506142772095\\
3695.70000000005	62.446560961091\\
3696.90000000004	62.4560454422505\\
3697.10000000004	62.4526668830466\\
3697.40000000001	62.4557133198107\\
3698.49999999999	62.4641756340237\\
3698.80000000001	62.4591094649761\\
3698.99999999997	62.4611391960207\\
3699.49999999997	62.4581035787653\\
3699.90000000003	62.4621621621622\\
3700.10000000002	62.458786011567\\
3700.70000000001	62.4621703415478\\
3701.59999999998	62.4658940486803\\
3702.20000000001	62.4638738081733\\
3703.19999999995	62.4551076067291\\
3703.60000000003	62.4591624591625\\
3704.39999999995	62.4483735996761\\
3704.60000000004	62.4504008421735\\
3704.90000000002	62.4480431848853\\
3705.39999999999	62.4450141681285\\
3705.79999999997	62.4490677028522\\
3706	62.445697633631\\
3706.30000000005	62.4433412475718\\
3706.49999999995	62.445367722441\\
3707.99999999995	62.4524689193927\\
3709.10000000003	62.4555160142349\\
3709.69999999998	62.4588926626772\\
3710.49999999999	62.4508165795289\\
3711.20000000002	62.4578988494598\\
3712.19999999997	62.4626242491178\\
3712.50000000004	62.4602704304261\\
3712.7	62.4622926093514\\
3713.20000000002	62.4592680365174\\
3713.60000000002	62.4633115222016\\
3714.20000000002	62.4612982257761\\
3714.60000000004	62.4572643820497\\
3714.90000000001	62.4602960969044\\
3715.10000000006	62.4569336778639\\
3716.09999999996	62.4535816156289\\
3716.80000000002	62.457962280395\\
3717.30000000005	62.4603217302416\\
3717.60000000001	62.4579713263577\\
3718.40000000002	62.4633588812693\\
3718.60000000003	62.4599994621776\\
3719.30000000001	62.4670645803087\\
3719.90000000004	62.4569892473118\\
3720.70000000001	62.4543109008815\\
3720.89999999995	62.4563289438323\\
3722.20000000003	62.4613814039707\\
3722.49999999995	62.4590340084887\\
3723.70000000006	62.4550190665449\\
3724.10000000004	62.4509961871006\\
3725.20000000002	62.4567148954447\\
3725.69999999994	62.4483332438671\\
3726.30000000003	62.4543795620438\\
3726.60000000001	62.4520353127432\\
3726.99999999995	62.4560650371603\\
3727.20000000006	62.452713760631\\
3728.10000000005	62.4617778016201\\
3728.29999999997	62.4584272073812\\
3728.79999999995	62.4634610743115\\
3729.30000000002	62.4604494020486\\
3729.50000000004	62.4624624624625\\
3729.70000000003	62.4591130891737\\
3730.30000000005	62.4624705125456\\
3730.90000000002	62.4658268560708\\
3731.40000000002	62.4628165617044\\
3731.60000000003	62.4648283624086\\
3731.90000000006	62.459807073955\\
3732.10000000004	62.4618187664112\\
3732.30000000005	62.4584717607973\\
3732.69999999995	62.4624946420917\\
3733.09999999999	62.4584806600236\\
3733.89999999999	62.4638457418318\\
3734.1	62.4605002410155\\
3734.89999999999	62.4685408299866\\
3735.30000000001	62.4645285645446\\
3735.69999999999	62.4685475667862\\
3736.10000000005	62.4645361597345\\
3737.19999999994	62.4622053354026\\
3737.69999999997	62.459200599283\\
3738.4	62.4662297712987\\
3739.3	62.4725891854308\\
3739.5	62.4692480479196\\
3739.70000000003	62.4712551473341\\
3740.20000000004	62.4735983744619\\
3740.40000000003	62.4702579869002\\
3740.89999999998	62.4672547447207\\
3741.1	62.4692611996151\\
3741.8	62.4655923461343\\
3742.89999999998	62.4739513759017\\
3743.60000000002	62.4702834094612\\
3743.9	62.4732905982906\\
3744.10000000005	62.4699535281235\\
3744.80000000001	62.4636171860397\\
3745.10000000003	62.4666239453167\\
3745.59999999998	62.4689644125264\\
3745.90000000001	62.4639615589963\\
3746.30000000005	62.4599615631006\\
3747	62.4509620773398\\
3747.60000000006	62.4543053072551\\
3747.79999999995	62.450972544625\\
3748.19999999997	62.4549795907478\\
3748.99999999994	62.4576565042277\\
3749.80000000006	62.454998799968\\
3750.39999999999	62.4583388881482\\
3750.99999999999	62.4616779078137\\
3751.30000000006	62.4593485098897\\
3751.60000000003	62.4623504011515\\
3752.19999999997	62.4656877115369\\
3752.79999999995	62.4636947427323\\
3753.10000000002	62.4666950868592\\
3754.50000000004	62.4727001544772\\
3754.8	62.4703720471917\\
3755.09999999997	62.4680443118875\\
3756	62.4743750166396\\
3756.60000000005	62.472382676285\\
3756.80000000003	62.4743804732625\\
3757.49999999996	62.4787098147754\\
3757.69999999997	62.4753845335036\\
3757.99999999999	62.4783800324632\\
3758.60000000002	62.4817091015511\\
3758.9	62.4793828145783\\
3759.40000000003	62.4817129937492\\
3759.60000000001	62.4783892331835\\
3759.90000000004	62.4813829787234\\
3760.19999999997	62.4763981597213\\
3760.60000000004	62.4803892892281\\
3761.50000000003	62.4760740110591\\
3762.19999999996	62.4830555777051\\
3762.8	62.4810651359324\\
3763.09999999998	62.484056122449\\
3763.30000000002	62.4807355051283\\
3764.59999999994	62.488378888092\\
3765.30000000004	62.4926966590535\\
3766.69999999996	62.5013273866412\\
3767.40000000004	62.4976775049768\\
3768.09999999999	62.4913751923996\\
3768.39999999998	62.4890540002654\\
3768.70000000003	62.4867331776693\\
3769.09999999995	62.4907142099119\\
3770.29999999995	62.4946955230214\\
3770.50000000001	62.4913806821196\\
3771.20000000001	62.498342746533\\
3772.50000000003	62.4927105974659\\
3772.69999999999	62.4946988973707\\
3773.2	62.4917181247184\\
3773.99999999996	62.4864206035876\\
3774.29999999995	62.4841034336583\\
3774.79999999999	62.4890725582135\\
3774.99999999996	62.4857619665704\\
3775.30000000001	62.488742914658\\
3775.80000000003	62.4910617336264\\
3776.70000000001	62.494704511756\\
3776.99999999997	62.4923883402611\\
3777.40000000001	62.4963600264725\\
3778.4	62.4983459044594\\
3778.70000000005	62.4960304858685\\
3779.20000000002	62.4983462545974\\
3779.50000000001	62.4960313260662\\
3779.70000000001	62.4980157680301\\
3780.19999999995	62.5003306615877\\
3780.90000000003	62.5046284051838\\
3781.59999999998	62.5089245577386\\
3782.09999999996	62.5112368462799\\
3782.79999999998	62.5155304131751\\
3783.10000000005	62.5132163248044\\
3783.40000000002	62.5161887141536\\
3783.69999999996	62.513874940536\\
3784.49999999996	62.5217988691011\\
3784.9	62.5178335535007\\
3785.20000000006	62.5208041634745\\
3785.70000000006	62.517829784986\\
3786.00000000001	62.5155172869179\\
3787.00000000005	62.5174935966835\\
3787.29999999994	62.5125415852564\\
3787.59999999998	62.5102304828788\\
3787.80000000005	62.5122099316244\\
3788.10000000004	62.5072593844042\\
3788.30000000001	62.5092387287509\\
3788.60000000002	62.5069284979017\\
3788.8	62.5089075985114\\
3788.99999999999	62.5056081919189\\
3789.29999999996	62.5085765556553\\
3789.5	62.5052776018577\\
3790.00000000001	62.5102240046437\\
3790.30000000006	62.5079147319544\\
3790.79999999998	62.5102218470548\\
3791.00000000004	62.5069241117354\\
3791.60000000006	62.5128570298283\\
3791.80000000003	62.5095598512619\\
3792.09999999999	62.5125257106693\\
3793.19999999997	62.5102153797485\\
3793.99999999998	62.4996705411033\\
3794.70000000002	62.496047222515\\
3795.09999999996	62.5\\
3795.6	62.4970361198198\\
3795.90000000003	62.4947312961012\\
3796.19999999997	62.4976951241999\\
3796.99999999997	62.5029627873904\\
3797.69999999997	62.4993417241561\\
3798.30000000005	62.5026326874473\\
3799.69999999999	62.4901310595295\\
3799.99999999997	62.4930922870451\\
3800.50000000005	62.4953954638741\\
3800.80000000005	62.4930937409561\\
3801.09999999998	62.4960538777228\\
3801.39999999996	62.4937524661318\\
3801.90000000001	62.4907943187796\\
3802.20000000002	62.4937537806065\\
3802.49999999994	62.488823436596\\
3803.30000000001	62.4940842404165\\
3804	62.4904708078126\\
3804.8	62.4931010013404\\
3805.00000000005	62.4898162991774\\
3805.39999999997	62.4937590329786\\
3806.70000000006	62.4960596826731\\
3806.89999999998	62.4927764644077\\
3807.90000000003	62.5\\
3808.20000000003	62.4950765433396\\
3808.4	62.4970460811343\\
3808.80000000001	62.4904828165612\\
3809.70000000004	62.4940941781721\\
3810.20000000002	62.4963913602603\\
3810.50000000001	62.4914711593975\\
3810.69999999999	62.4934396977013\\
3810.89999999999	62.4901600629756\\
3811.90000000001	62.4947534102833\\
3812.50000000003	62.4980328384829\\
3813.30000000006	62.5032779147218\\
3813.60000000002	62.5009832970606\\
3813.90000000004	62.5039328788673\\
3815.40000000001	62.5003276110601\\
3815.69999999998	62.5032758530321\\
3816	62.4983622022484\\
3816.29999999998	62.501310135206\\
3816.59999999997	62.4963974113763\\
3816.99999999997	62.4924681040581\\
3817.40000000006	62.4963981663392\\
3817.79999999996	62.4924696822861\\
3818.20000000001	62.4885420213184\\
3818.70000000003	62.4855975699173\\
3819.10000000003	62.4895266024298\\
3819.40000000002	62.4846184055505\\
3820.4	62.4944379007983\\
3821.3	62.4901868425185\\
3821.80000000006	62.4950940631623\\
3821.99999999997	62.4918238664608\\
3823.80000000006	62.5094798504145\\
3824.10000000003	62.5045761204958\\
3824.70000000004	62.5104580631667\\
3825.49999999999	62.5078419071518\\
3825.79999999998	62.5107817768368\\
3826.29999999995	62.5078402676145\\
3826.9	62.5137183172198\\
3827.30000000003	62.5071850342269\\
3828.19999999996	62.5133871431184\\
3828.4	62.5101214574899\\
3828.59999999998	62.5120798182151\\
3829.20000000001	62.5075078996161\\
3829.69999999998	62.5097916340279\\
3830.19999999999	62.5120747722111\\
3830.50000000001	62.5071790320054\\
3831.2	62.5009787800485\\
3832.30000000004	62.4986953345162\\
3832.49999999999	62.5006522987006\\
3833.09999999999	62.5039131795889\\
3833.40000000005	62.5016303638972\\
3833.69999999997	62.5045646616934\\
3834.50000000005	62.5071715433161\\
3834.70000000002	62.5039115468864\\
3835.20000000003	62.5061924751649\\
3835.8	62.5042362939597\\
3836.00000000002	62.5061911837543\\
3836.19999999996	62.5029325130986\\
3837.09999999997	62.5117273011571\\
3837.30000000005	62.5084692760723\\
3837.5	62.5104231811549\\
3838.09999999998	62.5136782866969\\
3838.69999999998	62.5117224132541\\
3839.20000000006	62.5139999479072\\
3839.80000000003	62.5120445844944\\
3840.00000000001	62.5139970313273\\
3840.49999999994	62.5110659792741\\
3840.80000000005	62.5087870030462\\
3841.29999999999	62.5136668922788\\
3841.60000000001	62.5087851732306\\
3841.80000000004	62.5107368749837\\
3842.60000000002	62.5159393135035\\
3842.79999999996	62.5126857321294\\
3843.10000000005	62.5156119900083\\
3843.30000000003	62.5123588489358\\
3843.50000000002	62.514309501509\\
3843.69999999994	62.5110567667412\\
3843.99999999998	62.5087796883536\\
3844.59999999994	62.5042265976539\\
3845.50000000004	62.5\\
3846.10000000004	62.4980500233997\\
3846.9	62.5058487132831\\
3847.40000000004	62.508122157245\\
3847.59999999998	62.5048730410375\\
3847.9	62.5077962577963\\
3848.89999999996	62.5123408677579\\
3849.70000000001	62.5097407657541\\
3849.99999999997	62.5074673385107\\
3850.69999999996	62.5116858834528\\
3850.90000000004	62.5084393663983\\
3851.30000000003	62.5045438022537\\
3851.70000000006	62.5084376135833\\
3852.29999999994	62.5064894611152\\
3852.9	62.5045419153906\\
3853.89999999999	62.5064867669953\\
3854.29999999997	62.502594437526\\
3854.59999999999	62.505512750668\\
3855.3	62.5019453234424\\
3855.50000000005	62.5038904450669\\
3855.79999999995	62.501620892658\\
3856.10000000004	62.5045381463617\\
3856.3	62.5012965460015\\
3856.80000000001	62.5061577951204\\
3857.09999999999	62.5038888312766\\
3857.80000000003	62.5106923455766\\
3858.30000000002	62.5129587393738\\
3858.90000000004	62.516195905675\\
3859.09999999999	62.512956053068\\
3859.39999999999	62.5158699313383\\
3859.90000000002	62.5181347150259\\
3860.30000000004	62.5142472282665\\
3860.60000000003	62.5119796928018\\
3861.09999999996	62.5168341448254\\
3861.70000000006	62.5200683619038\\
3862.29999999998	62.5103562551781\\
3862.60000000004	62.5132679214021\\
3862.80000000005	62.510031323617\\
3863.19999999998	62.5061475940258\\
3863.99999999994	62.5113221707513\\
3865.00000000004	62.5158469379835\\
3865.29999999999	62.5135820354944\\
3865.90000000003	62.5168132436627\\
3866.80000000002	62.5126069978536\\
3867.30000000004	62.5096964368827\\
3868.19999999995	62.5132487139053\\
3869.59999999996	62.5190583249348\\
3870.20000000003	62.5145337570731\\
3870.39999999999	62.5164707402144\\
3870.59999999997	62.5132404991345\\
3873.59999999997	62.5164571340062\\
3873.90000000003	62.5116159008777\\
3874.20000000006	62.5093565289214\\
3874.50000000002	62.5122593299954\\
3875.10000000002	62.5077415359207\\
3876.00000000005	62.5112871184954\\
3876.20000000003	62.5080618115213\\
3876.6	62.5119302499549\\
3876.80000000001	62.5087054089608\\
3877.3	62.5109609532161\\
3877.49999999998	62.507736744378\\
3878.30000000003	62.5\\
3878.89999999997	62.4980665119876\\
3879.30000000001	62.4942001340413\\
3879.89999999999	62.4922680412371\\
3880.39999999998	62.4945239015591\\
3880.70000000001	62.4896928468357\\
3880.90000000006	62.4916258696212\\
3881.40000000001	62.4835759371377\\
3881.90000000002	62.4858320453375\\
3882.39999999999	62.4880875724404\\
3882.69999999996	62.4832595034511\\
3883.00000000004	62.4861579665731\\
3883.29999999997	62.4813307926044\\
3883.70000000003	62.4851949121994\\
3885.00000000007	62.4900259967568\\
3885.49999999997	62.4871319744698\\
3885.80000000003	62.49002805013\\
3886.09999999996	62.4877772631362\\
3886.7	62.4909951631162\\
3887.29999999995	62.4890672428873\\
3888.19999999998	62.495177841216\\
3889.20000000001	62.4996786054046\\
3889.40000000006	62.4964648412392\\
3890.80000000005	62.50224883703\\
3891.70000000001	62.5083508916183\\
3892.00000000005	62.5035327972046\\
3892.5	62.5006422442583\\
3893.29999999998	62.5083474598038\\
3894.29999999996	62.5102711585867\\
3895.00000000005	62.5144412210213\\
3895.19999999999	62.5112314840962\\
3895.39999999999	62.5131562058786\\
3895.89999999996	62.5102669404517\\
3896.50000000003	62.5032079248576\\
3896.7	62.5051324163416\\
3897.30000000006	62.5083388925951\\
3897.9	62.5038481272448\\
3898.6	62.5080154923436\\
3898.79999999994	62.5048090487061\\
3899.20000000004	62.5086553996871\\
3899.40000000001	62.5054494165919\\
3899.59999999997	62.5073723619766\\
3899.89999999995	62.5051282051282\\
3900.50000000001	62.5108957596267\\
3901.40000000002	62.5067281814687\\
3901.89999999994	62.5012813941568\\
3902.10000000002	62.5032033212034\\
3902.40000000005	62.5009609224856\\
3902.70000000002	62.4987188685047\\
3903.09999999999	62.5025620004099\\
3903.49999999999	62.4987191310585\\
3904.09999999995	62.5019210081451\\
3904.70000000006	62.4948780987502\\
3905.39999999994	62.4913583408014\\
3905.69999999994	62.4942393363715\\
3905.90000000003	62.4910394265233\\
3906.19999999999	62.4888001433582\\
3906.69999999994	62.4910412613904\\
3907.90000000002	62.4872057318321\\
3908.40000000005	62.4843290264808\\
3908.90000000004	62.4891276541315\\
3909.50000000003	62.4923265807244\\
3909.79999999995	62.4900892605949\\
3910.20000000004	62.4939262972151\\
3910.90000000003	62.4904116594222\\
3911.60000000006	62.4843418462561\\
3911.80000000006	62.4862598737187\\
3911.99999999994	62.4830653613149\\
3912.29999999995	62.4859421327063\\
3914.19999999998	62.4964872390977\\
3914.70000000003	62.4987227955451\\
3915.00000000005	62.4964879568849\\
3915.19999999998	62.4984036983118\\
3915.89999999998	62.5025536261491\\
3916.49999999998	62.4955318388398\\
3916.69999999998	62.4974468954248\\
3917.00000000001	62.4952132955503\\
3917.39999999997	62.4990427568602\\
3917.59999999999	62.495852158154\\
3918	62.499680967816\\
3918.19999999996	62.4964908251027\\
3919.19999999997	62.5035082795397\\
3919.89999999996	62.5076530612245\\
3920.10000000005	62.5044640579562\\
3920.69999999995	62.502550499898\\
3921.10000000001	62.5063755993063\\
3921.29999999996	62.5031876370684\\
3922.29999999997	62.5\\
3922.50000000003	62.5019119971448\\
3923	62.4990441232699\\
3923.19999999997	62.5009558280019\\
3923.59999999995	62.4971328083187\\
3924.10000000002	62.501911217573\\
3924.30000000003	62.4987259198858\\
3924.8	62.5035032739688\\
3925.49999999994	62.4949052374159\\
3925.79999999999	62.4977712116967\\
3926.90000000006	62.5031830914184\\
3927.19999999999	62.4984085758664\\
3927.79999999994	62.4964994017159\\
3927.99999999994	62.4984088999771\\
3928.29999999998	62.4936360859383\\
3928.49999999998	62.4955454869419\\
3929.60000000001	62.4984095478026\\
3929.79999999997	62.4952288862312\\
3930.00000000004	62.4971374774179\\
3930.19999999997	62.4939572042847\\
3930.39999999999	62.4958656659458\\
3930.80000000005	62.4920501666285\\
3931.80000000006	62.4965029629441\\
3932.1	62.4942780123086\\
3932.60000000005	62.4888753273832\\
3933.59999999997	62.4958690291583\\
3934.39999999998	62.490786631084\\
3935	62.4965058067139\\
3935.29999999995	62.4917416272806\\
3935.69999999994	62.4955536358555\\
3936.19999999997	62.4876152732261\\
3936.5	62.4853934867652\\
3936.69999999998	62.4872993294046\\
3937.30000000002	62.4904759485955\\
3937.70000000006	62.4841281934075\\
3938.19999999995	62.4787344793439\\
3939.60000000004	62.4844531309491\\
3939.80000000003	62.4812812507932\\
3940.29999999997	62.4784285859304\\
3940.90000000004	62.481603653895\\
3941.19999999999	62.4793849745008\\
3941.39999999995	62.4812888494228\\
3941.7	62.4790704754173\\
3942.50000000005	62.4866839141683\\
3942.90000000004	62.4828810550342\\
3943.30000000006	62.4790789673885\\
3944.49999999994	62.4879582213659\\
3944.70000000005	62.4847901034273\\
3944.99999999998	62.4825733188005\\
3945.79999999996	62.4876454040903\\
3946.20000000006	62.481311608342\\
3946.40000000003	62.4832129735208\\
3947.30000000004	62.4867001063992\\
3947.59999999994	62.4844846366239\\
3948.00000000001	62.4882855044198\\
3948.50000000005	62.4854378767158\\
3949.00000000001	62.4876554151579\\
3949.30000000005	62.4854408264546\\
3949.69999999999	62.489239961517\\
3950.00000000004	62.4844940634414\\
3950.79999999997	62.4895593409097\\
3950.99999999994	62.4863961934651\\
3951.30000000001	62.4841828212785\\
3952.19999999995	62.4876654100144\\
3952.80000000004	62.4857699410559\\
3953.30000000003	62.4829260889361\\
3954.20000000004	62.491464987482\\
3954.50000000003	62.4892530217974\\
3955.29999999997	62.4968397633615\\
3955.80000000003	62.491468439546\\
3956.00000000002	62.4933646773337\\
3956.79999999999	62.4958932497662\\
3957.09999999997	62.4936824016982\\
3957.50000000006	62.4974732160906\\
3957.70000000004	62.4943150234979\\
3958.20000000002	62.4889472753455\\
3958.39999999997	62.4908424908425\\
3958.69999999994	62.4886329190664\\
3960.39999999998	62.494634515844\\
3961.1	62.4861153185903\\
3961.30000000004	62.4880092896451\\
3962.5	62.4917983142381\\
3962.79999999998	62.489590956118\\
3962.99999999995	62.4914839393404\\
3963.30000000002	62.4892768834839\\
3963.60000000005	62.4921159522668\\
3964.70000000006	62.4949556093624\\
3964.90000000004	62.4918032786885\\
3965.50000000004	62.4899132539843\\
3965.69999999995	62.4918049321701\\
3966.30000000006	62.4899152884228\\
3966.69999999996	62.4936976908339\\
3967.29999999997	62.4892876947119\\
3968.09999999997	62.4918098886145\\
3968.29999999999	62.4886604172967\\
3968.49999999994	62.4905508239681\\
3968.89999999995	62.4842529604434\\
3969.2	62.4820497316907\\
3969.40000000001	62.4839400428266\\
3970.20000000005	62.486461980203\\
3970.50000000003	62.4817407948421\\
3970.89999999998	62.4855200201461\\
3971.10000000002	62.4823730862208\\
3971.40000000007	62.4852071005917\\
3971.80000000004	62.4814320602231\\
3972.19999999998	62.4852100798026\\
3972.50000000001	62.4830086089715\\
3972.79999999999	62.485841576682\\
3973.50000000001	62.4899335615059\\
3974.10000000005	62.4880479090131\\
3974.30000000003	62.4899355877617\\
3974.99999999999	62.4940253075394\\
3975.19999999998	62.4908811913566\\
3975.60000000004	62.494655029303\\
3976.10000000003	62.4867964388109\\
3976.29999999997	62.4886832310633\\
3976.90000000004	62.4867990947951\\
3977.19999999997	62.4896286425464\\
3977.59999999999	62.4858586620409\\
3977.80000000002	62.4877447899646\\
3978.09999999994	62.4855462269368\\
3978.69999999996	62.4912033779029\\
3979.09999999999	62.4874346602332\\
3979.29999999998	62.4893199979897\\
3980.00000000006	62.4934046883244\\
3980.60000000003	62.4915215916799\\
3981.19999999997	62.4946625474091\\
3981.79999999997	62.492779828725\\
3981.99999999995	62.4946636196982\\
3982.29999999994	62.4924668541583\\
3982.50000000004	62.4943504243459\\
3983.4	62.4978034391867\\
3983.59999999994	62.4946657629842\\
3984.19999999997	62.5003137313957\\
3984.50000000007	62.4981177533504\\
3984.69999999997	62.5\\
3985.30000000002	62.4981181311788\\
3985.69999999996	62.4943549601084\\
3986.00000000003	62.4921602568927\\
3986.29999999995	62.4899658840056\\
3986.50000000006	62.4918476897607\\
3987.10000000005	62.4874598715891\\
3987.70000000001	62.4931039670997\\
3988.50000000002	62.4981196409768\\
3989.00000000004	62.492792860545\\
3989.20000000005	62.4946732509463\\
3989.90000000005	62.4912280701754\\
3990.10000000006	62.4931081148815\\
3991.19999999998	62.4884122967454\\
3991.50000000001	62.4862210642349\\
3991.80000000005	62.4840301610762\\
3992.1	62.481839587195\\
3992.39999999998	62.4846587351284\\
3993.1	62.4887308424321\\
3994.20000000007	62.4840397566532\\
3994.40000000003	62.485918137439\\
3994.89999999997	62.4831038798498\\
3995.30000000004	62.4868598888722\\
3995.89999999995	62.484984984985\\
3996.10000000002	62.4868625193934\\
3997.19999999998	62.484677157081\\
3997.99999999999	62.4896825992346\\
3998.69999999995	62.4837451235371\\
3999.19999999995	62.4884354762083\\
3999.49999999996	62.4837483748375\\
3999.8	62.4865621640541\\
4000	62.4834379140522\\
4000.29999999996	62.4862513748625\\
4000.50000000002	62.4831275308704\\
4000.70000000006	62.4850029994001\\
4000.99999999998	62.4828172252631\\
4001.49999999995	62.4800079968013\\
4002.10000000001	62.4781370246365\\
4002.60000000002	62.4828240937367\\
4002.90000000005	62.4781413939545\\
4003.60000000002	62.4697155131503\\
4003.99999999994	62.4734646986838\\
4004.40000000003	62.4672243725808\\
4004.60000000001	62.4690988088995\\
4004.9	62.4669163545568\\
4005.59999999998	62.4734752977008\\
4005.80000000003	62.4703562245688\\
4006.00000000007	62.4722298494795\\
4006.2	62.4691111499389\\
4006.69999999999	62.4638115204153\\
4007.49999999995	62.4663140033935\\
4008.10000000002	62.4644478818422\\
4008.6	62.4591513458228\\
4009.09999999999	62.4638331836776\\
4009.89999999998	62.4688279301746\\
4010.19999999995	62.4641548013864\\
4010.50000000005	62.4619757642248\\
4011.90000000006	62.4651046859422\\
4012.20000000001	62.4604341649428\\
4012.50000000002	62.4582564920501\\
4012.7	62.4601275917065\\
4012.89999999999	62.4570147022178\\
4013.50000000003	62.4626270679689\\
4013.80000000004	62.4579585938862\\
4013.99999999996	62.459829102414\\
4014.2	62.4567172358817\\
4014.69999999997	62.4589020623692\\
4014.99999999996	62.456725859879\\
4015.39999999997	62.4604656954302\\
4015.99999999999	62.4635840741017\\
4016.89999999995	62.4670151854618\\
4017.50000000001	62.4601752289924\\
4017.99999999995	62.464846569274\\
4018.30000000005	62.4626717101334\\
4019.00000000003	62.4692095245204\\
4019.20000000005	62.4661010623741\\
4019.7	62.4633066321708\\
4020.19999999994	62.4654876501754\\
4020.50000000001	62.4633139332438\\
4020.70000000004	62.4651810584958\\
4020.89999999996	62.4620741109177\\
4021.19999999997	62.4648745430582\\
4021.50000000003	62.4627014123732\\
4021.79999999997	62.4605286058828\\
4022.80000000001	62.4574312063437\\
4023.00000000006	62.4592975566106\\
4023.69999999995	62.4558874695561\\
4024.50000000001	62.4633503950703\\
4024.89999999997	62.4571428571429\\
4025.19999999997	62.4549723002012\\
4026.60000000002	62.4531253880349\\
4027.00000000005	62.4568548086713\\
4027.49999999999	62.4590326745456\\
4027.90000000004	62.4553128103277\\
4028.49999999995	62.4584222806931\\
4029.20000000001	62.4624624624625\\
4029.60000000006	62.4587438270839\\
4029.79999999997	62.4606069629519\\
4030.10000000001	62.458438787157\\
4030.40000000005	62.4562709341273\\
4030.60000000003	62.4581338229092\\
4031.20000000005	62.4612407908119\\
4031.40000000004	62.4581421307206\\
4031.59999999995	62.4600044646179\\
4031.9	62.4578373015873\\
4032.30000000006	62.4615613530404\\
4032.7	62.4578456655426\\
4033.00000000005	62.4556792541717\\
4033.39999999999	62.4594025040288\\
4034.49999999995	62.4646805135577\\
4034.80000000001	62.46251456046\\
4035.19999999994	62.46623547196\\
4035.70000000006	62.4684077506319\\
4036.00000000001	62.4662421644657\\
4036.3	62.4690318105242\\
4036.49999999997	62.4659366793836\\
4036.7	62.4677962742767\\
4039.50000000005	62.4814338053273\\
4039.70000000004	62.4783405119065\\
4039.89999999998	62.480198019802\\
4040.6	62.484223030663\\
4042.30000000006	62.4975262220463\\
4042.80000000006	62.4947438719731\\
4043.10000000004	62.4975267115156\\
4043.30000000003	62.4944353761686\\
4043.50000000004	62.4962904342665\\
4044.20000000005	62.4928912296318\\
4044.50000000003	62.4907283785789\\
4044.80000000004	62.4885658483523\\
4045.09999999997	62.4864036388807\\
4045.29999999994	62.4882582686508\\
4046.2	62.4941304401552\\
4046.8	62.4898070127752\\
4047.29999999998	62.4944408756239\\
4047.50000000006	62.4913529004842\\
4047.79999999994	62.4941327601966\\
4048.49999999996	62.4907375389024\\
4048.69999999997	62.4925903971547\\
4048.99999999994	62.4904299720926\\
4049.30000000001	62.4882698671408\\
4050.19999999995	62.494136236822\\
4051.00000000005	62.4892004640715\\
4051.19999999999	62.4910522548318\\
4051.40000000005	62.4879674194743\\
4051.70000000007	62.4858087763463\\
4051.90000000002	62.4876604146101\\
4052.40000000004	62.484885872918\\
4053.00000000005	62.4879721694505\\
4053.49999999998	62.4901322281429\\
4053.79999999996	62.4855077826291\\
4054.10000000004	62.4833505993784\\
4054.39999999998	62.4811937353558\\
4054.70000000007	62.4839696162573\\
4055.99999999994	62.4787357313676\\
4056.3	62.4815106991421\\
4056.80000000004	62.4787399245729\\
4057.20000000001	62.4824390604589\\
4057.70000000002	62.4796687860417\\
4058.20000000004	62.4744351082966\\
4058.40000000002	62.4762843415055\\
4059.09999999994	62.4802916830903\\
4059.50000000001	62.4741353827963\\
4059.7	62.4759840386226\\
4059.90000000003	62.4729064039409\\
4060.29999999995	62.4766032903162\\
4060.59999999995	62.4719875883468\\
4061.10000000003	62.4766078991431\\
4061.30000000006	62.4735312946275\\
4061.49999999994	62.4753791609218\\
4062.09999999994	62.4784599478115\\
4063.19999999994	62.4812344645977\\
4063.80000000006	62.4744703363764\\
4064.30000000005	62.4790867040646\\
4064.50000000002	62.4760123997441\\
4064.79999999994	62.4787817658491\\
4065.39999999999	62.4769401057681\\
4065.89999999994	62.4741760944417\\
4066.49999999997	62.4723356120592\\
4066.70000000003	62.4741811743877\\
4067.20000000001	62.4763356526443\\
4067.79999999995	62.4720371690553\\
4067.99999999996	62.4738821562892\\
4068.19999999997	62.4708109038173\\
4068.39999999994	62.4726557699398\\
4068.99999999997	62.4683590966061\\
4069.5	62.4705130725378\\
4069.70000000002	62.4674431175979\\
4070.30000000004	62.4631485849057\\
4070.70000000003	62.4594674265501\\
4071.49999999994	62.4668435013263\\
4071.9	62.4607072691552\\
4072.29999999996	62.4643944602691\\
4072.79999999999	62.4665471776867\\
4073.10000000003	62.4644014534027\\
4074.19999999999	62.4622634562992\\
4074.79999999999	62.4677906206287\\
4075.10000000001	62.465645857872\\
4075.80000000001	62.46227826983\\
4076.29999999999	62.4668825434207\\
4076.60000000004	62.4647386366424\\
4076.79999999997	62.4665799995094\\
4076.99999999997	62.4635157342229\\
4077.50000000006	62.4681185010791\\
4077.99999999994	62.465363772345\\
4078.69999999999	62.4718054329705\\
4078.99999999994	62.4672109043662\\
4079.30000000002	62.4650683924107\\
4079.60000000006	62.4678285168027\\
4079.80000000005	62.4647662932915\\
4080.09999999999	62.4675261016617\\
4081.80000000007	62.4733579950513\\
4082.19999999998	62.4696862063053\\
4082.60000000002	62.4733632155191\\
4082.80000000004	62.4703029709275\\
4083.10000000004	62.4730603448276\\
4083.3	62.4700004897879\\
4083.60000000003	62.4727575483018\\
4083.90000000004	62.4681684622919\\
4084.40000000005	62.4727628840739\\
4085.10000000001	62.4767453245863\\
4085.70000000001	62.4700181115082\\
4085.9	62.4718551150269\\
4086.10000000002	62.4687974156919\\
4086.7	62.4718606244494\\
4087.99999999999	62.4691176830312\\
4088.70000000001	62.4755429465858\\
4089.40000000002	62.4795207238049\\
4090	62.4752451040317\\
4090.29999999994	62.4731077645218\\
4091.30000000006	62.479835753043\\
4091.49999999995	62.4767816990908\\
4091.69999999997	62.4786157681216\\
4091.99999999996	62.4764790694265\\
4092.90000000004	62.4798436354752\\
4093.09999999997	62.4767907749438\\
4093.60000000004	62.474045484525\\
4094.49999999995	62.4627558247448\\
4094.7	62.4645892351275\\
4095.30000000005	62.467646627924\\
4095.50000000002	62.464596151968\\
4096.00000000001	62.4618539586436\\
4096.49999999996	62.4664355807255\\
4097.10000000005	62.4597285951381\\
4097.99999999998	62.4679729630805\\
4098.19999999997	62.4649244808823\\
4099.29999999997	62.4627994340635\\
4099.69999999998	62.4664617786234\\
4100.20000000002	62.4685998585469\\
4101.30000000006	62.4762276295899\\
4101.49999999998	62.4731811975814\\
4101.90000000003	62.4768405655778\\
4102.79999999995	62.4704477320919\\
4103.10000000004	62.4731916552934\\
4103.8	62.4771558761178\\
4104.10000000002	62.4750255835486\\
4104.70000000003	62.4805106217112\\
4104.89999999996	62.4774665042631\\
4105.50000000002	62.4756430241621\\
4105.99999999995	62.4802123669662\\
4106.39999999995	62.4741263849994\\
4106.69999999994	62.4719976624136\\
4107.40000000005	62.4637857577602\\
4107.69999999996	62.4665270947953\\
4108.69999999999	62.4634929906542\\
4108.89999999996	62.4653200292042\\
4109.10000000004	62.4622797624842\\
4109.40000000006	62.4601533033216\\
4109.69999999997	62.4628935714633\\
4109.89999999998	62.4598540145985\\
4110.19999999997	62.462593971243\\
4110.70000000006	62.4647270604262\\
4110.89999999994	62.4616881537339\\
4111.09999999999	62.4635143023935\\
4111.59999999996	62.4656468127539\\
4111.80000000003	62.462608526472\\
4112.10000000001	62.4604834395214\\
4112.60000000005	62.4577528144528\\
4113.19999999999	62.4632290375125\\
4113.39999999995	62.4601920505652\\
4113.89999999994	62.4623237724842\\
4114.10000000006	62.4592873462642\\
4114.30000000002	62.461112191328\\
4114.59999999995	62.456558193793\\
4115.80000000004	62.4626448650356\\
4116.39999999994	62.4656868699138\\
4116.69999999996	62.4611348620288\\
4117.40000000002	62.4578020643595\\
4117.70000000006	62.4556802175919\\
4118.30000000005	62.4538655788656\\
4118.90000000007	62.4593347899976\\
4119.20000000001	62.4547860073314\\
4119.40000000001	62.456608811749\\
4120.19999999999	62.4541902288668\\
4120.60000000001	62.4578348338874\\
4121.19999999998	62.4560211583724\\
4121.90000000006	62.45269286754\\
4122.10000000005	62.4545145795934\\
4122.50000000005	62.4508805122981\\
4123.30000000003	62.4484648590969\\
4123.49999999998	62.4502861577263\\
4123.79999999995	62.4457431072528\\
4124.90000000002	62.4484848484849\\
4126.19999999997	62.4409277076315\\
4126.4	62.4427480916031\\
4126.69999999997	62.4382087816226\\
4127.1	62.4418491955805\\
4127.50000000005	62.435798042446\\
4127.90000000005	62.4394379844961\\
4128.40000000001	62.4415647329539\\
4128.99999999995	62.4397568477392\\
4129.20000000006	62.4415760540528\\
4129.39999999997	62.4385518827945\\
4129.80000000006	62.434925785128\\
4130.19999999999	62.438563784713\\
4130.4	62.4355404914659\\
4131.30000000006	62.4388827031999\\
4131.8	62.4410077688231\\
4132	62.4379855279398\\
4132.50000000001	62.4425301263127\\
4133.30000000004	62.4377026176997\\
4133.70000000007	62.4413372683729\\
4134.20000000006	62.4434608035218\\
4134.40000000006	62.4404401983311\\
4134.89999999996	62.4449818621524\\
4135.50000000002	62.4431763226618\\
4135.79999999999	62.4459005295099\\
4135.99999999998	62.4428809748314\\
4136.29999999994	62.4456048738033\\
4137.10000000002	62.4480324857391\\
4137.50000000002	62.4419953596288\\
4137.89999999999	62.4456259062349\\
4138.10000000005	62.4426078971534\\
4138.30000000004	62.4444229653973\\
4139.29999999999	62.4486640575929\\
4140.40000000006	62.4465644245864\\
4140.70000000007	62.4444551777434\\
4141.20000000007	62.4465747470601\\
4141.39999999999	62.4435590969456\\
4142.60000000007	62.4399546189683\\
4143.10000000001	62.4444873527708\\
4143.30000000006	62.4414731862721\\
4143.79999999996	62.4460049711624\\
4144.30000000003	62.4384711900396\\
4144.70000000001	62.4348581354951\\
4144.99999999994	62.4327519239584\\
4145.49999999995	62.4372829023543\\
4145.69999999995	62.4342708283082\\
4145.89999999994	62.4360829715388\\
4146.20000000003	62.4339772809493\\
4147.00000000005	62.4412239878469\\
4147.40000000005	62.4376130198915\\
4147.7	62.4403298133951\\
4147.89999999996	62.4373191899711\\
4148.09999999999	62.4391302251579\\
4149.29999999994	62.447582783053\\
4149.70000000002	62.4439732035279\\
4150.30000000001	62.4494024672321\\
4150.60000000006	62.4472980461127\\
4151.00000000003	62.4509166245092\\
4151.70000000004	62.4476130834819\\
4152	62.4503263408877\\
4152.29999999998	62.445814468741\\
4152.59999999999	62.4485274640595\\
4152.99999999996	62.4449206616744\\
4153.50000000007	62.439811248074\\
4153.69999999995	62.4416197217006\\
4154.09999999998	62.4356073371528\\
4154.49999999996	62.4392239926828\\
4154.70000000001	62.4362183498604\\
4155.2	62.4407383341756\\
4156.00000000002	62.4431558432184\\
4156.60000000003	62.4413597324801\\
4157.20000000004	62.4467803622544\\
4157.40000000002	62.4437763078773\\
4157.60000000003	62.4455828943887\\
4158.30000000003	62.4494998076183\\
4158.50000000002	62.4464964170634\\
4159.10000000007	62.4519138295826\\
4160.19999999999	62.4498233300483\\
4160.70000000005	62.4447221688137\\
4161.00000000003	62.4474297661676\\
4162.20000000004	62.4534512168753\\
4162.59999999994	62.4498522593509\\
4162.90000000003	62.4477540235407\\
4163.40000000005	62.4522637204275\\
4163.7	62.4501657140112\\
4164.60000000005	62.4558791749706\\
4165.69999999996	62.4513898890969\\
4165.99999999994	62.4540937567509\\
4166.39999999999	62.4504980199208\\
4166.69999999994	62.4484016511472\\
4167.10000000006	62.4448070646957\\
4167.80000000005	62.441517310876\\
4167.99999999996	62.4433194980927\\
4168.30000000001	62.4388254486134\\
4169.10000000007	62.4364386453037\\
4169.50000000002	62.432847275518\\
4170.30000000003	62.4400537118742\\
4170.99999999999	62.4319723813862\\
4171.60000000002	62.4349785459165\\
};
\addplot [color=mycolor3, forget plot]
  table[row sep=crcr]{%
4171.60000000002	62.4349785459165\\
4172.29999999999	62.4388840954846\\
4172.69999999999	62.4352952453988\\
4172.90000000007	62.4370956146657\\
4173.80000000003	62.4427993004145\\
4174.10000000003	62.4383115327488\\
4174.40000000005	62.4410108995089\\
4174.70000000001	62.4389192296637\\
4176.59999999999	62.4416405296047\\
4176.89999999999	62.4395499162078\\
4177.10000000003	62.4413482715695\\
4177.59999999998	62.443449745075\\
4177.80000000003	62.4404605184423\\
4177.99999999997	62.4422584428329\\
4178.29999999996	62.4377752249665\\
4178.49999999999	62.4395730627483\\
4178.69999999999	62.4365846654542\\
4178.89999999999	62.4383823881311\\
4179.30000000004	62.4324065655357\\
4179.50000000003	62.4342042300699\\
4180.19999999993	62.4381025285267\\
4180.49999999996	62.4336219681385\\
4180.69999999995	62.4354190585534\\
4181.20000000003	62.4279530289623\\
4181.89999999998	62.4318507890961\\
4182.09999999998	62.4288651905696\\
4182.59999999997	62.433356444402\\
4182.79999999999	62.4303712735184\\
4183.39999999998	62.4261981594359\\
4183.80000000001	62.4226200435001\\
4184.40000000004	62.4208388098937\\
4184.59999999997	62.4226348364279\\
4185.00000000006	62.4166686578576\\
4185.29999999996	62.4145840302002\\
4185.60000000006	62.4124997013642\\
4185.90000000001	62.41519350215\\
4186.20000000002	62.4131094283735\\
4187.00000000003	62.4155143177856\\
4187.29999999997	62.4110426517648\\
4187.50000000004	62.4128379023785\\
4187.69999999998	62.40985720426\\
4187.89999999999	62.4116523400191\\
4188.50000000002	62.4146492861577\\
4189	62.4119739323482\\
4189.49999999997	62.4140729425243\\
4190.09999999997	62.4075223139707\\
4190.30000000003	62.4093165330279\\
4191.49999999997	62.4129210802557\\
4191.69999999993	62.409943222482\\
4192.09999999998	62.4135298888412\\
4192.99999999998	62.4096730342706\\
4193.59999999993	62.4126666189761\\
4194.6	62.4144754094452\\
4195.19999999997	62.4126999261078\\
4195.40000000001	62.4144917173162\\
4195.60000000003	62.4115165526611\\
4195.79999999994	62.4133082294621\\
4196.60000000006	62.4180904043653\\
4197.20000000004	62.41631525028\\
4197.4	62.4181060154854\\
4198.39999999995	62.4199118732881\\
4199.00000000002	62.4181372198805\\
4199.29999999997	62.4160594370624\\
4199.59999999995	62.418744196014\\
4199.90000000006	62.4166666666667\\
4200.10000000004	62.4184562639874\\
4200.29999999994	62.4154842395962\\
4201.00000000003	62.4193663564305\\
4201.20000000006	62.4163949253802\\
4201.39999999999	62.4181839819112\\
4202.09999999998	62.4101661034696\\
4202.60000000005	62.4146382087706\\
4203.30000000005	62.4185183422943\\
4203.79999999996	62.4206094340969\\
4204.20000000001	62.4146706942892\\
4204.80000000007	62.4176555922852\\
4205.19999999999	62.4140964972773\\
4205.50000000005	62.412022065817\\
4206.60000000006	62.4099650557444\\
4207.90000000006	62.4215779467681\\
4208.19999999993	62.4195043129055\\
4208.69999999994	62.4239688272192\\
4209.00000000004	62.4218954170725\\
4209.40000000004	62.4254662073881\\
4209.59999999999	62.4225004157066\\
4210.50000000004	62.4281575072436\\
4210.79999999999	62.4260846849842\\
4211.29999999997	62.4281711544855\\
4211.70000000001	62.4222422717128\\
4212.49999999994	62.4270047001852\\
4213.09999999996	62.4252349757904\\
4213.39999999994	62.4231636406788\\
4213.70000000005	62.4258389102473\\
4214.00000000003	62.4237678270568\\
4214.29999999999	62.4264426727411\\
4214.59999999993	62.4243718414122\\
4215.50000000002	62.4300218237024\\
4216.60000000003	62.4350795645884\\
4217.10000000001	62.4300483733283\\
4217.4	62.4279786603438\\
4217.59999999998	62.4297602958959\\
4217.9	62.4276908487435\\
4218.4	62.4297736162143\\
4218.59999999996	62.4268139474246\\
4218.90000000002	62.429485660109\\
4219.10000000003	62.4265263557073\\
4219.60000000006	62.4286086688627\\
4219.9	62.4241706161137\\
4220.40000000003	62.4262528136477\\
4220.60000000004	62.4232947141469\\
4220.80000000005	62.4250752209244\\
4221.80000000002	62.4292380207963\\
4222.2	62.4256921583024\\
4222.39999999995	62.4274718768502\\
4224.00000000004	62.4346014535641\\
4224.20000000004	62.4316454797245\\
4225.59999999995	62.4393591594292\\
4227.60000000002	62.4547626368948\\
4227.8	62.4518082263062\\
4228.00000000003	62.4535843523095\\
4228.3	62.4491533440545\\
4228.49999999999	62.4509293856123\\
4229.00000000004	62.4530041852877\\
4229.50000000002	62.4503499148856\\
4230.09999999994	62.4485839913007\\
4230.59999999994	62.4530219585411\\
4231.39999999999	62.457757296467\\
4231.69999999993	62.4533295524363\\
4232.19999999997	62.4506769368901\\
4232.39999999998	62.452451269935\\
4232.69999999998	62.448024948025\\
4233.00000000004	62.445961588434\\
4233.30000000003	62.4438985212831\\
4233.9	62.4421350968352\\
4234.20000000004	62.4447960701887\\
4234.50000000002	62.440372172106\\
4235.09999999997	62.4433320740461\\
4235.70000000001	62.4415694792011\\
4235.89999999998	62.443342776204\\
4236.09999999996	62.4403946933573\\
4236.39999999994	62.4430544081199\\
4236.69999999996	62.4386329305136\\
4237.00000000006	62.4412923933823\\
4237.30000000001	62.4392316042857\\
4237.69999999998	62.442776912549\\
4238.30000000004	62.4339373348433\\
4239.29999999999	62.442798509223\\
4239.50000000002	62.4398528163034\\
4239.99999999999	62.437206669654\\
4240.40000000006	62.440749911567\\
4241.10000000003	62.4375176836744\\
4241.49999999995	62.4339871746511\\
4242.10000000004	62.4369430955636\\
4243.00000000003	62.4307699559285\\
4243.79999999997	62.4331393293904\\
4244.00000000002	62.4301972149572\\
4244.40000000005	62.4266698079868\\
4245.39999999999	62.435519962313\\
4246.20000000001	62.4402420931164\\
4246.80000000006	62.4384845416657\\
4247.00000000001	62.4402533493443\\
4247.29999999997	62.4358431040166\\
4247.89999999996	62.4387947269303\\
4248.60000000003	62.4355685268435\\
4249.19999999998	62.440872614313\\
4249.99999999995	62.4455895155408\\
4250.59999999996	62.4485378878773\\
4250.80000000004	62.4455997553459\\
4251.2	62.4491332063134\\
4251.99999999998	62.4514945556313\\
4253.39999999994	62.4497472669566\\
4253.59999999997	62.4515128006206\\
4254.80000000001	62.4574020541023\\
4255.09999999999	62.4529986839631\\
4255.49999999998	62.4494783344299\\
4256.39999999994	62.4574180664866\\
4256.89999999996	62.4547803617571\\
4257.10000000001	62.4565442074603\\
4257.4	62.4544920728127\\
4257.70000000002	62.4524402273475\\
4257.90000000005	62.4542038515735\\
4258.50000000004	62.4524491616963\\
4258.70000000002	62.4542124542125\\
4258.89999999994	62.4512796431087\\
4259.10000000004	62.4530428249437\\
4259.5	62.4495257770683\\
4259.89999999997	62.4530516431925\\
4260.09999999996	62.4501197126895\\
4260.39999999994	62.4480694754137\\
4260.79999999995	62.451594733507\\
4261.59999999998	62.4539503015229\\
4262.40000000001	62.4563049853372\\
4262.89999999998	62.4513253577293\\
4264.00000000006	62.4539762200699\\
4264.70000000006	62.450759707372\\
4265.19999999995	62.452816917919\\
4265.50000000007	62.4484246061515\\
4266.2	62.4545859409793\\
4266.60000000005	62.4510746009797\\
4267.09999999997	62.4554743157105\\
4267.29999999999	62.4525472184468\\
4267.60000000002	62.455186634487\\
4267.90000000001	62.4531396438613\\
4268.39999999994	62.4505095466792\\
4269.00000000006	62.4487596917383\\
4269.40000000006	62.4522777842839\\
4269.59999999997	62.4493524135185\\
4270.50000000003	62.457265957945\\
4270.99999999996	62.4546369787643\\
4271.40000000003	62.4581528736978\\
4272.40000000006	62.4552369806905\\
4273.30000000003	62.4584639865213\\
4273.69999999998	62.4526182788151\\
4274.99999999997	62.4546794227036\\
4275.19999999995	62.4517577713845\\
4275.39999999998	62.4535142088645\\
4275.60000000002	62.4505928853755\\
4275.90000000006	62.4532273152479\\
4276.19999999995	62.4511844351425\\
4276.40000000007	62.4529404887174\\
4276.70000000005	62.4485596707819\\
4276.99999999997	62.4511935657338\\
4277.49999999995	62.4532448101739\\
4277.90000000002	62.4497428705002\\
4278.10000000006	62.4514982936749\\
4278.80000000002	62.4553039332539\\
4280.10000000002	62.4526891266763\\
4280.30000000005	62.4544435099523\\
4280.80000000006	62.4518208787872\\
4281.19999999995	62.4553289888585\\
4281.40000000005	62.4524115380124\\
4281.59999999996	62.4541654015928\\
4281.79999999995	62.4512482776338\\
4282.20000000001	62.4547556219788\\
4282.90000000002	62.4515526500117\\
4283.09999999993	62.4533059394845\\
4283.69999999996	62.4562304496008\\
4284.00000000002	62.4541910786396\\
4286.09999999995	62.4562549577714\\
4286.40000000004	62.4518838212994\\
4286.69999999997	62.4498460390035\\
4287.09999999995	62.4463519313305\\
4287.89999999998	62.451026119403\\
4288.20000000002	62.4466571834993\\
4288.80000000005	62.4425843456364\\
4289.50000000005	62.4393882879523\\
4289.69999999994	62.4411394470605\\
4290.00000000001	62.4367730355936\\
4290.49999999999	62.4411504218524\\
4291.70000000002	62.4446619134163\\
4292.80000000001	62.4426378438818\\
4293.90000000002	62.4522589659991\\
4294.10000000002	62.4493502864329\\
4294.89999999999	62.4470314318976\\
4295.39999999996	62.45140263066\\
4295.59999999995	62.4484950066345\\
4296	62.4519913409837\\
4296.20000000001	62.4490840956172\\
4296.79999999999	62.454327538458\\
4297.00000000004	62.4514207256056\\
4297.2	62.4531682684476\\
4297.50000000003	62.4511355174981\\
4297.90000000007	62.4546300604933\\
4298.19999999997	62.4502710373869\\
4298.89999999997	62.4563852058618\\
4299.19999999994	62.4543530342149\\
4299.79999999994	62.4595920835368\\
4300.30000000002	62.4616314761418\\
4300.79999999994	62.4636703945686\\
4301.29999999998	62.4610591900312\\
4302.3	62.4628114540722\\
4302.49999999999	62.4599079626273\\
4302.90000000005	62.4633976295608\\
4303.29999999996	62.4575916717014\\
4303.50000000006	62.4593363695511\\
4305.39999999994	62.4549994193473\\
4305.70000000004	62.4576153095824\\
4306.20000000002	62.4596521375659\\
4306.40000000006	62.4567514222687\\
4306.60000000004	62.4584949032902\\
4307.09999999999	62.4605312035661\\
4307.50000000004	62.4570526511283\\
4308.09999999997	62.4622812311406\\
4308.59999999997	62.4596746118319\\
4309.00000000004	62.4561973497946\\
4309.60000000006	62.4614242290647\\
4310.20000000001	62.4643296290281\\
4310.39999999997	62.46143138847\\
4310.80000000004	62.46491451901\\
4311.09999999995	62.4605678233439\\
4311.30000000001	62.4623092267013\\
4311.80000000004	62.4643428650943\\
4312.40000000001	62.4672463768116\\
4312.69999999999	62.4629011315155\\
4313.39999999994	62.4689926973455\\
4313.59999999994	62.4660963905696\\
4314.09999999999	62.4681285058644\\
4314.70000000001	62.4663947344025\\
4314.89999999994	62.468134414832\\
4315.50000000003	62.4710353137455\\
4315.70000000002	62.468140321609\\
4316.00000000005	62.4661152429276\\
4316.20000000003	62.4678544123439\\
4316.69999999998	62.4652520385471\\
4317.09999999993	62.4687297322339\\
4317.69999999999	62.4716290703599\\
4317.90000000006	62.4687355257064\\
4318.09999999999	62.4704738085313\\
4318.30000000006	62.467580585402\\
4318.49999999997	62.4693187607095\\
4319.29999999996	62.4716395795712\\
4319.69999999996	62.4658549006898\\
4319.90000000005	62.4675925925926\\
4320.40000000001	62.4696215715774\\
4320.59999999998	62.4667299280209\\
4320.79999999997	62.4684672174779\\
4321.10000000002	62.4664445061557\\
4321.50000000006	62.4629766753054\\
4321.80000000001	62.4655822670585\\
4322.10000000001	62.46356022396\\
4322.69999999997	62.4618302951791\\
4322.90000000002	62.4635669673838\\
4323.09999999999	62.4606772760918\\
4323.40000000007	62.4632820631433\\
4323.99999999994	62.4569274531116\\
4324.49999999998	62.461268094159\\
4325	62.4632956463434\\
4325.89999999997	62.4595469255663\\
4326.39999999994	62.4569513463539\\
4327.40000000002	62.4656268053148\\
4327.90000000005	62.4584103512015\\
4328.69999999994	62.4537978192571\\
4328.90000000002	62.4555324555325\\
4329.59999999999	62.4523639051204\\
4329.79999999995	62.4540982470727\\
4330.29999999994	62.4515056345834\\
4330.90000000003	62.4567074578619\\
4331.20000000002	62.4523815020894\\
4331.89999999995	62.4584487534626\\
4332.20000000006	62.4564319183805\\
4332.50000000007	62.4590315284125\\
4333.00000000001	62.4564399621518\\
4333.50000000002	62.4607716448219\\
4334.20000000006	62.4529912557968\\
4334.89999999998	62.4590542099193\\
4335.10000000002	62.4561727255951\\
4335.69999999997	62.4613681442871\\
4336.00000000005	62.4593528747031\\
4336.19999999996	62.461084334571\\
4336.69999999999	62.4538830474082\\
4337.39999999997	62.4599423631124\\
4337.70000000002	62.4579279819263\\
4338.00000000005	62.4605241926189\\
4338.19999999998	62.4576446995367\\
4338.50000000004	62.4556308486609\\
4339.09999999998	62.4608222713864\\
4339.39999999994	62.4565042055536\\
4339.60000000005	62.4582344401687\\
4339.90000000002	62.4562211981567\\
4341.19999999999	62.45364291802\\
4342.10000000002	62.4614250840588\\
4342.29999999998	62.4585482682388\\
4342.89999999995	62.4568270780566\\
4343.09999999994	62.4585559034813\\
4344.40000000003	62.4674876280354\\
4344.6	62.4646120560683\\
4344.89999999997	62.4626006904488\\
4345.70000000002	62.4603065028303\\
4345.99999999994	62.4628977704149\\
4346.60000000006	62.4657786366669\\
4347.40000000003	62.4611845888442\\
4348.00000000002	62.466364619029\\
4348.50000000006	62.4683806282482\\
4348.69999999994	62.4655077262693\\
4348.99999999993	62.4680968476236\\
4349.19999999999	62.465224288966\\
4349.90000000006	62.4712643678161\\
4350.19999999999	62.4669563018642\\
4350.99999999998	62.4646641079267\\
4351.20000000001	62.4663893549054\\
4351.79999999997	62.4646706036444\\
4352.09999999999	62.4626625614632\\
4352.30000000006	62.4643874643875\\
4353.1	62.4689883304236\\
4353.29999999995	62.4661184361648\\
4353.60000000005	62.4687047798424\\
4353.90000000001	62.4666972898484\\
4354.19999999999	62.4692832372597\\
4354.49999999998	62.4672759840169\\
4355.19999999998	62.4641241705508\\
4355.40000000006	62.4658477786706\\
4355.99999999994	62.457243864925\\
4356.20000000001	62.4589674723963\\
4356.39999999996	62.4561000803397\\
4356.69999999998	62.4586852735953\\
4356.90000000002	62.4558182235483\\
4357.19999999999	62.4538131411654\\
4357.60000000005	62.4572595635312\\
4357.90000000006	62.4529600734282\\
4358.20000000004	62.4509556478443\\
4358.39999999995	62.4526786738557\\
4358.79999999999	62.4469476244007\\
4359.00000000001	62.4486705971416\\
4360.40000000001	62.4400871459695\\
4361.00000000006	62.4452546375914\\
4361.50000000004	62.4403888481291\\
4361.80000000007	62.4429720993145\\
4361.99999999999	62.4401091217533\\
4362.39999999998	62.443553008596\\
4363.00000000003	62.4418418097224\\
4363.20000000003	62.4435633580088\\
4363.39999999999	62.4407012719147\\
4363.79999999996	62.4372694149728\\
4364	62.4389908572214\\
4364.5	62.4364202905192\\
4364.79999999996	62.4390020389929\\
4365.09999999996	62.4347108952625\\
4365.29999999995	62.4364319420901\\
4365.80000000006	62.4338624338624\\
4366.09999999994	62.4364435893912\\
4366.50000000005	62.4330142444923\\
4367.20000000007	62.439035559728\\
4367.60000000003	62.4356068411292\\
4368.39999999997	62.4401968639121\\
4369.40000000004	62.4419269939352\\
4369.89999999999	62.4393592677346\\
4370.6	62.4430869197154\\
4370.79999999996	62.4402297009769\\
4371.39999999995	62.445384879332\\
4371.80000000004	62.4396715386903\\
4372.40000000002	62.4356775300172\\
4372.6	62.4373956594324\\
4373.00000000007	62.4339713246896\\
4373.20000000007	62.4356892964123\\
4375.30000000003	62.4445764958632\\
4375.99999999994	62.4414432942575\\
4376.30000000002	62.4440179142674\\
4376.50000000007	62.4411643741717\\
4377.1	62.4440281458467\\
4377.59999999997	62.4391803915298\\
4377.79999999996	62.4408963201535\\
4377.99999999997	62.4380439003221\\
4378.29999999995	62.43604969852\\
4379.10000000005	62.4429119473877\\
4379.90000000006	62.4406392694064\\
4380.29999999997	62.4440690347913\\
4380.70000000004	62.440650109569\\
4380.89999999999	62.4423647569048\\
4381.10000000006	62.4395142883228\\
4381.29999999999	62.4412288309673\\
4381.59999999993	62.4392359130018\\
4382.69999999999	62.4418180158803\\
4382.90000000005	62.4389687428702\\
4383.09999999993	62.440682606315\\
4383.60000000002	62.442685402742\\
4384.60000000007	62.4398476520629\\
4384.79999999999	62.4415608109649\\
4386.09999999998	62.4344535132917\\
4386.30000000005	62.4361663322998\\
4386.79999999996	62.4313296405206\\
4387.10000000006	62.4338986141503\\
4387.29999999993	62.4310525596025\\
4388.30000000006	62.4350560568772\\
4388.59999999994	62.4330667395812\\
4388.90000000005	62.4310776942356\\
4389.49999999999	62.4339347548752\\
4389.69999999998	62.4310902546813\\
4389.99999999993	62.4291018427826\\
4390.29999999997	62.4271137026239\\
4390.49999999999	62.4288252175101\\
4390.79999999993	62.4268373226446\\
4391.40000000002	62.4228623477172\\
4392.00000000004	62.4211652740147\\
4392.60000000001	62.4194686639197\\
4392.89999999997	62.4220350557706\\
4393.19999999997	62.4177725172422\\
4393.70000000007	62.4197733169466\\
4393.89999999996	62.4169321802458\\
4394.60000000002	62.4092657064191\\
4394.99999999999	62.4126868558167\\
4396.19999999993	62.4092987284762\\
4396.40000000007	62.4110087569658\\
4396.59999999999	62.4081697636864\\
4396.80000000002	62.409879687962\\
4397.29999999998	62.4050575339974\\
4397.49999999998	62.4067673276332\\
4397.69999999997	62.4039292373459\\
4397.89999999998	62.4056389267849\\
4398.39999999998	62.4008184608389\\
4398.60000000001	62.4025280196422\\
4398.89999999998	62.4005455785406\\
4399.40000000002	62.4048187294011\\
4399.69999999996	62.4005636619846\\
4400.10000000002	62.4039816371983\\
4400.39999999994	62.4019997727531\\
4400.90000000005	62.4062713019768\\
4401.10000000001	62.4034354267018\\
4401.39999999994	62.4014540497558\\
4402.80000000002	62.4043244225397\\
4403.80000000006	62.3969663253026\\
4404.1	62.3995277235366\\
4404.29999999999	62.396694214876\\
4405.40000000003	62.4060833049597\\
4405.99999999994	62.3998547468283\\
4406.39999999996	62.4032678996936\\
4406.79999999999	62.399872926547\\
4407.10000000002	62.4024323833727\\
4407.29999999999	62.3996006715978\\
4408.29999999995	62.4013247436712\\
4408.50000000002	62.3984938529238\\
4409.30000000004	62.4007801514945\\
4409.79999999993	62.3982403229098\\
4410.50000000007	62.404208044257\\
4410.70000000002	62.4013784347511\\
4411.1	62.3979869423286\\
4411.49999999995	62.4013963187959\\
4411.69999999996	62.3985674781269\\
4412.59999999998	62.4062365445192\\
4413.10000000001	62.4014320674341\\
4413.90000000007	62.4082464884459\\
4414.89999999994	62.4031710079275\\
4415.4	62.4074283773072\\
4415.59999999998	62.4046017618951\\
4416.60000000002	62.4131138632916\\
4417.50000000002	62.4094530967041\\
4418.40000000005	62.4148466674211\\
4418.69999999994	62.4128722730153\\
4418.90000000003	62.4145734329034\\
4419.19999999995	62.4103364786278\\
4419.49999999999	62.4128880441669\\
4421.80000000007	62.4233926592641\\
4422.29999999996	62.4185962373372\\
4422.79999999998	62.4228447398766\\
4423.29999999994	62.4203101686486\\
4424.69999999996	62.4096004339179\\
4424.90000000002	62.4112994350283\\
4425.10000000001	62.4084787128266\\
4425.40000000004	62.4065077392385\\
4425.90000000001	62.4039765024853\\
4426.50000000006	62.4090724257895\\
4426.69999999994	62.4062528237101\\
4426.90000000007	62.4079512084933\\
4427.39999999998	62.4054206662902\\
4427.7	62.4034509237093\\
4427.99999999995	62.4059980578578\\
4428.19999999993	62.4031795497143\\
4428.59999999997	62.4065752929754\\
4428.90000000004	62.4023481598555\\
4429.20000000005	62.4048946786174\\
4430.19999999996	62.4111234002212\\
4430.59999999997	62.4077459543639\\
4431.10000000002	62.4029608232533\\
4431.39999999995	62.4055060363308\\
4431.60000000003	62.4026897127513\\
4431.90000000003	62.4007220216607\\
4432.80000000005	62.4038439847504\\
4433.2	62.3982135204024\\
4434.29999999998	62.396265560166\\
4434.60000000002	62.3988093895867\\
4435.09999999994	62.3917748917749\\
4436.10000000006	62.3889815607953\\
4436.59999999995	62.3932201861744\\
4437.10000000007	62.395204182818\\
4437.80000000006	62.3876157642128\\
4438.09999999998	62.390158172232\\
4438.5	62.3845356643987\\
4438.99999999998	62.3865197900475\\
4439.20000000005	62.3837091433334\\
4439.79999999995	62.3887925403725\\
4440.19999999998	62.3854244082607\\
4441.50000000001	62.3874279538905\\
4442.30000000001	62.3919502971367\\
4443.3	62.3891614529415\\
4444.00000000006	62.3815845728044\\
4444.70000000005	62.3875089992801\\
4445.10000000001	62.3818950778368\\
4445.39999999998	62.3799347654932\\
4445.99999999993	62.3760149344369\\
4446.99999999998	62.3844752760226\\
4447.20000000001	62.3816697771682\\
4447.79999999994	62.3844960543178\\
4448.09999999998	62.3802886560856\\
4448.49999999993	62.383671267365\\
4448.79999999999	62.3794645867518\\
4449.10000000003	62.3775060685067\\
4450.60000000001	62.3811984631631\\
4450.89999999994	62.3769939339474\\
4451.19999999999	62.3750365061892\\
4451.60000000003	62.3784172338657\\
4452.50000000006	62.3747922562099\\
4453.19999999997	62.371724339254\\
4453.70000000007	62.3737033544389\\
4454.19999999997	62.3689468603372\\
4455.59999999998	62.3717934331306\\
4456.09999999995	62.3692832458148\\
4456.60000000005	62.3735050597976\\
4456.90000000002	62.3715503702042\\
4457.20000000002	62.3740829650237\\
4457.49999999997	62.3721284996411\\
4457.80000000006	62.3701742973149\\
4458.69999999999	62.3665560240423\\
4459.49999999997	62.3710646694771\\
4459.90000000001	62.3654708520179\\
4460.39999999998	62.3696894966932\\
4460.80000000004	62.3663386312179\\
4461.9	62.3644105782161\\
4462.39999999998	62.3619047619048\\
4462.89999999998	62.3638807976697\\
4463.60000000003	62.3585814458857\\
4464.29999999994	62.3644834692232\\
4464.50000000003	62.3616897370425\\
4465.00000000004	62.3659044590267\\
4465.30000000005	62.3617145160568\\
4466.40000000006	62.3575506548752\\
4466.80000000001	62.3609214444022\\
4467.70000000005	62.3662652759748\\
4467.99999999996	62.3643159284707\\
4468.50000000002	62.3618135433917\\
4469.30000000005	62.3685505884459\\
4470.10000000004	62.3730481857635\\
4470.60000000006	62.3660724271367\\
4470.89999999993	62.3641243569671\\
4471.49999999995	62.3691743447536\\
4471.99999999999	62.3711455468348\\
4472.20000000006	62.3683563267223\\
4472.49999999999	62.3664088002504\\
4473.30000000003	62.3641972548844\\
4473.69999999999	62.36756225133\\
4474.00000000002	62.3633803446503\\
4474.99999999999	62.3673214006391\\
4475.59999999999	62.3589606095136\\
4476.50000000006	62.3665281686995\\
4476.79999999999	62.3623489468159\\
4477.39999999994	62.358458961474\\
4478.09999999994	62.3621097762494\\
4478.40000000005	62.3601652338953\\
4478.7	62.3626864338662\\
4479.00000000005	62.360742113371\\
4480.40000000005	62.3568798125209\\
4480.70000000007	62.3594001071237\\
4481.3	62.3510510108448\\
4482.39999999995	62.3535973229225\\
4483.09999999996	62.3460920770878\\
4484.09999999994	62.3522590428616\\
4484.59999999999	62.3542265926372\\
4484.90000000002	62.3522853957637\\
4485.70000000005	62.3567702527977\\
4486.60000000007	62.3620924064457\\
4487.60000000007	62.3637943712815\\
4487.90000000005	62.3596256684492\\
4488.40000000002	62.3615907318703\\
4488.89999999999	62.3591000222767\\
4489.40000000004	62.3610647065375\\
4489.60000000004	62.3582867452168\\
4490.09999999998	62.3602512137544\\
4490.60000000005	62.3555347718618\\
4491.10000000005	62.3574991093694\\
4491.40000000003	62.3555605031727\\
4492.10000000006	62.3614264725524\\
4492.30000000005	62.3586501647226\\
4494.39999999999	62.365112915786\\
4494.8	62.3595630603573\\
4495.39999999997	62.3623623623624\\
4495.69999999997	62.3582009875884\\
4496.49999999995	62.3560023128586\\
4497.29999999994	62.3604749410771\\
4497.49999999998	62.35770188545\\
4497.80000000006	62.3602125436315\\
4497.99999999996	62.3574398079189\\
4498.49999999993	62.3616236162362\\
4499.19999999997	62.3652568177272\\
4499.59999999994	62.3597128697469\\
4499.89999999998	62.3577777777778\\
4500.19999999997	62.3558429438038\\
4502.10000000006	62.3539602860824\\
4502.80000000004	62.3598125652357\\
4503	62.3570429259843\\
4503.39999999996	62.3603863661597\\
4503.59999999999	62.3576170704088\\
4504.1	62.3617956573865\\
4504.89999999994	62.3662597114317\\
4505.40000000005	62.3682166241261\\
4506.19999999996	62.3726782504494\\
4506.40000000007	62.3699101298125\\
4506.70000000003	62.3724150173072\\
4507	62.3704821281977\\
4507.80000000002	62.3682867854211\\
4508.99999999995	62.3716484442572\\
4509.19999999995	62.368882088129\\
4509.49999999994	62.3669505055881\\
4510.69999999994	62.3636605480181\\
4512.50000000007	62.372024996676\\
4513.70000000005	62.368735876645\\
4514.50000000003	62.3731892083463\\
4514.79999999994	62.3690447185984\\
4515.10000000007	62.3715450035436\\
4515.59999999995	62.3734969107779\\
4516.09999999997	62.3754483858111\\
4517.30000000001	62.3721609775535\\
4517.60000000004	62.374659671957\\
4517.8	62.3718984483942\\
4518.80000000005	62.3691606364381\\
4519.20000000004	62.3724913150267\\
4520.09999999997	62.3777708950931\\
4520.99999999999	62.3830483731835\\
4521.20000000001	62.3802888549753\\
4521.99999999998	62.3825213949271\\
4523.00000000003	62.3797837766134\\
4523.29999999995	62.3822788168192\\
4523.6	62.3803523664257\\
4524.70000000003	62.3894978783593\\
4525.00000000005	62.3875715453802\\
4525.50000000006	62.3895174120559\\
4525.80000000005	62.3853819129897\\
4526.30000000004	62.3873276776246\\
4527.79999999996	62.3843282758012\\
4529.00000000001	62.3898787838643\\
4529.29999999995	62.3879542544266\\
4529.60000000003	62.3860299799104\\
4529.89999999999	62.3841059602649\\
4530.20000000002	62.3865969141116\\
4530.5	62.3846731117291\\
4531.89999999998	62.396293027361\\
4532.10000000002	62.3935395613609\\
4532.80000000008	62.3861104370271\\
4533.50000000004	62.3897123698606\\
4534.00000000005	62.3916543525727\\
4534.20000000006	62.3889023664072\\
4534.50000000005	62.3913906408504\\
4534.69999999997	62.3886389697451\\
4535.20000000007	62.3905805569643\\
4535.50000000005	62.3864538319076\\
4535.79999999999	62.384532286867\\
4536.69999999995	62.3897901604655\\
4537.20000000004	62.3917307649924\\
4537.89999999999	62.3953283384751\\
4538.09999999999	62.3925785553744\\
4539.30000000005	62.4003172225404\\
4540.20000000001	62.4033654163822\\
4540.50000000001	62.4014447429855\\
4540.90000000004	62.3981501871834\\
4541.79999999993	62.4011977366301\\
4542.09999999996	62.3970763066356\\
4543.00000000001	62.4023244040413\\
4543.60000000001	62.4050883641086\\
4543.90000000005	62.4031690140845\\
4544.49999999995	62.399331074242\\
4544.79999999993	62.3974124843231\\
4545.60000000001	62.3952306575445\\
4545.9	62.3977122745271\\
4546.10000000004	62.394967225375\\
4546.5	62.3982756345401\\
4546.8	62.3963579581693\\
4547.20000000006	62.3996657357113\\
4547.4	62.3969213853766\\
4547.69999999999	62.3994019086152\\
4548.19999999994	62.3969395158631\\
4548.79999999993	62.4018993602849\\
4549.10000000003	62.3977842257979\\
4549.40000000003	62.3958676777668\\
4550.09999999994	62.4016526746077\\
4550.40000000005	62.3975387320075\\
4551.69999999995	62.4016872446065\\
4552.09999999995	62.3962040332147\\
4552.40000000001	62.3986820428336\\
4553.50000000003	62.4011770906535\\
4555.60000000004	62.3987532102641\\
4555.99999999999	62.4020543886218\\
4556.49999999999	62.4039854277312\\
4557.29999999999	62.4018080484487\\
4557.90000000001	62.4067573497148\\
4558.79999999995	62.4119853473426\\
4559.09999999999	62.4100719424461\\
4559.69999999998	62.4128251239089\\
4561.09999999996	62.4068227659388\\
4561.59999999994	62.4043667930815\\
4562.30000000005	62.4079431877959\\
4562.50000000006	62.4052075570946\\
4563.39999999995	62.401665388408\\
4563.79999999994	62.3983873441574\\
4564.10000000003	62.4008588580693\\
4564.49999999994	62.3975813871971\\
4564.99999999998	62.4016998532343\\
4565.19999999997	62.3989661139465\\
4565.70000000003	62.4008936002453\\
4566.4	62.3957078725501\\
4566.89999999996	62.3976352091088\\
4567.30000000006	62.3943600297763\\
4567.90000000003	62.3905429071804\\
4568.70000000002	62.3883733146559\\
4569.50000000007	62.3862044817927\\
4569.80000000006	62.3886737127727\\
4570.70000000003	62.3917038592807\\
4571.20000000007	62.3936298208387\\
4571.70000000005	62.3955553611269\\
4572.49999999995	62.3977605738529\\
4573.00000000005	62.3953117141545\\
4574.89999999997	62.4021857923497\\
4575.20000000006	62.4002797630757\\
4575.60000000001	62.4035666673952\\
4575.80000000005	62.4008391791779\\
4576.29999999993	62.396206625295\\
4576.79999999998	62.4003146234351\\
4577.10000000007	62.3984095079962\\
4577.70000000003	62.3946000262135\\
4577.99999999999	62.3970642843101\\
4579.09999999994	62.3995457721873\\
4579.60000000004	62.4014673450226\\
4579.80000000006	62.3987423306186\\
4580.10000000006	62.4012051875464\\
4580.29999999998	62.398480482054\\
4581.39999999997	62.396595001637\\
4581.89999999999	62.3985159319075\\
4582.29999999993	62.395251396648\\
4582.7	62.3985336475517\\
4582.90000000007	62.3958106044076\\
4583.69999999998	62.4001919804529\\
4584.19999999994	62.4021115546539\\
4585.29999999996	62.3980459719981\\
4585.59999999996	62.4005059205792\\
4585.89999999995	62.398604448321\\
4586.39999999997	62.4027035866129\\
4586.80000000003	62.3994418888574\\
4587.09999999999	62.401900941751\\
4587.50000000005	62.3986398116662\\
4588.10000000003	62.4035569504381\\
4588.30000000005	62.4008368930346\\
4588.80000000007	62.3983961297915\\
4589.10000000005	62.4008541793777\\
4589.29999999998	62.39813483244\\
4589.70000000007	62.4014118262234\\
4589.90000000006	62.3986928104575\\
4590.30000000003	62.4019693272917\\
4590.99999999996	62.3989893489578\\
4592.00000000002	62.407177544043\\
4592.49999999998	62.4047380568741\\
4592.89999999999	62.4014805138254\\
4593.4	62.4055730924132\\
4593.59999999995	62.4028560855084\\
4593.89999999999	62.4053112755768\\
4594.30000000002	62.4020546752568\\
4594.59999999995	62.4045095436046\\
4594.80000000001	62.4017932925635\\
4595.30000000005	62.403708055882\\
4595.99999999996	62.4072583277126\\
4596.19999999997	62.4045427844136\\
4596.90000000004	62.3993909071133\\
4597.20000000002	62.4018445609379\\
4597.69999999999	62.3994084127191\\
4598.49999999993	62.405949636846\\
4599.09999999999	62.4086797703949\\
4600.39999999998	62.4106075426584\\
4600.9	62.4125190176049\\
4601.40000000007	62.410083668369\\
4602.70000000005	62.4120100808204\\
4602.90000000006	62.409298283728\\
4603.20000000003	62.4074033845285\\
4603.69999999997	62.4027976888657\\
4604.19999999998	62.4068805247269\\
4604.5	62.4049863180298\\
4604.89999999997	62.4082519001086\\
4605.19999999999	62.406357891994\\
4606.30000000008	62.4109934004863\\
4607.40000000001	62.415626695605\\
4607.69999999999	62.4137332349494\\
4608.00000000001	62.4118400208329\\
4608.49999999993	62.4094084971575\\
4608.79999999993	62.411855323396\\
4609.09999999999	62.409962683329\\
4610.19999999999	62.4124243541635\\
4611.10000000005	62.4089174184594\\
4611.79999999996	62.4124547366595\\
4612.00000000002	62.4097482708528\\
4612.30000000001	62.4121932182812\\
4612.49999999994	62.4094870571912\\
4613.00000000005	62.4070581604561\\
4613.50000000005	62.4089647997226\\
4613.79999999999	62.4070742755586\\
4614.29999999995	62.4089805825243\\
4614.60000000006	62.4070903850738\\
4615.20000000005	62.4119775529218\\
4615.69999999996	62.4138827505525\\
4616.40000000004	62.4174157911838\\
4616.79999999998	62.4120080573545\\
4617.09999999999	62.4101186866499\\
4617.40000000004	62.408229561451\\
4617.79999999994	62.4114857402716\\
4618.20000000002	62.4082454582855\\
4618.99999999998	62.4017665779048\\
4619.30000000006	62.4042083387453\\
4620.50000000005	62.4096437692075\\
4620.80000000008	62.4077560648359\\
4621.80000000005	62.4050715073887\\
4623.09999999997	62.4091538328431\\
4623.30000000007	62.4064541246702\\
4623.90000000007	62.409169550173\\
4625.19999999996	62.4132488703435\\
4625.49999999999	62.409200968523\\
4628.00000000003	62.4187031395173\\
4628.69999999998	62.413584514345\\
4629.3	62.4184559554154\\
4629.70000000006	62.4152231197892\\
4629.99999999994	62.4133388047774\\
4630.30000000001	62.4157740152039\\
4631.20000000002	62.4187593116404\\
4631.80000000005	62.4214685118418\\
4632.19999999996	62.4182371608056\\
4632.89999999994	62.4131232462767\\
4633.3	62.4163681098114\\
4633.60000000005	62.4144851846257\\
4633.89999999996	62.416918429003\\
4634.10000000002	62.4142246773985\\
4634.50000000003	62.4174686057049\\
4635.29999999997	62.4217974716314\\
4635.60000000001	62.4177578359255\\
4636.60000000004	62.4237065154097\\
4637.09999999996	62.4191322349694\\
4638.19999999997	62.4129530215812\\
4639.29999999995	62.4218648963228\\
4639.5	62.4191740667299\\
4640.40000000004	62.4221527852602\\
4640.60000000001	62.4194625810761\\
4641.30000000003	62.4251303486017\\
4642.60000000008	62.4184203157645\\
4642.90000000002	62.4165410295068\\
4643.40000000002	62.4184343706256\\
4644.10000000001	62.4154859825158\\
4644.59999999998	62.4173789480483\\
4644.90000000005	62.4133476856835\\
4645.30000000007	62.4101261462953\\
4645.69999999997	62.413362607086\\
4647.20000000005	62.4104318636628\\
4647.49999999997	62.4085549530941\\
4648.29999999996	62.4042681352724\\
4649.10000000001	62.4085864234707\\
4649.79999999996	62.4034925482269\\
4651.39999999996	62.4099752767924\\
4651.60000000001	62.4072919577789\\
4651.89999999993	62.4054170249355\\
4652.79999999999	62.4083904661609\\
4653.10000000006	62.4043668873034\\
4653.90000000006	62.4022346368715\\
4654.20000000003	62.4046580581398\\
4655.20000000005	62.4105857839452\\
4655.39999999996	62.4079046289335\\
4655.89999999997	62.4054982817869\\
4656.50000000002	62.4103423098398\\
4657.09999999998	62.4130378768359\\
4657.50000000001	62.4076777739608\\
4658.69999999998	62.4044818408174\\
4659.00000000007	62.4069026206778\\
4659.20000000005	62.4042238104436\\
4660.10000000001	62.4114844856444\\
4660.30000000006	62.4088061110634\\
4660.80000000003	62.410693213757\\
4661.10000000004	62.4066763923453\\
4662.89999999995	62.4126099077847\\
4663.10000000005	62.4099330931549\\
4664.00000000004	62.4150425591218\\
4664.39999999993	62.4118340658163\\
4664.70000000002	62.4142514148516\\
4665.19999999993	62.4097056995263\\
4665.50000000001	62.4078360768176\\
4665.79999999997	62.4102531130114\\
4666.10000000007	62.4083836955124\\
4666.39999999997	62.4065145183757\\
4667.4	62.410283877879\\
4667.60000000003	62.4076097435568\\
4668.30000000001	62.4111044469197\\
4668.60000000005	62.4092359757534\\
4668.9	62.4116513171985\\
4669.19999999999	62.4076414023515\\
4669.80000000008	62.403905865222\\
4670.40000000006	62.4001712878707\\
4671.20000000001	62.3980476526877\\
4671.79999999996	62.3943149468097\\
4672.09999999994	62.3967295920551\\
4672.39999999998	62.3948635634029\\
4672.79999999994	62.3916625650025\\
4673.30000000007	62.38926691488\\
4674.19999999994	62.3858117793039\\
4674.70000000005	62.3834174724052\\
4676.10000000007	62.3904024635388\\
4676.70000000007	62.3845364351694\\
4677.09999999998	62.3813392628068\\
4677.49999999995	62.3845561826578\\
4677.89999999995	62.3813595553655\\
4678.19999999995	62.37949682577\\
4679.29999999995	62.3862033594051\\
4679.79999999999	62.3880852154961\\
4680.09999999996	62.3840861501645\\
4680.39999999995	62.3864971691059\\
4680.70000000007	62.3824987181678\\
4681.90000000002	62.385732592909\\
4682.09999999996	62.3830677886464\\
4683.30000000003	62.3798949481146\\
4683.70000000007	62.3831077330373\\
4683.89999999995	62.3804440649018\\
4684.59999999998	62.3860652763251\\
4684.80000000005	62.3834019936391\\
4685.29999999994	62.3810133606522\\
4685.69999999995	62.3778223569081\\
4686.19999999996	62.3818364167894\\
4687.39999999995	62.3893333333333\\
4688.09999999994	62.3928160061431\\
4688.40000000003	62.3909565959262\\
4689.10000000007	62.3944382837158\\
4689.39999999997	62.3925791662224\\
4689.89999999998	62.3901918976546\\
4690.20000000003	62.3883333688677\\
4691.09999999996	62.3955491132333\\
4691.30000000003	62.3928891162553\\
4692.40000000005	62.3910495471497\\
4693.20000000003	62.3868067244796\\
4693.59999999995	62.3900121439376\\
4693.89999999999	62.3881550916063\\
4695.40000000004	62.3959109785965\\
4696.1	62.3993867382139\\
4696.70000000007	62.4020609776869\\
4697.00000000001	62.4002043814268\\
4697.80000000002	62.4066072074757\\
4698.20000000006	62.4034225145265\\
4698.69999999994	62.4052949689282\\
4698.89999999995	62.4026388593318\\
4699.50000000001	62.4074389309728\\
4700.90000000002	62.4101255052117\\
4701.89999999995	62.4074861760953\\
4702.19999999994	62.4098845245943\\
4702.8	62.4040485657786\\
4703.09999999994	62.4021942507229\\
4703.70000000008	62.3984863302011\\
4704.29999999997	62.4032820338407\\
4704.89999999996	62.4059511158342\\
4705.20000000002	62.4019722440652\\
4705.50000000006	62.4043692621557\\
4705.69999999994	62.401717030048\\
4706.00000000001	62.3998640062897\\
4706.30000000003	62.4022607513174\\
4707.50000000003	62.399099328745\\
4708.30000000003	62.4033642001529\\
4708.60000000004	62.4015120946333\\
4709.20000000001	62.4041789650266\\
4710.1	62.4007473143391\\
4710.90000000005	62.4050095521121\\
4711.09999999995	62.4023603328239\\
4711.50000000004	62.3991849902369\\
4712.69999999998	62.3960278390766\\
4713.7	62.391276676991\\
4714.59999999998	62.3942138418139\\
4714.90000000001	62.390243902439\\
4715.79999999997	62.3953010029899\\
4716.79999999994	62.3926731539783\\
4717.09999999995	62.3950648689901\\
4717.59999999998	62.3926913538377\\
4718.19999999997	62.3889960367081\\
4718.90000000005	62.3839796567069\\
4719.29999999996	62.3871678603212\\
4721.60000000006	62.3906643793549\\
4722.69999999997	62.3930719064961\\
4723.19999999995	62.386467088688\\
4723.49999999997	62.384621898552\\
4724.49999999993	62.3904669178343\\
4725.49999999995	62.3941933299475\\
4725.7	62.3915527529731\\
4726.79999999999	62.3897268823119\\
4727.20000000001	62.3929092716773\\
4727.49999999995	62.3910652339453\\
4728.19999999997	62.3881733392551\\
4728.70000000007	62.3858061241753\\
4729.19999999998	62.3876683652972\\
4729.39999999993	62.3850301300349\\
4729.89999999997	62.3826638477801\\
4730.30000000006	62.3795027904617\\
4730.80000000002	62.3813650679575\\
4732.09999999994	62.3790203288111\\
4733.20000000007	62.3814252213044\\
4733.39999999993	62.3787894792437\\
4734.49999999998	62.3833058758924\\
4735.6	62.389931794666\\
4735.79999999999	62.3872970290758\\
4736.30000000003	62.3912676294232\\
4736.89999999999	62.3854760396876\\
4737.30000000004	62.3886520032085\\
4738.10000000007	62.3928918154574\\
4738.80000000001	62.3963367026103\\
4739.40000000001	62.3926574533179\\
4739.69999999997	62.3950377653066\\
4740.30000000002	62.397687958822\\
4740.59999999997	62.3958487143249\\
4740.89999999996	62.3940097025944\\
4741.19999999993	62.3921709235864\\
4741.99999999998	62.3985154256553\\
4742.19999999995	62.3958838538262\\
4742.49999999998	62.394045460296\\
4743.00000000001	62.3916847631296\\
4743.49999999998	62.3872164600725\\
4743.79999999999	62.3853791184468\\
4744.90000000007	62.3877766069547\\
4745.19999999993	62.3859397719849\\
4746.30000000005	62.3883364233946\\
4746.59999999999	62.3843933680241\\
4747.40000000006	62.3823064770932\\
4748.29999999993	62.374694633982\\
4748.89999999998	62.3773425984418\\
4749.39999999995	62.3791978102958\\
4749.69999999997	62.3773632574003\\
4750.19999999994	62.3750078942383\\
4750.70000000005	62.3726530268586\\
4751.09999999994	62.3758208452602\\
4751.40000000001	62.371882563401\\
4752.00000000005	62.3682161570674\\
4753.60000000001	62.3640532637735\\
4755.20000000001	62.3619960885749\\
4756.50000000004	62.3596686708994\\
4757	62.3615227764815\\
4757.29999999993	62.3575902804053\\
4757.59999999996	62.3557601362003\\
4758.00000000001	62.3589247808999\\
4758.20000000008	62.3563037219175\\
4758.89999999999	62.3513343139315\\
4760.5	62.3555854304079\\
4760.69999999997	62.352965888086\\
4761.19999999993	62.3569193287548\\
4761.90000000001	62.3603527929441\\
4762.19999999993	62.3585242424879\\
4763.19999999998	62.3622278672349\\
4764.29999999993	62.3646209386282\\
4764.60000000004	62.3627930404852\\
4765.59999999993	62.3685922319911\\
4765.80000000001	62.3659749470195\\
4766.69999999997	62.3730804732735\\
4767.29999999999	62.3757184209422\\
4767.60000000001	62.3717935272773\\
4768.59999999993	62.3691991528089\\
4768.90000000002	62.3673726148039\\
4769.30000000004	62.3705287876882\\
4769.59999999994	62.366605866197\\
4770	62.3697616402172\\
4770.30000000006	62.3679356028845\\
4771.6	62.3781880671459\\
4771.89999999998	62.3763621123219\\
4773.20000000001	62.3782288982465\\
4773.40000000005	62.3756153765581\\
4773.79999999998	62.378767883701\\
4774.10000000004	62.3748481420971\\
4774.40000000006	62.3730233532307\\
4775.6	62.3699143581046\\
4776.10000000007	62.3675725472133\\
4776.39999999997	62.3699361457134\\
4776.89999999993	62.3717814527946\\
4777.60000000005	62.3668292274525\\
4777.99999999997	62.3699796990436\\
4778.90000000005	62.3728813559322\\
4779.19999999999	62.3710585232147\\
4779.70000000004	62.3749947696556\\
4779.90000000003	62.3723849372385\\
4780.59999999998	62.3758027067166\\
4781.00000000006	62.3705841751898\\
4781.29999999999	62.3729451625047\\
4781.79999999993	62.366423388193\\
4783.20000000003	62.3732569564945\\
4783.39999999995	62.3706491063029\\
4783.89999999997	62.3745819397993\\
4785.40000000004	62.3717479887159\\
4785.70000000002	62.3741067324167\\
4786.20000000003	62.3759480183022\\
4786.80000000005	62.3681296872715\\
4787.09999999996	62.3704879679144\\
4788.39999999995	62.3660854129686\\
4788.69999999997	62.3684430337454\\
4788.89999999995	62.36583837962\\
4789.40000000001	62.3697671990813\\
4790.30000000003	62.374749498998\\
4790.50000000005	62.3721454515092\\
4790.80000000002	62.3703270784195\\
4791.09999999998	62.3685089330439\\
4791.50000000008	62.3716503881793\\
4791.90000000004	62.368530884808\\
4792.49999999996	62.3648958811501\\
4792.89999999996	62.368036720217\\
4793.39999999994	62.3698758735788\\
4793.89999999996	62.3654568210263\\
4794.49999999998	62.3680807575189\\
4794.90000000005	62.3649635036496\\
4795.40000000001	62.3668022104056\\
4795.59999999997	62.364201263632\\
4795.99999999995	62.3673401305227\\
4796.79999999997	62.3652775751006\\
4797.09999999995	62.3634620195114\\
4797.79999999999	62.3564476124971\\
4798.10000000008	62.3588012171231\\
4798.39999999993	62.3569865582995\\
4799.00000000007	62.3533579212769\\
4799.30000000007	62.355711130558\\
4799.50000000007	62.3531127593966\\
4800.10000000006	62.3557351777009\\
4800.79999999995	62.3528921660522\\
4801.19999999998	62.3497802678441\\
4802.20000000001	62.3576203069363\\
4802.69999999995	62.3552927458982\\
4803.09999999998	62.3584277148568\\
4803.29999999996	62.3558312861723\\
4803.70000000002	62.3589658187268\\
4804.2	62.3608017817372\\
4804.40000000004	62.3582058486835\\
4805.19999999993	62.3561484194535\\
4805.70000000007	62.353822464522\\
4806.19999999994	62.3577388011568\\
4806.59999999997	62.3546299956311\\
4807	62.3577624763371\\
4807.30000000005	62.3538711153638\\
4807.79999999997	62.3557062334907\\
4808.60000000004	62.3578098030653\\
4808.90000000003	62.3559991682262\\
4810.70000000004	62.351376070508\\
4811.10000000002	62.348270701696\\
4811.40000000006	62.3506183102982\\
4811.60000000004	62.3480266849554\\
4812.09999999998	62.3519388221603\\
4812.39999999993	62.348051948052\\
4813.10000000003	62.3514501786753\\
4813.50000000006	62.3483463520027\\
4813.80000000006	62.3506927854754\\
4814.09999999997	62.3488845498733\\
4815.69999999999	62.3468582582333\\
4816.50000000004	62.3489598471951\\
4817.70000000002	62.3438083772676\\
4818.19999999997	62.3477159994189\\
4818.69999999993	62.3453971943222\\
4819.60000000003	62.3482789385232\\
4819.90000000006	62.3464730290457\\
4820.49999999995	62.351159606688\\
4820.9	62.3459863098942\\
4821.50000000004	62.3423759747802\\
4822.80000000004	62.3442327230505\\
4823.59999999997	62.3421854592947\\
4824.49999999993	62.3388467437715\\
4824.90000000005	62.3419689119171\\
4825.10000000006	62.3393848959629\\
4825.40000000007	62.3417262459849\\
4825.7	62.3399229143354\\
4826.39999999993	62.3453848544494\\
4828.09999999995	62.3400024853983\\
4828.40000000003	62.3382002692348\\
4829.40000000001	62.3356455119578\\
4829.69999999995	62.3379850097313\\
4830.10000000002	62.3328226574469\\
4830.80000000007	62.3300006210023\\
4831.10000000004	62.3323397913562\\
4832.09999999993	62.3359960266545\\
4832.40000000006	62.3321262286601\\
4833.10000000003	62.3355127037987\\
4833.39999999993	62.333712630599\\
4833.9	62.3376086057096\\
4835.50000000005	62.3417983290595\\
4836.00000000003	62.3394884307603\\
4836.30000000001	62.3418244975602\\
4836.59999999994	62.3400252238096\\
4837.30000000004	62.3434076156613\\
4837.69999999993	62.3403199801563\\
4838.20000000001	62.342144968274\\
4838.50000000006	62.3403463811846\\
4839.09999999994	62.3429492478096\\
4839.60000000006	62.34477343637\\
4840.60000000006	62.3484206829591\\
4840.79999999994	62.3458447809292\\
4841.70000000007	62.3528439836424\\
4842.3	62.3492483066248\\
4843.20000000002	62.3562447091859\\
4843.49999999997	62.3523825253943\\
4844.39999999994	62.3552482196305\\
4844.70000000006	62.3534511228534\\
4845.70000000008	62.3570927401048\\
4846.70000000004	62.3627960716349\\
4847.09999999998	62.3597128238983\\
4847.69999999995	62.3643714674698\\
4848.00000000003	62.362575029393\\
4849.29999999993	62.3706025487689\\
4849.6	62.3688063179166\\
4849.90000000001	62.3670103092784\\
4850.20000000002	62.365214522813\\
4850.69999999996	62.3690937577307\\
4851.79999999993	62.3714421154599\\
4853.59999999995	62.3689144364093\\
4854.39999999994	62.37305592749\\
4854.70000000007	62.3692016148966\\
4855.59999999996	62.3761764524167\\
4855.89999999997	62.3743822075783\\
4857.00000000008	62.3726091700809\\
4857.79999999996	62.3664546408942\\
4858.49999999999	62.3636438480221\\
4859.39999999995	62.3603251363309\\
4859.90000000001	62.3641975308642\\
4860.59999999994	62.3675602279507\\
4862.10000000003	62.3709431944387\\
4862.30000000005	62.3683777558407\\
4862.99999999996	62.3655692870803\\
4863.30000000004	62.3678907760003\\
4863.60000000006	62.3640438349405\\
4864.10000000005	62.3658566670778\\
4864.60000000007	62.3676691265648\\
4864.89999999994	62.365878725591\\
4866.29999999993	62.3705408515535\\
4866.59999999996	62.3666961185197\\
4866.99999999996	62.3636251566641\\
4867.90000000002	62.360312243221\\
4869.39999999995	62.36369237088\\
4869.69999999999	62.3598505072077\\
4870.00000000005	62.3621691546375\\
4870.90000000003	62.3670704167522\\
4872.80000000001	62.3735352664738\\
4873.09999999999	62.3717475170319\\
4873.90000000007	62.3779236766516\\
4874.59999999995	62.3812747451125\\
4874.80000000003	62.37871546083\\
4875.29999999993	62.3764203962752\\
4875.8	62.3802785126848\\
4876.70000000002	62.3831200787402\\
4876.9	62.3805618207915\\
4877.40000000003	62.3823680164019\\
4877.59999999993	62.3798101564262\\
4877.89999999996	62.3780237802378\\
4878.7	62.3821431499549\\
4879.1	62.377029021151\\
4879.40000000007	62.3793421457116\\
4879.60000000002	62.3767854581224\\
4880.49999999996	62.3734786706552\\
4880.79999999994	62.375791349956\\
4882.69999999994	62.3801916932907\\
4883.50000000007	62.3822589892702\\
4883.80000000007	62.3804746206925\\
4884.30000000006	62.3822782736877\\
4884.79999999998	62.3840815574526\\
4885.09999999996	62.3822975517891\\
4885.39999999995	62.3846075120254\\
4886.20000000008	62.3887194809979\\
4886.49999999998	62.3848892890763\\
4887.00000000006	62.3887376972028\\
4887.69999999993	62.392078235607\\
4888.00000000005	62.3902947975696\\
4888.30000000005	62.3926028966533\\
4889.10000000005	62.3946657939949\\
4890.20000000001	62.3990348240394\\
4890.50000000001	62.3972518709361\\
4890.79999999995	62.3995583634914\\
4890.99999999996	62.3970068082844\\
4891.40000000005	62.4000817745068\\
4892.00000000002	62.3944727213262\\
4892.30000000003	62.3967786771319\\
4892.60000000003	62.3949966276289\\
4892.90000000002	62.3973022685469\\
4893.40000000005	62.3929702666803\\
4894.10000000002	62.3963058313923\\
4894.49999999995	62.3932497037552\\
4894.99999999995	62.3970909685195\\
4895.8	62.401192834821\\
4895.99999999995	62.3986438185495\\
4896.39999999995	62.4017155110793\\
4896.89999999994	62.3973861547886\\
4897.40000000003	62.3950995405819\\
4898.49999999999	62.3974196709264\\
4898.70000000007	62.3948722136033\\
4899.70000000001	62.4005061431079\\
4899.89999999995	62.3979591836735\\
4900.39999999995	62.4017957351291\\
4900.60000000004	62.3992490868651\\
4900.99999999998	62.4023178470139\\
4902.39999999996	62.4048954614992\\
4902.79999999995	62.4018438067266\\
4903.69999999997	62.4087442391615\\
4904.19999999993	62.4105376914137\\
4904.49999999994	62.4067202218326\\
4905.19999999997	62.4120848877744\\
4906.50000000004	62.4098153507521\\
4906.79999999999	62.4121135543826\\
4907.10000000006	62.4082980110857\\
4907.50000000007	62.4052490015486\\
4908.20000000001	62.4106105983742\\
4908.49999999995	62.4088334759402\\
4909.50000000001	62.4164901417631\\
4909.69999999998	62.4139476149741\\
4910.49999999996	62.4078524009286\\
4911.00000000005	62.4096434607318\\
4911.4	62.4065967626998\\
4912.09999999995	62.4119539106714\\
4912.29999999994	62.4094129142578\\
4912.80000000007	62.413238616703\\
4913.50000000006	62.410452621296\\
4913.79999999993	62.4127475121594\\
4914.10000000005	62.4089373651866\\
4915.90000000005	62.4125305126119\\
4916.19999999997	62.4087220063869\\
4916.80000000001	62.4051739917428\\
4917.30000000001	62.4089966242323\\
4917.59999999998	62.4072228887488\\
4917.90000000005	62.4054493696625\\
4918.50000000007	62.41003537592\\
4918.70000000008	62.4074977636822\\
4919.79999999993	62.4159027622513\\
4920.60000000003	62.413884203467\\
4921.09999999994	62.415670974559\\
4921.59999999998	62.4113619277892\\
4922.09999999996	62.4151802039738\\
4922.7	62.4096042902413\\
4922.99999999993	62.4078324632853\\
4923.4	62.4047933380725\\
4923.69999999994	62.4070839595434\\
4924.00000000002	62.4053126459658\\
4924.29999999997	62.4035415482089\\
4924.79999999997	62.4012670308027\\
4925.20000000007	62.4043205490021\\
4925.50000000007	62.4025499431541\\
4926.00000000007	62.4002760804693\\
4926.40000000007	62.4033289353496\\
4926.60000000003	62.4007956644407\\
4927.10000000006	62.4025815879201\\
4929.40000000004	62.3998377117355\\
4930.19999999999	62.4039105125449\\
4931.30000000003	62.4102688891593\\
4931.79999999994	62.4059693018918\\
4932.30000000006	62.4077528181007\\
4932.49999999999	62.405222397924\\
4933.00000000004	62.4070057367578\\
4933.99999999995	62.4105713301311\\
4934.29999999995	62.4088035019455\\
4934.90000000001	62.4052684903749\\
4935.29999999995	62.4083154354257\\
4935.70000000001	62.4052838445642\\
4936.99999999999	62.4111320410768\\
4937.59999999993	62.4075986795472\\
4938.39999999994	62.413688366913\\
4939	62.4162296774716\\
4939.19999999993	62.4137023464863\\
4939.59999999998	62.4106727129178\\
4940.10000000005	62.4063803084895\\
4941.09999999996	62.4099409050433\\
4941.99999999993	62.4147629550191\\
4942.80000000004	62.4188229581824\\
4943.10000000008	62.4170577763392\\
4943.50000000001	62.4200987134881\\
4943.69999999997	62.4175735264372\\
4944.19999999998	62.4153065145723\\
4944.5	62.4175868624358\\
4944.69999999994	62.4150622876557\\
4945.40000000004	62.4203821656051\\
4945.70000000003	62.4186178171378\\
4946.19999999995	62.422416755959\\
4946.80000000001	62.4249530008692\\
4947.30000000008	62.4226866637022\\
4948.19999999998	62.4254794575915\\
4949.59999999996	62.4300462654302\\
4949.80000000008	62.4275237883594\\
4950.29999999992	62.4313186813187\\
4951	62.4346104905981\\
4951.99999999993	62.4381575493225\\
4952.29999999993	62.4343752524029\\
4952.59999999998	62.4366507157712\\
4953.10000000005	62.4384236453202\\
4954.19999999997	62.4366711745352\\
4954.50000000006	62.4389456262867\\
4954.89999999997	62.4339051463168\\
4955.20000000005	62.4321433616532\\
4955.60000000006	62.435175656315\\
4955.89999999999	62.4313962873285\\
4957.10000000002	62.440490599532\\
4957.29999999994	62.4379715173276\\
4957.60000000002	62.4362103394719\\
4957.90000000005	62.4344493747479\\
4958.89999999997	62.4400080661424\\
4960.00000000007	62.4382572931997\\
4960.30000000007	62.4405289895976\\
4960.49999999997	62.4380115308632\\
4961.00000000003	62.4417971820766\\
4961.49999999999	62.4435665914221\\
4961.80000000001	62.4418065660332\\
4962.10000000008	62.444077223812\\
4962.3	62.4415605352249\\
4964.19999999995	62.4438490824487\\
4964.40000000004	62.4413334676201\\
4964.80000000008	62.4443594030091\\
4965.00000000002	62.4418440716199\\
4965.39999999992	62.4448696002417\\
4965.60000000005	62.4423545522283\\
4966.1	62.4400950424872\\
4966.39999999997	62.4383368569415\\
4966.70000000001	62.4365788837884\\
4967.50000000006	62.4426282309365\\
4967.7	62.4401143363259\\
4968.00000000001	62.4423823997102\\
4968.70000000007	62.4456609241668\\
4968.90000000001	62.4431475145905\\
4969.4	62.4469262501258\\
4970.00000000003	62.4494476972294\\
4970.50000000004	62.4512131332233\\
4970.80000000007	62.4474441248064\\
4971.39999999996	62.4499647993563\\
4971.89999999993	62.451729686243\\
4972.79999999993	62.446459812182\\
4973.40000000001	62.4509902483161\\
4974.10000000006	62.4542640022516\\
4974.90000000001	62.4522613065327\\
4975.20000000003	62.454525355255\\
4975.49999999993	62.450759707372\\
4975.89999999993	62.4537781350482\\
4976.79999999998	62.4585585404569\\
4977.19999999999	62.455548188777\\
4977.49999999993	62.4578109932498\\
4978.00000000001	62.4595729294309\\
4978.29999999994	62.4558090952917\\
4979.30000000002	62.461340723782\\
4979.49999999992	62.4588320347016\\
4979.80000000001	62.461093596257\\
4980.29999999998	62.4628543892057\\
4980.6	62.4590920954886\\
4981.29999999999	62.4643674469025\\
4982.00000000008	62.4616125730114\\
4983.09999999995	62.4598651468936\\
4983.50000000002	62.4628782406293\\
4983.8	62.4591183611228\\
4984.70000000001	62.4658963248275\\
4984.90000000006	62.4633901705115\\
4985.29999999995	62.4664018935291\\
4985.60000000006	62.462643159436\\
4986.20000000001	62.4671600184506\\
4986.50000000006	62.4634019171379\\
4986.79999999999	62.4656600292767\\
4988.00000000008	62.4686754475652\\
4988.29999999995	62.4669232619678\\
4988.60000000006	62.4691803475855\\
4988.99999999998	62.4661762642561\\
4989.29999999997	62.4644245801098\\
4989.59999999997	62.466681363609\\
4990	62.4616741147472\\
4990.3	62.4599230522603\\
4991.2	62.4646885580911\\
4991.59999999993	62.461686399423\\
4991.99999999997	62.4646942168627\\
4992.79999999993	62.4687055618979\\
4993.09999999994	62.4649523351758\\
4993.39999999996	62.4632021628117\\
4994.30000000008	62.4699663623258\\
4994.60000000001	62.4682163092879\\
4995.20000000004	62.4647168338238\\
4996.39999999998	62.4717302111478\\
4996.60000000005	62.4692296915965\\
4997.30000000005	62.4724856925601\\
4998.49999999994	62.4794942583924\\
4998.89999999995	62.4744948989798\\
4999.60000000004	62.4797487849271\\
4999.90000000002	62.478\\
5000.70000000003	62.4800031994881\\
5001.1	62.4750059985603\\
5001.39999999993	62.4732580225932\\
5001.70000000004	62.475508816826\\
5002.10000000008	62.4725120946784\\
5003.10000000001	62.4680204669012\\
5003.59999999997	62.4657753262586\\
5003.99999999998	62.4687756040047\\
5004.20000000008	62.4662790000599\\
5004.50000000002	62.4685289533629\\
5006.09999999998	62.470536534697\\
5006.4	62.4687905722561\\
5006.99999999997	62.4732879311378\\
5007.20000000007	62.4707926427416\\
5008.09999999995	62.4775368395831\\
5008.39999999993	62.4757911550364\\
5008.80000000007	62.4727984188145\\
5009.20000000008	62.4757950212605\\
5009.59999999993	62.4708066351279\\
5010.09999999996	62.4745519140952\\
5011.10000000005	62.4720625798212\\
5012.20000000001	62.4743131895537\\
5012.59999999996	62.4713228399864\\
5013.50000000006	62.4680868038934\\
5014.10000000006	62.4725778788241\\
5014.60000000002	62.4663489341336\\
5015.09999999992	62.4700909235923\\
5015.29999999995	62.4675997926387\\
5016.59999999995	62.4653656786334\\
5017.19999999994	62.4678612002471\\
5017.49999999995	62.4661192602041\\
5018.60000000001	62.4643832068065\\
5018.89999999998	62.4666268180913\\
5019.59999999999	62.4619001135526\\
5020.49999999992	62.4666374536908\\
5020.70000000008	62.4641491395793\\
5021.19999999998	62.4678868022225\\
5021.50000000003	62.4661462482077\\
5021.89999999997	62.4691358024691\\
5022.49999999994	62.4656552383228\\
5023.00000000006	62.4674006091856\\
5023.80000000004	62.4654153147953\\
5024.60000000005	62.4674109897108\\
5025.49999999996	62.4721426297358\\
5026.49999999995	62.4696614013449\\
5027.1	62.4721514958625\\
5028.09999999996	62.4696710552484\\
5028.39999999993	62.4719101123596\\
5028.59999999994	62.4694254976435\\
5028.90000000006	62.4716643467886\\
5029.19999999999	62.4699262322789\\
5029.70000000002	62.4716688536324\\
5030.00000000004	62.469931015288\\
5030.29999999994	62.4721692111959\\
5030.80000000008	62.4679480808603\\
5031.09999999993	62.4701860391159\\
5031.90000000009	62.4682034976153\\
5032.69999999995	62.4721824829121\\
5033.60000000005	62.4649859943978\\
5034.19999999993	62.4674731342987\\
5034.79999999994	62.4699596814237\\
5034.99999999995	62.4674783023177\\
5036.59999999999	62.4714594873628\\
5037.90000000001	62.4692338229456\\
5038.39999999997	62.4709735040191\\
5038.60000000001	62.4684938575426\\
5039.49999999994	62.4732121596952\\
5040.19999999998	62.4704878677856\\
5040.59999999996	62.4734659868669\\
5041.09999999998	62.4752043164326\\
5041.40000000004	62.4734701973619\\
5041.90000000004	62.4771915906386\\
5043.30000000006	62.48165919816\\
5043.60000000003	62.4799254515534\\
5045.00000000001	62.474480188698\\
5045.30000000002	62.4767114599437\\
5045.79999999994	62.4725024277136\\
5046.99999999993	62.4695369618197\\
5048.29999999996	62.4732588542905\\
5049.39999999997	62.4715318348351\\
5050.1	62.4688131163122\\
5050.60000000003	62.4705486368226\\
5050.80000000005	62.4680749965353\\
5051.49999999999	62.4712962229789\\
5051.89999999999	62.4683293745051\\
5052.69999999999	62.472292590247\\
5052.89999999996	62.469819908965\\
5053.39999999999	62.4675967151479\\
5053.70000000007	62.4658672681942\\
5054.19999999998	62.4676018439744\\
5054.70000000008	62.4693360766005\\
5055.30000000005	62.4619219052894\\
5055.69999999994	62.4648918074291\\
5056.00000000005	62.4631633076878\\
5056.60000000002	62.4597069234876\\
5057.00000000003	62.4626762373693\\
5057.30000000001	62.4609483133626\\
5057.59999999996	62.4631749609506\\
5058.4	62.4671345260453\\
5059.00000000008	62.4617026743887\\
5059.70000000006	62.464919562038\\
5060.19999999999	62.4666521747722\\
5060.49999999993	62.462949057424\\
5060.99999999997	62.4666574460098\\
5061.7	62.4639456319886\\
5062.10000000005	62.466911619454\\
5062.40000000003	62.4651851851852\\
5062.70000000004	62.4674093387059\\
5063.99999999999	62.4711202385419\\
5064.29999999997	62.4674196350999\\
5064.69999999996	62.464460590744\\
5064.99999999997	62.4627351878541\\
5065.40000000005	62.4656993386635\\
5065.59999999992	62.4632331168447\\
5065.89999999995	62.4615080931702\\
5066.40000000007	62.463238922333\\
5066.80000000007	62.4583078410863\\
5067.50000000008	62.4615202462704\\
5067.70000000004	62.4590552113343\\
5067.99999999997	62.4612774017877\\
5068.80000000006	62.4593106985737\\
5069.29999999997	62.4551228942281\\
5069.60000000003	62.4573446160522\\
5069.99999999992	62.454389459774\\
5070.60000000001	62.4588321139093\\
5070.8	62.4563686919482\\
5072.20000000006	62.4608165920786\\
5072.49999999993	62.4571225801364\\
5073.00000000003	62.4608227710867\\
5073.29999999999	62.4591004060394\\
5073.70000000007	62.4561472663487\\
5074.00000000008	62.4583670010445\\
5074.2	62.4559052480145\\
5075.20000000001	62.4534510275255\\
5075.60000000007	62.4564099533069\\
5075.99999999993	62.4514883473533\\
5078.00000000008	62.4603690356629\\
5078.80000000002	62.4584063478312\\
5079.29999999996	62.46210182305\\
5080.10000000003	62.460139364592\\
5080.40000000002	62.4584194469048\\
5080.70000000001	62.460636120296\\
5081.09999999994	62.4557191214674\\
5082.30000000005	62.4586809381395\\
5083.49999999992	62.461641356519\\
5084.19999999995	62.4648427512145\\
5084.40000000008	62.4623856819746\\
5084.90000000002	62.4601769911504\\
5085.20000000006	62.4584586946689\\
5086.09999999995	62.4552711257914\\
5086.39999999993	62.4535535240342\\
5087.59999999997	62.4604438154765\\
5088.09999999995	62.4562713729806\\
5088.59999999994	62.4520997504274\\
5089.39999999994	62.4501424501425\\
5090.29999999992	62.456781392425\\
5090.80000000007	62.4526115225206\\
5091.30000000008	62.4562988568959\\
5091.49999999999	62.4538455495326\\
5092.19999999995	62.4511517388999\\
5092.70000000007	62.4469839773798\\
5093.20000000005	62.4487071250466\\
5093.49999999998	62.4450290560704\\
5093.89999999999	62.4420887318414\\
5094.99999999994	62.4384212282389\\
5095.59999999993	62.4428439664815\\
5095.99999999996	62.4399050254116\\
5096.29999999992	62.4381916647045\\
5096.60000000007	62.4404026134558\\
5096.8	62.4379524809198\\
5097.60000000006	62.4438472252192\\
5097.80000000008	62.4413974381608\\
5098.40000000008	62.443856036089\\
5098.89999999998	62.4377328888017\\
5099.20000000005	62.4399427372384\\
5099.39999999992	62.4374938719482\\
5099.79999999998	62.4404400086276\\
5100.10000000003	62.4387278930238\\
5100.39999999999	62.4370159788256\\
5100.69999999994	62.4353042659975\\
5101.49999999995	62.4333542418065\\
5102.00000000005	62.4350757531213\\
5103.90000000002	62.4294670846395\\
5104.40000000004	62.4311881673034\\
5104.60000000002	62.4287421395968\\
5104.89999999995	62.4270323212537\\
5105.30000000005	62.4299761037333\\
5105.60000000002	62.4263078520085\\
5106.29999999992	62.4295002349992\\
5106.59999999998	62.4277909413124\\
5107.49999999994	62.4226642650168\\
5107.90000000009	62.4256068911511\\
5109.00000000002	62.4239102777397\\
5110.30000000006	62.4217282404508\\
5111.29999999998	62.4290800954729\\
5113.09999999993	62.4266604083548\\
5113.89999999995	62.4286272976144\\
5114.09999999999	62.4261859137304\\
5114.4	62.4283898719328\\
5114.70000000001	62.4247282396184\\
5115.19999999998	62.4225363126307\\
5115.60000000006	62.4254745196161\\
5116.59999999998	62.4289092579201\\
5116.90000000005	62.4252491694352\\
5118.19999999998	62.4230701600141\\
5118.50000000006	62.4252725354589\\
5119.69999999997	62.428219852338\\
5119.90000000001	62.42578125\\
5120.5	62.4301839628169\\
5121.00000000003	62.431899396614\\
5121.20000000003	62.4294612695995\\
5121.49999999994	62.431661980631\\
5121.79999999994	62.4299576329097\\
5122.1	62.4321580570848\\
5122.99999999999	62.4289980675763\\
5124.49999999999	62.4263357140069\\
5124.79999999993	62.4285351909306\\
5125.09999999996	62.4268321236245\\
5126.30000000002	62.4297752808989\\
5126.80000000005	62.4314888139031\\
5126.99999999994	62.4290534610208\\
5127.59999999996	62.431499502701\\
5127.80000000006	62.4290645293395\\
5128.40000000006	62.4334600760456\\
5128.60000000004	62.4310254060483\\
5128.9	62.4293234548645\\
5129.2	62.4276217027665\\
5130.79999999997	62.4256952971214\\
5132.09999999997	62.4235220763026\\
5132.59999999994	62.4252342821517\\
5132.79999999996	62.4228019248378\\
5133.10000000006	62.4211018468012\\
5133.80000000001	62.4164864917509\\
5134.10000000008	62.4186825600873\\
5135.00000000001	62.4213744620358\\
5136.50000000008	62.4167737413854\\
5138.10000000005	62.4187458643105\\
5138.29999999995	62.4163163630702\\
5138.90000000006	62.4187585133295\\
5139.69999999993	62.420716759407\\
5140.10000000008	62.4178047546788\\
5140.99999999995	62.4107681235533\\
5141.99999999997	62.4161334863188\\
5142.29999999994	62.4124922215308\\
5143.80000000004	62.4156768210891\\
5144.00000000008	62.4132501312183\\
5144.89999999998	62.4101068999028\\
5145.70000000007	62.406234210424\\
5146.69999999998	62.3999378254449\\
5147.10000000007	62.4028598072739\\
5147.60000000008	62.4006838005323\\
5147.9	62.4028749028749\\
5148.29999999994	62.3999689223836\\
5149.80000000005	62.3973281034583\\
5150.89999999992	62.3995340710542\\
5152.40000000003	62.402717127608\\
5152.99999999994	62.4051541790379\\
5154.29999999994	62.3971752289306\\
5154.79999999997	62.4008225183806\\
5155.10000000001	62.397191185599\\
5155.90000000001	62.4010861132661\\
5156.10000000004	62.3986656840309\\
5156.70000000005	62.4011014582687\\
5157.00000000003	62.3974714471311\\
5157.29999999999	62.3996587427774\\
5158.09999999997	62.4016129657633\\
5159.50000000005	62.4040623304132\\
5160.00000000003	62.4057673300905\\
5160.30000000003	62.4021393690412\\
5161.00000000006	62.4053011954816\\
5162.10000000003	62.4075006780055\\
5163.4	62.4092185533069\\
5163.70000000001	62.4055927805105\\
5164.40000000008	62.4087520573144\\
5164.60000000001	62.4063353147327\\
5164.89999999996	62.4085188770571\\
5166.49999999995	62.4143537335966\\
5167.89999999993	62.4187306501548\\
5168.29999999996	62.4158346877177\\
5169.09999999996	62.4139131780546\\
5169.90000000001	62.4119922630561\\
5170.40000000007	62.415627115366\\
5171.20000000003	62.4137064181154\\
5171.69999999994	62.4115395026877\\
5172.59999999999	62.4084134011251\\
5173.40000000006	62.4103604909636\\
5173.90000000003	62.4120603015075\\
5174.49999999993	62.4067560777645\\
5175.89999999996	62.4111282843895\\
5176.20000000002	62.4075111566177\\
5176.50000000008	62.4096897577561\\
5177.29999999995	62.4077722408931\\
5177.69999999992	62.4106763490285\\
5178.40000000007	62.4080332142512\\
5179.39999999994	62.4133603629694\\
5179.89999999998	62.4150579150579\\
5180.19999999997	62.4114433527016\\
5181.39999999992	62.4182186625494\\
5181.60000000005	62.4158094833742\\
5182.09999999995	62.4194357608738\\
5182.39999999999	62.4158224794983\\
5184.50000000005	62.41754426571\\
5185.30000000008	62.413700003857\\
5186.20000000002	62.4163661955537\\
5187	62.4183069537892\\
5187.59999999994	62.4207259479153\\
5187.79999999995	62.4183195512635\\
5188.19999999994	62.4212169689494\\
5188.39999999992	62.4188108316469\\
5189.80000000006	62.421241257057\\
5190.39999999993	62.4178788170696\\
5191.50000000004	62.4200631789814\\
5192.80000000006	62.4217681834813\\
5193.59999999999	62.4198548241138\\
5193.89999999997	62.4220254139392\\
5194.10000000004	62.4196218859497\\
5194.49999999992	62.4167404612482\\
5195.69999999999	62.4196466376689\\
5196.09999999998	62.4148416150264\\
5196.90000000005	62.4071579757553\\
5198.70000000006	62.410556282219\\
5198.90000000003	62.4081554145028\\
5199.49999999996	62.412493268713\\
5199.70000000002	62.4100926958729\\
5199.99999999996	62.4122613026673\\
5200.20000000003	62.4098609695594\\
5200.89999999995	62.4033839646222\\
5201.39999999994	62.4069979813515\\
5201.60000000005	62.4045984966453\\
5202.10000000001	62.4005228557149\\
5202.90000000003	62.402460119162\\
5203.10000000004	62.400061500615\\
5203.69999999996	62.4043967869634\\
5204.19999999997	62.4022442979844\\
5204.8	62.4046571499933\\
5205.00000000008	62.4022593225875\\
5205.40000000002	62.40514840073\\
5205.79999999994	62.4003534451296\\
5206.60000000007	62.4022893579426\\
5207.40000000002	62.4003840614498\\
5208.29999999992	62.3972813147992\\
5208.59999999995	62.3994470789256\\
5208.89999999995	62.3958533307737\\
5209.49999999995	62.3925061425062\\
5209.80000000006	62.3946716827578\\
5209.99999999994	62.3922765397977\\
5210.99999999998	62.3898984859243\\
5211.50000000001	62.3935067925397\\
5211.70000000007	62.3911124755363\\
5212.39999999999	62.3961630695444\\
5212.59999999995	62.3937690640167\\
5212.90000000006	62.3959332438135\\
5213.79999999997	62.3928345384453\\
5214.20000000002	62.3957194637823\\
5214.7	62.3935721408299\\
5215.5	62.3993404402178\\
5215.99999999992	62.3971933053431\\
5216.90000000007	62.3998466551658\\
5217.39999999999	62.3977000479157\\
5218.09999999992	62.4008278716799\\
5218.60000000009	62.3986816640159\\
5219.70000000006	62.4008582704318\\
5220.10000000009	62.3960767786675\\
5220.70000000004	62.3984829911125\\
5221.10000000005	62.3956178656248\\
5221.60000000007	62.3992186452688\\
5222.59999999995	62.3949298255692\\
5222.90000000009	62.3970897951369\\
5223.29999999998	62.3923115212314\\
5224	62.3896939185697\\
5224.7	62.39281886388\\
5224.89999999993	62.3904306220096\\
5225.49999999997	62.3928352725046\\
5226.40000000007	62.39548454989\\
5227.50000000005	62.391919810238\\
5228.30000000001	62.3900237166246\\
5229.1	62.3881282031668\\
5229.70000000004	62.3847948296302\\
5231.29999999995	62.3905646672019\\
5231.70000000004	62.3857945640124\\
5232.20000000001	62.3893889876345\\
5233.20000000008	62.3870215733858\\
5233.49999999994	62.3891776215225\\
5233.99999999992	62.3870388414436\\
5234.69999999998	62.3844272942615\\
5235.10000000002	62.3873013447433\\
5235.49999999992	62.384444953778\\
5235.79999999996	62.3866002024485\\
5236.00000000007	62.3842172609385\\
5236.40000000008	62.3870906139597\\
5237.29999999994	62.3897353648757\\
5237.80000000006	62.3875980832013\\
5238.29999999999	62.3911881490531\\
5239.30000000001	62.3888231476887\\
5239.60000000007	62.3909765826288\\
5240.00000000006	62.388122364077\\
5241.00000000002	62.3857587147736\\
5241.40000000004	62.3829056567777\\
5241.70000000002	62.3850585676676\\
5242.6	62.3762565090507\\
5243.70000000007	62.380334871658\\
5245.00000000008	62.3782196716936\\
5246.20000000003	62.3734822636906\\
5246.70000000008	62.3770679271175\\
5247.80000000007	62.3792374092494\\
5248.19999999999	62.3744831659776\\
5248.60000000001	62.3773505820489\\
5249.09999999995	62.3752190810028\\
5250.20000000001	62.3792926118508\\
5250.90000000007	62.3824033517425\\
5252.19999999997	62.3745787559736\\
5252.70000000002	62.3781602193116\\
5252.90000000008	62.3757852655625\\
5253.79999999995	62.3784236471954\\
5253.99999999995	62.3760491806399\\
5254.50000000001	62.3796292772047\\
5255.10000000002	62.3820216166844\\
5255.80000000001	62.379421221865\\
5256.09999999994	62.3815684334691\\
5257.20000000006	62.3856352119909\\
5257.60000000006	62.3827909542195\\
5258.09999999993	62.3806625841543\\
5258.49999999992	62.3835241318982\\
5258.70000000002	62.3811515935194\\
5259.99999999997	62.3752400144484\\
5260.50000000003	62.3731133330799\\
5262.6	62.3748266099151\\
5263.4	62.3729457585257\\
5265.49999999994	62.3708599209967\\
5266.49999999994	62.36851099381\\
5266.79999999992	62.3706544646756\\
5267.30000000008	62.368530964043\\
5268.29999999994	62.3737757193835\\
5268.90000000002	62.3685708863162\\
5269.19999999998	62.3707133774885\\
5269.40000000002	62.3683461428978\\
5269.89999999999	62.3662239089184\\
5270.20000000004	62.3683661271654\\
5271.29999999993	62.370527753538\\
5272.60000000002	62.3741157281848\\
5272.80000000005	62.3717498909518\\
5273.99999999999	62.3746231584536\\
5274.20000000009	62.3722579299623\\
5274.89999999994	62.3772511848341\\
5275.19999999998	62.373703865183\\
5275.50000000003	62.3758435059519\\
5275.70000000002	62.3734789036734\\
5276.89999999994	62.380140231192\\
5277.40000000002	62.378019895784\\
5277.89999999996	62.3758999621069\\
5279.09999999997	62.3730868313381\\
5281.59999999998	62.3852168809285\\
5282.10000000004	62.3830979516111\\
5282.59999999996	62.3809794234009\\
5283.30000000001	62.3859635840557\\
5284.10000000007	62.3897657166648\\
5284.30000000002	62.3874044356975\\
5285.1	62.3930977068039\\
5285.30000000003	62.3907367465093\\
5285.99999999999	62.388150053915\\
5286.70000000002	62.385564046304\\
5286.99999999999	62.3876983601596\\
5287.80000000005	62.3858242402466\\
5288.09999999993	62.3879580953822\\
5288.89999999999	62.3898657591227\\
5289.50000000002	62.3865698729583\\
5289.79999999998	62.3887030000567\\
5290.49999999998	62.3917892110536\\
5290.79999999996	62.3882515262054\\
5291.09999999997	62.3903840338676\\
5292.00000000001	62.3930008881162\\
5292.49999999996	62.3908853871443\\
5293.10000000005	62.3951484924054\\
5293.30000000005	62.3927910227831\\
5293.60000000006	62.3949222660899\\
5294.40000000005	62.3892718859194\\
5295.09999999999	62.3942438434809\\
5295.40000000004	62.3907090926258\\
5295.89999999992	62.3885951661631\\
5296.70000000003	62.3942757891557\\
5296.89999999992	62.3919199546913\\
5297.70000000002	62.395711427385\\
5297.90000000001	62.39335598339\\
5298.30000000004	62.3961950777593\\
5299.8	62.3992905526519\\
5300.29999999995	62.397177571504\\
5300.89999999997	62.4014336917563\\
5301.70000000004	62.3920178052737\\
5302.89999999999	62.3967565528946\\
5304.50000000002	62.4043283188176\\
5305.30000000008	62.4024578731104\\
5305.99999999993	62.3998793841051\\
5306.89999999997	62.4024872809497\\
5307.39999999997	62.4003768252473\\
5307.80000000006	62.3975583564121\\
5308.50000000008	62.4006329352372\\
5308.79999999997	62.3971067452768\\
5309.20000000007	62.3999397284011\\
5309.59999999993	62.3952389023862\\
5311.1	62.4002108751318\\
5312.49999999993	62.4025900688928\\
5312.89999999995	62.3997741389046\\
5313.50000000004	62.402137910268\\
5314.10000000003	62.404501147868\\
5314.89999999999	62.4063969896519\\
5315.2	62.4028747201475\\
5316.00000000001	62.4047704144015\\
5316.19999999994	62.4024227376183\\
5317.09999999995	62.3993831339803\\
5317.70000000009	62.3961036518861\\
5318.60000000003	62.4005866095098\\
5319.09999999998	62.3966009926305\\
5319.40000000005	62.3987216843688\\
5320.20000000007	62.400616506588\\
5320.40000000001	62.3982708392068\\
5322.40000000007	62.4030061061531\\
5322.60000000007	62.4006613185038\\
5322.99999999998	62.397850876369\\
5323.39999999995	62.4006762468301\\
5323.60000000002	62.3983319871518\\
5324.19999999993	62.4006911706703\\
5324.79999999997	62.3955379443746\\
5325.60000000009	62.3936759487016\\
5327.09999999999	62.3911247935125\\
5328.20000000008	62.3951354090423\\
5328.99999999998	62.3932746617628\\
5329.59999999998	62.3956320243166\\
5330.00000000007	62.3909495131423\\
5330.49999999993	62.3944771695494\\
5330.70000000003	62.3921362647257\\
5330.99999999997	62.3942525932734\\
5331.4	62.3914470599269\\
5331.9	62.3893473368342\\
5333.50000000008	62.3856307184641\\
5334.3	62.3837732453509\\
5334.69999999992	62.3865936867361\\
5336.09999999993	62.39271391627\\
5336.30000000004	62.3903755340679\\
5336.80000000002	62.388277839195\\
5337.49999999996	62.3913369304556\\
5338.19999999992	62.385028941798\\
5338.59999999995	62.3878472287261\\
5339.40000000001	62.3859911976777\\
5339.79999999998	62.3888087791906\\
5340.19999999997	62.3860082766886\\
5341.10000000003	62.3886018123268\\
5341.39999999995	62.3850978189647\\
5342.19999999998	62.3869868783108\\
5342.90000000003	62.3844282238443\\
5343.40000000005	62.3879479741742\\
5343.99999999994	62.3846859153085\\
5345.30000000002	62.3826093463539\\
5345.69999999992	62.3854240712335\\
5345.90000000001	62.3830901608679\\
5346.50000000005	62.3854412149777\\
5346.8	62.3819409377396\\
5347.50000000007	62.3793851447378\\
5348.10000000007	62.3817359111477\\
5348.90000000009	62.3798840904842\\
5349.70000000003	62.3817712811694\\
5350.10000000006	62.3789764868603\\
5351.59999999994	62.3820468262421\\
5351.99999999995	62.379253003494\\
5353.00000000002	62.3825446937289\\
5353.49999999996	62.3804542737597\\
5353.80000000008	62.3825622443452\\
5354.10000000007	62.3790669007508\\
5354.59999999993	62.3769772349525\\
5355.39999999995	62.3788628512744\\
5355.59999999992	62.3765334129992\\
5356.29999999994	62.3795833022179\\
5356.70000000006	62.3767921146953\\
5357.60000000006	62.3812456837822\\
5357.80000000001	62.378917113048\\
5358.69999999995	62.3759050533702\\
5359.39999999992	62.3808191062599\\
5359.6	62.3784913334702\\
5360.29999999993	62.3815386911425\\
5360.79999999992	62.3775858531217\\
5361.10000000009	62.3796911139297\\
5361.90000000003	62.3834390152928\\
5362.30000000006	62.3806504550201\\
5362.59999999996	62.3827549555261\\
5363.19999999994	62.3776406317006\\
5363.60000000004	62.374853179708\\
5364.09999999992	62.3727676074718\\
5365.00000000006	62.3753518107771\\
5366.60000000002	62.377252315203\\
5368.29999999999	62.3798524700097\\
5368.59999999998	62.3763667181999\\
5369.40000000002	62.3801098798771\\
5369.69999999993	62.3766248277403\\
5369.99999999999	62.378726653135\\
5370.60000000008	62.3810676448135\\
5371.30000000006	62.3841084261087\\
5371.49999999999	62.3817856876908\\
5373.00000000009	62.37925964527\\
5373.60000000003	62.376016524927\\
5374.6	62.373713881705\\
5375.90000000004	62.3828125\\
5376.70000000001	62.3846897783068\\
5377.50000000008	62.3828473668551\\
5379.10000000001	62.3810232004759\\
5379.79999999998	62.3840591832562\\
5380.40000000008	62.3808196264288\\
5380.89999999994	62.3787400111504\\
5382.49999999995	62.3713447033032\\
5383.69999999998	62.376016939708\\
5383.99999999999	62.372541371817\\
5384.49999999999	62.376035360101\\
5384.99999999997	62.3739577723719\\
5385.29999999998	62.3760537750214\\
5385.49999999993	62.3737373737374\\
5386.00000000007	62.3716603850653\\
5388.79999999999	62.3745105680194\\
5389.49999999998	62.377541932611\\
5391.10000000002	62.379433150319\\
5391.4	62.3759621626635\\
5392.59999999998	62.3732082259351\\
5393.30000000007	62.3688211517781\\
5393.79999999996	62.3723094606871\\
5394.19999999992	62.3695382162653\\
5395.29999999994	62.3716499240093\\
5396.00000000008	62.3746780081911\\
5396.50000000001	62.3726049735018\\
5398.09999999996	62.3744952021044\\
5399.90000000003	62.3814814814815\\
5400.20000000008	62.3780160361461\\
5400.69999999992	62.3759443045475\\
5400.99999999996	62.378034104164\\
5401.70000000005	62.3755044614758\\
5402.59999999996	62.3725174449812\\
5402.9	62.3746066999815\\
5404.70000000005	62.3834369449378\\
5404.90000000005	62.3811285846439\\
5405.89999999998	62.378838327784\\
5406.39999999997	62.3767687043374\\
5406.9	62.3802478268911\\
5407.10000000005	62.3779405237461\\
5408.7	62.3816743085342\\
5409.00000000002	62.3782144903958\\
5409.80000000002	62.3837778886856\\
5410.30000000002	62.3817093006062\\
5410.70000000005	62.3789458120796\\
5411.20000000003	62.382421968843\\
5411.39999999994	62.3801164187379\\
5411.69999999997	62.3822018552053\\
5412.30000000009	62.3771339886187\\
5412.89999999993	62.3794568631073\\
5414.30000000008	62.3762559101655\\
5414.80000000004	62.3741897357292\\
5415.80000000006	62.3663657009915\\
5416.19999999999	62.3691449882761\\
5416.59999999998	62.3663854376281\\
5417.40000000004	62.3645592985695\\
5417.80000000007	62.3673378984477\\
5418.49999999993	62.36481748053\\
5418.80000000003	62.3669010315747\\
5419.39999999997	62.3692222529754\\
5420.40000000003	62.3743197122037\\
5420.69999999996	62.370867768595\\
5421.49999999993	62.369042349122\\
5422.70000000004	62.3718374271594\\
5422.90000000007	62.3695371565554\\
5423.19999999992	62.3716187561079\\
5423.40000000007	62.3693187056329\\
5424.29999999994	62.3737187523044\\
5425.10000000002	62.3755806237558\\
5425.90000000001	62.3792849244379\\
5426.60000000004	62.3822949490482\\
5427.50000000007	62.3848478148721\\
5428.30000000004	62.383022621767\\
5429.2	62.3874164256902\\
5429.59999999998	62.3828204136508\\
5431.70000000002	62.3899996317979\\
5432.60000000003	62.3925488247096\\
5433.00000000009	62.3879553109643\\
5433.39999999998	62.3907242109138\\
5433.99999999996	62.3875158719935\\
5435.09999999992	62.3914483367677\\
5435.59999999996	62.3893886711923\\
5435.99999999995	62.3866374790751\\
5436.60000000001	62.3834311254989\\
5436.90000000004	62.385506713261\\
5437.09999999999	62.3832119473258\\
5437.40000000006	62.3852873563218\\
5437.59999999994	62.3829928094599\\
5438.29999999995	62.3859958811415\\
5439.10000000007	62.3841741432564\\
5439.70000000005	62.388323100114\\
5439.89999999998	62.3860294117647\\
5440.69999999994	62.3842082046758\\
5441.80000000001	62.3881364964443\\
5442.29999999999	62.3842422460679\\
5443.1	62.3860964138742\\
5444.20000000003	62.391859375861\\
5444.70000000006	62.3861298853952\\
5445.30000000009	62.3884379476255\\
5445.49999999995	62.3861466137799\\
5446.59999999999	62.3882350781207\\
5447.40000000007	62.391922900413\\
5448.99999999993	62.3956249655907\\
5450.09999999991	62.3977101757734\\
5450.39999999998	62.3942757545179\\
5451.00000000005	62.3984149988076\\
5451.19999999998	62.3961256947884\\
5451.49999999999	62.3981950253137\\
5451.70000000002	62.3959059393228\\
5452.49999999998	62.3977551993544\\
5452.79999999995	62.3943222872233\\
5453.29999999998	62.3977701984083\\
5453.49999999994	62.3954818835265\\
5454.29999999995	62.3936638310355\\
5455.09999999996	62.3900131984162\\
5455.39999999996	62.3920813857575\\
5456.99999999996	62.3902805519415\\
5457.50000000005	62.3937261800059\\
5457.70000000006	62.391439774268\\
5458.69999999998	62.3946654942478\\
5458.90000000007	62.3923795566954\\
5459.39999999993	62.3903287846872\\
5459.79999999992	62.3930841224198\\
5460.40000000006	62.3898910356195\\
5460.99999999997	62.3940231821428\\
5461.19999999994	62.3917382308242\\
5462.60000000004	62.3885624324968\\
5462.99999999995	62.3913162856254\\
5463.89999999999	62.3883601756955\\
5464.80000000006	62.3854050394335\\
5465.30000000003	62.3815274270868\\
5465.79999999994	62.3849686236484\\
5465.99999999995	62.3826860101352\\
5467.29999999997	62.3806562534294\\
5467.89999999997	62.3774689100219\\
5468.29999999996	62.3802209055665\\
5468.49999999999	62.377939509198\\
5468.89999999995	62.3752057048821\\
5469.90000000009	62.3784277879342\\
5470.69999999999	62.3802734517804\\
5470.89999999993	62.3779930542863\\
5471.80000000005	62.375043403571\\
5472.10000000005	62.3771060999233\\
5472.99999999995	62.3796385960425\\
5473.29999999995	62.3762195344758\\
5473.79999999991	62.3741756334606\\
5474.70000000001	62.3803609264265\\
5475.50000000008	62.382204689897\\
5476.10000000004	62.3845002008692\\
5476.59999999993	62.378804754688\\
5478.50000000005	62.3809002299858\\
5479.29999999992	62.382742636055\\
5479.59999999997	62.3793273354381\\
5479.89999999998	62.3813868613139\\
5480.10000000002	62.3791102514507\\
5480.79999999997	62.3747924610922\\
5481.49999999995	62.3723000583771\\
5482.20000000008	62.3771044999362\\
5482.49999999997	62.3736913143399\\
5482.89999999997	62.3764362575233\\
5483.10000000006	62.3741610738255\\
5483.49999999995	62.3769056823984\\
5485.5	62.3815079480823\\
5485.69999999993	62.3792336578074\\
5486.20000000008	62.3771941016714\\
5486.79999999996	62.374018115876\\
5487.60000000007	62.3776809956813\\
5488.49999999992	62.3802062456729\\
5488.70000000009	62.3779332458825\\
5489.30000000008	62.3802237038656\\
5490.20000000006	62.3845691492268\\
5490.50000000003	62.3811605289039\\
5491.49999999996	62.3789059654745\\
5492.30000000008	62.3807443012162\\
5493.10000000001	62.3789412364378\\
5493.70000000008	62.3830499836179\\
5494.50000000005	62.3776071051578\\
5495.3	62.3721658114059\\
5495.80000000006	62.370130460889\\
5497.39999999998	62.3719872669395\\
5497.60000000006	62.3697182458119\\
5497.99999999997	62.3724559393245\\
5499.10000000003	62.3781640965959\\
5500.90000000009	62.3759316487911\\
5501.90000000002	62.3809523809524\\
5502.59999999994	62.3766514620096\\
5503.09999999999	62.3746184038378\\
5503.90000000007	62.3800872093023\\
5504.30000000007	62.3773708306082\\
5504.89999999992	62.3814713896458\\
5505.19999999995	62.3780720396709\\
5506.40000000007	62.3753745573413\\
5506.79999999997	62.3781074651801\\
5507.00000000005	62.3758420947504\\
5507.70000000001	62.3733614147209\\
5508.10000000003	62.3760938237537\\
5508.29999999999	62.3738290610704\\
5509.60000000009	62.3718169773309\\
5510.09999999998	62.3752313890603\\
5510.70000000002	62.3775132467156\\
5510.89999999995	62.375249500998\\
5511.40000000007	62.3732196316792\\
5511.70000000002	62.3752676076781\\
5512.30000000008	62.3775487990712\\
5512.5	62.3752857091028\\
5513.79999999997	62.3732748145596\\
5514.30000000009	62.3766864935442\\
5515.30000000006	62.3816948906698\\
5515.80000000006	62.3796660563099\\
5516.79999999998	62.3774221029926\\
5517.99999999997	62.3801670864972\\
5518.30000000008	62.3767758770658\\
5518.90000000007	62.3790541764812\\
5520.59999999993	62.3815820457551\\
5521.79999999993	62.3843242362231\\
5522.00000000003	62.3820647941906\\
5523.00000000006	62.3870652351035\\
5523.19999999995	62.3848061847084\\
5523.80000000002	62.387081590905\\
5525.19999999994	62.3839429533238\\
5525.59999999994	62.3812367663825\\
5525.89999999995	62.3832790445168\\
5526.29999999999	62.3787637521714\\
5526.80000000003	62.3767392209014\\
5527.69999999996	62.3810557545497\\
5528.10000000001	62.3783510003256\\
5528.90000000002	62.3801772472418\\
5529.39999999999	62.3781535401031\\
5530.70000000005	62.3761481160049\\
5531.00000000007	62.3781887870406\\
5531.49999999996	62.3761660279124\\
5532.89999999997	62.3730345201518\\
5533.49999999992	62.3753072141102\\
5534.20000000001	62.3782592197749\\
5534.89999999998	62.3757904245709\\
5535.89999999996	62.3735549132948\\
5536.50000000005	62.3758263194018\\
5536.79999999998	62.3724466759378\\
5537.20000000001	62.3751647915049\\
5537.50000000001	62.3717856110951\\
5538.10000000002	62.3686396302048\\
5539.09999999999	62.3664067013287\\
5539.60000000001	62.3643879632471\\
5542.20000000008	62.3621962001335\\
5542.59999999993	62.3649124073105\\
5543.90000000006	62.3683261183261\\
5544.10000000006	62.3660762598752\\
5544.50000000003	62.3687912563575\\
5545.30000000003	62.3652035921665\\
5545.70000000006	62.3679180641206\\
5546.49999999991	62.3697400209137\\
5546.70000000008	62.3674911660777\\
5547.00000000006	62.36952641921\\
5547.29999999991	62.3661535133576\\
5547.60000000006	62.3681886187069\\
5547.8	62.3659402656861\\
5548.39999999997	62.3682076236821\\
5548.60000000009	62.3659595941392\\
5550	62.3610385398461\\
5550.80000000008	62.3628600767443\\
5550.99999999991	62.3606132117959\\
5553.80000000003	62.363384288518\\
5554.79999999992	62.3611586167168\\
5555.49999999999	62.3641010871913\\
5556.49999999996	62.367274952309\\
5556.79999999995	62.3639079342799\\
5557.20000000004	62.366616882299\\
5557.39999999997	62.3643724696356\\
5558.30000000004	62.3704663212435\\
5558.50000000002	62.3682222142266\\
5559.99999999998	62.3729789032571\\
5560.49999999997	62.3709671618171\\
5561.20000000006	62.3757035225577\\
5561.79999999994	62.3725705244611\\
5563.09999999998	62.3705780845557\\
5563.99999999998	62.3730702180047\\
5564.89999999996	62.3755615453729\\
5565.39999999994	62.3735513430959\\
5566.90000000004	62.3765044009341\\
5567.70000000005	62.3783181867165\\
5567.89999999995	62.3760775862069\\
5568.29999999996	62.3787802600388\\
5568.49999999994	62.3765398843515\\
5568.79999999999	62.3785666828279\\
5569.40000000004	62.3754376514947\\
5569.90000000006	62.3788150807899\\
5570.29999999997	62.3761309780267\\
5571.30000000002	62.3739096097929\\
5571.59999999998	62.3759355313459\\
5573.20000000003	62.3777654172573\\
5574.60000000005	62.3746569322116\\
5575.20000000004	62.369737951321\\
5575.49999999997	62.3717626802497\\
5577.40000000006	62.3756163155536\\
5577.70000000002	62.3722614650938\\
5578.49999999999	62.3740723479009\\
5578.70000000008	62.3718362371836\\
5579.69999999991	62.3696189827592\\
5580.30000000008	62.373664970253\\
5580.49999999994	62.371429595384\\
5581.00000000001	62.3676336206124\\
5581.60000000003	62.3716788791945\\
5582.59999999998	62.3694628047361\\
5583.00000000007	62.3721588364887\\
5583.50000000001	62.3683644960241\\
5583.80000000009	62.3703862891527\\
5584.00000000005	62.3681524328003\\
5584.99999999995	62.3713093767345\\
5585.50000000004	62.3657261529648\\
5585.79999999995	62.3677473638984\\
5586.59999999996	62.3659763366567\\
5587.19999999997	62.370017718755\\
5587.99999999999	62.373615361214\\
5589.60000000006	62.3700735280963\\
5589.99999999999	62.3727661401406\\
5590.40000000002	62.3700921205617\\
5591.49999999997	62.3721296230059\\
5592.40000000006	62.369244523916\\
5592.79999999999	62.3719358472349\\
5593.19999999992	62.3674753723205\\
5594.00000000004	62.369281922025\\
5595.10000000006	62.3731055190163\\
5595.40000000003	62.3697614154231\\
5595.89999999992	62.373123659757\\
5596.19999999992	62.3697800332363\\
5596.49999999998	62.371797162563\\
5596.80000000005	62.3684539655881\\
5597.20000000008	62.3657835027603\\
5598.50000000001	62.3638052370235\\
5601.50000000008	62.3607540702656\\
5601.79999999997	62.3627697745408\\
5602.90000000004	62.3594502944851\\
5604.70000000002	62.362617756209\\
5605	62.3592799414819\\
5606.09999999996	62.361314259213\\
5606.29999999998	62.3590896118722\\
5607.09999999992	62.3608931373948\\
5608.00000000003	62.3633672723382\\
5608.60000000003	62.3656105692941\\
5609.50000000008	62.3680832857958\\
5610.30000000006	62.3698845002139\\
5610.60000000004	62.3665496283886\\
5612.99999999994	62.3701697814042\\
5613.40000000001	62.3657254832101\\
5614.59999999998	62.3684257395765\\
5614.8	62.3662042066644\\
5615.30000000004	62.3695551519037\\
5615.60000000001	62.3662232669124\\
5615.90000000009	62.3682336182336\\
5616.5	62.3633514937863\\
5617.49999999993	62.361150669325\\
5618.20000000003	62.3658402007725\\
5618.49999999999	62.3625102338661\\
5618.99999999996	62.3587407236034\\
5619.29999999993	62.3607502580347\\
5620.10000000005	62.3589907832462\\
5620.50000000001	62.3616695726435\\
5620.89999999994	62.3590108521615\\
5621.49999999998	62.3612494663441\\
5622.09999999995	62.3581516132475\\
5622.40000000008	62.3601600711427\\
5622.60000000006	62.3579419140271\\
5623.30000000002	62.353736173845\\
5625.40000000007	62.3517909519154\\
5625.70000000007	62.3537985708699\\
5626.99999999996	62.3500559791011\\
5627.30000000006	62.3520631197356\\
5628.09999999997	62.3503073806901\\
5629.19999999992	62.3470058444212\\
5630.80000000001	62.3452734021204\\
5631.60000000005	62.3506223697995\\
5632.39999999996	62.352418996893\\
5632.69999999994	62.3490981394688\\
5633.00000000009	62.3511033001367\\
5634.19999999996	62.3484727472801\\
5635.20000000005	62.3516050609551\\
5635.69999999998	62.3496220589801\\
5636.20000000004	62.3529620495715\\
5637.20000000004	62.3578663544605\\
5637.90000000003	62.3607662291593\\
5638.4	62.3570098430434\\
5638.69999999994	62.3590125558629\\
5639.20000000004	62.3570301278528\\
5641.69999999994	62.3524407104116\\
5643.09999999991	62.3546923731216\\
5643.49999999992	62.3502728754696\\
5645.80000000007	62.3549832621903\\
5646.00000000003	62.3527744815005\\
5646.29999999998	62.3547747237178\\
5646.49999999991	62.3525661459994\\
5646.89999999992	62.355232867009\\
5647.29999999992	62.3525870312002\\
5647.69999999997	62.3552533729948\\
5647.89999999994	62.353045325779\\
5648.40000000005	62.3510666548641\\
5648.69999999991	62.3530661379408\\
5649.19999999994	62.3510877453844\\
5649.99999999993	62.3493389497531\\
5651.20000000001	62.3537947021039\\
5651.80000000002	62.350713919213\\
5652.10000000002	62.3527122182513\\
5652.30000000004	62.3505059797608\\
5654.10000000005	62.3483428248028\\
5656.40000000004	62.3512772916114\\
5656.60000000009	62.34907278095\\
5657.30000000003	62.3519637996253\\
5657.49999999997	62.3497596153846\\
5658.29999999998	62.355082708893\\
5658.80000000004	62.3531074943894\\
5660.70000000006	62.3551441492368\\
5660.99999999994	62.3518397484588\\
5661.49999999996	62.3498657623287\\
5662.10000000005	62.3467909999647\\
5662.49999999995	62.3494507823261\\
5662.70000000009	62.3472487108851\\
5663.39999999995	62.3519025337689\\
5664.90000000009	62.3548102383054\\
5665.69999999994	62.3530657629991\\
5666.10000000001	62.3557234125163\\
5667.00000000005	62.3581726103298\\
5667.69999999997	62.3610572003247\\
5668.39999999995	62.3639410778866\\
5668.80000000003	62.3595406516255\\
5669.09999999996	62.3615324913568\\
5669.30000000003	62.3593325572371\\
5670.1	62.3646432224613\\
5670.59999999994	62.3591443737105\\
5671.39999999998	62.3556378383144\\
5671.89999999999	62.3536671368124\\
5672.70000000009	62.350162177408\\
5673.10000000008	62.3475287315801\\
5673.59999999996	62.3455593351781\\
5674.99999999993	62.3478000387658\\
5675.19999999992	62.3456028756189\\
5675.70000000001	62.3489199760386\\
5675.89999999993	62.3467230443975\\
5676.19999999991	62.3487130701337\\
5677.09999999993	62.3511590220531\\
5677.30000000001	62.3489625532814\\
5678.10000000004	62.3454615899405\\
5678.90000000001	62.3472442331396\\
5679.30000000004	62.3446138676621\\
5680.59999999996	62.3426690372665\\
5681.10000000006	62.3389424769415\\
5681.39999999998	62.3409310921412\\
5681.90000000003	62.3389651531151\\
5682.29999999995	62.3416162184992\\
5683.10000000008	62.3433980855856\\
5683.69999999999	62.3456138498892\\
5684.49999999993	62.3438764380959\\
5684.79999999994	62.3458636035814\\
5685.59999999992	62.347644089558\\
5686.50000000006	62.3448106073928\\
5687.59999999993	62.3468185734128\\
5688.19999999993	62.3490322240388\\
5689.40000000002	62.3464276298445\\
5690.00000000003	62.3486406214302\\
5690.89999999997	62.3510806536637\\
5691.09999999998	62.3488895136351\\
5692.50000000006	62.3563925095738\\
5692.69999999997	62.3542017987634\\
5693.50000000002	62.3507095686385\\
5695.10000000007	62.3525073746313\\
5695.30000000002	62.3503178003301\\
5695.8	62.3483558348988\\
5696.99999999999	62.3510206947394\\
5698.20000000004	62.355439341558\\
5698.40000000005	62.3532508554883\\
5698.69999999993	62.3552326805643\\
5699.09999999998	62.3526108927569\\
5699.49999999999	62.3552530002105\\
5700.29999999999	62.3535190512947\\
5701.30000000008	62.3566141649419\\
5701.79999999992	62.3511461091917\\
5703.50000000003	62.348341398415\\
5704.30000000004	62.346609634668\\
5704.70000000009	62.3492497545926\\
5705.80000000004	62.3442401724531\\
5706.10000000004	62.3462199011601\\
5706.29999999993	62.3440347679798\\
5707.7	62.3480149970216\\
5708.20000000002	62.3460574952262\\
5708.70000000002	62.3441003363229\\
5709.60000000001	62.3500359038128\\
5710.39999999996	62.3535592329919\\
5710.60000000001	62.3513754881188\\
5711.49999999998	62.3485538202955\\
5712.30000000005	62.3468244520692\\
5713.29999999998	62.3516645079987\\
5713.49999999994	62.3494819378326\\
5714.59999999995	62.351479517735\\
5715.39999999998	62.3497506779809\\
5715.89999999997	62.3477956613016\\
5716.20000000006	62.3497717054738\\
5716.59999999994	62.3454090646702\\
5716.99999999991	62.3427961728849\\
5718.19999999993	62.3402060052813\\
5718.89999999996	62.3378212974296\\
5719.39999999994	62.341113733718\\
5719.59999999993	62.3389338601675\\
5720.00000000001	62.3415674551144\\
5720.59999999998	62.3385250056811\\
5721.40000000003	62.3367997902648\\
5721.90000000009	62.3348479552604\\
5723.10000000001	62.3287671232877\\
5723.79999999993	62.3263858557976\\
5724.60000000007	62.3246633011337\\
5726.10000000009	62.3223079878453\\
5726.59999999994	62.325597639129\\
5727.49999999993	62.3227879041833\\
5729.00000000004	62.3256706987136\\
5729.49999999996	62.3237224239039\\
5729.79999999997	62.3256950383078\\
5731.40000000007	62.3309779289889\\
5731.99999999994	62.3331763228136\\
5732.40000000001	62.3305713039686\\
5733.20000000006	62.3288507491323\\
5733.9	62.3334495988839\\
5734.50000000001	62.3286715725595\\
5734.80000000007	62.3306422082338\\
5735.29999999995	62.3269519126826\\
5736.60000000006	62.3215437446616\\
5737.59999999997	62.3193962737682\\
5738.89999999994	62.3279316954173\\
5740.00000000003	62.3299245657741\\
5740.59999999996	62.3321197763339\\
5741.90000000001	62.3301985370951\\
5742.69999999992	62.3337048129832\\
5743.30000000003	62.330675209806\\
5743.80000000002	62.3339542819339\\
5744.40000000008	62.3309252328314\\
5744.70000000006	62.3328923548252\\
5744.90000000007	62.3307223672759\\
5746.19999999993	62.3357638828464\\
5749.79999999991	62.3419537731091\\
5750.40000000007	62.3441439874794\\
5750.69999999994	62.3408917020241\\
5750.99999999995	62.3428561492584\\
5751.49999999991	62.3409138326727\\
5752	62.3441873402757\\
5752.19999999992	62.3420197138536\\
5752.90000000007	62.3448635494525\\
5753.70000000001	62.3466231012548\\
5754.40000000007	62.344252324268\\
5754.70000000003	62.346215333287\\
5756.29999999995	62.3497324716837\\
5756.49999999998	62.3475662717576\\
5757.29999999994	62.3493243477959\\
5757.59999999991	62.3460756899456\\
5758.39999999992	62.3478336372319\\
5759.2	62.3461184518952\\
5759.70000000005	62.3441786173131\\
5761.30000000003	62.3459575797549\\
5761.49999999993	62.3437933907248\\
5762.80000000002	62.3418764857971\\
5763.30000000006	62.3382031439775\\
5763.99999999993	62.3358373380059\\
5764.7000000001	62.3404107688038\\
5764.90000000004	62.338248048569\\
5767.59999999995	62.3420080794771\\
5767.80000000003	62.3398463912342\\
5768.09999999996	62.3418050691724\\
5768.40000000001	62.3385628846321\\
5768.70000000008	62.3405214255998\\
5769.00000000005	62.3372796450053\\
5769.79999999992	62.3355690739874\\
5770.10000000007	62.3375272954144\\
5771.59999999992	62.3351872065423\\
5771.99999999996	62.3377973354585\\
5773.30000000005	62.3341531852981\\
5773.80000000007	62.337414918859\\
5774.39999999996	62.3344012468612\\
5775.09999999999	62.33723507411\\
5775.50000000001	62.3329177920909\\
5776.00000000009	62.3361783902633\\
5776.8	62.3275459156295\\
5777.79999999992	62.3306045449039\\
5778.30000000009	62.3286722968296\\
5778.59999999997	62.330627995916\\
5778.79999999993	62.3284708162453\\
5779.10000000003	62.3304263565892\\
5779.90000000006	62.3287197231834\\
5780.40000000005	62.3319782025776\\
5780.79999999999	62.3276652424363\\
5782.19999999993	62.3246804904623\\
5783.00000000007	62.3229755667376\\
5783.30000000008	62.3249299719888\\
5783.9	62.3219225449516\\
5784.29999999998	62.3245280409377\\
5786.19999999992	62.3299863470612\\
5787.30000000006	62.3319625393095\\
5787.80000000002	62.3283056030685\\
5788.20000000004	62.3309089024411\\
5788.40000000007	62.3287552906625\\
5788.80000000006	62.3261759574358\\
5789.09999999999	62.3281282387895\\
5789.50000000002	62.3255492607434\\
5790.30000000009	62.3238463663996\\
5790.90000000005	62.3260231393542\\
5791.39999999996	62.324095657429\\
5792.49999999993	62.3260711942824\\
5793.09999999992	62.3282469101705\\
5793.3	62.3260952117927\\
5794.10000000008	62.3243933588761\\
5794.40000000009	62.3263439468462\\
5795	62.3285189211575\\
5795.30000000009	62.3252924733409\\
5795.60000000008	62.3272426109012\\
5796.39999999995	62.3255412749073\\
5797.00000000006	62.3294405823601\\
5797.19999999999	62.3272902903076\\
5798.89999999999	62.3314364545611\\
5800.00000000007	62.3247875036637\\
5800.80000000002	62.3230877967212\\
5801.69999999992	62.3272087972698\\
5802.19999999996	62.3235613463627\\
5803.00000000001	62.3253088866296\\
5803.79999999997	62.3236099863885\\
5804.30000000003	62.3216869960719\\
5804.70000000004	62.3242833517089\\
5806.30000000003	62.3277762469\\
5806.69999999994	62.3252049321485\\
5807.7000000001	62.3230827507834\\
5808.59999999991	62.3185910788989\\
5808.89999999992	62.3205370976072\\
5809.59999999998	62.3181919892593\\
5810.19999999998	62.3203621155534\\
5810.49999999992	62.3171445289643\\
5811.39999999995	62.3212595715392\\
5811.79999999998	62.3186909616476\\
5812.70000000001	62.3159234792183\\
5813.70000000002	62.3189652206818\\
5814.10000000009	62.3163977847339\\
5815.89999999997	62.3143053645117\\
5816.30000000004	62.3168970497215\\
5817.99999999996	62.3210326395215\\
5818.60000000001	62.3180435492464\\
5818.90000000007	62.3199862519333\\
5819.20000000003	62.3167734950939\\
5820.19999999997	62.3146573200694\\
5822.80000000001	62.3177454532965\\
5823.30000000006	62.3158292406498\\
5823.69999999991	62.3184175280745\\
5823.90000000001	62.3162774725275\\
5824.69999999996	62.3180195028155\\
5826.79999999992	62.3213029226518\\
5827.30000000007	62.319387720081\\
5827.70000000002	62.3219739867532\\
5828.39999999991	62.3247833919533\\
5828.99999999993	62.3217992485976\\
5829.69999999998	62.3194620741706\\
5830.60000000004	62.3235632085341\\
5830.79999999997	62.3214255089266\\
5831.10000000008	62.3233639731102\\
5832.29999999997	62.3208284754132\\
5832.60000000005	62.3227664717884\\
5832.8	62.3206295324796\\
5834.19999999991	62.3228150763588\\
5834.50000000004	62.3196105988414\\
5834.79999999992	62.3215479271281\\
5835.70000000006	62.3187909112718\\
5836.79999999996	62.3224656924052\\
5837.09999999996	62.3192626601795\\
5838.10000000003	62.3120139769107\\
5838.49999999997	62.3145959647861\\
5839.49999999995	62.3073498184807\\
5840.00000000009	62.3054399753429\\
5841.40000000001	62.307626465805\\
5842.6	62.3102332825577\\
5842.79999999999	62.3081004295812\\
5844.2	62.3137073729959\\
5847.2000000001	62.3108101174901\\
5847.79999999992	62.3129670480001\\
5848.59999999995	62.3078632858584\\
5849.29999999999	62.30553561049\\
5849.80000000008	62.3036291218653\\
5850.29999999994	62.3068508136196\\
5851.09999999992	62.3102953240361\\
5851.99999999991	62.3143828711061\\
5852.20000000008	62.3122533021205\\
5852.89999999994	62.3167606355715\\
5853.20000000009	62.3135667059608\\
5853.49999999992	62.3154981549816\\
5853.70000000004	62.3133690935802\\
5854.69999999993	62.3163899706224\\
5855.20000000003	62.3144843133571\\
5856.30000000007	62.3181476675091\\
5856.50000000007	62.3160195335177\\
5857.19999999996	62.3136940228433\\
5858.0000000001	62.3154265034738\\
5858.50000000003	62.3135220018435\\
5859.90000000007	62.3156996587031\\
5860.39999999996	62.3137957512158\\
5860.80000000006	62.3163677933423\\
5861.40000000003	62.3134010065683\\
5862.69999999995	62.3166405130654\\
5862.90000000003	62.3145147535391\\
5863.70000000004	62.3179508168764\\
5863.89999999994	62.3158253751705\\
5864.69999999993	62.3175555858682\\
5865.79999999996	62.3195076629332\\
5866.1	62.3163206164127\\
5867.90000000003	62.3125426039536\\
5868.2	62.3144692670791\\
5869.00000000004	62.3110868787378\\
5869.4	62.3136553369112\\
5871.40000000009	62.3111640977604\\
5872.20000000008	62.3128927336819\\
5872.90000000007	62.3156819342755\\
5873.29999999992	62.3131405999932\\
5873.59999999991	62.3150654612936\\
5873.8	62.312943700097\\
5874.59999999995	62.3146713874751\\
5875.50000000001	62.3170399618762\\
5876.39999999998	62.3126010380329\\
5877	62.3164485885896\\
5877.49999999999	62.3145501565265\\
5879.5	62.3120620450371\\
5881.10000000002	62.3138135074475\\
5881.90000000002	62.315538932336\\
5882.39999999999	62.3102422439439\\
5883.30000000002	62.3143080531665\\
5884.3	62.3122153490585\\
5884.60000000001	62.3141366594729\\
5885.70000000009	62.3160827754936\\
5886.30000000009	62.3131285675455\\
5886.60000000009	62.315049178657\\
5888.09999999995	62.3127611154513\\
5888.59999999995	62.3108665749656\\
5889.20000000006	62.3130083371538\\
5890.30000000009	62.3149531440989\\
5891.10000000003	62.3098859315589\\
5891.99999999991	62.3156429795828\\
5893.49999999998	62.3201438848921\\
5893.89999999995	62.3176111299627\\
5894.90000000004	62.3223070398643\\
5895.70000000004	62.3240272736524\\
5897.79999999992	62.3272690279591\\
5898	62.3251555585697\\
5898.80000000006	62.3302649646544\\
5899.00000000002	62.3281517519622\\
5899.39999999994	62.3256208153233\\
5900.30000000004	62.3313673649244\\
5901.09999999999	62.3330847963126\\
5901.29999999998	62.3309723116549\\
5901.69999999994	62.3335253651428\\
5902.5	62.3369362653746\\
5902.70000000005	62.3348241512503\\
5903.00000000007	62.3367383239315\\
5903.19999999998	62.334626395406\\
5903.70000000005	62.3327348487415\\
5904.29999999996	62.3348689113204\\
5905.89999999997	62.3383000338639\\
5907.29999999993	62.3404543453973\\
5908.10000000004	62.3353982600454\\
5908.49999999999	62.3379480756863\\
5908.69999999997	62.3358380720282\\
5909.40000000006	62.338607327185\\
5909.99999999995	62.3356626791425\\
5910.79999999998	62.3306095518449\\
5911.20000000009	62.3331585268892\\
5911.79999999999	62.335289839138\\
5913.19999999996	62.3374427138823\\
5913.89999999992	62.3402096719648\\
5914.80000000007	62.3374866861654\\
5915.10000000005	62.3393968082229\\
5916.59999999995	62.335423462403\\
5917.30000000006	62.3331192753574\\
5917.70000000004	62.3356652810166\\
5918	62.3325053648975\\
5919.40000000006	62.3414139707746\\
5919.60000000005	62.3393077351893\\
5920.20000000001	62.3414354002331\\
5920.40000000002	62.3393294485263\\
5920.99999999992	62.341456823901\\
5921.39999999992	62.3389343916238\\
5922.19999999999	62.3406446819648\\
5922.70000000002	62.3387586952117\\
5923.90000000005	62.3345712356516\\
5924.7	62.3396570348366\\
5925.5	62.3363034966923\\
5925.79999999993	62.3382102296698\\
5926.70000000001	62.3405547681717\\
5927.30000000007	62.3376185173938\\
5927.99999999991	62.3353182301243\\
5929.20000000006	62.3328217496163\\
5929.49999999992	62.3347274689692\\
5930.50000000007	62.3326476241864\\
5930.79999999998	62.3345529346305\\
5931.49999999991	62.3373120237373\\
5931.7	62.3352102228666\\
5933.9000000001	62.3323222109875\\
5934.20000000005	62.3342264462531\\
5934.59999999998	62.3317101117159\\
5935.09999999994	62.3298288179\\
5935.50000000006	62.3323674102028\\
5936.20000000007	62.3267018176305\\
5936.60000000008	62.3292401502518\\
5937.1	62.3273596981742\\
5937.59999999996	62.3305320241844\\
5937.8	62.3284326108557\\
5938.59999999994	62.3250879822183\\
5939.09999999998	62.3215247844828\\
5939.79999999995	62.3242815535615\\
5941.20000000001	62.3314762762358\\
5941.59999999993	62.3289630913712\\
5942.50000000007	62.3262545013967\\
5943.1000000001	62.3283752860412\\
5943.70000000002	62.3304956425183\\
5944.09999999994	62.3279835806332\\
5944.39999999992	62.3298847674321\\
5944.70000000006	62.3267393352173\\
5945.99999999996	62.3316123173172\\
5947.5	62.334386979622\\
5948.40000000006	62.3384046398252\\
5950.00000000003	62.3401287373322\\
5950.80000000002	62.3418306474651\\
5951.49999999991	62.3445796088447\\
5951.70000000001	62.3424846264996\\
5952.10000000002	62.3450152884648\\
5952.59999999996	62.3397785878677\\
5953.20000000004	62.3435741521509\\
5954.10000000002	62.3408686305465\\
5954.69999999995	62.3446631289044\\
5956.30000000009	62.3463837217111\\
5958.00000000003	62.3420217854685\\
5958.80000000004	62.3453993186662\\
5959.00000000009	62.3433068751993\\
5959.90000000007	62.3489932885906\\
5960.30000000003	62.3448090732166\\
5960.9	62.3485992283174\\
5962.49999999999	62.3503169758159\\
5963.29999999991	62.3536908475031\\
5963.79999999999	62.3518167641979\\
5964.30000000002	62.3499429951043\\
5964.89999999997	62.347024308466\\
5965.89999999995	62.3449547435468\\
5966.39999999999	62.3430822090003\\
5966.89999999994	62.3412099882688\\
5967.3	62.3437342896404\\
5967.80000000006	62.3418622966203\\
5968.70000000005	62.3391636509851\\
5969.00000000004	62.3410564406694\\
5970.30000000001	62.3442315422752\\
5971.89999999999	62.3509711989283\\
5972.09999999992	62.348883158635\\
5972.39999999997	62.3507743825869\\
5972.60000000003	62.3486865236828\\
5972.90000000005	62.3505775991964\\
5973.29999999993	62.346402383902\\
5973.80000000001	62.3428580993991\\
5974.20000000007	62.3403578661935\\
5975.60000000001	62.3424870726442\\
5976.10000000006	62.340617783876\\
5976.59999999999	62.3437683002326\\
5977.60000000004	62.3467219833715\\
5978.00000000005	62.3425503086265\\
5978.30000000003	62.3444399839422\\
5979.49999999998	62.3419626730885\\
5981.30000000001	62.3399204199686\\
5981.7	62.3424387308168\\
5982.69999999996	62.3453901183392\\
5983.70000000004	62.3433269828537\\
5984.59999999994	62.3406352866476\\
5985.09999999998	62.3437813272739\\
5985.30000000002	62.3416981321215\\
5986.30000000008	62.3479887745557\\
5987.39999999997	62.344885177453\\
5987.80000000006	62.3474005911922\\
5988.70000000002	62.3447101255677\\
5989.79999999995	62.3499557588608\\
5990.69999999993	62.3539427121586\\
5992.09999999999	62.3510563732853\\
5993.10000000004	62.3489955282654\\
5993.59999999996	62.3454627358727\\
5993.89999999998	62.3473473473474\\
5994.10000000006	62.3452670915218\\
5994.90000000004	62.3502919099249\\
5995.79999999998	62.3542754215381\\
5995.99999999996	62.3521955938026\\
5997.4000000001	62.354314297624\\
5998.19999999993	62.357668005935\\
5998.90000000007	62.3603933988998\\
5999.40000000009	62.3585298774898\\
6000.40000000003	62.3614698775102\\
6003.10000000008	62.3634061833689\\
6003.60000000001	62.3615437147093\\
6004.00000000001	62.3640512316584\\
6004.69999999992	62.3667732480682\\
6005.50000000005	62.3701212201945\\
6005.69999999998	62.3680442239169\\
6006.99999999998	62.3661999966706\\
6007.6	62.3682940226709\\
6007.79999999996	62.3662178132126\\
6008.60000000004	62.3712283855077\\
6009.3	62.3689553033581\\
6010.40000000003	62.3658597454455\\
6011.80000000003	62.3696335601058\\
6012.20000000003	62.3671473479367\\
6012.69999999996	62.3702767429484\\
6013.4000000001	62.366342396275\\
6014.30000000004	62.3686485767491\\
6014.99999999995	62.364715466077\\
6015.49999999991	62.3678436066228\\
6016.39999999999	62.3701487575833\\
6016.69999999994	62.3670389575854\\
6016.99999999997	62.3689152581808\\
6017.59999999998	62.3660202402911\\
6018.20000000009	62.3697721948058\\
6018.49999999995	62.3666633436347\\
6019.20000000001	62.369378499161\\
6019.79999999999	62.3714679645841\\
6021.09999999997	62.3696273168139\\
6021.49999999995	62.3671449448651\\
6021.89999999999	62.3696446363334\\
6022.1	62.3675733120786\\
6022.40000000003	62.3694479036945\\
6023.09999999996	62.3655199893744\\
6024.40000000001	62.3636816333306\\
6024.69999999996	62.3655557030939\\
6024.89999999999	62.3634854771784\\
6025.29999999999	62.3659840010622\\
6026.00000000007	62.362058379383\\
6026.50000000008	62.3602030995918\\
6026.99999999993	62.3583481276236\\
6027.39999999994	62.3608461219411\\
6028.10000000002	62.3635579443283\\
6029.10000000003	62.3615073309892\\
6029.90000000004	62.3665008291874\\
6032.00000000002	62.3646822831187\\
6033.29999999994	62.3628468193722\\
6034.59999999999	62.3709546456328\\
6034.79999999994	62.368887636912\\
6035.10000000003	62.3707582184518\\
6035.3000000001	62.3686913874805\\
6036.19999999993	62.3643622749035\\
6036.79999999996	62.3681028342361\\
6037.89999999991	62.3699900629347\\
6038.10000000007	62.3679242158259\\
6039.10000000006	62.3658762750033\\
6039.6000000001	62.3640247032137\\
6040.50000000007	62.3663212263682\\
6041.60000000006	62.3698627869639\\
6041.90000000004	62.3667659715326\\
6042.19999999999	62.3686344603876\\
6042.39999999998	62.3665701282582\\
6042.90000000002	62.3696839318219\\
6043.40000000004	62.3678332092331\\
6044.29999999994	62.363510025809\\
6044.69999999996	62.3610375860244\\
6047.39999999993	62.3646134766432\\
6048.30000000004	62.3669069505985\\
6049.40000000006	62.360525663278\\
6049.69999999991	62.3623921451949\\
6049.9000000001	62.3603305785124\\
6050.39999999993	62.3568300140484\\
6050.90000000006	62.3549826474963\\
6051.29999999999	62.3525134679578\\
6051.7	62.355001817641\\
6052.49999999995	62.3583253477844\\
6052.69999999992	62.3562648691515\\
6054.49999999997	62.3542430548674\\
6055.19999999994	62.3519891665153\\
6055.60000000007	62.3544759482801\\
6056.19999999993	62.3516008123772\\
6056.70000000004	62.3547087570995\\
6056.90000000005	62.3526498266469\\
6058.19999999991	62.350824488718\\
6059.50000000001	62.348999933989\\
6060.10000000008	62.351077522194\\
6060.60000000006	62.3492335868794\\
6061.6000000001	62.3471963310622\\
6061.90000000004	62.3490597162653\\
6062.69999999999	62.3523784390051\\
6062.89999999997	62.3503216229589\\
6063.59999999999	62.3480713095965\\
6063.89999999992	62.3499340369393\\
6064.59999999992	62.3476841393639\\
6065.00000000002	62.3501673509093\\
6065.90000000001	62.3524563138807\\
6066.99999999994	62.3543373275535\\
6069.30000000001	62.3570698915873\\
6069.60000000008	62.3539878412442\\
6070.50000000009	62.356274503344\\
6070.89999999995	62.3521660352496\\
6071.60000000004	62.3482714890393\\
6072.39999999999	62.3515850144092\\
6072.59999999992	62.3495315098721\\
6073.10000000006	62.3526312322993\\
6073.7000000001	62.3481181467944\\
6074.4000000001	62.3458720882377\\
6074.8000000001	62.3434130602973\\
6075.19999999995	62.3409543561635\\
6076.39999999999	62.3467456595079\\
6076.70000000004	62.3436677198526\\
6076.99999999997	62.3455266492241\\
6077.9000000001	62.3428759460349\\
6078.3000000001	62.345354040537\\
6079.20000000002	62.349283634629\\
6079.40000000007	62.3472325026729\\
6079.89999999997	62.3453947368421\\
6080.19999999997	62.3472526026676\\
6080.40000000008	62.3452018748458\\
6080.69999999995	62.3470595974214\\
6083.39999999997	62.3571956932687\\
6083.59999999998	62.3551457172444\\
6084.49999999994	62.357426946718\\
6085.50000000002	62.3553963454713\\
6086.80000000003	62.3601504871117\\
6087.19999999993	62.357695530038\\
6087.60000000004	62.3552408955763\\
6088.69999999999	62.352187623177\\
6089.19999999994	62.3503522572381\\
6090.2	62.3532502503982\\
6090.49999999996	62.3501789643057\\
6091.90000000009	62.3555482600131\\
6092.10000000007	62.3535011982535\\
6092.59999999993	62.3565906740854\\
6095.4	62.3541957181527\\
6096.40000000007	62.3570901336833\\
6096.90000000002	62.353616532721\\
6097.29999999992	62.3560862006757\\
6097.90000000004	62.3532305673992\\
6099.80000000003	62.3583993180216\\
6100.39999999998	62.360462257192\\
6100.7	62.3573957513769\\
6101.09999999991	62.3598636333836\\
6103.7	62.3578754218684\\
6104.10000000009	62.3603420595656\\
6105.2	62.3654857255172\\
6105.39999999999	62.3634427974777\\
6105.69999999993	62.3652920174261\\
6105.89999999997	62.36324926302\\
6106.40000000008	62.361418160976\\
6106.9	62.3595873587686\\
6107.29999999995	62.3620525919377\\
6107.49999999995	62.3600104787478\\
6109.19999999999	62.3573895536314\\
6109.79999999999	62.3610861061556\\
6110.00000000006	62.3590448601496\\
6111.1	62.3609111140202\\
6112.19999999998	62.3644127415212\\
6112.39999999996	62.3623721881391\\
6113.80000000007	62.3660838417377\\
6114.60000000009	62.3611951526649\\
6115.19999999996	62.3648880676336\\
6115.39999999992	62.3628484997138\\
6117.19999999999	62.3657495954097\\
6117.40000000001	62.3637106661218\\
6117.79999999997	62.3661713986826\\
6119.19999999993	62.3633422123446\\
6119.50000000001	62.3651872671416\\
6119.99999999993	62.3633600758158\\
6121.49999999995	62.3611474124412\\
6122.09999999993	62.3648361700043\\
6122.3	62.3627989023912\\
6123.39999999996	62.3597615742631\\
6123.69999999999	62.3616055390444\\
6124.30000000004	62.3636601136438\\
6125.49999999994	62.3677680553742\\
6125.99999999993	62.3659424429898\\
6126.49999999993	62.3690138086377\\
6126.7	62.3669778677287\\
6127.59999999991	62.3627135793201\\
6128.89999999999	62.3609071626693\\
6129.50000000008	62.3645914904725\\
6129.89999999998	62.3621533442088\\
6130.40000000001	62.3652230650029\\
6131.19999999999	62.3685026014059\\
6131.90000000007	62.3662752772342\\
6132.80000000008	62.3685369074989\\
6133.59999999991	62.3718147284673\\
6133.8000000001	62.3697810528375\\
6134.49999999994	62.3740749193101\\
6134.90000000002	62.3700081499593\\
6135.20000000009	62.3718481573843\\
6135.60000000002	62.3694118030543\\
6135.89999999999	62.3712516297262\\
6136.80000000003	62.3735110560707\\
6137.00000000003	62.3714783855567\\
6137.69999999993	62.3741405715403\\
6138.49999999994	62.3774150457759\\
6139.80000000003	62.3756087232691\\
6142.09999999992	62.3799290156621\\
6142.89999999993	62.375061045092\\
6144.50000000001	62.3832308042834\\
6144.70000000006	62.3812003645359\\
6145.7	62.3856942952911\\
6146.4000000001	62.383470267632\\
6149.20000000006	62.3875888312491\\
6150.30000000003	62.3845603537981\\
6150.99999999995	62.3872152948253\\
6151.59999999996	62.3843815530666\\
6151.99999999998	62.3868272622357\\
6152.90000000004	62.3890784982935\\
6153.09999999997	62.3870506403172\\
6154.40000000007	62.3901210496385\\
6155.00000000003	62.3921625968709\\
6155.49999999996	62.3903437520307\\
6155.89999999991	62.3927875243665\\
6156.10000000007	62.3907605340957\\
6156.59999999991	62.3889421280881\\
6157.10000000001	62.3871240174105\\
6158.5	62.3908031046017\\
6158.90000000003	62.3883747361585\\
6160.5	62.3916501639451\\
6160.80000000007	62.3886120534338\\
6161.6000000001	62.3918723728841\\
6162.00000000001	62.3878223332955\\
6164.19999999993	62.3898901740668\\
6165.89999999997	62.387285111904\\
6166.30000000004	62.3897249610794\\
6166.69999999996	62.3856781474995\\
6167.09999999996	62.3881177844078\\
6167.30000000004	62.3860946265849\\
6167.89999999994	62.3897535667964\\
6169.99999999992	62.3928299379265\\
6170.20000000005	62.3908075782377\\
6170.80000000007	62.3944643406958\\
6171.89999999993	62.397926117952\\
6172.99999999993	62.3916670716496\\
6174.20000000005	62.3876390845926\\
6174.7	62.3858262615793\\
6175.59999999997	62.3832116197354\\
6177.00000000001	62.3852616923799\\
6177.89999999994	62.387504046617\\
6179.19999999993	62.3905620377713\\
6180.20000000002	62.3885571897804\\
6180.90000000001	62.3863452515774\\
6184.29999999997	62.3892374361296\\
6184.70000000001	62.3868192989264\\
6185.30000000003	62.3888511656481\\
6185.50000000005	62.3868339368857\\
6185.99999999999	62.3850244903897\\
6186.60000000001	62.388672474825\\
6186.79999999999	62.3866556757019\\
6187.20000000006	62.3842386824625\\
6188.20000000008	62.3870853061422\\
6189.20000000005	62.3899310099688\\
6190.29999999993	62.3917678986818\\
6190.50000000001	62.3897522049559\\
6191.20000000002	62.3875438114774\\
6191.79999999999	62.3847284355368\\
6192.50000000009	62.3889803959565\\
6194.70000000009	62.391037644476\\
6195.8000000001	62.3880307945577\\
6196.29999999995	62.3910657801304\\
6196.90000000007	62.393093432306\\
6197.09999999998	62.3910798425095\\
6197.59999999998	62.3941139454959\\
6198.10000000003	62.3890807008486\\
6198.40000000004	62.3909010244414\\
6200.5	62.3955746218108\\
6200.90000000004	62.3931623931624\\
6202.20000000008	62.3913709430373\\
6202.69999999994	62.3879538273038\\
6204.7	62.3936307374936\\
6204.89999999999	62.3916196615633\\
6205.20000000008	62.3934378676293\\
6206.19999999999	62.3898296891868\\
6206.60000000006	62.3874200460792\\
6207.30000000007	62.3852176434578\\
6207.69999999994	62.3828087245079\\
6208.59999999998	62.3850403466104\\
6208.80000000005	62.3830308106106\\
6209.4999999999	62.385660912136\\
6209.69999999998	62.3836516473961\\
6211.59999999997	62.3903279295523\\
6212.40000000001	62.3935613682093\\
6213.10000000004	62.3913603296208\\
6213.50000000009	62.3937813827733\\
6213.7999999999	62.3907690822189\\
6214.69999999999	62.396215485615\\
6215.79999999997	62.3980437265722\\
6216.19999999994	62.3956372761932\\
6217.10000000006	62.393038666924\\
6217.4	62.3948532368315\\
6217.59999999993	62.3928462293131\\
6218.89999999999	62.3894516803345\\
6219.80000000006	62.3852473512436\\
6220.09999999994	62.3870615092762\\
6222.39999999998	62.3848935315388\\
6222.69999999994	62.3867069486405\\
6223.2	62.3849083283788\\
6223.59999999993	62.3873258672494\\
6224.20000000003	62.3893449865848\\
6224.40000000001	62.3873403486224\\
6224.90000000002	62.3839357429719\\
6225.50000000001	62.385954767412\\
6226.10000000003	62.3879734027175\\
6227.20000000008	62.3897997527018\\
6227.80000000005	62.3853947558567\\
6228.20000000009	62.382993754315\\
6228.79999999995	62.3850117998362\\
6230.20000000001	62.3886490217164\\
6230.59999999991	62.3862487360971\\
6231.50000000002	62.3884716605687\\
6231.69999999995	62.3864693988896\\
6233.39999999991	62.3903104195075\\
6233.6999999999	62.3873079020822\\
6234.20000000004	62.3855124071668\\
6234.49999999991	62.3873223623007\\
6234.70000000008	62.3853211009174\\
6235.39999999991	62.3879400208484\\
6235.6000000001	62.3859390284972\\
6236.19999999994	62.3895579109408\\
6236.69999999994	62.3877629553617\\
6237.59999999993	62.3915866425125\\
6238.70000000004	62.3886003718664\\
6239.20000000001	62.391614443928\\
6239.60000000004	62.3876147891726\\
6240.2	62.3912311908081\\
6240.80000000002	62.3932445640853\\
6242.49999999997	62.398680037164\\
6243.59999999994	62.4004997037013\\
6245	62.3961185569487\\
6245.79999999991	62.3993339630798\\
6246.40000000006	62.3965420635556\\
6246.99999999995	62.3985529285589\\
6247.49999999993	62.3951597413407\\
6247.80000000004	62.3969653803678\\
6248.79999999991	62.394981516747\\
6249.19999999996	62.3925879698526\\
6251.8000000001	62.3954317887362\\
6252.9999999999	62.3978506660696\\
6253.50000000008	62.3960598695152\\
6254.40000000002	62.3982732432649\\
6254.79999999995	62.395881628803\\
6255.20000000005	62.3982862532572\\
6255.5999999999	62.3958949438113\\
6255.89999999993	62.3976982097187\\
6256.20000000001	62.3947061362147\\
6256.70000000003	62.3977112901164\\
6257.30000000003	62.3949244094992\\
6257.69999999993	62.3973281344882\\
6258.60000000001	62.4011376164379\\
6260.90000000005	62.3989777990736\\
6261.99999999991	62.3944044330177\\
6262.40000000006	62.3968063872256\\
6263.59999999993	62.3896419049444\\
6264.29999999992	62.3858629717132\\
6264.79999999996	62.3888649459688\\
6265.00000000009	62.3868733140732\\
6265.29999999996	62.3886743065088\\
6265.50000000002	62.3866828396323\\
6266.00000000002	62.384896506599\\
6266.50000000006	62.3831104586219\\
6266.89999999995	62.3855114089676\\
6268.00000000003	62.3873263030264\\
6268.80000000001	62.3825551532167\\
6270.2999999999	62.3851747894871\\
6270.50000000003	62.3831850221669\\
6271.00000000007	62.3798057757012\\
6271.39999999996	62.3822052140636\\
6272.90000000008	62.3784473138849\\
6274.00000000009	62.380261710843\\
6275.60000000006	62.3755118950874\\
6276.00000000005	62.3779098484728\\
6276.60000000006	62.3799130116144\\
6278.7000000001	62.3781614321208\\
6279.79999999999	62.3847513495438\\
6280.00000000007	62.3827646056591\\
6280.69999999995	62.3869570755318\\
6281.20000000007	62.3819909891265\\
6281.7000000001	62.3849851953262\\
6281.89999999992	62.3829990448902\\
6282.59999999991	62.387190220765\\
6283.20000000009	62.389190393583\\
6283.40000000009	62.3872045834328\\
6284.1	62.3913942904427\\
6285.20000000007	62.393203188392\\
6286.49999999999	62.3914357522349\\
6287.10000000001	62.3934342791704\\
6287.50000000005	62.3910554106495\\
6288.00000000007	62.3940458962167\\
6288.90000000008	62.3962474161234\\
6289.99999999993	62.3996438848349\\
6291.80000000007	62.397685913635\\
6292.30000000002	62.4006738287458\\
6292.79999999996	62.3988939916414\\
6293.10000000002	62.4006864552215\\
6293.59999999997	62.3989068433513\\
6294.60000000011	62.3969371058192\\
6295.30000000007	62.3995298154208\\
6296.00000000001	62.3973570940741\\
6296.50000000007	62.3955785662103\\
6297.00000000007	62.3985644185419\\
6297.29999999998	62.3955918315495\\
6297.80000000005	62.3938138109529\\
6298.79999999993	62.3918461953674\\
6299.10000000008	62.3936372872746\\
6300.09999999996	62.3916701057109\\
6301.10000000002	62.3944645464356\\
6301.60000000001	62.3926876874494\\
6302.19999999997	62.3899211399013\\
6303.99999999995	62.3927285417427\\
6304.19999999991	62.3907491712006\\
6304.90000000004	62.3949246629659\\
6305.99999999998	62.391969680151\\
6306.69999999999	62.396143844739\\
6307.2000000001	62.3943684302316\\
6307.90000000008	62.3985415345593\\
6308.09999999994	62.3965632034495\\
6309.09999999991	62.402523299309\\
6309.59999999995	62.3991631931788\\
6310.4	62.40234529752\\
6310.70000000008	62.399378842619\\
6311.10000000004	62.4017619470148\\
6311.69999999994	62.398998700846\\
6312.60000000004	62.3964389247073\\
6314.39999999999	62.4008235014649\\
6314.80000000001	62.3968708926507\\
6315.69999999993	62.3990626682289\\
6317.39999999997	62.396517609814\\
6317.69999999996	62.3983032068125\\
6318.19999999996	62.3965307123752\\
6319.90000000005	62.3924050632911\\
6320.60000000009	62.3965700001582\\
6321.19999999995	62.3922294464746\\
6321.50000000005	62.3940141736269\\
6324.79999999991	62.3962434188683\\
6325.00000000006	62.3942704463171\\
6325.60000000008	62.3915139826422\\
6327.29999999997	62.3952966463318\\
6329.19999999995	62.3907857109001\\
6329.60000000004	62.3931623931624\\
6329.8	62.3911910140761\\
6330.50000000009	62.3890310555082\\
6330.90000000006	62.3914073606066\\
6331.40000000005	62.389639106057\\
6333.39999999996	62.3873055972211\\
6334	62.3908684738163\\
6334.90000000002	62.3930544593528\\
6335.49999999998	62.3950375655029\\
6335.69999999998	62.3930679630039\\
6336.60000000005	62.3889406157779\\
6337.09999999995	62.385596162343\\
6337.4000000001	62.3873767258383\\
6338.19999999994	62.3905463610116\\
6338.79999999994	62.386218429065\\
6339.29999999991	62.38445278733\\
6339.79999999993	62.3826874240919\\
6340.3	62.385653901962\\
6340.79999999992	62.3838887224211\\
6342.60000000002	62.3898339823734\\
6343.40000000005	62.3866950421692\\
6344.19999999992	62.3898617656794\\
6345.19999999998	62.3926370699573\\
6345.50000000011	62.3896873424105\\
6347.09999999999	62.3849886564154\\
6347.40000000003	62.3867664434817\\
6347.89999999997	62.3834278512918\\
6348.29999999996	62.3857979963455\\
6348.59999999997	62.3828500322901\\
6351.19999999996	62.3856533308142\\
6351.6999999999	62.3807424667024\\
6352.30000000009	62.3842956992633\\
6352.89999999996	62.3862742011648\\
6354.19999999999	62.389248225611\\
6354.40000000002	62.3872846014635\\
6357.00000000006	62.390083528653\\
6357.20000000003	62.3881207430828\\
6358.19999999998	62.3861724045736\\
6358.90000000005	62.3887403679824\\
6359.10000000009	62.3867782110957\\
6359.40000000004	62.3885525591635\\
6360.09999999995	62.3911197761077\\
6360.79999999996	62.3936864280212\\
6361.20000000004	62.3897630987377\\
6362.19999999992	62.3925310029392\\
6364.3	62.3892275784049\\
6364.89999999998	62.3912018853103\\
6365.0999999999	62.3892415006598\\
6366.19999999999	62.3863154422506\\
6366.80000000005	62.3882894344186\\
6367.10000000008	62.3853499183314\\
6367.69999999999	62.3873237224787\\
6369.39999999997	62.3832325928252\\
6370.10000000009	62.3873661737465\\
6370.50000000002	62.3850186795592\\
6370.80000000004	62.3867899354879\\
6371.70000000006	62.3842556263536\\
6372.29999999998	62.3862281087189\\
6372.49999999994	62.384270156608\\
6372.79999999994	62.3860408919017\\
6373.70000000008	62.3882142520945\\
6373.89999999998	62.3862566677126\\
6375.10000000007	62.3886309449115\\
6375.8000000001	62.3864866136546\\
6376.39999999993	62.3884576178154\\
6377.40000000004	62.3833790670325\\
6378.60000000002	62.3794817125747\\
6379.10000000009	62.3824303987961\\
6379.99999999992	62.3799000015674\\
6381.0999999999	62.3832507992227\\
6381.49999999998	62.3793406042372\\
6381.90000000008	62.3816985271075\\
6382.40000000006	62.3799451625539\\
6383.00000000006	62.3772148329182\\
6383.5000000001	62.3801616642647\\
6384.40000000007	62.3776333307229\\
6384.70000000002	62.3794010775592\\
6385.40000000004	62.3772609819121\\
6385.79999999994	62.379617595014\\
6385.99999999999	62.3776639889761\\
6386.70000000011	62.3755245193211\\
6389.00000000008	62.3734172262134\\
6389.69999999995	62.3681492378478\\
6389.99999999994	62.3699159637564\\
6390.50000000001	62.3681657434357\\
6391.19999999992	62.3722873280866\\
6392.1000000001	62.3760207753199\\
6392.79999999992	62.3785762330085\\
6393	62.376624798611\\
6393.89999999993	62.3741007194245\\
6394.39999999996	62.3723512393463\\
6395.39999999993	62.3704166992417\\
6397.10000000006	62.3679109610455\\
6397.50000000009	62.3655745904714\\
6398.29999999992	62.3702800700175\\
6398.50000000009	62.3683305723127\\
6399.60000000009	62.3701110989578\\
6399.80000000001	62.3681620025313\\
6400.70000000002	62.3656417947757\\
6401.40000000003	62.3681949543076\\
6401.80000000002	62.3642980990019\\
6402.30000000008	62.3625515431713\\
6403.40000000003	62.3596470680097\\
6403.9	62.3625858838226\\
6404.10000000009	62.3606383310952\\
6404.39999999995	62.36240143649\\
6404.60000000009	62.3604540415632\\
6405.19999999995	62.3624186220786\\
6405.6	62.3600855488081\\
6407.20000000008	62.3648026469808\\
6407.8	62.3667660231901\\
6408.00000000009	62.3648195252883\\
6408.89999999991	62.36698392885\\
6409.10000000002	62.3650377582226\\
6409.5999999999	62.3679735401033\\
6409.80000000005	62.3660275511319\\
6410.19999999994	62.3636959268677\\
6411.39999999997	62.3660609841691\\
6413.19999999999	62.3688272808071\\
6413.70000000004	62.3639652000374\\
6414.10000000009	62.3663122447071\\
6414.29999999992	62.3643676727363\\
6415.49999999996	62.355820188291\\
6416.20000000001	62.3583685301498\\
6416.40000000007	62.3564248422037\\
6416.80000000008	62.3587713693528\\
6417.20000000002	62.3548844529631\\
6418.30000000009	62.3597781378537\\
6419.00000000007	62.3576513841504\\
6419.50000000009	62.3543522960932\\
6420.00000000003	62.3526113300416\\
6420.30000000011	62.3543704442091\\
6421.69999999991	62.3516770998785\\
6423.40000000006	62.3476297968397\\
6424.49999999999	62.3494069669707\\
6425.69999999992	62.3517694294874\\
6426.19999999999	62.346918133296\\
6427.10000000009	62.3490789146129\\
6428.20000000005	62.3508548138699\\
6428.79999999993	62.3481466502823\\
6430.2	62.3532339082158\\
6430.70000000005	62.3499409093736\\
6431.50000000006	62.3546240437838\\
6432.30000000004	62.34686897581\\
6432.80000000002	62.3497955820858\\
6433.8000000001	62.3478760938156\\
6434.09999999994	62.3496316558391\\
6435.90000000001	62.3570540708515\\
6437.4999999999	62.3601963464645\\
6437.70000000007	62.3582590325888\\
6438.80000000008	62.360030439982\\
6439.9999999999	62.3577273644819\\
6440.70000000003	62.3556079990063\\
6441.69999999993	62.3583470458567\\
6442.10000000002	62.3544751792866\\
6443.49999999996	62.3626544167856\\
6443.79999999995	62.359751082419\\
6444.30000000001	62.3564645273416\\
6444.60000000004	62.3582168293326\\
6445.19999999999	62.3539633531411\\
6445.69999999999	62.3522293586522\\
6447.49999999999	62.356535765246\\
6448.49999999994	62.3499674347921\\
6448.90000000002	62.3523026825865\\
6450.5	62.3476885871082\\
6451.60000000007	62.3494582823132\\
6451.79999999992	62.3475255351137\\
6452.60000000008	62.3506439165001\\
6453.80000000006	62.3545453136863\\
6455.29999999996	62.3601945657899\\
6456.69999999996	62.3637095774997\\
6457.20000000005	62.3619779164666\\
6458.99999999995	62.3600811258535\\
6460.19999999996	62.3624289893658\\
6461.39999999999	62.3647759808094\\
6461.7000000001	62.3618805905475\\
6463.40000000002	62.3655913978495\\
6464.30000000005	62.3692840789555\\
6464.49999999996	62.3673545153606\\
6465.20000000004	62.3652421388025\\
6465.59999999996	62.3675704100097\\
6466.20000000007	62.3695157972875\\
6466.4	62.3675867934741\\
6466.79999999992	62.3699144876216\\
6467.19999999994	62.3676031728851\\
6467.80000000011	62.3649097852471\\
6471	62.3680672528627\\
6471.20000000005	62.3661397246303\\
6472.60000000002	62.3634650145998\\
6472.89999999991	62.3652093310675\\
6475.90000000003	62.3672019765287\\
6476.29999999995	62.3648940769563\\
6476.80000000007	62.36316756473\\
6477.59999999994	62.3678157370672\\
6477.99999999998	62.3655084052423\\
6478.39999999991	62.3678320598904\\
6479.30000000004	62.3699725283205\\
6479.7	62.3661224111855\\
6480.40000000005	62.3701874855335\\
6480.60000000001	62.3682626876726\\
6481.60000000009	62.3740685314038\\
6482.09999999993	62.3708000370245\\
6484.90000000002	62.3685427910563\\
6485.30000000004	62.3708637863509\\
6485.4999999999	62.3689404218577\\
6486.1	62.372421448614\\
6486.29999999993	62.3704982733103\\
6486.59999999998	62.3722385804801\\
6486.80000000006	62.3703155590498\\
6487.60000000003	62.365707415571\\
6489.09999999993	62.3682426185046\\
6489.50000000001	62.3659393491124\\
6490.10000000009	62.3632553696342\\
6491.30000000001	62.3655913978495\\
6491.69999999991	62.3632890723682\\
6493.29999999993	62.3664028090061\\
6493.79999999999	62.3646807003496\\
6494.10000000008	62.3664192664224\\
6495.50000000009	62.3683724367264\\
6495.99999999997	62.3666507596866\\
6496.30000000001	62.3683886460193\\
6496.49999999999	62.3664686143521\\
6497.1999999999	62.3689840395241\\
6497.40000000009	62.3670642554829\\
6498.20000000009	62.3701583491067\\
6498.39999999991	62.3682388243441\\
6498.70000000008	62.3699759955684\\
6500.20000000009	62.3755826654155\\
6500.3999999999	62.3736635643412\\
6501.59999999994	62.3775320300844\\
6501.80000000008	62.3756132822713\\
6502.09999999992	62.3773492048845\\
6504.10000000008	62.3750807170751\\
6504.69999999996	62.3785512237117\\
6505.30000000008	62.3804839056784\\
6505.50000000005	62.3785661583866\\
6506.20000000007	62.3764658869096\\
6507.09999999994	62.3739857388739\\
6507.80000000001	62.3780328523794\\
6507.99999999995	62.3761159170879\\
6509.89999999999	62.3824884792627\\
6510.60000000009	62.3803892054618\\
6511.90000000004	62.3786855036855\\
6513.1999999999	62.3769824819984\\
6514.00000000008	62.3739273268755\\
6514.50000000007	62.3768151536549\\
6515.10000000009	62.3710707269155\\
6515.79999999994	62.3735784772633\\
6516.79999999991	62.376283202136\\
6518.70000000011	62.3841811376327\\
6518.89999999994	62.3822672188986\\
6519.60000000011	62.3863061183797\\
6519.79999999992	62.3843923986564\\
6520.2000000001	62.3866999984663\\
6520.39999999993	62.3847864427575\\
6520.79999999996	62.3870938060697\\
6522.10000000002	62.3853914323388\\
6523.19999999995	62.3871353456073\\
6525.09999999997	62.3888922944891\\
6525.70000000007	62.390817984002\\
6527.29999999996	62.3847167325428\\
6528.90000000001	62.3770868433144\\
6529.3999999999	62.3799678382725\\
6529.59999999994	62.378057184863\\
6530.59999999993	62.3761618203255\\
6531.20000000002	62.3780870577067\\
6532.09999999992	62.3756161783166\\
6532.39999999997	62.3773440489858\\
6533.10000000009	62.3798444866222\\
6533.50000000008	62.3760254683482\\
6534.09999999996	62.3779498637936\\
6534.59999999994	62.3747073316296\\
6535.29999999997	62.3772072099642\\
6536.60000000008	62.3755105787324\\
6537.20000000006	62.3789637923914\\
6537.40000000008	62.3770554493308\\
6538.60000000003	62.3747839784667\\
6539.09999999995	62.3776608759481\\
6539.30000000005	62.3757531271982\\
6541.00000000007	62.3794163061259\\
6541.60000000009	62.381338184264\\
6542.00000000006	62.3790525977897\\
6547.19999999994	62.3814396774243\\
6547.79999999991	62.3833595503902\\
6549.00000000006	62.3810905315234\\
6549.60000000002	62.3830099088508\\
6550.00000000007	62.3807270118014\\
6550.49999999997	62.3790187158428\\
6550.99999999996	62.3773106806491\\
6552.10000000002	62.3790482586002\\
6553.80000000009	62.3750743831917\\
6554.2	62.3773705811452\\
6554.80000000007	62.3747120474759\\
6555.79999999996	62.3774005094648\\
6556.09999999993	62.3745462310485\\
6556.49999999999	62.3722661135345\\
6556.79999999998	62.3739877076057\\
6559.60000000009	62.37175480586\\
6560.09999999994	62.3746227249169\\
6560.80000000003	62.3710161715618\\
6561.30000000005	62.3693114274393\\
6562.19999999999	62.3744723648721\\
6562.39999999997	62.3725714285714\\
6563.20000000009	62.3756342084013\\
6564.70000000004	62.378137947843\\
6564.90000000006	62.3762376237624\\
6565.49999999994	62.3796758864384\\
6566.19999999993	62.3775946880283\\
6567.19999999996	62.3802780442495\\
6567.40000000001	62.3783783783784\\
6567.69999999998	62.3800968360791\\
6569.70000000001	62.3778501628665\\
6570.6000000001	62.3799595172508\\
6571.7999999999	62.3822638810694\\
6572.3999999999	62.3796120197794\\
6572.69999999993	62.3813291139241\\
6573.20000000002	62.3796266715349\\
6574.09999999998	62.3771713668583\\
6575.50000000005	62.3745361640002\\
6575.7999999999	62.3762526802415\\
6576.50000000008	62.3741751056777\\
6578.89999999998	62.3711810305518\\
6580.79999999992	62.3729277150542\\
6581.9999999999	62.3752297898847\\
6582.19999999996	62.373334548714\\
6582.50000000001	62.3750493725884\\
6583.60000000002	62.3782979175843\\
6584.10000000007	62.3750797363385\\
6585.5999999999	62.3730203319313\\
6586.49999999997	62.3766434882944\\
6586.99999999992	62.3734268494482\\
6587.69999999998	62.3759069795683\\
6588.29999999999	62.3778155546111\\
6589.39999999993	62.3795432126869\\
6591.90000000003	62.3771237864078\\
6594.09999999996	62.374510933851\\
6594.70000000006	62.3764177837084\\
6595.20000000003	62.3747213925068\\
6595.60000000002	62.3770031990539\\
6597.09999999995	62.3810101255078\\
6597.49999999991	62.378743785619\\
6597.90000000002	62.3810245528948\\
6598.80000000007	62.3725166315598\\
6600.79999999993	62.368767895287\\
6601.60000000011	62.3718133208113\\
6603.10000000011	62.374303368064\\
6604.99999999991	62.3790707180815\\
6605.39999999991	62.3768072061161\\
6606.20000000006	62.3737947111091\\
6606.89999999997	62.3717269562585\\
6607.3999999999	62.3745743473326\\
6608.80000000003	62.37649230583\\
6609.80000000001	62.3791585349249\\
6610.50000000009	62.3770913381539\\
6612.39999999992	62.3818525519849\\
6613.29999999993	62.3778994163365\\
6613.59999999993	62.3796059694271\\
6613.90000000007	62.3767765346235\\
6614.39999999994	62.3796205306524\\
6615.6000000001	62.3819096996539\\
6615.80000000009	62.3800238818604\\
6616.70000000007	62.3821182444686\\
6617.59999999996	62.3796787403479\\
6618.29999999993	62.3776139248157\\
6619.00000000011	62.3755495460108\\
6620.00000000003	62.3736801558889\\
6620.89999999992	62.3787947439964\\
6621.09999999994	62.3769105298133\\
6621.7999999999	62.3748470982649\\
6622.90000000001	62.3765665106447\\
6623.6	62.3790328668267\\
6624.3	62.3754604190568\\
6625.69999999993	62.3728455431797\\
6626.09999999996	62.3751169599469\\
6627.49999999991	62.3770293922385\\
6627.89999999994	62.3747736873868\\
6630.10000000001	62.3767005520196\\
6631	62.3787908491804\\
6631.20000000002	62.3769095049236\\
6631.49999999992	62.3786114964714\\
6633.20000000002	62.3807154809823\\
6635.49999999999	62.3786846705648\\
6637.39999999993	62.3758945386064\\
6639.50000000008	62.3787577564914\\
6641.00000000009	62.3767147008779\\
6641.89999999995	62.3742848539597\\
6642.59999999995	62.37824980806\\
6643.80000000007	62.3805295082707\\
6644	62.3786517361268\\
6644.60000000008	62.3760290156064\\
6645.70000000001	62.3807517529869\\
6647.09999999992	62.3871705379709\\
6650.00000000005	62.3900392475301\\
6650.6999999999	62.3924941360438\\
6651.0999999999	62.3887418811643\\
6651.99999999997	62.3863140963004\\
6653.19999999994	62.3840800805615\\
6653.79999999999	62.3859691308856\\
6654.1	62.38315650266\\
6654.50000000003	62.3854176058666\\
6654.8	62.3826052983516\\
6656.60000000008	62.3807592350564\\
6657.29999999999	62.3787064018986\\
6658.19999999994	62.3807878888005\\
6658.49999999997	62.3779773525966\\
6659.49999999998	62.376118685807\\
6659.89999999993	62.3783783783784\\
6660.10000000007	62.3765052100538\\
6660.49999999995	62.378764675855\\
6661.60000000003	62.3759700977228\\
6663.40000000006	62.3786298491784\\
6664.10000000003	62.38108099997\\
6664.30000000005	62.3792089310366\\
6665.30000000003	62.3773516968224\\
6666.40000000001	62.3790594764869\\
6667.19999999992	62.3760742729441\\
6667.79999999995	62.379459799937\\
6668.09999999997	62.376653369725\\
6669.3	62.3744264851411\\
6669.7	62.3766829590093\\
6670.00000000008	62.3738774531116\\
6670.90000000008	62.3714585519412\\
6671.9	62.3696043165468\\
6672.90000000009	62.367750636895\\
6674.00000000005	62.3739530423578\\
6674.20000000002	62.3720839638614\\
6675.10000000006	62.3696668264621\\
6676.19999999999	62.3713733654869\\
6677.09999999998	62.3674594141257\\
6677.49999999995	62.3697136695819\\
6677.79999999996	62.366911753695\\
6678.79999999996	62.3695518723143\\
6679.30000000011	62.3663802137917\\
6679.99999999992	62.3643358632356\\
6681.50000000003	62.3623084291188\\
6682.89999999999	62.3657040251384\\
6683.09999999998	62.3638376825473\\
6683.5	62.3660901310671\\
6683.69999999995	62.3642239444627\\
6684.90000000008	62.3620044876589\\
6685.49999999993	62.3638865621635\\
6685.89999999992	62.3601555489082\\
6686.5000000001	62.3620375078515\\
6686.70000000009	62.3601722797153\\
6687.90000000011	62.3624401913876\\
6688.6	62.3648840581877\\
6688.79999999993	62.3630193305327\\
6689.20000000007	62.3652699086601\\
6689.39999999999	62.3634053367217\\
6689.99999999993	62.3608017817372\\
6692.10000000008	62.3696243387825\\
6694.10000000006	62.3629410534492\\
6694.59999999997	62.3657520127862\\
6696.0000000001	62.3706336524245\\
6696.59999999994	62.3650454701569\\
6698.20000000003	62.3695564546228\\
6698.40000000005	62.3676942599089\\
6698.90000000009	62.3705030601582\\
6700.30000000007	62.37239567787\\
6701.20000000008	62.3699879127931\\
6702.10000000011	62.367580794366\\
6702.69999999998	62.3694575401325\\
6703.09999999999	62.3672275927915\\
6703.99999999992	62.36929640071\\
6705.20000000008	62.3715568281807\\
6705.39999999994	62.3696965177839\\
6706.90000000011	62.3721485015655\\
6708.40000000007	62.3745993888351\\
6709.00000000008	62.3720022059591\\
6710.39999999999	62.3694210565532\\
6711.09999999997	62.3718560019073\\
6711.80000000007	62.368330875013\\
6712.69999999998	62.365927779764\\
6715.30000000006	62.3641182952616\\
6715.69999999999	62.3618928496977\\
6716.40000000004	62.359860046155\\
6718.29999999993	62.3615741843296\\
6719.79999999995	62.3640232741559\\
6720.6999999999	62.3675752886561\\
6721.8000000001	62.3707582677517\\
6722.20000000008	62.3685345789388\\
6722.7999999999	62.36594326853\\
6723.89999999994	62.3706127305175\\
6724.10000000004	62.3687576217245\\
6725.29999999999	62.3650637880275\\
6725.8999999999	62.3609872137972\\
6726.50000000008	62.3643445425624\\
6727.00000000007	62.3611957604317\\
6727.3999999999	62.3589743589744\\
6728.49999999999	62.3606693814464\\
6729.70000000008	62.3629231180719\\
6729.99999999999	62.360143237099\\
6730.60000000009	62.3620128664181\\
6731.19999999992	62.3594253710279\\
6732.30000000003	62.3566632998634\\
6732.7	62.3588997148289\\
6734.99999999995	62.3569063562531\\
6735.99999999999	62.3624946185478\\
6736.89999999995	62.3601009351343\\
6737.50000000006	62.3634528615531\\
6738.99999999998	62.3614429226455\\
6739.6000000001	62.3588587029096\\
6740.2999999999	62.3612841967836\\
6740.60000000007	62.3585087602178\\
6741.19999999995	62.3603755952116\\
6742.40000000004	62.3566926214312\\
6743.60000000004	62.3589424203331\\
6743.79999999994	62.3570930767064\\
6745.90000000003	62.3628817076786\\
6746.10000000003	62.3610328777682\\
6746.99999999994	62.3586429725364\\
6747.39999999998	62.3608743979252\\
6747.79999999994	62.3571777886454\\
6748.70000000004	62.3547889995259\\
6749.70000000003	62.3529586061809\\
6750.29999999996	62.3548234178715\\
6750.50000000005	62.3529760317602\\
6751.5	62.3555897861248\\
6751.79999999999	62.3528192064456\\
6754.59999999997	62.3551008927117\\
6755.50000000008	62.3571555450293\\
6756.7000000001	62.3594008998342\\
6757.40000000009	62.3559008509064\\
6759.29999999994	62.3576057046483\\
6759.8	62.3544726993003\\
6761.8000000001	62.3567340540381\\
6763.79999999993	62.3604725084641\\
};
\addplot [color=mycolor3, forget plot]
  table[row sep=crcr]{%
6763.79999999993	62.3604725084641\\
6764.90000000002	62.3577235772358\\
6765.79999999995	62.3597747528045\\
6767.20000000003	62.3616508799669\\
6767.40000000009	62.3598079054304\\
6768.10000000002	62.3622233385538\\
6769.10000000001	62.3648289310406\\
6771.90000000006	62.3685764914353\\
6773.10000000001	62.3722907931259\\
6774.00000000006	62.3743375503757\\
6774.79999999992	62.3714003158718\\
6775.30000000003	62.3741771703516\\
6775.50000000003	62.3723360292815\\
6776.50000000011	62.3705102853939\\
6776.90000000006	62.3727312970341\\
6777.09999999993	62.3708906332999\\
6777.49999999995	62.3731114258735\\
6778.80000000006	62.3759016949653\\
6780.80000000007	62.3781503930157\\
6780.99999999996	62.3763106280692\\
6781.70000000002	62.3787195139933\\
6783.20000000003	62.3767192959179\\
6783.90000000006	62.3806014150943\\
6785.4	62.3844963525164\\
6785.59999999999	62.3826576477003\\
6786.5	62.3861727521881\\
6787.10000000003	62.3836044318718\\
6787.60000000004	62.3863753554223\\
6787.79999999992	62.3845371911784\\
6788.4000000001	62.3863887456728\\
6788.59999999991	62.384550797649\\
6788.99999999997	62.3823481757523\\
6790.29999999998	62.3792412818096\\
6791.39999999996	62.3823897518958\\
6792.59999999997	62.3846187819277\\
6793.00000000002	62.3824174530038\\
6793.39999999997	62.3802163833076\\
6793.90000000003	62.382984986753\\
6794.09999999991	62.3811486267699\\
6794.99999999994	62.384659534076\\
6795.40000000011	62.3824589802075\\
6795.8	62.3802586853838\\
6796.70000000004	62.3822975517891\\
6797.39999999993	62.3802868701729\\
6797.80000000004	62.3825004780888\\
6799	62.3803150416967\\
6800.60000000004	62.3832840736983\\
6801.20000000011	62.3851322541279\\
6802.30000000009	62.382394449018\\
6803.00000000011	62.3803854125325\\
6803.5	62.3831500970075\\
6804.89999999995	62.3850110213079\\
6805.7	62.3820858679362\\
6807.09999999994	62.3839464096839\\
6807.59999999993	62.3808334679848\\
6808.50000000001	62.3755250712334\\
6808.89999999998	62.3733294169482\\
6809.70000000001	62.3762812417399\\
6809.99999999992	62.3735334282903\\
6811.39999999997	62.3753945533289\\
6811.6000000001	62.3735631340194\\
6812.09999999994	62.3763248289833\\
6812.80000000001	62.3713836985718\\
6813.1999999999	62.3691896731393\\
6813.89999999999	62.3715879072498\\
6815.40000000001	62.3740004401731\\
6815.6999999999	62.3712550250888\\
6816.8999999999	62.3690773067332\\
6817.5999999999	62.3729410211655\\
6818.30000000011	62.3694708435997\\
6821.79999999996	62.3667893108958\\
6822.19999999996	62.3689957932076\\
6823.29999999998	62.3735967406278\\
6824.69999999997	62.3710584925566\\
6825.20000000008	62.3679545221456\\
6826.59999999994	62.3698126473992\\
6826.80000000001	62.3679854692467\\
6829.89999999996	62.3718887262079\\
6831.10000000007	62.3741070382949\\
6832.10000000008	62.3766868651386\\
6833.59999999994	62.3805551897215\\
6834.30000000009	62.3785555425494\\
6836.09999999997	62.3811474210819\\
6837.59999999996	62.3777001038361\\
6838.69999999996	62.3808270456805\\
6838.89999999991	62.3790027781839\\
6839.90000000003	62.3771929824562\\
6841.30000000007	62.3746601572778\\
6842.20000000007	62.3766862020081\\
6842.39999999992	62.3748629886737\\
6845.59999999992	62.3719999415692\\
6847.2999999999	62.3755001898531\\
6848.60000000002	62.3782615678888\\
6849.2999999999	62.376266534295\\
6850.19999999999	62.3797497919799\\
6850.80000000006	62.3815849012539\\
6851.09999999997	62.378853339561\\
6851.70000000004	62.3806882862897\\
6852.8	62.3838083147281\\
6853.70000000005	62.385829758674\\
6854.40000000004	62.3838354365745\\
6857.9000000001	62.3855351414407\\
6858.09999999995	62.3837158438074\\
6859.60000000008	62.3788212312492\\
6861.40000000009	62.3828608904759\\
6862.99999999992	62.3858023342221\\
6864.19999999992	62.380723453229\\
6865.40000000009	62.3771029058335\\
6866.00000000003	62.3803906147595\\
6866.99999999998	62.3844126341542\\
6867.20000000008	62.3825957800009\\
6868.20000000001	62.3807929181894\\
6869.40000000006	62.3757187568236\\
6871.00000000007	62.3713815837348\\
6871.79999999994	62.3743069602293\\
6872.90000000002	62.377418885494\\
6873.09999999993	62.375603794448\\
6874.59999999992	62.3838131118449\\
6874.90000000006	62.3810909090909\\
6875.6999999999	62.3840134966113\\
6876.10000000001	62.3818388063174\\
6877.09999999999	62.3844006281626\\
6877.49999999996	62.380772362452\\
6878.39999999994	62.3827869448281\\
6878.89999999998	62.3797063526675\\
6879.5	62.3815338101052\\
6879.69999999994	62.3797203407076\\
6880.5999999999	62.3817344165565\\
6881.00000000011	62.3795614073331\\
6881.80000000005	62.3824815821212\\
6882.70000000004	62.3859475794735\\
6883.4	62.3839616474177\\
6884.20000000002	62.3810699708031\\
6884.89999999994	62.3790849673203\\
6886.20000000001	62.3760219566385\\
6886.59999999998	62.3738510462195\\
6887.79999999998	62.376050755673\\
6889.29999999995	62.3784364385868\\
6889.99999999993	62.3764531719423\\
6890.90000000005	62.3813670004353\\
6891.80000000003	62.3848285668684\\
6892.09999999997	62.3821131133745\\
6894.40000000007	62.3801580970339\\
6895.09999999992	62.37817612252\\
6895.4999999999	62.3803584894715\\
6896.09999999999	62.382181491256\\
6896.49999999998	62.3800133399066\\
6899.8999999999	62.3782608695652\\
6903.19999999999	62.3759651181319\\
6903.5999999999	62.3781450526529\\
6903.99999999993	62.3759794904477\\
6904.3999999999	62.3738141791585\\
6904.99999999992	62.3770836048718\\
6905.40000000005	62.3749185431902\\
6906.40000000004	62.373126764642\\
6907.09999999994	62.3711489460273\\
6908.80000000007	62.3746182460305\\
6908.9999999999	62.3728126673518\\
6909.59999999995	62.374632762638\\
6910.5	62.3766387867913\\
6911.30000000002	62.3737592962352\\
6911.69999999993	62.375936803727\\
6913.19999999989	62.3783142638103\\
6914.40000000011	62.3819509725938\\
6917.00000000007	62.3801882291712\\
6918.19999999997	62.3838226153824\\
6920.20000000005	62.390358799474\\
6921.40000000003	62.3867658744492\\
6922.10000000008	62.3891248447026\\
6922.89999999993	62.3920265780731\\
6923.90000000007	62.3945696129405\\
6924.10000000005	62.3927673955114\\
6924.79999999999	62.3965689035221\\
6927.20000000006	62.4009354293881\\
6927.50000000003	62.3982331543392\\
6928.30000000004	62.4011315743895\\
6928.59999999993	62.3984297198609\\
6929.20000000006	62.4002424487322\\
6929.40000000009	62.3984414459918\\
6931.4999999999	62.4011772173813\\
6932.40000000008	62.398846015146\\
6933.0000000001	62.4006577144423\\
6933.19999999998	62.3988576868158\\
6933.80000000001	62.402111365898\\
6934.50000000007	62.3972543477634\\
6936.00000000007	62.3952941854933\\
6936.39999999998	62.3974626973258\\
6937.60000000009	62.3996425328279\\
6938.10000000011	62.3965870110403\\
6938.6000000001	62.3992966982288\\
6941.69999999996	62.3973609150365\\
6942.20000000002	62.3943073621134\\
6943.40000000008	62.3921653344855\\
6944.99999999991	62.3878705850168\\
6947.50000000002	62.3913293799298\\
6950.40000000008	62.3940723688943\\
6950.69999999999	62.3913794095644\\
6952.10000000006	62.3874457006415\\
6954.2999999999	62.3892787300127\\
6954.90000000006	62.3910855499641\\
6955.79999999992	62.3945140096896\\
6955.9999999999	62.3927200586536\\
6956.40000000008	62.3948824840078\\
6957.2000000001	62.3977692495652\\
6957.39999999999	62.395975565936\\
6957.7999999999	62.3981373690338\\
6958.39999999989	62.3999425163469\\
6959.00000000004	62.4017473523875\\
6959.2000000001	62.3999540183639\\
6960.90000000002	62.4019537422784\\
6961.3000000001	62.3998046369983\\
6962.19999999991	62.4017925110955\\
6962.59999999996	62.3982075918825\\
6963.40000000003	62.3939111079199\\
6965.70000000001	62.3905366217807\\
6968.00000000001	62.3972101433676\\
6968.2	62.3954192557726\\
6969.30000000002	62.4013544924958\\
6969.49999999995	62.3995638200184\\
6970.59999999999	62.402628143515\\
6970.80000000008	62.4008377684374\\
6972.80000000002	62.4030173959185\\
6973.40000000001	62.4048182404818\\
6975.0000000001	62.4019727315737\\
6975.49999999993	62.3989334250817\\
6977.30000000009	62.3957348009287\\
6977.90000000011	62.3975351103468\\
6978.30000000002	62.3939585005159\\
6978.90000000002	62.3971915747242\\
6979.10000000007	62.3954034846401\\
6980.00000000005	62.3988195011533\\
6981.30000000005	62.4029564270777\\
6982.29999999993	62.4011801099908\\
6983.49999999994	62.4047769058938\\
6983.69999999996	62.4029897763395\\
6984.10000000001	62.4008476275021\\
6985.30000000008	62.3987173247058\\
6985.89999999998	62.400515316347\\
6986.60000000003	62.3985572587917\\
6988.00000000002	62.3960733246519\\
6989.20000000003	62.3982373055957\\
6990.70000000006	62.4005836241918\\
6990.90000000004	62.3987984551566\\
6992.29999999999	62.4048967450375\\
6992.6999999999	62.4027571216108\\
6993.59999999999	62.4004461157899\\
6994.30000000012	62.398490220748\\
6995.8999999999	62.401372212693\\
6996.5999999999	62.399416867952\\
6997.1	62.4021036986223\\
6997.59999999998	62.3976449404804\\
6998.49999999996	62.4010516388992\\
7000.30000000011	62.3964344894577\\
7000.80000000005	62.3991201131283\\
7003.60000000003	62.3955909019518\\
7004.49999999993	62.3918567798304\\
7005.90000000009	62.387953182986\\
7006.29999999991	62.3901004795615\\
7006.69999999993	62.3879659759091\\
7007.19999999994	62.390649750974\\
7007.39999999994	62.3888690688548\\
7008.80000000007	62.3863944413531\\
7009.80000000007	62.3889071170773\\
7010.5	62.3855304824123\\
7011.10000000009	62.3873231401187\\
7011.29999999992	62.3855435433722\\
7013.89999999991	62.3809523809524\\
7014.4999999999	62.3784677672284\\
7015.99999999998	62.3765339718647\\
7016.60000000001	62.3740504795702\\
7018.39999999999	62.3765761914939\\
7019.99999999995	62.3737553596103\\
7022.70000000008	62.3853733553568\\
7023.1000000001	62.3818202528762\\
7024.79999999998	62.3852296829848\\
7025.80000000002	62.3820435816052\\
7026.20000000002	62.3799154604842\\
7027.80000000008	62.3756740989485\\
7028.39999999991	62.3774631856015\\
7028.59999999998	62.3756882496052\\
7029.3999999999	62.379970125898\\
7031.20000000011	62.3782230881914\\
7033.1	62.384121025991\\
7033.80000000008	62.3821777392343\\
7034.99999999995	62.3857514463192\\
7036.80000000004	62.3840043200841\\
7037.99999999993	62.3875761924383\\
7038.50000000001	62.3845651123803\\
7040.00000000007	62.3826366102754\\
7040.69999999994	62.3863765481195\\
7040.99999999994	62.3837184530826\\
7041.39999999995	62.3858552865157\\
7043.20000000003	62.3883690883535\\
7043.40000000011	62.3865975722297\\
7044.00000000005	62.3898013940745\\
7044.30000000009	62.3871443983874\\
7045.30000000008	62.391063672751\\
7045.50000000001	62.3892926081526\\
7045.90000000004	62.3871700255464\\
7046.79999999999	62.3848784571939\\
7048.49999999992	62.3911131288483\\
7049.30000000002	62.3939626067467\\
7050.90000000012	62.3911501914622\\
7052.59999999992	62.3888723467608\\
7054.39999999995	62.3871287830463\\
7058.49999999995	62.3834754767234\\
7059.10000000006	62.3866727107888\\
7061.30000000005	62.388478205455\\
7061.49999999992	62.3867112269174\\
7062.09999999996	62.3899068279007\\
7062.80000000005	62.3922184938198\\
7066.80000000012	62.3951095954379\\
7068.2999999999	62.3931865768774\\
7068.80000000002	62.3958465956514\\
7069.19999999997	62.3923160708981\\
7070.69999999995	62.3903942976749\\
7071.19999999999	62.3930536110758\\
7074.2000000001	62.3906252208699\\
7074.6999999999	62.393283202352\\
7075.39999999993	62.3913504345983\\
7076.39999999997	62.3938387620999\\
7076.60000000007	62.3920754023768\\
7077.19999999991	62.3952637305187\\
7078.29999999995	62.3926311030741\\
7079.00000000007	62.3892867737424\\
7079.90000000001	62.3870056497175\\
7080.40000000001	62.3896617470518\\
7080.60000000009	62.3878995014617\\
7081.00000000005	62.3900241487904\\
7082.10000000011	62.3873937477055\\
7083.10000000004	62.3856449062571\\
7083.99999999995	62.3876004008978\\
7086.8	62.3841171739407\\
7090.19999999999	62.3824097710957\\
7090.80000000001	62.3841825438238\\
7092.39999999991	62.3870285512866\\
7092.60000000008	62.3852693614561\\
7093.20000000009	62.3870412924873\\
7095.0000000001	62.3824893236177\\
7096.80000000009	62.386394059378\\
7098.00000000001	62.3899353348079\\
7099.2999999999	62.3869622785024\\
7099.70000000011	62.3890813825742\\
7100.6000000001	62.3868069345276\\
7101.90000000011	62.3894677555618\\
7102.10000000001	62.3877108501591\\
7102.60000000009	62.3903585960269\\
7104.00000000001	62.3865091989133\\
7104.60000000004	62.3896857010148\\
7105.40000000002	62.386883400183\\
7105.99999999992	62.3886520032085\\
7106.19999999992	62.3868961344159\\
7107.30000000009	62.3828685595295\\
7109.8000000001	62.3904696268583\\
7110.00000000007	62.3887146453636\\
7110.80000000002	62.3929460405856\\
7112.89999999992	62.3970195416842\\
7113.50000000002	62.3987854251012\\
7114.40000000004	62.39651416122\\
7114.7999999999	62.3986282308957\\
7115.39999999991	62.4003935071323\\
7115.60000000005	62.3986396278651\\
7116.19999999998	62.4018099293172\\
7117.19999999996	62.400067441305\\
7117.59999999989	62.4021804796493\\
7118.00000000004	62.3986738034026\\
7119.10000000003	62.4016743454321\\
7119.29999999995	62.3999213416861\\
7120.2000000001	62.4018650899541\\
7121.59999999994	62.4050437395566\\
7122.30000000009	62.3989104796136\\
7122.69999999999	62.4010220699725\\
7122.8999999999	62.399269970518\\
7123.59999999995	62.401560986566\\
7124.2000000001	62.4033238353242\\
7124.39999999993	62.4015720401432\\
7125.29999999992	62.4035141886771\\
7126.70000000008	62.4010776224954\\
7127.30000000004	62.4028397452086\\
7128.79999999989	62.4009314199947\\
7129.79999999991	62.4062048556081\\
7130.20000000008	62.4041064190848\\
7130.90000000009	62.3993829757397\\
7132.60000000006	62.3971287170356\\
7133.00000000003	62.3992373582313\\
7133.20000000002	62.3974878387282\\
7134.20000000005	62.3957501086302\\
7134.80000000012	62.3975108270613\\
7134.99999999999	62.3957617973119\\
7136.39999999994	62.3989350521965\\
7137.00000000001	62.3964915722072\\
7138.19999999992	62.4000112071502\\
7139.19999999996	62.3982743406216\\
7140.69999999996	62.400571364553\\
7141.99999999998	62.3976141470996\\
7142.50000000011	62.4002464088707\\
7142.89999999996	62.3967520649587\\
7143.2999999999	62.3988576868158\\
7144.20000000002	62.4007950394021\\
7144.89999999995	62.4030790762771\\
7145.19999999992	62.4004590430073\\
7146.49999999996	62.404500041978\\
7147.19999999991	62.4025855917619\\
7148.39999999989	62.4005036021543\\
7148.79999999994	62.4026073941446\\
7148.99999999993	62.4008616469206\\
7150.80000000011	62.3991385699702\\
7151.40000000001	62.4008949171503\\
7151.59999999997	62.3991498524826\\
7152.00000000011	62.4012527789041\\
7152.19999999989	62.3995078506215\\
7153.39999999998	62.4030195009436\\
7153.89999999998	62.4000559127761\\
7155.99999999999	62.4027053842177\\
7156.40000000001	62.400614825683\\
7157.30000000002	62.4025484114343\\
7158.90000000008	62.4067607207711\\
7159.29999999994	62.4046707824678\\
7159.89999999996	62.4064245810056\\
7160.50000000005	62.4081780856353\\
7160.89999999999	62.4060885351208\\
7161.79999999991	62.4080202180986\\
7162.69999999993	62.4043670073156\\
7163.89999999995	62.4022892238973\\
7164.49999999998	62.3998548418614\\
7166.29999999999	62.4037173476222\\
7166.69999999989	62.4016297371212\\
7167.70000000002	62.4054800636178\\
7168.29999999992	62.4072317393003\\
7168.89999999998	62.404798437718\\
7169.50000000001	62.4079446552109\\
7172.30000000008	62.4100719424461\\
7172.5000000001	62.408331706773\\
7172.9999999999	62.4109520291088\\
7174.10000000001	62.4083521507625\\
7175.00000000003	62.4047051609037\\
7175.99999999995	62.4001895179833\\
7176.90000000003	62.4035112163857\\
7177.10000000005	62.4017722788831\\
7179.4999999999	62.4045907850019\\
7180.60000000005	62.4019942345454\\
7181.59999999993	62.4058370580782\\
7182.90000000012	62.4084644299039\\
7184.10000000002	62.4105676345313\\
7184.70000000002	62.4067475782207\\
7185.10000000003	62.4088403941435\\
7185.50000000002	62.4053662881318\\
7186.49999999992	62.4036401079787\\
7187.10000000011	62.4053873552983\\
7188.40000000008	62.4080127982194\\
7189.60000000004	62.4045509548382\\
7190.09999999992	62.4071653083363\\
7190.50000000004	62.405084415765\\
7191.1999999999	62.4087438988778\\
7191.99999999994	62.4115348785473\\
7194.29999999998	62.4138218614478\\
7194.49999999995	62.41208684291\\
7195.19999999996	62.4157436104124\\
7197.70000000012	62.413515240768\\
7198.40000000007	62.4157810654998\\
7198.60000000007	62.4140469807049\\
7199.3	62.4177014751229\\
7199.90000000002	62.4138888888889\\
7200.30000000011	62.4159768901728\\
7202.09999999992	62.4198161672822\\
7202.29999999992	62.4180828612685\\
7204.49999999995	62.4212308802709\\
7204.69999999994	62.4194981123695\\
7205.50000000002	62.4153436216276\\
7206.10000000002	62.4184729815992\\
7206.79999999995	62.4151854472797\\
7207.9999999999	62.4172805593707\\
7208.39999999998	62.4138170215718\\
7209.2	62.4166007795486\\
7210.2	62.4148787151714\\
7212.49999999999	62.4129994731442\\
7213.2	62.4111017148878\\
7213.79999999993	62.4128418747141\\
7214.20000000004	62.4107675034307\\
7216.60000000003	62.4135685285518\\
7217.00000000011	62.4101093236896\\
7217.80000000004	62.4142756203328\\
7219.5999999999	62.4181060154854\\
7220.19999999992	62.4143041147875\\
7220.99999999999	62.4184681004279\\
7222.40000000007	62.4202146071305\\
7223.29999999996	62.4235124733505\\
7223.49999999991	62.4217841519464\\
7224.50000000008	62.4269855770562\\
7225.10000000008	62.4287216962852\\
7225.39999999996	62.4261296796069\\
7227.99999999991	62.4244268895007\\
7228.40000000012	62.4265061907726\\
7229.60000000005	62.4299763475663\\
7231.00000000012	62.427569802658\\
7232.39999999992	62.4293121327342\\
7233.10000000005	62.427418016922\\
7233.49999999992	62.429495686795\\
7233.69999999997	62.4277696369819\\
7234.3999999999	62.4258760107817\\
7234.90000000008	62.4284727021424\\
7235.39999999996	62.4255407366457\\
7236.59999999996	62.4234803156135\\
7237.19999999989	62.4252138228345\\
7237.39999999998	62.4234887737478\\
7238.00000000008	62.4252220886697\\
7238.40000000003	62.4231539683636\\
7238.90000000007	62.4257494129023\\
7240.19999999996	62.4228277833791\\
7241.10000000007	62.4205932718334\\
7241.79999999998	62.4242256866292\\
7242.40000000007	62.4259578874698\\
7242.60000000005	62.4242340563602\\
7243.79999999997	62.4207954278773\\
7244.7999999999	62.4190810087096\\
7245.30000000003	62.4216744417148\\
7245.60000000005	62.4190899430007\\
7246.2000000001	62.4208216607096\\
7247.39999999998	62.4256640220766\\
7248.20000000011	62.4284314942814\\
7248.49999999991	62.4258477499103\\
7249.59999999994	62.4232726871457\\
7250.00000000008	62.4253458572985\\
7250.19999999996	62.4236238500476\\
7255.9000000001	62.4214443219405\\
7256.29999999994	62.4193815114933\\
7257.20000000008	62.4157744615766\\
7257.90000000003	62.41388812345\\
7258.29999999995	62.4118262978067\\
7259.30000000002	62.4101165385569\\
7260.50000000011	62.4066881524943\\
7261.50000000002	62.4091109397378\\
7261.70000000005	62.4073921066402\\
7262.40000000002	62.4096385542169\\
7262.60000000008	62.4079199195891\\
7263.20000000009	62.409648506877\\
7263.49999999998	62.4070708739468\\
7264.10000000009	62.4087993171994\\
7264.39999999993	62.4062220386813\\
7266.29999999989	62.4091709787515\\
7266.79999999989	62.4062530102244\\
7267.70000000001	62.4040287294642\\
7268.30000000006	62.4016289692367\\
7268.90000000002	62.404732425368\\
7269.09999999998	62.4030154624993\\
7269.80000000005	62.401133440625\\
7270.50000000008	62.4047533903667\\
7271.1000000001	62.4064803608758\\
7273.09999999997	62.409943353682\\
7273.39999999992	62.4073692170207\\
7274.7	62.4099631604993\\
7277.6999999999	62.4144659100278\\
7279.09999999989	62.4120782503572\\
7280.00000000007	62.4153514374803\\
7280.29999999991	62.4127795176089\\
7280.69999999992	62.4148445225799\\
7281.59999999992	62.4167433429007\\
7281.89999999995	62.4141719307882\\
7282.40000000004	62.4112598695503\\
7282.99999999995	62.4129834823084\\
7284.5000000001	62.4111138566291\\
7285.60000000005	62.4154165008167\\
7286.49999999995	62.4131968270524\\
7287.40000000003	62.4150943396227\\
7287.79999999993	62.4116686562659\\
7288.49999999996	62.4097906319458\\
7290.50000000002	62.4077579348751\\
7292.80000000003	62.4045304337095\\
7294.59999999993	62.4110655681522\\
7295.69999999989	62.4139916116122\\
7295.90000000002	62.4122807017544\\
7296.69999999989	62.4164017103388\\
7297.90000000009	62.4198410523431\\
7298.30000000001	62.4177902005919\\
7298.80000000001	62.4203647124909\\
7301.10000000007	62.4226154604723\\
7301.30000000002	62.4209055797518\\
7301.79999999991	62.423478820581\\
7302.90000000006	62.4277694098316\\
7303.29999999996	62.4257195278911\\
7304.19999999996	62.4303492463344\\
7304.3999999999	62.4286398795263\\
7305.4	62.425569776196\\
7306.40000000001	62.4279750906727\\
7306.99999999989	62.4296916697459\\
7308.49999999996	62.4319295077033\\
7309.00000000012	62.4276586720663\\
7309.99999999995	62.4300625162446\\
7310.40000000003	62.4280144996922\\
7312.40000000012	62.4300854700855\\
7312.60000000006	62.4283780272676\\
7313.40000000003	62.4311205305257\\
7313.99999999999	62.4328352087065\\
7314.89999999995	62.4306220095694\\
7315.49999999989	62.4323363770573\\
7322.09999999989	62.4375187785092\\
7323.50000000011	62.4392375334535\\
7323.70000000008	62.4375324285207\\
7324.7000000001	62.4412953254696\\
7326.69999999998	62.4392640716275\\
7327.7000000001	62.4348371953383\\
7328.09999999992	62.4368876395295\\
7328.89999999991	62.4328012007095\\
7330.29999999996	62.4290625341046\\
7330.6999999999	62.4270202433568\\
7331.90000000012	62.4249863611566\\
7332.90000000011	62.4205645711169\\
7333.39999999995	62.417672325629\\
7334.6000000001	62.4210942506169\\
7334.79999999999	62.4193922207528\\
7335.19999999998	62.4214415225008\\
7335.39999999998	62.4197396223843\\
7340.30000000003	62.4216663942019\\
7340.59999999998	62.4191153432234\\
7341.1	62.4216749305291\\
7341.90000000001	62.4244075183874\\
7342.1000000001	62.4227070905178\\
7342.49999999992	62.4247541742707\\
7342.70000000001	62.4230538759057\\
7343.79999999992	62.4205122618772\\
7345.79999999994	62.4184919478893\\
7348.49999999994	62.421413602591\\
7348.90000000001	62.4193767859573\\
7349.39999999997	62.4219334648616\\
7350.80000000005	62.4236488049082\\
7352.09999999999	62.4262125622263\\
7353.49999999999	62.4224869451697\\
7354.29999999995	62.4197759164582\\
7355.10000000003	62.4225038068305\\
7355.30000000001	62.4208064823123\\
7355.99999999994	62.4189448213048\\
7359.3000000001	62.4208495257766\\
7360.00000000001	62.4189888724338\\
7360.59999999999	62.4206936840246\\
7361.29999999991	62.4188333740864\\
7362.00000000001	62.4224066502764\\
7362.30000000002	62.4198630881234\\
7362.90000000006	62.4215672959392\\
7363.49999999998	62.41783910044\\
7364.89999999992	62.4154786150713\\
7365.80000000006	62.418713259751\\
7366.09999999992	62.416171160164\\
7366.60000000004	62.4187220872304\\
7367.39999999998	62.4160162877503\\
7370.39999999991	62.4136761413744\\
7371.80000000009	62.4167446655543\\
7374.80000000012	62.4184734708267\\
7375.20000000001	62.4164440768511\\
7375.69999999993	62.4189918381735\\
7376.39999999998	62.4171354978648\\
7377.60000000008	62.4205375659081\\
7377.90000000008	62.4179994578477\\
7379.49999999999	62.4139519757168\\
7380.09999999996	62.415652692339\\
7383.09999999999	62.4173799978329\\
7385.20000000001	62.4199423178476\\
7385.90000000011	62.4221500135391\\
7386.29999999997	62.4201234701614\\
7387.8000000001	62.4182785365259\\
7389.99999999997	62.4213474783832\\
7390.7000000001	62.419494506684\\
7391.40000000011	62.4217006020429\\
7391.70000000009	62.4191671852593\\
7393.89999999991	62.4235866919124\\
7395.29999999998	62.4252913973551\\
7397.10000000009	62.4222678851457\\
7397.69999999992	62.4253156343778\\
7398.50000000006	62.4226204957695\\
7399.10000000003	62.4202616499081\\
7400	62.4221294306834\\
7400.70000000012	62.4243325045941\\
7401.09999999992	62.4223098956926\\
7402.19999999999	62.4197884441322\\
7403.30000000007	62.4172677418483\\
7403.90000000004	62.4203133441383\\
7404.19999999995	62.4177842604973\\
7405.09999999995	62.4155998487549\\
7405.49999999999	62.417629901696\\
7406.3999999999	62.4194963883076\\
7406.80000000001	62.4174755970784\\
7408.50000000001	62.419350484572\\
7412.49999999993	62.4247902220543\\
7413.29999999996	62.4221005206788\\
7414.39999999992	62.4195832490391\\
7415.70000000011	62.4167318428221\\
7418.59999999997	62.4206397347244\\
7419.29999999992	62.4187939725584\\
7419.80000000002	62.4213264329708\\
7420.60000000005	62.4186397509669\\
7422.10000000012	62.4208455713939\\
7423.69999999995	62.4235566690913\\
7424.20000000012	62.4206995945746\\
7425.30000000003	62.4262665984324\\
7426.2000000001	62.429473627513\\
7427.20000000005	62.4331856798567\\
7427.50000000001	62.430664009909\\
7429.0000000001	62.4328653538114\\
7429.3999999999	62.4308499899051\\
7431.00000000011	62.4281734871015\\
7431.80000000007	62.4308723206717\\
7432.19999999992	62.4288578232849\\
7433.80000000002	62.4315635130954\\
7435.50000000007	62.4347732529991\\
7435.99999999995	62.4319199580425\\
7436.40000000004	62.4299065420561\\
7437.00000000007	62.4329375697516\\
7437.50000000003	62.4300849736474\\
7438.89999999994	62.4331227315499\\
7440.90000000009	62.4351565649778\\
7442.6000000001	62.432988028538\\
7442.99999999991	62.43500691916\\
7443.69999999993	62.4291356565195\\
7444.50000000007	62.4318297826613\\
7446.39999999993	62.4360437789566\\
7450.00000000011	62.4380880793546\\
7451.50000000012	62.4416232755381\\
7451.89999999992	62.4396135265701\\
7452.49999999991	62.4426374688028\\
7454.80000000012	62.4408107419281\\
7455.40000000006	62.4384682449199\\
7455.99999999999	62.4414908598329\\
7456.89999999995	62.4433418264718\\
7457.19999999997	62.4408297909431\\
7457.5999999999	62.442844308567\\
7457.99999999998	62.4394953138199\\
7460.69999999993	62.4370040746301\\
7461.59999999989	62.4348338850396\\
7462.29999999998	62.4329974271012\\
7463.20000000007	62.4348478555063\\
7463.49999999989	62.4323382817943\\
7464.20000000008	62.4345216564179\\
7465.79999999999	62.4318568424437\\
7466.50000000002	62.4340395896392\\
7467.90000000003	62.4317086234601\\
7468.50000000011	62.4293709664462\\
7469.40000000005	62.4312202958699\\
7470.60000000002	62.4292234998059\\
7471.49999999989	62.4337491300391\\
7474.30000000009	62.4317670983624\\
7474.99999999994	62.4285962729596\\
7475.70000000001	62.4307766392894\\
7476.30000000012	62.4271039537745\\
7477.30000000007	62.4294540883195\\
7479.60000000009	62.4263005200743\\
7480.50000000009	62.4281474748015\\
7480.89999999998	62.4248095174442\\
7481.79999999997	62.4266563306112\\
7485.30000000004	62.4295294840623\\
7486.1000000001	62.4268654323956\\
7486.49999999998	62.4288729196164\\
7487.09999999996	62.4265412971471\\
7488.6999999999	62.4238863369298\\
7489.0999999999	62.4258932863323\\
7489.70000000005	62.4235627119549\\
7491.30000000002	62.4262487652508\\
7491.60000000002	62.4237489488367\\
7491.99999999995	62.4257551287356\\
7492.9000000001	62.4235953556653\\
7494.09999999988	62.4216060420058\\
7495.20000000012	62.4271209958241\\
7497.20000000012	62.4291411574834\\
7497.6	62.4271443242594\\
7497.99999999989	62.425147704085\\
7498.90000000008	62.4269902653687\\
7500.59999999991	62.4248403482342\\
7501.49999999993	62.4226831609257\\
7502.39999999993	62.4245251582806\\
7503.09999999995	62.4226996481501\\
7503.5000000001	62.4247028093182\\
7505	62.422885771009\\
7507.79999999999	62.4249124255784\\
7508.90000000005	62.4277533626315\\
7509.5	62.4254287844892\\
7511.90000000011	62.4227902023429\\
7513.29999999989	62.4257992386935\\
7514.3999999999	62.4299687271276\\
7514.79999999992	62.4279764201786\\
7516.20000000004	62.4256615622048\\
7516.89999999995	62.4291605693761\\
7517.20000000003	62.4266691498277\\
7517.90000000004	62.4288374567704\\
7519	62.426354217925\\
7520.39999999995	62.4333488464863\\
7521.29999999999	62.4351849389741\\
7521.99999999997	62.4333630236237\\
7522.60000000006	62.4363592858947\\
7524.79999999997	62.4393679650228\\
7525.29999999996	62.4352193903314\\
7526.19999999996	62.4370540637498\\
7528.09999999998	62.433250976329\\
7529	62.4350852027467\\
7530.20000000007	62.4331035948103\\
7531.10000000009	62.4349373273847\\
7531.79999999997	62.4331178056002\\
7534.39999999991	62.4381179905767\\
7534.89999999993	62.435301924353\\
7535.60000000008	62.4374643364253\\
7535.89999999997	62.4349787685775\\
7539.00000000008	62.4331816795108\\
7540.39999999998	62.4295471122605\\
7540.80000000002	62.4315400018565\\
7541.60000000001	62.428895341899\\
7542.20000000002	62.4318841732628\\
7543.10000000006	62.4270866475766\\
7543.79999999999	62.4252707485518\\
7544.50000000011	62.423455186491\\
7545.19999999996	62.4256159463507\\
7550.79999999993	62.4283198029374\\
7551.19999999993	62.4263371869744\\
7551.70000000002	62.4288249159141\\
7552.39999999994	62.4270109235353\\
7553.79999999996	62.4300030447848\\
7557.10000000004	62.4278833430371\\
7558.20000000004	62.4307053173333\\
7558.79999999994	62.4283956660361\\
7559.29999999994	62.4308807577321\\
7559.79999999993	62.4280744454292\\
7561.90000000012	62.4318963237239\\
7562.59999999998	62.4340513308739\\
7564.79999999994	62.4383666671073\\
7565.59999999996	62.4357296747162\\
7566.3000000001	62.433918375978\\
7567.60000000001	62.4390501737648\\
7568.29999999999	62.4412028962528\\
7569.09999999994	62.4385668234424\\
7570.29999999994	62.4445207650851\\
7571.20000000001	62.4410603198922\\
7572.20000000008	62.4446997609709\\
7574.70000000006	62.4465332418018\\
7576.7	62.4485270826734\\
7577.79999999998	62.4460602541601\\
7579	62.4440896676386\\
7579.90000000006	62.4459102902375\\
7581.80000000006	62.4434508500508\\
7582.60000000012	62.4460943990927\\
7583.4999999999	62.4426393797141\\
7584.90000000004	62.4390243902439\\
7586.09999999993	62.4449658590599\\
7586.39999999994	62.4424965399064\\
7587.39999999995	62.4448105436573\\
7590.49999999998	62.4430216320186\\
7590.9999999999	62.4454953827508\\
7595.80000000011	62.4481628246817\\
7597.20000000003	62.4445526700275\\
7597.69999999998	62.4470241385664\\
7598.60000000008	62.4488399331465\\
7599.19999999995	62.4465411287882\\
7599.80000000001	62.4495059145515\\
7600.29999999997	62.4467133308773\\
7600.70000000002	62.4486896116198\\
7601.89999999989	62.4440936595633\\
7604.09999999997	62.4418084742642\\
7605.70000000011	62.4470798601068\\
7606.29999999989	62.4434686579722\\
7608.19999999991	62.446275777769\\
7610.00000000009	62.4499021037831\\
7613.9999999999	62.4525551279862\\
7614.69999999992	62.4507537952408\\
7615.70000000008	62.4556842354053\\
7616.89999999999	62.4589733490876\\
7617.60000000002	62.4611103088859\\
7618.49999999988	62.4629196965322\\
7618.79999999996	62.4604601714158\\
7620.19999999999	62.4581709381521\\
7622.3999999999	62.4545752705805\\
7623.29999999992	62.4563842904741\\
7624.2999999999	62.4534389591312\\
7624.90000000011	62.4511475409836\\
7625.59999999993	62.4532829773004\\
7626.49999999989	62.455091390659\\
7627.20000000004	62.4532927772606\\
7628.80000000012	62.4585457929715\\
7630.39999999991	62.4559334250704\\
7631.60000000013	62.4539748679849\\
7632.30000000002	62.4521775588282\\
7632.8000000001	62.4546371628084\\
7633.60000000003	62.4494019937907\\
7634.90000000005	62.45317616241\\
7635.7000000001	62.4505618271825\\
7636.49999999998	62.4479480397035\\
7638.80000000004	62.446163714671\\
7641.19999999989	62.4487979794014\\
7642.00000000003	62.446186257704\\
7642.39999999989	62.444226365718\\
7643.09999999992	62.4463575465773\\
7644.10000000009	62.4421129745428\\
7644.60000000003	62.4445694402658\\
7645.09999999992	62.4417935436614\\
7646.29999999993	62.4437643858548\\
7646.99999999993	62.4419714663075\\
7647.70000000002	62.4401788749706\\
7648.4	62.4383866117539\\
7649.40000000007	62.4432969475129\\
7650.29999999997	62.4451009097564\\
7651.49999999997	62.4470698938784\\
7651.80000000005	62.4446215972504\\
7653.59999999997	62.4469210969858\\
7655.30000000006	62.4500352692217\\
7657.39999999992	62.4524975514202\\
7658.59999999995	62.4505464373849\\
7659.39999999997	62.4531627390822\\
7659.70000000001	62.4507167288963\\
7660.39999999998	62.454147901573\\
7662.1000000001	62.4559525984704\\
7663.90000000001	62.4582463465553\\
7665.90000000012	62.4602139316462\\
7666.70000000001	62.4576094328794\\
7668.19999999992	62.4597368386735\\
7669.40000000004	62.4616989373492\\
7670.10000000004	62.4599097807098\\
7670.80000000009	62.4581209506055\\
7673.30000000002	62.4547136862408\\
7674.10000000005	62.4573245419718\\
7674.40000000007	62.4548830542706\\
7676.09999999989	62.452776113181\\
7676.60000000006	62.4552216447171\\
7678.89999999995	62.4573512176065\\
7679.3999999999	62.4545868871671\\
7680.1000000001	62.4567068565923\\
7681.09999999996	62.4589907826902\\
7681.89999999988	62.4537880760219\\
7682.40000000002	62.4510250569476\\
7685.50000000013	62.4492557510149\\
7687.39999999988	62.4520325203252\\
7688.10000000007	62.450248432663\\
7689.60000000012	62.4471695904912\\
7690.40000000004	62.4497756972889\\
7691.1	62.4518930726025\\
7691.39999999994	62.4494571930053\\
7691.79999999999	62.451409924726\\
7692.20000000009	62.4494624494624\\
7693.39999999994	62.4475206343017\\
7693.89999999995	62.4499610085781\\
7695.19999999996	62.4446090470807\\
7696.40000000002	62.4465666211915\\
7697.90000000001	62.4486879709015\\
7698.3	62.4467421801933\\
7699.89999999998	62.4493506493507\\
7700.19999999994	62.446917652559\\
7702.09999999994	62.4444963776583\\
7702.50000000001	62.4464466543764\\
7703.00000000008	62.4436915008243\\
7705.8	62.4469562283445\\
7706.19999999996	62.4450125222221\\
7707.8999999999	62.441619097042\\
7708.59999999994	62.4398406994694\\
7709.49999999998	62.4429282971879\\
7710.29999999989	62.4403403195684\\
7711.00000000011	62.4424530871082\\
7711.90000000011	62.4455394190871\\
7713.50000000009	62.4481435386849\\
7715.40000000011	62.4457261357009\\
7716.09999999991	62.449132992924\\
7716.40000000008	62.4467051124214\\
7717.19999999992	62.4428232672049\\
7718.10000000011	62.4407245212614\\
7719.59999999992	62.4428410430457\\
7720.80000000003	62.4409071481304\\
7721.6000000001	62.4435033736094\\
7722.00000000002	62.4415638233123\\
7723.40000000012	62.4393086036124\\
7724.29999999997	62.4410957485371\\
7724.70000000008	62.4391570008285\\
7725.30000000002	62.4420741968054\\
7726.49999999995	62.4453187689281\\
7727.60000000002	62.4429002161057\\
7728.19999999996	62.4393462986685\\
7729.39999999989	62.4412963322337\\
7733.49999999999	62.4431054101583\\
7734.40000000003	62.4410110543668\\
7735.20000000005	62.4436027044846\\
7735.59999999989	62.441666558941\\
7737.49999999988	62.4379652605459\\
7738.0000000001	62.4352231167858\\
7740.99999999996	62.4329875599075\\
7742.20000000009	62.4310605375664\\
7745.50000000012	62.4328651105144\\
7746.50000000008	62.4351328324684\\
7746.99999999998	62.4323940571311\\
7747.80000000006	62.4298196930781\\
7748.50000000002	62.4332137418372\\
7750.49999999992	62.4351663097051\\
7750.90000000005	62.433234421365\\
7751.40000000007	62.4356576146552\\
7751.90000000011	62.4329205366357\\
7752.29999999999	62.4348588824106\\
7753.79999999995	62.4330981828499\\
7755.49999999996	62.4310175872918\\
7756.20000000008	62.433118884004\\
7757.89999999999	62.4349059035834\\
7758.99999999993	62.4376538516065\\
7761.19999999992	62.4405705229794\\
7761.69999999999	62.4378365842975\\
7762.50000000001	62.4352665344086\\
7763.4000000001	62.437045147163\\
7765.10000000001	62.4349662597229\\
7766.30000000008	62.4330449114133\\
7766.69999999998	62.431117062368\\
7767.10000000008	62.4330518076012\\
7769.90000000006	62.4311454311454\\
7771.19999999988	62.4335696730277\\
7772.3	62.4311666923987\\
7773.29999999994	62.4359996912548\\
7773.70000000011	62.4340734261236\\
7775.49999999992	62.4299089459334\\
7778.20000000008	62.4339508632992\\
7780.50000000001	62.4322031719919\\
7782.39999999995	62.4349502088018\\
7782.79999999995	62.4330262498555\\
7783.19999999989	62.4311024886616\\
7784.09999999996	62.4341615066417\\
7784.70000000009	62.4319186106258\\
7785.4000000001	62.4301586282191\\
7785.89999999989	62.4325712817878\\
7786.30000000001	62.4306483098736\\
7786.69999999995	62.4325782092772\\
7787.49999999997	62.428732857363\\
7788.59999999991	62.4314712339672\\
7789.90000000013	62.4287548138639\\
7790.29999999992	62.4306839186691\\
7790.70000000009	62.4287621296914\\
7792.00000000009	62.4324636490805\\
7792.50000000004	62.4297410363678\\
7792.9000000001	62.4316694469396\\
7793.20000000008	62.4292661645259\\
7795.00000000002	62.4340932124027\\
7799.00000000001	62.4405380107961\\
7799.90000000002	62.4423076923077\\
7800.20000000012	62.4399061574555\\
7800.60000000009	62.4418321432692\\
7800.89999999995	62.4394308421997\\
7802.09999999992	62.4413626925739\\
7803.99999999997	62.4440998962084\\
7804.7000000001	62.4410619106191\\
7806.39999999994	62.4428360981234\\
7807.59999999994	62.4409237035235\\
7808.50000000007	62.4375688343621\\
7809.00000000011	62.4399738766311\\
7809.80000000008	62.4374191731008\\
7811.19999999989	62.4351900451909\\
7811.70000000007	62.437594408459\\
7812.99999999997	62.4425644110532\\
7815.60000000011	62.4397046969561\\
7817.10000000001	62.4417950161183\\
7819.20000000009	62.4378141265842\\
7820.80000000008	62.4403840990167\\
7821.1999999999	62.4384693081713\\
7821.59999999999	62.4365547131698\\
7822.00000000005	62.4384756011813\\
7822.49999999999	62.4357630455347\\
7823.29999999998	62.4383260474985\\
7824.19999999995	62.4362562785169\\
7825.70000000006	62.4345114876434\\
7826.40000000005	62.4365936242254\\
7826.90000000006	62.4338827136834\\
7827.30000000012	62.431969747298\\
7827.70000000006	62.4338894708603\\
7828.60000000004	62.4318213751964\\
7830.29999999996	62.4335921536575\\
7832.30000000011	62.431693989071\\
7833.50000000004	62.4348958333333\\
7833.89999999992	62.4329844268573\\
7835.59999999992	62.4347537552484\\
7836	62.4328428682635\\
7836.69999999994	62.4285422621478\\
7837.29999999995	62.4314185826932\\
7837.69999999998	62.4295082803848\\
7838.09999999997	62.4314255824041\\
7839.10000000013	62.4336667006837\\
7839.79999999992	62.4357453538948\\
7840.2000000001	62.4338354399704\\
7841.49999999993	62.4362375025505\\
7844.20000000004	62.4415180449498\\
7845.50000000002	62.4439176098705\\
7846.39999999994	62.4418530555024\\
7846.99999999988	62.4447248027934\\
7849.90000000004	62.4420382165605\\
7851.19999999992	62.4393412555882\\
7851.90000000007	62.4426897605706\\
7853.89999999993	62.440794499618\\
7854.29999999995	62.4388877571807\\
7855.40000000006	62.4352364585322\\
7856.10000000001	62.4385835391156\\
7859.40000000011	62.4403588014505\\
7861.70000000011	62.4386272863721\\
7862.10000000006	62.4405382717306\\
7862.4999999999	62.4386335308931\\
7864.50000000004	62.4405564173639\\
7867.79999999999	62.4423289569008\\
7869.10000000011	62.4447211914807\\
7869.79999999997	62.4429789450946\\
7870.59999999991	62.4404436708298\\
7871.60000000006	62.4426743905383\\
7873.60000000007	62.4458640791496\\
7874.9	62.4495238095238\\
7875.6000000001	62.4515916045558\\
7876.49999999992	62.4533428128888\\
7877.70000000013	62.4514458351316\\
7879.30000000003	62.4451100337589\\
7881.10000000001	62.4422676749734\\
7881.49999999999	62.444173771823\\
7881.90000000012	62.4422735346359\\
7883.00000000011	62.44497722977\\
7883.49999999995	62.4422852503932\\
7884.39999999995	62.4402308326463\\
7885.09999999995	62.442296961396\\
7885.80000000009	62.4443627233417\\
7886.89999999999	62.4419931532902\\
7887.30000000004	62.4400943276619\\
7888	62.4421597089286\\
7889.09999999991	62.4397911068296\\
7895.90000000003	62.4417426545086\\
7896.20000000001	62.4393703380064\\
7897.29999999999	62.4370045837871\\
7897.80000000009	62.4393826206966\\
7899.80000000002	62.4374992088508\\
7901.3999999999	62.4324495348984\\
7902.60000000002	62.4343578776874\\
7906.30000000001	62.4380248912274\\
7908.59999999994	62.4363043230872\\
7911.49999999993	62.4323777744072\\
7911.89999999993	62.4342770475228\\
7912.2999999999	62.4323846114959\\
7913.19999999995	62.4341298826027\\
7914.80000000012	62.4303528787477\\
7915.49999999998	62.4324119460306\\
7915.99999999992	62.4297318123824\\
7918.09999999991	62.4346442373267\\
7919.00000000002	62.4313368943441\\
7919.4	62.433234421365\\
7919.70000000005	62.4308694664007\\
7920.70000000011	62.4330875669124\\
7921.39999999997	62.4351448589282\\
7921.79999999989	62.4332546485061\\
7922.2	62.4313646289588\\
7922.89999999991	62.4296352391771\\
7923.30000000008	62.431531918116\\
7924.00000000001	62.429802753625\\
7925.29999999998	62.4321800792389\\
7926.3000000001	62.4343964473153\\
7927.50000000006	62.436298501438\\
7928.19999999995	62.4333085276793\\
7928.99999999999	62.4370987880087\\
7929.40000000011	62.4352102906867\\
7930.19999999996	62.4389997856323\\
7931.3999999999	62.4421610035933\\
7933.60000000012	62.4399712618324\\
7936.69999999989	62.4445620401169\\
7937.7999999999	62.4422076367805\\
7938.69999999993	62.4401672796896\\
7941.40000000006	62.4441226468551\\
7942.29999999994	62.4420829975826\\
7943.20000000005	62.4438205783491\\
7943.70000000009	62.4411490722324\\
7944.79999999988	62.4438318921572\\
7945.69999999992	62.4417931485816\\
7946.50000000004	62.4392822087434\\
7946.89999999998	62.4411727695986\\
7949.19999999994	62.4432339954461\\
7949.6	62.4400920789464\\
7954.7	62.4428018303414\\
7955.29999999999	62.4406063805717\\
7955.70000000005	62.4424947836798\\
7956.30000000011	62.4402996329999\\
7958.50000000005	62.438117256804\\
7958.89999999999	62.440005025757\\
7960.10000000009	62.4418984447627\\
7960.70000000013	62.4397045523063\\
7961.89999999997	62.4428535543833\\
7963.99999999989	62.445222937934\\
7965.00000000008	62.4474268998506\\
7965.50000000009	62.4447624786582\\
7966.99999999994	62.446812516474\\
7967.90000000005	62.4447791164659\\
7969.20000000012	62.4421216418004\\
7969.99999999988	62.4458915195543\\
7970.3000000001	62.4435411020777\\
7971.39999999993	62.4399422944239\\
7972.49999999996	62.4426159596619\\
7973.49999999996	62.4448178990669\\
7974.39999999999	62.4478023700546\\
7975.10000000003	62.4448289698064\\
7976.20000000002	62.4487544350137\\
7977.50000000002	62.4511131167268\\
7977.79999999992	62.4487647125183\\
7979.8000000001	62.450657276407\\
7981.4999999999	62.4523904981457\\
7982.39999999997	62.4503601628562\\
7983.6000000001	62.4560041083708\\
7985.40000000004	62.4594577672031\\
7985.99999999996	62.4572695057663\\
7987.19999999995	62.4553979442365\\
7988.59999999996	62.4519633983001\\
7990.29999999988	62.45369443332\\
7992.59999999994	62.455740863538\\
7993.20000000006	62.4535548521887\\
7994.00000000004	62.4560613452421\\
7995.19999999989	62.4541918377047\\
7995.60000000007	62.456070137699\\
7995.9000000001	62.4537268634317\\
7997.20000000012	62.451077238568\\
7997.89999999995	62.448112028007\\
8000.1000000001	62.4509387265318\\
8000.80000000003	62.449224462248\\
8001.99999999996	62.4473575686382\\
8003.49999999993	62.445649457744\\
8004.0000000001	62.4479954023563\\
8006.19999999996	62.445823913668\\
8007.40000000011	62.4439587886357\\
8009.8	62.4477209453302\\
8010.10000000013	62.4453821377743\\
8010.79999999995	62.4474153965222\\
8011.99999999987	62.4455511039553\\
8012.70000000011	62.4475838658147\\
8014.10000000012	62.4504005390432\\
8014.99999999989	62.4521216204414\\
8016.70000000001	62.4563416824668\\
8019.50000000013	62.4544865080553\\
8020.69999999993	62.4576102134451\\
8021.00000000009	62.4552742142599\\
8021.60000000009	62.4580824513507\\
8022.59999999992	62.455283134107\\
8024.49999999999	62.452957156743\\
8025.00000000004	62.4552965072086\\
8026.30000000005	62.4576397886973\\
8028.70000000007	62.4601434834595\\
8030.4	62.4581283855302\\
8031.59999999995	62.4624923739681\\
8032.59999999988	62.4646756383283\\
8033.69999999995	62.4623465856755\\
8035.90000000013	62.4589347934296\\
8036.90000000004	62.4611173323379\\
8038.09999999993	62.4642332860591\\
8038.49999999999	62.4623690692409\\
8041.00000000002	62.4653343448036\\
8041.99999999988	62.4625408786262\\
8042.39999999993	62.4644078333851\\
8045.19999999999	62.4625557779076\\
8047.00000000008	62.4585254315219\\
8047.39999999991	62.4603914259087\\
8047.90000000004	62.4577534791252\\
8049.90000000003	62.4608695652174\\
8050.8000000001	62.4625818231502\\
8052.30000000012	62.4670905568526\\
8053.10000000009	62.4646103412308\\
8054.0000000001	62.4688047081611\\
8054.49999999993	62.4661684006655\\
8054.99999999999	62.4684982185199\\
8056.90000000002	62.4711431053742\\
8057.50000000005	62.4677323272439\\
8059.20000000013	62.4706860397305\\
8060.20000000002	62.4728608116323\\
8061.40000000013	62.4710041555542\\
8062.10000000009	62.4742626082211\\
8063.59999999997	62.472562223297\\
8064.30000000007	62.470859580378\\
8066.0000000001	62.4688511176405\\
8068.19999999995	62.466690628757\\
8069.20000000004	62.4688634701895\\
8070.09999999994	62.4631359817601\\
8070.5	62.4649964067108\\
8072.10000000012	62.4674809841183\\
8073.59999999993	62.4694997337033\\
8076.40000000011	62.4676530675416\\
8077.00000000009	62.4704411236706\\
8078.29999999999	62.4740047534165\\
8078.79999999989	62.4713760536707\\
8079.40000000009	62.4741630051365\\
8079.79999999997	62.4710701865122\\
8080.29999999988	62.4733924063165\\
8085.1999999999	62.4825300236231\\
8086.8999999999	62.4854705082231\\
8087.99999999989	62.4831542636713\\
8088.49999999999	62.4854733822911\\
8088.80000000007	62.4831559297309\\
8090.79999999989	62.4800702023261\\
8091.20000000005	62.4819250305884\\
8092.19999999988	62.4840898137736\\
8092.80000000011	62.4819286040851\\
8093.19999999987	62.4837828821371\\
8093.69999999992	62.4811584175542\\
8096.70000000011	62.4790040509831\\
8100.19999999997	62.4766983938867\\
8100.9999999999	62.4791694955006\\
8101.90000000009	62.4771661318193\\
8102.80000000011	62.4788655913315\\
8103.50000000004	62.4808726985537\\
8105.29999999989	62.4867372369038\\
8105.90000000005	62.4821120157908\\
8107.30000000007	62.4799565828749\\
8108.50000000007	62.4781096613472\\
8109.40000000013	62.479807633023\\
8110.09999999992	62.4818130255727\\
8111.70000000003	62.4867477008802\\
8112.39999999992	62.4825885978428\\
8112.80000000006	62.4807405490022\\
8113.50000000003	62.482745020706\\
8113.90000000008	62.4808972146907\\
8115.10000000004	62.4839806782335\\
8115.99999999989	62.4819802614556\\
8116.40000000007	62.48382923674\\
8116.69999999997	62.4815198107629\\
8117.79999999988	62.4841399869424\\
8118.59999999995	62.4866049983372\\
8120.80000000004	62.4832223029467\\
8122.99999999989	62.4810724969531\\
8123.99999999989	62.4783052892013\\
8125.09999999998	62.4809235464973\\
8125.70000000008	62.4787713209776\\
8127.50000000007	62.4809291795856\\
8127.99999999991	62.4783159656008\\
8131.0999999999	62.4803227075954\\
8131.60000000013	62.4777106878021\\
8132.19999999991	62.4804790772598\\
8132.59999999996	62.4786356314631\\
8133.20000000001	62.476485559367\\
8134.09999999991	62.4732610459541\\
8134.79999999988	62.4703438272136\\
8136.79999999988	62.4746525089407\\
8137.09999999988	62.4723492110308\\
8137.99999999993	62.4678979122891\\
8139.30000000004	62.4652922819864\\
8141.09999999999	62.4625362354444\\
8142.70000000013	62.4662278331778\\
8143.9999999999	62.463623973183\\
8145.29999999997	62.4610209443367\\
8146.79999999997	62.456885441088\\
8147.20000000001	62.458728658574\\
8148.39999999998	62.4618027857888\\
8150.09999999996	62.4634978282742\\
8150.89999999996	62.4598208808735\\
8151.8	62.4578319164857\\
8152.30000000003	62.4601344389382\\
8153.59999999995	62.4624403644971\\
8153.89999999995	62.4601422614668\\
8154.80000000002	62.4642852763369\\
8155.30000000003	62.4616818304436\\
8155.69999999998	62.4598445278207\\
8157.99999999987	62.4618477341538\\
8161.30000000006	62.4672237606293\\
8163.39999999991	62.4646291419122\\
8164.59999999991	62.4664715176308\\
8169.99999999998	62.4631277462944\\
8171.00000000013	62.4652739533233\\
8172.80000000004	62.4674228242117\\
8173.0999999999	62.4651299368668\\
8173.4999999999	62.4669668200059\\
8174.19999999995	62.4628408548744\\
8174.59999999989	62.4646776028478\\
8176.7000000001	62.4620878583309\\
8177.69999999991	62.464232434151\\
8178.99999999991	62.4665305473707\\
8180.09999999992	62.4691327840395\\
8180.80000000003	62.4711217592197\\
8181.19999999999	62.4692897216824\\
8182.20000000006	62.4665436368747\\
8184.29999999992	62.4688431650457\\
8184.79999999994	62.466248824054\\
8185.19999999988	62.4680830268897\\
8186.40000000004	62.4699199902278\\
8187.90000000007	62.4719101123596\\
8188.80000000012	62.4699288060667\\
8189.29999999998	62.4722201870711\\
8191.59999999994	62.4742117020887\\
8192.99999999992	62.4769623219538\\
8193.39999999991	62.4751327271618\\
8195.79999999994	62.4726997645164\\
8196.30000000006	62.4701088282661\\
8197.89999999996	62.4725542815321\\
8199.59999999988	62.469358635072\\
8201.00000000004	62.4721074002268\\
8201.60000000006	62.4699757367375\\
8205.19999999995	62.4730357208146\\
8205.69999999991	62.4704477320919\\
8206.79999999996	62.4669485432989\\
8209.20000000009	62.4693944672506\\
8210.00000000012	62.4718334734047\\
8211.59999999989	62.474274510759\\
8211.99999999989	62.4724491908282\\
8214.50000000007	62.4704793903538\\
8215.00000000009	62.4727635695244\\
8215.39999999991	62.470939078571\\
8215.89999999995	62.4732229795521\\
8217.70000000006	62.4753583684198\\
8218.40000000013	62.471253878445\\
8218.80000000011	62.4730803392181\\
8219.39999999995	62.4709532209989\\
8220.70000000013	62.4756714674971\\
8221.79999999988	62.4733942276116\\
8224.6999999999	62.47568329929\\
8225.09999999987	62.4726450420658\\
8226.70000000014	62.4690037438615\\
8230.10000000013	62.4711428640859\\
8230.39999999987	62.4688658040216\\
8232.79999999987	62.4713041577087\\
8237.19999999992	62.4694985007223\\
8238.50000000013	62.4657101934795\\
8239.09999999987	62.4684435382076\\
8239.89999999999	62.4708737864078\\
8240.29999999999	62.4690549002476\\
8240.69999999996	62.470876613921\\
8240.99999999998	62.4686024923857\\
8241.89999999993	62.4714875030332\\
8244.20000000009	62.4734665162597\\
8244.79999999991	62.4713459229342\\
8245.99999999991	62.4695310510423\\
8246.99999999996	62.4716567035685\\
8249.19999999997	62.4695428703042\\
8250.80000000006	62.4719727544874\\
8254.10000000006	62.470015264956\\
8255.10000000002	62.4745614885163\\
8255.90000000012	62.4769864341085\\
8257.40000000009	62.4741144414169\\
8259.89999999989	62.4769975786925\\
8260.19999999996	62.4747285207559\\
8260.79999999999	62.4774540304325\\
8261.1	62.4751852031182\\
8261.49999999994	62.4770020335044\\
8264.00000000001	62.4750426543725\\
8265.10000000005	62.4788268886415\\
8265.79999999992	62.4759554313505\\
8266.99999999991	62.4741445004899\\
8268.10000000002	62.4767180281077\\
8268.40000000012	62.4744512305739\\
8268.80000000011	62.4726384404213\\
8270.69999999995	62.478841224549\\
8273.40000000005	62.4765818577386\\
8274.29999999999	62.4746205162912\\
8276.60000000008	62.4765909118368\\
8278.2	62.480219368711\\
8279.1	62.483090153638\\
8280.29999999996	62.4812810975315\\
8281.30000000002	62.4785664259666\\
8281.70000000002	62.4803786616436\\
8282.09999999994	62.4785684962932\\
8282.79999999997	62.4817394873776\\
8284.69999999991	62.4891367323291\\
8285.20000000014	62.4865726044923\\
8286.49999999987	62.48883740014\\
8286.79999999993	62.4865751969977\\
8293.80000000004	62.4844765430015\\
8294.69999999991	62.4825191686358\\
8295.89999999987	62.4843297974928\\
8296.30000000009	62.4825225398968\\
8297.10000000002	62.4801137733211\\
8297.80000000013	62.4820737777028\\
8298.79999999997	62.4781597561122\\
8300.19999999988	62.4832837367324\\
8301.20000000013	62.4781660703745\\
8303.79999999993	62.4754633365045\\
8306.10000000006	62.4786304206496\\
8308.39999999993	62.4757778178973\\
8309.10000000013	62.4777355220719\\
8309.79999999988	62.4748793607625\\
8310.20000000011	62.4766855588848\\
8310.59999999995	62.4748817789115\\
8312.00000000005	62.4775929067263\\
8315.70000000006	62.4750474999399\\
8316.79999999991	62.4788082097897\\
8317.99999999988	62.4806145634219\\
8319.60000000013	62.4758104258567\\
8320.09999999994	62.4732578543785\\
8320.50000000009	62.4750618945749\\
8323.00000000013	62.471915512249\\
8323.40000000001	62.4701147353878\\
8324.5000000001	62.4726713595849\\
8326.49999999987	62.4708764681863\\
8328.20000000005	62.4689312344656\\
8330.09999999997	62.472689731339\\
8331.70000000014	62.4750954175568\\
8332.89999999997	62.4732989319573\\
8335.39999999998	62.4713574470638\\
8336.29999999989	62.4754090494698\\
8338.50000000001	62.4733168637421\\
8340.40000000009	62.475870751154\\
8340.80000000014	62.4740735412246\\
8341.60000000002	62.4776724168934\\
8341.99999999998	62.4758753791012\\
8343.79999999997	62.4731840026846\\
8344.20000000001	62.474982922474\\
8344.59999999998	62.4731865735137\\
8345.80000000014	62.4749877185205\\
8347.39999999987	62.4725965858041\\
8348.29999999994	62.4706530592688\\
8348.99999999988	62.4737995712113\\
8349.89999999989	62.4766467065868\\
8351.20000000013	62.4741058278352\\
8351.80000000013	62.4768016858439\\
8352.09999999988	62.47455760159\\
8353.19999999998	62.4723163300731\\
8353.5999999999	62.474113267175\\
8354.70000000006	62.4778570402643\\
8355.4999999999	62.4754655560343\\
8356.09999999998	62.4733730643115\\
8360.09999999989	62.4709935169015\\
8360.79999999995	62.4729395160808\\
8361.30000000007	62.4692037218648\\
8361.90000000008	62.4718966754365\\
8362.20000000003	62.46965547756\\
8362.60000000006	62.4714506080572\\
8365.10000000004	62.4754937120451\\
8366.20000000002	62.473255800055\\
8366.6000000001	62.4714642571145\\
8367.39999999994	62.4738571855393\\
8367.90000000005	62.4713193116635\\
8370.10000000006	62.4692360994958\\
8374.30000000012	62.4713412304165\\
8376.10000000001	62.473436641914\\
8376.59999999992	62.4709014289637\\
8377.19999999993	62.4735893426283\\
8378.10000000014	62.4716526222816\\
8381.4000000001	62.4756905088588\\
8382.10000000007	62.4728591539214\\
8382.50000000011	62.4746498699687\\
8382.80000000001	62.4724140810459\\
8384.4999999999	62.4704815972139\\
8385.39999999992	62.4733170353587\\
8386.79999999991	62.476004244715\\
8389.10000000007	62.4791398464693\\
8389.60000000003	62.4754162842533\\
8390.80000000003	62.4795909854724\\
8394.19999999995	62.4769188616085\\
8399.39999999992	62.4751473301982\\
8400.49999999998	62.4729186010523\\
8401.69999999994	62.469946916137\\
8402.69999999996	62.4720331318132\\
8406.39999999992	62.4742758579671\\
8407.69999999993	62.4765099074669\\
8408.10000000008	62.4747270521634\\
8411.60000000011	62.4725085297859\\
8412.49999999998	62.4753346171219\\
8415.30000000009	62.4771252703377\\
8415.90000000006	62.4750475285171\\
8416.99999999995	62.4787634696035\\
8418.10000000014	62.4765389275617\\
8419.79999999991	62.4805520255585\\
8420.90000000004	62.4783279895499\\
8422.50000000011	62.4759575428015\\
8423.70000000005	62.4741802986776\\
8424.99999999988	62.4764097755516\\
8426.19999999993	62.4746329943154\\
8426.79999999987	62.4773048214646\\
8428.89999999997	62.4854668406691\\
8432.30000000003	62.4875480290309\\
8432.59999999991	62.4853249848803\\
8433.39999999999	62.4888836189008\\
8433.79999999995	62.4871056095045\\
8435.09999999992	62.4822173748103\\
8439.50000000013	62.4840039812313\\
8440.10000000007	62.4819317077794\\
8440.70000000005	62.4845986162449\\
8441.99999999995	62.4820838416982\\
8445.19999999991	62.4844588114099\\
8446.49999999987	62.4819453981484\\
8447.20000000013	62.4779515348099\\
8449.19999999994	62.4761814588191\\
8449.6000000001	62.4779577973183\\
8450.7999999999	62.4761859683584\\
8452.90000000007	62.4784100319413\\
8455.90000000005	62.4751655629139\\
8456.59999999999	62.4770891718992\\
8457.90000000013	62.4733979664223\\
8459.70000000014	62.4766542944278\\
8461.50000000011	62.4740001890895\\
8462.70000000006	62.4757763388004\\
8464.79999999997	62.4779973774055\\
8466.90000000005	62.4813983701429\\
8467.8	62.479481335396\\
8470.19999999994	62.4771259577583\\
8470.99999999997	62.4747671494847\\
8472.00000000002	62.4768357314007\\
8472.40000000003	62.4750663912659\\
8473.39999999992	62.4783147459727\\
8474.1999999999	62.4747766777197\\
8475.69999999997	62.4778781943887\\
8477.89999999999	62.4758197688134\\
8480.1	62.4737624112639\\
8481.40000000009	62.4712609797795\\
8482.20000000006	62.4736215413272\\
8482.70000000007	62.4711180270665\\
8483.50000000001	62.4734782403697\\
8483.80000000012	62.4712691097255\\
8485.29999999993	62.4672967685672\\
8485.90000000001	62.4699505067169\\
8487.40000000002	62.4718703976436\\
8489.79999999987	62.4742340899186\\
8492.50000000012	62.4720344770742\\
8493.10000000001	62.4746856308576\\
8494.40000000005	62.4721878862794\\
8496.39999999998	62.4704290001766\\
8497.20000000006	62.473962317442\\
8497.60000000005	62.4721983595561\\
8498.00000000005	62.4704345677269\\
8498.69999999999	62.4676424907046\\
8500.30000000006	62.4700014116983\\
8500.69999999994	62.4682382834557\\
8501.70000000009	62.4703004069727\\
8502.59999999992	62.4660401989956\\
8504.69999999999	62.4623741886934\\
8505.09999999991	62.4606123312797\\
8505.50000000001	62.4623777276148\\
8506.70000000014	62.4653218601589\\
8507.50000000006	62.4617988621938\\
8508.30000000002	62.4582765267265\\
8510.29999999987	62.4612239142696\\
8513.00000000005	62.4649070256428\\
8513.5999999999	62.4616794108319\\
8514.00000000013	62.463442994562\\
8516	62.4605159638802\\
8516.40000000007	62.458756531439\\
8516.79999999997	62.4605196726508\\
8518.10000000004	62.4627268671785\\
8518.50000000003	62.4609677646562\\
8519.30000000005	62.4644928046576\\
8521.30000000001	62.4674349285329\\
8524.0999999999	62.464512798855\\
8527.60000000009	62.4670192431723\\
8528.79999999989	62.4640926731466\\
8529.20000000003	62.4658530008324\\
8530.69999999998	62.4677638673981\\
8533.00000000013	62.4708488122722\\
8536.99999999997	62.4743765447283\\
8537.30000000006	62.4721812261344\\
8537.89999999993	62.4701335207309\\
8539.00000000013	62.4679415863499\\
8541.10000000011	62.4631199363087\\
8541.49999999992	62.4648777746558\\
8542.1	62.462831589052\\
8542.49999999998	62.4645892351275\\
8543.00000000003	62.4621039201227\\
8545.49999999999	62.4590432503277\\
8546.10000000012	62.4616788748216\\
8550.50000000007	62.459944331392\\
8551.50000000013	62.4619954160625\\
8552.10000000009	62.4599518252613\\
8552.49999999987	62.458199845661\\
8553.0999999999	62.4608333723051\\
8554.19999999991	62.4586465286464\\
8555.90000000012	62.4567554932211\\
8556.79999999992	62.4595355794739\\
8559.70000000004	62.4617397602748\\
8562.80000000009	62.4636513330764\\
8564.40000000002	62.4671609551054\\
8564.80000000003	62.4642435988745\\
8566.29999999999	62.4673141576392\\
8566.70000000005	62.464397441285\\
8567.90000000002	62.4661531279178\\
8568.30000000002	62.4644040894449\\
8569.50000000002	62.4626587005228\\
8571.00000000013	62.4692279870728\\
8575.39999999992	62.4674946067285\\
8575.79999999996	62.4692452104152\\
8576.19999999991	62.4674976388419\\
8576.9999999999	62.4698324608551\\
8577.60000000009	62.4677943970995\\
8578.1000000001	62.4699820475158\\
8579.29999999997	62.4717346201366\\
8583.89999999997	62.4743709226468\\
8584.30000000008	62.4726247611947\\
8586	62.4695729143616\\
8586.59999999999	62.4721953719124\\
8587.80000000004	62.470452613561\\
8589.79999999998	62.4722057299852\\
8590.70000000009	62.4703170833915\\
8591.7999999999	62.4739580302378\\
8593.50000000006	62.4720722398064\\
8594.19999999996	62.4739653025843\\
8595.10000000005	62.4720774385704\\
8597.50000000001	62.4744114636643\\
8600.30000000004	62.4726756895028\\
8601.59999999988	62.4748596207726\\
8603.50000000005	62.4796596773444\\
8603.89999999994	62.4779172477917\\
8604.6	62.4809697026044\\
8605.59999999996	62.4771953472698\\
8605.99999999997	62.4789393569677\\
8606.80000000014	62.476617597509\\
8608.1	62.4741525522177\\
8609.89999999992	62.4715447154472\\
8611.8	62.4682125895563\\
8613.4999999999	62.4663323116931\\
8614.19999999998	62.4682214457356\\
8616.09999999999	62.4706947378195\\
8616.60000000012	62.4682302969814\\
8617.40000000009	62.4647519582245\\
8618.10000000001	62.4666403657376\\
8618.80000000011	62.4685284665097\\
8619.20000000012	62.466789646491\\
8619.60000000008	62.4685313874033\\
8620.60000000007	62.464765042282\\
8621.29999999989	62.4678126522375\\
8624.09999999994	62.4707219220333\\
8625.10000000007	62.4727542549738\\
8625.90000000002	62.4750753535822\\
8626.40000000013	62.472613458529\\
8629.49999999987	62.4745063502364\\
8631.40000000005	62.4769738747611\\
8632.60000000003	62.4740811101973\\
8633.90000000012	62.4762566597174\\
8635.39999999999	62.4735105089456\\
8636.39999999987	62.475539859897\\
8637.59999999988	62.4772798314366\\
8639.89999999992	62.4803240740741\\
8640.40000000012	62.477865864244\\
8640.90000000001	62.475407938896\\
8641.59999999994	62.4772903479639\\
8641.99999999994	62.4755557098391\\
8644.89999999999	62.4731058415269\\
8645.49999999992	62.4757101878412\\
8646.49999999996	62.4788934378831\\
8647.79999999988	62.4810647671689\\
8648.99999999995	62.4770207304806\\
8650.90000000006	62.4748583978731\\
8651.79999999994	62.4729828130237\\
8653.50000000009	62.4699547009337\\
8655.50000000011	62.4670733398031\\
8657.09999999996	62.4693896409925\\
8658.89999999994	62.4667975516803\\
8660.50000000008	62.4702676488927\\
8661.1999999999	62.4721462136169\\
8661.9	62.4740244747172\\
8662.69999999999	62.4717181511751\\
8668.89999999996	62.4697196908525\\
8670.09999999999	62.4726073216304\\
8673.89999999995	62.4694489278303\\
8674.89999999989	62.4668587896254\\
8675.30000000004	62.4685893445835\\
8675.7999999999	62.4661418411923\\
8676.29999999988	62.4683048268867\\
8677.50000000005	62.46658062137\\
8680.1999999999	62.4644309528473\\
8681.40000000007	62.4661636813915\\
8682.49999999992	62.4628567479787\\
8684.30000000002	62.4602735940307\\
8684.99999999995	62.4621478163752\\
8686.10000000005	62.4645990191338\\
8687.09999999994	62.4666175522608\\
8688.29999999999	62.4706505225358\\
8690.40000000011	62.4751165065301\\
8691.99999999988	62.477422026898\\
8692.29999999994	62.4752657493903\\
8694.69999999992	62.4729723512904\\
8695.70000000003	62.4703880034039\\
8696.39999999995	62.4734088426379\\
8696.70000000002	62.4712537944991\\
8697.09999999996	62.4729798095939\\
8697.49999999992	62.4712564385578\\
8697.8999999999	62.4729822947804\\
8699.70000000004	62.4784477804087\\
8703.50000000002	62.4764465278735\\
8705.20000000001	62.4745844485543\\
8707.2000000001	62.4717191322224\\
8709.7000000001	62.4698615352821\\
8710.80000000012	62.4677128654904\\
8711.60000000006	62.4711594751885\\
8712.3	62.4672880033056\\
8712.69999999991	62.465567900101\\
8713.1999999999	62.4631310754823\\
8714.19999999996	62.4674385779695\\
8715.10000000013	62.4655773820452\\
8715.50000000005	62.4638579099546\\
8715.90000000008	62.4655805415328\\
8716.40000000009	62.4631446107956\\
8717.09999999993	62.4650117010049\\
8717.59999999998	62.4625761382016\\
8718.10000000013	62.4647289578124\\
8719.39999999998	62.462297150066\\
8721.90000000007	62.4661774822288\\
8722.70000000008	62.463887742468\\
8724.29999999997	62.4673329971116\\
8725.00000000007	62.4646135860907\\
8726.29999999997	62.4667675100844\\
8728.69999999993	62.4644853817249\\
8730.39999999989	62.4626310062425\\
8731.00000000015	62.4606292448832\\
8732.80000000002	62.4649314660651\\
8733.40000000003	62.4629300967539\\
8733.90000000002	62.4650790016029\\
8734.30000000008	62.4633632533431\\
8735.09999999987	62.4668009891016\\
8735.9	62.4645146520147\\
8740.39999999991	62.4666781076597\\
8740.79999999992	62.4649635621046\\
8741.2	62.4666811572649\\
8741.50000000011	62.4645373844605\\
8744.00000000012	62.4684072689013\\
8744.4	62.4666933501058\\
8745.99999999986	62.4735596437269\\
8748.80000000014	62.4695676027844\\
8751.59999999991	62.4712912919776\\
8751.90000000006	62.4691499085923\\
8753.80000000008	62.4750111378928\\
8754.50000000003	62.476869302995\\
8754.89999999997	62.4751570531125\\
8756.20000000014	62.4807281614381\\
8757.60000000001	62.4787330006737\\
8758.99999999992	62.4767384776975\\
8760.59999999993	62.4790256486354\\
8761.60000000001	62.4764600477076\\
8762.39999999987	62.4787446504993\\
8763.49999999988	62.4766077867543\\
8763.90000000001	62.4783204016431\\
8765.1	62.4800346826085\\
8765.39999999997	62.4778962979864\\
8766.80000000002	62.4804662993761\\
8768.10000000008	62.4780456650168\\
8769.4999999999	62.4828954570334\\
8769.79999999991	62.4807580474122\\
8770.89999999998	62.483183217421\\
8774.49999999997	62.481480637294\\
8776.99999999989	62.4841918173429\\
8777.40000000005	62.4824836228995\\
8782.29999999997	62.4806431043906\\
8783.19999999989	62.4833490829187\\
8785.89999999999	62.4857728203961\\
8787.69999999998	62.4877671317053\\
8788.90000000013	62.483786551371\\
8789.69999999991	62.4860633916585\\
8790.30000000013	62.4840735347652\\
8792.59999999999	62.485925824832\\
8792.90000000001	62.4837939269874\\
8794.20000000012	62.4859283854315\\
8794.59999999995	62.4842234527613\\
8794.99999999997	62.4859296653819\\
8795.39999999995	62.4842248877267\\
8797.59999999987	62.4867863191516\\
8798.10000000006	62.4843718033234\\
8798.59999999999	62.4865036880448\\
8801.00000000009	62.4899160332231\\
8801.8000000001	62.4921891864257\\
8802.79999999989	62.4884981085779\\
8804.80000000004	62.4867971243285\\
8806.00000000004	62.4850955587604\\
8806.70000000003	62.4869419085252\\
8807.70000000007	62.4843888371671\\
8809.99999999999	62.4862373866358\\
8810.70000000001	62.4835429245926\\
8811.20000000008	62.4811321825383\\
8811.99999999988	62.4834035020029\\
8813.80000000003	62.478584962389\\
8814.20000000004	62.4802877142825\\
8815.79999999996	62.4848285484182\\
8816.69999999998	62.4829870247709\\
8818.19999999993	62.4848326775002\\
8818.70000000014	62.4812899714247\\
8819.90000000005	62.4829931972789\\
8820.70000000002	62.480727371667\\
8821.19999999998	62.4783195220659\\
8821.60000000011	62.4800208576578\\
8822.80000000006	62.4783234537397\\
8827.10000000009	62.4807413449338\\
8828.29999999991	62.4790449005482\\
8829.39999999997	62.4814542159805\\
8829.69999999986	62.4793313551836\\
8830.50000000006	62.4827305052884\\
8831.19999999996	62.4845719203288\\
8833.20000000005	62.4862735331077\\
8834.89999999996	62.4844368986984\\
8836.19999999998	62.4865611172097\\
8837.99999999987	62.4885439178104\\
8838.3999999999	62.486847315721\\
8839.89999999998	62.4898190045249\\
8840.69999999988	62.4875576870871\\
8841.90000000001	62.4847319610948\\
8842.90000000003	62.4878434920276\\
8843.49999999997	62.4858654846443\\
8844.19999999992	62.4888346166457\\
8844.90000000005	62.4861503674392\\
8845.30000000014	62.4878467904221\\
8848.59999999989	62.4905353328738\\
8850.00000000008	62.4885594513056\\
8853.9000000001	62.4903998192907\\
8854.89999999987	62.4923771880294\\
8856.39999999986	62.4896968328346\\
8857.39999999989	62.4928027095682\\
8857.89999999986	62.4904041544367\\
8860.60000000013	62.4939338878418\\
8861.19999999994	62.4919594190469\\
8861.90000000002	62.4949221394719\\
8862.59999999998	62.4899861216108\\
8864	62.4880134475017\\
8866.00000000013	62.4851964223277\\
8867.59999999986	62.4829437170856\\
8869.20000000005	62.4795643399141\\
8870.49999999999	62.4816810587784\\
8871.5000000001	62.4847829027458\\
8871.79999999991	62.4826700030433\\
8873.49999999987	62.4853498016589\\
8874.30000000006	62.4876047958172\\
8878.90000000012	62.4845140218493\\
8880.30000000002	62.4870501328769\\
8880.89999999994	62.4850805089517\\
8883.60000000007	62.4874770647365\\
8887.80000000015	62.4905770766998\\
8888.19999999996	62.487764814419\\
8889.0000000001	62.4900158621233\\
8890.00000000005	62.4874860800216\\
8891.10000000013	62.4898776318157\\
8896.10000000003	62.4918504530024\\
8897.29999999994	62.4946613617461\\
8900.19999999994	62.4990168870712\\
8901.79999999988	62.5023871308372\\
8902.50000000006	62.49971918316\\
8905.10000000009	62.5016844091093\\
8906.60000000002	62.5035085946535\\
8907.1999999999	62.5015436776577\\
8908.39999999995	62.4976146377056\\
8910.00000000009	62.4942481004703\\
8911.30000000008	62.4918643535247\\
8912.3999999999	62.4953716690042\\
8913.29999999992	62.4935490385263\\
8915.50000000001	62.4915877787249\\
8916.10000000003	62.4941118413674\\
8917.39999999999	62.4917297448837\\
8918.30000000006	62.4887872264083\\
8920.49999999992	62.4868282402529\\
8922.2	62.4838886834112\\
8923.50000000011	62.4859922004572\\
8924.79999999996	62.4836132617732\\
8926.99999999998	62.4816569770698\\
8927.89999999986	62.4787186379928\\
8929.20000000011	62.4808215649603\\
8931.29999999996	62.4829254092304\\
8933.69999999995	62.4784526181468\\
8934.49999999993	62.480693036062\\
8936.3000000001	62.4826552079137\\
8937.00000000005	62.4799991048551\\
8937.7999999999	62.4833573882008\\
8938.09999999998	62.4812602089906\\
8939.39999999986	62.4833603669109\\
8940.40000000014	62.4808455902914\\
8941.69999999995	62.4784718960388\\
8942.79999999995	62.4819689362511\\
8943.10000000011	62.4798729761159\\
8943.59999999991	62.4819705491016\\
8943.99999999998	62.4791762167239\\
8944.49999999991	62.4812736176017\\
8946.1999999999	62.4794607826699\\
8947.1000000001	62.4776466380544\\
8947.90000000005	62.4798837729102\\
8948.19999999993	62.4777890772549\\
8949.89999999998	62.4759776536313\\
8951.19999999998	62.4814272787193\\
8953.10000000007	62.479336996828\\
8953.59999999999	62.4814322570557\\
8954.39999999998	62.4836674297839\\
8955.89999999987	62.48771773113\\
8957.69999999993	62.4941391859608\\
8958.30000000013	62.4910698338989\\
8960.19999999993	62.4889791636441\\
8960.8999999999	62.4863296507086\\
8964.40000000014	62.4887054492721\\
8964.69999999999	62.4866143137605\\
8965.90000000012	62.483827793888\\
8967.49999999986	62.4871760560239\\
8968.89999999993	62.4852268926302\\
8969.50000000005	62.4877363539065\\
8971.19999999999	62.4859273460925\\
8973.00000000014	62.4912237688201\\
8973.30000000007	62.4891345532351\\
8974.4	62.4915037049418\\
8975.20000000014	62.489276124475\\
8976.50000000012	62.4869104115144\\
8978.10000000003	62.4891403622107\\
8978.69999999999	62.4871920523901\\
8979.99999999993	62.4892818565495\\
8980.79999999992	62.4915097596009\\
8981.80000000001	62.4934590676806\\
8982.40000000008	62.4915112719176\\
8986.29999999988	62.4933232440132\\
8987.9999999999	62.495966889554\\
8989.29999999989	62.4924911562507\\
8991.20000000009	62.490407393814\\
8992.9999999999	62.48790739567\\
8994.39999999998	62.4859636444494\\
8996.29999999987	62.4838824418656\\
8996.89999999999	62.486384350339\\
8997.69999999986	62.48416279535\\
9000.39999999995	62.486528526193\\
9003.29999999997	62.4830619543728\\
9004.79999999995	62.48486934891\\
9005.10000000014	62.4827877226491\\
9006.59999999991	62.4845948016477\\
9007.59999999988	62.4865392941594\\
9007.89999999997	62.4844582593251\\
9008.39999999998	62.4865404895377\\
9008.90000000014	62.4830724830725\\
9011.40000000001	62.481273927759\\
9012.10000000013	62.4786400656887\\
9013.69999999999	62.481972087244\\
9013.99999999999	62.4798926126846\\
9015.09999999989	62.4778152453634\\
9016.00000000001	62.4760151284924\\
9017.39999999997	62.4796229553646\\
9017.90000000005	62.4772676868485\\
9018.79999999986	62.4810120968189\\
9021.49999999987	62.4833732375632\\
9023.99999999987	62.4804689664343\\
9026.09999999995	62.4781192528417\\
9028.09999999997	62.4808932013026\\
9029.80000000014	62.4790972214532\\
9030.40000000008	62.4815901666574\\
9031.9000000001	62.4856067316209\\
9033.29999999986	62.4836717072199\\
9034.29999999988	62.4878243159479\\
9034.8	62.4854729991478\\
9035.6	62.4876877275695\\
9036.39999999992	62.4899020638522\\
9039.29999999986	62.4930858242804\\
9039.89999999999	62.4900442477876\\
9042.10000000007	62.4881113003473\\
9043.09999999998	62.4922593772116\\
9044.70000000007	62.4889439235804\\
9045.40000000014	62.4918467746393\\
9046.39999999989	62.4893605261703\\
9047.69999999993	62.4947501049979\\
9049.20000000003	62.4921264628203\\
9049.79999999999	62.4946132001459\\
9050.50000000001	62.4919894813604\\
9051.30000000002	62.4941997922973\\
9051.69999999988	62.4914381669944\\
9052.30000000014	62.4895055454907\\
9052.9	62.4875731801613\\
9053.70000000014	62.4853652609954\\
9054.39999999995	62.4882655033409\\
9055.19999999995	62.4904751913244\\
9056.29999999995	62.4884059891348\\
9057.40000000007	62.490753519183\\
9057.80000000015	62.4879939058722\\
9058.49999999987	62.4853730156978\\
9059.20000000006	62.4882717207731\\
9061.59999999996	62.4904819184038\\
9062.70000000013	62.4884141766342\\
9063.39999999989	62.4913113035803\\
9066.49999999996	62.4964154148192\\
9069.79999999998	62.4946250785566\\
9073.10000000007	62.4928360446149\\
9074.50000000009	62.4953165979768\\
9074.79999999988	62.4932506143318\\
9075.70000000005	62.4914608078627\\
9078.19999999986	62.4940792879724\\
9080.49999999991	62.4914653216748\\
9081.4999999999	62.4944943622269\\
9084.59999999988	62.497385714443\\
9086.40000000003	62.5015132339185\\
9087.99999999992	62.4993122874968\\
9089.40000000011	62.4962869244733\\
9090.99999999994	62.4984875317618\\
9091.69999999987	62.4958754042104\\
9093.89999999986	62.4939520563009\\
9096.30000000014	62.4917549799921\\
9097.50000000003	62.4890080900457\\
9097.99999999992	62.4910695639749\\
9103.00000000005	62.4951939449199\\
9103.30000000009	62.4931344332887\\
9104.80000000002	62.4949203176312\\
9105.90000000004	62.4928618493301\\
9107.00000000006	62.4951960558246\\
9107.59999999997	62.4932749212205\\
9108.59999999998	62.4951968996673\\
9109.49999999992	62.4934135417581\\
9111.70000000013	62.4903970675388\\
9115.69999999986	62.4925952741394\\
9116.99999999996	62.490265544965\\
9117.59999999991	62.4883468418571\\
9119.10000000011	62.4901307132205\\
9120.40000000013	62.4921879282934\\
9124.09999999989	62.4942460708884\\
9126.40000000013	62.4960280501835\\
9128.99999999988	62.497946128315\\
9129.89999999988	62.4928806133625\\
9130.69999999988	62.4906908485565\\
9131.49999999996	62.4885014674318\\
9132.00000000012	62.4861751404387\\
9132.69999999997	62.4890504555011\\
9133.20000000014	62.4856295095967\\
9133.79999999991	62.483714514063\\
9135.99999999989	62.4861811932882\\
9136.59999999999	62.4831722613197\\
9138.19999999989	62.4875523893941\\
9138.79999999999	62.4845440917397\\
9143.90000000002	62.4825021872266\\
9144.90000000009	62.4800437397485\\
9147.60000000003	62.4834657892148\\
9149.19999999993	62.4856546402457\\
9149.50000000012	62.4836058406925\\
9150.90000000003	62.4816959895093\\
9151.7	62.4849756332088\\
9153.89999999987	62.4830675114704\\
9155.09999999999	62.4879849702901\\
9156.20000000001	62.4859386433385\\
9161.19999999986	62.4878565269121\\
9163.1999999999	62.4905874521188\\
9168.09999999988	62.4866385986344\\
9169.89999999997	62.484187568157\\
9172.00000000004	62.4862354313625\\
9173.40000000011	62.4843298631929\\
9176.89999999997	62.4812030075188\\
9180.1999999999	62.4794396697276\\
9181.30000000006	62.4773999607903\\
9182.60000000002	62.4750890261034\\
9183.49999999996	62.4711442135982\\
9184.20000000015	62.4729157366375\\
9185.29999999992	62.4752324340802\\
9187.90000000006	62.4717022202873\\
9188.79999999998	62.4742896320561\\
9190.30000000012	62.4760619777159\\
9193.99999999987	62.4813739245821\\
9195.19999999999	62.4840951355584\\
9195.50000000012	62.4820566357823\\
9196.09999999995	62.4790674408995\\
9198.40000000007	62.4819264010437\\
9199.10000000015	62.4836942342812\\
9199.49999999989	62.4809774338015\\
9200.39999999987	62.4846475735014\\
9200.90000000008	62.4823388762091\\
9201.50000000007	62.4847852547383\\
9202.00000000009	62.4813901174732\\
9202.60000000007	62.479489714975\\
9205.69999999992	62.4769167264116\\
9206.40000000007	62.474338782382\\
9207.5	62.4723054867718\\
9208.50000000013	62.4742088916882\\
9209.70000000005	62.4704119524854\\
9210.99999999999	62.4681091292028\\
9211.60000000001	62.4705537522933\\
9212.40000000014	62.4683853459973\\
9213.20000000014	62.465131928842\\
9214.20000000009	62.4616085866534\\
9214.89999999994	62.4590341833966\\
9217.40000000013	62.4572823433686\\
9218.10000000004	62.4590484042438\\
9218.50000000015	62.4563382726227\\
9220.09999999989	62.4520075486432\\
9225.49999999991	62.4501387443635\\
9226.80000000007	62.4521778712244\\
9227.50000000009	62.4539425202653\\
9227.90000000008	62.4512353706112\\
9228.5999999999	62.4529998808066\\
9230.80000000002	62.4511152758669\\
9232.39999999988	62.4532900081235\\
9233.89999999997	62.4550573965779\\
9234.29999999999	62.4523520748506\\
9235.89999999996	62.4545257687311\\
9236.20000000002	62.4524972120871\\
9237.40000000013	62.4497970230041\\
9238.20000000012	62.4519662708507\\
9240.19999999986	62.455764423233\\
9241.00000000009	62.4536040081808\\
9244.10000000012	62.4499686289782\\
9247.39999999991	62.448229251149\\
9248.89999999989	62.4499945940102\\
9253.20000000013	62.447991527347\\
9254.90000000001	62.4462452728255\\
9256.20000000011	62.448278469799\\
9256.50000000006	62.4462545643109\\
9259.79999999987	62.442358988758\\
9262.10000000012	62.4441277450282\\
9264.1999999999	62.441846658679\\
9265.70000000002	62.4436098340133\\
9266.79999999993	62.4405140877748\\
9267.99999999989	62.4432192142942\\
9270.19999999992	62.4456597952601\\
9270.50000000007	62.4436390309149\\
9271.00000000003	62.4456644842575\\
9276.49999999993	62.4474484186017\\
9278.09999999994	62.450690866763\\
9278.70000000002	62.4488080355218\\
9280.49999999992	62.4507036183005\\
9282.70000000011	62.4488300943681\\
9285.30000000003	62.4507291016004\\
9285.60000000012	62.4487114595561\\
9286.10000000015	62.4507333462557\\
9287.1000000001	62.4536997157378\\
9288.80000000015	62.4562650044677\\
9289.09999999995	62.4542479438488\\
9289.80000000011	62.4516948513978\\
9290.70000000014	62.4542558229647\\
9292.40000000002	62.4525154694646\\
9294.29999999994	62.4494319159924\\
9294.79999999995	62.4514518714564\\
9296.59999999991	62.4533436595781\\
9298.59999999987	62.456042242464\\
9301.10000000015	62.4543069711435\\
9302.70000000007	62.4596895558327\\
9304.90000000013	62.4578183772166\\
9305.89999999995	62.4543305394369\\
9307.09999999995	62.4570225201994\\
9307.99999999991	62.4595782168219\\
9308.29999999993	62.4575652099179\\
9308.89999999994	62.4599849607906\\
9309.3999999999	62.4577044954079\\
9310.69999999987	62.4607982128281\\
9311.20000000006	62.45851814462\\
9312.40000000008	62.4558389261745\\
9314.00000000004	62.4579937943548\\
9315.50000000015	62.459744943965\\
9319.70000000005	62.4616408077427\\
9321.39999999991	62.4599045218044\\
9322.59999999991	62.4561554056228\\
9323.59999999987	62.458037045379\\
9324.79999999999	62.4607234393934\\
9326.99999999985	62.4652893182232\\
9328.6999999999	62.4635537260955\\
9330.90000000001	62.4659736362662\\
9331.49999999996	62.4630288482147\\
9332.70000000005	62.4657123264187\\
9334.4999999999	62.4675936837144\\
9335.3999999999	62.4658561405388\\
9336.00000000013	62.4682683347436\\
9336.90000000004	62.4665310056763\\
9337.99999999996	62.4688105717437\\
9339.09999999991	62.4710896008223\\
9340.99999999997	62.4733703739388\\
9342	62.4752464649276\\
9343.70000000009	62.4735118474282\\
9344.60000000012	62.4760559461513\\
9347.40000000001	62.4787376303825\\
9349.00000000001	62.4766020258635\\
9349.69999999992	62.474063616334\\
9351.60000000008	62.4763411999957\\
9353.29999999996	62.4746081638762\\
9354.40000000004	62.4779517879096\\
9356.59999999988	62.4760866544829\\
9358.3999999999	62.4779612117326\\
9359.20000000013	62.4801000074792\\
9362.10000000004	62.4821089060264\\
9362.80000000007	62.4838458170011\\
9365.49999999987	62.4818484667293\\
9368.80000000015	62.4854572041542\\
9371.0000000001	62.4878616170994\\
9371.29999999991	62.4858612373818\\
9372.7	62.4882639125982\\
9373.19999999987	62.4849306007489\\
9373.69999999998	62.4869316605859\\
9374.30000000002	62.4850657108722\\
9375.10000000012	62.4882669169724\\
9376.19999999988	62.4862685707582\\
9377.89999999989	62.4845382810834\\
9379.99999999994	62.4865406552169\\
9381.60000000003	62.4886747604379\\
9383.9999999999	62.4908089214736\\
9387.99999999992	62.4886824810132\\
9388.99999999991	62.4905475498184\\
9390.40000000007	62.4876204674937\\
9392.10000000011	62.4912161155001\\
9393.49999999989	62.488289899506\\
9394.00000000001	62.4860284646746\\
9395.49999999997	62.4877602281919\\
9396.99999999996	62.4852348064829\\
9398.3	62.4829758256725\\
9399.59999999996	62.4849729246678\\
9403.30000000014	62.4869727970734\\
9403.90000000013	62.4851127179924\\
9404.59999999995	62.4868416855402\\
9405.69999999994	62.4837865997576\\
9407.39999999985	62.4820621844273\\
9408.29999999997	62.4803367203775\\
9409.40000000003	62.483660130719\\
9411.89999999993	62.4819379515512\\
9416.30000000004	62.4846013338431\\
9417.79999999995	62.4820819928009\\
9420.70000000008	62.479831861413\\
9423.00000000008	62.4762551601914\\
9429.10000000013	62.4782590251559\\
9431.00000000009	62.4815769104346\\
9435.09999999993	62.4798626420214\\
9436.29999999988	62.4825145182485\\
9438.4999999999	62.4806645053292\\
9439.19999999994	62.4781498628076\\
9440.10000000008	62.4764305840978\\
9440.80000000012	62.4781535658677\\
9441.80000000016	62.4757728846948\\
9442.60000000015	62.4778929755261\\
9443.20000000012	62.476041214406\\
9447.2000000001	62.4781683655648\\
9448.49999999993	62.4801557902758\\
9449.30000000008	62.4822740068153\\
9450.69999999992	62.4793668260888\\
9453.09999999996	62.4836034358736\\
9453.90000000002	62.4814893166914\\
9457.10000000008	62.4836103709343\\
9458.50000000004	62.4817626287188\\
9461.2000000001	62.485070761946\\
9462.90000000003	62.4812427348621\\
9467.09999999989	62.4841558222072\\
9468.19999999996	62.4821773708057\\
9468.8	62.4803303446018\\
9469.5999999999	62.4782200069696\\
9470.39999999999	62.4761100258698\\
9471.59999999999	62.4734736108618\\
9472.60000000006	62.4753238253085\\
9472.89999999992	62.4733452971604\\
9475.20000000003	62.4750667525039\\
9475.89999999991	62.4767834529337\\
9479.00000000015	62.4795602958087\\
9480.69999999986	62.4778499704666\\
9483.50000000015	62.4752203804462\\
9483.99999999996	62.4771986798958\\
9486.90000000009	62.4791820385791\\
9487.80000000014	62.4774713055576\\
9491.90000000006	62.4757690686894\\
9492.70000000012	62.4789314006405\\
9493.90000000013	62.4815673056667\\
9495.49999999985	62.4794641728801\\
9499.39999999994	62.4811832201695\\
9499.70000000008	62.4792100886335\\
9500.3000000001	62.4763167866616\\
9501.10000000015	62.4784237780491\\
9501.90000000014	62.4752683645548\\
9502.39999999986	62.477242830834\\
9503.79999999988	62.4754048338051\\
9504.39999999993	62.4735651533484\\
9505.00000000015	62.4717257051478\\
9505.90000000002	62.4742268041237\\
9506.50000000013	62.4723876044012\\
9507.29999999992	62.4744935523908\\
9508.19999999992	62.4769937843779\\
9508.59999999989	62.4743655809943\\
9509.09999999993	62.4763387035713\\
9510	62.4746322330996\\
9510.69999999996	62.4763426841065\\
9512.69999999993	62.4789756959045\\
9513.79999999986	62.4759562324599\\
9518.19999999996	62.4785938665518\\
9522.4999999999	62.4755843992187\\
9528.09999999993	62.4776977813228\\
9529.99999999988	62.4799320049108\\
9531.49999999993	62.4774434512569\\
9533.29999999987	62.4792833616548\\
9534.00000000004	62.4757449575733\\
9534.69999999994	62.4784998112179\\
9536.40000000002	62.4767996644471\\
9537.80000000004	62.4749682844232\\
9539.10000000006	62.4769372693727\\
9539.99999999999	62.4752361086362\\
9541.49999999992	62.4717028590593\\
9542.9000000001	62.4740647595096\\
9545.09999999995	62.4774755898252\\
9547.30000000007	62.475647820349\\
9548.20000000008	62.472900935245\\
9549.50000000013	62.4706794001843\\
9550.09999999993	62.46884881992\\
9552.79999999993	62.4721288823289\\
9554.89999999985	62.4740973312402\\
9555.80000000002	62.4723992507247\\
9556.5	62.4741016679572\\
9560.39999999993	62.4768579049213\\
9562.10000000012	62.4793457572525\\
9562.40000000015	62.477385620915\\
9565.10000000011	62.4754317735123\\
9567.49999999986	62.480663907354\\
9568.19999999996	62.4781831673338\\
9574.30000000013	62.4801554144385\\
9574.80000000014	62.4779371063928\\
9577.20000000015	62.4800309064141\\
9578.60000000005	62.4782068547924\\
9579.20000000009	62.4763813639828\\
9580.80000000002	62.4805602813932\\
9583.90000000006	62.4843489148581\\
9585.69999999994	62.48722068059\\
9587.50000000001	62.4848762985523\\
9588.00000000006	62.4868326362887\\
9594.40000000008	62.4889259471572\\
9601.49999999999	62.4906265622396\\
9602.49999999988	62.4882844229688\\
9603.80000000015	62.4860733660284\\
9605.60000000005	62.4878978106749\\
9606.09999999997	62.4856863275801\\
9606.59999999993	62.4876388353961\\
9607.30000000013	62.4851676832442\\
9608.50000000011	62.4877713714797\\
9608.80000000008	62.4858204373029\\
9609.69999999989	62.489333805074\\
9610.80000000012	62.4873841159517\\
9613.09999999999	62.4849165730454\\
9614.60000000011	62.4886891946707\\
9615.2000000001	62.4858298753029\\
9617.50000000007	62.4833638329729\\
9620.70000000013	62.4854481955763\\
9622.59999999993	62.4876593887371\\
9624.70000000013	62.4844152605768\\
9625.59999999987	62.4879229562525\\
9627.39999999991	62.4855881589198\\
9628.59999999995	62.4829935505312\\
9629.59999999993	62.4868895188843\\
9629.90000000016	62.484942886812\\
9631.69999999988	62.4867625988912\\
9632.30000000009	62.4849466384286\\
9634.60000000009	62.4824851837629\\
9635.20000000013	62.484821437838\\
9637.20000000002	62.4822305002438\\
9640.60000000006	62.4840519879262\\
9641.60000000011	62.4858686746113\\
9642.20000000002	62.4840546342678\\
9642.70000000015	62.4818517443066\\
9643.49999999989	62.4839271641296\\
9644.69999999989	62.4813370935634\\
9646.29999999998	62.4844501575717\\
9649.39999999998	62.4819938856936\\
9650.59999999993	62.4856228045634\\
9653.30000000011	62.4836845049413\\
9655.70000000008	62.4878311481182\\
9657.00000000012	62.489774362904\\
9657.29999999988	62.4878331642057\\
9658.99999999994	62.490294126782\\
9660.9	62.4873201531933\\
9663.30000000003	62.4893929672786\\
9663.69999999991	62.4868064322523\\
9664.30000000008	62.4849964819337\\
9666.59999999985	62.4877155596015\\
9667.39999999992	62.4856477889837\\
9667.89999999991	62.4875879189077\\
9669.3999999999	62.4851336677181\\
9671.29999999989	62.4831978824162\\
9672.70000000007	62.48139111736\\
9676.30000000008	62.4839816460667\\
9677.60000000013	62.4817880281472\\
9678.09999999999	62.4795933128061\\
9679.20000000013	62.4776585083632\\
9682.10000000014	62.4796017434054\\
9684.29999999994	62.4839948783611\\
9684.80000000004	62.4818015673884\\
9686.50000000007	62.4842566019037\\
9686.79999999995	62.4823214857178\\
9687.90000000001	62.4803881090008\\
9688.99999999992	62.4774230836714\\
9690.09999999986	62.4806505541681\\
9691.89999999987	62.4783326454808\\
9693.69999999988	62.4760155975985\\
9695.59999999993	62.4740864506946\\
9698.00000000003	62.4761551231685\\
9698.30000000002	62.4742225521736\\
9699.69999999997	62.4724221117961\\
9700.69999999997	62.4701055583045\\
9701.4999999999	62.4721695390451\\
9704.40000000011	62.4741099489927\\
9705.40000000013	62.4769460615115\\
9706.90000000002	62.4745029360255\\
9707.99999999993	62.4725744481412\\
9709.10000000008	62.4706463972315\\
9710.30000000003	62.4680754654803\\
9711.89999999991	62.4742586490939\\
9712.90000000006	62.4719448162257\\
9713.70000000003	62.4740060532438\\
9714.79999999999	62.4679615847821\\
9715.80000000013	62.4656490906658\\
9717.59999999986	62.4674562910977\\
9718.40000000009	62.4695169007563\\
9719.19999999994	62.4674616484726\\
9720.10000000004	62.4699080265838\\
9721.39999999985	62.4677261739444\\
9723.9999999999	62.4705628284366\\
9726.10000000001	62.4735251177233\\
9730.29999999989	62.4763627394557\\
9735.59999999985	62.4741929188451\\
9736.40000000015	62.4772762286242\\
9739.39999999999	62.4795934082859\\
9740.3999999999	62.4813921256609\\
9742.30000000016	62.4794711775333\\
9745.79999999985	62.4816589540217\\
9748.99999999991	62.4837164456206\\
9749.30000000008	62.4817937514104\\
9752.19999999988	62.4837217887063\\
9752.80000000002	62.4819284520502\\
9753.5999999999	62.4839804381927\\
9753.89999999989	62.4820586426082\\
9754.80000000006	62.4844949717578\\
9756.20000000014	62.4878283775612\\
9757.30000000009	62.4859081312645\\
9758.09999999989	62.4879588448689\\
9762.10000000009	62.4848907008666\\
9763.59999999993	62.4875815520756\\
9765.70000000006	62.485408261484\\
9766.50000000011	62.488481150042\\
9770.00000000011	62.4855426249475\\
9771.30000000011	62.4874634136357\\
9773.80000000015	62.4909196942879\\
9777.30000000006	62.4890052570213\\
9778.89999999989	62.4910522548318\\
9779.39999999985	62.4888797995808\\
9781.19999999994	62.4927156921882\\
9782.80000000011	62.4886281164072\\
9784.09999999992	62.490545982298\\
9785.10000000011	62.493357315129\\
9787.80000000004	62.4914435169955\\
9791.30000000017	62.4936168474376\\
9795.09999999998	62.4918327344005\\
9797.80000000013	62.4950244440135\\
9798.10000000016	62.4931109795677\\
9800.00000000001	62.4911990693973\\
9802.60000000001	62.4960470074571\\
9803.10000000006	62.4938795495349\\
9803.69999999998	62.4910748893286\\
9809.09999999985	62.4882763120336\\
9814.20000000012	62.4863719266784\\
9816.80000000016	62.4881581762063\\
9817.70000000013	62.4854855466601\\
9819.70000000006	62.4880343795189\\
9821.59999999991	62.4851095024283\\
9822.99999999986	62.4833301096395\\
9823.59999999998	62.4856215071714\\
9824.90000000004	62.4875318066158\\
9827.89999999992	62.4847374847375\\
9828.89999999995	62.4865194831621\\
9830.6	62.4909721586459\\
9832.39999999993	62.4886854818205\\
9833.10000000007	62.4913558149941\\
9836.49999999991	62.4880548156883\\
9837.40000000016	62.4914866581957\\
9839.3000000001	62.489582698132\\
9840.4	62.4876784716224\\
9841.79999999993	62.4930145601967\\
9842.19999999989	62.4904747873973\\
9842.70000000002	62.4923802170114\\
9843.70000000005	62.4941587598285\\
9844.50000000013	62.492127663897\\
9846.29999999999	62.4949220019499\\
9847.10000000003	62.4928913802909\\
9847.5999999999	62.4947957391066\\
9848.20000000016	62.4930190997431\\
9849.50000000007	62.4949236517219\\
9851.10000000002	62.4928942666883\\
9852.30000000002	62.4954325849539\\
9855.00000000015	62.4935312680744\\
9857.00000000002	62.4970833206521\\
9860.2999999999	62.4994929211797\\
9860.6	62.4975914488829\\
9861.69999999996	62.4956904419072\\
9864.99999999997	62.4980993603714\\
9866.89999999995	62.4961994527212\\
9871.50000000004	62.4934154544349\\
9872.20000000016	62.4910102002573\\
9872.70000000005	62.4929098128191\\
9873.19999999988	62.4907579026263\\
9875.00000000009	62.4925317211978\\
9876.49999999987	62.490128181763\\
9879.00000000013	62.4925347450679\\
9879.6999999999	62.4901313791777\\
9880.69999999996	62.491903489596\\
9881.30000000005	62.4891209747607\\
9882.90000000005	62.4921582515431\\
9883.20000000002	62.490261349954\\
9885.09999999992	62.4883664468094\\
9885.70000000015	62.4865969370208\\
9886.89999999994	62.4840699908971\\
9887.79999999996	62.4814166809939\\
9888.90000000004	62.4795227019921\\
9891.30000000004	62.4764947327982\\
9892.70000000004	62.4787724405628\\
9894.00000000014	62.4766274850669\\
9895.1999999999	62.4801673521773\\
9898.49999999997	62.4775220738286\\
9899.00000000013	62.4794173207665\\
9900.59999999985	62.4834607654004\\
9903.40000000014	62.4880092896451\\
9905.50000000014	62.4858665805201\\
9906.30000000005	62.4888960671889\\
9906.90000000008	62.4871303119007\\
9907.59999999987	62.4897806756361\\
9908.4999999999	62.4861231657348\\
9910.59999999994	62.4920540426004\\
9915.69999999987	62.4952096653825\\
9917.19999999988	62.492815584887\\
9918.1999999999	62.4945807245193\\
9918.99999999989	62.4925648496335\\
9920.3999999999	62.4908018749055\\
9921.79999999986	62.489039397696\\
9922.80000000003	62.490804099608\\
9923.10000000015	62.4889148661722\\
9923.60000000003	62.4908048409363\\
9924.20000000016	62.4880344205637\\
9928.50000000016	62.4901798843744\\
9929.70000000007	62.4876633970473\\
9930.39999999994	62.4903076380847\\
9931.20000000009	62.4872876662673\\
9933.90000000009	62.4854036641836\\
9934.39999999985	62.4832653882933\\
9936.29999999989	62.4854071897267\\
9940.39999999985	62.4827724963533\\
9941.70000000005	62.484660725422\\
9944.10000000001	62.4866756501277\\
9944.50000000005	62.4841622589144\\
9946.49999999999	62.4866788651399\\
9950.00000000015	62.4898242228721\\
9950.50000000002	62.4876891845718\\
9952.89999999985	62.4897015975083\\
9955.29999999987	62.4917130401591\\
9956.50000000008	62.4952292951409\\
9957.60000000001	62.493346857206\\
9959.29999999994	62.4957326746591\\
9961.89999999994	62.4984942782574\\
9962.20000000009	62.4966122280999\\
9963.59999999984	62.4928490420226\\
9964.70000000015	62.4949823378292\\
9968.5	62.4982444876914\\
9972.19999999992	62.501128124906\\
9973.30000000003	62.503258668057\\
9973.9000000001	62.5015039101664\\
9974.49999999993	62.498746816915\\
9977.20000000011	62.4948633397813\\
9980.40000000005	62.5008767095837\\
9983.60000000006	62.5048829592235\\
9984.10000000002	62.502754351876\\
9986.40000000007	62.4993741551094\\
9987.29999999992	62.5017522077818\\
9988.40000000002	62.4998748560845\\
9991.79999999985	62.5036279386303\\
9993.39999999989	62.501626056937\\
9995.49999999998	62.5035015406779\\
9996.90000000011	62.5017505251576\\
10000.9	62.4997500249975\\
10001.8999999999	62.50149970006\\
10003.7999999999	62.499625146193\\
10005.2000000001	62.5028734770572\\
10010.3	62.5009989610805\\
10011.7000000001	62.4992508839569\\
10012.7000000001	62.5009987216363\\
10014.1	62.4992510634899\\
10014.9000000001	62.4972541188218\\
10015.9	62.5009984025559\\
10019.1999999999	62.5043665725151\\
10022.2000000001	62.5026191592748\\
10023.7	62.4992517807618\\
10024.1999999999	62.5011222728769\\
10024.6999999999	62.4990024738648\\
10028.2999999999	62.4945155757648\\
10030.5	62.4987538133312\\
10030.9	62.4962615890739\\
10031.7000000002	62.4982555473594\\
10032.2	62.4961374759527\\
10037.1	62.4935240903838\\
10038.7000000002	62.4895405825397\\
10042.7	62.4915362249572\\
10043.0000000001	62.4896695243501\\
10044.3999999999	62.4879287172084\\
10046.1000000001	62.4853178316179\\
10048.3	62.4835794753394\\
10049.2000000001	62.4859442946275\\
10050.0000000001	62.4879354434284\\
10051.8000000001	62.4896785682309\\
10054	62.4919187197263\\
10055.1000000002	62.4900548969687\\
10055.8999999999	62.4920445505171\\
10057	62.4901810661125\\
10057.6000000001	62.4874474283385\\
10060.4000000001	62.4849659559664\\
10061.8000000001	62.4832288136435\\
10062.8999999999	62.4813673854715\\
10063.6999999999	62.4833561875236\\
10064.4000000001	62.4809975657012\\
10065.6	62.4834835133175\\
10066.6	62.4812500620859\\
10067.8000000002	62.4837354363869\\
10068.1	62.4818736218987\\
10069.3	62.4843585516515\\
10070.6	62.4812575094085\\
10071.2	62.4795210151619\\
10072.3000000001	62.4816329772448\\
10075.6999999999	62.4833760098454\\
10076.2000000001	62.4802754979506\\
10078.3999999999	62.4835044897554\\
10079.5	62.4816460970674\\
10083.1	62.4841320215805\\
10084.9999999999	62.4872336417091\\
10085.3	62.4853748983679\\
10085.8000000001	62.4872346543194\\
10087.3	62.4848821301822\\
10090.0999999999	62.4824086737627\\
10093.5999999999	62.4805571792306\\
10094.4000000001	62.4785774431621\\
10096.6999999999	62.4811821567229\\
10099.7	62.4794550387136\\
10101.7999999999	62.483295221691\\
10102.1000000001	62.4814396864049\\
10107.9	62.4831816383063\\
10109.3	62.4814529052169\\
10110.6999999999	62.4797246508684\\
10111.4999999999	62.4826931445073\\
10115.8000000002	62.4808469834617\\
10117.4	62.482826785273\\
10117.9000000001	62.4807274164855\\
10122.2999999999	62.4782660238679\\
10123.4	62.4803674618462\\
10123.9	62.478269458712\\
10124.7	62.4762958280657\\
10126	62.4811131630144\\
10126.9000000001	62.4834600572726\\
10127.8999999999	62.4792654028436\\
10128.7000000001	62.4763051891636\\
10129.7	62.4780351043456\\
10132.3000000001	62.4817417393707\\
10133.0000000001	62.479399196692\\
10134.9	62.4765663542181\\
10136.6999999999	62.4802699076632\\
10139.2000000001	62.4826171431953\\
10139.8000000001	62.4808923164923\\
10141.8000000001	62.4833611059072\\
10143.2000000001	62.4865674878984\\
10143.7000000001	62.4844732743153\\
10148.0000000002	62.4826322168682\\
10149.0999999999	62.4807866629882\\
10151.3000000001	62.4839923557342\\
10153.3999999999	62.4819027921406\\
10156.0000000002	62.4836305274662\\
10157.5	62.4812947940458\\
10158.8999999998	62.4795747612954\\
10159.6	62.4821599062964\\
10161.1	62.484745896154\\
10161.9999999998	62.4811800710483\\
10164.2	62.4833977745639\\
10164.5000000002	62.4815536272947\\
10167.1000000001	62.4793453458179\\
10167.6	62.4811904363819\\
10169.6	62.4787358525817\\
10170.8000000002	62.4811963543049\\
10173.0999999999	62.4778830653088\\
10174.8	62.4802209358323\\
10179.4	62.4785107323543\\
10180.5999999999	62.4809688921194\\
10180.9999999999	62.4785141094774\\
10181.9	62.4808485562758\\
10183.1000000002	62.483305837065\\
10183.7999999999	62.4809748721021\\
10185.4000000001	62.479014285013\\
10186.5	62.4811026250172\\
10187.9	62.483313702395\\
10188.2999999999	62.4808605865494\\
10188.9	62.4791441750908\\
10190.9000000001	62.4766951231479\\
10194.6	62.4785427722248\\
10197.5000000002	62.4813681650585\\
10200.9000000001	62.4840701891971\\
10202.6999999999	62.481867722586\\
10205.7000000001	62.4850575163143\\
10209.6999999999	62.4909400771808\\
10210.5	62.4928995357765\\
10211.3000000002	62.4899621991108\\
10211.7999999999	62.4917987837719\\
10213.6999999999	62.4938808278995\\
10214.5000000001	62.495839288861\\
10215.6000000001	62.4940043266736\\
10217.1	62.4916806952981\\
10218.8999999999	62.4933946570115\\
10221.1000000002	62.4955973858255\\
10223.9999999999	62.493520212048\\
10224.5999999999	62.4957211458527\\
10228.9000000002	62.4978003714928\\
10233.1000000001	62.4946253371379\\
10234.1000000001	62.498290047097\\
10236.6	62.4947492844374\\
10237.5000000001	62.4970696256935\\
10237.8000000001	62.4952382812882\\
10238.8	62.493041244665\\
10239.2999999999	62.49487274645\\
10240	62.4915772306911\\
10240.7999999999	62.4935308420158\\
10241.3999999999	62.4908460674706\\
10242.5	62.4890164606643\\
10245.1000000001	62.4868230976457\\
10246.8000000002	62.4832876284535\\
10249.8999999999	62.4809756097561\\
10251.1999999999	62.4789051144733\\
10252.3000000001	62.4809800632047\\
10252.7000000001	62.4785424469413\\
10253.4	62.4762276295899\\
10255.1999999999	62.4779382368141\\
10256.3	62.4761124761125\\
10258.5	62.4744117130993\\
10260.4000000001	62.4764875006091\\
10260.9999999998	62.4747834052879\\
10262.3999999999	62.4730816077954\\
10263.5	62.4751549164036\\
10265.3000000002	62.4719932978744\\
10269.1	62.474194679235\\
10270.1000000002	62.4759011509026\\
10272.6000000001	62.4782189687229\\
10274.7999999999	62.4765204527538\\
10275.4000000001	62.4748187436135\\
10277.1999999999	62.4726338629796\\
10277.6999999999	62.4744595146821\\
10279.7999999999	62.4772614519597\\
10281.0000000002	62.4748324595617\\
10281.6000000001	62.473131875079\\
10282.4999999999	62.4696088537919\\
10283	62.4714337116239\\
10283.3000000002	62.4696112180796\\
10287.7	62.4720542778825\\
10291.1	62.4737639925373\\
10291.7	62.4710934919062\\
10293.5	62.4727986321598\\
10299.2	62.4692940296913\\
10300.1999999999	62.4709959904081\\
10305.5000000001	62.4679785747555\\
10306.7	62.4713781192999\\
10307.8000000001	62.4734426992889\\
10310.4	62.4693273847049\\
10312.0999999999	62.4726052636683\\
10313.2	62.4707901447645\\
10314.2000000001	62.4744287057774\\
10315.8	62.4783101813705\\
10316.6000000001	62.4763732588909\\
10317.5999999999	62.4780716632583\\
10319.6000000001	62.4814674845199\\
10324.1000000001	62.4794172914124\\
10325.4000000001	62.4773618710958\\
10326.1999999999	62.4802688281379\\
10327.8000000001	62.4822083869906\\
10328.3	62.4791836102397\\
10330.1	62.4808812994908\\
10332.0000000001	62.4829415123741\\
10333.6000000001	62.4800410308021\\
10334.6	62.4778658306482\\
10337.8	62.4740034242931\\
10340.2000000001	62.4759436379989\\
10341.0999999999	62.4782423703245\\
10342.0999999999	62.4760689215061\\
10343.3000000002	62.478488698107\\
10343.8	62.4764353870397\\
10344.9000000001	62.4794586756887\\
10346.4999999999	62.477528850057\\
10347	62.4793420378657\\
10347.9000000001	62.4768071124855\\
10349.5	62.4787431398315\\
10349.7999999998	62.4769321442719\\
10351.9000000001	62.4787480680062\\
10356.9999999999	62.4808102654218\\
10357.4999999999	62.4777940835715\\
10358.0000000001	62.4796053330244\\
10359.6000000001	62.4815390407058\\
10361.4999999999	62.47973285979\\
10364.7000000001	62.4816687249151\\
10368.8000000001	62.4781799419418\\
10369.4000000001	62.4803510294614\\
10371.8999999998	62.4778249132279\\
10373.5	62.4759003624585\\
10374.0999999999	62.4780705982148\\
10374.3999999999	62.4762639163333\\
10375.1999999999	62.4743380914287\\
10379.6000000001	62.4767575170766\\
10380.1	62.4747114699139\\
10387.1000000001	62.4768946395564\\
10389.8	62.4750960066988\\
10391.0000000001	62.4775047877511\\
10391.9999999998	62.4753418462101\\
10393.5000000001	62.4730603448276\\
10394.7999999999	62.4748674830927\\
10395.6	62.4729455447926\\
10396.7	62.4769159741459\\
10398.6000000001	62.4751170819429\\
10402.0999999999	62.4733229509142\\
10405.5000000001	62.4750134542938\\
10406.2000000001	62.472732863746\\
10407.5000000001	62.4706944924863\\
10410.2000000001	62.4679404051757\\
10411.2999999999	62.4661428818411\\
10412.7000000001	62.4683082360172\\
10415.2000000002	62.4705961422139\\
10417.1000000001	62.4726414007603\\
10418.0999999999	62.4685646272869\\
10418.9000000001	62.4704866109991\\
10421.9000000001	62.4678564574938\\
10422.9	62.4647414372062\\
10426.0000000002	62.4624739835605\\
10427.5	62.4659557328628\\
10428.6000000002	62.4679969699004\\
10429.4999999999	62.4654828564854\\
10432.1	62.4633346753321\\
10433.4000000001	62.4651363396751\\
10434.5000000001	62.4633431085044\\
10436.0000000001	62.4658636847098\\
10436.4000000002	62.4634695539692\\
10439.6000000001	62.4615649874996\\
10440.2000000002	62.4637223068303\\
10441.5999999999	62.4658819923959\\
10442.1	62.4628909616747\\
10443.2000000001	62.4649296678253\\
10449.5999999999	62.4630372163794\\
10451.4000000001	62.4656747835239\\
10451.7000000001	62.4638818193995\\
10453.5000000001	62.4617356700084\\
10453.9999999999	62.4597048048134\\
10462.5000000001	62.4615296389043\\
10464.1	62.4634467995642\\
10466.6000000001	62.45999216563\\
10467.6000000001	62.4568911986396\\
10471.7999999999	62.4538049446614\\
10472.5	62.4515402096901\\
10473.7	62.448204090206\\
10474.3	62.4503551516077\\
10474.5999999999	62.4485665460586\\
10477.5	62.4503703138123\\
10480.3	62.4537231403382\\
10481.7999999999	62.4571881052099\\
10485.1	62.4546980505856\\
10488.4000000001	62.4512561376746\\
10489.5999999999	62.4488784235965\\
10491.0000000002	62.4510299205994\\
10492.1999999999	62.448652821593\\
10492.9000000002	62.4511579148003\\
10494.6000000001	62.4477117020973\\
10496.5	62.4497456319189\\
10496.8	62.4479608265297\\
10497.9	62.4499904743761\\
10498.9999999999	62.4520196969264\\
10499.6000000001	62.4484509081212\\
10501.6	62.4508412923622\\
10502.1000000001	62.4488202471863\\
10506.8999999999	62.446940135148\\
10507.3999999999	62.4487270996907\\
10508.5000000002	62.4507546200255\\
10510.1000000001	62.4488592034405\\
10512.1000000001	62.4464907440878\\
10512.8	62.4489912393345\\
10515.1000000001	62.4514987827146\\
10518.5000000001	62.4550795733273\\
10518.8999999999	62.4527046297177\\
10519.8999999999	62.4553231939163\\
10521.1999999999	62.4571108133025\\
10523.9	62.4553401748385\\
10524.7	62.4534432958346\\
10525.2	62.4552269293987\\
10525.5	62.4534468343847\\
10526.8	62.4552337345277\\
10527.5000000001	62.4529807363502\\
10528.9999999999	62.455480525401\\
10529.6999999999	62.4522782958841\\
10530.6	62.4545376850542\\
10536.2999999999	62.4511218252914\\
10538.0999999999	62.4537397278473\\
10538.8	62.4514892446081\\
10541.2999999999	62.4490105678563\\
10542.3999999999	62.4510315390088\\
10543.5	62.4530520884707\\
10544.5999999998	62.4512788415033\\
10545.1000000001	62.4530592117741\\
10548.4000000002	62.4553254017159\\
10550	62.4534364603179\\
10551.1	62.4554553036621\\
10552.0000000001	62.4520237677808\\
10554	62.4544016069584\\
10554.5000000001	62.4523904269229\\
10556.7000000001	62.4488481357987\\
10557.6999999999	62.4467218549319\\
10559.8	62.4437731417911\\
10561.0000000001	62.4404654818154\\
10564.1000000001	62.4382347929801\\
10569.5	62.4347184377838\\
10572.4000000002	62.4365098131946\\
10572.7	62.4347381961259\\
10575.8	62.4391304758933\\
10576.3000000002	62.4361786619266\\
10578.7	62.4333572805989\\
10580.6000000001	62.4315971533074\\
10583.7	62.4293731929931\\
10585.4000000001	62.4335175475887\\
10588.0999999999	62.4364858993974\\
10590.4000000002	62.4323686322648\\
10591.1	62.4301306745222\\
10591.7999999998	62.4326136009592\\
10593.0000000001	62.4349812613871\\
10596.6000000001	62.4373625751413\\
10600	62.43431665739\\
10601.5000000001	62.4311424690613\\
10605.5000000001	62.4330542354982\\
10605.9	62.4306996039977\\
10608.6000000001	62.4289498242009\\
10610	62.4310798201713\\
10611.3000000002	62.4328552311665\\
10612.3999999999	62.4358068315666\\
10614.2000000001	62.4336979358036\\
10615.1	62.4312306880699\\
10615.6999999998	62.4286440965354\\
10617.8	62.4313659009785\\
10618.5999999999	62.4294876020605\\
10620.9999999999	62.4323281016091\\
10621.3999999999	62.4299769335781\\
10622.6999999999	62.4326919456264\\
10623.8	62.4346991217914\\
10628.4000000001	62.4368443336313\\
10630.3000000001	62.4341511137869\\
10630.9	62.4362712820995\\
10638.7000000001	62.4337331277964\\
10639.6999999999	62.4316246545988\\
10641.6	62.4336337239351\\
10642.7000000001	62.4356372383207\\
10644.9000000001	62.4377642085486\\
10650.7000000001	62.4356855823037\\
10654.5000000001	62.4378202841965\\
10656	62.4337234072503\\
10659.2	62.4365577476945\\
10661.6999999999	62.4388002025924\\
10663.3999999999	62.4410371829137\\
10664.6	62.4386996352452\\
10666.9	62.4364863598013\\
10667.5000000002	62.4385991225768\\
10669.0999999999	62.4367337757283\\
10670.3999999998	62.4347500117145\\
10670.9	62.4365101677444\\
10672.6000000001	62.4387455845288\\
10675.4999999999	62.4358349881974\\
10676.8	62.433852522736\\
10678.3000000002	62.4363200479473\\
10678.6000000001	62.4345660052254\\
10680.2000000001	62.4327032012209\\
10680.7000000001	62.4344618380646\\
10682.7999999999	62.4324855610368\\
10686.0000000002	62.4343773687313\\
10686.8000000001	62.4315751059709\\
10694.2000000001	62.4295185285619\\
10695.3000000001	62.431512612899\\
10697.4	62.4295396120589\\
10700.2000000001	62.4318944328664\\
10708.0999999999	62.4343960703013\\
10709.3	62.4376715782397\\
10710.2	62.4352259040363\\
10711.2000000002	62.438732926909\\
10712.0000000001	62.4368704549061\\
10713.9	62.438865036401\\
10715.1	62.4412050171719\\
10716.1000000001	62.4437767118941\\
10721.5999999998	62.4462538589962\\
10723.1999999999	62.4443967808417\\
10724.0999999999	62.4475485350889\\
10725.1999999999	62.4448733368764\\
10727.6000000001	62.4430213372857\\
10728.8999999999	62.4410476279243\\
10729.9000000001	62.4445479962721\\
10730.3	62.4422202341012\\
10730.8999999999	62.444320193831\\
10736.7	62.4422546755085\\
10738.5999999999	62.444243716651\\
10739.7	62.4462280489395\\
10743.5	62.448341338099\\
10747.4000000001	62.4452198185625\\
10750.0000000001	62.4422098399085\\
10750.4999999999	62.4439566163749\\
10751.3000000002	62.4411704522202\\
10752.2	62.4387340382988\\
10753.2999999998	62.4407164245727\\
10757	62.4424798505173\\
10758.3000000002	62.4395820939917\\
10758.9999999999	62.4420258200035\\
10761.9	62.4400669020628\\
10764.7999999999	62.4427537645496\\
10766.9999999999	62.4402113846811\\
10771.2999999999	62.4384945318157\\
10772.5000000001	62.4408220856618\\
10773.2	62.4367649652381\\
10777.9	62.4410836889961\\
10780.0000000001	62.4381963061567\\
10782.5000000001	62.4422680986033\\
10786.1000000001	62.4446051436094\\
10786.5000000001	62.4422895073517\\
10788.6	62.438477295689\\
10792.7999999999	62.4364165330912\\
10793.6	62.4392006448206\\
10795.1000000001	62.4370090410553\\
10796.1999999999	62.4399099691561\\
10798.3000000001	62.4425840865313\\
10798.8	62.4396929316875\\
10799.6	62.4424752539422\\
10800.1	62.4405103609192\\
10800.8000000001	62.442944569434\\
10804.5	62.4391462895433\\
10805.4000000001	62.4348711304428\\
10808.2000000001	62.4390514697038\\
10810.6999999999	62.4347874347874\\
10813.4000000001	62.4367688537476\\
10820.3000000001	62.4385420132343\\
10821.1999999999	62.4407418702004\\
10824.2999999999	62.44410775655\\
10824.6000000002	62.4423771559489\\
10826.6000000001	62.4474678341508\\
10828.4000000002	62.4453987163504\\
10831.2000000001	62.4477209568565\\
10833.0000000001	62.4456526755961\\
10835.3000000002	62.4490097273751\\
10836.2999999999	62.4469380975232\\
10837.7	62.4499437155142\\
10837.9999999999	62.4482150930514\\
10839	62.4516795674918\\
10839.2999999999	62.4499511043047\\
10839.7999999999	62.4516831336083\\
10843.7	62.4495103192608\\
10845.5999999999	62.4477903685331\\
10846.3000000001	62.445604071397\\
10849.7999999999	62.4475801620292\\
10850.1	62.4458535326538\\
10850.8	62.4436682671484\\
10857.5000000001	62.441976127321\\
10859.3000000002	62.4389929462033\\
10863.0000000001	62.4416602995462\\
10864.4000000001	62.4455796401123\\
10869.4000000001	62.443534661208\\
10870.4999999999	62.445495188858\\
10871.1999999999	62.4423941938867\\
10874.8000000001	62.4401143918565\\
10875.4	62.4375890763643\\
10875.8999999999	62.4393159249724\\
10876.1999999999	62.437593666964\\
10877.4999999999	62.4393248510701\\
10884.4000000002	62.443842160871\\
10885.9000000002	62.4416681976851\\
10891.2	62.4434181410851\\
10891.9000000001	62.440323172971\\
10892.6999999998	62.4375734430082\\
10894.0000000001	62.4356302952975\\
10894.8000000001	62.4383886038422\\
10895.5999999999	62.4365575410483\\
10897.2	62.4347315390051\\
10897.8999999998	62.4325564323729\\
10898.8999999999	62.4350857876869\\
10899.4000000001	62.4331391348227\\
10901.3000000002	62.435100078889\\
10902.6	62.4377447788163\\
10904.0999999999	62.4401606720346\\
10912.7	62.4450186936442\\
10914.2000000002	62.4474313515296\\
10915.3000000001	62.4493834399106\\
10917.6999999999	62.4475626957812\\
10918.5000000001	62.445734801165\\
10920.1	62.4439112836761\\
10922.7000000002	62.441864723331\\
10923.7000000001	62.4398103224153\\
10924.6999999999	62.4377562975981\\
10925.4000000001	62.4401629216054\\
10928.4000000001	62.4422381845633\\
10929.2000000002	62.4404124692341\\
10930.8	62.4385915157947\\
10931.5	62.4364228475246\\
10932.8999999999	62.4384889783225\\
10934.7999999999	62.436784972885\\
10937.5999999999	62.4345154831454\\
10939.2	62.4326967904711\\
10939.9999999998	62.4345298489045\\
10940.7999999999	62.4327066329095\\
10941.3000000001	62.4344233827481\\
10944.6999999998	62.436956362839\\
10944.9999999999	62.4352449954774\\
10947.3	62.4385698887407\\
10948.6000000001	62.4366363129869\\
10949.8999999999	62.4383561643836\\
10951.0000000002	62.4366501995233\\
10952.2999999998	62.4392827142909\\
10952.9999999998	62.4362052752189\\
10955.2000000001	62.4382718866667\\
10955.6	62.4359922232263\\
10958.5	62.4377201467341\\
10959.6	62.4360155843682\\
10960.2	62.4335100316597\\
10962.6000000001	62.4317002198364\\
10965.9	62.4338865584534\\
10966.4000000002	62.4319518533716\\
10968.6000000002	62.4349284783065\\
10969.7	62.4332257652829\\
10972.2999999999	62.4357478764901\\
10974.8000000002	62.4388377115053\\
10976.5000000001	62.4410108776853\\
10976.9999999998	62.4390777163367\\
10978.6000000002	62.4372648856422\\
10982.5000000001	62.4396773077413\\
10983.2999999998	62.4415026312435\\
10985.6	62.4375324285207\\
10986.5999999999	62.4409513320651\\
10989.5000000001	62.4426730727233\\
10992.0999999999	62.4388202543622\\
10995.2000000001	62.4421343665021\\
10996.8999999998	62.439756297172\\
10998.8000000002	62.4416987153261\\
11001.8000000002	62.4392150446741\\
11002.6000000001	62.44103719996\\
11002.9	62.4393347268927\\
11006.8	62.4371984845869\\
11007.8999999999	62.4355014534884\\
11010.4999999999	62.4389224928705\\
11011.3	62.4371106308008\\
11013.4	62.438825078313\\
11017.4000000001	62.4406625822555\\
11018.3000000001	62.438285050461\\
11020.9000000001	62.4353506941294\\
11021.9	62.4333151878062\\
11022.4000000001	62.435019278748\\
11027.5	62.4369763139758\\
11029.6000000001	62.4350616970543\\
11031.4999999999	62.4388121396715\\
11032.7000000001	62.4410847654267\\
11033.1000000002	62.4388210129428\\
11033.9000000001	62.4406380279137\\
11036.6000000002	62.4425779444943\\
11039.8000000002	62.4444061993315\\
11041.5000000001	62.4420373858861\\
11042.4000000001	62.4450984831333\\
11048.5000000001	62.4468258421881\\
11053.3000000001	62.4450395353466\\
11054.9	62.4432383536861\\
11056.3	62.4452805614848\\
11058.2	62.4426901060742\\
11059.4	62.4404358244044\\
11062.1000000001	62.4432752978612\\
11067.6999999999	62.4396899112741\\
11070.7000000002	62.4417386277415\\
11072	62.4443420850607\\
11072.8	62.4425398946979\\
11073.9000000001	62.4444645114683\\
11075.2000000001	62.4425523462118\\
11077.2	62.4457223330595\\
11078.4999999999	62.4438105897857\\
11079.2999999998	62.4456197989061\\
11081.8	62.4495799456772\\
11083.1999999999	62.4516163958388\\
11087.2000000002	62.4534377170276\\
11089.2000000001	62.4511916892861\\
11089.9999999999	62.4493917998936\\
11090.8	62.4511987304908\\
11092.1000000002	62.4528948269955\\
11092.7000000001	62.4495168036925\\
11096.3000000002	62.4517861648823\\
11097.4999999999	62.449538638985\\
11098.5999999999	62.4514582788975\\
11105.6000000001	62.4544153002512\\
11110.4	62.4562350929301\\
11112.3	62.4590547496491\\
11113.1000000001	62.4572580354893\\
11114.6000000001	62.4596255409503\\
11117.9000000002	62.4572764885771\\
11119.4999999998	62.4554840102162\\
11120.9999999999	62.4533544343635\\
11124.7000000001	62.4514598015245\\
11126.1	62.4489942657871\\
11127.9	62.446980589504\\
11130.2000000001	62.4439592823194\\
11131.0000000001	62.4457600776204\\
11131.7999999999	62.4430690178676\\
11132.9000000001	62.444983382736\\
11133.3	62.4427398638332\\
11136.3000000002	62.4402859092705\\
11137.1000000001	62.442085982114\\
11138.6000000001	62.4390638045733\\
11141.5000000001	62.4362748617793\\
11143.5000000001	62.4394271151154\\
11144.5000000001	62.4374136353032\\
11150.1000000001	62.4392387580492\\
11151.6999999999	62.4419376244194\\
11152.1000000001	62.4396979968078\\
11152.7	62.4417186715444\\
11154.2999999999	62.4399340170695\\
11155.2000000001	62.4420678959777\\
11157.7000000001	62.4442094319669\\
11158.5000000001	62.4460057713333\\
11159.0000000001	62.4441039151903\\
11161.2	62.4470267800346\\
11161.6000000002	62.4447888762464\\
11164.1	62.4469285752674\\
11164.7999999999	62.443909036355\\
11166.0000000001	62.4461539839335\\
11168.2	62.4499700043874\\
11168.6	62.4477333977992\\
11170.1999999999	62.4495313465171\\
11172.2	62.453568199923\\
11175.5000000002	62.4557070761301\\
11179.0000000001	62.4594108649176\\
11180.3	62.4566205144718\\
11181.8	62.4536080630304\\
11184.6	62.455854873175\\
11186.2	62.4540732860731\\
11187.7999999999	62.4522922085467\\
11189.4000000001	62.4505116403771\\
11192.3999999998	62.4525351798079\\
11194.5000000001	62.4560055741161\\
11196.3000000001	62.4540030724161\\
11198	62.4561309507863\\
11200.2000000001	62.453684276314\\
11200.7999999998	62.4556955244668\\
11204.4000000001	62.4534785130974\\
11209.6999999999	62.45160484576\\
11211.3999999999	62.4483788966686\\
11212.4999999999	62.451171004049\\
11215.7	62.4494017368355\\
11217.2999999999	62.4467345374151\\
11218.4	62.4486339528458\\
11219.8999999999	62.4465240641711\\
11221	62.4484230601278\\
11221.6999999999	62.4463098611631\\
11222.7999999999	62.4490996088355\\
11226.0999999999	62.4521209313926\\
11228.1000000002	62.4543559965088\\
11235.9000000001	62.4572801708793\\
11238.1	62.4601804559449\\
11239.0999999998	62.4581820770162\\
11240.9000000001	62.4561871719598\\
11241.5000000002	62.4581910048392\\
11243.5	62.4559749546409\\
11248.4000000001	62.4536604880651\\
11253.9999999999	62.4563492416097\\
11254.8999999999	62.4540204353621\\
11255.8000000001	62.4516920015281\\
11257.2000000001	62.4492551499916\\
11261.1999999999	62.4474971806097\\
11262.7000000002	62.4498348545655\\
11265.3999999999	62.4535084994008\\
11266.6999999998	62.4489651010047\\
11267.5	62.4516312258156\\
11270.4999999999	62.449204124004\\
11271.5000000002	62.4472124631818\\
11272.4999999999	62.4505437964622\\
11273.1999999999	62.4475530678683\\
11277.2000000002	62.4457981963768\\
11278.8	62.4475791078917\\
11282.2000000002	62.4500323515595\\
11283.1999999999	62.4480426825485\\
11286.1999999998	62.4500500606931\\
11287.2	62.4524908525511\\
11288.2	62.4505018470452\\
11290.4	62.4480758159515\\
11291.2000000001	62.4507364076767\\
11294.4	62.4534065252999\\
11296.4	62.4512016996415\\
11298.0000000001	62.4494384011471\\
11298.8999999999	62.4515443844588\\
11300.2	62.4540941390937\\
11302.3000000002	62.4522225368063\\
11309.3999999999	62.4501525266369\\
11311.4	62.4523714803519\\
11319.1999999999	62.4543920560459\\
11320.5	62.4525201844425\\
11321.7999999999	62.4506487427022\\
11322.9999999999	62.4484460969169\\
11323.4999999999	62.4465717616306\\
11325.8000000001	62.4489003081433\\
11327.4999999999	62.4510046258696\\
11328.2000000002	62.4489111340625\\
11329.6000000001	62.4517860137515\\
11330.3999999998	62.4500242707736\\
11331.1999999999	62.4517928216533\\
11332.3999999999	62.449591881756\\
11333.0999999999	62.4466170190238\\
11334.8	62.4487203239552\\
11335.8000000001	62.4467400029993\\
11338.1999999999	62.4449873437817\\
11339.8	62.4467587897601\\
11343.4000000002	62.4489795918367\\
11347.7999999998	62.4467963235489\\
11354.7	62.4449571987177\\
11356.2999999999	62.4432038322004\\
11358.7999999999	62.4453072040426\\
11360.4	62.4435544210202\\
11363.4999999999	62.4458798268154\\
11365.6000000001	62.4484193670429\\
11370.9000000002	62.4465746196465\\
11373.7000000002	62.4435105241872\\
11375.7	62.4465971623974\\
11377.9000000001	62.4485849885744\\
11382.9000000002	62.4457524378459\\
11384.3	62.4486138926953\\
11385.7999999999	62.4438999112938\\
11388.0999999999	62.44182574946\\
11392.9	62.4436057228123\\
11393.7000000002	62.441854341835\\
11396.2999999999	62.4451581201081\\
11397.3999999999	62.4470278569862\\
11399.2	62.4450624161133\\
11404.9999999998	62.4475015563213\\
11405.7	62.4454225043399\\
11406.9000000001	62.4476198825283\\
11407.5999999999	62.4446645686685\\
11408.4000000001	62.4464215278082\\
11410.8000000001	62.4446800865839\\
11412.5999999999	62.4427173236833\\
11414.1999999999	62.4444775413297\\
11417.3000000001	62.4476675950742\\
11421.2	62.4456060168282\\
11424.4	62.4482471880607\\
11427.5999999998	62.4500118134008\\
11429.1999999999	62.4465190344116\\
11431.7999999999	62.4445630210201\\
11433.3000000001	62.4477408294995\\
11434.0999999999	62.44949362439\\
11439.9000000002	62.4475524475525\\
11441.9000000001	62.4497465478063\\
11442.6000000001	62.4468001433228\\
11444.8999999998	62.4447356924421\\
11445.8999999998	62.4471431067622\\
11447.3	62.4447472788581\\
11448.1000000001	62.4464981394455\\
11448.6	62.4437709084874\\
11451.3000000001	62.4456398344307\\
11454.9000000001	62.4478393714535\\
11456.4999999999	62.4452280781384\\
11457.3000000001	62.4469774992581\\
11458.9000000001	62.4434941967013\\
11460.4000000001	62.446664630688\\
11461.7000000001	62.444816695458\\
11468.1000000001	62.4431035384803\\
11469	62.4408192447533\\
11473.7999999999	62.439100915992\\
11475.4000000001	62.4373665635484\\
11478.7999999999	62.4345538335555\\
11479.5999999999	62.4363006001899\\
11480.1000000001	62.4344523614571\\
11482.7	62.4325077507228\\
11483.2999999998	62.4344706271662\\
11485.1	62.4325218542124\\
11486.9	62.4270914947332\\
11489.8000000001	62.4252604461309\\
11491.1	62.4234196602618\\
11494.6999999999	62.4256185405575\\
11495.6	62.4276903537845\\
11496	62.4255182192222\\
11497.2000000001	62.4216120306507\\
11499.3	62.424126476164\\
11500.9999999999	62.4279416751441\\
11501.8999999999	62.4256651017214\\
11505	62.428835907554\\
11505.6000000001	62.426449498944\\
11507.7999999998	62.424073897062\\
11509.2	62.4269069361299\\
11512.2	62.4288804148606\\
11514.2999999999	62.4270478704926\\
11519.1999999999	62.4291406595887\\
11519.7000000002	62.4272990850536\\
11520.7999999998	62.4291505003949\\
11523.2999999999	62.4312268948401\\
11524.3000000001	62.4292804831488\\
11527.9999999999	62.4274598589533\\
11528.5999999999	62.4294152853314\\
11530.0000000001	62.4313752699456\\
11530.7000000001	62.4293197349707\\
11531.8000000001	62.4311691915469\\
11532.7999999999	62.4292242194071\\
11535.4999999998	62.4310829085613\\
11537.4000000002	62.4329360780065\\
11538.0000000001	62.4305561574263\\
11539.5000000001	62.4337065409546\\
11543.1000000002	62.4315614387692\\
11545.9	62.429412783648\\
11548.0999999999	62.4313745865156\\
11549.6999999999	62.4296524615145\\
11556.1	62.4262300756304\\
11557.5999999999	62.4241847426391\\
11558.5000000001	62.4262453930407\\
11560.2	62.4283106839788\\
11563.7000000001	62.4258461751327\\
11566.1999999999	62.4235926787305\\
11567	62.4218689213372\\
11569.9000000001	62.4200518582541\\
11571.8000000001	62.4227654922701\\
11574.2000000001	62.4245094735751\\
11591.2000000002	62.4269926582868\\
11592.7999999999	62.4252775405636\\
11594.4	62.4270128077968\\
11599.2	62.4253187692361\\
11601.6999999999	62.4222103466704\\
11603.0000000001	62.4246968482561\\
11605.0000000001	62.4225556005549\\
11606.1000000002	62.424393858455\\
11606.9000000002	62.4261221676575\\
11608.7999999999	62.4288261592399\\
11609.2	62.4266751656\\
11610.0000000001	62.4292641751578\\
11610.4999999999	62.4274369972267\\
11612.4999999999	62.4296023285052\\
11613.5	62.4276710064063\\
11614.4999999999	62.4300449434333\\
11615.2	62.4280044424165\\
11616.6000000001	62.4308108154639\\
11619.1000000002	62.4345910217571\\
11620.7000000002	62.4328789756299\\
11621.7000000001	62.4309487342752\\
11623.5000000001	62.4333253036925\\
11624.2999999998	62.431609373387\\
11628.2000000002	62.4295898798621\\
11629.7	62.4318560938279\\
11630.2	62.4291720763867\\
11630.9000000001	62.4271343822543\\
11631.7000000002	62.4297185302361\\
11632.4999999999	62.4271444045183\\
11633.3000000002	62.4254302267609\\
11633.9000000001	62.427368059137\\
11634.3000000001	62.4252217561714\\
11636.4000000002	62.4285652902505\\
11639.2999999999	62.4267573929928\\
11642	62.4243048934471\\
11643.7999999999	62.4266783466021\\
11644.2	62.4245338921189\\
11647.1000000001	62.4218696339034\\
11647.7000000001	62.4238053538007\\
11652.2	62.4220111051037\\
11653.0999999999	62.4197645281983\\
11659.7000000001	62.417880238083\\
11660.2999999999	62.4198140715584\\
11660.7000000001	62.4176728869374\\
11662.5000000001	62.4200435580402\\
11665.3999999999	62.4182418241824\\
11667.9000000001	62.4220089132671\\
11671.1000000001	62.4237439166495\\
11673.5	62.4254728618421\\
11674.0000000002	62.4236557850284\\
11676.2000000001	62.425597149782\\
11677.6999999999	62.423572933258\\
11679.8000000002	62.4217673096516\\
11681.1	62.4199568537479\\
11683.1999999998	62.422432018351\\
11684.6999999999	62.420409420786\\
11687.6000000001	62.4228890200809\\
11689.8000000001	62.4248282705584\\
11694.2999999999	62.4230400875633\\
11696.1000000001	62.4202732511414\\
11697.5	62.4179318834633\\
11700.7000000001	62.4145357582388\\
11701.3999999998	62.4125112165107\\
11702.1000000001	62.4147596178496\\
11702.7999999998	62.4127353049244\\
11703.4000000001	62.4146622805144\\
11705.5000000001	62.4120079278294\\
11706.9999999998	62.4159697961066\\
11709.2000000001	62.4179071336459\\
11710.6000000001	62.4155686679703\\
11712.1000000002	62.4178207339356\\
11712.9000000002	62.416118842312\\
11714.6000000001	62.4138902404671\\
11717.7000000001	62.4118861902405\\
11719.8000000001	62.4100888232835\\
11720.7	62.4078561190362\\
11723.0000000001	62.4049952657573\\
11725.8000000001	62.4080027972266\\
11728.2	62.4105795383815\\
11731.4	62.4123087414227\\
11735.6000000001	62.4104228976542\\
11737.5999999999	62.4083082716376\\
11738.8	62.4061879733195\\
11740.1999999999	62.4081156358867\\
11744.5000000001	62.4099586192803\\
11747.1000000001	62.4080631980387\\
11750.6000000001	62.4098989847413\\
11751.7000000001	62.4074609847002\\
11755.7999999999	62.4095135208704\\
11756.3000000001	62.4077098431493\\
11757.7999999998	62.4048512064229\\
11759.5	62.4026327426103\\
11762.4	62.4\\
11763.0999999999	62.4022374863983\\
11766.8999999999	62.404181184669\\
11769.8	62.4023993406911\\
11770.7000000001	62.4044245081048\\
11771.4000000002	62.4024126067196\\
11774.3	62.4006318793314\\
11776.7999999999	62.4026696329255\\
11778.7	62.4044894216728\\
11780.0000000001	62.4026960721895\\
11781.0000000002	62.4007944928742\\
11785.3000000001	62.4034822746789\\
11786.1	62.4051857256792\\
11787.7	62.4068952645956\\
11789.9	62.4045801526718\\
11790.9	62.4026800101773\\
11792.2000000002	62.4008887155178\\
11796.4000000002	62.3990166574832\\
11804.7	62.4008877744646\\
11805.2000000001	62.3982448561239\\
11805.7	62.3964492029341\\
11807.2	62.3944508905508\\
11808.4999999999	62.3977440170723\\
11810.0999999999	62.4002980474505\\
11812.6	62.3980969634377\\
11817.3000000001	62.3961277438354\\
11819.6999999999	62.3978409110137\\
11820.1000000001	62.3957293446811\\
11820.8999999999	62.3923525928433\\
11822.1000000001	62.3902488538512\\
11822.7999999999	62.3924756193489\\
11827.4000000002	62.3952652716128\\
11829.0999999999	62.3972880668177\\
11830.4	62.3955031486412\\
11832.2999999999	62.3930901592238\\
11832.8999999999	62.3949970421702\\
11835.4000000001	62.3970258966668\\
11836.3	62.3948159913487\\
11844.8999999999	62.3925707049388\\
11846.0000000001	62.3943745198842\\
11848.3999999999	62.3960838924758\\
11851.7000000001	62.3981167417607\\
11852.6	62.395066103082\\
11855.6	62.3927730964852\\
11860.1999999999	62.3947117695168\\
11864.1	62.3969589184269\\
11865.5999999998	62.3991842032076\\
11866.9000000001	62.3965618943288\\
11869.5000000002	62.3946889532924\\
11871.7000000001	62.3966037163699\\
11873.7999999998	62.393990180143\\
11877.0000000001	62.3957026546884\\
11879.8999999999	62.3981481481482\\
11884.6000000001	62.3953486415307\\
11885.2000000001	62.3972470194274\\
11888.7000000001	62.399064665904\\
11890.2999999998	62.4007602771984\\
11892.5	62.3984662731446\\
11893.2	62.4006793741014\\
11893.8000000001	62.3975315077477\\
11894.3999999999	62.3994283072008\\
11898.4999999999	62.4014589951759\\
11898.9000000002	62.3993612908648\\
11900.2000000001	62.3975866154635\\
11903.2000000002	62.3995026589265\\
11904.4999999999	62.3977286091091\\
11910.3000000001	62.3958893068243\\
11912.1000000001	62.3982135961451\\
11913.3000000001	62.3961253714305\\
11916.9999999999	62.399409252251\\
11917.5	62.3976303953816\\
11918.0999999999	62.3995234179658\\
11919.9	62.4026845637584\\
11921.0999999998	62.4005972553099\\
11922.3000000002	62.4035429108233\\
11928.1000000001	62.4017035260978\\
11928.7000000001	62.4035946616592\\
11931.7999999999	62.4016292459709\\
11935.1	62.3994570681681\\
11937.9	62.3973864969007\\
11942.2000000002	62.399202833625\\
11945.2999999998	62.4014264905319\\
11945.7000000002	62.3993370054747\\
11947.5000000001	62.4016538886471\\
11950.6999999999	62.4041905144425\\
11953.5000000002	62.4063043769241\\
11954.3	62.4038011108881\\
11956.0000000001	62.4016192571156\\
11957.1999999998	62.3987020481212\\
11957.7999999999	62.4005887321352\\
11958.8000000001	62.397043206315\\
11960.3000000001	62.3992508611752\\
11963.2000000001	62.3974990178295\\
11964.5000000001	62.3957340822092\\
11971.5000000002	62.3993451167764\\
11978.3000000001	62.3973151673011\\
11980.3999999999	62.3997328992947\\
11980.9000000001	62.3971287872465\\
11981.9999999999	62.3947388187379\\
11983.9000000002	62.3923564753004\\
11989.1	62.3894838688153\\
11990.2000000001	62.3870962361242\\
11992.5000000002	62.3843036539199\\
11993.6999999999	62.3872334039254\\
11994.7000000001	62.3853669923634\\
11995.3000000001	62.3872484452373\\
11997.0999999999	62.3853899243157\\
12002.6000000002	62.3834637206628\\
12003.7999999999	62.3855580269746\\
12005.2000000001	62.3882785103246\\
12007.4	62.3860087445347\\
12008.7999999999	62.387895644064\\
12019.4999999999	62.3905953609105\\
12020.6999999999	62.3926860109144\\
12022.4999999998	62.3949894365611\\
12030.8000000001	62.3926722024121\\
12031.8999999999	62.3944481382979\\
12035.4000000001	62.3962444435213\\
12036.2999999999	62.3932405038051\\
12036.9999999999	62.3954274700717\\
12037.7999999999	62.3929422905989\\
12038.7	62.3949230820348\\
12041.1999999999	62.3969172763738\\
12041.5999999999	62.3948445817451\\
12043.6	62.3927862700001\\
12049.8999999999	62.3900414937759\\
12050.6	62.3922261777324\\
12052.5	62.3940062725055\\
12055.7	62.3965228354817\\
12058.2000000001	62.3985138867005\\
12059.8999999999	62.4004975124378\\
12060.8000000001	62.398328482949\\
12062.4999999999	62.4003117072605\\
12064.2999999998	62.3984615894698\\
12068.9000000001	62.4020217085094\\
12069.9	62.3993371996686\\
12070.5000000001	62.4012062366411\\
12071.0000000001	62.3986215009403\\
12072.6999999999	62.3964614670996\\
12073.7999999999	62.3990591275396\\
12077.5999999999	62.4009538239897\\
12082.0000000001	62.4038867415433\\
12082.5	62.4021319914588\\
12084.8	62.4001853552781\\
12086.0999999999	62.4034022273337\\
12089.3000000002	62.4083908217116\\
12090.3	62.4106729305896\\
12091.2999999999	62.4079924574491\\
12093.3000000001	62.4059404303174\\
12094.7999999999	62.4039884579451\\
12095.7999999999	62.4013095346357\\
12102.0999999999	62.399398456479\\
12107.5	62.4021275892828\\
12107.8999999998	62.4000660720185\\
12108.9999999999	62.4018300286561\\
12111.5000000002	62.4062881865319\\
12115.2	62.4045628255182\\
12117.3	62.408602505488\\
12118.1999999998	62.4105691392357\\
12120.2999999998	62.4088313917032\\
12121	62.4060522559834\\
12121.8	62.4085333157343\\
12122.5	62.4065794466534\\
12124	62.4038072929125\\
12127.0999999999	62.4018734745036\\
12129.8000000001	62.4036471858795\\
12130.8000000002	62.4018003610614\\
12131.9999999998	62.3997494250789\\
12139.9000000001	62.4019769357496\\
12141.0000000002	62.3996178270503\\
12142	62.4018909414352\\
12144.0999999999	62.400158100163\\
12144.9	62.4026348291478\\
12145.3999999998	62.400889218229\\
12152.2999999998	62.3991968664626\\
12156.6000000002	62.4009805292555\\
12159.2999999999	62.4027501356975\\
12160.3999999999	62.4045063936516\\
12163.3000000002	62.4068928095763\\
12164.5000000001	62.4089571379248\\
12165.1999999998	62.4070101025047\\
12167.2	62.4099019503094\\
12171.8999999998	62.412914886625\\
12173.9999999998	62.4111843996682\\
12175.3999999998	62.4130425855201\\
12177.8999999999	62.4100837575957\\
12179.7999999999	62.4126634865639\\
12182	62.4145262311096\\
12184.2000000001	62.4180297596087\\
12184.8999999998	62.4152646696758\\
12185.9999999998	62.4178367156022\\
12186.9000000002	62.4197915811931\\
12188.7000000002	62.4220595956944\\
12191.8000000002	62.4242324822218\\
12194.5999999998	62.4295800634702\\
12195.6999999999	62.4313288181177\\
12196.4000000001	62.4293854794408\\
12197.8	62.4320579771928\\
12200.7	62.4303324372172\\
12202.7000000002	62.4323925656407\\
12204.4999999999	62.4305589695688\\
12205.7999999999	62.4288254041079\\
12207.0999999998	62.4270922078773\\
12212.5000000001	62.4289668047754\\
12216.6	62.4309347041345\\
12221.3000000001	62.4290179521168\\
12224.3	62.4325120251301\\
12225.7999999999	62.4297597722867\\
12227.9000000001	62.4321229964017\\
12230.7000000002	62.4341825555156\\
12231.4000000001	62.4322446143155\\
12235.1	62.4305283117562\\
12238.6	62.4322844746583\\
12240.1	62.434437345795\\
12241.4999999999	62.4370997255261\\
12242.2	62.4351633271526\\
12242.9	62.4332271502083\\
12247.1	62.4314128943759\\
12251.3000000001	62.4295998824624\\
12254.3	62.4265569917744\\
12256.1000000002	62.4247319723895\\
12258.4000000001	62.4228086633764\\
12259.6000000001	62.4207770173822\\
12260.9999999998	62.4226211351347\\
12262.7000000002	62.4245686140196\\
12263.1999999998	62.4228388769744\\
12265.0999999999	62.4245833740991\\
12267.4000000002	62.4267373140412\\
12272.8000000002	62.4286028567005\\
12276.9999999999	62.4267946013309\\
12277.5000000001	62.4250667882974\\
12279.7	62.4269124904314\\
12282.9999999999	62.4288656772313\\
12285.7999999999	62.4260330948486\\
12287.1000000001	62.4243114786119\\
12288.9000000002	62.4224916592074\\
12292.6000000001	62.4207863203365\\
12294.5999999999	62.4187658096578\\
12295.6	62.4169425083566\\
12298.4000000001	62.4214335081514\\
12300.7	62.4195174297607\\
12301.7000000001	62.421759417321\\
12303.4	62.4196366887471\\
12305.1000000002	62.4167018821311\\
12307.7999999999	62.4184466887121\\
12309.8000000001	62.4164290530386\\
12311.9	62.4147173489279\\
12315.5999999998	62.4130175304692\\
12316.7000000001	62.4147505845674\\
12318.5000000001	62.4129365349959\\
12319.5000000001	62.4151758173967\\
12324.8999999999	62.4170385395538\\
12328.5000000001	62.4190905698944\\
12332.1000000002	62.4211414021829\\
12337.2000000002	62.4228964198001\\
12338.1999999999	62.4251315010982\\
12343.7999999999	62.4227351161302\\
12344.2999999998	62.4210168173423\\
12345.2999999999	62.4232507654673\\
12346.6999999999	62.4250818025723\\
12350.0000000002	62.4270248823896\\
12352.3000000001	62.4251157669765\\
12357.7999999998	62.4224180483739\\
12359.6000000001	62.4206089144559\\
12364.1000000001	62.4189191375099\\
12367.6999999999	62.4209641164961\\
12368.8	62.4226891639515\\
12372.5999999999	62.4245314280634\\
12378.4000000001	62.4227491214606\\
12385.4	62.4246094223084\\
12387.8999999998	62.4225056506296\\
12389.0999999999	62.4245310431666\\
12395.0000000001	62.4262813531153\\
12400.7999999999	62.4245014474756\\
12402.8000000001	62.4265292794427\\
12405.8999999998	62.4246332419797\\
12407.9999999999	62.4269630322128\\
12409.5000000002	62.429087158329\\
12411.4	62.4267816138259\\
12413.9	62.428709521508\\
12414.3000000001	62.4266980280964\\
12415.6999999999	62.4285185006202\\
12420.5	62.4261307827319\\
12421.1	62.4279457701349\\
12423.5	62.43037444863\\
12427.4000000002	62.4284852142426\\
12428.4999999999	62.4302013098821\\
12429.6000000001	62.4319171017804\\
12431.1999999999	62.4295126012565\\
12432.1	62.4274062515082\\
12432.9000000001	62.4298238558675\\
12433.4000000001	62.4273133067921\\
12439.0999999999	62.4300598109203\\
12439.6999999999	62.4270486663773\\
12445	62.4293898803545\\
12445.9999999999	62.4267842938752\\
12447.5000000002	62.4297053247212\\
12447.8999999998	62.4276992287918\\
12450.4	62.4304244809445\\
12453.0000000001	62.432647292642\\
12453.3999999998	62.4306419881961\\
12455.2000000001	62.4336627780945\\
12459.9000000002	62.4357945425361\\
12460.9999999999	62.4334930303103\\
12461.9	62.4354036270262\\
12463.3999999999	62.4375175512496\\
12465.3999999999	62.4355220408327\\
12466.5	62.4372322846646\\
12467	62.4355303157912\\
12467.7000000001	62.4376393589887\\
12468.1	62.435636258642\\
12469.5000000001	62.4374478732277\\
12471.9	62.4350545221296\\
12473.5999999999	62.4377690661151\\
12475.9	62.4398845783905\\
12476.4000000002	62.4381837855168\\
12478.7000000001	62.4402987466744\\
12479.2000000001	62.438598318816\\
12480.5000000002	62.4369020720158\\
12481.7999999998	62.4392119789455\\
12482.7000000002	62.4371134681322\\
12485.1999999999	62.4390282972776\\
12489.3000000002	62.4409499255369\\
12490.0999999999	62.4369505692463\\
12492.5999999998	62.4388642967493\\
12495.3000000001	62.4405781327529\\
12496.0999999998	62.437380963813\\
12497.0999999999	62.4395864673687\\
12499.2999999998	62.4421972254668\\
12502.6000000001	62.4441120717925\\
12503.7000000001	62.441817687423\\
12508.3000000001	62.4452368008698\\
12510.9000000001	62.4482455439214\\
12511.7999999999	62.4461512639967\\
12512.7000000001	62.4480531935298\\
12514.4000000002	62.4459626832874\\
12516	62.4491654748684\\
12520.9999999998	62.4473888076926\\
12522.5	62.4454985386421\\
12523.4999999998	62.4429077900923\\
12525.7000000002	62.4447141100768\\
12526.2999999999	62.4425213948142\\
12527.3999999999	62.4442227100379\\
12528.0000000002	62.44203031585\\
12528.6999999999	62.444128727412\\
12530.5	62.4423411488676\\
12532.4000000001	62.4440454817475\\
12533.4000000001	62.4462440659034\\
12533.9999999999	62.4440526244405\\
12538.3	62.4417788553563\\
12540.6000000002	62.4438827178706\\
12544.7999999999	62.446093631675\\
12546.3999999999	62.4437094010282\\
12551.2000000001	62.4413407376128\\
12552.3000000002	62.4430387814283\\
12553.5	62.445832271221\\
12556.8999999998	62.4440551087043\\
12557.8	62.445950357942\\
12559.9000000001	62.4490445859873\\
12563.3000000002	62.4472674594457\\
12567.1	62.4490737793622\\
12572.3	62.4518787184627\\
12575.2000000002	62.4494047855717\\
12577.2000000001	62.4514005390664\\
12582.2000000001	62.4496316253785\\
12583.6000000001	62.4514252564826\\
12584.0000000001	62.4494401665594\\
12589.4	62.4528376821955\\
12590.6000000001	62.450856584622\\
12594.2000000001	62.4488856069809\\
12595.8999999999	62.4468085106383\\
12599.4	62.4485098615024\\
12605.2999999999	62.4502197470925\\
12606.2	62.4473477546941\\
12609.7	62.4490475661787\\
12611.5000000002	62.4464778457928\\
12614.4999999998	62.4482742219333\\
12615.0999999998	62.4453040776206\\
12619.2999999998	62.4435393124871\\
12620.8	62.4416642236291\\
12622.5	62.4435536260359\\
12623	62.441080241779\\
12624.9000000001	62.4388118811881\\
12626.9	62.4408014571949\\
12627.5	62.4386265006811\\
12628.2000000001	62.4407085672656\\
12633.3999999998	62.4427118375747\\
12635.6	62.4452938895352\\
12639.1999999999	62.4425403305563\\
12652.7000000002	62.445466616085\\
12659.3000000002	62.4437177117399\\
12660.2999999999	62.4458942845408\\
12664.9000000001	62.4484800631662\\
12670.1999999998	62.4460352162143\\
12673.7999999999	62.4488121257072\\
12674.8000000002	62.4470410022959\\
12677.9	62.4451806278593\\
12685.1000000001	62.4428467820768\\
12686.5000000001	62.4446266139076\\
12690.8999999999	62.4426759120637\\
12692.0000000002	62.4451430417346\\
12693.4999999999	62.4432785025525\\
12696.8	62.4451637801353\\
12699.9999999999	62.4475397831513\\
12705	62.4497249136174\\
12707.6000000002	62.4479646198761\\
12709.6999999999	62.4502352515382\\
12710.6999999999	62.4484690184725\\
12713.7000000001	62.4518240022653\\
12714.2999999999	62.4496633738124\\
12717.8000000001	62.4521343932568\\
12720	62.4539115258528\\
12722.5999999999	62.4513664552336\\
12726.9000000001	62.4538382965349\\
12727.3000000001	62.4518754812452\\
12728.0000000001	62.4500121777799\\
12733.2000000002	62.4480692357834\\
12736.2	62.4459222851221\\
12738.9999999998	62.4431867243369\\
12746.2000000001	62.4408651922519\\
12747.7	62.4390090839204\\
12752.4000000001	62.4363850225446\\
12754.4999999998	62.4386495852477\\
12755.7000000001	62.4406152495336\\
12759.7999999999	62.4424956308435\\
12761.3	62.4445593743633\\
12764.4	62.4427122096439\\
12767.6999999999	62.4445871645859\\
12768.2999999999	62.4424360139093\\
12780.7999999999	62.4400472580178\\
12783.3000000001	62.441916860929\\
12786.1	62.4399743473432\\
12788.4000000002	62.4381280056301\\
12792.3000000001	62.4409805822207\\
12793.3000000002	62.4392264761518\\
12795.7	62.4423639006549\\
12797.3000000001	62.4400268804601\\
12804.6000000002	62.4426968222606\\
12807.5	62.4402698397826\\
12809.7000000001	62.438133304189\\
12813.6000000001	62.4362986491021\\
12819.9000000001	62.4383775351014\\
12827.4999999998	62.4364651220805\\
12831.4	62.4400888438608\\
12833.1999999999	62.4383439956987\\
12833.8000000002	62.4401000475304\\
12835.0000000001	62.4420534316055\\
12839.3000000001	62.444506752652\\
12842.3000000002	62.4462717249112\\
12844.0000000001	62.4481279342266\\
12845.4999999999	62.4501774926823\\
12846.1	62.4480391088415\\
12848.7000000001	62.4462984870183\\
12850.4999999998	62.444555118049\\
12852.1000000002	62.4468962512255\\
12857.8999999998	62.4451703219785\\
12861.6	62.4474214139655\\
12862.6000000002	62.4448988159562\\
12866.2000000001	62.4476345180821\\
12868.8999999999	62.4516279431191\\
12870.8	62.4494013627641\\
};
\addplot [color=mycolor3, forget plot]
  table[row sep=crcr]{%
12870.8	62.4494013627641\\
12872.8	62.4521281141002\\
12875.6000000001	62.4501968824996\\
12881.8999999999	62.4483775811209\\
12886.6000000002	62.4465534232969\\
12888	62.4498568446862\\
12889.5999999999	62.4521905087008\\
12897.5999999998	62.4545461593928\\
12905.5000000001	62.4581576989834\\
12911.8999999998	62.4605018587361\\
12912.5	62.4583739912953\\
12913.9000000002	62.4608951525476\\
12916.2999999999	62.4585797900344\\
12917.2	62.460421295472\\
12924.2999999999	62.4578316981833\\
12927.5000000002	62.455521519849\\
12928.9999999999	62.4536897386516\\
12929.7000000002	62.4518554037959\\
12930.6999999999	62.4501190954929\\
12931.8999999999	62.4528301886792\\
12935.6000000002	62.4550662121107\\
12937.3000000002	62.4530431153091\\
12939.9999999998	62.4570134697568\\
12947.1999999999	62.4601268218084\\
12952.1999999999	62.4622653891587\\
12954.0000000001	62.4605337306336\\
12954.9000000002	62.4623697414126\\
12957.6000000001	62.4655610177732\\
12960.8999999998	62.4681737520253\\
12963.8999999999	62.4660598580685\\
12964.5000000002	62.4677969239313\\
12964.9000000001	62.4658696490552\\
12966.2	62.4680903573109\\
12968.5000000001	62.4662646700492\\
12969.8	62.4684847223186\\
12971.6999999998	62.4655020891472\\
12973.4	62.4673372644236\\
12975.9000000001	62.469944512947\\
12977.6999999999	62.4682149516867\\
12978.9	62.4701440788967\\
12982.2000000001	62.4727513614691\\
12985.5000000001	62.4745872350912\\
12986.5999999999	62.4723755842516\\
12987.8	62.4743030051047\\
12991.3999999999	62.4762344609937\\
12991.7999999998	62.474310916802\\
12996.1000000002	62.4767239654668\\
13003	62.4743330436588\\
13005.7000000002	62.4721278198957\\
13006.5000000001	62.4698230129319\\
13007.4999999999	62.4719394815339\\
13009.5000000001	62.4738654532038\\
13011.1000000002	62.4715629611412\\
13012.7999999999	62.4733917881487\\
13013.7999999999	62.4716649121324\\
13014.4999999998	62.4698415625528\\
13017.0999999998	62.47272839013\\
13021.2000000002	62.4768648291645\\
13023.3999999999	62.4793642262065\\
13026.6999999998	62.4811926182946\\
13027.2999999999	62.4790825490889\\
13028.4000000001	62.4814829028668\\
13030.3999999998	62.4780323088139\\
13036.4000000001	62.474590572623\\
13038.9	62.477183833116\\
13041.5999999999	62.4795847167164\\
13045.0000000002	62.477865252087\\
13046.3999999999	62.4795922278006\\
13047.3999999998	62.4778693236252\\
13053.6	62.4803695503957\\
13055.1999999999	62.4834358459783\\
13057.4999999999	62.4816198995221\\
13059.7000000002	62.4833458399057\\
13062.3000000001	62.4816266536012\\
13065.4999999999	62.479335047759\\
13069.9000000002	62.4812547819434\\
13073.4999999998	62.4831721943459\\
13074.4000000001	62.4811656277487\\
13082.2999999999	62.4785971992907\\
13087.2999999999	62.4768861653193\\
13088.5000000001	62.4787983435967\\
13090.0000000002	62.476222488751\\
13090.6999999999	62.4782289852415\\
13094	62.4808119687493\\
13095.4000000001	62.4825321675385\\
13096.2000000001	62.4802425112436\\
13100.1	62.4784354437337\\
13103.7000000001	62.4803492116791\\
13108.7	62.4832173806908\\
13109.3999999999	62.4814066135246\\
13110.8000000001	62.4838874524251\\
13111.7	62.4818865449443\\
13113.8000000001	62.4840817758257\\
13114.1999999999	62.4821759453421\\
13115.8999999999	62.4801768831961\\
13124.5000000001	62.481904210414\\
13127.3000000001	62.4838124838125\\
13128.2000000002	62.4818140962653\\
13134.8	62.4793489101554\\
13139.9	62.482496194825\\
13140.8000000001	62.4843047279867\\
13143.9	62.4863055386488\\
13147.5999999999	62.488496086768\\
13151.2000000002	62.4904001885745\\
13156.9	62.492209470244\\
13164.5999999999	62.4943978974075\\
13167.8999999998	62.4916464155529\\
13173.7000000002	62.4944966524465\\
13177.3	62.4963953435427\\
13184.2999999999	62.4981038196657\\
13191.3000000002	62.4998104825871\\
13193.5	62.5022738297356\\
13195.5999999999	62.5044522079162\\
13196.2000000001	62.5023680880247\\
13197.4999999998	62.5\\
13200.7000000001	62.5030301193867\\
13205.4000000001	62.5012305478778\\
13209.3	62.4994322225082\\
13209.9	62.5011355034065\\
13212.4000000001	62.5029328287606\\
13215	62.5080400450999\\
13216.8000000001	62.5108762266492\\
13217.3000000002	62.5085115075582\\
13218.3	62.5113478181928\\
13218.7000000001	62.5094562290072\\
13220.0999999998	62.5111571685754\\
13221.9	62.5086976251702\\
13224.5	62.5069945404776\\
13226	62.5097345400383\\
13226.9999999999	62.5080327509431\\
13228.7999999999	62.5101104400215\\
13230.3000000001	62.5136050308381\\
13232.2999999998	62.5117136725008\\
13233.8000000001	62.5099177113323\\
13235.0999999999	62.5068000483559\\
13239	62.505004116594\\
13240.0000000001	62.5033043557073\\
13249.7	62.5058491448928\\
13250.7000000001	62.5033960213723\\
13254.9000000001	62.5016974726518\\
13259.3	62.4998114545153\\
13263.9000000002	62.501507840772\\
13265.4999999999	62.5045229767217\\
13268.9	62.5028261361067\\
13271.5999999999	62.5006592976032\\
13275.0000000001	62.5027306762284\\
13275.8000000001	62.5004707778757\\
13278.3000000002	62.5037655139173\\
13282.9999999999	62.5019761953159\\
13287.8	62.4997177883639\\
13290.5999999999	62.5031036739976\\
13292.0999999998	62.5050781661426\\
13297.3000000001	62.502444086814\\
13298.5000000001	62.5005639691396\\
13299.4000000002	62.4985901725629\\
13302.8	62.4961474565696\\
13305.9	62.4943634450624\\
13308.1000000001	62.4975578966352\\
13312.7000000002	62.4992488432186\\
13313.1000000001	62.4973710302557\\
13318.3000000002	62.4992491590582\\
13319	62.497466045003\\
13321.0999999998	62.4996246584392\\
13322.3	62.4977481534859\\
13325.0000000002	62.4948405640483\\
13326.9	62.4971861634276\\
13331.6	62.499156146628\\
13333.9000000002	62.4966251687416\\
13339.9000000001	62.4947526236882\\
13341.4999999999	62.4917551118307\\
13342.4000000001	62.4935356942102\\
13349.2	62.4961608473853\\
13352.9999999999	62.4941024930541\\
13354.9999999999	62.4959753202896\\
13358.2	62.4982220791568\\
13358.7999999998	62.496163606285\\
13364.6	62.5019641293856\\
13366.4000000001	62.4995324131224\\
13373.4000000002	62.5012150895428\\
13381.3999999999	62.4989724619811\\
13382.9000000002	62.5009340207726\\
13385.2000000002	62.5028949668666\\
13385.6	62.5010272156107\\
13387.8999999999	62.5044816253361\\
13388.8000000001	62.5025207447961\\
13394.4	62.5047594161783\\
13395.7000000002	62.5024261335643\\
13397.2	62.5006531166728\\
13398.9000000001	62.5031718784984\\
13399.7000000001	62.500932849744\\
13401.8000000002	62.4986009446422\\
13404.1000000001	62.5020515957685\\
13405.9999999999	62.4999067588635\\
13410.6000000002	62.4963648429985\\
13412.3000000002	62.4996272106409\\
13414.0999999998	62.4972044549805\\
13416.6000000002	62.4989751578257\\
13418.4000000002	62.5010247046987\\
13419.7999999999	62.4989754022012\\
13422.0000000001	62.5013969498067\\
13425.6000000001	62.499534474925\\
13428.2999999999	62.4973935837479\\
13429.2999999999	62.5001861587264\\
13429.8999999998	62.4981384959047\\
13431.9000000001	62.5007444907683\\
13432.5000000002	62.4986971993508\\
13434.2000000002	62.4967434105238\\
13434.8999999999	62.4986974320804\\
13436.8000000001	62.5010233015056\\
13437.4	62.498976744186\\
13441.2999999999	62.4972101120419\\
13442.8	62.4991631270038\\
13447.6999999999	62.5009295200702\\
13452.6	62.5026946263575\\
13453.5000000002	62.5\\
13455.2000000002	62.5017651037138\\
13457.2000000001	62.4991640225008\\
13460	62.4973068550754\\
13460.9999999999	62.4993499788279\\
13461.4000000001	62.4974928499796\\
13463.1999999998	62.4950792153484\\
13468.3000000002	62.4929464524368\\
13468.9999999999	62.4948957242874\\
13470.2000000001	62.4967521139099\\
13470.7	62.494432401936\\
13473.7999999999	62.4971240694973\\
13475.8999999999	62.5\\
13478.1	62.497959668205\\
13485.7000000002	62.4998146198223\\
13491.8999999998	62.5037058997925\\
13493.1000000001	62.5018527851066\\
13493.9999999999	62.5036126899904\\
13495.2	62.5010188732374\\
13497.9	62.5033338272337\\
13498.9999999999	62.5004629938292\\
13501.0000000001	62.5023146262156\\
13502.2999999999	62.5\\
13505.6000000002	62.5017585167744\\
13509	62.5037937390352\\
13511.3999999998	62.5015727343374\\
13516.1000000001	62.4998150367707\\
13521.1000000001	62.5018489483182\\
13521.6999999999	62.4998151133725\\
13524.9000000001	62.502033271719\\
13525.5999999998	62.500277249976\\
13526.8000000002	62.5028646622656\\
13529.6000000001	62.5010162826966\\
13530.2000000002	62.4989837623704\\
13531.7000000002	62.5016627499668\\
13535.7999999999	62.4989841828028\\
13539.5000000001	62.4966764158468\\
13540.7999999999	62.4995384354068\\
13544.2999999999	62.4974159062048\\
13545.5000000002	62.4992617528939\\
13548.9	62.4968632371393\\
13549.7999999999	62.4993542387767\\
13553.1999999999	62.5028590822899\\
13555.5000000001	62.5047950662457\\
13556.4999999999	62.5068232447664\\
13557.2000000001	62.5050710687231\\
13558.5999999999	62.5023047932324\\
13559.2999999999	62.5042406006166\\
13564.3999999999	62.5065428139629\\
13570.5	62.5042371007914\\
13573.3000000002	62.5023943890256\\
13574.7	62.5003683295518\\
13578.7999999998	62.5021172554478\\
13581.5999999999	62.5039575310896\\
13583.9000000001	62.5058892815077\\
13586.5000000002	62.5020240531112\\
13589.9999999999	62.5043229998308\\
13590.7	62.5025752715072\\
13592.9999999999	62.5008276257807\\
13593.9000000002	62.5025746652935\\
13598.8	62.5006434343954\\
13599.7000000001	62.5023897410256\\
13600.1	62.5005514624785\\
13605.4000000002	62.4982543824189\\
13609	62.4964178380642\\
13610.7999999998	62.5006428671139\\
13613.9000000002	62.5025708829146\\
13615.9000000001	62.5007344300823\\
13619.4999999999	62.4988986460689\\
13620.7	62.4970633149301\\
13622.4000000002	62.4951367223344\\
13624.0999999999	62.4976145388353\\
13624.8	62.4958715293323\\
13626.5000000001	62.4983488177535\\
13628.7000000001	62.4963312984269\\
13629.6999999999	62.4983492054175\\
13641.7999999998	62.5000916294651\\
13646.5000000001	62.4976184544136\\
13648.2000000001	62.500091586498\\
13651.7000000002	62.5023806384506\\
13652.1000000002	62.5005493620076\\
13655.6999999998	62.498718493241\\
13660.8000000002	62.4944183765345\\
13662.5999999998	62.4964318912075\\
13663.3000000001	62.4946938536528\\
13664.9999999999	62.4971643090793\\
13665.8999999998	62.495243670423\\
13668.7	62.4934156619454\\
13670.4000000002	62.4951537983249\\
13671.5999999998	62.4925941909199\\
13673.5999999999	62.4944236015124\\
13674.2000000001	62.4924127743285\\
13679.9999999999	62.4944262103347\\
13680.6000000002	62.4924163237261\\
13681.6999999999	62.4947009896359\\
13682.8999999998	62.4928743696558\\
13687.5000000002	62.4952511762471\\
13687.9999999998	62.4929683447666\\
13691.4999999998	62.4952525636157\\
13692.3000000001	62.4930618445269\\
13695.5	62.4952539501738\\
13699.7000000001	62.4972627337625\\
13700.9	62.4990876578352\\
13701.6000000001	62.4973543428918\\
13702.9000000002	62.4994526745968\\
13705.1999999998	62.5020977286159\\
13710.6000000002	62.5044673138498\\
13711.5000000001	62.5025525832142\\
13712.7999999998	62.5002734651314\\
13714.8000000002	62.5028253942792\\
13715.5000000002	62.5010936451923\\
13717.8	62.5030070200249\\
13726.4000000002	62.5061013368302\\
13728.0999999999	62.5078305968736\\
13738.8000000001	62.5108269220971\\
13743.1000000002	62.5087315908959\\
13744.1999999999	62.5110045618911\\
13747.2	62.5090017676198\\
13755.8999999999	62.5109043326548\\
13762.4	62.5126248864669\\
13763.9	62.510897994769\\
13765.6000000001	62.5089897353567\\
13766.3	62.5072640632264\\
13772.7999999999	62.5053547183237\\
13774.3000000002	62.5087118132187\\
13775.9	62.5116144018583\\
13779.9000000001	62.5137880986938\\
13780.3999999999	62.5115199013098\\
13781.9000000002	62.5083442170948\\
13784.9	62.506347479144\\
13788.4999999999	62.5081589138854\\
13803.2	62.5060673896822\\
13804.0999999998	62.5077874849683\\
13805.8000000001	62.5058851650381\\
13809.3000000002	62.5081466247628\\
13810.3999999999	62.5060642264943\\
13815.9999999998	62.5082331482835\\
13817.9	62.5061513967289\\
13819.6000000002	62.5078692012128\\
13821.6999999999	62.5048835896917\\
13824.8000000001	62.5067812425406\\
13827.1999999999	62.5046104445553\\
13833.6999999999	62.5063250878284\\
13834.6000000002	62.5080413742257\\
13835.9000000002	62.5057820179243\\
13836.5000000002	62.5037942847231\\
13837.4999999998	62.5014453373417\\
13839.2	62.5031612870593\\
13840.9000000001	62.5012643595116\\
13841.5999999998	62.5031607389266\\
13843.2	62.5009932602775\\
13844.3000000001	62.5032504117188\\
13846.7999999999	62.5056871935235\\
13848.2000000001	62.5037008152625\\
13850.1999999998	62.5018952658065\\
13851.6	62.4984658922731\\
13854.4999999998	62.49621064484\\
13856.6000000001	62.493955992408\\
13857.3000000002	62.4958505924632\\
13859.3000000002	62.4983765530975\\
13861.9999999999	62.4963028689737\\
13865.0999999998	62.4945907740242\\
13865.9	62.4924275205539\\
13867.5000000002	62.490265078312\\
13869.2999999999	62.4922491239708\\
13871.9000000001	62.4942329873126\\
13872.8	62.4959453322665\\
13875	62.4932432919402\\
13883.3999999999	62.4914466813124\\
13887.7999999998	62.4896492630275\\
13895.4000000002	62.4878557806484\\
13897.7	62.4897465785952\\
13899.2999999998	62.4868699368318\\
13900.4999999999	62.4893889472397\\
13902.5999999998	62.4914584936739\\
13904.4999999998	62.4893919997699\\
13908.5000000002	62.4872381116719\\
13910	62.4891266058475\\
13919.8000000001	62.4911098499271\\
13923.5000000001	62.4938952569738\\
13928.7000000001	62.4956923783815\\
13929.7999999998	62.4936288128414\\
13931.5000000002	62.4917453845933\\
13933.4999999998	62.4935407934776\\
13934.0999999999	62.4915675101549\\
13943.6000000002	62.4934558259286\\
13948.5000000001	62.4915762155342\\
13951.2999999999	62.4897859713004\\
13952.0000000002	62.4880842310477\\
13953.4999999999	62.4863834422658\\
13955.0000000001	62.4882659386174\\
13957.3999999998	62.485402113559\\
13958.3000000002	62.4871045392022\\
13958.6999999999	62.485313923833\\
13963.3	62.4833493275277\\
13964.1999999999	62.4850511661881\\
13966.9999999999	62.4832642423982\\
13969.5000000001	62.484967357691\\
13970.2000000001	62.4832680758466\\
13971.4000000001	62.4807644132699\\
13973.3000000001	62.4779939027008\\
13975.3000000001	62.4797859095267\\
13977.2999999998	62.4815774035228\\
13979.7999999998	62.4797030021674\\
13982.2999999999	62.4814051951024\\
13982.9999999998	62.4797076470883\\
13987.7999999999	62.477569899699\\
13992.2	62.4793636500075\\
13998.7	62.4753550304312\\
14003.1999999998	62.4731313333286\\
14004.4	62.4713484951266\\
14006.9000000001	62.4687656171914\\
14009.5	62.4707343535861\\
14012.3999999999	62.4727921498662\\
14013.8999999998	62.4696731839589\\
14016.5	62.4673601301314\\
14017.9000000001	62.4654016264802\\
14018.9000000001	62.4673657179542\\
14023.8999999998	62.469338277239\\
14027.0000000001	62.4719293367838\\
14027.7999999998	62.4697923424034\\
14029.6000000001	62.4717563454671\\
14034.1999999999	62.4698061178684\\
14038.9000000002	62.4716860175226\\
14048.7	62.4736632310233\\
14049.6	62.471796550816\\
14053.2000000002	62.4700248340248\\
14055.5	62.4718973220638\\
14058.3000000002	62.474392534001\\
14063.1999999998	62.4725348957926\\
14072.1	62.4749506118446\\
14074.2000000001	62.4769970797837\\
14075.3999999999	62.4752229050478\\
14078.1	62.4781577190266\\
14079.7999999998	62.4762959964204\\
14080.9999999999	62.4780734459666\\
14081.3999999999	62.4762986897703\\
14082.6	62.4780759371427\\
14085.1000000002	62.4762161701644\\
14088.4000000002	62.4736487205877\\
14089.7999999998	62.4716995862284\\
14091	62.4699278267843\\
14092.7	62.4680688010899\\
14093.7000000001	62.4700222793001\\
14105.3000000001	62.4718193032456\\
14106.6999999998	62.4691638075255\\
14108.4	62.4673069426233\\
14111.7999999999	62.4692635293617\\
14114.1	62.4711283671763\\
14115.8	62.4735227651089\\
14118.6	62.475298717304\\
14130.1000000001	62.4732841714909\\
14134.2000000002	62.4714347367751\\
14136.1000000001	62.4757714237207\\
14136.8999999999	62.4736507038268\\
14144.0999999998	62.4715431060081\\
14156.0000000002	62.4741277611772\\
14159.1	62.4759873439177\\
14160.5999999999	62.4785497892053\\
14166.3000000002	62.4809408177095\\
14173.4999999998	62.4830671106847\\
14179.1000000002	62.4851895734597\\
14181.7000000001	62.4892467810856\\
14182.7	62.4911865076008\\
14190.9999999998	62.493393746785\\
14192.7000000002	62.4915450087368\\
14196.9	62.4934845389871\\
14201.7000000002	62.4913743328311\\
14204.9000000001	62.489264343541\\
14210.0999999999	62.4910275717442\\
14214.2000000001	62.4891834279563\\
14221.7000000002	62.4871675877878\\
14224.3999999999	62.4851488628774\\
14229.6999999999	62.4829582987814\\
14235.0000000001	62.4807693658633\\
14236.2000000001	62.4825270611044\\
14241.8999999998	62.4849038056453\\
14244.9999999999	62.4825378551221\\
14253.4000000002	62.4807941909005\\
14255.8000000001	62.4829018160902\\
14256.2000000002	62.4811486851427\\
14257.9	62.4849207462477\\
14260.0999999999	62.4872021430274\\
14269.1000000001	62.4842317719283\\
14272.3999999998	62.4823962165003\\
14277.4999999998	62.4845912478288\\
14280.7999999998	62.4869581048813\\
14281.5	62.4845955635223\\
14282.5	62.4865220618095\\
14293.8000000001	62.4846962690378\\
14295.3999999999	62.4867965443671\\
14298.1999999999	62.4885475895736\\
14299.4000000002	62.4868002377706\\
14301.6000000002	62.4848794199291\\
14303.7000000001	62.4826969057174\\
14305.6999999998	62.4809517817948\\
14308.7999999998	62.4827904311303\\
14310.1	62.480608237481\\
14312.7000000001	62.4825331172098\\
14319.0999999998	62.4797474719258\\
14321.0999999999	62.478004636483\\
14325.3000000001	62.4799307523699\\
14326.1999999998	62.4780997186992\\
14331.6	62.480375670716\\
14334.2	62.4822977055036\\
14337.3999999998	62.4802092414996\\
14338.9000000001	62.4820419834019\\
14343.9999999999	62.4856212658863\\
14345.9999999998	62.4838806365493\\
14347.2000000001	62.4856244728973\\
14348.6999999998	62.4832738626227\\
14351.1000000001	62.4811862422655\\
14364.1000000001	62.483813926289\\
14365.4999999998	62.4819012084424\\
14367.4000000002	62.4792065425439\\
14370.4999999999	62.4810376741403\\
14372.9999999999	62.4785188998894\\
14375.2999999998	62.4747833103774\\
14377.0000000002	62.4729604718615\\
14379.9000000002	62.4749652294854\\
14380.8999999998	62.4727070440164\\
14383.9999999999	62.4745378577735\\
14385.9	62.4767134714306\\
14386.3000000002	62.4749763665684\\
14391.6000000001	62.4769832611853\\
14392.5	62.4744660450509\\
14400	62.476649467712\\
14401.1000000001	62.4739605032914\\
14407.1	62.4722361041702\\
14415.8	62.4685243377104\\
14426.2999999999	62.4701935340764\\
14427.8	62.4720160245081\\
14428.9999999998	62.4737509615984\\
14429.9000000002	62.4719334719335\\
14436.7000000002	62.4702150060955\\
14437.8999999999	62.4726416401164\\
14438.9999999999	62.4706526030016\\
14440.0999999999	62.4728189360258\\
14440.7000000003	62.4709157387402\\
14444.2000000002	62.4737785839397\\
14445.4999999998	62.4764634213878\\
14446.5999999999	62.4744751396513\\
14449.9000000001	62.4726643598616\\
14454.0999999998	62.4745748640534\\
14462.4000000002	62.4726015557476\\
14467.7999999999	62.4748581342144\\
14470.7999999999	62.4729629808789\\
14474.9999999999	62.4755614814406\\
14476.2000000002	62.4738365466314\\
14477.4000000001	62.4762562597134\\
14478.6999999999	62.4741000635412\\
14480.3999999999	62.4764338247989\\
14489.0999999999	62.4789498385004\\
14493.3000000001	62.4808533539404\\
14495.2999999999	62.4791313106227\\
14497.3	62.4808586367211\\
14503.2	62.4830211055415\\
14505.2000000002	62.4812999386431\\
14513.7999999998	62.4787272890126\\
14514.6	62.4807953316293\\
14517.5999999999	62.4830379467822\\
14520.2000000002	62.4856235752705\\
14525.2000000002	62.4833910487219\\
14529.8000000002	62.4815036579742\\
14530.3999999999	62.4796118509342\\
14533.9999999999	62.4778968081959\\
14537.1999999998	62.4806532162093\\
14540.8999999999	62.4785090433945\\
14541.9000000002	62.4810892586989\\
14543.5999999998	62.4779113980624\\
14544.4	62.4799752483757\\
14548.2000000001	62.4774028580659\\
14552.5999999998	62.4791275845719\\
14558.6000000001	62.4808533728973\\
14561.3	62.4830030079525\\
14564.2000000001	62.4808607348105\\
14574.8999999999	62.4782161234992\\
14576.3999999998	62.4807052447432\\
14582.4999999999	62.4785703509662\\
14585.0999999999	62.4811452705482\\
14586.2000000001	62.4791756648362\\
14588.2000000001	62.480892221849\\
14591.9999999999	62.4790126164157\\
14594.0999999998	62.480985597018\\
14596.1999999998	62.4829580099066\\
14597.7000000001	62.4785926646481\\
14598.8999999998	62.4768819782177\\
14601.3000000001	62.4789403755804\\
14603	62.4771452636769\\
14603.9999999998	62.4790298614773\\
14604.9999999998	62.4768060472027\\
14607	62.4792053179618\\
14610.9999999998	62.4812642443074\\
14613.0000000002	62.4795560148086\\
14617.5999999998	62.4776811673519\\
14622.4000000002	62.4804240041033\\
14626.3	62.4829076190997\\
14627.0000000002	62.4806010760848\\
14630.1999999998	62.482655858048\\
14631.9999999999	62.4845374211494\\
14632.9000000002	62.4827444816511\\
14634.1	62.4810375695289\\
14640.0000000001	62.4831797597011\\
14641.0999999998	62.481217386553\\
14645.9999999998	62.4787486088447\\
14650.9999999999	62.4806328535059\\
14653.8000000001	62.4782481114243\\
14666.2000000002	62.4813347606417\\
14675.4000000001	62.4762359033764\\
14676.8	62.4784525342545\\
14678.3000000002	62.4809243514279\\
14679.6999999999	62.4790528481314\\
14684.9000000001	62.4821246169561\\
14686.3000000001	62.4802538402876\\
14687.9999999999	62.4784689646721\\
14691.1000000002	62.4802602918754\\
14692.9000000001	62.4821343496903\\
14693.3000000002	62.480433391863\\
14694.1999999998	62.4827313992501\\
14696.1000000002	62.484860031845\\
14697.2999999998	62.4831602868535\\
14698.9000000002	62.4811211647051\\
14699.6999999998	62.4790813480456\\
14704.7	62.4823186986562\\
14711.4000000002	62.484450939741\\
14712.5000000002	62.4824979949159\\
14718.4999999999	62.4842036606743\\
14721.2	62.486329332328\\
14723.1	62.483699195827\\
14727.4999999999	62.4820065726935\\
14730.9000000001	62.479125653384\\
14733.0000000001	62.4770075544183\\
14733.7000000001	62.4787902645618\\
14735.8	62.4807443047252\\
14738.0999999998	62.4825283955978\\
14743.4000000001	62.4851629531658\\
14749.9000000002	62.4827118644068\\
14750.5999999999	62.4844922613842\\
14755.0000000002	62.4861912152408\\
14756.6999999998	62.4844139650873\\
14759.9000000001	62.4823848238482\\
14761.7000000003	62.484249888225\\
14762.5999999999	62.4824727184052\\
14766.5000000001	62.4849322118836\\
14768.1	62.482902452567\\
14769.4999999999	62.4864586718665\\
14770.5000000001	62.4842592717967\\
14772.0000000001	62.4860378686849\\
14778.8000000003	62.4843526920136\\
14780.3000000002	62.4861302806419\\
14783.9	62.4878246753247\\
14786.1000000001	62.485966644574\\
14789.8000000001	62.4892663236398\\
14795.2999999999	62.4917203995837\\
14799.3999999998	62.494003175783\\
14801.1999999999	62.4958618499726\\
14802.1000000001	62.4940887165421\\
14809.1000000002	62.4922345568971\\
14811.3000000002	62.4944299661072\\
14815.1999999998	62.4921533819767\\
14816.9000000001	62.4903826685564\\
14818.4000000002	62.4921550764247\\
14824	62.4901343083223\\
14825.5	62.4871843298079\\
14827.5999999998	62.4891250834587\\
14828.9999999999	62.4872716483131\\
14834.0000000001	62.4850850405485\\
14835.6999999999	62.4873616522196\\
14841.6000000002	62.4894722302701\\
14842.4999999998	62.4870305741582\\
14852.7000000001	62.4899008941075\\
14855.3999999999	62.4879674194743\\
14859.6999999999	62.4900738906311\\
14861.8999999999	62.4882250033643\\
14864.3999999998	62.4864610313162\\
14870.0999999999	62.4847009455152\\
14874.9	62.4867226890756\\
14877.4	62.4849605108385\\
14880.9999999998	62.487316125824\\
14884.1	62.4843794090378\\
14886.4999999999	62.4863971625489\\
14887.0000000001	62.4842984866092\\
14893.7	62.4864037384684\\
14896.3000000002	62.4842243763594\\
14911.5000000001	62.4822286005526\\
14916.1999999999	62.4799715747203\\
14920.9000000001	62.4817371489846\\
14924	62.4794794996013\\
14925.6000000001	62.4814916553327\\
14929.8	62.483338803341\\
14932.4	62.4811652435962\\
14934	62.483176086942\\
14937.0000000002	62.4853552563751\\
14940.9	62.4831001940968\\
14948.0000000001	62.4848642971348\\
14951.0000000001	62.4830280046284\\
14958.2999999999	62.4806128997754\\
14959.8	62.4823695345557\\
14961.1000000002	62.4849610993771\\
14969.0000000002	62.4867226486562\\
14971.5999999999	62.4845541922427\\
14975.0999999999	62.4866445857151\\
14979.8	62.4884011241731\\
14984.1000000002	62.4931594612992\\
14985.5000000002	62.4913250053385\\
14986.1999999998	62.4930770103361\\
14987.4999999998	62.4909925538445\\
14988.5000000001	62.4928278825241\\
14990.1999999998	62.4910775634911\\
14993.6000000002	62.4875781161421\\
14997.8000000002	62.4894151847925\\
15000.7000000002	62.4913337955309\\
15003.6000000001	62.4892526510127\\
15007.0000000002	62.4917539031525\\
15007.8999999998	62.4900053304904\\
15015.7000000002	62.4928408742791\\
15019.3000000001	62.4905122707964\\
15026.2000000001	62.4884369405642\\
15026.8000000003	62.4866073508175\\
15029.7	62.4845307322785\\
15030.7999999999	62.4866109148488\\
15033.5000000002	62.4906875266071\\
15039.7999999999	62.492436784819\\
15041.1000000001	62.4896949711459\\
15042.6	62.4914410311979\\
15043.4999999999	62.4883671461618\\
15045.2999999998	62.4901963390804\\
15047.2	62.4882869351977\\
15048.2999999999	62.491028946599\\
15052.9999999999	62.4927755744664\\
15053.9999999998	62.4906171740589\\
15057.7	62.4925287890661\\
15060	62.4942729463948\\
15061.9	62.4917009693268\\
15076.1999999999	62.4934499844126\\
15077.6000000001	62.4916267069911\\
15083.5	62.4956906839216\\
15090.2000000002	62.4937873998529\\
15091.0000000002	62.4957756558501\\
15094.2000000002	62.4931265444572\\
15096.8000000001	62.490974968371\\
15103.4	62.4927996821929\\
15104.3000000002	62.491062207039\\
15107.9999999999	62.4889959690497\\
15119.3000000002	62.487268013281\\
15121.5000000002	62.4847899693154\\
15131.0000000002	62.4871952468756\\
15134.9000000001	62.4909150974562\\
15137.1000000002	62.4890997013979\\
15141.2	62.4873689841691\\
15143.2	62.4896819055292\\
15144.3	62.4871239534085\\
15144.9999999998	62.4888577823851\\
15155.2	62.4870507347265\\
15163.1999999999	62.4890360277776\\
15164.6000000001	62.4911801750117\\
15165.2000000003	62.4893671737453\\
15167.8	62.4911820357465\\
15169.1999999999	62.4893699775204\\
15172.9000000002	62.4873129901799\\
15174.3999999999	62.4897031203664\\
15175.0999999999	62.4874795719332\\
15184.0000000001	62.4850995449187\\
15185.0000000001	62.4869115119426\\
15187.5	62.4838684189734\\
15188.8999999998	62.4820593850813\\
15191.3000000003	62.4794291507037\\
15192	62.4811579702609\\
15192.8999999999	62.4794313170539\\
15195.2000000001	62.4811619382309\\
15196.5000000002	62.4791071687088\\
15202.6	62.4816644411848\\
15207.1000000002	62.4796149192488\\
15212.3000000001	62.4819226420552\\
15216.7999999998	62.4844744987481\\
15220.0000000002	62.4825066852386\\
15225.2999999999	62.4804602834737\\
15230	62.4821898739995\\
15231.0000000001	62.4800572512819\\
15237.3	62.4817882315881\\
15239.2000000001	62.4844973194307\\
15242.3999999999	62.4864687551255\\
15244.3000000001	62.483928524573\\
15256.1000000001	62.4821384093025\\
15262.2000000001	62.4840292747489\\
15270.3	62.4823187342833\\
15273.7000000001	62.48412313897\\
15276.8000000001	62.4858446412558\\
15279.1000000001	62.4836378868004\\
15281.2	62.4868303089397\\
15288.2000000002	62.4889621475246\\
15289.3	62.4910068413411\\
15292.2999999998	62.4892103266982\\
15293.6000000001	62.4910911028724\\
15294.4999999999	62.4893753350856\\
15297.9	62.4911753170349\\
15300.5999999999	62.4892978752606\\
15303.1999999999	62.487175968582\\
15314.8999999998	62.4890630101208\\
15316.5999999998	62.4873504083778\\
15323.6	62.4894770845162\\
15326	62.4875212872159\\
15328.6999999999	62.4895621314128\\
15336.3999999999	62.4914419848075\\
15338.1000000001	62.4897315199958\\
15342.1	62.4916895881947\\
15346.9000000002	62.4942985599791\\
15349.2000000002	62.4921006169663\\
15351.3999999999	62.4955216102661\\
15359.8	62.4925943528278\\
15374.3999999998	62.4904874955283\\
15375.9	62.4921956295526\\
15378.7999999998	62.4901650963333\\
15382.5000000002	62.4939867122593\\
15385.2000000002	62.4960189271577\\
15385.8999999999	62.4938255557\\
15387.6999999998	62.4917142151575\\
15390.1	62.4936647996777\\
15391.6	62.495370881709\\
15393.0999999998	62.4970766312398\\
15395.5999999998	62.4992692764863\\
15397.3999999998	62.5010553661309\\
15399.1000000001	62.4993506156164\\
15399.9999999999	62.5015421977779\\
15408.4999999999	62.5034071881936\\
15410.7	62.5009733433696\\
15413.9	62.4990268586999\\
15415.3	62.501135228408\\
15418.2999999999	62.4993514242723\\
15423.1999999998	62.4976496599301\\
15428.7000000001	62.4993518614539\\
15431.8999999999	62.5012960082944\\
15434.8999999999	62.4995140913508\\
15436.4	62.502510284067\\
15441.1	62.504209517395\\
15443.7	62.5059894585529\\
15446.3	62.5077688004972\\
15446.8999999999	62.5059882177769\\
15453.1000000003	62.5080889395077\\
15455.5	62.5106757421258\\
15456.3000000002	62.5087342460081\\
15458.9999999998	62.51075418362\\
15463.1999999999	62.5086495120705\\
15466.2000000002	62.50686977493\\
15467.0999999999	62.5051722354402\\
15469.6999999998	62.5069490232582\\
15470.3	62.5051711655807\\
15472.5000000002	62.5085635251994\\
15478.8999999998	62.5066218747981\\
15482.6000000002	62.5084772035885\\
15484.8000000001	62.5067000755575\\
15489.7000000002	62.5088768092551\\
15493.0999999999	62.5106498334753\\
15496.8000000003	62.5131477908485\\
15499.1999999999	62.5150813262535\\
15500.6	62.5126607185482\\
15503.6000000001	62.5102394912182\\
15509.3000000002	62.5085432060557\\
15510.6999999999	62.5061247646801\\
15513.8000000001	62.5039480723738\\
15515.7000000001	62.5020946390131\\
15517.2000000001	62.504430538818\\
15520.3999999999	62.5063625527528\\
15521.9000000002	62.5041876046901\\
15523.9	62.5019324916259\\
15525.8000000001	62.5039450209006\\
15527.2	62.5066817798329\\
15529.6999999998	62.508853945318\\
15535.1000000001	62.5070806941655\\
15537.8000000001	62.5052291493703\\
15542.8	62.5031364803222\\
15550.0000000001	62.5057073587951\\
15551.3999999999	62.5039385268302\\
15553.6999999998	62.5062685645951\\
15554.3000000002	62.5045003343106\\
15558.8000000003	62.5069895686713\\
15561.4000000002	62.5036146901006\\
15565.6999999999	62.5004818255406\\
15568.5999999998	62.5029707040408\\
15569.4000000002	62.5010437072482\\
15573.7999999998	62.5032907621084\\
15591.9	62.5051308363263\\
15592.9999999998	62.5032867101474\\
15595.1999999998	62.5053702076908\\
15598	62.5024842769312\\
15600.6000000001	62.504246604319\\
15612.5	62.5059247018434\\
15618.1000000001	62.5078434134535\\
15620.1999999999	62.5052015646306\\
15622.9000000001	62.5033604301351\\
15628.0000000001	62.5079184289837\\
15632.9999999999	62.509674984488\\
15635.6000000002	62.5114321712491\\
15638.9000000001	62.5135878253085\\
15642.0999999999	62.5116671567938\\
15643.5000000001	62.5137436395715\\
15652.6	62.5157321101152\\
15654.0000000002	62.5139739748692\\
15656.2999999998	62.516287269104\\
15659.2000000002	62.5142886335915\\
15662.9000000001	62.5161207942284\\
15665.7999999999	62.5179530062109\\
15668.1	62.5202639741642\\
15668.9999999998	62.5179493397834\\
15672.7	62.5197794905824\\
15673.5000000001	62.5172264189465\\
15675.7999999998	62.519536358359\\
15677.7000000002	62.5215272550996\\
15679.7000000001	62.5243944438067\\
15680.2999999999	62.5226397285784\\
15681.5999999998	62.5244711989134\\
15683.7000000001	62.5218378199161\\
15688	62.5187243834499\\
15693.6	62.5206292971065\\
15694.9999999998	62.5182381762461\\
15703.3000000001	62.5202185513965\\
15706.8000000001	62.5228402803863\\
15708.3999999998	62.5247477480345\\
15710.2	62.5277684067141\\
15715.4999999998	62.5251342614981\\
15716.9000000001	62.5278361010371\\
15717.9	62.525130423718\\
15718.9000000003	62.5268783001463\\
15724.4999999999	62.5287765666535\\
15725.7000000001	62.5265487288405\\
15735.1000000002	62.524149677157\\
15737.4999999999	62.5266876779179\\
15740	62.5288276440429\\
15749.2999999998	62.530636087724\\
15754.7999999999	62.5335609873754\\
15758.9	62.5356938892062\\
15762.5000000003	62.5379061829901\\
15764.9	62.5398033618776\\
15771.7999999998	62.5378045764936\\
15777	62.5406443516236\\
15781.7999999998	62.5425329016151\\
15786.1999999998	62.5453716196956\\
15789.9000000003	62.5427485750475\\
15796	62.5445521362868\\
15802.5999999998	62.5424769185013\\
15806.8000000002	62.5404095679735\\
15811.5000000001	62.5426901768322\\
15815.2000000001	62.5400719556379\\
15817.8000000001	62.5424361008731\\
15822.2000000002	62.5402122321028\\
15829.0999999999	62.5382205038789\\
15831	62.5363998711397\\
15833.1000000001	62.5344213424955\\
15837.7000000002	62.5326749927389\\
15839.5	62.5344074345312\\
15840.8999999998	62.532668392147\\
15843.0000000002	62.5344787320663\\
15849.9	62.5318611987382\\
15852.3999999998	62.5295694685381\\
15860.9	62.5269529033478\\
15862.6999999998	62.5286834606753\\
15865.4999999998	62.5308844292053\\
15883.0000000002	62.5280959006743\\
15883.9999999999	62.5298254229072\\
15885.7000000001	62.5319467700714\\
15889.3999999999	62.5343780483967\\
15894.4000000001	62.5360973921797\\
15908.2999999999	62.5392874204823\\
15910.1000000002	62.541011426632\\
15912.8	62.539197757794\\
15918.4000000001	62.5410685680183\\
15920.9999999998	62.5390205450628\\
15922.6000000002	62.5409007266356\\
15927.4000000001	62.5433997802543\\
15929.4	62.5411971499419\\
15939.9	62.5395232120452\\
15941.2000000001	62.5413234805192\\
15943.1000000003	62.5395152792413\\
15950.6	62.5370673387375\\
15952.6000000002	62.5398835306876\\
15955.8000000001	62.5373686222651\\
15959.3000000002	62.5393185207464\\
15962.1	62.541504303918\\
15964.4999999998	62.5396189068314\\
15967.5	62.537889225682\\
15969.5	62.5350666265905\\
15970.9000000001	62.5377246258844\\
15974.1999999999	62.5404556068185\\
15974.9000000003	62.5383411580595\\
15985.7999999999	62.5407390262669\\
15988.0999999999	62.5430004628413\\
15989.5999999998	62.5408856951663\\
15992.2999999998	62.5390810635052\\
15993.4000000002	62.5410322943696\\
16002.0000000001	62.539291717962\\
16003.9000000001	62.5362409397651\\
16005.9000000001	62.5396726227665\\
16014.5999999998	62.5419146159466\\
16015.1000000002	62.5399620360657\\
16017.0000000001	62.5419083354665\\
16018.0000000002	62.5398767644103\\
16020.5999999999	62.541586822049\\
16023.8000000001	62.5397063136939\\
16028.8999999998	62.5379000561482\\
16029.8999999999	62.5396132252028\\
16030.6999999999	62.5371160516007\\
16033.9000000002	62.5352376200574\\
16034.9000000001	62.5375740567509\\
16036.0999999998	62.5353886831045\\
16036.6999999998	62.5336725531278\\
16038.5000000001	62.5353833875775\\
16039.7999999998	62.5371729250182\\
16040.4	62.5354571241545\\
16046.3999999999	62.5376250272645\\
16050.8999999998	62.5394056445081\\
16052.2999999999	62.5376890683013\\
16053.6000000002	62.5394768807191\\
16055.9999999999	62.5413394286284\\
16058.4	62.5394650807983\\
16060.4999999999	62.5375141650997\\
16066.7000000002	62.5401448950631\\
16070.8000000002	62.5378790235768\\
16071.8000000003	62.5395877276489\\
16074.4999999998	62.5377925422717\\
16075.7999999999	62.5401999265982\\
16082	62.5422053090082\\
16083.8999999998	62.5441432479483\\
16089.1000000001	62.5475474231161\\
16092.0999999998	62.5458296566038\\
16094.5000000002	62.5476868017845\\
16097.1000000002	62.5456601148026\\
16101.8000000002	62.5435507610903\\
16106.1	62.541754107114\\
16108.0000000002	62.5399643657539\\
16110	62.5421319544882\\
16111.4000000002	62.5391800887565\\
16113.5	62.537235627048\\
16117.9999999998	62.5352864171335\\
16123.1	62.5334921107472\\
16124.0999999998	62.5351955445852\\
16128.7000000002	62.5328604731908\\
16134.3	62.5347084490282\\
16138.5999999999	62.5329177690892\\
16141.6000000001	62.5349250698501\\
16155.1000000001	62.5328067742894\\
16160.3999999998	62.534575044089\\
16168.2000000001	62.5378054588299\\
16170.0000000002	62.5339360919228\\
16175.2000000001	62.5367072017211\\
16178.6999999999	62.5343041511113\\
16180.1000000003	62.5363098107564\\
16184.2999999999	62.5386174340723\\
16187.5000000003	62.5410808272999\\
16189.6999999998	62.539376644554\\
16193.1	62.5429192500556\\
16193.9000000002	62.5410645918241\\
16197.0000000001	62.5383556315637\\
16198.7	62.5404350939576\\
16200	62.5385028487479\\
16207.1999999999	62.536634726327\\
16210.2000000002	62.5349314941735\\
16219.7	62.532830244516\\
16221.9	62.5311305634324\\
16223.9	62.5332840236686\\
16228.9	62.5306550003081\\
16233.8000000001	62.5333407252724\\
16236	62.5316424510812\\
16238.6999999998	62.5298667389216\\
16244.0999999998	62.5281638984992\\
16246.5000000001	62.5300062782367\\
16249.7	62.5318465458036\\
16257.6000000002	62.5346758766615\\
16260.9999999998	62.536974743406\\
16265.5000000002	62.5350432815267\\
16269.0000000001	62.5369565618258\\
16269.5	62.5350346658799\\
16271	62.5329572063352\\
16273.6999999998	62.535486487483\\
16275.3000000001	62.5336397262126\\
16276.2	62.5357114331881\\
16281.4999999998	62.5374656053459\\
16287.6	62.5355329481756\\
16296.0000000002	62.5376623854787\\
16299.8000000001	62.5396474824999\\
16300.8999999998	62.5378811115882\\
16303.3	62.5403290111265\\
16305.7	62.5372566816716\\
16309.3000000002	62.5393944596368\\
16318.8000000001	62.5373033721636\\
16320.1000000001	62.5390620212988\\
16321.6999999998	62.5408962246811\\
16325.7000000002	62.5384360950153\\
16329.0000000002	62.5362083642087\\
16331.9999999999	62.5381916593702\\
16339	62.5364922180536\\
16347.0999999999	62.5385387099932\\
16348.1999999999	62.5361658398733\\
16354.1000000002	62.5343948343545\\
16360.3999999998	62.5317074661532\\
16365.2000000002	62.5298650192786\\
16368.5999999998	62.5321497736534\\
16369.9000000001	62.5302382406842\\
16375.2	62.5319841468553\\
16388.5999999999	62.5339410691513\\
16390.1999999999	62.5321074049895\\
16393.0000000002	62.5342369655526\\
16401.7999999998	62.5360476530158\\
16404.7000000001	62.537793816444\\
16408.3	62.5399185782892\\
16418.8000000001	62.5376852286085\\
16420.0999999998	62.5357791013508\\
16429.7000000002	62.5339322450669\\
16430.5	62.5357564544204\\
16440.4000000002	62.5382439706822\\
16444.1000000002	62.5399837024604\\
16444.8999999999	62.5381574946792\\
16454.3000000001	62.5364644107351\\
16458.9	62.5384288231363\\
16460.8000000003	62.5409303258024\\
16470.9000000001	62.5390079533726\\
16472.1	62.5411299037166\\
16478.7999999999	62.5393685258118\\
16480.4999999999	62.5420191012463\\
16484.4999999999	62.5401890249081\\
16497.3000000001	62.5419763114188\\
16502.3000000003	62.5442359899166\\
16507.5000000001	62.5463422908236\\
16515.6999999998	62.5443514695019\\
16517.8999999998	62.5463131129677\\
16518.8999999998	62.5443428778982\\
16519.9999999999	62.5462315603417\\
16521.3000000002	62.5437311607975\\
16525.0000000002	62.5460662870421\\
16527.3999999999	62.544244441083\\
16528.6999999998	62.5459803494507\\
16533.3999999999	62.5433211358757\\
16535.4999999999	62.5450543070708\\
16540.5000000002	62.5473078364751\\
16542.5000000002	62.5451863673183\\
16550.2999999998	62.5471287703016\\
16551.6	62.5446328775896\\
16553.6000000002	62.5425131541589\\
16563.5000000001	62.5407520104325\\
16566.6000000002	62.5387071655791\\
16568.0000000001	62.5364405091712\\
16570.4000000003	62.5388491596512\\
16577.5	62.5428288775215\\
16578.4999999999	62.5408659356037\\
16591.2000000001	62.5430195343343\\
16593.1999999999	62.5409050640921\\
16596.3999999999	62.5433073238333\\
16599.8999999998	62.5451807228916\\
16601.5	62.5433693138011\\
16608.8000000002	62.5453822950346\\
16610	62.5432718647088\\
16611.0999999998	62.5451502600655\\
16613.7000000002	62.5480022631788\\
16616.2	62.5458134482406\\
16617.0000000003	62.5476166118035\\
16620.4999999999	62.5452751404883\\
16623.1000000002	62.5433129602002\\
16624.1000000002	62.545566102429\\
16626.2000000001	62.543079338157\\
16629.7000000001	62.5407401171391\\
16632.4	62.5426123553284\\
16641.5000000002	62.5408614556293\\
16642.8000000002	62.5437874408907\\
16648	62.5416714219641\\
16652.6	62.5442120496976\\
16659.9999999998	62.5464432986597\\
16661.9000000003	62.5447125195055\\
16664.2999999998	62.5429058351936\\
16665.8999999999	62.5447017880715\\
16674.1999999999	62.5465536784153\\
16686.1	62.5444978485215\\
16688.2000000002	62.5426196796558\\
16692.5	62.544480787894\\
16695.2000000001	62.5427515528322\\
16696.9999999999	62.5396026854963\\
16698.3	62.5413213241987\\
16704.1999999998	62.5395856156798\\
16707.6999999998	62.5378565699853\\
16709.7999999999	62.5395723493259\\
16712.4999999998	62.5414358029271\\
16713.5999999998	62.5397129301112\\
16715.7000000002	62.5366419794446\\
16717.3	62.534245755919\\
16718.4000000001	62.5361126895355\\
16721.3000000002	62.5342375638403\\
16722.1000000001	62.5324419035773\\
16729.0000000002	62.5341470850195\\
16738.7000000002	62.5361435706263\\
16741.4999999998	62.538228126344\\
16745.0000000002	62.5400863536199\\
16747.5000000001	62.5379158804844\\
16755.2000000002	62.5360333745143\\
16756.8999999998	62.53804380259\\
16759.2999999999	62.5362483143788\\
16760.2000000002	62.538260055011\\
16763.5999999999	62.5410857984812\\
16767.1	62.5393625650079\\
16774.2000000001	62.5373338976887\\
16782.1000000002	62.5353052639106\\
16784.5000000003	62.537683352597\\
16786.5999999999	62.5399870135286\\
16787.3999999998	62.5381980640358\\
16793.8000000002	62.5399698700123\\
16808.7000000001	62.5416448526962\\
16814.1999999999	62.544381865436\\
16815.5	62.5425200409144\\
16819.1000000002	62.5445918949772\\
16823.4	62.5428715784468\\
16830.3999999998	62.5447847657527\\
16832.4000000001	62.5474528442002\\
16836.6999999999	62.5451392188539\\
16839.8999999999	62.5433491686461\\
16842.8999999998	62.5458647509351\\
16846.9000000002	62.5476345936962\\
16848.1999999999	62.5493373218663\\
16849.9000000001	62.5513353115727\\
16853.0999999999	62.5548857190326\\
16853.9000000003	62.5531031209209\\
16855.1999999999	62.554804720177\\
16857.5999999998	62.5530173155294\\
16863.1000000001	62.5557426822904\\
16865.0999999998	62.5536607926381\\
16866.8	62.5515062044596\\
16869.8000000001	62.5540163249338\\
16882.2000000002	62.551903472868\\
16892.4	62.5496522125203\\
16893.8000000002	62.5521637987676\\
16895.7999999999	62.5542291325114\\
16896.6	62.5524510703274\\
16900.1999999999	62.5503689283622\\
16905.4000000003	62.5471000561947\\
16907.6000000001	62.5496075752468\\
16908.5000000002	62.5468696403014\\
16911.7	62.5450868624274\\
16913.0999999997	62.5475959605515\\
16914.7000000001	62.545817863646\\
16919.7000000001	62.5486116857173\\
16920.7999999998	62.5469094433511\\
16925.1999999998	62.5489651586678\\
16926.0999999999	62.5468209048694\\
16937.2999999998	62.5485611723169\\
16938.8000000001	62.5459740597087\\
16940.3999999999	62.5477406215873\\
16943.0999999999	62.5460361679022\\
16944	62.5480255664214\\
16944.7000000002	62.5460318209716\\
16950.5999999998	62.5484493265765\\
16952.1999999998	62.5466750824372\\
16953.3000000002	62.544976228957\\
16956.3999999998	62.5417981305104\\
16958.2999999999	62.5395084441928\\
16961.6000000002	62.5414905345573\\
16964	62.5432531050866\\
16969.6999999998	62.545816686113\\
16970.7999999999	62.5441196400898\\
16977.4999999999	62.5459428894544\\
16988.1999999999	62.5442216113443\\
16989.8000000001	62.5424516918875\\
16990.7999999999	62.544656257173\\
16997.7999999998	62.5465498679249\\
17000.9000000003	62.5445562025763\\
17005.9999999998	62.5463804164388\\
17008.4000000001	62.5446100479172\\
17011.6	62.5428381643222\\
17014.3	62.5446680458905\\
17030.1000000001	62.5424246338857\\
17031.6000000002	62.5445492816337\\
17032.2000000002	62.5423460131632\\
17034.1	62.5447628887767\\
17036.7999999999	62.547763971145\\
17040.3999999999	62.5456999501188\\
17051.4999999999	62.5436909146356\\
17053.4000000002	62.5455185152608\\
17058.0999999999	62.5476310513419\\
17076.3999999998	62.5502884080462\\
17078.8000000003	62.5479392700935\\
17079.7999999999	62.5501320265341\\
17080.7999999999	62.5482263815139\\
17083.9999999998	62.5464613295403\\
17085.0999999999	62.5482874066443\\
17087.5000000003	62.546524965472\\
17089.6999999998	62.5484206953856\\
17092.9	62.5513368045399\\
17094.1999999998	62.5495048056955\\
17101.9	62.5476552450006\\
17104.2999999998	62.5494024929258\\
17106.2999999998	62.5514427348829\\
17108.6999999999	62.5496820349762\\
17112.8000000002	62.5522266827949\\
17117.3000000001	62.5503873251779\\
17125.9999999999	62.5483910522536\\
17130.3999999999	62.5504217623537\\
17135.1999999998	62.5480732756357\\
17140.2999999999	62.5463816480362\\
17142.8999999999	62.5485620953159\\
17145.2999999998	62.550305038086\\
17146.3999999998	62.5521243402444\\
17148.8	62.5503676620658\\
17159.3000000002	62.5528864645617\\
17160.0999999999	62.5511357676484\\
17166.8000000001	62.5529361736831\\
17169	62.5507452341707\\
17174.2	62.5527677983964\\
17175.2999999998	62.5505082851055\\
17180.9	62.5522379372563\\
17192.7999999999	62.5502387613492\\
17194.0999999999	62.5484174896186\\
17198.7999999998	62.5510933838792\\
17199.4999999997	62.5491290495128\\
17200.6000000001	62.5509426941927\\
17202.3999999998	62.5490481034733\\
17204.2000000001	62.5471539091971\\
17205.5	62.5494025201097\\
17207.5000000002	62.5520119017178\\
17213.1999999998	62.5493078026875\\
17216.4999999999	62.5471928255289\\
17220.7999999998	62.5495763868323\\
17227.4999999998	62.5478882723072\\
17231	62.5456297044298\\
17243.0999999998	62.5475549781943\\
17262.7000000001	62.5495284658341\\
17264.8000000001	62.5471332008874\\
17273.9	62.5454440199143\\
17281.9	62.5488948038421\\
17287.9	62.5509023600185\\
17290.5999999998	62.5486533223062\\
17294.9000000003	62.5504481063891\\
17301.6000000003	62.5481889062925\\
17304	62.5504938136049\\
17315.9000000002	62.5485100485101\\
17317.0000000002	62.5503115417709\\
17319.6000000001	62.5478501359723\\
17328.9000000003	62.5500605920711\\
17330.8	62.5472422090024\\
17336.3000000001	62.5493181975497\\
17344.4999999999	62.5514569376059\\
17346.9	62.5531792240733\\
17349.1999999998	62.551226850651\\
17350.6999999998	62.5533116628628\\
17354.1000000001	62.5560383077296\\
17363.4999999998	62.5584556198023\\
17364.6999999999	62.5564360084769\\
17367.0999999999	62.5547008153301\\
17368.3000000002	62.5567121899542\\
17370.9000000003	62.5548327672558\\
17378.7000000002	62.5526503556057\\
17385.9000000003	62.5543540779938\\
17399.9000000001	62.5568965517241\\
17405.2000000002	62.5591055598008\\
17407.5000000001	62.5571589420713\\
17409.9000000002	62.5594485927628\\
17413.9000000002	62.5577121855978\\
17417.6000000003	62.5599246743255\\
17420.4000000001	62.5619241698\\
17426.2000000001	62.5646293246415\\
17430.0000000001	62.5664798251301\\
17430.7999999999	62.5647556924771\\
17432.7000000002	62.566541232619\\
17435.9000000001	62.5688231245699\\
17441.1000000001	62.5708093479806\\
17447.5	62.5690639400261\\
17448.9999999999	62.5671238058123\\
17451.6999999998	62.5689040672022\\
17455.6	62.5669552066087\\
17464.5000000002	62.5688535666434\\
17465.6999999998	62.5708527522358\\
17466.5000000001	62.5691319432517\\
17476.9999999997	62.571021508145\\
17484.1999999999	62.5732800283683\\
17489.7999999997	62.5715412895443\\
17498.9999999999	62.5695035744696\\
17500.6000000002	62.5677830029656\\
17503.0000000002	62.5660597265627\\
17503.9000000001	62.5679844606947\\
17505.5	62.5662645096426\\
17511.1000000002	62.567956507835\\
17511.8999999998	62.5662402923709\\
17514.0000000002	62.5684448530042\\
17515.3999999999	62.5662984213982\\
17518.8999999997	62.5680689537074\\
17523.6	62.5701193241153\\
17525.7999999999	62.5719649204891\\
17529.8999999998	62.5698802053622\\
17532.6	62.5659481996498\\
17535.9000000001	62.5678604014599\\
17543.1	62.5661224862055\\
17544.1999999998	62.5639096458679\\
17546.1	62.5656837377894\\
17547.3000000001	62.5636846484379\\
17553.3999999997	62.5664397413621\\
17562.4000000001	62.5645551601423\\
17564.0999999998	62.5624850548274\\
17566.1000000002	62.5604854777926\\
17570.4999999998	62.5630314274982\\
17575.5999999999	62.5647911605228\\
17581.0999999999	62.5622824380588\\
17588.0999999998	62.5641054798103\\
17595.4999999999	62.5622314669577\\
17597.1	62.5639306253268\\
17598.7999999998	62.5618646619959\\
17601.2000000003	62.5635606460886\\
17604.6	62.5656784835868\\
17606.2000000002	62.563968579429\\
17617.8999999999	62.5621523441934\\
17620.1999999998	62.5642015175678\\
17621.6000000001	62.5620683588984\\
17627.6000000001	62.5640327439201\\
17639.2999999999	62.5616517568625\\
17643.9000000001	62.5634776694627\\
17662.1999999999	62.5660304716826\\
17663.2999999998	62.5638325577182\\
17666.4000000001	62.5658732629553\\
17667.3	62.5638181056635\\
17669.7	62.5609797507612\\
17676.7999999998	62.55904598657\\
17682.3999999999	62.5618549413262\\
17685.1000000002	62.5636125121571\\
17687.1000000001	62.5661495318649\\
17691.8000000003	62.5687461493678\\
17695.1	62.5706406257064\\
17697.1999999998	62.5688664372531\\
17708.4000000002	62.5671287799644\\
17713.3999999999	62.5652750726847\\
17716.2999999999	62.5629360366666\\
17719.2000000001	62.5611621226572\\
17721.6000000003	62.5594609997912\\
17726.0999999999	62.5576829777392\\
17727.4	62.5598646171203\\
17729.9000000002	62.5617597292724\\
17734.9999999997	62.5635040118184\\
17736.5999999998	62.5618068750106\\
17739.7999999998	62.5601046229122\\
17741.6999999998	62.5618595632912\\
17746.2	62.5600829468678\\
17748.7999999997	62.5582430460479\\
17763.0999999999	62.5562961628535\\
17765.8000000002	62.5586094709528\\
17782.8000000002	62.560662209201\\
17799.5000000001	62.5632036674981\\
17802.9	62.5652979834859\\
17819.6999999998	62.5635529018283\\
17823.0999999997	62.561156245792\\
17826.5999999998	62.5589705329646\\
17843.6999999999	62.5606653291339\\
17844.2	62.5589123697764\\
17849.9	62.5613445378151\\
17850.9000000003	62.5595204750434\\
17852.8000000001	62.5618246895462\\
17864.8999999998	62.5636719843269\\
17867.6999999997	62.5617031755448\\
17873.4	62.5646907432792\\
17878.1000000003	62.5627859627927\\
17881.9999999999	62.5653586547441\\
17898.2999999998	62.5636928440531\\
17901.4000000001	62.5617964975002\\
17908.5	62.563796164971\\
17912.0000000001	62.5660866118434\\
17918.1999999998	62.5689937103408\\
17925.5999999998	62.5671521893148\\
17933.1999999999	62.5690754072034\\
17939.4999999999	62.5671698365627\\
17954.4999999998	62.5694808015773\\
17955.8000000003	62.567735396165\\
17957.1000000002	62.5659902434678\\
17960.4999999998	62.5641682349142\\
17965.5000000001	62.5662376987131\\
17968.2999999998	62.5642795129227\\
17974.7000000001	62.5625876226717\\
17976.6999999998	62.5606337056651\\
17988.9000000002	62.5637889821558\\
17993.1999999998	62.566066257996\\
17994.5000000002	62.5643248530115\\
18001.0999999998	62.5613847965691\\
18007.6000000002	62.563236837575\\
18013.5999999998	62.5657138733297\\
18017.3000000003	62.5639659440319\\
18019.1000000002	62.5621559225715\\
18019.9999999998	62.5640257268273\\
18025.4	62.5663643172173\\
18030.1000000002	62.5644751583454\\
18039.9999999999	62.5667263485236\\
18054.2999999998	62.5642502658632\\
18058.4000000002	62.5622283135366\\
18068.5000000002	62.5604640093865\\
18072.6999999998	62.5625248992962\\
18074.0999999999	62.5598920007525\\
18077.9000000002	62.5616771766788\\
18080.8999999999	62.5595929428682\\
18085.3999999998	62.5578502114954\\
18088.3999999997	62.5596373386406\\
18090.2000000001	62.5578348617768\\
18093.4	62.5600353718186\\
18100.6999999998	62.5618757182003\\
18101.6000000003	62.5598700674522\\
18106.2000000003	62.5577837548257\\
18110.8000000003	62.559563577735\\
18115.6999999999	62.5575464511642\\
18117.1	62.555472148014\\
18122.8000000003	62.5573169856921\\
18123.7000000001	62.5553140069963\\
18126.4999999999	62.5572363267243\\
18127.5000000001	62.5554403230433\\
18132.4	62.5578381359438\\
18134.4000000002	62.5597617800325\\
18140.0000000002	62.5575382715641\\
18146.8999999998	62.5546922356312\\
18148.8000000001	62.5564083773672\\
18156.2000000001	62.5545953746083\\
18158.7000000001	62.5564464612199\\
18162.5999999998	62.5545761368079\\
18163.6000000002	62.5527838491056\\
18169.1999999998	62.5500156857997\\
18170.0000000002	62.5478120648758\\
18174.1	62.5496583068306\\
18176.5999999998	62.5520584044409\\
18178.7000000003	62.550333355337\\
18187.4	62.5523024054983\\
18190.1000000002	62.5540126001913\\
18199.9000000003	62.5521978021978\\
18210.2000000001	62.5547080498399\\
18218.9999999999	62.5579748725239\\
18227.2999999999	62.5596629250469\\
18232.2000000003	62.5614980008008\\
18232.9000000003	62.5596446004497\\
18241.5999999998	62.561603359336\\
18251.8000000002	62.563349569086\\
18257.8000000002	62.5657934373614\\
18261.9000000002	62.5632460847662\\
18265.6	62.5609749421046\\
18270.3	62.5591120063053\\
18271.8000000001	62.5572600550572\\
18277.7999999998	62.5553263777568\\
18280.9000000002	62.5572999288879\\
18282.0000000002	62.5551769216884\\
18285.9000000002	62.5571475445696\\
18289.7	62.5589126179619\\
18298.6000000001	62.5569029493899\\
18300.8000000003	62.5586719778809\\
18304.4000000001	62.5605725368079\\
18306.6000000003	62.5623405638373\\
18307.5999999998	62.5605619493437\\
18309.7999999999	62.5585066002545\\
18315.8	62.5614902898574\\
18318.7	62.563595868725\\
18320.4000000003	62.5605196364728\\
18327.9000000003	62.5627455259712\\
18332.8999999998	62.5598647248132\\
18339.0999999998	62.5616166463095\\
18341.2999999999	62.5633812031797\\
18344.3999999998	62.5615307040257\\
18350.4	62.5634178905207\\
18351.8999999998	62.5615736704446\\
18353	62.5632726896274\\
18353.6999999997	62.5614314201964\\
18355.6000000002	62.5631275298681\\
18360.4999999999	62.5611363463068\\
18364.9000000002	62.5592158998094\\
18369.5999999998	62.5573634844336\\
18374.2	62.5591178983689\\
18375.5000000002	62.5574130912732\\
18376.7000000001	62.55550476688\\
18378.3	62.5587646367475\\
18380.9000000002	62.5569881943311\\
18385.6000000002	62.5551379604802\\
18392.3000000002	62.5573606489637\\
18393.2	62.5553870159243\\
18405.6000000003	62.5572512862863\\
18409.4000000003	62.555202476982\\
18413.6	62.5577694868495\\
18417.2	62.5553148398517\\
18420.2000000003	62.5532700336042\\
18422.1	62.5555036857704\\
18428.1999999999	62.5537895519391\\
18434.6	62.5575680645739\\
18436.9	62.5557303248902\\
18447.8999999998	62.5574588031223\\
18449.9000000003	62.5555555555556\\
18454.5999999998	62.5537126043772\\
18460.0999999998	62.5518683437883\\
18465.7000000002	62.5545603223256\\
18467.1	62.556316063074\\
18468.9000000002	62.5545508690238\\
18470.6000000002	62.5574558625282\\
18472.7	62.5557576544974\\
18474.1000000001	62.5575126392482\\
18483.4999999999	62.5592417061611\\
18485.7999999998	62.5574086195425\\
18489.3999999998	62.5592904080694\\
18490.1999999999	62.5571245463838\\
18504.6000000002	62.5554588834188\\
18506.1000000002	62.5536306751251\\
18510.2999999998	62.5556443945026\\
18524.1	62.5576273199382\\
18525.8999999999	62.5558674295585\\
18532.6000000002	62.5532167466154\\
18536.6999999997	62.5512494065858\\
18549.2000000003	62.5533038982603\\
18549.8000000001	62.5512805998954\\
18558.3999999998	62.5530080556079\\
18566.1	62.5513029052795\\
18574.5000000001	62.5493954109375\\
18576.9999999999	62.55174381362\\
18586.9999999999	62.5546750165437\\
18591.1000000003	62.557016222729\\
18597.1999999997	62.5553171696967\\
18598.2999999997	62.5575318306951\\
18602.6000000001	62.5554355012982\\
18607.8000000002	62.5535390882367\\
18612.9999999998	62.5554045269192\\
18623.1999999999	62.5576562692971\\
18632.1999999997	62.5601777558326\\
18635.4999999997	62.5630513640559\\
18639.3000000001	62.5604901445326\\
18643.9000000002	62.5622184080669\\
18648.8	62.5640118183914\\
18650.3999999999	62.5618616122892\\
18663.1999999999	62.5639624289381\\
18668.9999999998	62.5659512242154\\
18672.1999999999	62.5637977110479\\
18674.7000000001	62.5618480519202\\
18680.1999999999	62.5637703891265\\
18681.4999999999	62.5615578965399\\
18683.3000000002	62.5635590952396\\
18686.3999999998	62.5617424343778\\
18698	62.5641107920056\\
18701.8999999999	62.5622928029088\\
18703.6999999999	62.5642917482009\\
18704.7000000001	62.5625507891023\\
18715.7999999999	62.5607104120026\\
18717.5000000002	62.5587682181476\\
18722.6999999998	62.5568825175722\\
18727.9000000001	62.5592695429304\\
18732.0999999998	62.557521273529\\
18736.7	62.5555057427095\\
18744.7999999998	62.5530144199222\\
18745.9999999999	62.5506105269896\\
18758.5000000001	62.5489108995341\\
18766.0999999999	62.5534205113449\\
18774.7000000002	62.5551270852419\\
18785	62.5570265795764\\
18792.0999999998	62.5546769404327\\
18795.8999999999	62.5563949776548\\
18808	62.5581531361488\\
18814.8000000003	62.5557403972384\\
18815.4999999999	62.5539445991624\\
18825.9999999999	62.5557072362306\\
18830.7999999999	62.5578172047008\\
18831.7000000002	62.5558895060483\\
18832.7000000003	62.5578777452105\\
18845.1000000003	62.555982425233\\
18851.6000000003	62.5540402191845\\
18852.8	62.5521803011738\\
18857.1000000002	62.5543558958912\\
18868.1999999998	62.5562451307219\\
18892.6999999998	62.5587525406504\\
18926.5999999999	62.5571282896649\\
18928.1999999999	62.5592366984885\\
18929.5999999997	62.5567230331173\\
18935.3999999998	62.5549893058013\\
18938.1000000002	62.5529353370436\\
18942.3000000002	62.5512078722865\\
18964.0000000001	62.5492377703134\\
18970.8000000002	62.5510650522643\\
18977.1999999998	62.5531556122315\\
18977.9	62.5513752766361\\
18980.4999999998	62.5533439406552\\
18983.8999999998	62.5558364938896\\
18989.4	62.5540430237763\\
18992.4000000003	62.5520600236936\\
18994.9999999998	62.5545535427558\\
18998.7000000003	62.5565825210013\\
19006.1	62.5585335311635\\
19009.4000000001	62.5608248507325\\
19011.3999999999	62.5589774610104\\
19014.5999999998	62.5568639000352\\
19016.5999999997	62.5539657248629\\
19018.7000000001	62.5565230193283\\
19020.1	62.5545472707963\\
19032.1	62.5576654301657\\
19035.8999999997	62.5593612103383\\
19039.8999999998	62.5614495798319\\
19044.8999999999	62.5634024678393\\
19046.1	62.5610357971669\\
19049.5000000002	62.5593188308416\\
19083.1999999999	62.5615066576536\\
19087.1999999998	62.5593981338377\\
19105.0999999998	62.5573142390553\\
19107.3000000003	62.5605786239886\\
19109.1000000001	62.5625353232998\\
19111.6	62.5606303991796\\
19114	62.5585300903521\\
19117.4000000002	62.560481234471\\
19119.6000000001	62.5585129473789\\
19123.1999999998	62.5603321602443\\
19125.9999999999	62.5621532879155\\
19128.0999999998	62.5641722692151\\
19131.1000000002	62.5622020573723\\
19133.6999999999	62.5641534875456\\
19137.8999999998	62.562441216428\\
19140.9999999998	62.5601454461865\\
19144.4000000001	62.5584371490506\\
19146.7000000001	62.5566674326781\\
19149.3	62.5591402341588\\
19157.2000000001	62.5615300694774\\
19169.0999999999	62.5633829267784\\
19178.1000000001	62.5611371244434\\
19186.1999999998	62.5587007395902\\
19188.5000000001	62.5569348467319\\
19189.9999999999	62.5546505750361\\
19192.7000000002	62.5567921303822\\
19195.0000000003	62.5550270641987\\
19197.7999999998	62.5531959224707\\
19206.0999999998	62.5511553560829\\
19210.2000000001	62.5529013081524\\
19212.5000000002	62.5547817578048\\
19215.9999999999	62.5527552416984\\
19220.9999999998	62.551050668276\\
19226.6999999997	62.5491501445898\\
19228	62.5470015238115\\
19231.1999999999	62.5490736455674\\
19237.1000000003	62.5470442683966\\
19237.9999999999	62.5487963988128\\
19244.6000000002	62.5507282524539\\
19260.0000000001	62.5490002647961\\
19274.7999999997	62.5471468075061\\
19278.8999999997	62.5452564967063\\
19283.8000000002	62.5469951617671\\
19303.3000000003	62.5449402695898\\
19305.9	62.5468766186678\\
19325.5000000001	62.5486401457135\\
19337.7000000003	62.5469288129984\\
19349.0999999997	62.5447046906332\\
19352.6999999998	62.546504898516\\
19354.9000000003	62.5445621286489\\
19361.6999999999	62.5427387949468\\
19369.6999999997	62.5447862135902\\
19376.6999999999	62.5423186491062\\
19383.3000000003	62.5406275472827\\
19397.7999999999	62.5387284190557\\
19401.0999999999	62.5414922788281\\
19402.6	62.5397496224752\\
19406.0999999999	62.5377456689099\\
19412	62.5398591600084\\
19419.8999999998	62.538105046344\\
19430.3000000003	62.5411725955204\\
19434.3999999999	62.5392986698912\\
19441.9000000003	62.537290402222\\
19443.2000000003	62.5351663555055\\
19444.6	62.5373495091207\\
19448.7	62.5349635967258\\
19457.8999999999	62.5367458114914\\
19473.4000000002	62.5393483451871\\
19478.5	62.5373486800899\\
19482.6999999998	62.5392654033301\\
19484.6999999999	62.5410576449335\\
19486.4	62.542786031355\\
19489.5	62.5410475330433\\
19496.2999999997	62.5428284196057\\
19504.2999999998	62.5448616722381\\
19509.1000000001	62.5469009492957\\
19510.3999999997	62.5447835780733\\
19524.5999999998	62.5428303635907\\
19531.1999999999	62.5447358854761\\
19533.1999999998	62.5429394930708\\
19535.5	62.540183050431\\
19538.5	62.5382576028989\\
19547.8000000002	62.5402217117951\\
19549.7000000003	62.5382356852756\\
19562.0999999999	62.5364222837922\\
19564.6000000003	62.5381426753285\\
19566.8999999999	62.5399908008381\\
19586.8	62.5418008975387\\
19588.2999999997	62.5400747381103\\
19591.3	62.5376440683157\\
19595.6000000003	62.5356583332058\\
19607.4	62.533724340176\\
19624	62.5317848971418\\
19626.5000000002	62.5335004534662\\
19629.7000000003	62.5314572741444\\
19634.8999999998	62.533231474408\\
19643.8000000003	62.5349345089315\\
19644.9999999999	62.5331507602405\\
19646.5999999998	62.535184025816\\
19653.1999999999	62.5370802867712\\
19654.7000000001	62.539430571667\\
19657.2999999998	62.5413330348876\\
19661.7000000002	62.5395436836912\\
19666.2999999998	62.5355937029655\\
19669.5999999998	62.5373035684327\\
19674.4999999999	62.535451800799\\
19675.6999999998	62.5336708037284\\
19679.3000000003	62.5359512993282\\
19681.3000000002	62.5382340687146\\
19685.2000000002	62.5360040233067\\
19693.8000000001	62.5386541010161\\
19698.7000000003	62.5403577882917\\
19701.2999999998	62.5422558802928\\
19709.8	62.5447110335415\\
19712.2000000001	62.5426764000142\\
19713.9	62.5398194176727\\
19717.8000000003	62.5380998990765\\
19720.7000000003	62.5400592268062\\
19723	62.5418925016858\\
19738.1	62.540150570974\\
19741.4000000002	62.5438796444039\\
19754.0999999998	62.5421429366919\\
19760.9000000002	62.5449116947523\\
19767.9999999997	62.5431882679671\\
19771.3	62.5458996328029\\
19775.3999999998	62.5435513640616\\
19782.6000000001	62.5455574820424\\
19783.8000000003	62.5437856034452\\
19787.2	62.5456732348527\\
19793.1	62.5477436695431\\
19796.7999999997	62.5496921235143\\
19806.1000000001	62.5516252486595\\
19826.2999999998	62.5499334221039\\
19827.5999999999	62.5518844848369\\
19828.7000000003	62.5499273783588\\
19830.4000000001	62.551625022062\\
19836.6000000003	62.5537513800178\\
19840.2	62.5555057131193\\
19846.2000000003	62.5582602298665\\
19860.0000000001	62.5565833001848\\
19868.5000000003	62.5544829530012\\
19871.3	62.5562366013467\\
19873.3	62.5579920899293\\
19881.3000000002	62.5559568239661\\
19886.8	62.5542442512408\\
19889.1	62.5525410775697\\
19891.3	62.5556773278905\\
19893.1000000002	62.5575573562825\\
19899.9	62.5557788944724\\
19905	62.5538178657731\\
19917.3999999999	62.5555416091377\\
19922.7999999999	62.5576597784459\\
19928.0000000003	62.5594010467631\\
19940.2000000002	62.5572333415245\\
19943.5000000001	62.5549048316252\\
19944.7000000002	62.5566563715856\\
19946.9999999998	62.5549578635491\\
19955.5000000003	62.5528673655515\\
19957.9999999999	62.5505433883987\\
19967	62.5478912811575\\
19971.4999999999	62.5498207454586\\
19979.4000000003	62.5516154057909\\
19989.5999999998	62.5497131022477\\
19999.4999999999	62.5517510350207\\
20005.7	62.5498605404433\\
20007.2999999997	62.5478572928017\\
20010.7000000002	62.550222879645\\
20018.5000000003	62.5483300530507\\
20019.8000000001	62.5507619918181\\
20027.9999999999	62.5481198915524\\
20030.3	62.5504233564981\\
20031.4999999998	62.5486730965075\\
20035.5999999999	62.5468538658495\\
20043.2000000001	62.5485823192788\\
20045.2000000002	62.5468314268183\\
20063.8000000003	62.5451681876405\\
20068.6000000001	62.5471505379023\\
20069.8000000001	62.5454038136712\\
20073.2	62.5432788828942\\
20074.7000000002	62.5450813955805\\
20088.6999999999	62.5433077137509\\
20100.9000000002	62.5451470076116\\
20103.4000000002	62.5433382246872\\
20111.1000000002	62.5452484187915\\
20115.9999999998	62.5474122717624\\
20119.6000000002	62.5491433768893\\
20121.1999999998	62.5471515259948\\
20124.7999999999	62.5454039523178\\
20126.8999999997	62.5473244894917\\
20127.9999999998	62.5453967339192\\
20132.3999999999	62.5486154228238\\
20137.8999999999	62.5504022246499\\
20141.2000000001	62.5485941821034\\
20144.9999999999	62.5516875071357\\
20151.1999999998	62.5498106821892\\
20157.2000000003	62.5525243956284\\
20170.9000000002	62.5492043032076\\
20174.8000000001	62.5509915786448\\
20180.1000000001	62.5533939207738\\
20190.2999999999	62.5554719074412\\
20194.7000000001	62.5537267019233\\
20197.3	62.5555764603365\\
20221.3999999999	62.5537175778256\\
20225.8000000001	62.5564251776188\\
20228.8000000002	62.5545630261655\\
20235.6000000002	62.5528150743488\\
20259.3999999999	62.5553444063279\\
20261.9	62.5535485144606\\
20272.9999999999	62.5558005435774\\
20277.3000000001	62.5578229950585\\
20304.6000000001	62.5559599501593\\
20307.3	62.554044338517\\
20308.7000000002	62.5566256992043\\
20311.5999999997	62.5545867652634\\
20315.7000000002	62.556729245218\\
20322.0000000002	62.5584954310824\\
20336	62.5601762383151\\
20338.9999999997	62.5583236229725\\
20344.2999999998	62.5602131299031\\
20351.8999999997	62.5584709119497\\
20353.9	62.5567456028299\\
20361.9000000003	62.5547588645516\\
20366.9999999999	62.5567704778785\\
20371.5000000002	62.5547330597499\\
20372.9000000003	62.5523977813773\\
20389.6	62.5506996179443\\
20396.5000000001	62.5481697930047\\
20406.9999999998	62.5463686658075\\
20409.0000000003	62.5480790431719\\
20412.4999999998	62.5461724621067\\
20414.5000000001	62.5444534793726\\
20429.6000000001	62.5461950004161\\
20435.4999999999	62.544285462624\\
20437.5000000003	62.5459936587466\\
20444.1000000002	62.5434108451297\\
20445.6	62.5451806492319\\
20454.9000000003	62.5475433879247\\
20455.7999999998	62.5457691912847\\
20464.8000000003	62.5480701102864\\
20475.8	62.5462128648802\\
20496.4999999998	62.5479347794268\\
20513.9999999997	62.5462486777387\\
20524.2999999997	62.5494533335932\\
20542.6000000004	62.5511738963233\\
20545.2999999998	62.5531749199334\\
20556.9000000001	62.5514423310794\\
20559.1	62.5539904276431\\
20561.7999999997	62.5520987846454\\
20563.6999999998	62.5540999231659\\
20570.4000000003	62.5560876011764\\
20575.8000000002	62.5595964210557\\
20576.9000000001	62.5577100646353\\
20579.9999999997	62.5594627820079\\
20582.0000000003	62.5577564971505\\
20592.4999999999	62.5559667064868\\
20594.1999999997	62.5542018908145\\
20596.5000000003	62.5564413543983\\
20611.7999999998	62.5585220188338\\
20614.2999999997	62.556271344303\\
20621.2000000002	62.5542521567506\\
20625.9	62.5559972849801\\
20630.5000000001	62.5580448460054\\
20634.1000000004	62.5563385059755\\
20636.3000000003	62.55451532244\\
20640.5000000002	62.5577744832999\\
20644.2999999997	62.5559473755595\\
20648.8000000002	62.55781179627\\
20651.9000000001	62.555200464846\\
20655.1999999998	62.553436648221\\
20659.3999999998	62.5552409303226\\
20668.4000000003	62.5570312311005\\
20672.2000000002	62.5552067259086\\
20676.6999999997	62.5575524259073\\
20679.3	62.5554900045456\\
20695.1000000003	62.55750125633\\
20707.1000000001	62.5593996291145\\
20709.1999999998	62.5564359973538\\
20716.2999999999	62.558166476801\\
20723.8	62.5562756044953\\
20732.0000000001	62.5542033850888\\
20733.8	62.5560073116973\\
20738.9000000003	62.5584647282897\\
20741.6000000003	62.5565889006205\\
20753.0000000003	62.5593284858648\\
20757.8000000003	62.5617234884068\\
20769.5	62.5635544256991\\
20770.9999999998	62.5652950493715\\
20771.8000000002	62.5633668561855\\
20788.4000000002	62.5615123746302\\
20794.2999999999	62.5634786288616\\
20798.7000000001	62.5656287862761\\
20801.6999999999	62.5633358651655\\
20806.9000000001	62.5606766953429\\
20816.2000000001	62.5625111090828\\
20822.4000000002	62.560691559611\\
20827.4999999999	62.5631373754057\\
20834.5	62.5651560385129\\
20836.0999999998	62.5627513654121\\
20845.1999999998	62.5608650391215\\
20855.9999999997	62.5629911632568\\
20863.6000000002	62.56129066273\\
20867.0000000001	62.5630777635608\\
20874.3000000004	62.5613191277354\\
20877.4999999997	62.5593938000536\\
20880.8999999998	62.561658924381\\
20883.9999999997	62.563385542111\\
20890.0999999999	62.565700663469\\
20892.7999999999	62.5638374758889\\
20907.0999999998	62.5617012321114\\
20910.3000000003	62.563604713444\\
20911.3999999999	62.5617483203022\\
20939.3000000001	62.5600542517933\\
20942.3	62.558255023302\\
20952.2000000003	62.5601962553037\\
20962.7	62.5579598145286\\
20964.4999999998	62.5597435677285\\
20966.6000000002	62.5615857526459\\
20973.1000000001	62.5598382697919\\
20981.0000000002	62.561543484374\\
20982.5999999997	62.5591558760312\\
20985.3000000003	62.5573017431167\\
20986.8999999998	62.5592033163387\\
20989.7999999997	62.5567534862005\\
20991.3	62.5589527139686\\
21001.7999999998	62.5571972059671\\
21009	62.5590815408561\\
21009.9000000003	62.5573536411233\\
21016.3000000001	62.5554329000209\\
21019.7000000004	62.5534020304665\\
21021.5999999998	62.5515538705243\\
21035.8000000003	62.5535394254584\\
21046.0000000001	62.555532854068\\
21048.2999999998	62.5572490070504\\
21052.4000000003	62.5550409690061\\
21054.1000000003	62.5571144949701\\
21056.6999999999	62.5588883401087\\
21060.5000000003	62.557097138733\\
21065.2000000001	62.5550075242223\\
21070.5999999997	62.5579596311466\\
21074.6000000001	62.5560506199377\\
21079.0999999997	62.5578769592774\\
21079.9000000003	62.5559772296015\\
21085.8	62.557917850317\\
21089.5	62.5597450876261\\
21092.0999999997	62.5615156313708\\
21097.5000000001	62.5597224328834\\
21109.0999999997	62.5580315691736\\
21129.1	62.5603430323912\\
21137.5000000001	62.5624479600333\\
21141.2999999998	62.5606629646097\\
21144.4000000002	62.5585849748161\\
21147.5000000001	62.5602905294218\\
21150.4999999997	62.5627641769028\\
21153.7000000002	62.5646455955904\\
21156.7999999998	62.5663495124522\\
21158.3000000003	62.5642770719903\\
21160.9999999999	62.5666907674932\\
21166.3999999998	62.5649020858432\\
21172.8	62.5681885806857\\
21176.5999999997	62.570183267459\\
21181.5000000002	62.5722325036824\\
21188.9999999997	62.5755695145145\\
21192.9	62.5772660784221\\
21194.8999999998	62.5746638358103\\
21198.8000000003	62.5763600941558\\
21204.5999999997	62.5738633416177\\
21207.6999999997	62.5755618215939\\
21209.8999999997	62.5775577557756\\
21212.5999999997	62.5799638895567\\
21216.5999999999	62.5780635065774\\
21219.2	62.5802924695914\\
21238.0000000001	62.5781025609636\\
21246.4000000003	62.5801896783\\
21249.3999999997	62.5821784041977\\
21256	62.5839170873302\\
21258.9999999998	62.5821412947867\\
21260.3	62.5839589095219\\
21262.0000000001	62.5822472850753\\
21264.5	62.5805329044515\\
21265.8000000002	62.5828203838069\\
21267.1000000002	62.5804055070719\\
21271.7000000001	62.5786252221251\\
21275.4000000003	62.5809029165002\\
21278.6000000002	62.5827705639912\\
21297.9999999997	62.5844558904315\\
21298.5999999998	62.5826928404081\\
21302.7999999998	62.5806815034573\\
21305.2999999999	62.5785012250415\\
21308.6000000003	62.5767878847607\\
21314.4999999999	62.5749486267629\\
21328.7000000001	62.5731405423653\\
21330.8	62.5711995274461\\
21337.0000000001	62.5694213365453\\
21340.6999999998	62.5712250712251\\
21342.2999999998	62.5693455281505\\
21345.6999999999	62.5715597447741\\
21347.4000000003	62.5698559550299\\
21364.9000000003	62.5724315469225\\
21380.3000000002	62.5746010364633\\
21381.3000000001	62.5726098384577\\
21385.3000000003	62.5749343009717\\
21388.0000000002	62.5768534839467\\
21400.4000000002	62.5751734772552\\
21408.1	62.5727524967069\\
21409.5000000001	62.5751999103206\\
21410.8999999997	62.5734435570501\\
21415	62.5754724470117\\
21428.1999999997	62.5775259819958\\
21433.3000000003	62.5794321012998\\
21435.5999999998	62.5820477054633\\
21448.9999999998	62.5802481222988\\
21459.0999999997	62.5782881002088\\
21465	62.5759954530843\\
21467.3999999999	62.577850238733\\
21469.2	62.5795903918619\\
21476.2999999999	62.5770613324393\\
21481.2000000003	62.5790804094724\\
21489.5000000003	62.5772466681558\\
21495.8999999997	62.5753628582062\\
21499.3000000004	62.5770951747491\\
21518.8999999999	62.5791161299317\\
21521.2000000002	62.5766101490152\\
21528.4000000003	62.5747265253037\\
21533.2999999999	62.5730260897025\\
21534.8999999999	62.5711632226608\\
21550.5	62.5694876244745\\
21555.7999999999	62.5717321011881\\
21560.9000000001	62.5750197115162\\
21562.7	62.5725787003543\\
21563.7999999998	62.5707780132536\\
21567.4000000003	62.5686797264403\\
21569.5999999998	62.5669341715462\\
21571.0000000002	62.5651913903324\\
21575.9999999998	62.5669143172306\\
21577.3999999998	62.5642451627853\\
21579.0000000002	62.5660940447007\\
21590.3999999998	62.5640906880341\\
21597.0999999998	62.56227659141\\
21611.0000000003	62.5641452771955\\
21618.1999999998	62.5659741977861\\
21621.6999999997	62.5632463532176\\
21636.3000000002	62.5653990497495\\
21639.5000000002	62.5686241889869\\
21645.7999999999	62.5665830480599\\
21650.0000000004	62.569226008194\\
21654.4000000003	62.5671338520862\\
21662.4000000002	62.5689555683785\\
21669.8000000003	62.5706625318991\\
21672.5999999997	62.5727297475626\\
21675.5999999998	62.5746804024784\\
21678	62.5765173147093\\
21691.1999999997	62.5785453154029\\
21697.3	62.5807700461806\\
21700.8	62.5826578621163\\
21703.6999999997	62.5844322192427\\
21723.7999999997	62.5868283319294\\
21731.6000000002	62.5887528357193\\
21748	62.5870765722063\\
21760.6	62.5848433184594\\
21770.4000000003	62.587905652144\\
21775.5000000001	62.5902386156983\\
21780.8000000002	62.5878636787277\\
21784.8000000003	62.5860114115741\\
21793.4000000001	62.5842567738087\\
21813.5999999998	62.5822304331681\\
21818.0000000003	62.5847346927551\\
21829.2999999998	62.5830302252925\\
21842.9999999999	62.5872701219149\\
21845.3999999997	62.5895493351033\\
21848.1000000001	62.5868492598933\\
21859.8000000002	62.5885754280669\\
21872.5999999998	62.5908095479753\\
21885.1999999999	62.5926992090581\\
21886.6000000002	62.5909799101738\\
21895.3999999999	62.5886597702724\\
21900.1999999999	62.5863572645124\\
21910.6000000002	62.5881418667592\\
21914.3999999997	62.5864153870725\\
21931.2000000001	62.588173067716\\
21934.6000000001	62.5898690203194\\
21946.7999999998	62.5919833780625\\
21950.1	62.5898625069476\\
21957.0000000001	62.5879556043376\\
21959.7000000003	62.5861802020055\\
21964.3000000002	62.5880971025842\\
21967.3	62.5859227764779\\
21969.1000000003	62.5876226717404\\
21973.6000000001	62.5893681992564\\
21976.4999999999	62.5911196454411\\
21978.0000000002	62.5932177940768\\
21981.2	62.5913844950026\\
21982.5999999999	62.589672788147\\
21986.2999999998	62.5914201506386\\
21992.8	62.5892901800126\\
22005.2999999998	62.5910003908132\\
22010.6000000003	62.5891043901375\\
22015.2	62.5910162477913\\
22017.4	62.5893039627569\\
22020.5999999998	62.5911074579827\\
22022.8000000002	62.5893955836879\\
22030.3000000003	62.5912375626407\\
22044.7000000002	62.5889098562926\\
22052.7000000002	62.5870637742146\\
22066.4000000001	62.5889923639907\\
22072.6000000002	62.5872684356694\\
22076.0000000003	62.5894066433836\\
22102.6999999998	62.5875454693523\\
22114.7000000003	62.5893067086295\\
22123.9000000004	62.5913035617429\\
22127.7000000004	62.5936604633086\\
22135.1999999998	62.5954922680063\\
22151.5000000003	62.5972841690894\\
22155.3999999998	62.5952923653269\\
22156.9999999997	62.5970907745147\\
22161.7	62.5991571081771\\
22165.1999999999	62.6010024678213\\
22187.7000000003	62.6033225466247\\
22197.3000000001	62.6014758485228\\
22208.7000000002	62.6035625517813\\
22221.6000000002	62.6018711439719\\
22232.0000000001	62.6000242892034\\
22238.9999999997	62.5983065861478\\
22249.6999999999	62.5965177215076\\
22264.5000000002	62.5944324173801\\
22266.3999999997	62.5922349718187\\
22271.1999999999	62.5899700511421\\
22276.3000000001	62.591801188702\\
22299.2000000001	62.5934446372756\\
22312.9	62.5917626495765\\
22326.0000000004	62.5940043267745\\
22328.5999999997	62.5961206877248\\
22330.1	62.5941550008509\\
22334.3999999999	62.5959837918915\\
22348.8999999998	62.5942995212314\\
22351.6000000003	62.596581020683\\
22353.2000000004	62.5947846626673\\
22361.2000000002	62.5965395571814\\
22369.0000000004	62.5943824293333\\
22382.3000000004	62.5924833797984\\
22398.8999999999	62.5952051430867\\
22401.5999999997	62.5934638889013\\
22416.9000000003	62.5917830218138\\
22419.6000000004	62.5895975414479\\
22427.3000000004	62.591294577169\\
22430.0000000003	62.5895559984128\\
22436	62.5915377449735\\
22437.6	62.5933139314636\\
22449.0000000002	62.5913733735428\\
22456.6999999997	62.5935128780592\\
22457.8999999998	62.5915041410633\\
22465.0999999998	62.5892491497961\\
22488.1000000002	62.5910477494864\\
22489.6999999999	62.5892626879741\\
22493.6999999999	62.5874685468885\\
22505.7999999997	62.5893654552806\\
22517.1000000003	62.5912635674063\\
22519.2999999998	62.5891453591126\\
22526.4000000002	62.5871751048765\\
22536.4000000001	62.589133183946\\
22557.6000000001	62.5870545312687\\
22563.7999999999	62.5889141504793\\
22569.3999999997	62.5871197855513\\
22581.8000000003	62.5890646934049\\
22586.0999999999	62.587332087735\\
22588.6	62.5892592313856\\
22596.6000000003	62.5874574606027\\
22601.6999999998	62.5857232609792\\
22607.4999999997	62.5878023319592\\
22616.2	62.5902557005346\\
22619.5000000003	62.5921766963165\\
22627.7999999997	62.5939658563101\\
22632.3000000002	62.5969848535728\\
22645.2000000002	62.5988615739248\\
22651.8999999999	62.597121666961\\
22653.9999999999	62.5988231710816\\
22663.3999999998	62.5971275398769\\
22665.2000000001	62.5996567440096\\
22672.2999999999	62.5976958769253\\
22685.9000000001	62.5994005113286\\
22702.6000000001	62.6013645954006\\
22707.2999999998	62.6033803958181\\
22713	62.6052806530152\\
22719.2999999999	62.6077273167425\\
22724.1000000002	62.6059443236726\\
22726.6	62.6078577180145\\
22733.9000000001	62.6101873845342\\
22741.3999999998	62.6084471121078\\
22759.4	62.6063841472792\\
22771.1000000003	62.6084703485104\\
22778.2000000004	62.6065158506122\\
22781.8999999998	62.6046879115091\\
22797.3999999999	62.6024783419235\\
22801.0999999998	62.6006525972317\\
22808.6	62.598043728928\\
22811.8999999999	62.5999473961073\\
22819.9000000001	62.5981595092025\\
22822.2000000003	62.6001761435088\\
22825.6999999998	62.5984631425843\\
22827.0999999998	62.5963762528913\\
22836.7000000001	62.5985251874168\\
22844.6	62.6005156557101\\
22855.0000000004	62.5987197605786\\
22867.7999999999	62.6004136803117\\
22876.0999999997	62.5986833477588\\
22878.6000000001	62.596651033494\\
22893.6000000002	62.5949497023199\\
22907.4999999999	62.5923274371824\\
22919.5999999999	62.5902607800277\\
22923.5000000001	62.5879006787765\\
22926.1999999997	62.586200128237\\
22927.9999999998	62.5882650546709\\
22931.2000000001	62.5865083968201\\
22935.5	62.588290692199\\
22939.4000000002	62.5907277839535\\
22944.9999999999	62.5889623492598\\
22949.9	62.5869281045752\\
22960.7999999999	62.5890100126737\\
22967.4000000002	62.5871339936867\\
22969.5999999998	62.5850577064568\\
22975.9999999999	62.5867749531034\\
22983.8000000001	62.5846788404057\\
23001.7	62.5829282925684\\
23007.4999999997	62.5797562544551\\
23013.5999999999	62.5818534177468\\
23014.3999999998	62.5796780290686\\
23018.3999999999	62.5779264504638\\
23026.6000000001	62.5756187382473\\
23038.8999999997	62.5773688094101\\
23041.0999999998	62.5752998975748\\
23044.5999999999	62.5775123998143\\
23049.4999999999	62.579394002499\\
23056.7000000001	62.5776343638319\\
23063	62.5757161873295\\
23065.5	62.5776047447281\\
23069.7999999998	62.5793783241367\\
23081.6999999999	62.5817743850133\\
23090.5	62.5834755268378\\
23092.5999999997	62.5816816569739\\
23102.6000000003	62.5844598250421\\
23104.6000000002	62.5820720459474\\
23114.5000000001	62.5803604648144\\
23138.4999999999	62.582005825763\\
23146.8000000002	62.5837585162592\\
23151.1000000001	62.5855247244203\\
23166.0999999997	62.5873039169134\\
23168.1	62.5853540628966\\
23171.2999999997	62.5836160093909\\
23180.2999999998	62.5817500992218\\
23182.4999999997	62.5840069707453\\
23191.6000000001	62.5823031515585\\
23193.0000000002	62.5802501606081\\
23207.7999999998	62.5821379788779\\
23220.2000000001	62.5801561564665\\
23228.9999999998	62.5784038124594\\
23238.9000000004	62.5767029562374\\
23254.0999999999	62.5749327003294\\
23257.3	62.576642272997\\
23266.6999999998	62.5745697732391\\
23285.0000000001	62.5769268759851\\
23287.8000000004	62.5788499607092\\
23305.7	62.5814175012229\\
23318.9000000003	62.5794416570179\\
23322.8999999998	62.5811430776487\\
23326.4999999998	62.5792014266974\\
23332.6999999998	62.5814304326956\\
23336.2	62.5836143690302\\
23338.6000000002	62.5810349333939\\
23353.0000000003	62.5792721308948\\
23355.3000000003	62.5812445943979\\
23358.1000000003	62.583161373736\\
23370.7000000003	62.5849350471529\\
23377.5999999997	62.5869952989387\\
23380.7	62.5846848696366\\
23385.3999999997	62.5870731863762\\
23388.8999999999	62.5888238060627\\
23399.7000000001	62.5924153197891\\
23409.9999999999	62.5905058073225\\
23430	62.5887213456195\\
23446.5999999997	62.5866326604597\\
23452.8999999999	62.5847439559971\\
23461.0000000001	62.5865794868953\\
23478.0999999999	62.5848659607636\\
23482.6000000001	62.5869256942345\\
23486.4000000004	62.5848891916633\\
23489.2000000004	62.5829633066971\\
23493.5999999998	62.5848631760855\\
23503.0000000004	62.5866375073926\\
23510.8999999998	62.5890008931989\\
23523.2	62.5907079363865\\
23525.5999999998	62.5889984144999\\
23533.0000000001	62.5909888624958\\
23539.1999999998	62.5940448526507\\
23540.5000000003	62.5922873673568\\
23586.7	62.5939084572727\\
23591.6000000004	62.5915046393435\\
23602.5000000004	62.5897146924491\\
23610.9000000002	62.5915886662996\\
23614.6999999999	62.5933736470349\\
23619.9	62.5914479254869\\
23626.5999999999	62.5931678990295\\
23639.3999999997	62.5956555764716\\
23641.8999999997	62.5974959817274\\
23649.7000000003	62.6005293913691\\
23655.7000000001	62.6024061752298\\
23661.9999999997	62.6005299614151\\
23681.4000000001	62.5986529569495\\
23686.8	62.6004247073277\\
23700.8000000004	62.5984667248923\\
23718.4	62.5967072116702\\
23735.6000000004	62.5985330114553\\
23743.2000000001	62.596606200486\\
23747.2	62.5986954306384\\
23751.5000000001	62.6004142878796\\
23767.0000000002	62.598297646747\\
23772	62.5964891616643\\
23783.7000000001	62.5984914101195\\
23805	62.5966704613717\\
23811.4000000002	62.5945446527938\\
23835.2000000002	62.5928769514124\\
23849.8999999998	62.5911949685535\\
23864.6	62.589515057805\\
23874.2000000002	62.5878036214674\\
23877.8999999998	62.5856436887511\\
23885.2000000003	62.5874491842263\\
23893.3000000003	62.5892505880285\\
23898.2999999998	62.587453553376\\
23902.5000000002	62.5848234083322\\
23909.1	62.5830224348786\\
23916.3000000001	62.5855061798598\\
23923.7999999999	62.5872035913877\\
23930.6	62.5852983824125\\
23947.6999999999	62.5836193721344\\
23953.8000000003	62.5856332371764\\
23957.1	62.5836909154659\\
23958.6	62.5818596167572\\
23969.3999999999	62.5837001189011\\
23978.9999999999	62.581998490352\\
23988.5999999997	62.580298223747\\
24008.7999999999	62.5822090974597\\
24019.8	62.5843571372071\\
24025.9999999996	62.581942137925\\
24028.1	62.5839638424851\\
24036.5999999997	62.5822180249369\\
24044.1999999999	62.580320491759\\
24049.4999999999	62.5785875856563\\
24051.9000000003	62.5765009146849\\
24054.3	62.5744146601038\\
24065.2000000001	62.5722513328319\\
24069.2999999998	62.5744721513623\\
24071.5999999997	62.5726475487813\\
24074.5999999998	62.5702500965744\\
24077.4999999999	62.5685284247599\\
24079.9999999998	62.5703381630475\\
24082.5999999998	62.5681505811226\\
24086.3999999997	62.5661677703278\\
24090.6000000003	62.5643920683085\\
24093.2	62.5663566219655\\
24100.0000000002	62.5644706868436\\
24110.1999999998	62.562473299793\\
24116.5000000003	62.5643747460256\\
24118.0999999998	62.5622973522071\\
24127.4	62.564293855559\\
24144.7000000003	62.5625393459461\\
24150.5000000001	62.5603504674832\\
24156.5000000003	62.5621983226116\\
24161.2000000003	62.5603754764851\\
24165.8000000004	62.5583984043632\\
24175.9000000003	62.5566677696889\\
24222.4000000003	62.5583651563629\\
24225.0000000003	62.5566045135005\\
24239.4999999998	62.554662618195\\
24244.8	62.5529492800548\\
24249.1000000003	62.5550533625852\\
24252.1	62.5526756335508\\
24270.2999999997	62.5543872371284\\
24273.8999999997	62.5525253357502\\
24285.8000000001	62.5544039957342\\
24286.7999999997	62.552651841116\\
24291.4000000001	62.5548031204331\\
24294.8000000004	62.5530461125586\\
24313.0000000003	62.5547544327955\\
24317.5000000003	62.5530479981577\\
24318.8999999998	62.5510917389695\\
24320.9999999999	62.5530917598299\\
24325.8000000004	62.5551367061445\\
24329.1000000003	62.553228219588\\
24355.0000000002	62.5515805724468\\
24357.4999999998	62.553371432325\\
24363.1999999999	62.5551546793743\\
24384.3999999998	62.5569521622342\\
24386.5000000001	62.5552557552098\\
24395.8999999999	62.5532874241679\\
24406.5	62.5552104758549\\
24408.6999999998	62.5532594801875\\
24419.7999999998	62.5514436996057\\
24423.1000000003	62.5495430574208\\
24429.6000000001	62.5476366881296\\
24437.1	62.5497192804413\\
24440.5000000001	62.5479734540069\\
24442.2000000003	62.5497600471314\\
24445.5000000002	62.547861373826\\
24460.3999999998	62.5461458269455\\
24499.5999999997	62.5444393196651\\
24503.9999999997	62.5425949126881\\
24507.8	62.5443224429674\\
24513.8999999999	62.5426287019662\\
24522.0999999998	62.5408813238616\\
24529.4000000002	62.5426527242708\\
24548.4000000002	62.5451656924048\\
24552.7999999996	62.5433248211006\\
24560.8000000004	62.5457536165206\\
24562.8999999999	62.5477343972642\\
24568.2	62.5497083640301\\
24576.8000000002	62.5522340083574\\
24579.0999999998	62.5545176409322\\
24582.8999999999	62.5562380506854\\
24593.2	62.5580950909394\\
24605.6000000002	62.5562369694827\\
24612.9000000004	62.5543412018039\\
24617	62.5561093711282\\
24620.3999999999	62.5543754188583\\
24629.8999999998	62.5562322371092\\
24637.4	62.5538305428716\\
24661.8999999997	62.5557537912578\\
24667.5	62.5581734745172\\
24674.8999999998	62.5600810536981\\
24681.3000000004	62.5624964548202\\
24705.2999999997	62.5648643616375\\
24710.7999999997	62.5667215682148\\
24714.2	62.5649927369984\\
24717.9999999996	62.5675112569332\\
24725.0999999998	62.5697668775177\\
24728.5000000001	62.5716781378647\\
24732.8999999997	62.5738891359722\\
24738.6000000004	62.5756405955042\\
24744.1	62.5738556914348\\
24749.4999999999	62.5755567766752\\
24751.1999999999	62.5736829984688\\
24758.3000000001	62.5755299211581\\
24759.8000000002	62.5737583754377\\
24761.9000000004	62.5757208626121\\
24791.6999999999	62.5779491606096\\
24797.3999999998	62.579695533824\\
24804.6999999999	62.5814358511256\\
24808.8000000003	62.5831858728118\\
24813.8999999996	62.5812041589425\\
24818	62.5829535701766\\
24825.3000000001	62.5810661661041\\
24835.5000000004	62.5831467731804\\
24851.8	62.5807282340586\\
24871.2000000004	62.5789564678967\\
24875.0000000004	62.5770348661915\\
24879.3999999998	62.5788299604092\\
24885.8999999997	62.5765490637306\\
24895.0999999997	62.5783283524535\\
24896.3999999999	62.5762657401643\\
24904.9999999996	62.574332164898\\
24908.5999999996	62.5761280195273\\
24915.8000000002	62.5781127713629\\
24922.9999999999	62.580096376454\\
24926.6999999996	62.5820402137459\\
24936.6999999998	62.5802027525585\\
24942.4999999996	62.5824893956524\\
24947.4000000001	62.5846277182082\\
24952.2999999997	62.5867652009426\\
24955.4000000001	62.5850013023181\\
24961.1000000004	62.5879364774129\\
24967.4999999997	62.5859113410981\\
24970.0999999999	62.5842003668373\\
24992.9999999999	62.5860737563568\\
24995.1999999997	62.5841658231748\\
24998.4	62.5861551693102\\
25002.3000000001	62.5839919367741\\
25007.5999999998	62.5819247671717\\
25020.5999999997	62.5801836079726\\
25024.8999999998	62.5822177822178\\
25031.5000000002	62.5844932005945\\
25046.0999999999	62.5823478212264\\
25051.3	62.5841270348164\\
25053.4999999998	62.5822237123607\\
25069.6	62.5839160420747\\
25071.0999999999	62.5817671272217\\
25092.0999999998	62.5800049417747\\
25097.0999999999	62.5782955867587\\
25111.0999999997	62.580442193125\\
25114.5999999997	62.5824716202065\\
25117.2000000002	62.5807710223631\\
25141.3000000001	62.578854001766\\
25161.0000000003	62.5771528271816\\
25163.7999999997	62.5793299130898\\
25166.6000000003	62.5775330098901\\
25191.8999999996	62.5758177199111\\
25196.9999999997	62.5738676276238\\
25206.7999999999	62.5757233138545\\
25209.8000000001	62.5738301222932\\
25235.5999999997	62.5756368953506\\
25265.9000000004	62.5773767117866\\
25270	62.5755339314051\\
25273.8999999997	62.573395584395\\
25275.6000000004	62.5751215594425\\
25277.1999999999	62.57313874504\\
25283.2999999998	62.575049241795\\
25294.2000000004	62.5769442127278\\
25296.6999999997	62.5786660763417\\
25299.9000000001	62.5762845849802\\
25304.9	62.5781466113416\\
25318.4000000002	62.5799316705176\\
25330.5000000002	62.5780676336131\\
25338.1	62.580609514488\\
25340.8000000001	62.5782825392942\\
25345.5000000002	62.5800927971719\\
25356.4000000004	62.5780371896752\\
25370.0000000004	62.5807545102305\\
25376.3000000001	62.5790104191296\\
25386.7999999997	62.5814888781222\\
25403.9	62.5838450637695\\
25406.6999999999	62.5816710486956\\
25413.7999999998	62.5799267330083\\
25424.6000000002	62.5816627138177\\
25437.9999999998	62.5836835298234\\
25440.7999999997	62.5854431250467\\
25457.5999999998	62.5873507818852\\
25467.7000000002	62.5892303222108\\
25492.4999999999	62.5911048696484\\
25498.7000000002	62.5931416380379\\
25516.5	62.5910191796713\\
25529.2999999999	62.5890150179793\\
25541.5000000002	62.5916152472829\\
25555.6000000001	62.5895592764041\\
25573.2999999998	62.5919901147286\\
25575.2999999997	62.5902234178156\\
25577.7	62.5921697722244\\
25584.6000000001	62.5944412090038\\
25600.9999999997	62.5965290553922\\
25601.8000000001	62.5945730590308\\
25610.1999999997	62.5927849341866\\
25616.2999999999	62.5907621679861\\
25623.3000000004	62.5884933303153\\
25630.5999999999	62.5905652206143\\
25633.9000000004	62.5926503862058\\
25639.3999999998	62.5944343688449\\
25645.3999999998	62.5965569008208\\
25655.2000000003	62.5948634395232\\
25663.4000000001	62.5966840064683\\
25669.6000000004	62.5948102237268\\
25674.8999999996	62.5927945472249\\
25712.3000000003	62.5912011325275\\
25725.3000000004	62.5930014693649\\
25730	62.5947819868559\\
25738.1999999997	62.5927120283779\\
25739.6000000004	62.5908615873534\\
25749.7000000004	62.5931075192817\\
25752.7999999997	62.5913974736826\\
25768.4000000001	62.5895958243592\\
25776.9000000004	62.5914575008729\\
25778.0999999997	62.5897075823758\\
25786.3999999999	62.5877881837396\\
25802.2000000001	62.5901566914578\\
25809.1000000004	62.5920214497156\\
25821.3000000004	62.5899447744894\\
25825.2000000001	62.5924965053649\\
25829.9000000002	62.5903987611305\\
25846.4000000003	62.5922271874335\\
25871.6999999999	62.5940212895894\\
25875.2000000001	62.5921245357542\\
25879.2999999999	62.5941868822307\\
25882.1000000003	62.5924380462248\\
25910.2000000003	62.5905527917469\\
25922.9	62.588049222698\\
25936.4999999998	62.586075275865\\
25955.6000000003	62.5885643615853\\
25961.2999999997	62.5867634257012\\
25967.4999999999	62.5887644603275\\
25975.9999999998	62.5906121396206\\
25981.9000000002	62.5925640828266\\
25989.4000000004	62.590276842571\\
25992.0000000004	62.5878632353677\\
25998.6000000003	62.5858215987723\\
26005.9000000003	62.5840190725217\\
26018.1000000002	62.5858053208908\\
26023.0999999999	62.5876141289311\\
26027.8000000003	62.5859174193846\\
26037.3000000002	62.5842057962777\\
26058.8000000001	62.5824574329692\\
26068.9999999998	62.5806030894814\\
26079.0000000002	62.5788466626532\\
26088.6999999996	62.5770445555181\\
26097.1000000004	62.5752954339929\\
26103.0000000004	62.5772417835429\\
26110.9000000002	62.5755428746505\\
26112.4999999998	62.5736234614707\\
26119.1	62.5761891635272\\
26125.0999999997	62.5740664186303\\
26130.4999999997	62.5760602511997\\
26134.0000000003	62.5741846859084\\
26142.7	62.5759291277139\\
26144.5999999999	62.577883854089\\
26148.1999999998	62.5795940845103\\
26162.7999999998	62.5813652156298\\
26168.8999999998	62.5832091405862\\
26186.4000000002	62.5853015866954\\
26204.7	62.5831908657956\\
26212.7000000004	62.5808765183422\\
26218.7999999999	62.5827170476259\\
26224.5000000003	62.5809354575475\\
26235.9	62.5827107790822\\
26239.7999999997	62.5844610688303\\
26243.2999999999	62.5818300982342\\
26247.7	62.5801019514016\\
26256.2999999998	62.5824560868969\\
26284.8	62.5807973399176\\
26293.3000000001	62.5826252976032\\
26305.8999999997	62.5845814643047\\
26325.1000000001	62.5867989606917\\
26329.6000000002	62.5848376548157\\
26333.6999999999	62.5830681481594\\
26339.3000000002	62.5853284433207\\
26354.5	62.5833820281848\\
26356.4999999997	62.5850830532011\\
26358.1999999998	62.583322900187\\
26361.0000000005	62.5854004574923\\
26368.8999999996	62.5894042246577\\
26388.0000000002	62.5876815685859\\
26394.2	62.5904077774368\\
26396.2000000003	62.58869614302\\
26400.8999999999	62.5904321805992\\
26404.1999999999	62.5879118173934\\
26414.1000000002	62.5860332699836\\
26422.6999999999	62.5879921885644\\
26425.5999999997	62.5860431322538\\
26428.1999999997	62.5840481604946\\
26442.1999999997	62.581923660196\\
26443.6999999998	62.5836680053548\\
26445.2000000005	62.5812526233395\\
26451.6000000003	62.5838792969828\\
26462.2000000001	62.5856407039449\\
26468.9	62.5837772488572\\
26477.5999999996	62.5854964743917\\
26484.8999999997	62.5837266377195\\
26502.8999999997	62.5857450099989\\
26520.4000000001	62.5878094304406\\
26540.2	62.5859541904199\\
26543.8999999998	62.587778782399\\
26552.0999999999	62.5899172196654\\
26554.1999999999	62.5879801011512\\
26571.6999999996	62.5896627251447\\
26582.9000000003	62.5915058496031\\
26587.1999999996	62.5896574680392\\
26593.9999999999	62.5917026708932\\
26597.7000000004	62.5935227725601\\
26602.0999999997	62.5918157144898\\
26606.1999999997	62.5938217640183\\
26612.1999999999	62.5921096635766\\
26634.2999999999	62.5938635749257\\
26645.4999999997	62.5915723421503\\
26651.4000000002	62.5897229049022\\
26656.6999999997	62.5874073407161\\
26662.7000000003	62.5856999264893\\
26666.5999999998	62.583296770879\\
26672.2000000001	62.5855288070395\\
26694.6999999997	62.5837241709996\\
26703.9000000002	62.58201018574\\
26724.6000000002	62.5836772723361\\
26745.8000000003	62.5864151140923\\
26755.0999999997	62.5882071522545\\
26756.8000000003	62.5902103756414\\
26766.4000000004	62.5883100143837\\
26783.3999999998	62.5900274422686\\
26787.3999999998	62.5881474568362\\
26792.3000000004	62.5864050999537\\
26797.4999999996	62.5847090784249\\
26802.9999999996	62.5864172427816\\
26806.0999999999	62.583656019876\\
26813.7999999997	62.5854500837252\\
26819.1999999997	62.5877632898696\\
26821.4999999997	62.5857517821457\\
26828.5000000003	62.5839589095219\\
26831.4000000003	62.5857667294039\\
26836.2999999998	62.5840276639192\\
26838.1000000001	62.5857918936441\\
26844.2000000003	62.5879609451541\\
26848.3999999998	62.5859917686277\\
26859.0000000004	62.5877263199437\\
26862.6000000004	62.5897620120092\\
26863.7000000002	62.5879436267393\\
26882.1999999996	62.589882562132\\
26918.6000000002	62.5880893208067\\
26938.6000000001	62.5902511999466\\
26943.4999999999	62.588147092445\\
26949.7000000005	62.5863642772859\\
26955.8999999998	62.5842113073156\\
26965.9999999997	62.5859875918283\\
26976.7	62.5841463776282\\
26979.4000000001	62.5864081988176\\
26988.4999999999	62.5845727455296\\
26990.3999999999	62.5864656082696\\
26997.2	62.5888514777404\\
27020.9000000002	62.5905776988268\\
27025.4000000001	62.5927364896117\\
27030.5000000004	62.590545529881\\
27033.4000000004	62.588640020715\\
27057.5999999999	62.5906858306508\\
27062.4	62.5888221709007\\
27067.3000000003	62.590791875097\\
27077.7	62.5926035349992\\
27081	62.5908844175458\\
27105.2999999997	62.5930626369653\\
27109.4000000004	62.5909736439256\\
27121.0999999997	62.5927318850198\\
27125.6000000002	62.5945136899693\\
27134.4000000003	62.596325710811\\
27137.5999999996	62.5981568076882\\
27143.8000000001	62.600068523683\\
27157.8999999997	62.6021798365123\\
27180.8000000001	62.6002082344588\\
27193.0000000004	62.6019100433566\\
27196.7000000003	62.6036886692552\\
27200.7999999999	62.6019727288435\\
27225.3999999999	62.5997685992911\\
27229.5999999998	62.5974579227828\\
27232.0999999997	62.5994227421949\\
27234.7999999999	62.5976229029664\\
27246.0000000002	62.5957476482873\\
27247.6999999999	62.5977143108801\\
27250.2000000003	62.5996778017123\\
27255.1000000001	62.5975960550647\\
27266.3000000001	62.5993897250829\\
27286.3000000004	62.601149290489\\
27296.9999999996	62.602987130501\\
27298.6999999997	62.6009201869679\\
27314.7999999998	62.5991674873421\\
27320.1999999997	62.6010695343755\\
27324.2000000002	62.6028846118656\\
27343.7999999999	62.6048222821178\\
27354.1000000005	62.602817848813\\
27367.0000000003	62.6047334207863\\
27397.0999999997	62.602747726045\\
27401.5999999999	62.6045099391644\\
27417.9999999996	62.6028061754826\\
27422.4999999997	62.6045670359485\\
27436.9000000003	62.6026897984474\\
27439.7999999997	62.6004467946312\\
27443.9000000005	62.6023903221105\\
27466.6	62.5990745156863\\
27469.7000000004	62.5963785684643\\
27478.5	62.5945281055076\\
27486.6000000001	62.5964557404126\\
27488.8000000004	62.5983578826363\\
27501.4000000002	62.6002218060833\\
27542.4000000004	62.6017972224744\\
27577.1999999998	62.603663157742\\
27585.6	62.6016378050947\\
27588.6000000001	62.5991800990985\\
27596.1000000004	62.6013726527565\\
27603.6999999998	62.6036994906498\\
27612.6999999998	62.6014022482327\\
27626.8	62.5994954193196\\
27630.0000000001	62.601293516853\\
27640.4000000002	62.5994464644272\\
27649.2999999996	62.5977417231477\\
27656.1999999999	62.5994800461378\\
27663.4999999996	62.5974204369641\\
27689.0000000003	62.5957506744531\\
27693.2000000003	62.5974513691037\\
27704.1000000004	62.5952021715119\\
27718.0000000004	62.5933956512171\\
27720.5000000003	62.5953262194902\\
27728.4000000004	62.593360621743\\
27762.0999999997	62.5908609548235\\
27773.3999999999	62.5891587304445\\
27787.6000000002	62.5910024939092\\
27791.3000000002	62.589146282663\\
27797.5000000004	62.5870578755001\\
27809.1999999999	62.5887742589709\\
27814.6999999996	62.5864647597682\\
27849.5999999999	62.5881068736827\\
27856.9000000005	62.5900132821194\\
27859.8000000003	62.5878054120797\\
27873.8999999998	62.5895099375762\\
27879.8999999997	62.5914634146341\\
27884.3000000005	62.593421411255\\
27905.5999999997	62.5950970590238\\
27914.1999999998	62.5969485174266\\
27919.2999999998	62.5948265363869\\
27936.2000000002	62.593113619198\\
27943.2000000004	62.5913904227489\\
27947.2000000003	62.5935242402665\\
27957.2999999998	62.5955918647657\\
27983.5999999996	62.5939386142647\\
27986.3999999996	62.5962517642435\\
27988.6000000003	62.5945470850736\\
28000.5000000002	62.5925873017007\\
28007.0999999999	62.5906909651804\\
28014.9999999997	62.5926732369329\\
28016.4999999997	62.5907497697794\\
28021.0000000002	62.5889062171007\\
28032.7000000001	62.5870408949516\\
28036.2000000004	62.5888580162147\\
28040.3999999998	62.5869724148999\\
28053.4999999997	62.5851940570907\\
28067.4999999999	62.5831920078667\\
28072.1000000002	62.5811300859925\\
28077.4	62.5792894666548\\
28079.7000000002	62.5816423193897\\
28093.2999999997	62.5798230189297\\
28100.1999999997	62.5815382753921\\
28103.8000000001	62.5834848544152\\
28105.6999999998	62.5853026777391\\
28112.4999999997	62.5872384624688\\
28134.6000000005	62.589613537731\\
28139.7999999997	62.5876424578623\\
28144.8999999998	62.5894475039971\\
28148.9000000004	62.5876585313866\\
28171.2000000002	62.5856811719729\\
28176.5000000002	62.5873952144687\\
28183.2000000001	62.5902573509844\\
28194.9	62.5884021989715\\
28204.1999999997	62.5901015093443\\
28210.7999999999	62.5924731220911\\
28222.0000000002	62.5906647627214\\
28232.3999999996	62.5927565748694\\
28234.8999999998	62.590401983354\\
28274.1000000003	62.5885082513387\\
28286.0999999997	62.5867030566142\\
28288.7999999999	62.5849714905846\\
28293.2999999996	62.5866809927404\\
28298.9000000004	62.5845436234496\\
28302.2000000004	62.5871395610957\\
28312.0000000004	62.5891403322255\\
28321.1999999999	62.5871693742872\\
28330.5000000001	62.585331761417\\
28347.2999999997	62.5835173596168\\
28354.2	62.5852163516645\\
28369.9999999998	62.5834945946613\\
28377.5000000002	62.5817546233649\\
28385.4999999996	62.5834930387239\\
28389.8000000004	62.5852856121367\\
28396.9999999999	62.5835032450497\\
28402.2	62.5864806723399\\
28423.1000000003	62.5847898899491\\
28433.8000000001	62.58304347979\\
28435.7000000002	62.5813235428579\\
28441.4999999998	62.5794610711071\\
28458.5	62.5775688192673\\
28461.2	62.5758486084613\\
28480.2000000003	62.5776413872045\\
28487.4000000004	62.5758666081615\\
28497.5999999997	62.5780326131583\\
28510.3999999996	62.5797513196892\\
28516.8000000003	62.5779800749731\\
28521.3000000005	62.5796770144523\\
28528.6999999998	62.577115055663\\
28541.0000000001	62.5753737592454\\
28548.6000000003	62.5772802264201\\
28554.4999999999	62.5755570030748\\
28570.4999999998	62.5772647406775\\
28596.7	62.5790298215185\\
28599.0999999997	62.5807714901116\\
28610.5000000001	62.583098571858\\
28624.6000000004	62.5854594109283\\
28633.6000000002	62.5835990458795\\
28641.0999999996	62.5818750611008\\
28653.3000000004	62.5796589584482\\
28661.3000000001	62.5813812305051\\
28666.3999999998	62.5831545532242\\
28672.9999999999	62.5813044281923\\
28683.2999999996	62.5832362969523\\
28712.7000000005	62.5814967540609\\
28719.7000000003	62.5794747874289\\
28724.9999999997	62.5776759697965\\
28749.6999999996	62.5795657708923\\
28755.9999999996	62.5776791706803\\
28758.0000000003	62.5795862730848\\
28767.9999999998	62.5776467684693\\
28777.5999999997	62.5758834097235\\
28782.7000000004	62.5779979710105\\
28793.1000000004	62.5797063195477\\
28797.8	62.5778268554304\\
28802.9000000004	62.5795924035691\\
28844.5999999999	62.5778739248458\\
28871.1000000003	62.5762005043088\\
28874.8999999996	62.5780086580087\\
28893.1000000003	62.576315534451\\
28915.7000000004	62.5744402714087\\
28920.4999999996	62.5761567878951\\
28930.0000000001	62.5780761214099\\
28936.2000000002	62.5798737226252\\
28939.8999999997	62.5780926053905\\
28943.2000000003	62.5761402466201\\
28972.2000000004	62.5783938451555\\
28991.0999999997	62.5765749606777\\
29013.1000000002	62.5784125846167\\
29027.5999999996	62.58022509534\\
29032.3000000004	62.5780162852537\\
29047.0999999999	62.5757387975433\\
29056.0999999999	62.5780384220924\\
29067.7999999997	62.575899875808\\
29094.3000000001	62.5742410910691\\
29112.6000000003	62.576126570191\\
29116.1000000001	62.5778776076548\\
29129.0000000001	62.5796883528842\\
29132.1999999996	62.5779632916042\\
29138.1000000003	62.5797063648407\\
29162.9000000002	62.5775811816343\\
29175.3999999996	62.5795616184813\\
29188.8000000005	62.5813237223739\\
29190.7000000001	62.5789632349918\\
29200.5999999997	62.5772669833257\\
29208.9000000004	62.5789996234037\\
29222.9999999998	62.5813140974092\\
29230.9000000004	62.5839006534159\\
29238.3000000004	62.5858460107256\\
29249.3999999996	62.5836339082719\\
29263.3000000003	62.5819282790107\\
29266.6000000001	62.5837555993672\\
29270.3999999999	62.5855383406501\\
29289.6999999996	62.5832200970986\\
29294.6999999997	62.5861927714134\\
29297.4000000001	62.5841795375032\\
29310.4000000001	62.5823510346122\\
29317.7999999997	62.5801984453184\\
29329.3999999999	62.582041971394\\
29331.7999999997	62.5799897040424\\
29336.9000000004	62.5817227392031\\
29341.2999999996	62.5835849686791\\
29343.8000000005	62.581660924417\\
29357.0000000004	62.5794100915962\\
29360.9000000003	62.5813153502946\\
29363.8999999997	62.5793488625528\\
29367.7999999996	62.5812536817409\\
29394.4999999996	62.5795214086941\\
29402.5000000003	62.5777992422439\\
29407.4000000003	62.5796140440364\\
29424.4999999999	62.581309516527\\
29440.1000000002	62.5831346254441\\
29454.7	62.5813110257072\\
29458.0999999997	62.5835930233348\\
29471.1000000004	62.585507207036\\
29473.1000000002	62.5832960112916\\
29487.4000000003	62.5814328105129\\
29532.5999999998	62.5831705194581\\
29535.6999999996	62.5813419646666\\
29555.9	62.5832318311003\\
29565.6999999996	62.5810903138085\\
29575.2000000003	62.5829661913827\\
29579.4999999996	62.5806298935753\\
29588.5000000002	62.5788310362775\\
29595.8999999999	62.5770374374916\\
29598.0000000003	62.5753004415824\\
29609.2999999999	62.5770870061535\\
29624.1	62.5795802080731\\
29630.2999999999	62.5776229818025\\
29658.1000000001	62.5793203903136\\
29662.7	62.5810779831978\\
29666.7999999999	62.5784965736225\\
29677.5000000003	62.5805321185001\\
29683.1000000004	62.5784955799914\\
29701.3	62.580551758503\\
29707.3999999996	62.5825128334596\\
29712.2	62.5808166988082\\
29727.2000000003	62.5828783643318\\
29736.9999999999	62.5810855799658\\
29760.6000000002	62.5828693545515\\
29765.6000000001	62.5810916591916\\
29769	62.5833498493404\\
29773.1000000001	62.581449088442\\
29781.7000000001	62.5831883902249\\
29784.3999999997	62.5812083466232\\
29788.1000000001	62.579477779792\\
29810.8000000002	62.581471877736\\
29835.5000000004	62.5792677204414\\
29863.7999999996	62.5775601980987\\
29875.0999999997	62.5756480291345\\
29883.6	62.5739115303661\\
29885.5999999999	62.5720662390374\\
29904.4000000004	62.5738601213864\\
29918.5000000002	62.575788974083\\
29927.4999999997	62.5740119488365\\
29955.7999999998	62.572314635848\\
29963.9	62.574088906688\\
29970.4000000004	62.5758662684974\\
29976.8999999999	62.573973379591\\
29999.7	62.5757505050034\\
30020.2999999997	62.5777804426323\\
30023.6999999998	62.5796867818198\\
30033.9000000001	62.5817406938803\\
30048.5999999997	62.5837390635868\\
30072.6999999999	62.5814689686361\\
30077.0000000002	62.5834937543846\\
30089.9999999999	62.5853686096091\\
30102.4	62.5871605348393\\
30108.2000000001	62.5854000391919\\
30119.1999999998	62.5834597749616\\
30139.8999999998	62.5816191108162\\
30151.0999999996	62.5835787630343\\
30162.1999999999	62.5854129161238\\
30169.0000000005	62.5835706070118\\
30180.9000000005	62.5814254000861\\
30186.8999999996	62.5795872395402\\
30197.1000000002	62.5812989283775\\
30200.5000000003	62.5831937113832\\
30207.2999999998	62.5849957295232\\
30228.4000000002	62.5833236846023\\
30238.8999999996	62.5850722576805\\
30256.7000000003	62.5829565585257\\
30262.6999999996	62.5807922598041\\
30271.3	62.5825036172757\\
30281.3000000003	62.5842926681065\\
30285.7000000004	62.586096454444\\
30312.1000000001	62.5880008709365\\
30320.4	62.5899968668063\\
30326.3999999999	62.5881654658467\\
30356.0999999999	62.5898498494542\\
30359.5000000004	62.5917337514328\\
30367.7	62.589980176371\\
30384.1999999997	62.592193994925\\
30402.8999999999	62.5941518929053\\
30406.6999999997	62.5922491021745\\
30416.7999999996	62.5944787272865\\
30438.8000000005	62.5922750165085\\
30443.2999999997	62.5905779249361\\
30447.4000000005	62.5923310616635\\
30468.5000000001	62.5942773872117\\
30477	62.5925695030039\\
30493.0999999997	62.5900200700484\\
30506.7000000004	62.5880131642781\\
30519.1	62.5861752601641\\
30529.5999999999	62.5895439522825\\
30559.4999999999	62.5917878506263\\
30580.5000000001	62.5945861101483\\
30587.0000000001	62.5927269992905\\
30592.8999999999	62.5947112084464\\
30610.9000000002	62.5964522557251\\
30616.2000000004	62.5944349905116\\
30620.8	62.5961353193407\\
30625.5000000005	62.5933859255002\\
30645.6000000003	62.5950785917763\\
30651.5999999996	62.5932656263764\\
30664.0000000004	62.5950215398463\\
30677.6999999996	62.5967311867213\\
30682.1999999998	62.598631784449\\
30691.4000000001	62.6007200690745\\
30695.2999999997	62.5989561953908\\
30702.7000000003	62.5972224031684\\
30707.1000000004	62.5954173614006\\
30712.5000000003	62.593528389\\
30728.8999999998	62.5916886328875\\
30736.4000000005	62.5897548517235\\
30753.4000000002	62.5915749426895\\
30761.8999999996	62.5895585462584\\
30768.2000000004	62.5916933987253\\
30785.6000000001	62.5936717372027\\
30806.4000000004	62.5958807394543\\
30809.3999999997	62.5978999983771\\
30816.0000000002	62.5961753758587\\
30833.0000000002	62.5944196334459\\
30857.8999999996	62.596085293927\\
30885.1	62.5979433515082\\
30890.6999999996	62.5998679218408\\
30896.6000000002	62.5979473536009\\
30902.6000000003	62.6006789050795\\
30920.2000000001	62.6025620708725\\
30927.2000000003	62.6006796584248\\
30931.8999999999	62.598926677874\\
30949.3999999997	62.5971340409377\\
30953.2999999998	62.5953853211602\\
30960.0000000002	62.5973430318378\\
30976.2000000004	62.5991483811817\\
30979.4999999999	62.5973221087425\\
30994.3999999998	62.5956218038684\\
31020.0999999996	62.5975976944056\\
31059.1000000003	62.5953018751288\\
31063.5000000001	62.5970589371483\\
31068.1999999998	62.5953141948546\\
31074.8000000003	62.5971443190485\\
31082.5000000003	62.5990103787971\\
31087.3000000001	62.6012468073882\\
31090.3000000005	62.5990659496179\\
31103.1999999998	62.6007529747647\\
31151	62.6026047234287\\
31163.3999999999	62.6046496702874\\
31186.2000000004	62.6028095670214\\
31215.2999999997	62.6046758971533\\
31238.8999999999	62.6028361983418\\
31250.5	62.6045579924865\\
31262.9000000005	62.6062757892717\\
31299.3000000003	62.6079094167939\\
31308.3	62.6100343677735\\
31329.8999999996	62.6077242259815\\
31342.3000000005	62.6094364183981\\
31346.9000000005	62.6075860528918\\
31351	62.610243340744\\
31360.4000000003	62.608376779707\\
31384.0999999998	62.606661950918\\
31390.4000000005	62.6087510552556\\
31395.8999999997	62.6105236335839\\
31410.1999999999	62.6122641299192\\
31428.6999999996	62.6142264419895\\
31439.2000000001	62.6123991310239\\
31443.8999999997	62.6106729423738\\
31472.3999999999	62.6124394312495\\
31481.9000000003	62.6141922368337\\
31491.6000000001	62.612370878676\\
31502.7999999997	62.6142355148256\\
31507.4000000002	62.6162024914703\\
31512.1	62.6182875203889\\
31525.0000000005	62.6164548248856\\
31578.4999999997	62.6180387984268\\
31588.2999999996	62.619822466475\\
31600.9999999997	62.6215543129828\\
31608.0999999999	62.6198265007182\\
31648.7999999997	62.6214497186316\\
31689.5000000004	62.6195975966879\\
31694.2	62.6178839728279\\
31708.5	62.6158203137319\\
31752.0000000001	62.617905587347\\
31768.8999999999	62.6160722717114\\
31780.3000000001	62.6178399264956\\
31798.2000000005	62.6200142774928\\
31802.0999999999	62.6217683053374\\
31827.3999999999	62.6200612677716\\
31835.2999999999	62.6183431023326\\
31840.2000000006	62.6165582610717\\
31847.2999999999	62.6148445398997\\
31861.2000000002	62.6167168320188\\
31894.6999999996	62.6149090133815\\
31918.3	62.6131009073137\\
31928.6000000004	62.6113809832533\\
31942.7999999999	62.6132880859283\\
31953.1000000003	62.6150119549842\\
31958.2999999995	62.6132722539301\\
31967.4	62.6151560178306\\
31976.9000000004	62.6168808831348\\
31992.7000000003	62.6147133104949\\
31997.7	62.6164923838514\\
32022.5999999996	62.6146452360357\\
32031.9999999995	62.61281651843\\
32047.6000000002	62.6110454104351\\
32054.0000000002	62.6091514034086\\
32063.9999999996	62.6111445510711\\
32069.4999999996	62.6128794871155\\
32086.3000000003	62.6146903360926\\
32110.2000000005	62.6163567453434\\
32146.2	62.6180929064931\\
32150.9000000001	62.6160928120432\\
32185.3999999997	62.6142206894409\\
32194.7999999997	62.6170604660987\\
32207.8999999999	62.6151887729757\\
32218.1999999996	62.6134836412846\\
32227.3999999997	62.6117446280351\\
32241.5	62.6097960398988\\
32259.5000000001	62.6080298577788\\
32263.5999999999	62.6062726841621\\
32266.2999999996	62.6044430119257\\
32279.7000000005	62.6026183557519\\
32282.4000000005	62.6048168512352\\
32297.1999999998	62.6030658909568\\
32323.7999999998	62.6013568907217\\
32344.1999999999	62.603302591183\\
32354.5000000002	62.6050082523042\\
32368.0999999997	62.6071267478575\\
32372.5000000004	62.6054132198217\\
32389.1999999995	62.6036993698536\\
32397.8999999997	62.6054077412186\\
32420.1000000006	62.6072633728355\\
32432.7000000004	62.6051404750746\\
32443.8	62.6031395732326\\
32447.9999999995	62.6012000702661\\
32457.9000000005	62.6030562573171\\
32470.5000000004	62.6012454343314\\
32480.9999999998	62.5994809289094\\
32484.5999999999	62.5977767995395\\
32495.4999999996	62.5995519393395\\
32504.8999999996	62.6017535763729\\
32511.9999999997	62.603461480495\\
32515.8000000001	62.6013734818966\\
32522.6	62.6033508903012\\
32527.8000000005	62.6053326528918\\
32548.7999999998	62.6036517363106\\
32589.3000000005	62.6062462027531\\
32592.6000000004	62.6045095987752\\
32594.4000000003	62.6062679286383\\
32598.1000000005	62.6043769287875\\
32605.9000000004	62.6025884806477\\
32620.5000000001	62.6006265979167\\
32630.3999999999	62.6024731462895\\
32634.8000000005	62.6007740179991\\
32656.6000000004	62.6024674875294\\
32668.2999999999	62.6042291633505\\
32672.0999999996	62.6024571348119\\
32706.2000000002	62.6007833353207\\
32737.1999999997	62.6029024995953\\
32743.0000000001	62.600975472695\\
32758.3000000001	62.6031796424734\\
32776.4000000004	62.6049151068601\\
32788.5000000003	62.6068206632793\\
32797.3000000001	62.6049625884979\\
32818.4999999997	62.6032189063519\\
32840.0999999998	62.6052825500454\\
32847.0999999997	62.6035095837697\\
32849.6999999998	62.6052517823548\\
32866.8	62.6073648564361\\
32871.7000000004	62.6056376590269\\
32891.6000000003	62.6075879325179\\
32898.2000000002	62.6093141590903\\
32912.5	62.6112795707419\\
32914.9999999999	62.6095621766302\\
32918.9000000006	62.6112579361463\\
32943.2000000002	62.6133386758461\\
32946.2999999997	62.6113930505306\\
32953.4000000005	62.6133794589346\\
32958.8000000003	62.6152571839473\\
32966.0000000002	62.6134119595584\\
32975.1000000004	62.6155413765497\\
32977.5000000004	62.6134103148804\\
33001.4999999998	62.6157519635412\\
33013.4000000002	62.6131733987611\\
33025.1999999995	62.6110890741341\\
33039.1000000002	62.6128961960338\\
33047.4999999995	62.6109006402886\\
33071.3999999996	62.60919522852\\
33075.2000000005	62.6074442257515\\
33088.5999999995	62.6092895761997\\
33135.8000000002	62.6076249626538\\
33157.1000000005	62.6099308747421\\
33200.9000000005	62.6116683232433\\
33211.6000000002	62.6134765760259\\
33235.8000000003	62.6154248869445\\
33249.4000000001	62.6174829696687\\
33266.7000000001	62.6155806990754\\
33285.7999999996	62.6181055642179\\
33300.3000000005	62.616364968589\\
33319.5999999997	62.6143092524843\\
33324.1000000004	62.6160567995631\\
33329.7000000005	62.6139370773302\\
33353.5000000004	62.6157296363811\\
33397.2999999999	62.6174492625175\\
33407.3000000002	62.6193597825632\\
33415.9000000003	62.6176083313383\\
33466.9999999999	62.6158227034909\\
33490.0000000004	62.6176093830714\\
33512.2000000006	62.6158156855841\\
33521.1000000004	62.613808574872\\
33543.4000000004	62.6121305170898\\
33554.4000000004	62.6139563992907\\
33577.3999999998	62.6115702479339\\
33582.1999999999	62.6097676454561\\
33588.6999999996	62.6080717381984\\
33615.9000000004	62.6100666349357\\
33638.4000000005	62.6083208228667\\
33653.2999999998	62.6064528398319\\
33662.0999999998	62.6043455270303\\
33679.1000000003	62.6062970616879\\
33685.0999999996	62.6082077588971\\
33712.0999999997	62.6064154816357\\
33720.2000000003	62.6082804720005\\
33727.6999999997	62.6062180159987\\
33745.8999999996	62.6080128015172\\
33753.9000000001	62.6059133732298\\
33761.4999999998	62.6078147955073\\
33771.6000000003	62.6095221738912\\
33773.1999999995	62.6077404340115\\
33779.2999999997	62.609460203556\\
33785.2000000005	62.6077021663268\\
33793.5000000003	62.6059372188817\\
33810.1000000006	62.6077337608177\\
33843.8999999998	62.6096206122208\\
33855.9000000005	62.6078095463138\\
33865.5	62.6095506945101\\
33893.3000000004	62.6074693008078\\
33913.2000000002	62.6093597497147\\
33940.3000000002	62.6115190156863\\
33951.3999999996	62.6096048775459\\
33961.1999999995	62.607732919529\\
33978.2999999998	62.60977562216\\
33979.7000000002	62.6080789174745\\
33996.6000000001	62.6063706183247\\
34010.2000000003	62.6080922544053\\
34019.6000000004	62.606372190231\\
34034.7999999999	62.6080875806892\\
34045.1999999999	62.6101106466972\\
34056.2000000005	62.6080930694174\\
34073.0999999996	62.6063885986641\\
34076.6999999996	62.6082848154756\\
34080.4999999997	62.6065855648081\\
34087.9999999995	62.6045452811978\\
34114.2999999996	62.6025959712028\\
34121.1000000005	62.6047735718556\\
34129.6999999997	62.6068714144238\\
34134.9000000001	62.6087593379229\\
34141.3999999997	62.6067981781703\\
34154.4000000005	62.6090266289947\\
34159.0999999997	62.6068526194993\\
34177.7999999999	62.6047826226889\\
34199.8999999999	62.6029239766082\\
34205.0000000004	62.6011910504574\\
34207.9	62.6028999064546\\
34221.0000000002	62.6011437388044\\
34226.4999999999	62.6030631146535\\
34233.1999999995	62.6048321371296\\
34254.0999999996	62.6031260400185\\
34260.3000000005	62.601429055119\\
34278.8000000001	62.5997333636727\\
34285.3999999997	62.5978912368202\\
34291.2000000004	62.5960520598519\\
34297.5000000004	62.5985491696212\\
34313.8000000004	62.596498794949\\
34335.4999999998	62.5947995666306\\
34365.9000000005	62.5964616190421\\
34376.4000000002	62.594504967056\\
34388.0000000002	62.5963632768312\\
34389.8000000002	62.5942500559757\\
34405.9000000004	62.5960588269488\\
34421.7000000003	62.5943442818214\\
34437.6	62.5962244865366\\
34442.7000000002	62.5979885491307\\
34446.1	62.5961644535537\\
34472.5000000002	62.5943502955971\\
34488.4000000004	62.5921684039607\\
34504.7000000003	62.5941897938838\\
34521.7000000001	62.5960986970552\\
34567.4999999999	62.5944526088013\\
34580.8000000005	62.5923558958847\\
34617.0999999999	62.5905619171972\\
34627.7000000002	62.5924834959195\\
34634.0999999995	62.5941987977202\\
34654.4000000006	62.5959110649411\\
34663.5000000004	62.5941910245906\\
34668.6000000001	62.5965207809926\\
34676.4	62.5983014433406\\
34688.2	62.6000697641568\\
34718.4000000002	62.6017829111281\\
34725.0999999999	62.6000714178752\\
34732.3000000005	62.6017781667838\\
34736.8000000002	62.6037441452749\\
34755.3000000006	62.6055231705001\\
34776.4000000004	62.6037697870689\\
34780.7000000006	62.6055179869353\\
34790.7000000005	62.6036193476436\\
34863.1000000003	62.6052685926708\\
34867.3999999997	62.6035706603571\\
34871.2000000002	62.6053516788878\\
34873.2999999998	62.6035889818601\\
34889.8999999999	62.6016050444253\\
34899.6999999997	62.6035106218374\\
34902.4999999996	62.6016399924361\\
34936.0999999998	62.6035458922264\\
34949.7999999996	62.6053293428593\\
34969.6999999999	62.6031604412951\\
34980.4999999999	62.6012704184605\\
34987.6999999995	62.5995346949508\\
34992.4999999998	62.5978063933517\\
35013.4000000002	62.5995687377726\\
35031.2000000001	62.5977340264278\\
35035.2999999998	62.5995421773406\\
35047.4000000006	62.601326770811\\
35050.2999999998	62.5995709036131\\
35071.8000000001	62.6025393548682\\
35077.4000000003	62.6008124866367\\
35115.7000000004	62.5991718827422\\
35133.8999999999	62.597199294131\\
35182.4999999995	62.5991257041833\\
35196.8000000003	62.6009676988598\\
35202.5999999996	62.5991756314146\\
35218.4000000005	62.600905773954\\
35234.9000000005	62.5991201929899\\
35248.8999999997	62.5969531050526\\
35260.7000000002	62.5950063526636\\
35301.0999999995	62.5933396031863\\
35349.1999999995	62.5913384423454\\
35386.3000000006	62.5895824384509\\
35404.1000000003	62.5914439529773\\
35426.4	62.5932564605592\\
35459.8000000003	62.5949311757788\\
35467.9000000004	62.5967068907184\\
35484.3999999996	62.5985993884654\\
35491.8	62.5965361110563\\
35506.4999999999	62.5948415224212\\
35533.2999999996	62.5926593008268\\
35560.5000000001	62.5909011658971\\
35585.0999999998	62.5892224857525\\
35596.7000000002	62.5910194174757\\
35605.8000000001	62.5927163756568\\
35608.5999999999	62.5908836885368\\
35625.8000000002	62.5890152950522\\
35657.3000000005	62.5909348410148\\
35698.2000000002	62.5892549505159\\
35729.6000000003	62.5910656960456\\
35734.3000000002	62.5892697232918\\
35741.8999999996	62.5910693301998\\
35745.5999999998	62.5932629658952\\
35757.9999999999	62.5914128547098\\
35774.5999999996	62.5931174824667\\
35788.6000000005	62.5912648405782\\
35794.7000000002	62.5895381452055\\
35801.1999999999	62.5913025504661\\
35811.3000000004	62.5895664509067\\
35814.4	62.5916877242458\\
35833.9	62.589719261037\\
35858.4	62.5879498584715\\
35868.1000000001	62.5897034141663\\
35885.8999999997	62.5915398762749\\
35927.9000000002	62.5899020262748\\
35945.6999999999	62.5917353348653\\
36012.8999999998	62.5898980923555\\
36041.0000000002	62.5915968158575\\
36053.4999999999	62.5934719417756\\
36060.2000000004	62.591270732634\\
36070.7000000006	62.5930115217849\\
36097.8999999995	62.5912792952518\\
36103.6	62.5891529123054\\
36116.9000000004	62.5874242046682\\
36138.7000000002	62.5892392663841\\
36159.2	62.5874947800428\\
36177.3	62.585757959389\\
36221.3999999996	62.5874135527242\\
36266.3000000004	62.5890631548761\\
36270.0000000002	62.5873653505229\\
36289.1999999995	62.5856657472037\\
36300.9000000001	62.5839508553483\\
36339.4000000006	62.5856162027546\\
36347.3999999996	62.5833963821446\\
36360.7000000004	62.581681371147\\
36387.2000000004	62.5798011943728\\
36396.8000000001	62.5817033868269\\
36427.4000000001	62.5837622675177\\
36440.9000000003	62.5860980763426\\
36448.9999999997	62.5878279573433\\
36465.2999999994	62.5859033494765\\
36470.5999999995	62.5877759406866\\
36489.8000000005	62.586085464745\\
36498.6999999998	62.5878111061186\\
36559.3000000001	62.5893750991537\\
36562.5999999998	62.5913841155057\\
36572.4	62.5896506938273\\
36601.5000000003	62.5879742962056\\
36615.1999999997	62.5896824551485\\
36626.6000000004	62.5914974595036\\
36643.4999999995	62.5931949917585\\
36654.6999999995	62.5950762246691\\
36669.1999999995	62.5967771405508\\
36686.1999999998	62.5950286619256\\
36697.1999999999	62.5931608047458\\
36699.7000000001	62.5948915252944\\
36706.4000000004	62.5930012395625\\
36714.3999999999	62.5948875784772\\
36729.8000000004	62.5928739256028\\
36738.6000000003	62.5909463317973\\
36762.4000000004	62.5890513430806\\
36772.4000000002	62.5910666938609\\
36780.0999999995	62.5929168411265\\
36788.8000000004	62.5946956826651\\
36804.2999999996	62.5927878188478\\
36806.9999999996	62.5909131662098\\
36814.5000000003	62.5928300185253\\
36847.9000000001	62.5949848024316\\
36865.2000000002	62.5932787743488\\
36911.0000000006	62.5949917504491\\
36922.0999999998	62.5970283460899\\
36934.1000000002	62.5991628355291\\
36965.2000000003	62.6008716282568\\
36977.2999999997	62.599046985456\\
36990.4999999996	62.5972544376139\\
36995.3999999998	62.5989647389547\\
37047.2999999997	62.6006143481054\\
37052.8000000001	62.5988789001671\\
37059.4000000005	62.6009525222952\\
37080.2999999999	62.5988392789722\\
37082.8000000003	62.6008214028569\\
37088.8000000006	62.6025576385388\\
37093.7999999998	62.6043635206867\\
37096.4000000005	62.6024018438397\\
37140.2999999997	62.6002950964449\\
37172.4999999997	62.5985268719433\\
37195.6000000001	62.6002468027218\\
37212.1999999998	62.5983881673533\\
37219.2000000003	62.6008549327903\\
37259.4	62.6028261248809\\
37275.2999999996	62.6010720206893\\
37282.8999999997	62.6030630582303\\
37301.9999999995	62.6047863257028\\
37308.1	62.6028594250058\\
37319.2000000005	62.6046040520589\\
37336.5999999997	62.6064970926729\\
37353.8999999997	62.6048080526851\\
37359.7000000002	62.6065985363947\\
37365.7	62.6085886024118\\
37379.1999999998	62.6103217556241\\
37394.6999999999	62.6084375367698\\
37430.9999999995	62.6067628255648\\
37444.2000000002	62.6084611008885\\
37500.4000000005	62.6100985320196\\
37518.7999999997	62.6081788112125\\
37523.7000000004	62.6101301040939\\
37527.4000000006	62.6082206381987\\
37533.9999999997	62.6099999733576\\
37542.3000000004	62.6121398738493\\
37568.9000000006	62.6103968697596\\
37574.0999999999	62.6086516812068\\
37589.5999999996	62.6105023450573\\
37593.0000000001	62.608563805592\\
37665.0000000006	62.6067632901545\\
37678.1999999998	62.6050007564036\\
37688.0000000006	62.6067644694214\\
37698.6000000006	62.6048113064907\\
37701.6999999999	62.6031117877661\\
37707.8000000003	62.601205582915\\
37712.2000000002	62.5994701993779\\
37784.3000000002	62.5977916812229\\
37799.8999999997	62.5960317460317\\
37855.8000000002	62.5976399979924\\
37865.8000000004	62.595897628209\\
37889.5000000006	62.5976521261771\\
37898.6000000003	62.5958146321642\\
37913.7000000001	62.5935675136758\\
37923.9999999997	62.5952890114729\\
37930.7000000006	62.5934596686598\\
37937.4	62.5916309719934\\
37960.7000000004	62.5898295083349\\
37982.4000000006	62.5917198709932\\
37995.8999999996	62.5900094746815\\
38036.2000000001	62.588369531211\\
38068.5000000002	62.5909542247417\\
38110.5999999994	62.592920098555\\
38113.5999999997	62.5948149877865\\
38127.1000000003	62.592847101282\\
38132.2999999997	62.5911298528286\\
38136.7000000006	62.5894149482914\\
38144.0000000006	62.5913312936994\\
38184.2999999999	62.5896963157729\\
38215.1000000001	62.5879231300634\\
38253.6999999995	62.5862005866084\\
38256.1000000003	62.5882863431287\\
38260.1000000005	62.5901066905034\\
38269.1999999997	62.5882887850052\\
38289.6000000002	62.586544162007\\
38373.7999999997	62.5881132749082\\
38422.0000000004	62.5861158031446\\
38431.5000000003	62.5878183578097\\
38443.1000000005	62.589742789362\\
38448.7999999997	62.5880064189093\\
38470.0000000003	62.5857484124034\\
38495.6999999998	62.5876069597203\\
38501.2999999994	62.589412333058\\
38509.4000000003	62.5876731715551\\
38519.6000000006	62.5858976056408\\
38543.8000000003	62.5842221466951\\
38579.3999999996	62.5860884666727\\
38589.5000000006	62.5878475029542\\
38602.5000000004	62.5859398071633\\
38617.1999999997	62.5880110727576\\
38625.1999999997	62.5861805604099\\
38650.7999999995	62.5879345629727\\
38660.0000000004	62.5862323170401\\
38664.8000000003	62.5841525517976\\
38675.0999999998	62.5861016879034\\
38678.7	62.5841546273411\\
38687.2000000002	62.5822944480489\\
38700.4000000002	62.5805868141238\\
38714.9999999997	62.5822999294849\\
38726.2	62.5843419071794\\
38744.6000000001	62.5822370543584\\
38755.0999999998	62.5802472958467\\
38771.2999999999	62.581954739834\\
38786.0999999996	62.580247613842\\
38800.3000000002	62.5782208430841\\
38812.1999999998	62.580419093947\\
38904.2999999998	62.5790399029416\\
38911.1999999996	62.5807927260205\\
38914.1999999995	62.5790519166476\\
38919.1000000005	62.5773397192131\\
38966.2000000001	62.5756102067684\\
39003.2000000006	62.5739360515649\\
39024.3000000005	62.5757218560695\\
39045.6000000004	62.5774413059569\\
39053.0999999995	62.5756660145647\\
39075.1000000006	62.5739599541397\\
39081.2000000004	62.5757075634639\\
39096.0000000005	62.5775972539461\\
39099.3000000005	62.5758963053141\\
39109.0000000003	62.5777632315753\\
39123.7999999999	62.575816828077\\
39127.1000000007	62.5776953117013\\
39130.9	62.5759627916486\\
39155.6000000004	62.574286757739\\
39170.1000000003	62.5720573292963\\
39202.3000000003	62.5701487664021\\
39207.4000000006	62.5681310973666\\
39211.9999999999	62.5699720239416\\
39245.5999999998	62.5716957526557\\
39279.4000000006	62.57360709785\\
39292.4999999999	62.5756503769158\\
39346.9	62.5773248278141\\
39350.7000000007	62.5756020207975\\
39417.8000000005	62.577154034081\\
39429.1000000001	62.5754516956976\\
39459.9999999999	62.577134878016\\
39504	62.5790740707927\\
39510.5000000004	62.5811807464326\\
39556.7999999999	62.5830133301649\\
39561.7999999999	62.5806647304604\\
39590.4999999995	62.5825322172435\\
39597.7000000001	62.584537524812\\
39604.2000000007	62.5863858217416\\
39613.6000000006	62.5846613671533\\
39625.6000000002	62.5828691985252\\
39637.4000000001	62.5849258908862\\
39691.0000000006	62.586574824079\\
39712.0999999996	62.584545807082\\
39736.1000000001	62.5827331249591\\
39750.9999999995	62.5809097106747\\
39764.4000000006	62.5791849514014\\
39775.7999999998	62.5811106725429\\
39824.3000000006	62.5794738903788\\
39830.4000000001	62.5814388471147\\
39842.3999999999	62.5831712367447\\
39858.5000000001	62.5814755159489\\
39872.8	62.5833586220215\\
39877.4999999997	62.5814993881277\\
39917.3000000004	62.5797271365369\\
39923.9999999998	62.5779917393254\\
39936.5000000002	62.5796888067587\\
39943.5999999999	62.5778282933229\\
39994.0000000002	62.5757299201632\\
39998.9999999997	62.5779080029301\\
40022.2000000003	62.5796118663845\\
40031.2	62.5777828848926\\
40047.1999999996	62.5760038754173\\
40055.7000000004	62.5777041027766\\
40066.7999999998	62.5798352255852\\
40085.1999999995	62.5817943236049\\
40091.5999999996	62.5797858409596\\
40129.7	62.5819216641997\\
40134.4999999996	62.5801677355698\\
40161.1000000002	62.5820443612243\\
40193.5000000006	62.5798634608495\\
40204.3000000006	62.5817074748037\\
40217.6999999998	62.5800018897105\\
40224.9	62.5819763828465\\
40227.2999999994	62.5802313845787\\
40241.0000000005	62.5822852755019\\
40280.1000000003	62.5840983907726\\
40286.4000000003	62.5820063793082\\
40291.2	62.5837339574548\\
40306.6999999997	62.585717546419\\
40314.1999999998	62.5839962494698\\
40324.4999999999	62.5861136874265\\
40339.5000000005	62.584408372914\\
40356.3000000001	62.586107779683\\
40367	62.5843818357028\\
40383.3000000004	62.5861121153741\\
40398.8	62.5878427382924\\
40405.3000000006	62.5896538581477\\
40410.6000000003	62.5878789528516\\
40420.3999999994	62.5860640021771\\
40491.5999999995	62.5844308833662\\
40498.3000000006	62.5861762439998\\
40501.1000000003	62.5843184893287\\
40511.3000000007	62.5860868792488\\
40543.4000000005	62.5880844031719\\
40558.4000000005	62.5898393678267\\
40566.4	62.5915472126015\\
40588.2000000007	62.5897118135029\\
40609.4000000006	62.5915118383629\\
40657.3000000003	62.5898360446069\\
40667.5000000001	62.5879078185091\\
40676.1000000006	62.5859839414695\\
40697.5000000002	62.5877201604026\\
40703.9000000001	62.5894261006289\\
40760.8999999996	62.5877677191433\\
40825.6999999995	62.5893430134866\\
40847.3999999997	62.5874288512149\\
40874.4000000001	62.5857197029933\\
40898.7000000007	62.5874108775808\\
40949.6	62.58922531789\\
40968.4000000001	62.5873536985733\\
40977.7999999996	62.589346940668\\
41009.0999999996	62.5876632560499\\
41015.5000000006	62.5856990998547\\
41106.5999999998	62.5871208343166\\
41147.1999999996	62.5887968347863\\
41179.0000000002	62.5870890815972\\
41205.6999999995	62.5853641963025\\
41235.0999999999	62.5870615396555\\
41258.8999999996	62.5890108824741\\
41286.7999999995	62.5907975653294\\
41301.9999999999	62.5890693209304\\
41312.6000000003	62.5872915592542\\
41322.9000000006	62.5891150206907\\
41329.5000000006	62.5909759591189\\
41340.8000000004	62.5927350396339\\
41349.3999999994	62.5944690987799\\
41356.7999999998	62.5924573650346\\
41363.9999999996	62.5907489828136\\
41398.3000000007	62.588892324341\\
41416.4000000002	62.5871331474171\\
41445.0999999997	62.5889125881887\\
41472.9000000007	62.5906011139778\\
41497.1000000001	62.5924158738421\\
41502.6000000006	62.5944818047982\\
41518.4000000005	62.5961920589617\\
41550.3000000003	62.5943432554199\\
41566.2999999996	62.592622887717\\
41576.9	62.5908555210814\\
41586.8	62.5889402672476\\
41606.2000000003	62.5907614952543\\
41634.2999999994	62.5888688200142\\
41637.8999999998	62.5906623757145\\
41656.8000000004	62.5925116847389\\
41706.3000000003	62.59087334318\\
41745.6000000007	62.592554442733\\
41751.1999999999	62.5944581366329\\
41805.6000000004	62.5926608094112\\
41826.8999999996	62.5909101776365\\
41852.9999999996	62.5891511023078\\
41862.5000000005	62.5873691552842\\
41868.8000000004	62.5851168767271\\
41879.5999999994	62.583304082885\\
41884.4999999997	62.5850551276603\\
41890.9999999998	62.5868024472979\\
41962.3999999995	62.5885016383676\\
41983.6000000005	62.5902433563502\\
41991.9999999995	62.5884392540502\\
42026.9999999996	62.5867594956588\\
42038.4000000001	62.5850113586355\\
42047.7000000001	62.5871032491593\\
42050.8000000004	62.5853430009821\\
42084.9000000006	62.5835808482832\\
42123.6999999998	62.5852843285743\\
42147.4000000001	62.5835458805386\\
42161.2000000003	62.5853567133841\\
42171.2999999996	62.587203649867\\
42178.6999999997	62.5854694775575\\
42189.2000000007	62.5871962796254\\
42216.6999999999	62.5855109814103\\
42236.0000000006	62.5879283361863\\
42241.0000000007	62.5862015903942\\
42257.1999999995	62.5842162182629\\
42264.0999999998	62.5825166452932\\
42288.9999999997	62.5842120073494\\
42294.8999999995	62.582338337865\\
42325.9999999997	62.5805354143188\\
42333.7000000005	62.5788377136\\
42349.3	62.577037691207\\
42376.7000000003	62.5750410602028\\
42429.8999999998	62.5731793542305\\
42448.1999999994	62.5749441084802\\
42458.5999999996	62.5768099353018\\
42472.7000000006	62.5751068919403\\
42500.1	62.5768819911436\\
42508.9999999994	62.5786008172368\\
42532.9999999995	62.5769106883815\\
42578.3000000004	62.5746857561581\\
42593.8000000002	62.5728097215798\\
42615.6000000006	62.5745910544705\\
42633.9	62.5728291973542\\
42657.7999999997	62.5745758698858\\
42690.9	62.5764212597503\\
42726.6000000006	62.5782005163048\\
42739.3000000002	62.5799613471411\\
42777.3000000007	62.5816435781511\\
42803.8000000006	62.5795780291048\\
42852.6000000003	62.581354267059\\
42881.1	62.579638629516\\
42884.3000000001	62.5814981671657\\
42895.6	62.5836622318787\\
42904.7000000004	62.5818090283605\\
42953.9999999996	62.5837812921235\\
42966.3999999999	62.5855026590483\\
42987.3999999995	62.5837743530096\\
43070.1999999995	62.5821041413689\\
43106.3000000007	62.5839782491695\\
43123.7999999995	62.5857123312131\\
43134.2000000003	62.5875463378332\\
43139.0000000004	62.5896228711308\\
43172.0999999999	62.5916677862143\\
43194.6999999999	62.5934140220582\\
43211.5000000001	62.5915263494062\\
43237.5000000002	62.5934371935538\\
43243.1999999994	62.5916615984441\\
43275.1999999997	62.5899762682177\\
43286.7000000003	62.5881331029321\\
43296.9999999995	62.5864087895032\\
43317.3999999998	62.5846366941767\\
43327.8999999994	62.582856351551\\
43336.8999999996	62.5846274546\\
43343.7000000004	62.5828838265219\\
43359.6000000001	62.5848426073059\\
43369.0999999995	62.58681276113\\
43396.9999999995	62.5885139790447\\
43406.2	62.5867673586553\\
43417.2000000002	62.5886455399115\\
43466.4000000003	62.5902706682158\\
43495.9	62.5919624793084\\
43526.3000000001	62.5937362152625\\
43539.0000000001	62.5954601725805\\
43567.1999999995	62.5974067706743\\
43575.8999999996	62.5956948779144\\
43588.6999999998	62.5975021106339\\
43602.5999999998	62.5954356037585\\
43615.0000000004	62.5936888829786\\
43642.1999999998	62.5954635754761\\
43645.6999999993	62.5936516228366\\
43661.1000000004	62.5953936217969\\
43769.8999999996	62.5969842357779\\
43789.8000000001	62.5986814311063\\
43815.6000000002	62.5969686664826\\
43840.2000000007	62.5951464748188\\
43847.4999999995	62.5970406590098\\
43883.4999999993	62.595365922577\\
43896.2000000002	62.5970753799295\\
43908.4000000003	62.5988134415888\\
43918.6	62.597025868252\\
43940.3999999994	62.5953277727836\\
43947.7999999999	62.5970751731027\\
43973.1999999995	62.5948018456654\\
43991.1999999999	62.593058172866\\
44017.0000000006	62.5947643075078\\
44025.9999999998	62.5928710469472\\
44035.2000000003	62.5911484649815\\
44080.0000000001	62.5894678097373\\
44129.8999999997	62.5878087468842\\
44148.5999999996	62.5859878093806\\
44191.7999999994	62.5881666097181\\
44200.4999999994	62.5898743455971\\
44213.7999999996	62.5916284245452\\
44262.7000000004	62.5934193046983\\
44285.4999999996	62.5916776559423\\
44291.9000000008	62.5898582136729\\
44312.7999999999	62.5880951145152\\
44365.6000000005	62.5861870769536\\
44382.2000000003	62.584408649394\\
44387.8999999994	62.5826800036046\\
44430.3000000001	62.5810256040909\\
44443.4000000002	62.5792298086334\\
44463.2000000003	62.5814997987104\\
44483.7999999994	62.5833166606345\\
44489.4	62.5851043504647\\
44502.2000000008	62.58328221238\\
44537.7000000006	62.5850401232212\\
44565.0000000001	62.5832770486322\\
44613.3	62.581645873213\\
44630.5999999999	62.5836027667054\\
44643.7000000006	62.5853981963901\\
44655.7000000003	62.5835837673942\\
44660.3999999996	62.5852822964365\\
44667.7999999993	62.5870031946879\\
44696.1000000002	62.5887659353592\\
44704.6999999997	62.5870152645801\\
44711.0999999997	62.5887920699959\\
44712.7999999997	62.5870833696763\\
44751.4999999999	62.5888236398252\\
44779.2000000004	62.5867309225468\\
44806.0999999995	62.5884364217452\\
44912.7999999999	62.5869182350728\\
44917.1999999994	62.5888020873917\\
44928.3000000005	62.586693494538\\
44943.8000000007	62.5844664125721\\
44960.2999999998	62.5826282684318\\
44972.1000000005	62.584663414287\\
45013.4000000005	62.5830028769147\\
45081.7000000001	62.5813521199242\\
45090.6999999997	62.5830546364225\\
45108.1000000001	62.5813045078278\\
45113.2999999997	62.5791893317728\\
45118.4000000008	62.5772133382094\\
45133.2999999995	62.5791542405403\\
45145.2999999997	62.5773611486442\\
45213.4999999994	62.575640957588\\
45239.5000000003	62.5739396457971\\
45254.6999999996	62.5721470429656\\
45279.5000000008	62.5738743275117\\
45364.3	62.5754997310667\\
45426.9999999994	62.573882110018\\
45449.3000000007	62.5717831258498\\
45493.4000000001	62.5700374778815\\
45524.6999999995	62.5718289811268\\
45536.8	62.5701354286304\\
45579.0999999996	62.5719626496297\\
45603.4999999995	62.5702795393346\\
45636.3000000002	62.568476041055\\
45651.3000000002	62.5667558935761\\
45659.8000000003	62.5686871850354\\
45672.0000000004	62.5705846676636\\
45693.2000000003	62.5724121479517\\
45726.7000000006	62.574245300349\\
45738.7000000005	62.5724767593378\\
45753.3999999997	62.5742292939338\\
45805.5000000002	62.5759732434462\\
45856.3999999996	62.5776062281247\\
45903.7000000008	62.5793507291335\\
45927.6999999997	62.5775673992658\\
45978.1000000005	62.5792223271029\\
46017.7000000004	62.581001264728\\
46046.5000000001	62.582687972619\\
46073.8999999994	62.5845379172635\\
46099.4000000007	62.5826744324776\\
46102.6000000007	62.5809334377379\\
46112.7999999993	62.5827046227845\\
46120.1999999995	62.5845885651221\\
46124.9999999994	62.5826285471468\\
46129.4999999999	62.5843276334501\\
46141.1000000002	62.5863653307673\\
46174.2000000002	62.5882796274118\\
};
\addplot [color=mycolor3]
  table[row sep=crcr]{%
46174.2000000002	62.5882796274118\\
46194.9000000002	62.5864271025003\\
46217.3000000002	62.5846542644112\\
46235.6000000005	62.5828093875512\\
46253.8999999995	62.5848575258356\\
46269.4	62.5831271139736\\
46302.6999999995	62.5854160007602\\
46315.4000000006	62.5874707171465\\
46373.3000000004	62.5891135866682\\
46388.7999999994	62.5873862066141\\
46421.4	62.5856553536616\\
46432.0999999998	62.5839395936441\\
46450.4999999999	62.5856286032904\\
46458.7999999999	62.5839182589342\\
46475.5999999996	62.5858244200733\\
46531.6	62.5840018739913\\
46562.9000000006	62.5823078409896\\
46576.0000000005	62.580593909752\\
46585.3999999995	62.5784847216409\\
46602.9000000005	62.5803059888848\\
46621.1999999999	62.5784780776169\\
46666.2000000008	62.5766345306999\\
46673.1999999995	62.5783906430443\\
46683.7000000002	62.5801669958315\\
46694.3999999997	62.5818886592639\\
46717.6000000008	62.5801355803047\\
46763.6000000007	62.5784529453401\\
46819.7999999996	62.5802276382478\\
46907.1000000003	62.581863765051\\
46916.5999999998	62.5798489663595\\
46927.2999999994	62.581562157716\\
46940.1000000003	62.5836702868756\\
46985.1999999997	62.5819139177573\\
47000.3000000008	62.5798929370814\\
47051.3000000004	62.5781592046145\\
47144.9000000005	62.576519249125\\
47159.6	62.5784303123218\\
47173.9999999994	62.576710525479\\
47202.9000000005	62.5784378111561\\
47211.9000000005	62.5804880115225\\
47248.3000000003	62.5830715960752\\
47273.1000000005	62.5851433793354\\
47306.4999999994	62.5868272080428\\
47318.2000000003	62.5888926694323\\
47334.2999999997	62.590631760411\\
47361.3000000007	62.592533159915\\
47509.5000000007	62.5936652802802\\
47525.2000000005	62.5919247222006\\
47611.0000000006	62.5902363104402\\
47626.2000000004	62.5920972235928\\
47688.3000000008	62.5904832202381\\
47699.4000000002	62.5922703592281\\
47732.3999999999	62.5940397004138\\
47756.9	62.5918713487028\\
47784.2000000002	62.5935715287239\\
47804.5999999997	62.5911259771529\\
47811.5999999996	62.5928381546776\\
47873.1000000003	62.5951471804684\\
47937.5	62.5934548246053\\
47954.6000000007	62.5913622648041\\
47958.5000000006	62.5894000241875\\
48013.7999999993	62.5875007029215\\
48055.4999999994	62.5893756398838\\
48107.9000000002	62.5877192982456\\
48138.7000000001	62.5894289014267\\
48163.4000000002	62.5913814403023\\
48171.1000000005	62.589680140831\\
48265.7000000002	62.5882094567996\\
48278.3999999995	62.5864515260416\\
48315.4	62.5881963345096\\
48340.6000000006	62.5899087104655\\
48349.0999999997	62.5919353370894\\
48360.6000000001	62.5938003378777\\
48404.5000000003	62.5954971221743\\
48479.1999999997	62.5972322207623\\
48493.6999999998	62.5989301725169\\
48507.9999999994	62.6010913641227\\
48513.8000000007	62.5993787347544\\
48542.3999999995	62.5975176391822\\
48619.3	62.5957950941394\\
48653.3000000002	62.5974752021442\\
48692.7999999993	62.5957788507154\\
48724.1999999997	62.597512945286\\
48755.8999999996	62.5957830831077\\
48795.8999999993	62.5940650873022\\
48820.9000000001	62.5958091804756\\
48830.6999999997	62.5941004448012\\
48841.2000000001	62.5923142913886\\
48912.1000000002	62.5939949542241\\
48922.8999999999	62.5922367802465\\
48999.6000000001	62.5940566983063\\
49013.3999999997	62.5958154386037\\
49107.4000000008	62.5973629282696\\
49137.6999999998	62.5990581588919\\
49146.8999999994	62.6007691212078\\
49171.4999999995	62.6025998747244\\
49241.0999999998	62.6008301991016\\
49291.6000000007	62.5991799836484\\
49303.8999999997	62.60120882687\\
49315.2000000008	62.6030866688431\\
49363.3999999999	62.6051637343381\\
49382.3999999993	62.6033513896623\\
49419.2000000005	62.6052979301609\\
49424.3999999998	62.6070066465012\\
49458.3999999997	62.6088538876027\\
49500.5000000003	62.6105542155045\\
49538.9999999993	62.6087272477699\\
49557.2999999998	62.6104275042678\\
49577.4	62.6086430336342\\
49618.1000000001	62.6068660289974\\
49641.3999999998	62.6050784122156\\
49653.1000000008	62.6068410495195\\
49659.0000000002	62.604839797741\\
49678.8999999998	62.6065339479458\\
49683.4999999998	62.6047629398836\\
49726.9000000002	62.6030124479659\\
49736.4999999997	62.6011830322137\\
49759.6000000004	62.6028693902897\\
49769.0000000005	62.6010918421269\\
49775.3000000009	62.5991955865749\\
49860.4999999996	62.6007308375752\\
49895.8999999993	62.6024130190797\\
49918.0999999999	62.6006146054946\\
49959.5000000004	62.602382725242\\
49973.7000000004	62.6006027158231\\
50012.9000000002	62.5987243316738\\
50032.1999999996	62.6005600382153\\
50049.1999999994	62.6022741576805\\
50070.6000000007	62.6004829171551\\
50086.1999999995	62.5987545496473\\
50112.4000000006	62.6005487652781\\
50134.5999999992	62.5987589434065\\
50163.3999999998	62.6004963768477\\
50184.7000000008	62.6022221867976\\
50205.0000000009	62.6004131054415\\
50226.4999999994	62.6022864378636\\
50251.9	62.600493512696\\
50327.3000000003	62.6020815698804\\
50334.8999999997	62.6003774709447\\
50385.2999999994	62.5986893028536\\
50396.7000000002	62.6004032002032\\
50415.6999999996	62.6022001039357\\
50494.7000000008	62.6038720818777\\
50505.6000000002	62.6020429377278\\
50643.8000000004	62.6037884128197\\
50651.2000000004	62.6019470378845\\
50661.8999999994	62.6037266590344\\
50682.1999999996	62.6019340085197\\
50716.8999999999	62.5999171875308\\
50731.8000000007	62.6016372341663\\
50762.9000000005	62.5999251423281\\
50837.1000000002	62.5982548212726\\
50886.7000000008	62.6001241972378\\
50896.5000000005	62.6018240904108\\
51011.3999999998	62.6003940287974\\
51036.0000000006	62.5986311649989\\
51066.8999999996	62.6003093974582\\
51084.9000000004	62.6021336987374\\
51122.4000000002	62.6004205584625\\
51134.5	62.5987100710673\\
51165.6000000006	62.6009221021114\\
51179.7000000001	62.6026283807283\\
51213.4999999993	62.6044644391334\\
51258.5999999996	62.6061527116373\\
51354.0999999996	62.6079269076337\\
51358.3999999996	62.6096945977784\\
51367.7999999996	62.6079711259366\\
51395.5999999999	62.6062102471608\\
51429.8999999996	62.604510985806\\
51437.5	62.6063424421046\\
51461.9999999994	62.6045186651924\\
51482.0999999997	62.6062988761164\\
51530.8	62.604573178423\\
51589.5000000007	62.6025400468312\\
51606.2999999994	62.6007626960997\\
51640.4000000005	62.6026084178116\\
51650.2999999996	62.6047426544615\\
51741.4	62.6031328817294\\
51786.7000000003	62.6014737346196\\
51795.6000000002	62.6032662943063\\
51804.9	62.6051539426696\\
51853.8999999998	62.6034635707949\\
51866.8000000005	62.6052453491533\\
51942.0999999995	62.6036247983335\\
51953.3000000006	62.6053347807843\\
51964.4999999995	62.6070440261255\\
51977.3000000007	62.6095187523809\\
52000.6000000008	62.6076187435938\\
52043.3999999997	62.6093556351898\\
52174.2999999993	62.6107822993652\\
52199.1000000005	62.609005502\\
52256.0999999996	62.6107524083267\\
52275.8999999997	62.612479914301\\
52303.9999999995	62.6107704749723\\
52318.3000000004	62.6089482858803\\
52365.0999999999	62.610665098195\\
52446.0999999995	62.6087304704631\\
52467.6000000009	62.6107109707496\\
52472.7000000002	62.6126297815249\\
52499.4999999998	62.6107627486686\\
52511.4999999995	62.6090235300391\\
52553.5999999993	62.607199873653\\
52594.9999999993	62.6054518386694\\
52628.2000000009	62.6071904279637\\
52647.6999999993	62.6088839419691\\
52683.7999999995	62.6107786249689\\
52702.0000000004	62.6090800935826\\
52732.4999999995	62.6073813921559\\
52829.5999999992	62.6058826758433\\
52949.7000000002	62.6074130591617\\
53050.7999999992	62.6089284064926\\
53122.3999999997	62.6071815144242\\
53170.8000000003	62.6092467872464\\
53208.4999999999	62.611119255158\\
53267.1000000004	62.6094482157876\\
53295.3999999994	62.6077248548189\\
53407.2000000001	62.6092313223099\\
53483.3000000003	62.6108287805188\\
53551.4000000003	62.6126252299189\\
53649.6000000008	62.6141432291327\\
53714.5000000009	62.6123996083002\\
53737.9000000001	62.6141650228888\\
53751.7000000005	62.6124148400612\\
53792.2999999996	62.6105174708695\\
53844.5999999992	62.6122905318444\\
53896.8000000004	62.613990786112\\
53926.5999999993	62.6122125032683\\
53936.6999999997	62.6104996959404\\
53951.6000000007	62.6087778513003\\
53962.0999999998	62.6104940124754\\
54015.2999999995	62.6121439441344\\
54081.3999999998	62.6138328263824\\
54101.5999999995	62.6155924860032\\
54147.5999999999	62.6176181075096\\
54162.4000000001	62.6193399492269\\
54169.4999999992	62.621285739603\\
54181.7999999992	62.6194356417918\\
54214.3999999998	62.6177498639663\\
54225.4000000005	62.6198006472969\\
54242.6000000004	62.6215140470515\\
54255.9999999998	62.6233732243932\\
54288.9999999998	62.6250941717582\\
54356.1999999996	62.6234677489086\\
54406.5999999999	62.625191382679\\
54446.7999999993	62.6268896851795\\
54453.2999999996	62.6251069722001\\
54478.4999999994	62.6231217395454\\
54484.8000000004	62.6213868429602\\
54553.4999999997	62.6191488737682\\
54566.8999999996	62.6173328202027\\
54597.6999999996	62.6191897842038\\
54653.3999999993	62.6173987027363\\
54706.3	62.6191816679584\\
54718.8000000008	62.6174868281343\\
54765.4	62.615697838968\\
54770.3000000009	62.6139301520529\\
54839.5999999995	62.6155868832251\\
54880.2000000005	62.6137247792013\\
54934.7999999994	62.6153865757469\\
54990.2999999993	62.6171113503448\\
55027.9000000009	62.619575488842\\
55091.1999999996	62.6215028507224\\
55182.5	62.6199200472613\\
55191.8	62.6216890521979\\
55242.9000000008	62.6198794417392\\
55247.7000000008	62.6180589996344\\
55294.4000000009	62.6163542486142\\
55318.9999999999	62.6181553929836\\
55346.7000000002	62.6198804628271\\
55384.8000000001	62.6181504345047\\
55404.9000000007	62.6161898745601\\
55446.6000000001	62.6143665898962\\
55492.9999999995	62.6160729892545\\
55526.4000000007	62.6180292292869\\
55561.8000000004	62.6161092403248\\
55587.9000000007	62.6144131827013\\
55629.7999999993	62.612731642516\\
55674.7999999993	62.6143917636134\\
55698.1999999992	62.6126829723708\\
55712.4000000006	62.6109041956473\\
55727.1999999994	62.6088111212996\\
55796.3000000002	62.6070857618054\\
55865.9999999996	62.6088092778984\\
55878.8999999994	62.6070616868591\\
55941.9000000005	62.6053769976047\\
56020.6000000004	62.6036447241823\\
56049.3999999999	62.6053756054916\\
56091.3000000008	62.6035363709945\\
56116.1000000003	62.6052726307198\\
56172.5000000003	62.6034757159184\\
56210.6000000003	62.6017822229576\\
56233.4000000008	62.60396382939\\
56247.4000000008	62.602248988844\\
56384.0999999999	62.6008704566173\\
56404.9999999995	62.5991266747156\\
56467.4000000007	62.5974232965865\\
56479.5000000006	62.5953441596612\\
56508.4999999993	62.5936583104165\\
56522.2000000006	62.5919327415905\\
56531.3000000009	62.5901711261352\\
56546.2000000002	62.5918937224894\\
56590.9000000006	62.5901645137919\\
56607.3999999992	62.5922360111293\\
56630.4000000002	62.5940085289729\\
56656.5000000003	62.595708178747\\
56661.6999999996	62.5938462950348\\
56682.9000000007	62.5921352080871\\
56704.4999999992	62.5903365864498\\
56758.5000000001	62.588576885265\\
56775.7999999995	62.5867665682094\\
56918.3999999992	62.5880864745206\\
56994.8000000009	62.5864770356646\\
57027.9999999994	62.584585493818\\
57064.1999999998	62.5829108567003\\
57078.6000000009	62.5846419067008\\
57122.2000000004	62.5829142033847\\
57266.1000000008	62.5842119784445\\
57286.6000000002	62.5859056290553\\
57351.9000000006	62.5877040033478\\
57372.7999999997	62.5859944329117\\
57503.4999999997	62.5846033987437\\
57520.1999999999	62.5864260096001\\
57533.9000000008	62.5847325059965\\
57566.9999999999	62.586442603501\\
57647.3000000006	62.5847826614904\\
57664.6999999995	62.5830662726655\\
57710.7000000002	62.5813539233558\\
57777.4000000003	62.5830124183289\\
57783.0000000009	62.581273763436\\
57832.1000000003	62.5833013442338\\
57867.0000000005	62.5813285960416\\
57886.8	62.5796164589916\\
57907.6000000005	62.5778609753107\\
58014.2	62.5793985276044\\
58066.5000000007	62.5810707015737\\
58085.500000001	62.5793656259038\\
58109.4999999999	62.581053732946\\
58178.7000000005	62.5829339897007\\
58224.0999999992	62.5846297587601\\
58246.000000001	62.5829025462649\\
58269.0000000004	62.5811965518603\\
58305.3999999993	62.579345001758\\
58387.5	62.5776706012921\\
58418.7999999998	62.5794049528492\\
58542.2000000009	62.5810738559981\\
58555.8000000007	62.5791764792275\\
58570.0999999996	62.5773857695552\\
58719.6	62.5791684903022\\
58767.1999999993	62.5774878206077\\
58839.2000000001	62.5758634110195\\
58858.5999999995	62.5776648142076\\
58873.0000000006	62.575947249253\\
58884.5000000001	62.5778217055053\\
58970.5000000004	62.5760972416764\\
59017.7999999996	62.5779636347617\\
59033.6000000004	62.5800178542087\\
59108.9000000003	62.5782875704207\\
59147.3999999992	62.5766093241473\\
59176.8000000008	62.5783033582361\\
59237.9999999992	62.5764499536616\\
59291.2999999998	62.5782153904276\\
59308.9999999996	62.5799413580715\\
59376.1999999992	62.5817034742818\\
59420.2999999996	62.5835571621867\\
59448.4999999998	62.5817933475305\\
59490.5999999999	62.5798654243436\\
59577.9999999999	62.5815526174887\\
59631.200000001	62.5798867373343\\
59683.2999999996	62.5782043248206\\
59723.4000000007	62.5763727845823\\
59773.4999999996	62.5781281368363\\
59897.1000000007	62.5762139131712\\
59923.8999999993	62.5782658033509\\
59951.0999999996	62.5800651196306\\
59982.1999999993	62.5821283945431\\
59992.2000000002	62.5803644801083\\
60040.0000000001	62.5786765844827\\
60083.6000000001	62.5808663580971\\
60090.5999999996	62.5790679755769\\
60173.1999999999	62.5774222121772\\
60196.9000000003	62.5791982989186\\
60226.0000000001	62.5810072377258\\
60255.0999999997	62.5828144292941\\
60273.2000000001	62.5807778900442\\
60341.3999999991	62.5826338423805\\
60411.1999999997	62.5810071956737\\
60445.600000001	62.5827809091465\\
60471.6999999991	62.5810708462457\\
60512.3999999994	62.5827721545135\\
60554.1999999993	62.5810223221142\\
60666.7	62.5792031226305\\
60691.1999999995	62.5809630045822\\
60715.6999999991	62.5792627289766\\
60728.5000000003	62.5772700177511\\
60797.7000000005	62.5756195125482\\
60902.3000000004	62.5737245166036\\
60931.400000001	62.5756792463668\\
60967.7000000002	62.5738504587668\\
61026.4999999997	62.5720587415979\\
61110.1999999992	62.5737396150895\\
61182.5999999995	62.5753685273778\\
61192.2999999998	62.5772154712023\\
61229.1000000009	62.5789655915805\\
61314.9999999996	62.5811586379212\\
61324.600000001	62.5791891358621\\
61344.2000000001	62.5774521838215\\
61363.3000000001	62.5755743651753\\
61413.3999999995	62.5772834962997\\
61427.8999999992	62.579117015042\\
61492.0000000006	62.5774693009346\\
61516.6	62.5755282711849\\
61527.0000000005	62.5773033346282\\
61547.9000000007	62.5755507896276\\
61568.9000000002	62.5738602218649\\
61635.8000000009	62.5755444473107\\
61753.2999999991	62.5740445060515\\
61764.0000000007	62.5722709470388\\
61829.100000001	62.5705977111138\\
61970.8000000002	62.5721104582958\\
62012.9000000001	62.5738151677874\\
62055.0000000005	62.5719723278183\\
62083.7999999994	62.5737107366\\
62115.7000000008	62.5720026144717\\
62157.7999999997	62.5703249305398\\
62220.9000000003	62.5685861686569\\
62244.0000000001	62.5667332325473\\
62265.2999999994	62.5649236975913\\
62323.0999999996	62.5667488190594\\
62341.7000000008	62.5649243364805\\
62383.1999999996	62.5632180407256\\
62424.2000000001	62.5615345306235\\
62458.3	62.5632420939377\\
62475.5000000009	62.5615440267881\\
62619.2999999991	62.5600053657493\\
62655.3999999991	62.5582750117707\\
62710.4000000004	62.560017859848\\
62722.5000000007	62.5619792546865\\
62768.5000000006	62.5601017069044\\
62789.2999999996	62.5618336853036\\
62805.6000000007	62.5600861068343\\
62851.4000000003	62.5584114937591\\
62862.2000000009	62.5602308537868\\
62900.900000001	62.5621214289121\\
62921.1999999995	62.5603730374293\\
62946.5000000003	62.5585814007429\\
63048.6999999996	62.5602707743843\\
63116.3000000006	62.5620282525619\\
63200.0999999998	62.560403289863\\
63259.9000000003	62.5624407208347\\
63300.1999999992	62.564158463704\\
63380.9999999993	62.5624989152918\\
63407.6	62.560540754514\\
63440.4000000002	62.5625586179176\\
63458.9999999991	62.5607674864598\\
63484.4999999994	62.5589513047259\\
63573.5000000009	62.5607170271937\\
63617	62.5624242538563\\
63633.1000000009	62.5605815831987\\
63685.899999999	62.5622585811638\\
63702.3999999991	62.5604960558848\\
63763.9999999993	62.5621627216569\\
63785.4	62.5604565300891\\
63824.7000000008	62.558754590692\\
63919.2000000008	62.55700547409\\
63998.9000000001	62.5551961749402\\
64016.9999999998	62.5534427520147\\
64042.6000000004	62.555138993203\\
64144.9000000001	62.5535895237353\\
64160.8000000006	62.5553880946184\\
64172.3000000005	62.5536835150314\\
64201.5000000004	62.5554503314559\\
64256.1999999999	62.5535550599708\\
64304.3000000011	62.555439441158\\
64314.6999999996	62.557140813623\\
64361.100000001	62.5552351416692\\
64388.7000000001	62.5569974902467\\
64412.4000000009	62.5552493692994\\
64470.6000000001	62.5569444724503\\
64478.800000001	62.555192473817\\
64491.0000000003	62.5532825459637\\
64509.400000001	62.5515621730133\\
64549.0999999991	62.5498069689477\\
64557.9999999994	62.547999398991\\
64592.3	62.5462128671485\\
64666.2000000001	62.5478804261261\\
64688.3999999997	62.5497576849054\\
64693.0999999998	62.5479957708074\\
64833.9999999994	62.5496459424901\\
64880.3000000005	62.5475490286743\\
64896.2999999999	62.5458422963369\\
64918.1000000007	62.5476368722485\\
64944.300000001	62.5493499054576\\
64966.2000000008	62.5476593249115\\
65015.6	62.5459696657884\\
65185.7	62.5473646100838\\
65293.9000000003	62.5489325205991\\
65397.9000000004	62.5506896235359\\
65431.8000000011	62.5490013280984\\
65468.9000000006	62.5473124684965\\
65556.6999999997	62.5489651721866\\
65647.4	62.55059217792\\
65684.7000000001	62.5523713248727\\
65706.1000000004	62.5540664351309\\
65744.8	62.5523804888288\\
65803.0000000006	62.5541957749711\\
65825.600000001	62.5524681089605\\
65877.9	62.55077567625\\
65911.400000001	62.5528170349636\\
65943.0000000011	62.5510477972676\\
66007.3999999997	62.5527402189145\\
66052.2999999994	62.5544264856387\\
66072.2999999996	62.5525938213233\\
66123.9999999992	62.5508702575914\\
66152.2000000002	62.5527759427866\\
66188.4999999999	62.5545788851857\\
66283.5000000005	62.5561979132093\\
66325.2000000011	62.5578776123139\\
66498.4000000004	62.5591554696722\\
66516.3999999992	62.5610186946096\\
66536.4999999999	62.5592531028036\\
66677.1999999991	62.5577220433341\\
66811.9000000001	62.5592707896785\\
66894.8999999995	62.5610284774647\\
67007.0999999999	62.5593966021562\\
67039.1999999996	62.5612140938226\\
67112.8000000006	62.5629349946136\\
67196.3999999994	62.5612941150209\\
67266.5000000009	62.5629956025725\\
67290.4000000009	62.5612827962342\\
67347.0000000006	62.5631987123425\\
67396.000000001	62.5648961883551\\
67598.1000000001	62.5636481444772\\
67629.5000000005	62.5653559979654\\
67675.8000000007	62.5633349538019\\
67791.9000000003	62.5618067028558\\
67876.6000000011	62.5634422415939\\
67942.4000000002	62.5652573867609\\
68136.7000000006	62.5664838971011\\
68145.3000000012	62.564751252469\\
68216.3999999992	62.5663878973562\\
68268.3000000004	62.5646712095201\\
68289.4000000006	62.5629123071629\\
68364.3999999991	62.5646351542102\\
68413.9000000003	62.5665799397784\\
68526.5000000005	62.5682581654423\\
68560.8000000006	62.5662731965304\\
68612.1999999995	62.5645839011373\\
68676.6000000006	62.5628488264579\\
68754.2000000004	62.560741655431\\
68790.5000000002	62.5624721982364\\
68888.0999999996	62.5641256412564\\
68983.2000000008	62.5657224284718\\
69004.6000000003	62.5639992638183\\
69010.4	62.5622187927924\\
69033.6000000005	62.563936164511\\
69081.4000000004	62.5621910352265\\
69098.2999999998	62.5604934412374\\
69127.499999999	62.5588042981385\\
69158.3	62.5605855543217\\
69334.2000000003	62.5621085090641\\
69376.4999999993	62.5604310387076\\
69399.8000000001	62.5623379860778\\
69468.7999999995	62.560656639158\\
69572.7999999997	62.5622907770123\\
69585.1999999997	62.5604833204714\\
69740.8999999992	62.5619076296583\\
69813.0000000005	62.5636162840498\\
69831.1999999992	62.5616306727785\\
69868.4999999998	62.5597192444102\\
69934.099999999	62.5613791249489\\
69999.7000000007	62.5630358943883\\
70099.5000000011	62.5614126186169\\
70165.9000000009	62.5596442721546\\
70220.5000000005	62.5614136022762\\
70387.0000000008	62.5600429624178\\
70420.4000000001	62.5583459361976\\
70487.0999999992	62.5600108955952\\
70502.5000000005	62.5582602627421\\
70578.5000000005	62.5598977593775\\
70638.0999999991	62.5582192071712\\
70665.6000000004	62.5599123761599\\
70838.8000000012	62.5585659856378\\
70898.9000000008	62.5602617808432\\
70939.8000000002	62.5621124360198\\
70973.8000000004	62.5601805734221\\
71083.6999999997	62.5585576460459\\
71124.7000000011	62.5568015656986\\
71219.1999999998	62.558604198581\\
71247.8999999995	62.5604929261172\\
71274.000000001	62.5586292917063\\
71423.1999999989	62.5601169366299\\
71457.0999999991	62.5584265826257\\
71549.9999999999	62.5602200416212\\
71600.2000000008	62.5585367659074\\
71648.2000000004	62.560311968323\\
71671.7000000007	62.5621234572035\\
71745.8000000002	62.560369303333\\
71798.2000000007	62.562066232766\\
71850.8999999993	62.5641953487077\\
71936.2999999994	62.5658220316835\\
71958.9000000011	62.5640990008199\\
72006.4000000005	62.5624075604286\\
72025.3000000012	62.5641787480528\\
72054.6000000009	62.5624699013388\\
72096.0999999992	62.5641850749415\\
72131.0999999999	62.5623863182645\\
72169.5999999999	62.5606036882514\\
72218.3000000001	62.562310989997\\
72268.2000000004	62.5642224875914\\
72277.7999999998	62.5624153441093\\
72316.9999999999	62.5607221528518\\
72337.6999999995	62.5624500606875\\
72386.8999999996	62.56413444403\\
72413.6999999992	62.5623845178681\\
72450.8999999998	62.5640777905067\\
72478.3000000002	62.5623634075807\\
72508.5999999996	62.5640785174745\\
72672.8000000003	62.562523306487\\
72806.8999999991	62.5610174845825\\
72857.2000000003	62.5627081980804\\
72994.9000000013	62.5611343242688\\
73106.999999999	62.5627333049731\\
73145.6000000011	62.5644432960516\\
73299.5999999998	62.5630118540731\\
73369.5999999993	62.5612480356332\\
73483.3000000007	62.562837321082\\
73608.2000000001	62.5644118937674\\
73659.4000000004	62.5627380039235\\
73694.9999999998	62.5610115190834\\
73726.8999999992	62.5626975192263\\
73844.7999999991	62.5611247357637\\
73862.4000000007	62.5628701980031\\
73891.5999999997	62.5646723515632\\
74014.599999999	62.5662199536038\\
74264.4	62.5673100875923\\
74301.5999999999	62.5690933047292\\
74408.1999999993	62.5674823910773\\
74425.9000000002	62.5657700266036\\
74447.9999999996	62.5674798953902\\
74489.800000001	62.5657975108035\\
74633.5000000002	62.5643141962869\\
74645.1999999993	62.5662968733464\\
74682.4999999998	62.5645063240969\\
74746.4999999993	62.5628456678966\\
74776.4000000007	62.5645757691253\\
74798.6000000011	62.562852028177\\
74915.9000000007	62.5612686208554\\
74931.3000000006	62.5630910406051\\
75058.9000000002	62.5646491426744\\
75164.3999999995	62.5662380512077\\
75198.399999999	62.5645458353558\\
75295.8000000002	62.5629018313082\\
75342.6999999996	62.5611206379375\\
75387.9000000005	62.5632726693904\\
75432.0000000007	62.56500879599\\
75568.6000000005	62.5665123258704\\
75654.0000000005	62.5647519433844\\
75672.9999999997	62.5630508066935\\
75698.6000000006	62.56131214935\\
75741.4000000001	62.5596271528818\\
75781.2000000001	62.557913363851\\
75859.5999999994	62.5596990233286\\
75912.8999999996	62.5616165874093\\
75940.6999999991	62.5634704928049\\
76120.0000000006	62.5649467092135\\
76231	62.5633107747363\\
76312.1999999991	62.5651434958716\\
76457.7000000006	62.5666707647879\\
76513.7000000006	62.5649229289357\\
76606.7999999997	62.5632939069457\\
76657.7999999998	62.5615885642576\\
76824.1999999996	62.5601534931005\\
76857.5999999993	62.5619814280157\\
76874.999999999	62.5638210551921\\
76912.0000000001	62.565578107996\\
76935.600000001	62.5674426826558\\
76980.0999999993	62.5692061075445\\
76989.1000000013	62.5674769967736\\
77021.700000001	62.5691687288534\\
77074.9999999992	62.5674180117833\\
77101.1000000004	62.5691947725846\\
77151.8999999992	62.5708990045624\\
77220.2999999997	62.569217460671\\
77267.1999999997	62.5712300028602\\
77307.0000000013	62.5694147109386\\
77345.4000000004	62.5711903084213\\
77367.5000000003	62.5730926124114\\
77422.3999999996	62.5748328974135\\
77474.2000000002	62.5731371564506\\
77565.7999999992	62.5713103309573\\
77641.3999999996	62.5696309319114\\
77649.8000000009	62.5678848266385\\
77716.0000000007	62.5661864143981\\
77824.8999999991	62.5679408930292\\
77883.5000000002	62.5661885172231\\
77978.9000000011	62.5645366060093\\
78043.2999999994	62.5663412921528\\
78145.0000000013	62.568094480652\\
78173.0000000007	62.5662791932263\\
78227.999999999	62.5680541902462\\
78267.8999999993	62.5697603107272\\
78289.0000000001	62.5679692319876\\
78344.0000000012	62.5697404144026\\
78417.7999999997	62.5679851156433\\
78486.1999999994	62.5662058219078\\
78642.4999999993	62.5646914013524\\
78680.8999999997	62.5630075876006\\
78736.0000000007	62.5648209652243\\
78833.1999999989	62.5664535164708\\
78853.6000000007	62.5647242932164\\
78894.9	62.5630268077825\\
78908.9000000003	62.5647264570581\\
78966.8000000007	62.5664677225521\\
79035.8000000011	62.5681494105843\\
79089.4000000008	62.5664595173822\\
79112.8	62.5683042841306\\
79162.8999999993	62.5666030847745\\
79339.2000000002	62.5681093732866\\
79410.0000000007	62.5698746129271\\
79449.6999999991	62.5715860832878\\
79496.4000000008	62.5697986703817\\
79510.3000000005	62.5715629653479\\
79563.7000000006	62.5697867623216\\
79683.699999999	62.5714386111104\\
79791.9000000009	62.5698064968919\\
79814.2000000008	62.5717446622974\\
80026.6000000012	62.5729912641656\\
80079.1000000004	62.5713044086355\\
80127.7999999992	62.5695918650058\\
80164.2000000009	62.5678762242046\\
80199.0999999994	62.5695767538828\\
80265.999999999	62.5678835772512\\
80346.0000000012	62.5695584477654\\
80555.0999999989	62.5682761634258\\
80608.600000001	62.5702932809982\\
80730.5999999989	62.5718592802986\\
80841.5000000001	62.5701371571072\\
80920.2999999992	62.5718607421614\\
80973.8999999989	62.5700842245659\\
81032.4999999998	62.5680281763142\\
81082.4999999998	62.5697745262239\\
81102.9999999997	62.571467675095\\
81206.9999999994	62.573100135333\\
81372.6000000011	62.574671849404\\
81398.4000000004	62.5728975349669\\
81549.1999999997	62.5713525437987\\
81604.2000000014	62.5696194930904\\
81619.2000000005	62.5714751290442\\
81634.3000000002	62.5697010083984\\
81766.2999999998	62.5681208907326\\
81956.2999999999	62.5666817966626\\
82016.1999999997	62.5685382052104\\
82217.5000000011	62.5699363639902\\
82268.2999999998	62.5716557025541\\
82302.6000000011	62.5735243193723\\
82339.1000000006	62.5752982783413\\
82366.599999999	62.5735885983049\\
82577.1999999988	62.5722807599667\\
82634.4000000002	62.5742274715766\\
82777.800000001	62.576122370826\\
82808.2999999993	62.5780717898184\\
82951.3000000013	62.5797756276567\\
82967.0000000003	62.578058049516\\
83212.2000000004	62.5768065538388\\
83289.4000000002	62.5749944470792\\
83347.100000001	62.5766672425708\\
83388.0000000007	62.5784734272636\\
83549.3000000001	62.5769903793444\\
83677.1000000013	62.5753490795581\\
83779.6999999993	62.5736752773342\\
83965.2	62.5722768810449\\
84103.0000000012	62.5739122576932\\
84116.4000000014	62.5721469628432\\
84231.1000000015	62.5738443712069\\
84327.8000000007	62.5754939942771\\
84478.7999999988	62.5739681743015\\
84516.3000000005	62.5756657879417\\
84543.2000000007	62.5736161233356\\
84643.0000000003	62.5753310074891\\
84745.1000000005	62.5769955112502\\
84857.7999999999	62.5753170889216\\
84986.3000000001	62.5770711549142\\
85067.3999999998	62.5752490669175\\
85119.5999999999	62.5735288070799\\
85244.2999999991	62.5718522272431\\
85365.6000000005	62.5702126263827\\
85460.7999999993	62.5684962362905\\
85537.1000000009	62.5666961275328\\
85602.9000000001	62.5683679310305\\
85820.4999999987	62.5670293612489\\
85899.1000000005	62.5653091064876\\
86016.5999999992	62.563664962734\\
86110.4999999991	62.565352000799\\
86185.0999999995	62.5670068642876\\
86201.2000000012	62.5652977391292\\
86290.1999999994	62.5669397371431\\
86429.0000000014	62.5685099115923\\
86456.2000000007	62.5702233382645\\
86559.2999999991	62.5719448147746\\
86588.5000000002	62.570245967714\\
86631.3000000005	62.568537504877\\
86817.9000000004	62.5668640143749\\
86864.9000000008	62.5650146779485\\
86903.5999999999	62.5667261578046\\
86954.9000000013	62.5650048875855\\
86970.7000000012	62.5632971066151\\
87002.4999999994	62.5651417313965\\
87051.3000000002	62.5668283336052\\
87150.2000000008	62.5684593168354\\
87226.1999999989	62.5701193332745\\
87248.3	62.5683680159178\\
87275.1999999998	62.5702804802734\\
87302.2000000005	62.5720055485365\\
87516.9000000014	62.570586286093\\
87742.4000000001	62.5692224406644\\
87824.3000000011	62.5674641671335\\
87907.2000000007	62.5657937395415\\
87924.300000001	62.5675011714609\\
88006.0999999989	62.5693417054707\\
88032.1999999994	62.5711244622712\\
88066.9000000013	62.5728138803411\\
88236.0999999992	62.5743175703396\\
88257.7	62.5725998155404\\
88337.5000000004	62.574260564018\\
88431.1000000003	62.5725988112793\\
88488.800000001	62.5708987228907\\
88563.3999999987	62.5692299875231\\
88618.7000000011	62.5709217457244\\
88651.1999999993	62.5691896227128\\
88714.0000000004	62.5675061799646\\
88950.7000000015	62.5688582902009\\
89089.9000000003	62.5672915029745\\
89294.5000000004	62.568733159676\\
89351.7999999997	62.5669963369553\\
89371.1000000001	62.5687022217448\\
89431.8999999998	62.5669782628142\\
89475.7999999996	62.5689152050999\\
89570.1999999992	62.5672795558349\\
89622.5000000006	62.5655805566899\\
89673.4000000014	62.563856657764\\
89935.6000000009	62.5626975717096\\
90048.2000000002	62.5643127077357\\
90129.5000000011	62.566016048002\\
90197.7000000005	62.5677122945349\\
90264.500000001	62.5693793580208\\
90310.5000000009	62.5676277203341\\
90368.699999999	62.5693823532016\\
90400.0999999997	62.5675606912374\\
90505.4000000013	62.5659214080912\\
90571.9999999996	62.5676118804797\\
90663.6000000014	62.5693634828493\\
90866.2999999993	62.567901886726\\
90915.3000000016	62.5701476317544\\
90950.7000000004	62.5684435980772\\
91044.4000000003	62.5701717292093\\
91239.2999999995	62.5686929111765\\
91393.4999999994	62.5702456189493\\
91498.8999999997	62.5720499677592\\
91620.2000000002	62.5703037427295\\
91825.1000000007	62.5688808736599\\
91840.5999999997	62.5706250061247\\
91971.6	62.5690293862134\\
92015.4999999994	62.5673255404518\\
92127.0000000008	62.5656294401973\\
92159.2	62.5674240147223\\
92201.8000000009	62.5656304262711\\
92241.1000000013	62.5673777010707\\
92361.9000000008	62.5657738030792\\
92594.7000000009	62.5644204642161\\
92749.0000000007	62.5628712300173\\
92793.0999999998	62.5648215602005\\
92824.600000001	62.5629816201938\\
92901.0999999997	62.5608711189952\\
93010.3000000004	62.5591331722044\\
93113.1999999988	62.5607727360109\\
93243.4999999998	62.5625780214406\\
93289.2000000001	62.5643026585042\\
93314.6999999986	62.5625302738687\\
93451.0999999989	62.5644186484497\\
93477.0000000002	62.5627025228639\\
93560.4000000001	62.5644369151512\\
93656.7000000009	62.5627824140906\\
93779.9000000004	62.5645126892728\\
93827.5000000006	62.562721416726\\
93900.7999999996	62.5644695631245\\
94099.0000000007	62.5629788170131\\
94164.5000000008	62.5613022303498\\
94269.8999999991	62.5630635408932\\
94362.4999999987	62.5647237358869\\
94433.8000000008	62.5630202713221\\
94511.2999999998	62.5611301917017\\
94614.5000000008	62.5594781355097\\
94860.2000000003	62.5582040115834\\
94874.3000000011	62.5563903434436\\
94923.4999999989	62.5580993556924\\
95017.7000000003	62.5563841722288\\
95080.3000000003	62.55463796955\\
95166.3000000014	62.5529598681888\\
95306.6999999997	62.5546131021081\\
95383.5999999993	62.5562858224204\\
95480.7999999999	62.55806135049\\
95518.9000000002	62.5563500455407\\
95590.9000000011	62.5546338044377\\
95735.9000000013	62.5529581348709\\
95817.7000000012	62.5546610337536\\
95862.5999999991	62.5529011805426\\
95964.3999999995	62.5512559331836\\
95992.3999999987	62.5530119540589\\
96041.4000000006	62.5547289452997\\
96065.0000000004	62.5529979149556\\
96269.7999999999	62.5515347995583\\
96498.5	62.5529282290106\\
96573.8999999994	62.5512042578748\\
96815.2999999994	62.5498629350289\\
96903.7999999988	62.5481533766959\\
96983.5000000009	62.5498537897129\\
97087.0999999992	62.5482040886955\\
97135.7000000016	62.5462496834329\\
97248.6000000014	62.5479826465547\\
97403.3	62.5462766186807\\
97605.1000000003	62.5477945847148\\
97783.3000000016	62.549369320355\\
97806.1000000016	62.5512493073036\\
97852.9999999993	62.5529492678311\\
97880.5999999987	62.5512486118305\\
97951.3999999993	62.5529981674604\\
98087.9000000014	62.5547467580132\\
98379.5000000002	62.5536188396782\\
98414.8000000006	62.5519103306512\\
98480.5000000011	62.5501875496291\\
98510.600000001	62.5521897621274\\
98548.6000000007	62.5504953388528\\
98589.0000000017	62.55235112198\\
98625.7000000016	62.5505699320056\\
98662.6000000002	62.5488659848149\\
98763.4000000002	62.5505373948878\\
98809.7999999994	62.5522341384821\\
98883.100000001	62.5505647066438\\
99034.1000000006	62.5521284566342\\
99139.7999999996	62.5539263202807\\
99237.7999999994	62.5522103954235\\
99350.5000000003	62.5505029662629\\
99376.5999999999	62.5487664613536\\
99460.5000000013	62.5505979252086\\
99714.1999999999	62.5491027866615\\
99804.8999999988	62.5507740093182\\
99984.3999999999	62.5522956058189\\
100000	62.5528374471626\\
};
\addlegendentry{QAM 16}

\end{axis}
\end{tikzpicture}%
        \caption{onvergence of BER results for QPSK, 16PSK and 16QAM modulations}
    \end{center} \label{fig:1}
\end{figure}



Figure \ref{fig:1} shows the stop time vs BER for QPSK, 16-PSK and 16-QAM modulations. The results are obtained using the Matlab script as shown in \ref{matlabs:task1}.


Figure \ref{fig:1} shows the percentage BER vs SNR for QPSK, 16-PSK and 16-QAM modulations for different stop times. The results are plotted on an x-log scale. The results are obtained using the Matlab script as shown in \ref{matlabs:task1}. 

The Graph shows to begin with the BER is not very precise, with lots of variation. As the simulation stop time increases, the \% BER converges on a value. at around 1000 intervals, the \% BER converges has converged really well. By 100000 intervals, the change in the BER is unobservable on the graph. This is because the noise model is based on a Gaussian probability distribution. As the stop time increases, the number of samples also increases, hence the noise model becomes more accurate and the BER converses on the true value.

The Graph shows PSK 4 has the lowest BER. This is because PSK 4 has the least number of symbols. This means the spacing between symbols on a constellation diagram is the largest, hence is the least susceptible to noise. The PSK 16 has the highest BER. Although it has the same number of symbols as QAM 16, due to the unit circle constellation of PSK 16, vs the lattice constellation of QAM 16, the distance between symbols on the constellation diagram is smaller, therefore PSK16 is more susceptible to noise.


